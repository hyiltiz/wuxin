%% LyX 2.0.3 created this file.  For more info, see http://www.lyx.org/.
%% Do not edit unless you really know what you are doing.
\documentclass[oneside,UTF8,adobefonts]{ctexbook}
\usepackage[T1]{fontenc}
\setcounter{secnumdepth}{3}
\setcounter{tocdepth}{3}
\makeatletter
\usepackage{url}


%%%%%%%%%%%%%%%%%%%%%%%%%%%%%% LyX specific LaTeX commands.
\providecommand{\LyX}{L\kern-.1667em\lower.25em\hbox{Y}\kern-.125emX\@}

%%%%%%%%%%%%%%%%%%%%%%%%%%%%%% User specified LaTeX commands.
% 如果没有这一句命令,XeTeX会出错,原因参见
% http://bbs.ctex.org/viewthread.php?tid=60547
\DeclareRobustCommand\nobreakspace{\leavevmode\nobreak\ }

\makeatother

\begin{document}

\title{无心法师}


\author{尼罗}
%\date{}

\maketitle

\tableofcontents{}

\part{民国时期}

\chapter{法师的来历}

无心法师永远不老,永远不死。

如此说来,他仿佛已经类似于神,可事实上他毫无神通,只是不老,只是不死。和凡人一样,他饿了要吃,渴了要喝,冷了要穿,累了要歇。所以在他无边无涯的人生之中,最紧要的一件事便是设法生存。当然,不吃不喝不穿不睡他也能活,至多是渐渐熬成一具人干,掩人耳目的蛰伏在僻静处守株待兔。然而饥寒交迫的感觉太不好受,而且无始无终的长久持续,让无心法师以为自己是堕进了阿鼻地狱。

无心法师不知道自己是从何处来,往何处去。太久远的往事他已经记不起了,他好像是从天而降落到人间,着陆之后就再没人管他。他不生不灭无魂无魄,只有一具不朽的躯壳。

因为头发至多只能长到睫毛的长度,所以无心在大部分的岁月里都在做和尚,做和尚好活,比卖苦力强。他自称会念经,会算命,会看风水,还会驱妖捉鬼。其中念经是真的,驱妖捉鬼也是真的,算命全是瞎诌,看风水更是胡说八道。凭着以上几样绝技,他浑浑噩噩的活了千百年,活到最后,就活腻歪了,不想活了。

无心法师的皮囊很体面,有着白皙的皮肤,浓秀的眉毛,眼窝微微凹陷着,由于常年的不想活,故而目光也是忧郁动人。他自认为挺英俊,可是难得拥有爱情,因为没有故乡,没有来历,没有家庭,没有亲人,又穷。凭他的资格,似乎只适合做上门女婿,但他的秘密瞒得过一时,瞒不过一世;一个永葆青春的女婿,足以令岳家上下毛骨悚然。况且根本无需一世的光阴,朝夕相处的日子过得稍微久一点,他的疑点便足以让家宅内外一起不宁了。

无心一度很爱和人亲近,想要找个姑娘作伴,结果天长日久露出马脚,被人当成妖怪烧过打过许多次。烧和打对他来讲,感觉都是统一的疼。他很伤心,并且也怕疼,所以渐渐离群索居,继续做他的游方和尚。

大概是在同治年间,无心法师终于坠入了爱河。一个十七八岁的丫头爱上了他,知道了他的所有底细之后,还依然爱他。无心法师快乐之极,当场脱了僧衣自行还俗,并且在瓜皮小帽后面掖了一条假辫子。带着媳妇在京城里过了十五年,媳妇长成了他的老大姐,两人就迁去了直隶一带居住。在直隶文县又过了十年,媳妇看起来开始像了他的娘。察觉到左邻右舍起闲话了,无心法师带着媳妇进了山,与世隔绝的度起了时光。媳妇最后是老死的,安安详详的无疾而终。无心法师含着眼泪伐大树做棺材,媳妇下葬这天,他稳稳当当的蹲在坟前,用媳妇留下的旧手帕蒙住了眼睛。

其实眼睛对他来讲,本是可有可无。他周身每一寸皮肤都能感知到颜色与光、空气与风。抬手向上招招摇摇,媳妇的魂魄缱绻缠绵,夏风一样掠过了他的指尖。

``玉儿,走吧。''他喃喃的说:``谢谢你用一生陪伴我,谢谢你。''

夏风稍纵即逝,旧手帕上还残留着玉儿的气息。无心法师在山里穷得很,平常的衣裳破到不能再穿,只好翻出了古旧的僧袍往身上套。午后的太阳照得他身上暖洋洋,像是玉儿伸出苍老干枯的双手,温柔的抚过了他的头脸。

在吃光家里最后一口杂合面之后,无心法师因为扛不住饿,所以独自下山谋生去了。

他当初上山之时,宣统皇帝还没有退位;如今下了山一打听,才知道民国的大总统都已经换了好几茬。坐在街边支起算命摊子,他打算糊弄几个钱买馒头吃,然而街上众人看了他的年轻面孔,一致认为他还是个小伙子,会算个屁。

无心法师没了生意,转而想去驱妖捉鬼。可镇子里面天下太平,并无妖鬼。无可奈何之下,他只得忍饿挨饥的踏上路途,直奔附近的文县而去。不料走到半路,他竟然出乎意料的得了个伴儿。

伴儿是个十七岁的姑娘,姓李,大名就叫月牙。月牙生得美人颈、流水肩、杨柳腰,身影比脸面更好看,当然脸面也不丑,明眸皓齿大辫子,是个干干净净的伶俐模样。月牙是从家里私逃出来的,因为爹娘要把她送给债主做八姨太。债主都六十二了,半脸褶子半脸麻,满嘴黄灿灿的大马牙。月牙不能坐以待嫁,于是趁着夜色深沉,收拾出个小包袱就跑了。

月牙一家是从关外迁过来的,家里丫头都不兴裹脚。月牙平日做惯活计,身体强健,又是一双大脚,奔跑起来分外得力。凌晨时分天蒙蒙亮,通往文县的小路上就只有她和无心两个人,她是有备而来,一边走一边从包袱里掏出一个棒子面窝头,一口一口的咬着吃。无心不远不近的跟在一旁,因为有日子没见干粮了,所以垂涎三尺,恨不能当场实行抢劫。

然而最后他并未真抢,因为月牙等他看到一定的程度了,主动掰了半块窝头递给了他:``师父,吃吧。''

无心几十年没有伪装过和尚,几乎连佛号都生疏了。对着月牙笑了一下,他接过窝头就往嘴里塞。而月牙看了他一眼,随即就转向了前方,不知怎的,忽然生出一阵心疼。

然后她自嘲的笑了,因为自己都是自身难保,居然还有闲情去心疼路人。

无心狼吞虎咽的吃了窝头,意犹未尽的伸舌头又舔了舔嘴唇上的渣滓。加快速度跟上了月牙的步伐,他终于开口说道:``姑娘,谢谢你。''

月牙自顾自的往前走,一边走一边又道:``文县外面的山上有座大庙,庙里和尚不少,也都吃得挺胖。你过去问问吧,要是能收了你,你不就有着落了?''

无心感觉到了对方的好意,于是跟得越发紧密:``姑娘,你是要去文县?''

月牙眼望前方,茫茫然的点了点头。到了文县又当如何?她不知道。

无心继续说道:``我也去文县。文县很大,我一定能弄到钱。等我有钱了,我请你去馆子里吃宴席。''

月牙本来都要愁死了,可是骤然听了无心的许诺,不由得愣了一下:``你个当和尚的,还要下馆子?''

无心望着月牙,不置可否的又是一笑。

月牙有一个好处,就是尽管时常感觉自己要``愁死了'',可是一分一秒的熬下去,她总有主意,从来没真愁死过。一个身无分文的大姑娘,回了家就得嫁给老头子做妾,离开家又无处投奔,怎么想怎么都没活路,身边还跟着一个招人心疼的怪和尚。和尚傻乎乎的真好看,让她看了心里难受得慌。为什么难受?说不清。总而言之,愁死了。

月牙存了寻死的心,什么都不在乎了,一边走一边对无心讲了自己的烦恼。无心歪着脑袋认真倾听,及至她说完了,两人也到了文县城门。

此时天已大亮,城门洞里人来人往,把姑娘和尚当成一对稀罕来看。月牙连活都不想活了,自然也就暂时不要了脸。而无心则是全不在意,只对月牙说道:``不至于。''

月牙十岁入关,身心都带着关外丫头的印记,问无心道:``啥不至于?''

无心从僧袍袖子里抽出一条旧手帕,双手抻开蒙上双眼。将手帕两端在脑后打了个活结,他迈步向前走去,同时头也不回的说道:``不至于死,也不至于愁!''

月牙拔脚追上了他:``你有眼睛不用,闹什么幺蛾子呢?''

无心灵灵巧巧的绕过脚下一块石头,然后轻声答道:``我在寻找财路。否则你没有钱,我也没有钱,到了中午,又该饿了!''

月牙连忙说道:``我包袱里还有一个窝头,一人一半,中午也能对付了——你慢点走,前面有臭水沟!''

无心不再理会她。长而柔软的僧袍袖子垂下来遮住了他的双手。他逆着晨风一路疾行。魂魄的光芒扑面而来,闭上眼睛,他才能看出人间有多拥挤。如此不知走了多久,张开的五指忽然合拢,他在袖内暗暗攥了拳头,鼻端掠过一丝阴冷的风。

天无绝人之路,文县果然没有让他失望。抬手解下眼上手帕,他扭头望向一旁,发现月牙已经追出了一头的热汗。月牙真不愿意追他,满大街的人都把他和她当疯子看,可是不追他追谁去?月牙现在没亲人了,就是走,也想在临走之前留给他半个窝头。

转回前方望出去,面前是两扇气派堂皇的黑漆大门。大门关得严丝合缝,无心伸出手去,猛然捶出一声大响。

门黑,显得他的手异常苍白。而院门后面立刻有了回应,声音苍老而又疲惫:``谁啊?''

无心清晰的答道:``法师!''

一阵铿锵之声过后,大门欠开一条大缝。一个形容枯槁的老头子探出头来,眯着眼睛去看无心:``谁?''

无心背过双手,直望进了老头子的浑浊眼中:``你家有鬼!''

此言一出,老头子当即一哆嗦。一只枯树枝似的老手伸出来,慌乱的扯住了无心的僧袍:``师父,请进来说——不,不,你别进来,我出去,我带你去找顾大人!''

\chapter{顾大人}

老头子是老派人物,言谈举止都带着前清气息。他口中的``'',乃是两个月前带兵打进文县的一位顾司令,而在顾司令之前,文县的主人翁乃是一位丁旅长,当然,也被老头子尊称一声丁大人。

老头子并非顾大人手下的听差,在顾大人手下吃饭的乃是老头子的三儿子。文县是个富庶繁华的地方,新近建造起了火车站,上了火车就能直奔天津卫和北京城。顾大人占据要地,十分得意,起了安家的心思,故而在旁人的撺掇下,就买了那处宽阔宅院——说是买,其实是抢,三进的大院子带东西跨院带后花园子,一共就给了人家房主一条小黄鱼。房主惹不起他,收下小黄鱼就拖家带口的跑了,跑到了哪里去,没人知道。而顾大人喜迁新居,没住几天就闹了怪事。

``我是亲眼看见的。''老头子带着无心和月牙穿大街走小巷,脸上始终是变颜失色:``我一大把年纪了,不会瞎说。''

无心走在一旁:``你看见什么了?''

老头子压低声音,在烈日之下出了冷汗:``一个女的,头发老长,贴着房梁一动不动。''

无心回头扫了一眼,见月牙跟得很紧,就放了心,继续问道:``还有呢?''

老头子像要晕厥似的,半闭着眼睛举起三根手指:``家里已经死了三个\ldots{}\ldots{}哎呀,死的都没法看哪!''

``顾大人怎么说?''

老头子放下了手:``妈了个×的顾大人搬司令部住去了,留下我家老三看房子。我能让老三送死吗?我就替他来了。小师父啊,不瞒你说,我现在一到天黑,就到门外坐着,一坐坐一宿,熬的我呀\ldots{}\ldots{}我都六十七了\ldots{}\ldots{}''

说到这里,老头子停了脚步。无心向前望去,就见前方是处青砖碧瓦的大四合院,院门口站着两名威武卫兵,想必就是顾大人的司令部。老头子上前和卫兵办交涉,月牙得了空,一把扯住无心的袖子,从牙缝里恶狠狠的挤出了话:``你个傻玩意儿,真是穷迷了心,连大长官都敢招惹!趁着人家没放我们进去,你跟我赶紧跑!''

无心莫名其妙的看着她:``你不想吃好的啦?''

月牙本来就觉得自己命运不好,如今遇上个二话不说就要捉鬼的和尚,越发的要愁死。鬼,她没见过,但是她信,也怕。无心贸贸然的就要揽差事,万一被鬼弄死了,自然是不好;可万一没被鬼弄死,而又没捉到鬼,同样还是不好。这些年各地都是一拨一拨的过大兵,月牙见得多了,还没遇过讲理的丘八。顾大人统领上万的人马,堪称丘八之王,更是不知道要嚣张成什么样子,兴许都能活吃人了!

月牙不想让无心被鬼或者丘八宰了,宁可饿着,也不想让他去冒险。然而未等她阐明利害,前方卫兵已经放行了。

无心随着老头子向院内走去,忙里偷闲的回头又对月牙一笑。月牙认了命,但是没理他。

四合院内青砖漫地,十分整洁。正房传出丝竹之声,正是一派吹拉弹唱的好气氛。一名副官上前挑起帘子,老头子立刻恭而敬之的把腰弯成九十度,四脚着地的就进去了。不过三言两语的工夫,乐曲歌唱一齐停止,老头子从门口伸出一张老脸,对着无心连连招手。而无心像怕月牙逃了似的,拉着她一起进了门。

门内窗明几净,家具华丽,有点小公馆的意思,并没有司令部的风格。无心向前一望,就见迎面一张大太师椅上,坐着个器宇轩昂的魁梧军官。此军官浓眉大眼高鼻梁,两条大腿分开来,被两个花枝招展的大姑娘分别盘踞。对着无心一扬下巴,他大喇喇的问道:``你就是会捉鬼的法师?''

无心脸色一正:``你就是顾大人?''

军官一晃脑袋:``对,是我,怎么的?''

无心凛凛然的又问:``顾大人见没见过鬼?''

军官摇了摇头:``我倒真是一直没见过,就听别人说来着!''

无心垂下眼帘,发现顾大人面前摆着个小茶几,茶几上面全是瓜果点心。不动声色的咽了一口唾沫,他的声音轻了些许:``顾大人杀气太重,鬼也怕你!''

军官当即仰天长笑,露出一口整齐的大白牙:``你这话说得很准,老子凭着刀枪打天下,的确是杀人如麻!''

无心听出顾大人是个难缠的货色,故而开动脑筋,沉默片刻后才又说道:``顾大人阳气重,杀气更重。想要除了恶鬼,顾大人非得和我一起回趟宅子不可!''

军官登时不笑了,望着无心反问道:``啊?还得让本司令亲自出马?''

无心正色答道:``对,顾大人是万里挑一的人中龙凤,想要引出恶鬼而又不为恶鬼所伤,非顾大人不可。''

随即他注视了军官的眼睛:``莫非,顾大人怕了?''

军官冷笑一声,眼睛瞪起来了:``我怕个屁!就算真有死鬼,老子也会让它再死一次!''然后他推开大姑娘站了起来:``现在就去?''

无心答道:``现在就去!''

顾大人向前迈了一步,这才发现无心身后还躲着个月牙。平心而论,月牙现在灰头土脸,没什么看头,不过身段袅袅婷婷的,让人一见便有印象。顾大人认为无心作为和尚,应该不能公然带着个大姑娘满街跑,于是就想不通了,笑嘻嘻的开口便问:``哟,这又是哪位仙姑啊?''

无心把月牙拽到了自己身后:``我妹子。''

顾大人出了司令部大门,骑着一匹菊花青往家走,无心和月牙合乘了一匹枣红马,紧紧的跟在后面。四周全被顾大人的卫队包围了,月牙心如死灰的垂了头,心想我反正也是没活路,索性跟着傻和尚混吧,就算没有傻和尚,吃完剩下的一个窝头之后也是饿死。

片刻过后,这一行人趾高气扬的回到了黑漆大门前。老头子一路随行,这时便上前打开门锁。卫士用力推开两扇黑漆大门,只听一阵生涩的吱吱嘎嘎,门外明明是艳阳高照,门内却像暗了一层似的,虽然也是花红柳绿,然而大概是无人的缘故,风景寂寞鲜艳的堪称刺目。

顾大人昂首挺胸,首当其冲的跨过门槛。无心跟在一旁,且走且对月牙低声说道:``你在我的后面,我走你走,我停你停。''

月牙当着许多大兵的面,不敢多说,一边点头一边跟在了无心的正后方。而顾大人抬手向前一指,开口说道:``一个月不到,家里死了仨,俩娘们儿一个半大小子。全不是好死,不知道让什么东西给撕了个碎,就只有脑袋是囫囵的。结果还吓疯了我一个姨太太。''

穿过两进院子,第三进院子院门紧闭。老头子又上来开了大锁。这回院门一开,月牙就觉得脊背一凉,从心里往外渗出了一层寒气。偷眼窥视了身边卫士的反应,卫士们都是年轻小伙子,其中有几人也是皱了眉头。

老头子开了门就退下去了,而顾大人若无其事的走进院内,对无心说道:``师父,看看吧,看够了再去跨院瞧瞧,后面还有个大花园子呢!''

无心没言语,转身把月牙推到卫士堆里,然后取出手帕,把眼睛又蒙上了。顾大人站在门口,就见他靠着四边套廊缓步前行。忽然蹲下来,他对着地面便是一掌。

随即起身再往前走,他第二掌拍在了一根廊柱上。

顾大人微微变了脸色,然而一言不发。而无心停下脚步,最后一掌拍上了套廊扶栏。抬手解下手帕,他转身望着顾大人问道:``是不是?''

顾大人走上前去,一手按着腰间的手枪皮套。用马靴靴底蹭了蹭无心拍过的地面,他抬头说道:``师父,要是提前没人向你通风报信的话,你就真是有两下子。那三个人,的确是死在了这三个地方。你看这砖缝里面,还干着黑血呢!''

无心慢吞吞的往东南角走,东南角有一口井,井台四面围着矮矮的小栏杆,旁边还扔着个挺新的小铁桶。院子里挖井是有讲究的,若论风水方位,这井并无问题。手扶井台探头下去,众人就见他越来越深入,最后竟然连肩膀都没入了井口。月牙怕他掉下去,正想上前揪住他的后襟;不料顾大人先行一步,直奔他而去。可是未等顾大人出言提醒,他慢慢抬头,离开了井口。只听``呸''的一声,他往井里啐了一口唾沫。

一转身坐在井台上,他面向前方开了口:``三人临死之时,饱受折磨,然而有身难动,有口难言。先被剥皮,后被拆骨,挖眼摘心,无所不至。''

顾大人在他面前蹲了下来,鬼鬼祟祟的低声问道:``师父,你认准了\ldots{}\ldots{}真是鬼?''

无心不看他,自顾自的继续说道:``比鬼厉害,是煞。鬼无形,煞有形。''

顾大人虽然自诩刚猛,可是听到此处,也有些胆寒:``反正我知道人死了就变鬼,变煞的可是没听说。煞是个什么东西?''

无心答道:``人有三魂七魄,三魂七魄便是人的光芒。人死如灯灭,三魂七魄消散开来,一生的爱恨也就烟消云散。顾大人,魂魄本来不灭,可若是你的三魂加上我的七魄,凑出的新灵魂却和你我都无关系。所以世间千百万人,大多是不知前世,只知今生。非得存有执着的信念,死后魂魄也不消散,依然是个完整的灵魂,且又不肯附在新生命上转世投胎,才能成为世人眼中的鬼。''

顾大人眨巴眨巴眼睛:``哎哟,当鬼也不容易啊!''

无心深以为然的一点头:``诚然,做一时的鬼容易,做一世的鬼,没有毅力是不行的。''

顾大人跟着他点头:``你接着说,惹上鬼了我该怎么办?''

无心毫无预兆的笑了,一边笑,一边侧身拍了拍井栏:``先吃午饭,吃饱了再说。办法不在你手里,在我手里。''

\chapter{午夜时分}

顾大人富可敌县,当然不在乎一顿午饭。他带着无心和月牙回到前院,支使副官前去附近的大馆子里要来一桌宴席。县里的高级宴席,其实也无非只是鸡鸭鱼肉而已,可无心在山中苦熬了许多年,连干粮都吃不足,如今见了荤腥,差点没当场香晕过去。

他知道顾大人是有求于己,所以并不客气。拉着月牙坐下来,他在桌子底下一晃腿,轻轻撞了月牙的膝盖,又低声催促道:``吃,多吃。''

月牙乃是平常人家的丫头,一年到头也见不到几次鱼肉,家里弟弟又多,有了好菜也轮不到她。她依然感觉无心是个耍嘴皮子的,虽然暂时唬住了顾大人,但是不定何时就可能被撵出去,所以她惜取眼前,决心一顿吃出三天的量。

顾大人坐在首席,还有心再谈两句,不料法师兄妹撩开嗓子眼颠起后槽牙,两只猪似的吃得头不抬眼不睁。顾大人现在有点尊敬无心,没敢贸然打断对方饮食,眼看着二人风卷残云,其中法师的妹子挺不要脸,剩下两个大白馒头还被她揣进小包袱里去了。

顾大人起初就只吃了一筷子凉拌菜,沉吟片刻之后还想再吃,结果一抬头,就见无心用半个馒头蘸着盘子里的汤汤水水大嚼,盘子全被他蹭得雪白锃亮。

顾大人放下筷子,认为自己遇到了饭桌上的对手:``师父饭量不错啊!''

无心一顿解了十年的馋,对着顾大人颔首微笑:``哪里,哪里。''

顾大人忍着饥饿又道:``师父,接着讲讲你的主意吧!你说我家里住着个煞,煞又是个什么东西?''

无心打了个饱嗝,随即答道:``人吃了饭,就有力量;鬼吃了鬼,也能壮大。壮大到了一定的程度,能够化成实在的形状,便是煞了。府上的煞大概是新化成的,之所以接二连三杀人,无非是要得到新鬼来吃。顾大人,此煞不除,府上宅院必定日益凶险,永无宁日。''

顾大人听他越说越真,不由得双手抱拳向他拜了拜:``师父,你说吧,怎么除?只要是成功了,我必定厚厚的酬谢你!''

无心穷的生疼,早就谋划着要敲他一笔。莫测高深的一笑,无心说道:``顾大人,要说除煞,虽不容易,但也有法可想。我下午筹备一切,今晚就要开始动手。但是要把煞引出来,须得要个勇猛的活人散发阳气才行。顾大人福大命大,非你不可了!''

顾大人张了张嘴:``我说师父,你不也是活人吗?''

无心微微一笑,随即斩钉截铁的答道:``我不行!''

顾大人真不想去做诱饵引鬼,想找几个副官代替自己出面,然而无心怀着鬼胎,坚决不允。顾大人没辙了,回到司令部打开一口木箱,从里面拎出一把一尺多长的砍刀。手握砍刀迎向阳光,他开口说道:``我家本是屠户,这把刀还是我爹传给我的。我用这把刀先杀猪,后杀人,死在刀下的肥猪不计其数,人命也有个二三十条!师父,这刀够凶了吧?''

无心正在盘算着如何从他身上诈出钱财,骤然听了这句问话,就不怀好意的一拍巴掌:``凶极了呀!''

顾大人听他语气轻松的诡异,不禁扭头看了他一眼:``师父,你原来都是怎么除鬼的?''

无心思索了一番,末了答道:``基本上是见到就骂,抓到就打,打服了算!''

顾大人深感意外:``怎么像是汉子打老婆?人家法师不都要掐诀念咒吗?''

无心摆了摆手:``低级伎俩,不值一提。顾大人,劳驾你给我捉几只黑狗,再来一只大公鸡。''

顾大人握着砍刀,乖乖出门找黑狗公鸡去了。

无心手刃黑狗,控出两大壶狗血。又把公鸡的爪子缚住,用红头绳缠住鸡头鸡嘴,不让公鸡随便开口鸣叫。晚上吃过一锅炖狗肉之后,无心带着月牙和顾大人,在卫队的簇拥下回了宅子。

傍晚时分,天光暗淡。看房子的老头子照例是搬了板凳坐在门外。卫队众人聚集起来守在前院,无心三人则是孤零零的一路前行,进了第三进院子。

月牙一手抱着大公鸡,一手拎着大铜壶,心里知道的不比顾大人更多。公鸡张不开嘴,路上一直从嗓子眼里低声咕咕。然而一进院内,它在月牙怀里抖了一下,一身的羽毛就乍开了。

无心转身接过月牙手里的大铜壶,在院子正中央用狗血浇出一个深红色的圆圈,口中说道:``月牙,你进来坐下。''

月牙果然是走进圈内席地而坐了,胆战心惊的仰头问道:``你到底要干啥呀?你可别整出大事啊!''

无心蹲下来,把大铜壶放到了月牙身边:``狗血能辟邪,公鸡阳气也重。把你放到外面我不放心,你好好坐在圈里,如果看到了什么不干净的邪祟,就用狗血泼它,狗血不顶用,你把公鸡脑袋上的头绳解开,公鸡也能帮你抵挡一阵。''

月牙和他认识了不过一天,没想到竟然成了生死与共的关系。她非常想埋怨他几句,可是转念一想,还是不说了,毕竟自己也吃了宴席和狗肉,死也是个饱死鬼。

月亮渐渐升上半空。月牙搂着一只臭烘烘的大公鸡,坐在狗血圈里环顾四周。房子真是好房子,雕梁画栋,她先前只在画片上见过。门窗都是关闭着的,白天来的时候没好意思细看,现在想看也看不清楚了,不知道屋子里都是什么样的摆设。忽然一阵凉风掠地而来,月牙打了个冷战,抬头再去望天,就发现星星减少了,已经成了个云遮月的天象。

顾大人个高腿长,正坐在套廊的扶栏上抽烟,脚边也摆着一壶黑狗血,砍刀则是被他系在了腰间。冷不丁的回头看了一眼,他见无心正直挺挺的站在套廊拐角处,并未远走,才放了心。

吸着香烟转向前方,顾大人心里犯了嘀咕。因为他到底也没见过``煞''的真面目,所以此刻感觉无心法师比煞还吓人——此君一直贴着墙壁站在暗处,不但不动,甚至连喘气的声音都没有,阴沉之中就见他微微低着一张雪白面孔,眼窝微微凹陷下去,乍一看仿佛两个黑坑。

一根香烟吸到了头,顾大人掏出烟盒,又续一根。如今正是夏季,他的两边衣袖全都挽到了肘际。□出来的小臂忽然过电似的一麻,他下意识的双手搓了搓胳膊,发现自己起了一身鸡皮疙瘩。

顾大人怀疑夜里风凉,自己穿少了。而无心站在原地一动不动,右手缩在袖子里,慢慢的揉搓着一团马粪。

良久过后,万籁俱寂。月牙抱着臭公鸡昏昏欲睡,朦胧中就见顾大人起身走到院内,一手夹着烟卷平伸出去,他自言自语的问道:``下雨了?''

月牙也伸了手,可是并没接到雨点。顾大人随手把烟头弹进井里,然后回到原位又坐下来。百无聊赖的打了个哈欠,他的后脖颈生出一点冰凉,正是落了水滴的感觉。正要抬手向后去摸,耳边响起``滴答''一声,又是一滴冷水落在了扶栏上。

顾大人怀疑是套廊顶上积了雨水,如今正在慢慢的渗漏。向上一摸头顶,他正打算换个地方,不料触手之处一片凉湿。他怔了一下,随即从头上摘下一缕水淋淋的长发。

水滴落得越发急了,顾大人猛然抽出砍刀,仰头向上望去,就见廊顶悬着一张惨白污秽的面孔,不但脸上血口纵横的没了好皮,两只眼睛也被戳成血洞,下巴嘴唇则是干枯焦黑,嘴唇皮已经没有了,两排牙齿齐齐露出,齿缝之间满是血涎。一头湿漉漉的黑发蜿蜒向下游去,顾大人看得清楚,发现上方的鬼脸子居然裂开了嘴,挤着满脸的伤口对自己狞笑!

顾大人吓疯了,大喝一声举起砍刀,不料未等他开始动作,长发已经向下缠上他的颈项。在半窒息的惊恐中哼出声音,长发如同触角,四处蔓延着覆上他的脸皮,竟是见洞便钻。

月牙远远的看在眼里,吓得立刻要嚎,哪知忽有一个人影飘然而现,正是无心。

无心神情平静的抬起双手,一上一下的抓住长发,轮换着慢慢往下拽。而那女煞顺势而下,对着无心张开血口,``呼''的一声喷出黑气。然而未等黑气出口,无心闪电般的骤然出手,将一团马粪直塞进了女煞嘴里,同时厉声喝道:``闭上你的臭嘴!''

女煞面容不动,脸上两个血窟窿里忽的翻出两只白眼珠,随即将一双冰冷的湿手合上无心的脖子,显然是要活活掐死无心。无心见她头发缠住了顾大人,双手钳住了自己,再无办法伤害月牙,便是放心大胆的抡起巴掌。只听噼里啪啦一阵脆响,他连着扇了女煞三十多个大嘴巴。而女煞死死掐住他的脖子,掐着掐着双眼转红,却是察觉出了异常——无心居然始终没有呼吸。而无心正视了她,看她不但眼珠变色,而且脑袋就像盛满脓血的皮囊一样,从大小伤口之中一股子一股子的往外喷起了血。

猛然向前直凑到了无心眼前,一条白色蛆虫蠕过了她血肉模糊的眼底。无心翘起嘴角笑了一下,随即低声说道:``臭娘们儿,你以为你长得丑,我就怕你了?本法师行走江湖的时候,你的三魂七魄还没凑齐呢!''

说到这里,他从袖子里抽出一条长长的粗麻绳:``来吧,让我带你晒晒明天的太阳!''

\chapter{讨价还价}

无心嘴上说的凶猛,手上却不十分加紧动作。而女煞再恶,也是由鬼化的,见了日光便要魂飞魄散。眼看无心不是善茬,女煞骤然松开双手,水蛇一样缩回廊顶,显然是要撤退。无心怕她会去袭击月牙,单脚踩上扶栏跳跃出去,他先一把夺过了顾大人手中的砍刀,随即几大步跑到月牙跟前。月牙此时已经解了鸡头上的红绳,正骇的双目圆睁,浑身乱颤。发现女煞沿着套廊廊顶移过来了,无心拎起铜壶,浇了月牙一头一脸狗血,紧接着一手抢过大公鸡,抡刀就追。而顾大人依旧满脸水淋淋的长发,直挺挺的瘫在地上,被那女煞一路拖行。

无心明知道女煞被自己打了个措手不及,现在正要逃命,可是并不痛打落水狗,一路谩骂着不使劲追。眼看女鬼行过套廊,逼近井口了,他这才一刀抹了公鸡脖子,然后对着女煞的长发用力砍下。只听``嗤啦''一声,就像火炭遇水了一般,浓厚长发迎刃而断。无心随即把公鸡向前一扔,公鸡挨了一刀,要死未死,拍着翅膀乱飞乱舞,正是撞上前方女煞。而女煞影子一晃,瞬间消失,似乎是投井了,但又没有听到水声。

夏季昼长夜短,如此闹过一场,天色黑的浓重,正是黎明将至的光景。月牙张着嘴怔了半天,最后忽然反应过来了,一身狗血一身冷汗,抬手一拍大腿,她打算像她家里的所有女眷一样嚎啕一场,可是嘴都咧开了,她又临时收了声,怕自己盲目撒泼,再把女煞招回来。无心从井旁把顾大人拽了过来,然后从怀里摸出半截蜡烛一根火柴。

蜡烛一亮,月牙心里就平定多了。她第一眼先去看无心的脖子,口中低声怨道:``你傻大胆,不要命啦?''

无心的脖子干干净净的,除了几点水珠血迹,再无其它。抬眼对着狗血淋漓的月牙一笑,他的脸孔像是一张细白瓷的面具,笑容很足,然而不带活气;眼珠子也亮,但是没有感情。

月牙一愣,感觉无心有点不大对劲,可又说不出来是什么问题。垂下眼帘扫了顾大人一眼,她吓得猛一哆嗦:``哎呀妈呀!''

顾大人满脸都是头发,头发顺着他的七窍钻进去,旁的部位不消说,就连上下眼皮都被头发攀住扒开了,眼珠子整个的晾在外面,四面全都露了白眼球。月牙看他,他黑眼珠一转,居然神志清醒,也能去看月牙。

无心起身走去,把顾大人的一壶黑狗血也拎了过来。安安稳稳的席地而坐,他开始用手指去摘顾大人脸上的头发。头发一层一层纵横交错,稍稍用力一扯,顾大人的眼珠子就要使劲的往外努。无心扭头对着月牙又是一笑,然后往顾大人的脸上浇了一层狗血:``顾大人,你别怕,我有法子救你。''

月牙伸手拍了他一下,又悄悄的一指水井,压低声音问道:``是不是跳进去了?''

无心一点头:``那是她的家,她在外面挨了打,不回家回哪里?''

月牙打了个冷战:``那是不是得把井填了?''

无心摇了摇头:``没用,几块石头堵不住她。''

说到这里,他再次去清理顾大人的面孔。头发本来勾结连环的紧贴皮肤,现在被狗血浸透了,就像失了生命一般,成了碎糟糟的一团一团。脸上露出本来颜色了,他捏开顾大人的嘴,从喉咙里又掏出几大团头发。顾大人呼呼的喘起了粗气,一翻身爬起来,``哇''一声就吐了。正在他吐得上气不接下气之时,遥遥起了鸡鸣,天亮了。

无心一行三人回了司令部,各自烧水洗澡。无心还特地向顾大人开口,给月牙要了一身干净衣裳。月牙锁了西厢房,又拉了窗帘;无心和顾大人则是在东厢房沐浴涤荡。

无心手持镊子,继续为顾大人清理七窍毛发。又掏耳朵又掏鼻子。顾大人忍痛皱眉,几乎被他把鼻毛拔光;同时自己举起一面小圆镜,仔细查看眼睑内外,生怕还有毛发残余。

及至顾大人确定自己七窍洁净了,才有闲心对无心问道:``师父,你昨夜让那东西跑了?''

无心和顾大人分别占据了两只大浴桶,此刻坐在热水里面,他一本正经的答道:``我当时若是再和她交战不休,恐怕顾大人要性命不保。''

顾大人挖了挖鼻孔,又问:``那\ldots{}\ldots{}今夜还去?''

无心在浴桶中轻轻巧巧的一转身,正视了顾大人的侧影:``女煞十分凶暴,我纵是去了,也没有十成的把握。顾大人,我愿意拼出性命去完此事,可你又当如何报答我呢?''

顾大人本来以为家宅闹鬼,找个和尚老道过来禳治禳治也就罢了。然而昨夜亲眼见识了女煞的本领,他不禁一身接一身的起鸡皮疙瘩,承认此事实在凶险,自己不多付出一点,恐怕真找不到高明人物降妖除魔。

``本司令肯定不能亏待了你。''顾大人试探着问:``师父,你开个价吧!''

无心竖起一根手指,望着顾大人没说话。

顾大人笑了:``一百大洋?''

无心摇了摇头。

顾大人想了想:``一千大洋?''

无心继续摇头。

顾大人有点龇牙咧嘴了:``总不会是\ldots{}\ldots{}一万大洋吧?''

无心这回点了头:``一万大洋,不划价!''

顾大人有点生气了:``你个出家人,怎么狮子大开口啊?张嘴就要一万大洋,你当本司令的钱都是大风刮来的?你要一万大洋干什么?大不了我给你修座庙,你守点和尚本分行不行?''

无心毫不动容:``顾大人,既然你我谈不拢,那我洗完澡后,立刻就走。顾大人另请高明吧!''

顾大人一听这话,脸色都变了:``放你娘的狗屁!你要是走了,万一那东西半夜过来找我怎么办?''

无心满不在乎的侧脸往窗外望:``你可以和她解释嘛,就说是法师打了你,不是本司令打了你。你通情达理,出门找法师去吧!''

顾大人沉默半晌,忽然把牙一咬:``老子这就去调几门大炮过去,对着井口开轰!''

无心面无表情的答道:``好主意,我听说大炮很厉害,大概真能把鬼打死。''

顾大人``哗啦''一声从浴桶中站了起来:``师父,你要么打个一折,要么我现在就去把你妹子奸了!''

无心靠在桶壁上,舒舒服服的闭了眼睛:``大人,你要么给我一万大洋,要么我夜里就去引来女鬼,把你奸了!''

顾大人高高大大的站在水中,双手叉腰怒道:``操!什么流氓和尚!''

无心和顾大人在东厢房内唇枪舌战,顾大人有求于人,夜里又受了大惊吓,当然底气不足。末了顾大人败下阵来,穿了军裤衬衫往外走,不料刚一出门,就见月牙蹲在院内树荫下,正就着一盆净水搓血衣。

月牙身上的一套豆绿衣裤,还是顾家姨太太的旧货。姨太太不缺穿的,再好的料子也就穿个两三次,所以衣裤看着堪称崭新。月牙一直灰头土脸,现在终于露出了本相,顾大人看在眼里,认为她虽然不算标准的美人,可是干干净净的有精神,眼睛明亮,脸形端正,一笑一口小白牙,带着一点良家丫头的俏皮。

顾大人素来自诩英俊潇洒、风流倜傥,故而如今走上前去,想要施展几分魅力和手段,迷倒月牙:``真勤快,不困啊?''

月牙仰脸看着他一笑,怕笑大了不庄重,所以一笑即收:``顾大人。''

顾大人一手伸出去扶了大树,一手插在裤兜里:``昨夜没吓坏吧?''

月牙都吓的麻木了,低头一边搓衣裳一边摇头:``没事,天一亮就不怕了。''

顾大人还要说话,不料无心无声无息的走了过来,对着月牙说道:``别洗了,回屋睡觉吧。我要是能把女煞宰了,顾大人就给我们一万大洋。有了钱,还怕没衣裳穿吗?''

月牙看看无心,又看看顾大人,就感觉自己像掉坑里了似的,没出路了。

\chapter{作恶}

顾大人的司令部,其实也是一处强占下来的民宅。东西厢房都砌着火炕,正房才是会客之所。夏天火炕上面铺了席子,硬邦邦的倒是凉快;月牙没了事做,靠边躺在炕上打盹。因为知道无心就坐在旁边,所以她睡不实,隔三差五的就醒过来眯了眼睛,偷偷窥视对方的行动。无心不声不响的总跟着她,让她有了个不大好意思的想法——她感觉无心好像是看上自己了。

此刻正是下午,窗外知了叫成一片。月牙侧身紧紧靠墙,就见无心脱下僧袍,换了一身黑色裤褂,打着赤脚盘腿而坐,身边高高堆起一摞古旧厚书。书籍乃是文县县志,无心想要找出女煞的来历,又打听不出,便让顾大人要来县志,专翻几十年上百年前的故事看。文县的县志是本县历代学究们联合撰写的,已经传了几辈,字字句句都很严谨,而且包罗万象,大事奇事全有记载。

无心读得认真,月牙也看得入迷。无心穿僧袍时就不大像正经和尚,脱了僧袍更不像了。月牙瞧他黑黑的短发白白的脸,分明是个美男子的模样,至多不会超过二十五岁。要说年纪,和自己倒也是很般配;但捉鬼可不是正经营生,年纪轻轻的,干点什么不能挣饭吃?

无心读书很快,唰唰的不停翻页。最后他心里大概有数了,收拾起一摞县志送出门去。片刻之后回了来,他上炕推了推月牙:``醒醒,再睡夜里就睡不着了。''

月牙故意打了个小小的哈欠,因为发现无心已经光脚蹲在了自己身前,便坐起来向后又躲了躲。而无心笑嘻嘻的把手一伸,送给了她一个很大的香瓜。香瓜白生生水淋淋,显然是被狠狠的洗过一次。

月牙一手接了香瓜,另一只手攥了拳头向瓜上一捶。香瓜应声裂成两半,月牙把大的一半给了无心:``你也吃。''

无心接过香瓜咬了一口,垂下眼帘美滋滋的。月牙问道:``师父,今夜\ldots{}\ldots{}还去吗?''

无心摇了摇头:``今夜不去了。那东西昨夜没讨到便宜,想必一时半会不敢出来,今夜去了,恐怕要白等一宿。明夜吧,明夜再去打她个措手不及。''

月牙看他紧挨自己蹲着,根本没有移动的意思,就往旁边又蹭了蹭:``干完这次可别再干了,太吓人了。''

无心笑着一点头:``干完这次我也就发财了,顾大人应该不敢和我耍赖。等一万大洋到了手,我们找个好地方买所小房,安安生生过几年日子。''

月牙含着一口香瓜,本来是一点也不生气,但是感觉不生气不像话,于是就很勉强的生气了:``你说啥呢?谁要跟你一起过日子了?你上那边蹲着去,别离我这么近!''

无心向后退了一寸,捧着半个香瓜对月牙拜了拜:``求求你了,跟我过吧!''

月牙起身走到大炕另一端去了:``你不是和尚吗?和尚还想着娶媳妇哪?''

无心转身面对了月牙,很认真的低头给她看:``我不是真和尚,你瞧,我头上没有戒疤。''

月牙抱着膝盖坐在角落里,低头不看他。而他抬头望向月牙,可怜而又谄媚的微笑不止。

无心的确是看上了月牙,因为月牙对他有善意,而且模样也挺可爱。他对于寂寞的岁月已经痛恨至极,只要有人肯和他作伴,无论是谁,他都热烈欢迎。当然,女人最好,因为男女凑起来是一户人家。

没有女人来和他做夫妇,来个男人和他做兄弟也行,他甚至捡过许多弃婴来养,可是养着养着弃婴就长大了,比他还大,比他还老,并且最终都是离他而去。他甚至和一只狐狸精相好过,好了没几天就不好了,因为他素来是按照人的方式来活,和妖精过不到一起去。

无心想要笼络月牙,所以格外殷勤。月牙刚吃完香瓜,他就拧了一把毛巾给她擦手。月牙受了他的照顾,心里十分为难——要说嫁,没有认识一天就嫁的;要说不嫁,自己心里其实也挺喜欢他,看他像个狗腿子似的跑前跑后,甭提自己多心疼了。

无心敲了顾大人一笔巨款,又奉承着心里看上的大姑娘,感觉生活很有奔头,暂时就不想死了。

转眼间天色擦黑,无心和月牙睡在了西厢房。一铺大炕分成两半,月牙和无心各占一端,中间隔开老远。夏天衣裳单薄,和衣而睡也不难受,月牙面对墙壁一动不动,无心却是审视着她的背影,越看越美。虽然月牙下午骂了他几句,让他闭上狗嘴。但无心自作主张,已经把月牙收为己有。

顾大人受了惊吓,不敢远离法师,此时在东厢房也上了炕,又让人把五姨太从小公馆接了过来。五姨太正受宠爱,昨夜没等到他,今夜见了面,格外温柔。为了彰显自己勾魂摄魄的媚态,五姨太没有开灯,只点了一双龙凤蜡烛。摇曳烛光之中,她一张浓妆艳抹的面孔没了血气,一色煞白,嘴唇却红的突兀,眉眼也黑的深邃。顾大人抱着棉被坐在炕上,本来觉得五姨太最美丽,然而自从经过昨夜惊吓之后,审美观忽然发生变化。眼看五姨太拔下发卡,甩出一头浓密青丝,他打了个寒颤,忍不住又挖鼻孔又抠耳朵,且把舌头伸了出来,咔咔的清喉咙,就觉得嗓子眼里有头发。

五姨太以为他是做鬼脸,便含着笑容翩然而来。不料未等她走进炕沿,顾大人忽然向后一缩,声音都变了:``你别过来!''

五姨太一愣,随即就不乐意了。抬腿迈上炕去,她直逼到了顾大人眼前,尖声尖气的怒问:``干嘛呀?看不上我啦?看不上你早说啊,何必还要派汽车去接我?你当我乐意来哪?''

五姨太是个苗条的小身材,一生气就张牙舞爪,手指头又长又细的,长指甲上的蔻丹鲜红欲滴。顾大人昨夜落了心病,眼看五姨太披着一头黑发凑上来了,两根枯骨一样的细胳膊还挥来挥去,不禁精神崩溃,大叫一声下炕就跑。一溜烟的横穿了整个院子,他一头撞进西厢房中。``啪''的一声打开电灯,他在光明之中蹦上大炕,一掀棉被拱到了无心怀里,又哆哆嗦嗦的叫道:``师父,快保护我!''忽见对面的月牙坐起来了,他连忙招手:``仙姑,你也过来!你们两个一起搂着我,我害怕!''

此言一出,月牙和无心全气笑了。未等无心出言讥讽,五姨太冲到院子里,开始骂起了顾大人,因为顾大人不爱她了。

前半夜,谁也没睡着觉。

后半夜,五姨太被副官开汽车送走了。而顾大人因为一闭眼睛就是鬼脸长发,所以死活不肯回房,定要占据大炕中间的位置。月牙忍无可忍了,气得说道:``我不能跟两个老爷们儿睡一铺炕,我下地用椅子拼张床去!''

顾大人以为无心和月牙是兄妹,忌讳不必太多,只是多出一个自己,比较难办。起身挤到了无心身后,他陪着笑对月牙说道:``仙姑,你就当没有我,我躺在他身后,也看不见你。''

月牙本来睡得挺好,远远的躺着一个无心,安安静静的,也挺好。冷不防来了个顾大人,就一点都不好了——可毕竟是睡着人家的屋子,又不好太挑剔。

月牙不再说话了,关了电灯躺下来。而顾大人守着无心,很有安全感,闭上眼睛也睡了。无心有心事,一边思索一边提醒自己别忘了喘气。等到月牙的呼吸粗重了,顾大人也打起了呼噜,他才放心大胆的吐出最后一口气息,瘪着胸腔彻底放松了。

翌日上午,无心等人刚刚起床,就有人急三火四的跑来报信,说是看房子的老头子被鬼杀了。

无心眼看天空一碧如洗,是个骄阳似火的好天气,想必阳光必会整日充足,不容邪祟作怪,便放心大胆的把月牙和顾大人留在司令部里,自己带上一把匕首,骑马去了宅子查看。宅子门口站着几名士兵,见法师来了,像见了救命星一般,立刻就给他让出了路,又有人轻声说道:``本来老头夜里都在外面坐着,可是昨晚\ldots{}\ldots{}一直没出来。''

无心停下脚步,开口问道:``谁发现的?''

士兵答道:``胡同里送水的人早上推门没见老头,就挑着水桶往里走,结果没走多远就吓坏了\ldots{}\ldots{}''

无心不再询问,跨过大门门槛之后,转身关拢了两扇黑漆大门。人死成鬼,大多是存有一段不散的怨气;可由于自身含怨便滥杀无辜,则是无心最深恶痛绝的行为!

仇再大也大不过一个``死''字,就算死了还放不下,那有冤报冤有仇报仇,也不该把恶气出在无辜的活人身上。老头子六十七了,要说价值,他没什么价值;可他是家里老妻的丈夫,是儿女们的老爹,他宁可自己整夜不睡觉,也要替三儿子冒险看房子。好好的一位老人家,凭什么恶煞说杀就杀?

院子地上凝结着一洼洼的黑血,成群结队的苍蝇盘旋不去。老头子真就只有一个脑袋还是完整的了,脸冲下滚在厢房门前的台阶旁。无心走过去蹲下来,捧起脑袋转过来一看,就见老头脸上肌肉狰狞,双眼被戳成了血洞,一张黑洞洞的大嘴张到极致,竟然占据了下半张脸。

无心闭上眼睛,觉察出老头子的血肉残肢上还附着残余的一魂两魄,魂魄凶气极重,正是惨死之人应有的现象。如何超度亡灵,无心在很久很久以前是会的,然而太久不做,已然忘记。出门向士兵要了几根火柴,他把满地的碎肉断骨收到大太阳下,又把人头恭恭敬敬的放到最上方。一把火点起来,他低声说道:``你的仇,我来报。有生有死是好事,该走就走吧。''

烈焰加上骄阳,足以使得魂魄四散。老头子的家人还没赶到,所以无心待到魂魄散开,便扑灭火焰,留了大半骸骨以便装殓下葬。想到恶煞狠毒,又见天色还早,距离正午三刻还有一段时间,无心索性大踏步走向后院。及至来到井边,他不假思索的脱了衣裤鞋袜,因见前夜用过的绳子还在廊前地上,他便过去拿起了绳子。

回到井边从衣堆里面翻出匕首,无心一道划开掌心。用力的按压掌心挤出了一点暗红鲜血,无心用伤手握住绳头向下一撸,在绳子上面留下了断断续续的浅淡血迹。

把绳子一圈一圈缠在臂上,无心跨上井台,低头向下望去。井水黑沉沉的深不见底,散发着隐隐的寒气。无心认为井中女煞已经恶到不可救药,所以懒得再等入夜。拎着绳子一头扎进井里,他决定速战速决,不再给她嚣张的机会。

\chapter{最后的异动}

无心将匕首衣物尽数留在井口,然后手无寸铁的带着一卷染血麻绳,毫无预兆的就大头朝下跳了井。他本来不怕受伤,然而感觉敏锐,很知道疼,手心上面新增了一道刀口,免不了要半轻不重的作痛。井是一口气派的好井,不但井台平坦坚固,下面长长一段井壁也是砌得笔直齐整,是个利利落落的正圆形。四周水汽阴森,青苔湿滑,无心像条鱼似的飞速下坠,瞬间周身一寒,已然无声的扎入了井水之中。

入水之后,无心一脚蹬上井壁,借力翻身改成了头上脚下的姿势,因为身无寸缕,皮肤光滑,所以无心在水中动作利落,毫无滞涩。抱住膝盖继续下沉,他闭上双眼沉静片刻,就觉水寒入骨,四面黑沉,简直和井外不是一个世界。耳孔中鼓出最后一个气泡,他睁开眼睛,像一尾深潭中的鱼,天然的不需要光,一样能够看清。皮肤有了麻麻痒痒的触感,他看见了无数长发如同细小的水草,无根无源的在四面八方飘飘摇摇。

无心知道女煞就躲在长发之中,如果下来的不是自己而是凡人,大概阳气一显,立刻就会被长发纠缠控制。然而无心非人非鬼,不死不生,一如木石一般,所以来就来了,并未轻易惊动女煞。

脚下忽然落了实地,无心在水中起起伏伏的勉强站住,不动声色的环顾周围,发现这口井是个大肚子壶,上面看着普通,井下却是四面扩张,最后竟是宽宽敞敞,足像一间小屋。仰头再向上望,因为头发太多太密,所以乌云盖顶,也不见光。抬手抓住一把头发,无心不再犹豫,开始混拽乱扯。而水中长发忽然像成了精似的乱舞起来,无心一边顺着头发寻找女煞,一边抡起绳子充当鞭子,四面八方的乱抽。一时间水中大乱,他竟是当真打的长发散开,不能缠拢。正是激烈之时,无心忽觉身后阴气一鼓,来势汹汹。一跃而起回手甩出一鞭,他耳边只听一声凄厉的惨叫,绳子正是狠抽上了突袭而来的女煞!而无心抬手一指面前翻翻滚滚的无边毛发,口中厉声喝道:``你再厉害,也无非是鬼煞一类。前夜我手下留情,是要让你反思悔改!没想到你不知好歹,反而变本加厉的继续害人,那就别怪我不客气了!''

此言一出,毛发阵中传出幽幽的回应:``口气不小,你又是个什么东西?''

无心抬起双手,缓缓抻直绳子:``我?不可言说!''

随即他纵身向前直冲而去,就要强行缚住女煞。此时正是天光大亮的时候,女煞一旦离了水井便是魂飞魄散,自然不能坐以待毙。一口咬上无心的喉咙,她虽然看出对方不是平常人物,但还以为他是法力高强的真正法师,用了法术闭住呼吸。煞的身体乃是大量怨气聚合而成,一呼一吸都带着毒,何况用牙鲜血淋漓的往肉里咬,就算只是破皮,也足以要人性命。无心忍痛不躲,自顾自的要用绳子把煞和自己捆在一起。煞本来不怕束缚,然而此刻一挨绳子,她再次哀号一声,松了血口就往后退——并非因为绳子上写了刚猛的符咒或者附了极阳的物事,绳子带着一股子诡异之气,如何诡异?说不清。

女煞躲进角落,身体完全躲在水草一般丰隆的长发之后:``你到底是个什么东西?''

无心看她有了畏缩之意,是个欺软怕硬的货,心中越发愤恨。回想起县志中所记载的内容,他忽然起了恶意,想刺激刺激对方,于是微微低头笑了一下,口中柔声唤道:``岳绮罗,我是你的段三郎呀!说好是要同生共死的,怎么我如约投河,你却还要继续活?''

话音落下,他拎着绳子再次冲向女煞。而女煞听了方才他的一番话,竟像是受到莫大威胁一般,骤然发疯一般开始迎击。井底再大,也无非是大过上方而已,容不下两个人你死我活的互斗。女煞施展种种毒术,连连击中无心的肩头腹部。无心是个光身子,随她去打,连个手印都留不下;女煞看得清楚,更加怒发如狂,伸出利爪猛然出击,``噗''的一声抓向无心胸膛,而无心不躲不闪,结成绳扣向下一套,正是套上了女煞的脖子。忍着剧痛一勒绳头,他低头再瞧,只见女煞的指甲已经刺入自己皮肉,正是个挖心的招数。

无心不怕她挖,只是害疼,所以迎头伸出两指,去戳对方脸上两个血洞。一戳之下,他骂起了街:``妈的,两个眼睛分得这么开!''

随即他手心朝上重新又戳一次,指头向上勾住了对方的眼眶骨头,他双脚蹬地,便要带着女煞往上游。女煞知道一上去就要魂飞魄散,所以拼命挣扎。脖子上的绳扣越勒越紧了,她终于意识到了敌人的诡异之处——敌人是死的!

不是生生死死的死,是在开天辟地之前就存在的、无始无终无声无色的死!活人死了还有轮回,鬼煞散了还有魂魄;可对方像个影子,只是存在,一无所有。如果最惨烈的失败就是死亡,那么对方永远不败!

井底黑透了,长发沸腾着纠结盘旋。女煞积蓄力量叫道:``你是傀儡!''

无心一手攥着绳子,一手勾着眼眶,不为所动的带着女煞继续向上游。眼看水中头发渐渐稀疏了,头顶渐渐显出光明了,他正要加快速度,不料身前女煞忽然一震,随即他手上一轻,低头一瞧,发现女煞竟然断了脖子,脑袋还在自己手中,身体却是目标明确的直往下沉。

无心不知道她这是什么意思,以为她打算当个无头煞,身残志坚继续害人。扔了脑袋俯冲下去,他穿过一层厚重头发坠入井底,结果就见无头女煞合身直扑前方,一下又一下的拼命狠撞。无心看得清楚,见前方漆黑一片,无非是井底四壁而已,可在女煞的几番撞击之下,井水渐渐被弥漫开来的泥土混成污浊,无心闭了眼睛,只觉前方阳气大盛,但又不是来了活人。

此时无头女煞依然在撞,无心在水中也照样耳聪目明,就听女煞隐隐撞出金石之声。感觉井水略略清澄些许了,他睁开眼睛再看,发现前方出现了一道石壁,石壁上面刻了阴阳八卦。女煞姿势扭曲抽搐,仿佛每撞一次都是苦楚难言。

无心没看明白,但是隐约预感不妙——一个将要魂飞魄散的女煞,忍受着比魂飞魄散更大的痛苦去撞施了法术的墙壁,图个什么?

女煞撞过几次之后,便漂在水中不再动了,一身乱糟糟的破烂衣裳随着水流摇曳。无心游上前去要抓女煞,哪知一只手都已经搭上女煞肩膀了,女煞忽然把身一挺,弓一般的腰背向后弯曲,随即竭尽全力,只听``咚''的一声巨响,女煞不知是下了多大的狠心,居然撞碎了半边身体。碎骨烂肉散于水中,女煞静静的歪在水中,又不动了。

无心莫名其妙,不过半边身子他也要。一手抓住女煞的臂膀,他转身游回井口正下方。女煞大概是真不行了,水中的长发全像被淋了狗血,丝丝缕缕的成了败絮。俯身捡起女煞的脑袋,无心也不用绳子了,一跃而起便要向上游去。

然而就在浮起的一刹那间,他的眼前忽然掠过一串小小气泡。井底既成了女煞的老巢,一般的活物也不会有。无心还未想出气泡的来源,头顶的光明再次出现了。

与此同时,顾大人和月牙也鸡飞狗跳的进了宅子大门。

无心说是出门察看,然后一去不复返。顾大人并不通晓鬼神的脾气,以为女煞会像姨太太一样无孔不入的追他,所以身边没了无心,不由得心中惴惴,站在大太阳下都冒冷汗。而月牙眼看到了午饭时分,无心该回来不回来,放着好菜吃不到嘴,不禁也着了急。月牙虽然也怕鬼煞,可是自认为活了十七岁,只有人负她,没有她负人,所以别有一番听天由命的坦荡。两人站在大太阳地里合计一番,末了就决定同去宅子,看看无心到底在干什么。

顾大人十分谨慎,披挂出门,身前身后各绑了一只大公鸡,汽车前后还跟着三条大黑狗。月牙心疼衣裳,不肯抱鸡,改抱一只毛茸茸的黑色小奶狗。

汽车走得顺利,没几分钟就到了宅子门口。顾大人有公鸡护体,牵着大黑狗往里走。月牙跟在后面,刚走几步便见了满地干血。守门的卫兵小跑上来,低声说道:``报告司令,看房子老头的尸骨,已经被他家人接走了。大法师不知是用了什么法术,把老头的尸首给烧了个七八分熟。''

顾大人停下脚步想了想,随即疯狂的挥手:``滚滚滚,听你说话我吃不下饭!''眼看卫兵真要退下了,他又把对方揪了住:``法师呢?''

卫兵弯着腰,低声说道:``报告司令,我上午从门缝里溜了一眼,看大法师往后院去了,一直没出来。''

顾大人沉吟着摸了摸左腰的手枪、右腰的砍刀,然后连人带鸡一起转向后方,高声命令道:``来人哪,齐步走,跟我上后院去!''

顾大人带着他的人与动物,一路杀气腾腾开进后院,不料刚一进去,就见无心穿着一身黑布裤褂,水淋淋的赤脚坐在井台上。无心的脚边地面摆着一堆物事,是一大团头发缠裹着半截躯干,正在阳光下面嗤嗤的蠕动。仿佛头发下面,有冰水与火炭共存。

顾大人愣了一下:``你——''

无心抬眼看他,忽然若有所思的笑了:``我——''

``我''字之后,戛然而止,无心又笑了笑,不肯再说下去了。

\chapter{消散}

光天化日之下,顾大人前有无心后有卫队,胆气极壮。``嚓''的一声拔出砍刀,他上前两步弯下腰来,用刀尖去挑那一大团头发,一边挑,一边忍不住又挖了挖鼻孔,掏了掏耳朵。自从经历过女煞的纠缠之后,他现在见了披头散发的娘们儿就害怕。

头发又长又湿又重,水淋淋的分不出个条理来。无心见顾大人挑个不休,索性伸手帮忙,拎起脑袋向顾大人一递:``看看,眼不眼熟?''

日光之下,女煞的头颅就像要消融一般,破烂皮肉塌了形状,眼窝伤口隐隐蠕动,一起向外流出腥臭脓血。院内响起一片惊叫,无心前方立时宽敞了一大片。

顾大人、月牙、以及卫队,一起向后退了老远。三只大黑狗夹了尾巴,从喉咙里面呜呜咽咽。公鸡倒还老实,并没有振翅鸣叫。无心放下脑袋,开口说道:``顾大人,你答应谢我一万大洋,不赖账吧?''

顾大人吓得想要含泪杀人,舌头都打了结:``不、不赖帐!''

无心点了点头,不知为何,看起来有点心不在焉:``好,谅顾大人也不敢。谁去找些干柴过来?''

顾大人立刻派出了身后的卫兵找柴。无心站了起来,不知是因为在冷水里泡久了,还是因为衣裳特别黑,他看起来是出奇的苍白,也带了几分鬼气。转身弯下腰扶住井沿,他把头向下探去,看到一个小小的水泡在黑沉沉的水面上破裂开来。

他没有动,继续等待,片刻过后,缓缓的又升上来一枚气泡。不动声色的闭了眼睛,无心除了井水,没有感觉到任何陌生魂魄。

直起腰面对了众人,他开口问道:``顾大人,搬进这所宅子里后,府上吃过这口井里的水吗?''

顾大人连连摇头:``没吃过没吃过,我们吃的都是胡同口甜水井里的水。刚搬进来的时候,厨子倒是从这井里里面打过一桶水,水混,有股子腥气,看着就不干净。不过都说这口井方位不错,所以我也没让人填了它。''

无心又问:``这处宅子一直风平浪静,只在近两个月才开始闹鬼的?''

顾大人皱着眉头``唉''了一声:``要是一直闹鬼,还能瞒得住人?街坊邻居不早就都知道了?我买房子的时候,左邻右舍都住得挺好;可是自打两个月前闹了鬼,你出门看看去吧,左右两家都没人了。说是一户回了乡下老家,另外一户跑天津去了。''

无心听得十分迷惑——大凡鬼要修炼成煞,免不得要吞没许多冤魂,然而人死成鬼的事情不算罕见,鬼本身也没什么稀奇,新鬼甚至连吓人的本领都没有,非得年深日久,力量壮大了,才能作怪。从鬼到煞,至少要有个几十年才能修成,而宅子里面先前并不闹鬼,可见女煞不是一直凶残,起码在两个月之前,女煞应该是另找孤魂野鬼来吃,并不伤人。可是这两个月到底发生了什么事情,让女煞性情大变呢?

这时卫兵抱着一大捆柴禾回来了。无心走去把柴禾一层一层的架好,然后回到井边拎起女煞的头颅躯干,放在了柴禾堆上,眼看就是放火要烧。卫兵察言观色,立刻把一盒火柴送到了他面前。他接过火柴,却是向着门口挥了挥手,口中说道:``都到前院等着吧,火一起来,这里会非常的臭。''

在场丘八本来不怕尸首,可现在不是练胆子的时候。眼看顾大人迈步向外走了,他们立刻跟了上去。月牙还抱着小黑狗,对着无心张了张嘴,一时也不知说什么才好,故而犹豫一下,也跟着出去了。

无心跟上去关了院门,随即脱下黑色衣裳,盖在了女煞的残体上面。阳光立时被遮住大半,无心蹲回原位,垂下头闭上了眼睛。

真正的眼睛一闭,他的周身便全是眼睛了。

鬼怕日光,见光便散。然而煞有了实形,虽然在阳光下也逃不过魂飞魄散的结局,但是身躯既由魂魄练成,身躯不散,魂魄便也能多存一阵。他看见女煞此时已然只剩下了两魂五魄,全凭着自己的黑衣挡了日光,才减了许多痛苦。抬手抚过高低不平的黑衣表面,他在心中向对方的残余魂魄说道:``不要怕,我不是段三郎。''

魂魄在黑衣下面战栗着做了回应:``不要伤害她\ldots{}\ldots{}不管你是谁,不要伤害她。她死的很惨,她已经赎罪了\ldots{}\ldots{}''

无心问道:```她'是岳绮罗?''

魂魄像一团光,闪烁的越发激烈了。

良久过后,黑衣也抵挡不住正午阳光的照射了。

无心对着女煞低声说道:``无论你所言是真是假,我都已经留不住你。走吧,魂飞魄散,一笔勾销,多么好。''

随即他伸手抓住衣领,猛然一掀!

耳中隐隐响起一声惨叫,女煞的魂魄在烈日之下无处遁形。而无心睁开眼睛划了火柴,一把火点燃了女煞身下的柴禾。烈焰腾空而起,无心盘腿坐在浓烟之中,轻声开口说道:``我真是天下第一大好人,你们活,我来陪,你们死,我去送。虽然你死后成了恶鬼凶煞,可是我也给你念一段往生咒。''

垂下眼帘清了清喉咙,无心微微仰起脸面向了太阳。干柴烧出噼噼啪啪的炸裂声音,而他低吟浅唱的声音却是穿透沉滞黑烟,被飘逸而出的魂魄一直带去很远很远。一门之外便是月牙、顾大人和他的卫兵们。无心平日声音清朗,念起经来却是带了一点嘶哑,众人一起静静倾听着,听无心把往生咒念得这么悠远、这么苍凉。

柴禾还未烧尽,女煞的残躯便已彻底消失,连一片灰都不曾留下。无心仔仔细细的穿好上衣,遮住了胸前的伤。喉头也被女煞狠咬过一口,好在咬的偏下,也能用衣领遮掩一阵。手心的刀伤已经开始愈合,他走去井边再次低头望下,结果又见到一枚晶莹剔透的小气泡炸裂开来。

女煞最后给他讲了个不怎么动听的小故事,可信度也不大高。不过,有点意思。

无心身上疼,肚里饿,决定先去吃顿好饭,顺便把钱收了。转身走去推开院门,他对着顾大人一笑:``灰飞烟灭。''

顾大人刚把两只公鸡卸下去了。一身轻松的走到无心面前,他扬起大巴掌就拍上了对方的肩膀:``完了?''

无心没有正面回答,只说:``先吃饭,吃饱了再说!''

顾大人欢天喜地,直接返回司令部。无心和月牙坐上汽车,月牙还抱着狗,一路也不说话,单是悄悄的盯着无心瞧。看完一眼,再看一眼,心里莫名的很知足。

无心生平第一次坐汽车,新奇极了,顾大人理直气壮的坐在后排正中央,因为月牙一直横着瞟人,他便沾沾自喜,以为仙姑已经被自己英俊的侧影所折服,只是另一侧的无心摇头摆尾,十分闹人。及至汽车开到司令部门前,顾大人和月牙都下车了,无心还赖在车上东翻西摸;顾大人也饿了,气得拉开车门骂道:``不要像个土包子似的,快点下来!''然后他又转向月牙,正色说道:``本司令摩登惯了,最看不得土鳖。''

月牙没理他,低头退了一步。顾司令一说话,两只眼睛就对着她的胸脯和细腰使劲。他要不是个大军官,她能挠他。

等到无心在车上坐够了,一行人进了司令部正房。正房里面支起桌子,饭菜已经摆好。无心很自觉的又去洗了洗手脸,然后坐下来抄起筷子便吃。狼吞虎咽的大嚼了一场,他忽然对顾大人问道:``你一定要搬回去住吗?''

顾大人愣了一下:``那宅子挺好的,为什么不住?''

无心不置可否的往嘴里扒了口饭:``我感觉\ldots{}\ldots{}那个地方不大干净。''

顾大人登时变了脸色:``啊?什么意思?''

无心放下饭碗:``那地方在上百年前,惨死过人。''

顾大人瞪着眼睛看他:``不就是那东西吗?''

无心摇了摇头:``惨死的不是一个人,死不是好死,埋也不是好埋\ldots{}\ldots{}这么着,你先吃,吃完了我再和你细说。''

顾大人把筷子往桌上一拍:``听了你的话,我心都拧起来了,还吃个屁啊!''

月牙不声不响的看了无心一样,心里怨他多嘴——反正该办的事情都办到了,有钱没钱都是小事,赶紧离开才是正经。两个人年纪轻轻的,远走高飞之后还怕没有活路?

\chapter{爱情故事}

午后天热,顾大人命令勤务兵在西厢房的大炕上摆了一张小炕桌。盘腿坐上炕去,他拎起茶壶先倒出了三杯冰凉的碧螺春,然后从衣兜里摸出一根明晃晃的小金条,``咚''的一声扔到了桌上。

无心赤着双脚也上了炕,又叫月牙过来坐。月牙不愿意和两个爷们儿围一张桌子喝茶,所以就不声不响的坐到了炕角,低头摆弄着两条九成新的绸缎手帕,想看看能不能用它缝个好荷包出来。无心端起茶杯喝了一口,发现茶水里面还放了糖,又甜又清香,就主动端起一杯,转身过去一直送到了月牙身边。

月牙没吭声,可是就像受了吸引似的,一双眼睛不由自主的总要往他身上瞄。忽然见他手心上面横了一条浅淡泛白的小伤口,她登时记住了,暗想等到顾大人出去了,自己得去给他瞧瞧,皮肉伤遭了水,可是爱闹炎症。

她不说话,无心也不说话,四脚着地的爬回了炕桌旁,和顾大人相对而坐。顾大人见自己那根金条无人问津,就伸手将其向无心一推:``谢礼,收着吧!''

无心本来说好要在饭后讲个小故事的,现在讲故事的排场都摆开了,他却又不急了。对着金条扫了一眼,他不动声色的说道:``一条小黄鱼,也不值一万大洋啊!''

顾大人素来是凭着刀枪讲道理,前两天他怕极了,别说一万大洋,十万大洋他也肯答应;但是今天中午他眼看着女煞被无心烧成了灰,心中的恐慌随之烟消云散,不由得本性上升,跃跃欲试的想要赖账。大模大样的对着无心一笑,他开口答道:``哼哼,本司令的钱也不是大风刮来的,明晃晃的十足真金,能说拿多少就拿多少吗?''

无心向他一探头,满脸都是阴沉神色:``顾大人,你要食言?''

不等顾大人回答,无心闭上双眼一扯右臂衣袖,右手食指蘸了茶水便在桌面上乱画起来,同时口中开始嘀嘀咕咕。顾大人见状,吓了一跳:``哎?你干什么?''

无心沉着脸,从牙关中挤出回应:``我咒死你!''

顾大人立刻伸出两只大巴掌,左右夹攻一把握住了无心的手:``别别别,我跟你闹着玩的!实不相瞒,我的钱在我姨太太的小公馆里,我晚上就去取,我再给你九条小黄鱼,说假话天打雷劈!''

无心睁开双眼,从顾大人的双手中抽出右手。手掌一抹桌面水渍,他拿起金条爬回月牙面前,把金条直接送到了月牙手里:``你收着。''

随即他调头爬回桌边重新坐好,皮笑肉不笑的一拍桌子:``原来顾大人是在我和闹着玩啊!哈哈,顾大人你真诙谐。''

顾大人把嘴一咧,苦涩的一笑,心想我买宅子也没花一万大洋。颇为尴尬的清了清喉咙,他很不自在的转移了话题:``师父,你不是说要给我们讲个小故事吗?讲讲吧,我这心里一直惦记着呢!''

无心点了点头:``好,故事不长,请顾大人和月牙都仔细听一听。故事说的是一百多年前,有个小小的京官,姓岳,受了陷害,被朝廷贬来了文县。京官有个庶出的小女儿,名叫绮罗,幼时常说自己前世如何如何,说得很真,家人听的惊恐,所以全都不甚喜爱她。及至她长大了些许,前世的话倒是不大提了,性情却是变得顽皮淘气,家中只有一个小丫鬟和她最好。京官来到文县之时,绮罗已经满了十三岁。一日岳家女眷乘了大马车去城外庙里上香,绮罗遇上了一位段家三郎。三郎英俊,绮罗秀美,两人就看对了眼。回城之后,绮罗和三郎想方设法见了许多面,渐渐爱成了死去活来。然而段家亲自登门向岳家提亲了,京官却是坚决不允,因为段家寒微,双方不能匹配。亲事既然不成了,绮罗便暗里和三郎做了约定,不能同生,便要共死。一天夜里,绮罗私自出门见了三郎,两人到了僻静地方,各自拿了刀子要抹脖子。哪知三郎一刀子真割下去了,绮罗却是生了怯,不肯动手。三郎死后,绮罗独自逃回家中,只对小丫鬟讲了此事。风平浪静的过了一年,岳家女眷照例又去上香,不料众人一时疏忽,回城时竟发现绮罗和小丫鬟双双丢了!''

说到这里,无心暂停下来,转而问道:``两位,你们有何评论?''

顾大人先开了口:``段家死了个儿子,就不声不响的算了?段三郎说死就死,也没给家里留句话?''

顾大人说完了,月牙才在炕角接着说道:``我看绮罗不是什么正经东西,十三岁就知道跟男人相好。再说俩人都定好了一起死,她既然胆小,怎么不想着提前拦一拦三郎?她不是喜欢三郎吗?就忍心眼看着三郎死了?三郎死了她还自己回家,安安生生过了一年?真没长心!''

无心等到二人都说完了,才继续又问:``那你们再猜一猜,绮罗和小丫鬟,是丢到哪里去了?''

月牙猜不出,顾大人迟迟疑疑的答道:``你要是原来问我,我肯定说是被人劫走了;但你现在问我,我就有点犯迷糊——总不会是被鬼抓了吧?''

无心端起茶杯喝了一口:``基本没错,她们是被段家的人掠去了。段家的方法,这里也不必细说,总而言之,就是趁着她们落单,使了迷香之类的手段。顾大人想的对,三郎殉情之前经过深思熟虑,当然会留下遗书,对父母做一番交待——''

不等无心把话说完,顾大人一拍桌子:``哎呀,那绮罗和丫鬟全完了,还不得被人先奸后杀?''

月牙本来也打算发些议论的,然而听到顾大人的妙语之后,立刻把脸一红,决定不再和他们掺和。

无心微微一摇头:``段家认为三郎全是绮罗害死的,所以把绮罗活着钉进了棺材里。那时候文县还没有这么大,棺材被埋进荒地之后,小丫鬟也难逃一死,被段家挖了眼睛,塞进了旁边一眼小小的水井之中。''

意味深长的看了顾大人一眼,无心忽然笑了一下:``段家从此销声匿迹,而岳家闹了一阵,找不到人,也就罢了。后来文县日益繁华,那片埋了绮罗尸骨的荒地渐渐起了人气,有了房子又有街,最后竟然也成了个热闹的好地方。''

顾大人白了脸:``荒地\ldots{}\ldots{}不会就是我家吧?''

无心笑吟吟的答道:``女煞当时已经收不住魂魄,时间有限,就只对我讲了这些。我想如果小丫头死后修炼成了女煞,那绮罗呢?''

顾大人直着眼睛发起了呆,而月牙在角落里发了话:``不好说,反正绮罗没有小丫鬟冤。''

无心知道她很看不上绮罗的所作所为,正要回答,不料顾大人忽然又一拍桌,怒发冲冠的骂道:``妈了个×的!老子活了二十八年,还没有受过这样的气!老子花钱买的宅子,那两个做了鬼的臭娘们儿又没出钱,凭什么老子不能住,要留给鬼?一百多年前的烂事,和老子有个屁关系?我告诉你们,本司令受够了!明天上午我就带一个营过去,掘地三尺埋炸药,管它水井棺材,炸没了算!''

说完这话,顾大人伸腿下炕穿了鞋,气冲冲的就往外走。无心并不拦他,趁着清静挪到了月牙身边。

月牙见顾大人真走了,不由得也松了口气。扯着衣袖拽过无心的右手,她正要去看对方的伤,然而定睛一瞧,却发现对方掌心平整,根本无伤。

她怔了一下,立刻望向无心的左手,无心的左手随意搭在炕上,掌心向上,也是完好。月牙自认为眼神很好,方才不会看错,可是方才没错,此刻也没错。连忙松开了无心的袖口,她又是疑惑,又是不大好意思。从口袋里掏出金条送到无心面前,她低声说道:``你的东西,你自己收着。''

无心把金条拿起来放回了她的手帕上:``不,你收着。''

月牙垂头说道:``丢了我可赔不起。''

无心对着她微笑:``我的就是你的。''

月牙像头牛似的,也说不出巧话,就单是脸红:``我不要。''

无心蹲起来,抱着拳头向他拜了拜:``求求你了,你要了吧。''

月牙浑身都发烧了,耳语似的哼唧道:``挺大个男子汉,一点儿都不值钱,说求就求。''

无心立刻用手帕包起金条,塞进了月牙的手里。顺势握住了月牙一只手,他美滋滋的不肯松开。月牙如今无依无靠,婚姻大事全凭她自己做主,所以他想让月牙尽快爱上自己,一旦爱上了,为情所困,想必就不会轻易离开了。然后他垂下脑袋,饶有兴味的又看了看月牙的手,月牙干惯了活,手比脸糙了许多。不过无心情人眼里出西施,只要月牙肯和他过日子,哪怕再丑十分,他也心满意足。

月牙任他握着手,一颗心快要从喉咙口蹦出来,不知为何,竟然慌得浑身肉颤。强挣着挤出了声音,她的面孔已经热到发烫:``一根金条就不少了,咱们\ldots{}\ldots{}走吧!''

无心并不是贪得无厌的人,如果顾大人一定要在酬金上面纠缠不休,他也懒得奉陪到底。用力攥了攥月牙的手,他轻声说道:``明天我们就可以走,今晚我还想再去宅子一趟。''

月牙猛一抬眼:``又干啥去?''

无心安抚似的松手拍了拍她的膝盖:``你别怕,我就是去看一看,不会惊动了谁。若是里面真没什么,那明天我们早早就走,顾大人爱怎么干就怎么干,我也不管了。好不好?''

月牙认为很不好,可是俩人毕竟还不是两口子,有些话她说不出口。

\chapter{井中密室}

无心要去夜探深井,顾大人没拦着,月牙想拦又拦不住。到了傍晚时分,顾大人以取金条为借口逃之夭夭,月牙守着一根金条坐在屋里,因为生平还不曾拥有过如此巨大的财富,所以谨慎得都不敢乱动。无心脱了缰,自己骑着马就去了宅子。

无心活了无始无终的这许多年,人见多了,鬼也见多了,无论人鬼,他都不会轻信。女煞生前作为一名冤死的小丫鬟,中午都要魂飞魄散了,还满口回护着岳绮罗,可见岳绮罗在她心中,比她自己更重。岳绮罗死得惨,难道她就死得轻松了?她在先前的上百年里一直安静修炼,近两个月怎么就急得开始杀起活人了?

宅子门口守着两名卫兵,虽然知道宅子干净了,但还是死活不肯进门一步,倒是正合了无心的心意。下马之后进入宅门,他形单影只的一直走到后院,见地面还余着焦黑灰烬,余晖之下,宛如火后残骨。

夕阳不落,阴气不起,纵是有了鬼魅,也不会出现。无心是来找鬼的,所以慢条斯理的脱了衣裤鞋袜,赤条条的又蹲上了井台,一边等着太阳下山,一边向井内水中张望。井中黑洞洞的深不可测,一串气泡漂浮上来,破裂之后再来一串。

天终于黑了,一轮明月升上了半空。夜空是黑丝绒,明月是白玉盘,周围散落着几点散碎星星。夜风清凉袭人,此刻虽然黑暗,却是一天中最为舒适的时候。无心很惬意的呼出一口长气,然后双手按着两边井沿,双脚向下坠入了井中。

井水之中少了盘旋长发,让无心行动起来自如了许多。沉到井底定了定心神,他睁开双眼望向前方,看到了一面平平整整的石壁。双手拨水向前游去,他停在石壁前方,没有轻举妄动,心里则是想起了女煞上午最后的举动——女煞疯狂的去撞石壁。

如果小丫鬟的目的是要撞破石壁,那非得修炼成女煞不可,否则没有实体,拿什么去撞?纵算魂魄可以穿墙,但是石壁上面八卦赫然,必定是有些威严力量,不许邪祟之物靠近,而小丫鬟大概是本领有限,以至于撞碎了半个身体还不成功。若是由着她再修炼几年几十年,兴许会有破壁的可能;而小丫鬟行为有异,难道就是因为心中急切、等不得了?

无心一边思索,一边上下审视着壁上八卦。八卦就是八卦,中间围着阴阳鱼,乍一看也无甚特别。无甚特别,却能挡住鬼煞,说明必是画它的人法力高强。向前凑近了些许,无心仔仔细细的将八卦细节又看了一遍,末了却是一惊——八卦图和阴阳鱼全是反的,而黑白二鱼的鱼眼,则被统一涂成了血红!

无心一直感觉石壁表面萦绕着一层纯阳之气,专克妖魔邪祟;万没想到纯阳之气虽然不假,可却是以毒攻毒,以至阳的法力布了个至阴的邪阵。一动不动的悬浮在水中,无心认为无论石壁后面镇着个什么,布阵之人都有些小题大作了。

一串气泡又掠过了眼前,无心沿着水泡的踪迹追寻来历。歪着身子越发靠近石壁,他在血红鱼眼处发现了一道细微裂缝。裂缝仿佛妇人生产一般,一枚一枚的分娩出小小气泡。

无心没敢妄动,心想女煞撞破石壁,是为了杀,还是为了救?如果石壁后面是岳绮罗,``杀''不大可能,因为小丫鬟魂飞魄散之前还求自己不要伤害岳绮罗。不是杀,就是救,可怎么救?岳绮罗已经死了一百多年,尸身早就烂没了,莫非魂魄被困在石壁后面,不得转生?

无心记得小丫鬟说过段家寒微,似乎只是平常门户,既然如此,怎会又杀人又做法?就算要给儿子报仇,一刀剁了岳绮罗也就是,何必大费周章?到底是岳绮罗有问题,还是段家有问题?

无心实在是想不明白了,眼看鱼眼鲜红异常,不知是用什么颜料涂抹的,浸在水中也不脱色。忍不住伸出一根手指,他堵上鱼眼裂缝轻轻蹭了一下;然而还未等他收回手指,忽然就听一声天崩地裂之响。排山倒海的气流爆破石壁鼓荡而出,井水混着大小石块,在气流的搅拌下一边旋转沸腾,一边滔滔的涌入石壁后方的干燥空室之中。无心随波逐流进入空室,就见室内四壁灰白平坦,龙飞凤舞的画满漆黑符咒,正中央停着一口腥红棺材,棺材不但被铁链道道捆住,而且周遭贴满黄符。晕头转向的被水流石块直冲向前,无心身不由己,猛的直撞到了棺材头上。忍着疼痛扶住棺材,无心总算有所依附,哪知棺材并未钉死,他就见棺盖在铁链的松松束缚下缓缓向后滑去,而一阵气泡直冲上来,带得两张黄符漂漂浮浮,正巧盖在了棺内之人的面孔上。无心一眼望去,就见对方穿着大镶大滚的旧式女装,两只手向上举起,蜷曲成爪,居然并非腐烂,骨肉俱全,正是个抓挠棺盖的姿势,可见此人十有八九便是岳绮罗。艰难的腾出一只手,无心想要揭开黄符去看对方面孔,不料一块大石顺流而至,正中他的脊背。他疼得双手一松,当即随着水流翻滚而上。张牙舞爪的在室内转了一圈,他在慌乱中只抓住了一张泡软的黄符。有心游回棺材上方再去查看,可是井水翻腾得厉害,并不容他自由行动。``咣''的一头撞上墙壁,他像条大鱼似的在水中打了个挺,随即哭丧着脸抬手捂住了额角。还未等他熬过疼痛,又一阵水流直冲过来,把他向前卷回了井下。

无心仰头向上游去,不敢再在水中停留。水流东一股西一股,力道惊人全无方向,他潜下去也是无用,只会撞出一身的皮肉伤。密室的邪门是不言而喻的,其中的玄机却是一时难以窥透。无心撑着井壁爬了上去,累倒不是很累,只是周身作痛。

水淋淋的坐上井台,他低头吐出一口井水。仰头又看了看天上星月,他忽然发现自己手中还攥着那张黄符。

黄符厚而柔韧,虽然经了水,但是不会立刻糟烂,可见不是普通黄纸。无心展开黄符看了一遍,见上面弯弯曲曲乱画一气,因为不懂,所以也无须细瞧。黄符大概是本是贴在棺材上的,棺盖一动,导致黄符散落。抬手向下一抹脸上的水珠,无心忽然起了疑心:``我捅破了石壁,又撞开了棺盖\ldots{}\ldots{}我是不是闯祸了?''

一转身俯向井口,他闭上眼睛,并未感觉到有魂魄出没,阴风寒气倒是依旧。

起身穿戴整齐了,他见黄符完好无损的挺结实,就将其叠起来也塞进了衣兜里。心想等到明日顾大人过来大炸一场,就算地下真有邪祟,想必见了火光日光,也无生路可逃。

思及至此,无心便湿漉漉的离去了。

无心骑马回了司令部,发现顾大人还没回来。摸着黑进了西厢房,他没开电灯,眼看炕上有人坐起来了,他连忙说道:``我什么事都没有,你睡吧,我也要睡了。''

屋里黑灯瞎火的,月牙听他语气平和,就放心的又躺了回去。无心蹑手蹑脚的上炕躺下,因为一时睡不着,于是望着月牙的背影发起了呆。

他眼神好,窗外又挂着一轮大月亮,所以他将月牙的背影看得十分真切。月牙侧身蜷着两条腿睡觉,腰太细了,显得屁股圆滚滚。无心一直认为月牙的身材像个葫芦,他想抱着葫芦睡觉,或者被葫芦抱着睡觉;两人挤着一个热被窝,你疼我我爱你的总在一起,多么好。

小心翼翼的向前挪了挪,他决定明天就带着月牙离开文县,只要有了伴儿,去哪里都可以的。

无心浮想联翩,从月牙想到葫芦,从葫芦想到被窝,想得沾沾自喜,连疼都忘了。及至想的差不多了,他心思一转,又回到了井里。

水为阴,深井加上冤魂,更是阴上加阴,加之一百年前周围荒凉,人气衰弱,所以井中阴气简直堪称纯粹。无心无意中把手伸进衣兜,摸到了又潮又软的黄符。心中忽然一动,他想当初段家的所作所为哪里只是单纯的复仇?分明就是凑齐了天时地利人和,专为了整治岳绮罗一个人!

不是杀,而是整治,如果岳绮罗真是人的话。

无心经过无数离奇事情,见怪不怪,想不出头绪,也就懒得再想。迷迷糊糊的闭了眼睛,他正要强迫自己入睡,不料窗外忽然响起一声惊天动地的大爆炸,气流冲击之下,窗户玻璃尽数粉碎。无心猛然坐起,就见外面腾起硝烟火光,伴随着年轻卫兵的狂呼乱叫。

月牙被崩了一被面玻璃渣子,幸而头脸安然无恙。嗷一嗓子坐起来,她六神无主的一把抓起枕边包袱,就听无心叫道:``月牙,下地!''

月牙吓得没了主意,可是手忙脚乱的很听话。慌里慌张的下地穿了鞋,她手上一紧,已被无心用力握住。无心把她护到身前,弯着腰就要带她往外跑。一脚跨出房门去,他听外面有人带着哭腔嘶喊:``司令呢?司令呢?张团长反了,张团在大街上开战了!''

无心不作停留,一鼓作气把月牙推出了司令部院门。沿着道路跑出没多远,忽听身后又是一声巨响,无心和月牙回头一看,发现司令部又中炮弹,半边房院都被夷平了!

\chapter{小两口}

无心已经许久没有遭遇过战火,没想到现在的枪炮如此厉害。眼看街上接二连三的爆起开花雷,他不敢停留,拽着月牙就往暗处跑。月牙胜在腿长脚大身体好,无心跑多快,她也跑多快,完全不拉后腿。一鼓作气不知逃出了几条街,无心开始遥遥的见了兵。

月牙小时候经过好几次兵灾,最怕丘八大爷们过境闹事。单手死死的把小包袱捂在胸前,她喘着粗气叫道:``当兵的要抢铺子了!''

街上闹得越厉害,四周的住宅越死寂。家家户户都黑了灯,噤若寒蝉的关了院门待宰。无心索性带着月牙拐进一条幽深胡同,胡同弯弯曲曲四通八达,他最后停在一棵黑黢黢的老树下面,搂着月牙蹲下了身。月牙的鬓角碎发都被汗水打湿了,一绺一绺的贴在耳边。口鼻之中呼出热气,她惊恐的瞪大了眼睛,极力想要屏住呼吸,连条野猫野狗都不敢惊动。耳边响起了无心的声音,无心告诉她:``别怕,当兵的都在大街上杀人放火,小胡同里要什么没什么,他们不会过来。''

月牙气咻咻的点了点头,也知道自己现在还算安全。下意识的又往无心怀里缩了缩,她恨不能在老树下面隐身。远远的起了一排枪声,她像是受了某种震动一样,忽然发现无心太安静了。

到底是怎么个安静法,她说不出来,总而言之,就是觉得他静。呼吸渐渐缓和下来,她在暗中轻轻靠近了无心。一场狂奔过后,她的脸蛋热得要起火,需要一点凉风的吹拂。

她不动声色的等了足有两三分钟,两三分钟之中,无心一口气都没有喘!

月牙的汗毛骤然竖起了一层,正在她要出言质问之时,无心突然低低咳嗽了一声,随即又打了个哈欠。

``完喽!''无心的气息活泛起来了,凑在月牙耳边嘀嘀咕咕:``顾大人今晚要是死在兵变里,我就算是给他白忙了一场。''

说这话时,他依旧亲亲热热的和月牙偎在一起,可是稍稍侧了身,不让月牙靠上自己的前胸。

月牙又出了一层透汗,出得畅快淋漓一身轻松,心想自己真是吓懵了累坏了,居然还怀疑起了无心的身份。无心能吃能喝能晒太阳的,难道还会是鬼不成?

``行了!''她一拍怀里的小包袱:``这就够——''

后面的半截话被她强行咽了下去,她想说``这就够咱们置办个家了'',可是大姑娘哪能主动说这个话呢?一拧薄薄的流水肩,她转移了话题:``你别搂我。''

无心轻轻的笑,手臂搂她搂得更紧了。月牙不理他,不料肩膀忽然一沉,却是他得寸进尺,歪着脑袋枕上来了。

月牙最受不了他这种小孤儿式的赖皮,好像全天下除了自己,就再没人肯要他了似的。
若无其事的一动不动,她由着无心把脑袋蹭上了自己的脖子,短短的一层发茬戳得她心疼。

两人在树下避了许久,直到天边隐隐有亮光了,胡同外面也彻底安静了,他们才起身试试探探的向外走去。

大街上正是一副劫后余生的惨象,体面的大商号全受了损,隔三差五还能见到断壁残垣冒着黑烟。尸首光明正大的躺在道路中央,比活人还要理直气壮;活人反倒成了鬼魅,悄无声息的游荡而出,有的抬尸首,有的翻废墟。

无心不让月牙乱看,怕她害怕,自己领着她快步往前走。无论夜里的兵变谁输谁赢,他都不在乎了。搂着月牙蹲了一夜,他现在只想快点远走高飞,和月牙过日子去。

城门大敞四开,盘查森严。月牙留了心眼,提前从包袱里掏出小金条藏在了身上,又在地上抓了把土,把自己抹成灰头土脸的样子。及至到了城门口,小包袱果然被士兵打开来检查了,当然是只有几件衣裳,并无其它。

出了文县,有两条路,一条路通往平镇,月牙的家就在那里,自然决不能去。两人商议一番,末了就决定前往相邻的长安县。长安县比文县还要繁华,那么热闹的大地方,三教九流俱全,自然也容得下他们一对小男女。

迈开大步踏上路途,两人一口气走了一个时辰。眼看前方路边出现一处小小的饭馆,月牙便拿出自己当初离家之时所带的一点私房钱,虽然加起来只有一块多,但是足够一路的吃喝了。

所谓饭馆,也就是在凉棚下面摆了桌椅而已。无心和月牙坐在了角落里,要了两碗汤面和一屉包子,一边吃一边倾听食客们高谈阔论。原来文县兵变尚未结束,顾大人和张团长目前还在城内僵持,双方实力相当,以至于都不占上风。

无心对于顾大人是没意见也没感情,月牙更是几乎有些烦他,所以全不关心顾大人的死活,吃饱了就走。

从文县到长安县,中间几十里地,说远不远,说近不近。两县之间有个挺大的镇子,叫猪嘴镇,名字虽然不好听,可是挨着交通要道,还是个有名的地方。无心和月牙本意是到镇子里吃顿饱饭,好赶在天黑之前到达长安县;然而下午进了猪嘴镇,他们直到夜里也没出来。

镇边有户人家出租房屋,是一排三间砖瓦房,玻璃窗户,外面还带着个栅栏围成的小院儿。除了位置太偏僻之外,没别的毛病。无心偶然发现此处,一眼就看中了。月牙其实比无心还盼着有家,无心说好,她也跟着说好。于是一下午的工夫,金条换成九百五十大洋,不但租下了房子,而且连锅碗瓢盆米面肉菜都一并置办齐全了。房东认准了他们是私奔出来的小两口,故而十分识相,并不多问。

三间屋子,只有中间一间堂屋开了大门,堂屋东西通着两间卧室,格局大小都相同,统一的在窗下砌了火炕。堂屋里面空空荡荡,门口两边各有一眼大灶。月牙乐坏了,两口大灶全生了火,一边蒸饭一边炒菜。崭新的锅铲磕着锅沿,她心里有种无法无天的痛快——当初要是不逃,现在自己早进了马家的门了!给马老头子做姨太太,和给无心做正经媳妇,两种生活孰好孰坏,一目了然。

两人七碟子八碗的吃了一顿丰盛碗饭。月牙二话不说,收拾了碗筷就去洗刷,一切活计全不用无心插手。等到屋里屋外都收拾利落了,无心已经在西屋炕上铺了被褥,又喊:``月牙,来睡觉了!''

月牙应声而入,却是站在炕前对着无心正色说道:``咱俩还没成亲呢,不能糊里糊涂的就往一个炕上睡,往后想起来了,都不知道哪天算是洞房。反正我都跟你来了,我对你是啥心思,你也全明白。明天咱们翻翻黄历,挑个好日子,也不用惊动谁,你我一人换一身新衣裳,再放一挂鞭炮就行。''

无心蹲在炕上,把铺好的被褥推向一边:``那我们还像在文县一样,各睡一边好不好?''

月牙``哎呀''了一声,又是不耐烦又是笑,自己弯腰抱起一套被褥:``你急啥呀?我还能半夜跑了啊?''

不等无心挽留,她快步去了东屋。无心倒是没有追逐——其实就算睡在了一个炕上,今夜他也不会去动月牙。他的底细迟早是瞒不住的,而在真相大白之前,他不能真碰月牙。

屋子里面渐渐安静下来,东西两屋的油灯也都先后灭了。无心没想到自己如此轻易的就安了家,心里高兴的睡不着。躺在炕上辗转反侧了一阵,末了他坐起身来,想要透过窗子看看月亮。

不料就在他靠近窗子的一瞬间,他忽然发现院门外面站了个人!

人不大,还没有门高,若不是栅栏稀疏,无心简直看不到。小人儿梳了两条垂肩的辫子,想必是个小姑娘,衣裳却是穿得乱七八糟,外面甚至套着一件男人的短褂。无心看不清她的面孔,只见她一动不动的站在清冷月光下,直对着自家院门。

她不动,无心也不动,静静的紧盯着她。如此过了良久,小姑娘像是看够了一般,姿态娇俏而又飘逸的转身便走。月光之中无心看得清楚,就见在她破烂凌乱的粗布裤脚之中,刹那间闪过一只鲜红底子绣金花的小鞋,倏忽而逝,鲜艳的像一点血。

无心眼看小姑娘越走越远,因为不明就里,所以若有所思的躺了回去。伸手从衣兜里摸出那张黄符,黄符早已彻底干燥了,他将黄符展开来看了一遍,依然是看不懂。

如果他是孤身一人,那来了什么他都不在乎;可是东屋里还睡着一个月牙,攥着黄符想了又想,他心中拉起了警铃。

\chapter{不速之客}

翌日天刚一亮,月牙就起床了。

她没有惊动无心,抄起笤帚扫了屋子扫院子。昨天买的一堆劈柴整整齐齐摞在院子角落,劈柴旁边的竹篮子里放着昨天买回来的小黄瓜小萝卜,一夜过后还是很水灵。

炉子里面生起了火,大铁锅里很快就咕咕嘟嘟的出了声音。月牙按照惯例,差一点就要煮粥了,可是转念一想,她把锅里的水又舀出许多——现在她是一家的女主人了,没人看着她管着她了,她可以随心所欲的多放米少放水,给她男人吃干饭。

无心早上一出卧室,就有净水摆在院子里让他洗漱。等他回了堂屋,房东留下的旧木桌也支起来了,上面摆着两碗米饭和一盘凉拌黄瓜。月牙进了西屋,正跪在炕上叠被,心想无心关门睡了一宿,房里居然丝毫不臭——李家从她往下,都是男孩,弟弟们的臭脚丫子和臭响屁可真是让她受惯又受够了。

下炕出门回了堂屋,她发现无心端端正正的坐在桌边,笑吟吟的望着自己不说话,一张脸白白净净的十分好看。月牙表面装成浑不在意,心里却是美得不行。走到无心对面坐下来,她垂下眼帘盯着米饭,无心的影子浮现在了心中,她对着自己的心,食不甘味的将他细细的端详。

早饭过后,两人并肩出门,去采办所欠缺的应用什物。月牙的脸蛋上透着两片似有似无的红晕,总像是在害热,可是天气并不算热,她的额上也没见汗。要买的东西就太多了,一时简直难以尽述。月牙预备先去布店,买了布好做新衣裳;然而无心另有主意:``正经成亲的话,也得有几件首饰才像样啊!''

月牙停了脚步:``首饰不顶吃不顶喝的,有没有还不都一样?''

无心不听她的,笑嘻嘻的把她往银楼里拽。两人在银楼里打了半天嘴皮子官司,最后月牙在现成的首饰里面挑了一副小小的金耳环。无心嫌少,不让她走:``我们有钱,再挑几样!''

月牙沉默了一阵,末了低头说道:``你要是真有心,就再给我买副镯子吧。戒指项链我都不爱,我就喜欢镯子。''

片刻之后,两人出了银楼,月牙耳垂上换了金耳环,手腕上也多了金镯子。走在通往布店的道路上,月牙告诉无心:``本来我娘有一副金镯子,还是我姥姥给她的陪嫁。我娘说等我长大了,就把镯子传给我。我七岁的时候我娘没了,镯子让我爹化成一条项链俩戒指,给我后娘戴了。''

无心知道月牙在娘家肯定是活得不容易,能把她送给老头子做小老婆的父母,想必平日也不会善待她。

月牙低头转了转腕子上的金镯子,又道:``我将来也要生个丫头,等丫头长大成人了,就让她把我的镯子带走,将来再传给我外孙女。''

无心默然的握住了她的手腕,手腕圆滚滚的有肉,显得镯子不甚宽松。他承认自己是太自私了——月牙直到现在,还是对他的秘密一无所知。

他的种子是死的,无论月牙的土地有多丰腴,都不可能孕育出生命的苗。月牙的镯子只能她自己戴,不会再有丫头和外孙女来继承。

猪嘴镇只有一家布店,布店里货物还算齐全,唯独缺少了大红的布,枣红和桃红倒是都有。月牙想要缝件大红的上衣做嫁衣,正经的新娘子,非得用大红才对劲。可是大红的布总要五天之后才能到货。月牙算了算日子,心想自己要做的活计还有很多,等上五天也没什么,于是扯了所需的几样布料,两人出门继续采购。

两人下午回家,到了傍晚时分,月牙连咸萝卜都腌进新坛子里去了。吃饭之前她把无心叫进东屋,要量量他的脚,有了尺寸好给他做新鞋。无心欢欢喜喜的坐在炕上,两条腿向前伸得直直的,一双赤脚整整齐齐的摆出去,是个讨好卖乖的模样。月牙一手拿着木尺,忍着笑给他量大小,同时发现无心的脚很干净。无心自称是个孤儿,被老和尚捡回庙里养大;月牙认为老和尚肯定是个文明人,看把无心教育的多讲卫生。

量完了脚,顺便把身材也一起量了。月牙低着头,用木尺从无心的脚踝开始往上比量,嘴里一五一十的记着尺寸。无心的腿又长又直,腰腹收紧胸膛开阔,肩膀端端正正的带着威风。月牙心里都幸福死了,疼他都要疼死了。

吃过晚饭之后,月牙在炕边点了一盏小油灯,借着光亮给无心纳鞋底。一灯如豆,光明有限,所以无心就蹲在了窗旁的阴暗角落里,一句递一句的和月牙说话。纳鞋底子是个力气活,月牙捏着大针,把线扯得嗤嗤直响,纳了许久也未见多少成绩;眼看外面夜色越来越浓了,无心不动声色的斜出目光,瞟向了窗外。

月牙下午把玻璃窗子擦了一遍,分外透明。院门外面并没有人,只有一条野狗施施然的经过。

月牙打了个哈欠,把针线一圈一圈的缠上鞋底。回头看了无心一眼,她轻声说道:``该睡觉了,你回屋吧。''

无心犹豫了一下,随即说道:``你做个荷包好不好?我有一张平安符,想给你带在身上。''

月牙立刻下炕找来自己的小包袱,打开来翻出一只小小的绣花荷包:``不用做,我有。''然后她又把荷包向前递向无心:``好看不?还是我去年绣的呢!''

无心从衣兜里掏出黄符,折好之后塞进小荷包里抽紧了口。眼看月牙把荷包挂到脖子上了,他才安心的下炕穿鞋,回房去了。

月牙没有多想,吹灯睡觉。而无心回到西屋又等了许久,见院外始终无人,便也睡下了。

天亮之后,月牙照例早起。梳洗过后进了院内,她正打算从篮子里取两个鸡蛋炒一盘子,不料未等弯腰,忽听院门响了。

响声很轻,是迟迟疑疑的``啪啪''两下。她直起腰望过去,因为自己在猪嘴镇并无亲友,所以打了个激灵,怕是娘家人追了过来。可是透过栅栏细细一看,她放了心,原来是个破衣烂衫的小人儿。

走过去打开了院门,她认定对方是个小叫花子,可是低头一瞧对方,她不禁愣了一下——多漂亮的一个丫头啊!

小人儿比她矮了一个脑袋,和她一样也梳两条大辫子,身上脏,一张小瓜子脸却是莹白如玉,两道浓淡相宜的眉,一双秋水盈盈的眼,连两片粉红色的小薄嘴唇都是特别的嫩。抬眼望向月牙,她用细细的声音说道:``姐姐,我饿,给我点吃的好不好?''

月牙看不出她的岁数,十一二岁也是她,十三四岁也是她,是一朵花要开没开的年纪,看着真是又可怜又可爱。连忙把她放了进来,月牙搬了个小板凳让她坐在院子里,又问:``你家大人呢?''

小人儿仰脸对她摇了摇头,一双眼睛水汪汪的,总像是含着点泪:``家乡打仗\ldots{}\ldots{}我爹我娘都没了。''

月牙本来就看她招人疼,又听她比自己还要命苦,就回了堂屋,要从锅里拿出热好的馒头给她吃。而小人儿扫过她的背影,随即垂下眼帘,眼珠子悠悠一转瞄向了西屋窗户。

无心苍白的面孔赫然紧贴在玻璃后面!

小人儿浓黑的睫毛一挑,紧接着转向了走出来的月牙。双手接过月牙递过来的热馒头,她细声细气的站起来道谢,然后像一切饿坏的大孩子一样,把馒头仓皇的往嘴里塞。月牙真有心把她引进堂屋坐坐,可又嫌她太脏,怕她带了虱子。低头看着狼吞虎咽的小人儿,她叹了口气,心想今天自己能喂她一顿饱饭,可是将来她又该怎么活呢?不知道镇子里有没有人家愿意要童养媳,她都这么大了,不养都能当媳妇,真要是有好人家肯收留她,对她来讲,也是条活路。

月牙蒸的馒头很大,小人儿一个馒头没吃完,无心披着褂子走出来了。

月牙一边忙碌,一边向他介绍了小人儿的来历,他带听不听的洗脸漱口,对小人儿是一眼不看。小人儿也像受气包一样,蜷成一团啃馒头。

无心从月牙手里接过新毛巾,满头满脸的擦了一气,又端起水盆,把水泼到了小人儿身后的土地上。他认得出,小人儿就是前天夜里出现在院门外面的小姑娘。破衣烂衫没有变,只是脚上的红色绣花鞋不见了。

把水盆放回堂屋的脸盆架上,他忽然没了主意。把小人儿赶出去?怕是从此对方在暗自己在明,反而不利;让小人儿留下来?他正想和月牙好好过几天日子呢,留个来历不明的东西干什么?

无心对小人儿的感觉很不好,尽管小人儿坐在光天化日之下,并无邪祟之气。

无心素来相信自己的感觉,并且预感到小人儿必定要赖下不走了。

\chapter{各怀鬼胎}

月牙看出无心不爱搭理小人儿,不禁有点心虚。虽然他们是小两口,家里没有上人压着,可无心毕竟是老爷们儿,是家里掌柜的,掌柜的没发话,娘们儿是不该私自往家里放人,好在对方是个小丫头,放进来了也不犯嫌疑。

小萝卜腌过一夜就有滋味了,鸡蛋也炒出了黄澄澄的一盘子。两样菜肴摆在无心面前,她本来热了四个馒头,现在只拿出了一个,伴着一碗粥送给无心,又小声说道:``你吃你的,人家穷的没活路了,咱们能帮一把就帮一把呗!反正也不差她一口吃的。等我再给她一口水喝,就让她走。''

无心不置可否的抄起了筷子,夹起一块炒鸡蛋站起来,伸长手臂先往月牙嘴里喂。月牙愣了一下,就见他诚心诚意的对着自己微笑,是在眼巴巴的等待自己张嘴。月牙一下子就幸福的无可奈何了,吃了一筷子炒鸡蛋后自去忙碌。

无心坐下来,喝了一口热米粥,大声唤道:``月牙,你怎么不来吃?''

月牙把锅里余下的两只大馒头拿出来放在笼屉布上,包裹起来送出去,一直递到小人儿怀里:``给你,拿着路上吃吧!''

小人儿仰起了头,小猫似的双手接过馒头,细声细气的说道:``姐姐,让我再歇歇脚行不行,我过会儿就走。''

月牙不忍心撵她,况且光天化日的家里俩大人,院子里多个生人也没什么。

无心对小人儿一直视而不见,吃完早饭也不出门,径自回了西屋睡觉。月牙正在洗碗刷锅,忽然眼角余光瞥到动静,直起腰向外一看,她发现小人儿不知何时站了起来,正在扶着笤帚扫院子。

两人就此开始交谈起来,小人儿自称姓李,是家里的老姑娘,小名就叫小妹。月牙问她一句,她答一句,老老实实毫无迟疑。月牙笑道:``巧了,我也姓李。小妹,你多大了?''

小妹扫了院子,又去把散落的劈柴摞好:``姐姐,我十四了。''

月牙加意看了看她的身段——衣裳太多太乱了,看不出具体模样。不过有的姑娘发育晚,又是``孩儿面'',所以要说小妹是十四,也差不多。

小妹把院子收拾的整整齐齐,连坐过的小板凳都规规矩矩的放回了角落里。抱着两个大馒头对月牙深深一鞠躬,她仰起脸,用她一双水盈盈的眼睛看人:``姐姐,谢谢你。我歇好了,我要走了。''

月牙从小没有妹妹,刚和她闲扯了半天,扯的还挺得趣。小妹要走,她也不能挽留,也不敢问小妹的前途,因为明知道小妹出去了只能是继续要饭。送着小妹出了院门,月牙正要说话,不料天边忽然响起一声闷雷,却是来了雷阵雨的光景。

夏天的大雨来势最猛,能浇得人睁不开眼睛。理所当然的,小妹走不成了。

月牙以为雷阵雨下不了多久,没想到阵雨下着下着就转成了滂沱大雨。转眼到了中午时分,无心哈欠连天的出了西屋,一屁股坐到了饭桌前,屋里暗,他一双眼睛阴沉沉的陷成了坑。很不耐烦的扫了小妹一眼,他声音不高不低的咕哝道:``还没走!''

月牙有点不好意思,一边摆饭菜一边横了他一眼,又把筷子塞进他的手里:``吃你的吧!''

小妹胆怯的退到了门口,月牙也不敢让她上桌,给她盛了饭菜,让她守着灶台吃。无心吃饱喝足之后,又回了西屋。而小妹一边帮着月牙洗涮,一边轻声问道:``姐姐,大哥是姐夫吗?''

月牙被她问笑了:``还不是呢!''

大雨下了一下午,小妹进了东屋,月牙坐在炕上做针线活,她就蹲在地上,守着个小笸箩挑碎布头,可怜巴巴的察言观色,殷勤的让月牙很不自在。及至天色晚了,大雨势头虽然弱了许多,可还是淅淅沥沥的不停。月牙没了办法,自作主张的烧了一锅热水,让小妹洗个澡,换身干净衣裳,留下住一宿。

小妹乖乖洗了,洗得兴高采烈,是舒服感激的了不得的模样。两条大辫子因为脏乱的不可救药了,所以她和月牙一商量,月牙干脆抄起剪刀,给她剪了个齐刘海的短头发——新学校里的女学生,现在全都剪发,小妹算是赶了个时髦。

剪了头发,月牙又检查了她的头皮,倒是没见虱子。而她穿上月牙的旧衣,虽然不大合体,但总比先前一身破烂好了千万倍。吃过晚饭之后,无心进了东屋,上了月牙的炕,像昨夜一样陪到她的身边。颇为生硬的聊了几句之后,他下炕回西屋去了。

他在的时候,月牙也觉得小妹挺碍事;他一走,月牙又觉得小妹是个伴儿。小妹凑到她的身边,拉拉扯扯的看她的镯子,看过之后天真的笑了,小声说道:``真漂亮。我大姐出嫁的时候也有一对镯子,比你的小多了。''

月牙挺得意,忍不住把镯子的来历讲了一遍,又撩起头发,让小妹看了自己的新耳环。小妹的头发干了,黑亮亮蓬松松,显出一种楚楚可怜的稚嫩。很艳羡似的轻轻摸了摸月牙的耳环,她垂下眼帘瞄了对方的胸前,没有再往近靠。而月牙显摆完毕了,收拾起了针线笸箩,开口说道:``趁着下雨凉快,咱也早点睡吧!''

小妹乖乖的``嗯''了一声,主动爬去铺开被褥。月牙吹了油灯,心里认为自己今天是做了好事,十分安然,又想小妹虽然小,可是真俊秀。无心也是个好样的,见了漂亮丫头毫不动心,一点奉承的意思都没有。

雨声淋漓,空气湿凉。月牙仰卧在被窝里,很快入了梦乡。小妹侧身直视着她,良久之后缓缓一眨眼,随即伸手摸向她的脖颈。脖颈隐隐可见一根五彩线绳,下面连着个香包似的小扁荷包。然而指尖都要触到五彩线绳了,她犹豫一下,把手又缩了回去。

凌晨时分,无心无声无息的坐了起来。

窗子傍晚就没有上闩,此刻被他伸手推了开来。起身赤脚踏上窗台,他轻飘飘的跳了出去。

踩着湿漉漉的泥水地走到东屋窗前,他停下脚步,向内望去。浓浓的黑暗之中,他看见月牙张着嘴正在酣睡,而小妹仰面朝天微抬双臂,手指蜷曲如同利爪!

无心冷笑一声,转身慢慢走回西屋窗前。伶伶俐俐的翻窗回房,他想岳绮罗真是在棺材里躺得太久了!

如此的妖孽,他先前似乎也曾见过,``似乎''而已,究竟见没见过,他也记不清了。女煞的话果然是信不得的——或许女煞自己也是蒙在鼓里。不知道岳绮罗追过来是什么意思,说起来自己也算是救了她,她总没理由恩将仇报。

无心不睡了,一直熬到天明。昨日下了半天大雨,今日天空一碧如洗,阳光明媚的让人睁不开眼。早饭桌上,无心依然是不理小妹,但是当着月牙的面,他开始鬼鬼祟祟的瞟她,一眼接一眼,全不是正眼。月牙留意到了,就有点不痛快,心想你昨天不看今天看,怎么着?看她今天洗干净有人样了?看在眼里拔不出来了?

家庭里的活计是干不完的。月牙昨天给无心做好了一件上衣,嫌新布有臭味,想要重新浆洗一遍。上衣泡在水盆里,她看小妹还没有要走的打算,就支使她去把上衣揉一揉。小妹蹲在院子角落里洗衣裳,洗着洗着,无心走过去,也蹲下了。

把手伸进水盆里,无心低声说道:``水凉,我洗吧,不用你。''

小妹没有动,手指头软软的在无心掌中一划,嫩得柔若无骨。无心抬眼看她,她的黑眼珠子在眼皮下面闪着水光一转,眼神像是阳光下的蜜,又甜又暖似有似无,仿佛是看了他一眼,又仿佛是没看。

无心温柔的和她争夺着衣裳,同时低声说道:``无处投奔的话,留下来多住些日子也无妨。''

小妹一歪脑袋,说起话来还是细声细气,可是吐字轻软,别有一种豆蔻初开的娇媚:``我怕大哥嫌我呢。''

无心抬眼看她,笑了一下,心想岳绮罗的小嗓子真够清甜,骂娘都能把男人骂酥了。

``我嫌你干什么?''他对小妹说道:``我不嫌。''

小妹的声音越发轻了,粉红的小薄嘴唇微微一撅:``你昨天不理我嘛\ldots{}\ldots{}''

月牙正在厨房煮淘米水,半晌不见无心出现,出门一瞧,发现他正和小妹相对而蹲,两人笑眯眯的搓着一盆衣裳。

她心里登时就不对味了,但因两人还未成亲,她顾忌着自己的姑娘身份,好些手段不便使出,所以压着一肚子醋唤道:``哎,你给我搬些柴禾进去。''

无心起身搬了柴禾,然后不等月牙说话,一转身又回到了小妹身边。月牙双手叉腰站在灶前,就觉形势变化太快,原来男人都是一个臭德行!

\chapter{岳绮罗}

月牙活了十七年,第一次正儿八经的吃醋。没想到吃醋的滋味是这么难受,她站在堂屋里叮叮咣咣的煮开一锅淘米水。双手垫着抹布端起大铁锅,她真想走到院子里泼了无心和小妹。事情不临到自己头上,她真不知道自己还有着杀人放火的狠心。

沉着脸把衣裳浆过一遍晾上,月牙开始忖度着如何让小妹离开。小妹正在低头扫院子,看起来小小的乖乖的,她真不忍心硬撵;可是想起无心方才那个色迷迷笑嘻嘻的贼样,她就气得恨不能撒泼一场。把牙一咬把心一横,她回屋掏了两块多零钱,出来塞进了小妹的口袋里,又低头说道:``妹子,姐姐知道你无处投奔。可是姐家小夫小妻的,也不富裕。姐姐给你两块钱,够你吃喝一阵子的,你自己想法子生活去吧。''

小妹立刻仰起了头,一张瓜子脸在阳光下白成了半透明:``姐姐,我吃得少,能干活,你留了我吧,我没地方可去了。''

月牙很为难的蹙了眉头,正要说话,不料无心悄无声息的从后方走了过来,不阴不阳的来了一句:``多个人吃饭也吃得起,做点好事,再留她几天吧。''

月牙咽了口唾沫,心里快要腾起大火——小妹昨天没洗脸的时候,也没见他起过善心;今天洗出好看模样了,他倒有脸来教自己``做点好事''了!眼角余光忽然一闪,她捕捉到了小妹的眼神。小妹方才向无心递了个眼风,好个眼风,大黑眼珠子差点没飞出去!

月牙压下一口恶气,脸上显出笑模笑样,姑且不再提撵人的话。坐在炕上又纳了一阵鞋底子,她让无心和小妹好生看家,自己出门买些肉菜回来。两人清清楚楚的答应了,及至她扭着小细腰真出了门,小妹推门进了西屋,抿着嘴对无心笑:``大哥,你怎么不出来见见天日呀?''

无心盘腿坐在炕上,这时就对她招了招手:``过来坐,上午累了你了。''

小妹果然坐到炕沿,娇声嫩气的说道:``我可不陪着你久坐,姐姐看不得你和我说话呢。''

无心微微俯身,向她探过头去:``那你愿不愿意和我说话呢?''

小妹用小白牙咬了嫩嘴唇,笑着抬起一根玉葱似的手指,轻轻点上了无心的眉心,一双眼睛幽幽的黑:``我不知道。''

眉心是人魂魄聚集之处,小妹的指尖像一滴水落上皮肤,软中透出寒意。无心一动不动的答道:``岳绮罗,你不知道,我也不知道。''

小妹不说话了,脸上的笑意渐渐加深,深到极致之时,竟然笑成了个狰狞的面目。而无心闭上眼睛,就见前方隐隐一团晦暗血光。

慢条斯理的开了口,他对着那团血光说道:``你不必笑。我真不知道究竟是人外有人天外有天,还是当初布阵的人弄巧成拙,用至阴的邪气既镇了你,也养了你。难怪你的小丫鬟拼着魂飞魄散也要去撞石壁,大概是石壁一碎,她就有解救你出棺的机会了。''

随即他睁开了双眼,抬手握住了小妹的手指:``别徒劳了。''

小妹骤然收敛了笑容:``你到底是什么人?''

无心把她的小手放了下去,又在她的手背上安抚一拍:``虽然我是无意之中破坏了石壁,但毕竟是让你重见了天日,纵然无功,也绝无过。所以你不要烦我,请快走吧!''

岳绮罗忽然又笑了,笑得天真无邪:``原来你是行尸走肉,怪不得神鬼无忌。可是你的魂魄到哪里去了?大热的天气,你等到了洞房花烛夜时,会不会已经烂成一堆臭肉?月牙真是够傻的,她不知道她要和死人成亲了吗?''

无心好脾气的笑了又笑:``是是是,我是行尸走肉,我是傀儡,我是影子,我是死人。你说我是什么,我就是什么,行不行?''

岳绮罗一甩乌黑的短发,稚气十足的又道:``我要去告诉月牙,让她记得在入洞房时掀开被子,给你挑一挑身上的蛆!''说到这里她叽叽嘎嘎的笑出了声,十足的女童模样:``怪不得你不肯出来晒太阳呢,是不是因为越晒臭的越快?''

无心笑微微的看着她,不言语。而她开心的几乎娇憨了,爬上前去一直坐到了无心腿上。抬手搂住无心的脖子,她斜着一双秋水眼瞟人:``我看你这副皮囊还算不错,要不然,你跟了我吧!我会找些零碎魂魄填进你的身体,让你总能有个人样,如何?''

无心低头望着她的眼睛,望着望着,忽然抱着她就往后仰。与此同时院门开了,拎着空篮子的月牙一步迈进院内,通过大开的两扇窗子,正见小妹趴在无心身上。

月牙登时就红了眼睛。大姑娘的身份拦不住她了,她像她的娘她的姥姥一样,指着窗内大吼一声:``你俩干啥呢?''

然后她扔了篮子抄起笤帚,一阵风似的就刮进西屋去了。无心和小妹已经分开坐了起来,无心往炕里一缩,指着小妹就嚷:``没我事啊,是她扑的我!''

月牙自有一套战略,安内必先攘外。一把将小妹从炕上扯下来,她指着对方的鼻子就骂:``好你个骚狐狸精!我好心好意给你吃喝,结果倒是引进一条小白眼狼!怎么着?你几辈子没见过汉子,毛没长全就勾上我家男人了?你个不要×脸的小贱货,你给我滚你娘的蛋!''

月牙有劲,骂完之后薅了她的厚头发就往外搡。无心见状,立刻下炕跟上,以防岳绮罗出手伤人。月牙没想那么多,拎鸡崽子似的先把小妹扔出去了,然后``咣啷''一声关严院门,回身对着无心就是一笤帚:``你还想不想和我过了?还没成亲呢你就敢偷腥,往后结婚了我还有好啊?一眼没看住你就带着她上炕了,你就那么着急?急得连廉耻都不讲了?''

月牙越说越气,因为外敌已被驱出,所以现在专心致志的处置内奸。无心被她狠打了好几下,抱着脑袋往房里逃。月牙挥着笤帚紧随其后追了进去,房门一关,无心转身一把抱住了她,低声问道:``荷包里的黄符还在吧?''

月牙一愣,随即开始挣扎:``别扯没用的,你——''

无心不肯松手,继续说道:``我告诉你,那个小妹\ldots{}\ldots{}有妖气!''

月牙奋力的仰起了头,想要对着他的脸骂:``有妖气你还往炕上拽她?知道你有点邪本事,是不是再过两天要去找女鬼睡觉了?''

无心一手环着月牙的腰,一手上下拍打了月牙的背:``是她拽我,不是我拽她。再说我能看上她吗?谁知道她是个什么东西?''

月牙恶狠狠的直瞪着他,瞪了半天,攥了拳头挥出一拳:``你敢说你没动心?''

无心理直气壮的答道:``敢说!''

月牙又给了他一拳:``你还嘴硬?''

无心针锋相对的掴了她一掌,巴掌蹭过她的脸蛋,轻的连只蚊子都拍不死,因为不是真要和她对着干,而是要表示自己行得正走得端,不受她的脏水。

月牙明白了他的意思,心火渐渐降下去了。抬手一拧无心的耳朵,她咬牙切齿的说道:``别看我没娘家,我可不是好欺负的!''

无心笑着从她领口里抻出荷包,打开来看了看,见黄符安然无恙,就把荷包口重新抽紧了,又对她正色说道:``别以为我是在和你闹着玩。这道符是有来历的,必定有些灵力。月牙,你猜那个小妹到底是谁?''

月牙被他说得心里发毛:``我哪知道。''

无心低声说道:``她就是岳绮罗!''

月牙一哆嗦:``啊?那她不是早死了吗?''

无心思索片刻,末了说道:``到底是怎么回事,我也不大清楚。总而言之,你记住她是个早该死了的人,见她等于见鬼!''

月牙知道无心是靠着招神惹鬼吃饭的,说出话来肯定有准。想着自己昨夜竟然还和岳绮罗睡了一宿,她不禁起了一身鸡皮疙瘩。忽然转身推门向外瞧了瞧,院子外面空无一人,岳绮罗已然没了影子。

月牙算是受了一大惊,好在不是娇滴滴的身体和性情,所以惊归惊,不耽误她干活吃饭,只是夜里她主动搬去了西屋,和无心平分大炕睡觉。如此过了五天,无心和她去镇上买来红布红烛。新衣缝出来,成亲的准备也就做齐全了。

因为距离吉日还有几天,所以月牙清闲下来,开始打扮起了自己。这晚她温了一大锅水倒进两只大木盆里,想要彻彻底底的洗个澡。无心为她把盆端进空着的东屋,随即就被她推了出去。无心隔着门板嘱咐道:``天快黑了,把灯先点上吧。''

月牙答应一声,依言点了油灯。顺势往空荡荡的大炕上扫了一眼,她怪不得劲的想起了岳绮罗。幸而无心在堂屋里走来走去,不是碰了桌子就是踢了凳子,总不安静,让她心里有了底。

散开左右两条大辫子,月牙低头去解衣裳纽扣。天气热,天天擦身也不够劲,到了晚上就能嗅到自己的汗酸气。月牙把脱下的衣裤放到炕上,然后自己蹲在一盆水前,俯身想要先洗头发。撩水打湿了厚厚的长发,她闭着眼睛抬手去摸摆在炕沿的新香皂。一摸没摸到,二摸又没摸到,三摸摸到了,冰凉黏湿一跳一跳,顺着她的手腕往下流。猛然一甩头发睁开眼睛,月牙大叫一声,就见一团紫红色的稀烂血肉糊在了自己的手掌上,正在活生生的沿着小臂流动蔓延。发疯似的将手臂在炕沿上狠磕了几下,她一边起身大喊无心,一边灵机一动,在血肉将要越过肘际之时,一胳膊抡到了炕上的衣裳堆里。血肉触到了她的小荷包,``嗤''的一声凝结成了一层凹凸不平的红皮,紧裹着她的手臂抽搐不止,皮内仿佛藏了筋脉一般不断勒缠,竟是直箍得她手腕关节都要脱臼。月牙忍痛捡起荷包,一边转身往门口跑,一边想要打开荷包取出黄符。前方房门已被撞得咣咣直响,可是门板不但纹丝不动,甚至紧密的连道缝隙都没有。月牙又疼又吓,猜出外面定然也出了事。手忙脚乱的取出黄符捂上手臂,她忽然听到窗外响起一串清脆笑声,嘻嘻哈哈的,还是小女孩子的童音。

当即转身面对了窗户,月牙在摇曳火光之下,看到玻璃外面贴上了一张雪白小脸,正是岳绮罗。

\chapter{夜战}

无心人在堂屋,既听到了月牙的惨呼,也听到了岳绮罗的娇笑。眼看门板坚实的如同厚壁一般,他转而冲向前方大门,想要冲进院内。然而大门也是同样紧闭。他合身向前狠撞几下,半边身体的骨头都震痛了,大门依然严丝合缝,毫无变化。

无心没想到岳绮罗真有几分不凡的妖术,定下心神思索了一瞬,他就近抄起灶台上的菜刀,对着左手掌心便划。一刀下去不见鲜血涌出,再划一刀才隐隐渗出了血色。无心是有办法破开妖术的,只是太过痛苦,难以忍受。横七竖八的将左手掌心划了个稀烂,他最后抬手一刀割开颈侧,随即扔下菜刀对着门板拍出一个血手印。只听一声巨响,房门应声而开,他冲进院内转身一看,正见到岳绮罗打开东屋窗户,要往内爬。

大踏步的冲向前去,他同时抬起右手按住颈部伤口,忍痛挤出一股鲜血。双手血淋淋的搓了搓,他对着岳绮罗的头脸就出了手。岳绮罗当即侧身一躲,然而面颊已被甩上了几滴血点。像挨了火烧一般哀鸣出声,她一边抬了袖子满脸乱抹,一边向后退出老远。而无心趁机转向窗户,大声问道:``月牙,你怎么样?''

月牙还在用黄符死死贴着手臂。紧缚在手臂上的一层血肉已然渐渐松弛,不再箍得她关节骨缝作痛。眼看无心站到了窗外,她蹲下来挡了胸前腿间,高声答道:``我有黄符,我没事!''

无心听她中气很足,便放心转向了岳绮罗。岳绮罗还穿着月牙的衣裳,领子袖子都宽大。放手抬头正视了无心,她的小脸上血点赫然,皮肤肌肉围着血点收缩抽搐,一张脸失控似的扭曲不止。抬手一指无心,她的声音粗哑起来:``你到底是什么?''

无心阴着面孔笑了一下,抬手捂上颈侧伤口,狠狠又挤一把:``你就当我是神吧!''

话音落下,他纵身扑上前去,伸着两只血手就要去抓岳绮罗。岳绮罗在至阴之地存活百年,自身就是个邪物,然而沾了无心的鲜血之后,竟然如同中毒一般身心俱乱。眼看无心已然逼近,她一甩衣袖凌空飘向后方,回身作势要逃。无心斗鬼斗出了经验,知道自己的血很能镇鬼,而且来之不易;所以开了院门拔腿就追。

无心前脚一走,月牙后脚也得了自由,手臂上的一层血肉越缩越小,最后成了一团皱巴巴的烂皮落在地上。月牙紧握着符咒蹲下去细看,认不出它到底是块什么东西,就见皮中嵌着几根萎靡的筋脉,还在长虫一般垂死挣扎的蠕动。月牙越看越觉恶心。起身跑到炕边把黄符装回荷包挂到脖子上,她手忙脚乱的穿了衣裤,光脚踩着布鞋再去开门。这回房门一拽便开,她从灶台下面找出两根未烧的劈柴,想要把东屋地上那团烂皮夹出去烧掉。

皱着鼻子拧着眉毛真把烂皮夹起来了,月牙壮着胆子向外走进院内。房子偏僻,左边邻着田野,右边走出不远是老树井台,过了井台才又有人家,所以她半夜点火也不惹人注意。一小堆火烧旺了,她一手握着火钳子,一手攥着胸前的小荷包,心里又是怕又是恨。眼看烂皮在火里一动一动的不老实,她把牙一咬,伸火钳子压住了它。腥臭的浓烟腾起来,她用小荷包堵了鼻子,像幼年跟她舅舅冬天进山打狐狸时一样,起了满心的杀机。不管岳绮罗是妖是鬼,如果此刻敢再出现,她会拼了性命给她一火钳子!

烂皮在火里烧得滋滋响,月牙又加了几根柴禾进去,把火翻得很旺。眼看烂皮快要化成灰烬了,院门忽然一响,一个黑影``呼''的冲了进来!

月牙正在脑海里大杀狐狸精,冷不丁的受了惊动,一火钳子就敲在了地上:``谁?''

人高马大的黑影猛然刹在了院门内,一脚前一脚后,一手拿刀一手拿枪。对着月牙上下打量了几眼,他忽然出了声:``哎?你不仙姑吗?''

月牙眨巴眨巴眼睛,也是十分意外:``哟,顾大人?''

顾大人抽了抽鼻子,问道:``怎么满院子都是屁味?师父呢?''

月牙经过了一场惊魂,现在瞧顾大人都顺眼多了:``收拾妖精去了!''

顾大人心里有了数,直通通的就往堂屋里走。月牙连忙回头看他:``顾大人,你来有事啊?''

顾大人头也不回的进了屋:``他妈的打仗没打好,有人追我,我到你家躲躲。''

顾大人的部下张团长,以及顾大人的宿敌丁旅长,两方联手出兵,把顾大人打了个人仰马翻。顾大人单枪匹马逃出文县战场,糊里糊涂的跑来了猪嘴镇,刚到镇子边就见了人家。他又累又饿,打算破门行凶抢些吃喝,不料院门大敞四开,他公然冲进去,迎面正是见到了月牙。

进了堂屋看到灶台,他揭开锅盖看到了几只大菜包子,当即抓起一只就往嘴里塞。而月牙熄灭了院内的火堆,回到堂屋点了油灯,眼看顾大人噎的上气不接下气,她便打算给他倒碗水喝。哪知一碗水端到顾大人面前,顾大人却是盯着她的胸脯直了眼睛。月牙低头一瞧,连忙放下瓷碗拢了前襟——纽扣没系全,前边露出了一大片胸脯。

顾大人一伸脖子,喉咙里的一口菜包子终于``咕噜''一声咽下去了,心想:``两个大馒头!''

月牙现在没心思和他计较,转身把纽扣一粒一粒系严实了,她迈步进院要等无心回来;而顾大人想着她的大馒头,不由自主的也跟了出去。

与此同时,无心已经追着岳绮罗上了荒野。

岳绮罗身形飘忽,不远不近的始终在前方。无心知道她是肉体凡胎,再有法力妖术,也做不到飞天遁地,如今又被自己的鲜血伤了,恐怕也只能逃到这种程度。提起一口气加快了速度,他对于岳绮罗既没意见也没兴趣,就是感觉此人讨厌难缠,虽然还未摸清她的底细,但他很想抓住她狠打一顿,打不死也打个半死。

两人之间的距离明显缩短了,岳绮罗还是个半大女孩子的身量,哪里比得过无心步大腿长?眼看就要没了生路,无心正要去抓她的蓬松短发,不料她毫无预兆的回手一甩,无心猝不及防,只感觉眼前一黑,脸上冰凉黏湿的糊了一层腥臭之物。收住脚步抬手一摸,触及之处一片细小的疙疙瘩瘩,宛如一片抻开了的筋膜皮肉。而岳绮罗微微喘息着面对了他,见自己扔出的一团血肉正中目标,而且已经流淌蔓延开来,不但包住无心的头脸,而且将要箍住他的脖子,便洋洋得意的一拍手:``大哥,你戴上了我的面具,看起来可就不那么好看了!''

无心手中鲜血已然干涸,想要咬破舌尖,面孔又全被血肉包住,牙关一动都不能动。抬手捂上颈侧伤口,他还想忍痛再挤鲜血出来,然而血肉凝结成皮,已然快要覆住伤口。无心深知自己若是再不行动,就会被血肉吞没整体,届时彻底没了还手之力,岳绮罗便可为所欲为。双手抓住血肉边缘,他想要将其撕脱,然而血肉仿佛已经和他的皮肤融为一体,一撕之下,颈侧伤口当即被扯了开。点点鲜血迸溅而出,血肉像被滚油浇过一般,立刻开始抽搐紧缩。

岳绮罗看了血肉的反应,不由得也抬手一蹭面颊。无心的血竟然邪到无法言喻,她的小脸上已经被血点蚀出了深深的孔洞。眼见无心颈侧的伤口被越拽越大,苍白的皮肤裂开来,露出里面层层筋肉,鲜血却是越来越少;她心生一计,右手状似无意的垂下来,一把锋利匕首倏忽间从袖内滑入她的手中;左手扬起来,她虚虚的对着无心一招:``大哥,你接住了!''

无心目不能视,依稀感觉她又扔了东西过来,生怕又是血肉一类扯不开甩不脱的东西,连忙挥手去挡。而岳绮罗趁此机会,狞笑着伸长舌头一舔匕首,随即纵身而上,对着无心的脑袋横砍一刀。只听一声凄厉惨吼,岳绮罗飘然退后,虽然手背上星星点点的溅了无心的鲜血脑浆,可是总而言之,还算胜利——无心的上半个脑袋被她横劈下来了!

笑微微的看着前方,她忍着手上脸上深入骨髓的痛楚,静观着无心的反应。她认定无心不是行尸走肉,否则没有魂魄支撑,肉体早就腐烂了。既然不是行尸走肉,就该有魂有魄。她要收住他的魂魄——收住了,他就是她的了!

至于躯壳上的损伤,实在不算什么。只要无心肯乖乖的听话,她会帮他修复身体,就算修不得了,再找一具更漂亮的皮囊也不是难事。

然而无心在熬过最初的剧痛之后,却是站在原地不动了。

岳绮罗把他劈得很平整,从鼻梁中段向上,是个齐齐的平面。他的脸上只剩下了鼻子和嘴,至于先前纠缠不清的肮脏血肉,已经被他的脑浆化成了灰烬。

忽然对着岳绮罗笑了一下,无心准确无误的踢开前方挡路的上半个脑袋,一步一步的走向了前方:``怎么?你以为你大功告成了?''

岳绮罗后退了一步,用她清甜的小嗓门说道:``我要你的魂魄!''

无心继续向前:``怪不得你能记得前世事情,原来你会控制魂魄。你爱做什么就做什么,本来与我无关,不过让零碎魂魄附上腐烂血肉,让它臭哄哄的四处乱爬,尤其是爬到了我家里吓人,就不对了。做了错事,不受惩罚,还砍掉了我半个脑袋——''

无心压低声音,下半张血迹斑斑的面孔忽然痉挛了一下:``小妹妹,你很过分啊!''

岳绮罗始终没有捕捉到无心的魂魄,于是暗暗蓄势预备逃跑。撒娇似的一扭肩膀,她故意说道:``我不管,我就要你的魂魄!我——''

话音未落,她已被无心扑倒在地。一滴鲜血滴进了她的眼中,让她发出了一声稚气的尖叫。紧闭双眼伸开双手,她在草地上飞快画出符咒,最后双手用力一拍地面,撕心裂肺的大喊一声:``生!''

荒郊野外,地下免不了会有骸骨埋葬。附近地面渐渐隆起,忽然一只白骨嶙峋的手破土而出,却是一只骷髅缓缓爬了出来。骷髅大概不是好死,魂魄缠绵人间,还未散尽,如今正被岳绮罗所操纵了,成了她的傀儡。眼看骷髅白骨从后方箍住了无心的身体,岳绮罗奋力一起,撒腿便逃。而无心疼到疯狂,起身拼命一挣,将副骷髅当即拆成碎骨。可是就在这短短的几秒钟里,岳绮罗已然隐于夜色,无影无踪。

\chapter{十分惊魂}

无心总不回来,月牙就搬了个小板凳,坐在黑洞洞的夜里等待。顾大人眼前晃着一对大馒头,叼着烟卷蹲在一旁陪她。眼看月牙心不在焉的直往院外望,他没话找话的开了口:``师父倒是总有生意上门,可半夜把你一个大姑娘留在家里,真是太不安全了。''

月牙没理他。

顾大人斜着溜了她一眼,天黑,看不清脸面,能看清身形:``我说,你也老大不小的了,师父没想着给你找个人家?妹子再好,也不能养一辈子不是?''

月牙终于开了口:``我不是他妹子。我俩也是前一阵子才认识的。我没家,他也没家,我俩说好了,过两天就成亲。''

顾大人一听,当场有了失恋的感觉,烟卷都灭了:``啊?你俩不是兄妹啊?''

月牙摇了摇头:``不是。''

无心蹲在荒野上,双手捧着自己的上半个脑袋。很怜惜的摸了摸脑袋上面的短头发和眉眼,他徒劳的想把它扣回头上。脑浆淋淋沥沥的流了他满脖子,他依然是疼。

他很冷,很累,疼得像堕进了火海里。他想回家去,让月牙拧把热毛巾给自己擦一擦,可是未等他站起身,半个脑袋自己落到了地上。一直想要对月牙讲明自己的真面目,始终是找不到机会,如今机会来了,他想瞒都瞒不住了。

或许,自己都不该再回去,免得把月牙活活吓死。吓不死,也可能吓疯,虽然月牙也算是胆子大的姑娘了。

夜色越来越浓了,浓到极致便会转淡,转淡了,天就亮了。回还是不回,他必须马上作出决定。如果真的拖延到了天亮,镇子边上人来人往,他想露面都不能够了。

无心解开衣裳,把自己那半个脑袋藏进了怀里。犹犹豫豫的站起身,他想自己迟迟不归,月牙一定担心极了。回去一趟吧,就算月牙不要他了,他也想再见月牙最后一面。

月牙坐在小板凳上,看出天要亮了。

自从在院子里烧过火之后,蚊子倒是被熏走许多,直到此时才渐渐重新聚拢。她一边啪啪的拍蚊子,一边对着门外望眼欲穿。顾大人百无聊赖的坐在一旁,想要强\textbar{}奸月牙,又怕无心回来饶不了自己,正是意淫之时,他忽然听到门外传来了无心的声音,轻轻的,怯怯的:``月牙,我\ldots{}\ldots{}我回来了。''

顾大人吓了一跳,月牙则是一跃而起:``你怎么才回来?''

院门一侧伸进一只苍白的手:``别过来,我受伤了。''

月牙一把攥住了他的手,不由分说的就要往里拽:``受伤了?赶紧让我瞧瞧!''

无心没有动,又说了一句:``你不要怕。''

夜黑如墨,月牙隔着一层篱笆,朦朦胧胧根本看不清他,急得都要生气了:``我怕什么?你让骚狐狸精把脸挠了?''

无心从大门一侧缓步走出。而月牙直勾勾的看着他,明明大概看清了轮廓,可就感觉自己没看清,看错了!后方的顾大人也站了起来,不说话,对着无心使劲揉眼睛,

末了,月牙颤巍巍的伸出了手,摸上了无心的面颊——面颊只剩下了一半,不够一手摸的。

``脑袋呢?''月牙的声音吊成了一根线,又高又细的重复了一遍:``脑袋咋了?''

随即她两眼一翻,向后仰了过去。

她一仰,顾大人怪叫一声,扶着她就往后退,一鼓作气退进了堂屋。``咣''的一声关了房门,顾大人哆嗦着掏火柴点油灯,而月牙背靠门板瘫在地上,一口气慢慢的缓过来,她睁开眼睛怔了一瞬,带着哭腔又开了口:``脑袋呢?''

顾大人扑到她的面前,巴掌在鼻梁上比量着一横,压低声音急促问道:``是不是往上就没了?我没看错吧?是不是没了?''

月牙把嘴一咧,呜呜哭着点了头。不料正在此刻,身后的门板有了震动,是被无心轻轻敲了一下。

无心站在门外,隔着房门开口说道:``月牙,你别怕,我做了鬼也不会害你。我是一时疏忽,被岳绮罗劈掉了半个脑袋,但是我不会死,你给我一点时间,我可以恢复成原来的样子。''

月牙抬手一拍大腿,哭得满脸都是眼泪:``哪有没了半个脑袋还不死的?你——你——''

说完两声``你''之后,她忽然一愣,抬眼去看顾大人,顾大人也是目瞪口呆。对啊,少了半个脑袋的人,怎么还可能一路走回家来?无心方才说的都是什么话?

顾大人慢慢抄起了刀,对着月牙做了个无声的口型:``鬼?''

月牙张着嘴挺身离了门板,四脚着地的向前爬去。而无心没有得到回答,忍不住抬手又敲了敲门:``月牙?''

月牙一转身坐在地上,几近崩溃的哭叫道:``别进来!你是人还是鬼啊?你别进来!''

门外果然安静了。

月牙缩在炉灶后面,抽抽搭搭的一直哭。好容易得了个如意郎君,眼看着就要成亲了,没料到一夜不见就少了半个脑袋。少了半个脑袋,不知道算人还是算鬼。让她跟半个脑袋的人过一辈子,吓都吓死她了,怎么过得下去?可是无心既然没有死,她不要他了,他怎么办?他脑袋缺了一半,到哪儿都是怪物了,还有谁能管他?

月牙哭得肝肠寸断,又心疼自己又心疼无心,哭的怕都忘了。窗外一点一点见了亮,顾大人怕鬼不怕人,一见太阳就有了底气。手里攥着他的砍刀,他不耐烦的对月牙说道:``哭能哭出个屁用来?我出去看看到底是怎么回事!他真要是半死了,我就给他补一刀,让他走个痛快,你也不用怕,难道我不是汉子吗?嫁不了他就嫁我,我不比他强?''

话音落下,月牙站起来,却是率先一步拉开了房门:``不用你,我自己出去,我不怕他。''

凌晨的空气是清凌凌的凉,月牙走进院子里,发现无心不见了,堆好的柴禾垛却是乱糟糟的没了形状。她奓着胆子靠上近前,就发现柴禾垛下伸出了两只脚,一只穿着鞋,一只光着,正是无心的脚。

犹犹豫豫的弯下腰,她试探着伸出一只手,在那赤脚脚背上摸了一下。赤脚的脚趾头立刻动了动,随即无心的声音从柴禾垛里传了出来:``月牙,你放心,我不会出来吓你。你如果还是害怕,那我天黑就走。''

月牙听了他的声音,还和平时一样沉沉稳稳的,不禁难过的心如刀割:``无心,你说实话,你到底是个啥?我都是要跟你成亲的人了,你不能瞒我骗我。''

无心沉默片刻,长长的叹了一口气——终于到了这一关。

``我不知道我的来历。''柴禾垛里的无心低声说道:``我也不知道我已经活了多少年。我不长大,也不衰老,更不会死。我的骨肉正在生长,过一阵子我又会有个囫囵脑袋,就和先前一样。''

顾大人走了过来,蹲在一旁静静的听。而无心继续说道:``月牙,我一直没有告诉你。我\ldots{}\ldots{}我也不能让你生儿育女。''

顾大人开了腔:``我明白了,你就是一个长生不老的太监呗!''

柴禾垛里猛然伸出一只惨白的手,分毫不差的扯住了顾大人的衣袖:``信不信我日了你?''

顾大人惊叫一声,很灵活的从外衣里面逃了出去:``我闹着玩的,你别当真啊!''

月牙默然无语的站起身,径自走进了西屋里去。关了房门又关了窗,她盘腿坐到炕上,把自己预备的嫁衣全翻了出来。布料全是镇上最贵的,摸着别提多厚实了,颜色又鲜又正。她没娘家,是自己嫁自己,嫁得满意极了,心里美得像是揣着一盆火,红红火火的要和无心过上一生一世。

没想到,无心都不是个真正的活人。

她把自己和无心的新衣裳全摸了个遍,摸完之后靠在墙上,眼泪就顺着眼角往下流。她小时候只在老家读过两年私塾,说不出``一见钟情''之类的好词,她只会说``一眼就相中了''。

对于无心,她便是``一眼就相中了''。一眼之间都能生情,她和无心都互相看了多少眼了?生出的感情比山都高,比海都深了。让她收拾起小包袱另寻夫君,她宁可剃了头发当姑子去。除了无心,她谁也看不上了。

到底应该怎么办,月牙也没了主意,自己在炕上坐着哭,躺着哭,把辫子扯散了打滚撒泼的哭。哭到最后哭不动了,她趴在炕上歇了一会儿,起身编好辫子擦了把脸,推开房门进了堂屋。

抬起袖子又抹了抹泪,她红着眼睛走到灶前,开始照常生火做饭。

\chapter{复生}

月牙和面,擀面,切面,烧开水煮面条,用三个鸡蛋伴着青菜豆瓣酱做了一大碗卤子。顾大人把他的刀枪放在了东屋的炕上,单手插兜靠墙站在灶旁,垂涎三尺的等着吃打卤面。月牙腰身秀气,动作可不秀气,干起活来大开大合,好像也就是一眨眼的工夫,面捞出来了,卤子也盛出来了,连锅都刷干净了,灶台都擦清洁了。

顾大人作为屠夫之子,勉强也算苦出身,虽然总有猪大油吃,苦的有限。他在文县吃惯了山珍海味,然而如今落魄了,能吃上打卤面也挺满意。老太爷似的坐在饭桌前,他理直气壮的等着上面。月牙站在灶台前,正用勺子往一海碗面条上舀卤子。卤子放足了,她又抄起筷子开始拌面;顾大人看见了,开口说道:``不用你拌,我自己来。''

月牙鼻音很重的说道:``没给你拌。''

顾大人愣了一下,随即反应过来:``你还给他吃面条啊?他还有嘴吗?''

月牙低着头,把面条挑起多高:``没嘴就直接往腔子里倒。''

顾大人咽了口唾沫,对月牙有点恨铁不成钢:``你个娘们儿真是不开窍,他都长生不老了,还少你一碗面吃?反正也饿不死他,你还喂他干什么!''

月牙不理他,自顾自的继续拌面。拌好之后端着海碗走出去了,她还是害怕无心的样子,走到近处就停了脚步,低声问道:``哎,你饿不饿?''

无心还躲在柴禾垛里,手里捧着自己的半个脑袋。每次重伤过后,他总要活一部分死一部分,活着的部分渐渐成长,死了的部分渐渐腐朽。如今他的身体活着,半个脑袋死了,所以他扒开眼皮凑上嘴唇,正要吮下一只眼珠充饥。月牙的声音刺激了他,让他含着一只眼珠立刻做了回答:``饿!''

月牙听他有声,显见真是活得挺旺,便很悲伤的放了心。眼看柴禾垛上开了一个隐隐约约的洞,是无心伸手抓顾大人时留下的,她便弯腰把一大碗面放在了洞前,又将一双筷子横架在了碗沿上。

``吃吧。''她小声说道:``不够再盛。''

然后她直起腰,转身走向堂屋门口。进门之后回头看了一眼,她见一只手从洞中伸出来了,先是拿走了碗上的筷子;然后再伸一次,稳稳的把大碗也端了进去。

月牙懒懒的肿着眼泡,顾大人说什么她都不听也不答。一锅面条,给无心盛了一海碗,她自己吃了小半碗,剩下的全被顾大人包了。

吃饱之后,月牙走进院内,见空碗和筷子已经全被摆在了洞外地上。过去蹲下收拾了碗筷,她正起身要走,不料前方洞中忽然挤着伸出了两只手,竟然合掌对她拜了一拜。

同时,无心的声音传出来,很轻很乖:``月牙,谢谢你。''

月牙气息一颤,眼泪落进了空碗里。一把握住无心的手,她狠狠攥了一下,喉咙哽咽的发不出声音。紧接着松手站起身来,她屏住呼吸快步走回了堂屋。

日子还得照常的过,月牙挎着空篮子出了门,要去附近的集市上买菜割肉回来。病一场还要补一补呢,何况无心少了半个脑袋。

她前脚一走,顾大人后脚就溜达出来了。光天化日的,他胆子特别壮,背着手围着柴禾垛转圈。末了停在无心伸出来的双脚前,他弯下腰细看了半天,发现原来长生不老的也长五根脚趾头,和自己是一个样。

无心知道他来了,然而缩在柴禾垛里没出声,手掌轻轻抚摩着自己的头皮,头皮上面生着一层睫毛长的短头发,毛茸茸的好像小狗的脊背。自从吃过一大碗打卤面之后,无心就没有胃口再吃自己了。

顾大人心里痒痒的挺好奇,走到柴禾垛上的小洞前蹲下来,他用一只眼睛往里看:``哎,你干什么呢?''

无心正抱着脑袋摸得心旷神怡,忽然受了他的打扰,就有些不大耐烦。侧过下半张脸凑上洞口,他把自己的嘴唇亮给了顾大人。嘴唇是薄薄的带着棱角,紧紧抿住了,里面的舌头则是在翻江倒海的搅动不已。顾大人以为他要啐自己,正想躲闪,不料无心的嘴唇忽然张开了,两排牙齿之间衔住了一颗黑白分明带血筋的人眼珠子!

只听``噗''的一声,眼珠子向前直打到了顾大人的脸上。而顾大人一屁股向后坐去,吓出了一脑袋白毛汗,耳边就听无心说道:``离我远点,否则我活吃了你!''

顾大人一翻身爬起来,回到堂屋自己舀了一盆水,开始疯狂洗脸。

月牙上午出门,中午回来,篮子里面除了肉菜水果之外,上面还盖了层层荷叶和几个莲蓬。莲蓬是买回来吃的,荷叶是她向卖莲蓬的孩子要来的,预备用来做荷叶粥。把荷叶随手放在柴禾垛上,她拿起一个大莲蓬,也不说话,直接俯身塞进了洞里,然后径自向房内走去了。

顾大人被眼珠子打了脸,越想越恶心,把脸洗了个通红,关公一样向月牙告状,说无心吃人。月牙面无表情的摆上切菜墩抄起切菜刀,低声说道:``爱吃啥吃啥吧,不吃\textbar{}屎就行。''

顾大人压低声音,皱鼻子瞪眼的对她说:``他可能是个妖怪!''

月牙垂着肿眼皮,审视着面前猪肉的肥瘦:``爱是啥是啥吧,是个男的就行。''

顾大人气的笑了:``我也是个男的啊!''

月牙开始切肉:``我爹也是男的。''

顾大人被她堵的没了话,心里知道自己不招对方待见,问题当然不在自己身上,而是月牙太过浅薄,被小白脸迷了心窍。

满怀自信的走去院子里,他找到无心的眼珠子一脚踢开,倒还没有离去的打算。平日里他飞扬跋扈,惹下不少仇家,如今队伍被人打散了,张团长和丁旅长绝不会放弃痛打落水狗的机会。他现在露面,等同于找死,不如等到风声弱了,再做打算。

月牙煎炒烹炸,做完午饭做晚饭,忙着忙着天就黑了。她也知道无心一个人睡柴禾垛不舒服,可是让他回屋上炕,她又实在害怕。自己关了西屋的门,她坐在窗前向外看,看着看着,却是忍不住一笑。

原来一只手从柴禾垛的洞中伸出来,向上摸索着拿下了一片大荷叶。片刻之后无心从柴禾垛里爬了出来,戴帽子似的顶着荷叶,一路跑进了茅厕里去;脑袋还是只有半个,不过好像比凌晨见长。

三五分钟过后,月牙眼看着无心鬼鬼祟祟的又溜出来钻回柴禾垛里了,才放心的躺了下去,心想:``这算个啥东西呢!''

无心在柴禾垛里一躲就是半个月。半个月后的一天清晨,月牙还在炕上睡觉,忽然听见有人敲窗户,睁开眼睛起身一瞧,她就见无心把脸贴上玻璃,眉毛是眉毛眼睛是眼睛的,还和先前一个模样,脸皮是粉红粉白的嫩。

她以为自己是在做梦,张着嘴看着无心不言语。而无心双手抱着臂膀搓了搓,对着她做了个口型:``冷。''

月牙一掀被子下了炕,连忙给他开门去了。

两小时后,蓬头垢面的顾大人从东屋走了出来,迎面就见无心穿着一身崭新的裤褂,正坐在桌边喝热汤。

``哟!''顾大人很惊愕:``活啦?''

无心抬眼看他:``你什么时候走啊?月牙可是已经伺候你半个多月了!''

顾大人装听不见,先是上下打量无心,打量够了走上前去,伸手指头去戳无心的脑袋。头骨硬硬的,皮肤却是又软又嫩;头皮泛着青,是将要生出头发的模样。

``嚯!''顾大人算是开了眼界,用他的大巴掌盖住了无心的头顶,试试探探的又拍又摸:``挺会长啊,新旧一个颜色,谁能看出你上半个脑袋是后来的?''

无心任他撩闲,自顾自的继续喝汤,月牙站在灶台前,也不理他。月牙不在乎多干点活,也不在乎顾大人一个人有两个人的饭量。顾大人的讨厌之处在于他总是粗豪的贫嘴恶舌,让人怒也不是,不怒也不是。月牙不是很敢惹他,只希望他尽早带着他的刀枪滚蛋。

然而顾大人无意滚蛋。大喇喇的坐在无心对面,他脸也不洗牙也不刷,一挽袖子开口说道:``师父,别不理人,你抬头看我一眼,我有正经事和你讲。''说到这里他一挥手:``月牙,给我盛碗汤,我得边喝边说!''

\chapter{合作}

顾大人喝了一碗鲜美滚烫的肉汤,然后抬袖子一摸额上的热汗。翘着二郎腿望向无心,他开口说道:``我有钱。''

无心东倒西歪坐没坐相,是个懒洋洋要瞌睡的模样,一双眼睛也是似闭非闭:``嗯。''

顾大人本来终日自我感觉良好的嬉皮笑脸,此刻却是难得的正了脸色:``昨天我出去溜达了一圈,听说张小毛子和丁大头闹崩了,正在文县对着打呢!''

张小毛子是张团长,丁大头是丁旅长,全是顾大人的仇敌。而无心身上暖和,腹中也暖和,舒服的一动不动,连呼吸都停了:``嗯。''

顾大人把胳膊肘架在桌上,浓眉之间闪过一道凶光:``我要拿钱出来招兵买马。等他们两个王八蛋打疲了,我再干他们个出其不意!''

无心微微一点头:``嗯。''

顾大人看他懒得刀枪不入,急得用手指一叩桌面:``所以我得拿钱哪!''

无心向外轻轻一挥手:``好,拿去吧,再见。''

顾大人登时气歪了鼻子:``放你娘的狗屁!我要是一个人就能拿到手,还和你废什么话?我告诉你,我有三箱金子,是前年在冯家屯挖墓挖出来的,能值多少钱我没算过,反正当初让我偷着藏到猪头山里了!现在你得帮我把金子运出来,我不让你白出力,肯定亏待不了你。你看你除了装神弄鬼之外也没别的本事,是,你是饿不死,可你也得顾着月牙不是?只要你乖乖帮了我,将来我从手指缝里给你漏下点金末子,都够你俩快快活活过完下半生了。''

此言一出,月牙登时就把青菜下进油锅里了。``嗤啦''一声大响过后,她稍稍痛快了些许,心想顾大人说话太气人了,明明有求于人,居然还敢大言不惭,好像自己两口子活不起了,全等着他手指缝里漏金末子呢!

月牙挥着铲子,把一锅菜炒得刀光剑影。而无心八风不动,彻底把眼睛闭上了:``为什么非要找我?''

顾大人在满屋油烟中咳嗽了一声,随即答道:``自从张小毛子造了我的反,我就谁也信不过了。''

无心反问道:``谁也不信,就只信我?''

顾大人呼吸着混合了饭香的油烟,忽然生出了蓬勃的勇气,暗暗攥起两只大拳头,他对着门外说道:``算命的说我是一将功成万骨枯,老子还没出将入相呢,还没杀出成千上万的人命呢,哪能说完蛋就完蛋了?师父,你看着吧,老子将来要是真发达了,不管你到底是个什么东西,都少不了你的荣华富贵!所以——''

只听``咣''的一声,顾大人用他的大拳头一敲桌面,随即虎视眈眈的转向无心,一字一句的说道:``你得跟我上猪头山!''

无心对着他眨巴眨巴眼睛,满脸都是莫名其妙:``顾大人,你东一句西一句说什么呢?我告诉你我的脑袋现在嫩得很,风吹一下都疼,我凭什么要跟你去上山?''

饭菜端上来了,无心和顾大人还在打嘴仗。其实上猪头山倒没什么的,猪头山不算很大,名副其实,远看非常的像猪头。而猪嘴镇正好位于山下,紧靠猪嘴,小镇的名字便是由此而来。无心一家住在镇子边上,上山真是太容易不过;要说去上一趟,倒也不算十分为难。只是顾大人说话太不中听,居高临下的总要替人做主;所以无心故意拖延着不肯答应,及至顾大人急成脸红脖子粗了,他才略略松了口风。

到了下午,顾大人也不怕人了,亲自前往镇里购买进山应用之物。留下无心和月牙在家。月牙坐在炕上,翻着针线笸箩问道:``真要上山去啊?''

无心四脚着地的跪在一旁,蓄谋靠近月牙:``我是想分一点金子回来。往后日子久着呢,钱不怕多。''

月牙低头说道:``那我也跟你们一起去。''

无心试探着把下巴搭上了月牙的肩膀:``上山怪累的,在家等我吧!''

月牙的面颊起了红霞,脸上热着,心里却是热中透凉——两人离得这么近,可她连对方的气息都感受不到。忽然放下笸箩伸出了手,她按上了无心的胸膛。

胸膛里面安安静静的,一点活蹦乱跳的意思也没有。

月牙没有多问,放下手答道:``累我不怕,我就怕好不容易把你等回来了,你身上又少了物件。''

无心对着月牙的侧影笑了一下,然后歪着脑袋越凑越近,最后嘴唇就贴上了对方的脸蛋。月牙哆嗦了一下,只感觉自己像是喝醉了酒,半边身子都麻了,一颗心几乎要从喉咙里直蹦出来。

与此同时,顾大人正在买绳子。

绳子一盘一盘的堆在地上,都是溜光水滑的好麻绳,普通的草绳人人都能编,犯不上摆出来卖。大下午的,猪嘴镇的大街上人来人往,十分热闹;顾大人蹲下来挑绳子,挑着挑着就感觉背上做痒,像是有人在隔着衣裳轻轻挠自己,不禁回头怒问:``谁啊?''

后面没什么正经人物,只有一个破衣烂衫蓬头垢面的半大丫头,满脸糊的都是泥,脏的看不出眉目。顾大人虽然很爱女色,但是对于小叫花子并无兴趣,于是张口便骂:``去去去,哪来的小兔崽子!''

半大丫头面无表情,俯身一指点上了他的眉心。顾大人被她轻轻戳了一下,竟是感觉心神一晃,仿佛要被人把脑子心肺全勾出去。满怀烦恶的用力打开对方手臂,顾大人挺身而起,急赤白脸的想要揍她。而半大丫头后退一步,转身撒腿就跑。旁边的小贩见了,连忙让顾大人小心身上钱物,只怕是街上的小贼采用声东击西的战术要作乱。

顾大人买好了所需之物往家走,一路上头晕目眩,隐隐的还有些作呕。到了树下井台旁边,他眼看着前方就是家门了,却是无论如何都走不动。坐在井台上喘了半天,他感觉心里清明些了,才起身拖了两条腿,继续向前走去。

及至顾大人进了门,树下闪出一个小小的人影,细瘦双臂在肮脏衣袖中垂下来,右手的拇指食指缓缓摩擦不止。

岳绮罗没想到顾大人的阳气如此之重。

阳气重,杀气也重,凭着她的道行,竟让没能一举引出他的魂魄。右手二指的摩擦速度渐渐加快,她用左手从衣兜里掏出一沓黄纸剪成的小小人形,放上井台一字摆开。眼看四周无人经过,她咬破右手食指指尖,快速在一排小人身上写下血咒,口中同时念念有词。用力写出最后一笔,她左手猛然挥向无心家门,右手衣袖随之对着纸人扇出疾风:``吾佩真符,役使万灵,上升三境,去合帝城,急急如律令!''

一团火光骤然腾起,纸人瞬间灰飞烟灭。井台上面干干净净,丝毫没有烟熏火燎的痕迹。而岳绮罗愤愤然的抬手捂住了脸,迈步消失在了老树后面。

她本来是有点喜欢无心的,不是因为无心冲破大阵解救了她,而是在解救她时,赤\textbar{}裸的无心看起来很好看。当时她仰卧在棺材里,目光透过黄符的缝隙看清了他的一举一动。他有着修长的四肢,俊秀的面孔,最要命的是,他仿佛无所畏惧,不知道怕。

她在人间的时候有意无意总是会吓到人,于是无心就显得很可贵。然而无心不但不识抬举,还用毒血伤了她的脸。天知道她对自己的脸有多满意,她认为自己真是可爱美丽极了,可是为了弥补脸上的孔洞,她须得打扮成个小叫花子,四处挖掘尸首炼丹,用法术来恢复容颜。

她忙极了,但又不肯轻易饶了无心。既然姓顾的男人利用不上,她只好自己制造了几名部下,权充是千里眼顺风耳,免得她一时疏忽,从此再失了对方的音讯踪迹。

顾大人无知无觉的进了院子,月牙透过玻璃窗看见了,连忙往墙角一躲。无心追上去,抢着又亲了她一口。两人的嘴唇都有些红肿,月牙下了死劲,把他的手从自己衣裳里面扯了出去:``别没完没了!迟早都是你的,大白天的你急个啥?''

房门一响,顾大人真进来了。无心皱成了八字眉,下炕出门看他:``东西买齐了?''

顾大人一手拎着绳子,一手扶着铁锹:``齐了,咱们一会儿就走,行不行?''

无心一点头:``行,趁夜去趁夜回。''

吃过晚饭之后,顾大人腰挎砍刀,扛着铁锹拎着绳子打了前锋。他是本地人,小时候没少在猪头山里野跑,闭着眼睛都能把山逛遍。如今只要进山挖出金子,再用绳子捆好了背回来,就算完活。箱子不算大,只要有劲,搬运不是问题;而自己很有劲,无心也有劲,月牙饭量不俗,想必也不是平常女子。三个大人,还弄不了三只箱子?

无心跟在后方,一手揣进兜里,一手拉着月牙。兜里毛茸茸的鼓起一团,是他暗暗藏起来的一片旧头皮——他的骨肉不会腐烂,只会一点一点的干软成絮,最后化灰。头皮如今还剩软而薄的一层,如果不去处理,最后也会自然的消失。横竖都是消失,不如先带在身上,反正不是坏东西,至少可以用来驱邪。

三人悄悄走过荒地,进了山中。猪头山并不险峻,远看就是个浑圆的大猪头,值此夏末秋初之际,山上草木葱茏,所以还是个绿猪头。月牙走着走着忽然一回头,没看见什么,随口对无心问道:``山上没狼吧?''

顾大人头也不回的答道:``没狼!有狼倒好了,我做个狼皮褥子!''

月牙回头又看了一眼,自己拍拍心口说道:``吓我一跳,我还以为有一双绿眼睛看我呢,原来是俩萤火虫。''

\chapter{火中取栗}

顾大人对猪头山真是太熟悉了,所以连火把马灯都不预备,顶着半空中的大月亮坦然前行。无心和月牙深一脚浅一脚的紧跟着他,耳边就听秋虫鸣叫此起彼伏,并不是个寂寞的夜晚。

不出片刻的工夫,领头的顾大人开始往草丛林子里钻。蚊子结成了阵,恨不能吃了他们三个,顾大人一边顶着蚊子开路,一边嘀嘀咕咕的骂街,不由自主的吃了许多蚊子;月牙则是单手抽出一条大手帕,满头满脸的乱挥;无心不招蚊子,一眼看着顾大人一眼看着月牙,承前启后的紧跟着。

不知走了多久,蚊子渐渐稀疏了,月牙终于有机会开了口:``顾大人,还没到?''

顾大人把绳子向后递给无心,自己挥着铁锹披荆斩棘:``快了快了,我藏的可是金子,还不得找个隐秘地方?''

三人又走了良久,末了顾大人终于停了脚步。无心带着月牙挤上前去一瞧,就见前方鼓起了个坟头似的小土包,四周林子遮天蔽日,把月光遮住了大半,小山包上面生满杂草,不走近来,绝对瞧不见。

顾大人扶着铁锹站住了,回头对着无心说道:``你看这地方没什么出奇吧?我告诉你,大白天的让你带着地图来找,你都未必能找得到!猪头山看着不险,可是山上除了野菜蘑菇没别的,谁往这深处走?挖坟掘墓的都不来啊!''

无心抬手摸了摸下巴,然后横了顾大人一眼:``你把金子埋进地下去了?''

顾大人竖起一根手指摇了摇:``错!三箱金子,我能就地刨坑随便埋了?''随即他迈步上前,围着土包转了一圈。仰头望着星月定了定方向,他低头一锹插下去,一言不发的挖了起来。

他力气大,而且挖的得法,十锹八锹过后,无心和月牙就听到了金石声响,走近一瞧,却是发现土包下方埋了一块青石板。而顾大人一锹插到石板边缘,弯下腰就开始撬,同时咬牙切齿的挤出话来:``师父,来帮一把!''

无心松开月牙,上前弯腰抬住石板,原来石板还不算厚,重也重得有限。月牙眼看顾大人放下铁锹,和无心合力把石板抬起来了,就没有上前帮忙。低头把手帕掖到肋下纽扣眼里,她冷不丁的回头又看一眼,心想:``林子里还是有野物。''

石板掀开之后,下方露出幽黑洞口。无心俯身细瞧,发现此洞起初一段虽是直上直下,然而不过一人来深,再往下便是斜着深入,既不陡也不险,只是阴冷的潮气太重。顾大人累出了汗,气喘吁吁的说道:``不知道这洞是怎么来的,反正石板一盖,就算是我的藏宝库了。我这便宜捡的挺俏皮吧?''

无心对着前方的月牙招了招手,随即问道:``顾大人,洞里会不会有毒蛇?''

顾大人当即做了答复:``放狗屁!猪头山上就不生毒蛇!''

无心虽然对顾大人没有爱意,但是也不想让他平白无故的死在山里。既然洞内就有箱子,他便决定率先下去,遇了毒蛇也能解决。他打头,顾大人殿后,中间比较安全,让月牙走。

三人商议好了,又坐在洞口等了片刻,估摸着洞中潮气散出大半了,才络绎下了洞。三个人一人分了一条麻绳,顾大人的麻绳打了个死结,低头费了不少力气才解开。伶伶俐俐的向下跳入洞中,他眼角余光向上一瞥,忽然发现月牙还站在洞口,仿佛是要跳又不敢跳的模样。

顾大人一笑,仰头问道:``这就害怕了?你要是再不吭声,我都能把你落在外面。''然后他把绳子搭在肩膀上,向上伸出双臂:``来,你跳,往我怀里跳,我接着你!快点,要不师父都走远了!''

月牙动作僵硬的弯下腰,果然向他一跳。顾大人抱了个满怀,心想月牙看着大馒头大屁股的,分量居然还挺轻。月牙不说话,他就自作主张的握了对方的手,兴致勃勃的弯腰往斜洞里走。洞里是彻底的漆黑一片,顾大人走着走着,听不见前方动静,就开口问了一句:``师父,走着哪?''

无心的声音很快传了回来,原来就在正前方:``还真是没有蛇——怎么着?洞里还带拐弯的?''

顾大人嘿嘿一乐:``拐不了几个弯。''

随即月牙也开了口:``你们说这洞是啥动物的窝?我看不像是人挖的,像是啥东西用爪子刨的。咱们弯腰都能在里面走,看来猪头山上原来肯定有大野兽。''

顾大人自认是个土著,所以立刻不屑一顾:``哪有什么大野兽,这山上连狸子都少见,我告诉你们——''

话到这里,他忽然打了个激灵。短暂的停顿过后,他轻声说道:``月牙,你说能是什么大野兽?''

月牙的声音从前头传了过来:``说不好,我也不懂啊。''

顾大人的头上立时出了一层冷汗——前面走着的是月牙,那自己手里拉着的人又是谁?

顾大人如今对于鬼神一道,也算是见多识广。强行压下一声惊叫,他若无其事的想要放开后方的手。不料他把五指一松,那只手却是紧紧的攥住了他的手掌。停下脚步发出颤音,他鬼哭似的开了口:``你们\ldots{}\ldots{}划根火柴\ldots{}\ldots{}''

只听前方``嗤''的一响,月牙双手笼着一点小火苗转过了身,嘴里唠唠叨叨的不耐烦:``是你领我们来的,我们都不怕,你还怕上了。''

话音落下,月牙弯着腰,瞪圆眼睛也僵住了。微弱火光之中,她就见顾大人颤巍巍的抬起右手,右手上面赫然箍着一只苍白的断手!

随即二人心有灵犀一般一起缓缓扭头,咫尺近处的洞壁上,横贴着长长一具女体,两条辫子垂向下方,遮住了半张雪白面孔,只露出一双黑不见底的弯弯笑眼。

火苗倏忽间熄灭了,随之而起的是月牙的惨叫和顾大人的哀嚎。近处忽然又起了火柴光亮,却是无心转身挤了上来。女体沿着洞壁游动到了洞顶,一张脸彻底露出,上方是两弯乌黑笑眼,下方是一抹嘴角上翘的鲜红嘴唇,中间没有鼻子,正是一张麻木不仁的诡异笑脸。顾大人拔出砍刀向上一捅,捅进了女体胸中。女体不能再动,然而双臂向下越伸越长,最后竟是眼看就要触到顾大人的脖子。顾大人背靠洞壁,躲无可躲,正是崩溃之际,一点火苗横空飞来,正中了女体的脑袋。凌空一团火光瞬间亮了又灭,洞中三人隐隐就听一声凄厉哭叫,女体已然灰飞烟灭!

顾大人收回了砍刀,右手上面也干净了,只在掌心留有几片纸灰。呼哧呼哧的喘了一阵粗气,他的力量又回来了:``师父,怎么回事?''

黑暗之中响起了无心的声音:``不要怕,是有魂魄附在了纸人身上,出来兴妖作怪。一把火烧了它的替身,它的魂魄自然就散了。''

顾大人立刻把砍刀系回腰间,又从怀里掏出火柴。月牙也是紧紧攥着一盒火柴,带着哭腔说道:``这玩意儿是从哪儿来的啊?洞里面的还是洞外面的?''

此言一出,顾大人头发都立起来了:``洞里面\ldots{}\ldots{}不会吧?''

无心弯腰向前,低声说道:``先出去再说,洞里太逼仄,万一再来了纸人,想打都没地方打!''

他动作灵活,扯着月牙弯腰疾行,月牙也顾不上再嫌顾大人了,一把拽住顾大人的袖子就往外走。然而没有走出多远,无心就见前方一片月光骤然消失,竟是石板落下,将要盖住洞口!

无心登时急了——他自己尽可以不怕幽禁,然而月牙和顾大人在洞内久了,却是熬不住的!而顾大人一眼看清,猛的挤开月牙无心冲上前去,在最后关头一挥砍刀。只听``嗵''的一声闷响,刀身正好垫在了石板与洞口之间。

石板是彻底砸严实了,砍刀却也没有断裂。顾大人红了眼睛,开始破口大骂:``妈了个×的,什么东西在跟老子做对?老子拿自己的金子,没偷没抢,碍着谁的事了?''

话音落下,石板上面响起了沉重滞涩的脚步声。无心带着月牙赶上去蹲下来,伸手又一拍顾大人的小腿:``纸人要往石板上面堆土了!你踩着我们上去,快把石板掀开!''

顾大人看也不看,抬脚就踩。踏上二人的脊背之后,顾大人双手向上推住石板,运力之余大喝一声:``我操\textbar{}你们的娘!!''

两人拼了命才能抬动的石板,如今被顾大人硬生生的强托了起来。无心和月牙作为垫脚石,差点被他踩进洞中土里。而石板和洞口之间一欠缝隙,就有一双惨白的手伸了进来,准确无误的掐住了顾大人的脖子。顾大人登时窒息,然而心里愤恨极了,不肯松手服输。正是生死攸关之时,他身体忽然歪斜着向上一升,却是无心凭着一己之力,用肩膀扛着他的两条大腿站了起来!而月牙得了自由,连忙起身踮脚将一团毛茸茸的物事点燃了,连烟带火的从缝隙中扔了出去。

合在顾大人脖子上的双手立时化为灰烬,与此同时,后方洞中深处传出一声呜咽,竟是个女子哭泣的声音。三人一起愣了一下,其中月牙和顾大人已经被吓得麻木了,以为又是纸人出现;无心则是心中一动,大声催促道:``顾大人,快顶开石板,洞里有东西!''

\chapter{鬼洞}

顾大人红了眼睛,气运丹田双臂发力,瞪眼咬牙的重重哼出一声,硬是把石板托起推向了一旁。双手按住地面爬出土洞,他随即转身蹲下来向下伸手,先把月牙拽了上去。地上那一团毛茸茸的物事还在乌烟瘴气的阴燃,月牙不知道它的来历,只是依着无心方才的嘱咐,伸脚轻轻的把它翻了个身,让它烟气腾得更浓。与此同时,顾大人把无心也拎出来了。

顾大人累得胳膊哆嗦,可还是拔出砍刀面对了四方。几个披头散发的人影包围了他们,长发之间依稀可见白脸笑颜,都和洞中纸人是一个模样。月牙一手捂着胸前荷包里的黄符,站在烟气中不敢妄动。无心挽起衣袖,心想能够自如驱使魂魄而又和自己有过节的人,只有一个岳绮罗。方才洞中惊魂一场,想来和她必有关系。本来自己也算岳绮罗的恩人,不料对方恩将仇报,不但要害月牙,甚至还砍掉了自己半个脑袋,让自己狠狠的受了半个多月的罪。虽说好男不跟女斗,可是从头到尾的细想一番,无心真是忍不住的要闹脾气。

两只衣袖挽到肘际,无心对顾大人说道:``你和月牙不要动,烟能驱鬼。我去捡些柴禾过来,拢一堆火!''

顾大人铿锵的答道:``你去吧,他妈的敢过来我就砍死它!''

无心眼看自己的头皮越来越缩,并不禁烧,就连忙走了出去,公然的四处捡起枯枝败叶。周围纸人飘忽不定,然而始终不敢靠近,显然十分惧怕烟火。无心抓紧时间划了火柴,连吹带翻的点起了一小堆火。火苗刚刚稳定,旁边的烟气就快速淡化了,原来是头皮已经彻底烧光。

无心把顾大人和月牙叫过来,让两人背对火堆坐下。顾大人手里有砍刀,月牙却是手无寸铁;于是无心把坑边的铁锹拿过来给了她,同时低声说道:``见鬼就拍,手别留情!''

月牙双手攥住锹把,心里起了狠劲:``你放心,我有劲!再说我胸前还有护身符呢!''

无心没言语,因为到底也不知道那道黄符是干什么的,反正肯定能治岳绮罗,可岳绮罗并不算鬼——岳绮罗基本就是个人,只是不生不死的被镇压了百年,百年间她停止了一切生长变化,并且在至阴之地修炼出了一身的妖气。回想起上次岳绮罗逃脱之时所画的符咒,无心几乎怀疑她所使用的乃是某种道术。

眼看二人都坐稳了,无心开始围着火堆缓缓走动,一旦火势见弱,他便立刻就近捡拾枝叶添火。他不远离,纸人也不靠近,而洞中依稀传出若有若无的呜咽,断断续续的,像是伤心虚弱到了极点。好在顾大人和月牙守着火堆,身后光明,所以心里有底,怕的倒还有限。

无心默然走动了片刻,忽然开始专心捡柴。自顾自的把火烧旺之后,他面无表情的经过月牙,弯腰一把扯下对方肋下的手帕。直起身用手帕蒙住了眼睛,他不受双眼干扰,更清晰的感受到了周遭的魂魄。

魂魄的怨气很强,全都带着刻骨铭心的恨意,虽然生前它们各有仇人,但是如今受了操纵,便统一的把无心三人当成了目标。无心若是单枪匹马,满可以对它们忽略不计,毕竟是个纸胎子,一把火就能将其燎成飞灰。问题是身边还跟着顾大人和月牙,并且还有个曲曲折折的深洞等着他去钻。他不能再带着一条阴魂不散的尾巴进洞,否则身后两位都可能在洞里被纸人掐死。

双方不知僵持了多久,顾大人渐渐松了劲头,开口问道:``师父,咱们要等到什么时候?那几个玩意不来也不走,它们是什么意思?''

无心停在了他的身边,轻声答道:``它们是专为我们而来的,当然不会轻易离去;只是怕火,不敢靠近而已。''

顾大人一挺身就要站起来:``我烧了它们去!''

无心点了点头:``好,去吧。''

顾大人舔了舔嘴唇,看了看无心,又回头看了看月牙。月牙手握铁锹,精神抖擞;无心此刻不见眼睛,鼻子和嘴唇都是雕像一样,不带活气。

末了又向远方望了望游移不定的鬼影,顾大人不由得生了怯意。自己抬手摸了摸脖子,他被纸人捏了一把,现在喉结还在作痛:``我真去啊?我也没干过这活啊!要不然还是你去吧,你连有骨头有肉的鬼都打过,还怕这几个纸糊的?''

月牙背对着火堆,听得清清楚楚,忍不住说道:``他要是能去,早就去了,还用你催?你别把他当枪使唤!''

顾大人弯着腰半站不站,手里掂着一把砍刀,心乱如麻的也不知道自己该不该出手。不料正在他为难之际,无心忽然伸手夺过他的砍刀,随即弯腰从火中捡起一团火炭,高高扔起来挥刀一打。刀背磕上火炭,夜色之中只见一颗火流星急速飞出,远方当即腾起一团无根的烟火,一个纸人立时灰飞烟灭。

``顾大人。''无心提着砍刀,围着火堆又绕一圈:``不是我不出手,是我怕我离了你们,你们会有危险。周围的几个纸人,是我们能看见的;林子深处我们看不见的,谁知道还有多少?谁知道除了纸人,会不会有其它的东西?''

月牙思忖着说道:``我没见过大白天还能满街走的鬼怪,它们再厉害,一见太阳也得完蛋。大不了咱们等到天亮,天亮之后再下洞拿金子。现在山上没野菜,不下雨也没蘑菇,谁没事往山里走?咱们白天把金子拿出来,应该也不能被人瞧见,要是怕下山遇到人,就忍一忍饿,天擦黑的时候再回家!''

月牙的主意虽然很笨,但是无心等人寡不敌众,也没有其它的法子可想。三人守着一堆火不再乱动,而待到天边泛起鱼肚白了,顾大人看得清楚,就见那些纸人如同影子一般越来越淡,最后竟是真的消失无踪了!

顾大人一放心就来了精神,月牙年纪轻身体好,熬过一夜也不痛苦,只有无心哈欠连天,睡眼惺忪。三人没有干粮也没有水,只怕再耽搁下去会体力不支,便一起张罗着要二次下洞。无心知道洞里不干净,然而洞外阳光明媚鸟啼四起,一派爽朗景象,并不是鬼神肆虐的时辰,所以他大喇喇的第一个跳下洞内,还像昨夜一样打了前锋。顾大人长了心眼,把外衣的两只袖子撕下来缠上一根粗树枝,找松树蘸了松香制成火把,让无心用它在前方照明开路。三人络绎的弯腰钻入斜洞,一路走得十分顺利。连着拐了几个大弯之后,无心心中忽然一凛,暗想洞外是白昼不假,可洞内不见天日,永远都是黑夜。天上的太阳,可驱不散地下的黑暗。

就在他生出念头的一瞬间,鬼哭似的呜咽又响起来了。月牙和顾大人双双打了个冷战,同时只听无心粗声吼道:``嚎你娘的丧!你当老子要抢你的骨殖吗?''

此言一出,洞内登时恢复了安静。月牙和顾大人全服了无心——把鬼都骂老实了!

拐过最后一个弯,无心停了脚步,就见前方已经到了底,空间也开阔了些许,靠着洞壁果然叠着三只古旧木箱。闪烁火光之中,木箱丝毫不见腐朽,上面花纹俨然,可见姑且不论箱中的金子,单说箱子本身,就不是普通的木料。

顾大人挤上前来,伸手一拍箱子:``没错,就是我的宝贝!''

无心总算是见了箱子的面,随手将火把交给后方的月牙,他就要帮着顾大人把箱子捆好背起来。哪知就在此刻,哭声又起来了,就在三人身边!

无心一把抢过火把觅声照去,只见旁边洞壁凹凸不平,暗处竟然摆着一只半米多高的大坛子,坛子外面凝固着一道一道干涸血迹,几乎遮住坛子本身的光滑釉质。而坛口黑瀑一般散垂了长发,竟仿佛是里面藏了一个脑袋!

月牙真是惊着了,嗷一嗓子藏到了无心身后。顾大人本来要搬箱子,此刻也傻了眼,扭头张嘴瞪着坛子发呆。坛子里面传出了微弱的抽泣,四周洞壁之中起了窸窸窣窣的细响,仿佛正有大变化处在酝酿之中。忽然一块泥土落在了月牙的肩上,月牙扭头一瞧,只见洞壁渐渐显出巴掌大的一片碎裂,同时就听无心大喊一声:``快跑!''

月牙想都没想,扭头便往外跑,而她前脚蹿出去,后脚便有一只血肉模糊的手臂横空伸出洞壁,一把薅住了无心的衣袖。顾大人看得清楚,拔出砍刀一刀劈断手臂,随即转身也向外飞跑。无心殿了后,要逃之前回头又看了坛子一样,就见坛口抬起一个描眉画眼的女人头,正在七窍流血的狞笑。

周遭泥土落得越来越快越来越密了,手臂接二连三的伸出来,像是洞壁上的寄生虫一样黏湿腥臭,抓挠不止。月牙算是三人中的小个子,腰身又是细而灵活,所以弯着腰摸着黑,一路跑得飞快。顾大人肩宽背阔膀大腰圆,且跑且碰壁,被洞中手臂纠缠的将要迈不开步,只能抡着砍刀一路披荆斩棘。无心手无寸铁,只有一支已然熄灭了的火把,自己连着咬破了三根手指,却是连一滴血也没挤出来。忽然一只血手死死抓住了他的火把,他猝不及防的一松手,黑暗中就见火把瞬间随着血手没入洞壁,从此便是无影无踪。

两人跌跌撞撞的杀向前方,顾大人知道外面是大白天,只要出洞便能安全,所以心劲很足。杀到半路他红了眼睛,将一柄砍刀舞的虎虎生风。眼看前方有了隐隐约约的光亮,他闭着眼睛乱砍乱劈,挣扎着拐过一道弯后,他握着砍刀睁开眼睛,就见月牙站在入口之处,正在焦急的往里面望。

身不由己的被无心推向前去,顾大人还保持着横眉怒目的神情,同时发现洞壁已经恢复原样,似乎只在深处才有怪手肆虐。三人连滚带爬的上了地面,月牙灰头土脸,后怕的没有话说,无心则是当胸给了顾大人一拳:``好家伙,你他娘的把金子藏进了鬼洞,怪不得让我过来帮忙!可是你骗我也就算了,你好意思让月牙也跟着过来冒险?''

顾大人精神一松懈,身体立刻就累酥了,顺着拳头的力道跌成了仰面朝天:``师父,我向天发誓,我真不知道里面有鬼\ldots{}\ldots{}我当时放金子的时候,根本没危险,进去就放,放完我就出来了\ldots{}\ldots{}我能把我的金子送给鬼?我疯了?''

无心真生气了:``你要是被鬼拉进墙里,大不了过几分钟就能憋死。我要是被鬼拉进墙里,我怎么办?我如果逃不出来,要在里面熬多久才算完?''

顾大人可怜兮兮的仰望着他:``师父,别说丧气话,你给我想想办法,怎样才能把我的金子弄出来?''

无心一挥袖子:``去你的吧!我和月牙一宿没睡觉,早饭也没吃,屁都没有挣到一个。我还给你想办法?给你一个嘴巴你要不要?''

说完这话他拽起月牙:``走,咱们回家去!''

\chapter{夜行}

顾大人跟着无心和月牙一路下山回了家。月牙累得都要发昏了,可因认定女人得负责起家里男人的吃喝,所以强挣着煮了一锅面疙瘩汤。自己没滋没味的喝了一碗,她见顾大人还在追着无心说话,只好天旋地转的自己去了东屋,倒在炕上就睡着了。

顾大人一张嘴兵分两路,说话之余站在灶台前,弯腰把锅里的面疙瘩全捞出来吃了。无心蹲在院子里洗头洗脸,回来之后一掀锅盖,就只看到小半锅稀溜溜的面汤。拧着眉毛看了顾大人一眼,他苦着脸长叹一声:``你倒是给我留点啊!''

顾大人顶天立地的站在一旁,双手叉腰吧嗒吧嗒嘴,很诚恳的向他一探头:``你不够吃呀?''然后他伸手戳了戳无心的肩膀:``你别光顾着吃,你听我说啊!''

无心喝了两勺子面汤,无精打采的被顾大人撵进西屋里去了。

顾大人盘腿坐在炕上,斩钉截铁的发誓,说自己当初带着两名卫士入洞之时,洞里干干净净,肯定没鬼。无心枕着手臂侧卧在一旁,懒洋洋的问道:``会不会后来又有人进洞做过手脚?''

顾大人一摆手:``不可能!下山之后我就把那两个小子给毙了!''

无心慢吞吞的扫了他一眼:``为什么?''

顾大人理直气壮的答道:``为什么?杀人灭口,图个心静呗!''

无心收回目光,认为顾大人基本就是个穷凶极恶之徒。这样的人是不该招惹的,但是如果顾大人不愿主动离去,无心也没有本事把他强行撵走。脑筋暗暗的开动起来,无心闭上眼睛,似睡非睡的说道:``大概是附在纸人上的魂魄在洞内冲撞了她,让她有了知觉。老实讲,坛子里面的女人到底是鬼是煞,我没有看清楚。不过无论她是个什么,都难缠得很。她若是肯出洞,我或许可以和她较量一番;她不出洞,我也没有办法。总而言之,我是不敢再进去了,万一被鬼手拽进洞壁里,可是不知哪年才能挣出来。''

顾大人抬手挠了挠头:``要不然\ldots{}\ldots{}我往下挖坑,把洞刨开?''

无心把脸在手臂上蹭了蹭:``好,去吧。''

顾大人一看他这态度,就知道自己说了蠢话。土洞内部虽然不算陡峭,可是一直曲折向下,真要是盲目开挖,不知挖到哪年哪月才能成功。很踌躇的望着无心,他希望无心能够给个主意,而无心如他所愿,果然低声又道:``我看你还是去找位真有法力的和尚老道,求几道镇鬼的符咒试一试吧!只要能够制住里面的东西,三箱金子又不长腿,还不是你想什么时候搬,就什么时候搬?''

顾大人一言不发的望着无心,发现不过是一天一夜的工夫,对方头上已经生出漆黑毛发,似乎也就是睫毛的长度,然而很密,是毛茸茸的一层。

``你说得对!''顾大人开了口:``可是我到哪儿找和尚老道去?''

无心把脸埋进手臂里去,声音越来越低:``我也不知道,我从来不和那些人打交道。你自己想办法吧\ldots{}\ldots{}''

话未说完,余音袅袅。顾大人等了半天,没等出下文。凑过去仔细一看,他发现无心竟然是睡着了。

顾大人从此存了心事,饭量都有所减小。无心趁机煽风点火,一力撺掇他出去寻找真正法师。月牙也很紧张,每天竖着耳朵等待顾大人告辞离去。结果这日清晨,顾大人早早起床,当真走了。

无心和月牙喜出望外,顾大人出了院门不久,月牙就也赶出去买菜割肉,还打了一斤好烧酒回来。两人把门一关,欢欢喜喜的过了一天静谧生活,到了傍晚,无心翻出一对红烛,眉飞色舞的就要布置洞房。不料天还没有黑透,顾大人却又回来了。

顾大人进门之后,先抄起葫芦瓢从水缸里舀了一瓢水喝,然后抹着嘴对无心说道:``我今天走了好几座庙,屁都没有找来一个,还差点跟和尚打了一架!怎么办吧?''

无心静静的看着他,看了半天,最后低声开了口:``帮人帮到底,送佛送到西。毕竟当初是我主动找上了你,所以我有始有终,如今再为你出一次力。六十里外有一座青云观,我亲自前去碰碰运气。不过我有言在先,无论此次运气如何,你都不能无限期的住在我家里不走。你若真是个有本领的人物,应该也不至于少了三箱金子就不能东山再起!''

顾大人知道对方是新鲜小两口,自己人高马大的住下来,的确是挺碍眼。恭而敬之的满口答应了,他问无心:``你趁夜就走?用不用我陪你?''

无心把月牙叫过来,自己咬破指尖狠狠的挤了半天,挤出一点淡淡鲜血,涂抹上了她的眉心。眉心是人魂魄聚集之所,眉心护住了,魂魄就稳。换上一双新布鞋,他又嘱咐了顾大人好好保护月牙,然后便推门走出去了。

猪头山下一带的县镇,近些年除了增添铁路火车之外,几乎没有大的变化。无心沿着小路走在黑夜中,心里想起了许久许久之前的往事。往事之中的他还带着一条假辫子,搀着玉儿走在这条路上。左邻右舍的闲话越来越盛了,于是他们决定躲进山里去生活。玉儿老了,走不多远便要喘粗气,他弯腰背了玉儿往前走,心里知道再过些年,玉儿就会死了。

前方隐隐有了光亮,是路边一家饭馆门外挑了灯笼。前后都荒凉,白天路上人多,还会有各种饮食摊子,晚上众人收了摊,就只剩下饭馆还亮着灯。无心不渴不饿,所以直走了过去。

良久过后,前方路边又挑出了一只灯笼,灯笼上的字号十分眼熟,后方房屋的轮廓被隐约照耀了,看着也是似曾相识。无心若有所思的停住脚步,对着门口看了又看,末了发现自己竟是兜了个圈子,这家饭馆,自己方才已然经过一次!

因为身边既无累赘也无牵挂,所以无心无所畏惧的迈步走向前去。天气还不算凉,饭馆门口垂下油腻的旧竹帘子,帘后隐约传来婴儿的啼哭声音。无心抬手一掀帘子,迎面就见一个直挺挺的男人,木雕泥塑一般僵硬的笑容可掬!

``请进,请进。''男人用掌柜的口吻招呼了他,语气之中毫无波动,像是照本宣科的在读文章:``店里什么都有,您要吃点什么?''

无心绕过掌柜,走了进去:``什么都有,都有什么?''

掌柜慢慢转过了身,一步一步沉重的跟上了他:``面条,包子,米粥,炒菜。''

一名敞着怀的妇人从前方缓缓经过,臂弯中的婴儿含了她的奶\textbar{}头,正在委委屈屈的抽抽搭搭。无心停下脚步,扭头望向了店铺角落。

角落处的桌子后面,坐着个俏生生的小姑娘,正是岳绮罗!

几盏油灯摆在周遭桌上,火苗窜起多高,把角落照得一片光明。岳绮罗双手扶着桌沿,对着无心粲然一笑,随即快乐的拍了拍手,用稚气的声音喊道:``大哥!好久不见,想我了吗?''

无心也笑了一下,绕过桌椅走向了她:``我不想你,但我看你好像是很想我。''

岳绮罗笑得双目弯弯,脸上阴影随着火苗一跳一跳:``哈哈哈,大哥没感情!''

无心在她对面坐了下来,同时发现她的面孔已经恢复光洁,只是右眼的眼珠有些异常,黑眼球的边沿缀了一个红点子。

``你有感情。''无心说道:``弄几个纸人对我装神弄鬼。''

岳绮罗从怀里摸出一张纸剪的人形,对着无心一挥:``知道是纸人,你还害怕不成?或者说,是我的纸人吓了你的月牙,你心疼了?''

话音落下,她手指一弹。纸人轻飘飘的飞出去,落上灯焰化为灰烬。

后方的厨房里响起了煎炒烹炸之声,显见岳绮罗是要在此地吃上一顿。无心垂下眼帘,就觉四周阴魂涌动:``没错,我心疼了。''

岳绮罗用手指轻轻一挠脸蛋:``哟,哟,真不知羞!''

无心抬眼看了她:``你把我引过来,除了倾诉相思之情,还有别的事吗?''

岳绮罗笑眯眯的看着他,脸是天真无邪的模子,一双黑洞洞的大眼睛却是显了岁数,仿佛已经见识过沧海桑田:``大哥,相思之情,还不够你听的吗?''

无心摇了摇头:``若论年纪,你至少该称我一声祖爷爷。对于祖爷爷,尊敬就可以了,没有必要再害相思。''

岳绮罗怔了一下,随即嘿嘿嘿的又笑了起来。正当此时,妇人捧着个大砂锅走了过来,两只巴掌似乎不知道烫,结结实实全贴在砂锅外层。岳绮罗欠身揭开砂锅盖子,很销魂似的深吸了一口气,然后对着无心笑道:``请你尝尝老板娘的手艺,也算我没有白白引你过来一趟。''

话音落下,她抄起筷子伸进锅内,连汤带水的夹起了一块东西往嘴里送。而无心向内一瞧,只见沸腾汤水之中窝着个小小的婴儿,周身皮肉都被煮烂了,一双眼睛却还睁着,似乎就是妇人方才怀里奶着的婴儿。

无心叹了一声:``岳绮罗,你收了人家夫妻俩的魂魄,又驱使着他们煮了自己的孩子给你吃。''

岳绮罗鼓着面颊,嘴唇蠕蠕的动,最后低下头去,她从嘴里吐出一串细细的骨头,正是婴儿的一只小手。咽下口中的嫩肉,她对着无心一挑眉毛:``我看出来了,你很想做人;可是我不一样,我很不想做人!怎么?不爱听?想杀了我?嘻嘻,别说你杀不了我,就算我真死了,也不会魂飞魄散。我可以投胎为人,接着这辈子继续往下活。''

无心知道自己奈何不了她,她也无法控制自己。被这么个东西看上了,彼此之间不分个胜负出来,恐怕将来总也没有安生日子可过。

\chapter{偶遇道长}

岳绮罗坚信自己需要补养,甚至希望自己可以接着一百年前的年纪继续成长。她认定了自己是个美人坯子,可惜年华凝固在了豆蔻梢头。一朵鲜花绽了骨朵,不盛开一次真是太可惜了。

撅着薄薄的小嘴唇,她津津有味的吮吸着嫩豆腐似的婴儿肉。肉软的像汤,汤又软的像肉,她连肉带汤连吃带喝,忽然打了个心满意足的饱嗝,她问:``大哥,你怎么不吃?''

无心在蒸腾的雾气中摇了摇头:``我是人,人不吃人。''

岳绮罗吐出一根细骨头:``谁说人不吃人?你没见过人吃人?''

无心答道:``被吃的要死,吃人的也要死。与其如此,不如不吃。''

岳绮罗伸长了手臂,用筷子在砂锅里捞来捞去:``大哥,可惜你的血肉有毒,否则我一定要尝一尝你。''

无心想了想,却是问道:``段三郎好不好吃?''

岳绮罗换了汤匙,意犹未尽的舀出碎肉:``你也知道段三郎?段三郎没什么好的,我当时只是收了他的魂魄来玩,玩腻了,就让他去死了。''

无心笑了一下:``可是段家也没轻饶了你!''

岳绮罗抿着嘴,笑微微的向他一歪脑袋:``段家算什么,破落户而已。有人想要对付我,怎样都能找到机会;段三郎的性命,就是他的机会!''

无心饶有兴味的看着她:```他'是谁?''

岳绮罗喝下一口肉汤,然后对他摇了摇汤匙:``我不告诉你。''

无心站了起来:``我走了。''

岳绮罗放下汤匙:``不许走!''

无心转身就跑,瞬间冲出饭馆大门。而岳绮罗眼看追逐不上,当即起身从怀中扯出长长一串纸人。纸人凌空飞起,而她同时念念有词,虚空画符。最后对着窗口猛然一挥衣袖,她大喝一声:``去!''

纸人随着疾风飘出窗外,隐隐约约的化成人形,张牙舞爪去追无心。无心怕是不怕,可也懒得和一群纸人撕撕扯扯。一口气跑出两里地,他突发奇想的在岔路口拐了个弯,结果差点被疾驰而来的敞篷大马车碾成饼子。

大马车十分威武,前头两匹阿拉伯马并驾齐驱,后方悬着两盏雪亮的风雨灯。车夫慌忙勒住缰绳,只听一阵人叫马嘶,车是急刹住了,车后座上的人却是猝不及防,惊叫着向前跌了下来。无心就听``咚''的一声,正是一柄拂尘从天而降,砸在了自己的头顶心上。

无心知道自己是惹了祸,连忙弯腰捡起拂尘。车上乘客本来摔了个大马趴,此刻也自己爬起来了。无心放眼一瞧,只见对方头戴道冠,身穿道袍,乃是个器宇轩昂的道士。道士一甩袍袖,对着无心一拱手,朗声说道:``福生无量天尊!''

无心没想到道士这么有涵养,摔成狗吃\textbar{}屎了还不骂人。毕恭毕敬的双手奉上拂尘,他正要道歉,不料道士忽然变脸,甩手就是一个大嘴巴:``好你个混账东西,大半夜的胡跑什么?万一把本道爷摔出个三长两短,你赔得起吗?''

无心冷不防的挨了一记耳光,登时捂着脸怒问:``你是谁?怎么随便打人?''

道士在风雨灯旁扬起大白脸,傲然答道:``贫道法号出尘子,当今大总统都要称我一声真人,今夜打了你,你还不服气么?''

无心咽下一口恶气,抬手向后一指:``道长,前方可是有鬼!我好心前来拦你,你还不领情么?''

出尘子冷笑一声,上前一把夺过拂尘,随即转身昂然上车。端端正正在坐稳当了,他一甩拂尘,目空一切的说道:``笑话!贫道在此,倒要看看谁敢作祟!''

话音落下,前方忽隐忽现的飘出了白色人影,正是纸人追踪而来。车夫坐上车去,显见是害怕了,挥着马鞭不敢出声,而出尘子嗤之以鼻,声若洪钟的说道:``不必怕,走!''

车夫闭了眼睛一甩马鞭,大马车呱嗒呱嗒的又上了路。马车越是向前,人影越淡,待到大马车一拐弯上了大路,人影竟是消失无踪。出尘子心中得意,摸出白绸子手帕擦净了掌心尘土,他将手帕顺风向后一抛,抛完之后感觉不对劲,猛然回头一瞧,正和无心打了个照面!

无心一直扒在车座后面,此刻被出尘子发现了行踪,就手足并用的翻过座位,坐到了出尘子身边。两人大眼瞪小眼的对视了片刻,最后还是无心先开了口:``道长,你是青云观的住持吗?''

出尘子没有回答,拧起眉毛质问他:``谁让你上车的?下去!''

无心上下打量着出尘子,暗想此人似乎真是有几分本领,自己可不能轻易放过了他。就算他不肯出面帮忙,能给出几道符咒也是好的。

无心打了如意算盘,赖在马车上死活不下。硬是一路赖到了青云观。而天亮之时,岳绮罗离开饭馆,独自也向文县方向走去了。

临走之时,她耍了个恶作剧,让掌柜夫妇坐到了狼藉桌前。出门之后她放出了二人的魂魄,不过片刻,夫妇便会一起还魂。还魂之后面对着满桌的骨头,岳绮罗想象不出他们会有什么反应。

诸如此类的把戏,她是永远玩不够的。如果无心不提起段三郎,她也许真就把对方彻底忘怀了。段三郎死的很热闹,是她第二个傀儡;第一个傀儡是她身边的小丫鬟,小丫鬟一定不明白自己为何如此深爱小姐——因为她的魂魄都落在小姐手里了。

岳绮罗想要找到栖身之处,无心喜欢做人,那她就用人来征服他。其实征服了又有什么用?好像也没什么用。她不能吃了无心,即便把无心炼成了丹,她也没胆子服用。让他爱她陪她?可是久了也会腻,况且他根本也不爱她。

岳绮罗所走的道路很偏僻,身边没有旁人经过。把手伸进衣裳里面捂住一侧微隆的小胸脯,她在刺目的阳光下眯起了左眼。

右眼点缀着无心的一滴血,已经瞎了。

当天晚上,无心回家了。

家里一切太平,月牙正在望眼欲穿的等着他。无心从怀里拿出一沓子纸符递给顾大人:``青云观的住持老道亲自画的,这要是再没用,那我也没法子了!''

顾大人半夜没睡好,落枕了,歪着脖子接了纸符一张一张的看。看过之后来了精神:``师父,还是你行!今晚咱们就再上山去?''

无心从月牙手中接过毛巾,满头满脸的擦了一遍:``我们不去,你自己去吧!''

顾大人登时张大了嘴:``啊?''

无心把毛巾交还给月牙:``你知道青云观那牛鼻子派头多大吗?我脸都不要了,硬是缠着他给我画了这么多张。顾大人,你自己摸着心窝想一想,我对你是不是也算仁至义尽了?''

月牙听了无心的话,感觉十分有劲,不是个懦弱的丈夫。而顾大人彻底傻了眼,捏着纸符张口结舌。

无心不再理他,把月牙叫进了西屋。翻出一张纸一支笔,他让月牙把荷包里的黄符拿出来,依样画葫芦的描了一张,打算再去趟青云观,让出尘子认一认它的来历。不能坐在家里等着岳绮罗打上门来,他得早早做下准备。

不过对着月牙,他可是没有多说,尤其是不提岳绮罗。只怕自己说多了,惹得月牙害怕。

三天过后,顾大人犹犹豫豫的并没有独自上山,而无心则是又跑长路去了青云观。

青云观位于青云山上,气势巍峨,宛如天宫。平心而论,青云山除了名头动听之外,各方面都未见得比猪头山高明多少,只因为有了青云观,才成了一处了不得的名胜。

青云观属于正一派,观内空气还算自由。无心在小道士的引领下绕过正殿,不知过了几道门拐了几道弯,最后在一处清幽如画的小小院落里,他见到了出尘子道长。

出尘子穿着一身雪白的绸缎裤褂,披头散发的站在游廊里面,手中端着一杯来自京城的马爹利。居高临下的望向无心,他侧身靠向廊柱,同时举杯抿了一口酒:``听说你有一张奇怪的符要给我看?''

无心有求于人,十分恭敬,双手把一张折好的白纸展开,上前送到了出尘子面前。出尘子接过去上下瞧了两遍,保养良好的白脸上没什么表情:``从哪里描来的?''

无心答道:``从一口棺材上。''

出尘子忽然笑出两道四十多岁的鱼尾纹:``我看不懂。''

无心点了点头,对着出尘子一拱手:``打扰道长了,既然道长看不懂,那我就只好告辞了。对了,道长,我再对你说一句——棺材里的人,前一阵子,出来了。''

只听``啪嚓''一声脆响,出尘子的玻璃酒杯脱手而落,在石板地上摔了个粉碎。

\chapter{前尘旧事}

出尘子避开了地上的玻璃碎片,飘然走下围着无心转了一圈,末了停在他面前问道:``你到底是谁?''

无心正在暗暗的盘算心事,忽然听他问了,也不说实话,只莫测高深的一笑:``我这个人,僧不僧俗不俗,也说不清究竟算是个什么人,四处漂泊,混口饭吃罢了。''

院内拂过一缕清风,吹动了出尘子一头乌黑亮丽的披肩长发:``你认识棺材里的那个人?''

无心抬头正视了他:``本来是不认识,但是她自从回到人间之后,自称是爱上了我,终日死缠烂打,让我不胜其烦。实不相瞒,我也不能算是全无本领,可是她道行极深,我竟拿她没有办法。''

出尘子抬手托着下巴,很有保留的扫了无心一眼:``爱上了你?''

无心一点头:``没错,可是我都有老婆了。''

出尘子张开五指向后一拢头发:``道行极深?''

无心继续点头:``没错,埋在地下的尸首都能被她召唤出来伤人。''

出尘子放下了手,从长发的中分缝隙中向外看他。而无心不等他再问,直接挑明了来意:``道长,请你告诉我她的来历,否则我心里糊涂着,想对付她都不知道从哪里下手。''

出尘子背了双手,又一阵风掠过院子,他的长头发全垂到了眼前:``我不知道。''

无心叹息一声:``好,既然你不说,我只好把她引到青云观来。久闻道长是位活神仙,活神仙见了活妖怪,想必会有一番切磋,定然十分好看。''

出尘子听到这里,抬手一撩长发,勃然变色:``胡说八道!这跟我有什么关系?我说不知道,就不知道!''

无心发现出尘子这个人不善于说谎,前言不搭后语的满口漏洞。本来他也没想过出尘子真能知道些什么,可是出尘子又摔酒杯又闹脾气,让他不得不相信对方和岳绮罗有些渊源。对着出尘子一拱手,他摆出一副死猪不怕开水烫的架势,转身就要真走。没走出三五步,他果然被出尘子叫住了。

出尘子带着无心进了屋子。青云观是大观,出尘子又是位常和达官交往的尊贵道士,所以他的住所外表幽雅,内中豪华。盘腿坐上一张红木大罗汉床,他不看无心,直接垂着眼帘开了口:``棺材里的人,应该还是个小姑娘吧?''

隔着一张小炕桌,无心倚着床围子也坐舒服了:``没错,据说是十四岁。''

出尘子接着说了下去,表情有些为难:``她\ldots{}\ldots{}她算是我的太师叔祖,无父无母,和我太师祖一起长大。我师父说师祖说太师祖说太师叔祖从小就痴迷于鬼神之术,先还只是画符念咒而已,后来竟然挖坟掘墓,对着死人活人一起演练起来,惹出许多凄惨祸事。太师祖看不下去,想要劝醒了她,不料未等开口,她竟是夜里自杀了。''

出尘子说到此处,长长的叹出了一口气:``人既然是死了,太师祖也就无话可说。哪知过了十几年,一个小女孩找上门来,言谈举止极其类似太师叔祖。太师祖先还以为她是借尸还魂,可是仔细一看,太师叔祖竟是魂魄不散,投胎成了人身。原来太师叔祖求的便是灵魂不灭,先拿着不相干的旁人练习够了,她才一索子吊死了自己,要试一试自己的真本领。太师祖预感不妙,可对她又奈何不得。在接下来的几十年里,太师祖和太师叔祖一直争斗不止,太师叔祖死了几次,可是对她来讲,所谓死亡,无非是换了一具皮囊而已。''

无心插了一句嘴:``所以你太师祖就决定把她封起来?''

出尘子点了点头:``太师祖年纪越来越大,自知太师叔祖已成妖物,所以带着我师祖多方寻找,最后终于在文县找到了太师叔祖。当时我的师父也还是个小孩子,亲眼见了太师叔祖一面,说太师叔祖被太师祖封进棺材之时,看起来就是个平平常常的丫头。

太师祖做完这件大事之后,就在青云山上修建了青云观,说要镇一镇太师叔祖的邪气。但是自从太师祖羽化之后,此事也就不再被人提起。到了如今,整座道观之内,除了本住持之外,更是无人知晓百年之前的这一段生死之斗了。''

无心不再言语,心想岳绮罗的确是邪,可你那太师祖也不算完全的正。你太师祖所布的阵,乃是以毒攻毒的法子,不但把岳绮罗埋在了一口荒井旁边,而且为了确保井中阴气旺盛,能够配合阵法压住岳绮罗,还虐杀了小丫鬟投入进去。殊不知岳绮罗在至阴之地被禁锢的久了,反倒邪气更盛;而小丫鬟生前一直爱戴小姐,死后心意不变,竟然成了厉鬼,一心要救小姐出来。

出尘子讲完这一段故事,扭头望向了无心:``是谁破了我太师祖的阵法?''

无心犹豫了一下,把小丫鬟拎出来当了挡箭牌——小丫鬟惨死,小丫鬟杀人,小丫鬟撞破石壁\ldots{}\ldots{}全是小丫鬟的错。而他之所以会被岳绮罗缠上,完全是出于偶然,以及他太英俊。

一场谎言说到头,他问出尘子:``道长,你有没有办法把岳绮罗重新镇住?''

出尘子摇了摇头:``没有。''

无心追问一句:``没有?''

出尘子摆了摆手:``没有。''

随即他伸腿下床,背着手在地上来回踱了一圈:``一百多年前的事情,我也就只知道这些,全部对你讲了。总而言之,我是无计可施。''

无心客客气气的跟了上去,对着出尘子一拱手:``道长,多谢解惑。不过你也是个有慈悲心的人,总不能看着我被你太师叔祖追得满街跑。''

出尘子以为他要赖上自己,登时有些紧张:``什么太师叔祖!我和她没有任何关系!''

无心笑了:``别误会,我无意把道长和妖孽归到一类,只想请道长按着黄符的样式,给我再画几张。实不相瞒,你太师叔祖挺怕这符!''

出尘子一甩衣袖:``放你的狗屁!我太师叔祖一百多年前就上吊死了,我没有太师叔祖!你再敢说她是我太师叔祖,当心本道爷抽死你!''

无心被他喷了一脸唾沫星子,可是人在屋檐下,不得不低头:``道长,你别急啊,我又没对别人说你太师叔祖从棺材里出来吃人。你太师叔祖死去活来把自己折腾成了妖怪,我更是当成秘密,一直保存在心里呀!''

出尘子抬手向他一指:``你敢威胁我?''然后不等无心回答,他移动手指虚空画符,最后一笔直接点上了无心的眉间。

无心满不在乎:``道长,我可没威胁你。我只是求你给我多画几道黄符。你不同意,我走就是了。''

出尘子气得长发凌乱。太师叔祖的威力他没见识过,他只觉无心比太师叔祖可恨多了!

出尘子无可奈何,只好更衣洁面梳头。画符乃是一件庄重之极的大事,仪式十分繁琐;但因出尘子已经颇有道行,所以不受束缚,自有一派潇洒形式。

待他画出三道黄符之后,无心恭恭敬敬的问道:``道长,这符也有名目吗?''

出尘子怔了一下:``名目?这是我太师祖自创的符咒,平日也用它不着,没有专门的名目。''然后他把毛笔往案上一掷,冷着脸说道:``故事我讲了,黄符我也画了。明日我就要去天津,不定何时才能回来。你我就此别过,我也不送你了!''

无心见好就收,立刻告辞。天黑的时候他到了猪嘴镇,敲开家门之时累得腿都直了。好在家里有月牙,还有顾大人。顾大人把他搀回了西屋炕上,没等他坐稳,月牙的热毛巾劈头盖脸拍下来,把他一头一脸的尘土全擦干净了。

到家之后的一搀一擦,让无心幸福的快要落泪。掏出怀中三道黄符,他让月牙再缝几个荷包,让顾大人装一张,他留一张,剩下一张还给月牙,横竖黄符没有分量,多带一张也不沉重。吃过一顿热气腾腾的晚饭之后,无心让月牙坐到自己身边做针线活,前些天本来想把顾大人撵出去的,如今他也舍不得撵了,让顾大人上炕一起坐。

双手分别搭上了月牙和顾大人的膝盖,无心慢条斯理的讲述了岳绮罗的来历。月牙和顾大人听是听了,然而没听明白,因为实在是算不清其中的辈分。等到无心说完了,月牙用牙齿咬断了一根线:``管她是个啥呢,反正离咱们远点就行。明天是不是该买大白菜了?多买点,囤起来够一冬吃的。''

顾大人掏着耳朵,也有话说:``师父,你真不和我上山去了?三箱黄澄澄的金子啊,你就忍心不要了?''

月牙立刻抬头看他:``你别撺掇他跟你往山里跑!见了鬼不躲着走,还要自己往门上送?我俩明天买大白菜去,没工夫跟你上山见鬼。''

顾大人皱起了眉毛:``这个小娘们儿啊,就是头发长见识短。''抬手一搡无心,他转移了目标:``你说句话,上不上山?''

无心挪到了月牙身边,对着顾大人笑:``我还是想和月牙去买大白菜。''

顾大人一拍大腿,十分失望:``你啊,就知道围着娘们儿打转,你再活一万年,也还是没出息!''

月牙告诉无心:``你别理他!''

岳绮罗独自走在黑暗的文县大街上,街上白天发生过激战,如今满街都是沙袋和死尸。

她找到一处漆黑的角落,抱着膝盖坐了下去。她很饿,右眼也有些疼痛。虚弱的时候她会压制不住右眼中的毒血,血点渐渐蔓延开来,她的右眼珠子变成了鲜红颜色。手指触到地面,她迟疑了一下,决定还是亲自动手,省点法力。

将一具年轻的尸体拖进角落,她捡起一把军刀,劈开了尸体的头颅。手指蘸了温热的脑浆送进嘴里,滋味淡而微腥。忽然听到遥遥传来一队马蹄声音,她眨着渐渐恢复黑白的右眼,用袖子擦了擦嘴,然后起身走了出去。

\chapter{百年好合}

节气一过白露,便是一天冷似一天。清晨起床之后,月牙试着烧热了西屋的炕,于是顾大人觅着热气溜出东屋,很自然的上炕取暖去了。

早饭是面疙瘩汤,配着腌萝卜条。顾大人捧着大碗坐在炕角,靠着墙壁喝出一头大汗。无心披着棉被跪在炕边,说自己昨天走长路累着了,已经连起床吃喝的力气都没有。于是月牙端着一碗面汤站在炕边,很有耐心的一勺一勺喂他。

顾大人有些嫉妒,偷眼审视前方二人,就见无心像条狗似的仰头对着月牙,两只脚垫在屁股下面,露出一排整整齐齐的脚趾头。无心的脸是白生生的,月牙的脸是粉嘟嘟的;无心的嘴唇红通通,睫毛随着咀嚼动作一颤一颤;月牙的嘴唇水嫩嫩,微微的撅了起来,仿佛是在替无心害烫。

一碗面汤喂完,无心闭着眼睛歪着脑袋,一头蹭上了月牙的胸口。月牙打了他一下,端着大碗往外走。顾大人依旧盯着月牙的身影,看她胸脯一颤一颤,屁股一扭一扭,胸脯和屁股之间是一段细长的腰。顾大人是识货的,认为凭着月牙的姿色,兴风作浪是不能够,可当个姨太太是太有资格了。忽然想起了落在文县的几个骚姨太太,顾大人有些怅然,因为不知道她们是死是活,还是跟着别的男人跑了。

顾大人有日子没碰过女色了,单是一动心思,裤裆里就支了帐篷。正当此时,无心忽然扭头,对他一笑。

顾大人猝不及防的和他打了个照面,不由得吓了一跳。而无心披着棉被爬向了他,四脚着地奇快无比,姿势与神情都不大像人,仿佛只是摇头摆尾的一瞬间,就已经凑到了他的面前。

从昨天晚上开始,无心对他生出了一点好感,此刻便从棉被下面伸出一只手,轻轻搭在了他的大腿上:``你夜里冷不冷?''

顾大人瞪着眼睛看他,头发都要竖起来了:``不冷。''

无心认真的告诉他:``如果冷了,就让月牙给你烧炕。''

顾大人运足力气,一脚把他蹬出老远:``去去去,大清早的你怎么像个鬼?离我远点,一边蹲着去!''

无心一片好心去关怀他,结果却换来一脚。两人当即开战,顾大人放下饭碗打开窗户,拎起无心就扔出去了。月牙正在院里思量着如何放置大白菜,眼看无心飞了出来,她也不思量了,追着顾大人好一顿骂,骂的顾大人一声不出。

到了中午,无心和顾大人讲了和。无心跟着月牙出去买大白菜,顾大人负责给柴禾垛搬个家,腾出地方放大白菜。生活琐事最耗时间,三个人忙了整整一个下午,总算安顿下了一百棵大白菜。

晚饭也是大白菜,顾大人吃饱喝足之后就没了雄心壮志,哈欠连天的只是想睡。他睡了,无心和月牙在西屋也上了炕。月牙终于买齐了大白菜,没了心事,背对着无心闭了眼睛,正是朦胧之际,身后忽然一暖,竟是无心横跨火炕,侵入了她的被窝。

她一哆嗦,一时也不知道怎样才好,索性一动不动的装睡。而无心抬手轻轻扳了她的肩膀,又低声说道:``月牙,顾大人不知道哪天才能走,我们\ldots{}\ldots{}别等了。''

月牙通身发起了烧,手脚都失了控制,躺在炕上动不得,唯有一颗心在扑通扑通的大跳。无心被被窝里挤挤蹭蹭,紧贴着翻到了她的胸前。她的手被他压在了身下,她的掌心贴上了他光裸的半个屁股。

月牙的呼吸和心跳全乱套了,拼了命的要把手抽出来。手抽出来了,又被夹在了两人之间。手背贴住了一根陌生东西,滚烫梆硬的一跳一跳。胸膛里立时起了狂风骤雨,月牙知道自己是碰上男人的命根子了。

翌日清晨,顾大人推门进了堂屋。眯着眼睛望向灶台前的月牙,他迷糊了半天才反应过来——月牙的头发换样式了。

两条垂肩的大辫子被拆开了,光溜溜的盘成了脑后一个圆髻,上面还插了一朵小红线花。恍然大悟的``哦''了一声,顾大人嬉皮笑脸的开了腔:``哟,看来昨夜有好事啊!''

月牙背对着他不回头,借着锅里腾出的热气肆意脸红:``怎么的?我俩本来就是两口子。''

顾大人抱拳拱手:``恭喜恭喜,祝你俩——''

他想说白头偕老,可是无心不会白头;又想说早生贵子,但是无心也没有种子。舌头在嘴里转了一圈,他思索着把话说完整了:``百年好合。''

月牙把今天算成了是新婚第一天,生怕顾大人胡说八道讲晦气话;如今听他狗嘴里终于吐出了象牙,脸上不禁有了笑模样:``承你吉言。''

不料顾大人随即又来了一句:``小白脸子是占便宜,说弄个老婆就能弄个老婆。''

月牙感觉顾大人就像脱缰野马似的,言行全都令人无法预料和控制,所以赶紧推门出去了,想要躲开顾大人的高论。而顾大人独自进了西屋,见无心又披着棉被坐在炕上。双方四目相对,无心对他一笑:``嘿嘿。''

顾大人微笑回礼,心想也不知道这玩意儿到底多大岁数了。

月牙和无心好的蜜里调油,顾大人看在眼里,心中就酸溜溜的不得劲。如此过了不久,文县忽然传出消息,说是张小毛子落败了,县城又回到了丁大头的手里。顾大人一看形势有变,立刻蛰伏下来,不敢妄动。

顾大人蛰伏了,张小毛子也蛰伏了,只有丁大头旅长君临文县,可以肆意的耀武扬威。在卫士的簇拥下踏进文县最大的戏园子里面,他看起来是异常的高。高的其实不是他,而是骑在他脖子上的九姨太。九姨太穿得花团锦簇,对待骡子大马一样驱使着丁旅长往楼上走。丁旅长似乎爱她爱到了肝脑涂地的程度,脸都不要了,驮着她就真上了楼。

卫士们跟在后方,暗笑不止。九姨太看起来只有十四五岁的年纪,还是一口未长成的小嫩肉,没想到却正投了旅座的胃口。前头八个女人一下子都不值钱了,编外的娘们儿更是彻底没了地位。丁旅长把九姨太驮进了雅间,她不下令,丁旅长能一直驮着她。

大戏唱起来时,九姨太和丁旅长并肩坐了。偏着脸望向丁旅长,她的右眼珠上缀着个针尖大小的红点子:``怎么,心疼你的老八了?''

丁旅长油光满面的看着她,心中一阵一阵的茫然,有点爱,更有点怕:``绮罗,我心疼她干什么?往后家里你说了算,你要怎样就怎样。''

岳绮罗满意的转向前方。丁旅长也面对了戏台,心中一片迷惘。

他是在不久前的一个夜里捡到岳绮罗的,捡她,无非是看她有个好模样,带回家里当个丫头,睡也行用也行。可是等岳绮罗到了家之后,空气就莫名的怪异了。

到底是怎么个怪异,丁旅长也说不清楚。是岳绮罗先勾引的他,豆蔻年华的小少女,脱光了别有一番诱惑力。然而一觉醒来,他就感觉自己仿佛失了魂魄一般,竟然思想主意都没有了,万事只想听凭岳绮罗的吩咐。本来他是很爱老八的,八姨太也才十八岁,漂亮得很,现在他想起老八,也还是喜欢。老八一直看不惯岳绮罗,又没心眼,昨天不知受了谁的撺掇,公然的想和岳绮罗打一架。今天早上他回了家,岳绮罗让他去杀老八,他梦游似的,就真把老八毙了。

``其实不至于。''他木然的想,姨太太之间闹矛盾,不至于让他动刀动枪。岳绮罗把老八的尸首拖到房里,用一把刀子砍下去,像砍瓜似的,很轻松的砍开了老八的脑袋。老八的脑浆还冒着热气,被岳绮罗用小勺子舀起来,送进粉红色的小嘴唇里,脑浆娇嫩,嘴唇也娇嫩。丁旅长眼看着岳绮罗吃饱喝足,心情是莫名的平静,仿佛吃活人脑浆是最天经地义的事情,只是老八没犯大错,所以``不至于''。

许多人都看出丁旅长最近有些呆,说话做事都有点出格,但又没过分到疯的程度。丁旅长自己也有些知觉,可是依旧麻木不仁。

他不知道自己是阳气重杀气也重,所以岳绮罗没能彻底收走他的魂魄。否则他完全变成行尸走肉,就不会有这些困惑了。

一场唱念做打的大戏看完,岳绮罗感觉很过瘾,拍着面前栏杆哈哈大笑。笑过之后她问丁旅长:``想不想彻底绝了后患?''

丁旅长没听懂:``什么?''

岳绮罗笑道:``给我一队兵,我帮你抓到顾玄武。''

顾玄武是顾大人的大名。丁旅长想了想,认为自己的确是有必要抓到顾玄武,九姨太的要求很合理,自己应该答应。

于是在隐隐的恐慌之中,丁旅长对岳绮罗点了头:``好。''

\chapter{借刀杀人}

岳绮罗无法完全控制丁旅长,丁旅长抱着她亲了几个嘴,她虽然不大耐烦,但是也让亲了。亲完之后丁旅长了结心愿,又想不起接下来要做什么,能够失魂落魄的安静许久。他一安静,岳绮罗也安静了,自己默默的坐在房里想心事。

她想自己总是对一些看起来不可能的事情着迷。魂魄不灭本来是不可能的,她研究了一辈子,把不可能变成了可能;肉体不灭也是不可能的,她研究了几辈子,还没得出眉目。无心的脑袋被砍掉了半个,说长出来就长了出来,她十分羡慕,认为无心很有资格做自己的伴侣,然而无心的心上人是个葫芦身材的小娘们儿;这个娘们儿能干活,更能撒泼,真的就只是个娘们儿而已,平心而论,是配不上无心的。

对着镜子扒开右眼眼皮,她仔细研究着眼珠上的血点子。看够了血点子,她向后一仰头,开始宏观的审视自己。审视完毕之后,她感觉自己很美,很可爱。

窗台上的瓷花瓶里汩汩的发出微响,是丁家八姨太的半瓶血肉在蠕动,心肝脾肺剁成的,附着几缕陌生的魂魄。魂魄本能似的不大安稳,但是被岳绮罗封住了,所以不安稳也没有用。没滋没味的吧嗒吧嗒嘴,她忽然感觉有些饥饿。她人小,胃口也小,于是认为应该用最好的食物填满自己有限的肠胃。吃好喝好才能长出好身体,才能有力量压制住右眼中的毒血,岳绮罗认为这是常识。吃鲜嫩的婴儿,喝年轻的脑浆,对她来讲,也都是常识。

她起身离开房间,给自己打猎去了。

无眠的夜里,岳绮罗躲在阴暗角落里吃吃喝喝,而顾大人躺在被窝里,也是长吁短叹。

西屋里始终是不安静,不是哼唧就是说笑。顾大人知道那是无心和月牙在干好事。这点好事干得月牙整天精神焕发,像架风车似的从早忙到晚,并且不闹脾气,总是喜上眉梢,大姑娘劲儿一点都没了,通身彻底换了媳妇做派。无心成了她的宝贝,被她伺候的面面俱到。顾大人年纪轻轻,还不到三十岁,身边又没女人,看得心里酸溜溜,也想享受宝贝待遇,可是月牙又不肯惯着他。

顾大人直勾勾的瞪着眼,等着西屋消停下来,自己也好安心睡觉。然而西屋二人不知道他的苦楚,在温暖的火炕上鲤鱼打挺鹞子翻身,十八般武艺都练绝了。末了月牙坐起来,掀了被子去看无心的下身。揪了揪鸟又摸了摸蛋,月牙心中暗想:``该长的都长齐了,犁是好犁地是好地,真就长不出苗结不出果吗?''

月牙摆弄着无心的东西,心里存着一份希冀,希望无心能和自己开花结果,养几个娃娃出来。而顾大人终于得了清静,便披着新制的薄棉袄下炕出门,要去外面茅厕里撒一泡尿。秋天短的似乎只有几天,夜里冷得有了冬天气息。顾大人打着哈欠哗哗撒尿,尿着尿着,忽然打了个冷战。撒尿打冷战是正常事情,不过此刻这个冷战打得很不舒服,心惊肉跳的难受。顾大人是出生入死过许多次的人,别有一番敏感。系好裤子吸进几口冷空气,他一俯身趴下去,把耳朵贴上了落着干白菜叶的地面。

隐隐的,似乎是有大队人马来了!

顾大人一挺身窜起来,想都不想,直接就要回屋拿刀拿枪。然而几大步迈进堂屋之后,他临时转弯敲响西屋房门,压低声音叫道:``你俩别日了,外面好像不大对劲!''

随即他扭头冲入东屋,瞬间就把武器披挂了上。再出门时,无心和月牙已经衣衫不整的站在了堂屋里。月牙自从跟了无心,已经被吓成了傻大胆,迎面就问顾大人:``又来鬼了?''

顾大人一摇脑袋:``不像是鬼,好像是人。''

无心也换上了新棉袄,一边系纽扣一边问道:``人?来兵抓你了?''

顾大人来不及多说,跑去院内又喘了几口粗气。扭头越过篱笆院望向老树井台的方向,他已经看清了黑黢黢的队伍影子——真的是过大兵了!

顾大人不知道来者的目标是不是自己,可不管是不是,他都决定躲一躲。眼看从井台到院门还有一段距离,他不走大门,翻了院栅栏就往外跳。落地之后回头一瞧,他发现无心和月牙也跟上来了。

``你们也要跟着我走?''顾大人轻声发问:``我就去野地里躲一躲,等兵过了,我再回来!''

无心一手领着月牙,一手向前一指:``上山!''

话音落下,无心和月牙撒腿就跑。顾大人莫名其妙的追了上去:``躲也不用往山里躲啊。''

无心头也不回的答道:``人来了,鬼也来了!''

顾大人回头一瞧,就见后方不远处浮现出了白色影子,一张脸上描出木然的笑眼笑嘴,正是纸人!

顾大人一声没吭,转向前方一大步迈出去,差点扯了裤裆。

丁旅的士兵按照九姨太的指点偷袭而来,踹开院门之后没有找到任何活物,不过院子栅栏歪了一片,点了火把往地面一照,赫然现出凌乱脚印。领头的军官没犹豫,顺着脚印就往猪头山里追去了。

军队和无心等人之间的距离,至多不会超过一里地,中间还夹了一个忽隐忽现的纸人。无心一边飞奔,一边让顾大人加快速度上前带路。顾大人跑得耳边风声作响,气喘吁吁的问道:``往哪里带?''

无心攥紧了月牙的手:``鬼洞!''

顾大人立时带了哭腔:``操,自杀去啊?''

无心自顾自的继续说道:``到了鬼洞之后,你立刻带着月牙就近上树,无论下面发生了什么事情,都不要动。''

顾大人情急之下拐了弯,一路跑成了草上飞:``行,去就去!''

无心等人会跑,后方的追兵也同样会跑。顾大人还要用心认路,追兵却是一心追逐便可。双方距离越来越近,忽然破空起了一声枪响,无心把月牙拽到胸前用力一推:``顾大人,带她上树吧!''

顾大人眼看前方就是鬼洞,脊梁骨正要冒寒气,忽然得了这句话,如同得了大赦。而无心停下脚步,眼看月牙和顾大人手足并用的真爬上一棵老树了,才转身面向了来路。纸人脚下无根,飘然而至,伸手一把掐住了他的脖子。而他顺势抱住纸人,扭头就往鬼洞跑去。树上二人看得清楚,急得要死,又见不远处一条火龙蜿蜒而至,正是追兵举着火把赶上来了。

月牙高高的骑在一股枝杈上,盯着洞口望眼欲穿。无心刚刚拖着纸人跳下去了,现在洞口一片漆黑,一点动静都没有。顾大人握了手枪,蹲在月牙的斜后方,小声问道:``师父进去干什么去了?再不出来就让人堵进洞里了!''

月牙也是不明所以,正要让顾大人想个办法,不料洞内忽然光芒一闪,随即就见无心连滚带爬的上了地面,离弦之箭似的直奔老树而来。及至无心上了树,追兵们也到达了。

鬼洞经过挖掘,本就十分显眼,洞口附近方才又被无心踩踏了一番,新土和荒草都搅拌在了一起。军官围着洞口走了一圈,随即对着身后发号施令,把三名士兵派进了洞内。

顾大人立刻就明白了——原来无心是在借鬼杀人!纸人被他消灭在了洞内,魂魄流动之时少不得要惊动深处的鬼,无知士兵下入洞中,正是羊入狼口。

无言的对着无心一挑大拇指,顾大人算是佩服了他。然而无心坐在下方的树枝上,并不得意。

借鬼杀人也是杀人,而无心根本不想杀人,好人不想杀,坏人也不想杀。``无可奈何''四个字是总逃不脱的,和士兵相比,月牙和顾大人的性命更重要。为了保护月牙和顾大人,他只好出此下策。

看到纸人,就不由得要想起岳绮罗。无心知道岳绮罗一直处在暗中,自己的一举一动都逃不过她的眼睛。不知道纸人与士兵之间有没有关系,对于岳绮罗,他也同样的是无可奈何。

三名士兵入了洞,再也不见出来,于是军官又派下十个人。十个人提了五盏马灯,络绎下去弯着腰往里走。

一个小时过去了,人和马灯都不见了踪影。军官急了,派出二十人继续下洞。二十人手拉着手连成了队。最后一人腰间还绑了条长绳子,绳子一头就握在军官手里。

当最后一人进入斜洞之后,地下忽然起了隐隐的枪声。未等军官拉扯绳子,最后一人连滚带爬的出来了,身后跟着同样连滚带爬的几名弟兄:``有蛇!洞里有蛇!''

军官气的双手叉腰:``你们他妈的没见过蛇?''

几个人自作主张的爬上了地面,惊慌失措的告诉军官:``蛇从土里往外钻,钻\ldots{}\ldots{}''

未等他把话说完,洞里又弯腰逃出了一个:``有人,洞里有人!''

及至此人也爬上地面了,洞内传出一声哀嚎,震得人心一跳。军官举着火把照向洞口,随即大惊失色的后退了一步——一名士兵东倒西歪的冲了出来,半边身体血肉模糊!

``手!''半死的士兵抽搐成了一条垂死的虫子,已经无力爬上地面,只能扭曲着身体发出惨叫:``手!''

军官大喝一声:``什么手?说!''

士兵张大嘴巴,火光之中露出一口带血的乱牙,脸皮像被溶过了一样,五官糜烂没了形状,眼珠几乎突出了破损的眼眶:``手抓我们,手\ldots{}\ldots{}''

接下来就是无意义的狂叫了。军官一枪击毙了他,然后六神无主的环顾了四周。末了他一挥手,对着部下发号施令:``先撤,天亮再说!''

士兵是撤得一干二净了,空旷的猪头山上渐渐恢复安静。顾大人开了腔:``师父,有你的!够狠!''

无心轻声答道:``顾大人,太平日子结束了。趁着天没亮,我们赶紧下山往远跑吧!''

仿佛是要回应他的话似的,洞内幽幽的传出一声呜咽,含着泪泣着血。而刚被击毙的士兵缓缓起立,动作僵硬的爬上了地面。

\chapter{逃之夭夭}

月牙死死的抱住身边的大树枝,尽可能的不添乱。顾大人紧紧的握了枪,随时预备扣动扳机。无心蹲在下方的树杈上,眼看着死而复生的士兵越走越近。月色朦胧,月牙和顾大人眼力有限,只看出士兵像是被人扒过一层皮似的,扒得还不干净利索,血肉淋漓的拖一片挂一片;而无心的视野更清晰,瞧出士兵根本就是受了腐蚀,也许是半边身子都被鬼手抓进洞壁里去了,然而垂死挣扎的又逃了出来,可惜最后还是没能逃脱长官的一粒子弹。

士兵似乎是追着人味过来的,一步一步走得东摇西晃,仿佛已经无法调动自己的双腿。停在树下仰起了头,他抬起双手抱住树干,面目模糊而又狰狞。忽然慢慢张开了嘴,他作势要往树上爬,同时一张嘴越张越大,嘴角竟然渐渐裂到了耳根。

月牙强忍着不哆嗦,而顾大人咬了牙,对着无心说道:``师父,你躲一躲,让我一枪把他打下去!''

无心背对着顾大人抬起了一只手:``他已经死了,不怕你杀。有符没有?''

顾大人握着手枪拍拍身上,一时回答不出;而月牙颤巍巍的开了口:``有,有,顾大人,你掏棉袄里面的暗兜!你不是天天吵着要上山搬金子吗?我怕符丢了,全都给你缝进棉袄里了!''

顾大人在树杈上坐稳了,腾出一只手往怀里一摸,果然摸到一个暗兜。暗兜开口被粗枝大叶的缝了几针,伸手指头勾开棉线,他从里面取出了一卷子纸符:``找到了,用哪张?''

无心向上伸出了一只手:``全是镇鬼的符,随便给我一张就行!''

顾大人立刻弯腰递去一张纸符。而无心接住纸符,随即纵身向下一扑,竟是大头冲下的紧贴了树干,大蛇一般的爬了下去。迎头遇到向上的士兵,无心一掌击出,正把纸符拍上了对方眉心!

士兵立时僵住了动作,不上不下的附在了树上。而无心紧盯着他,心中却是同时敲起了鼓,因为不知道出尘子所画符咒是否真有效验。如果纸符无用,他自己琢磨着,恐怕就得下去和活死人打一仗了。

如此过了片刻,士兵开始有了反应。摇摇欲脱的下颚张到极致,他似乎要去撕咬无心一般猛然一窜,然而无心稳稳按住他的眉心,并不退却。他的表情越发凶恶痛苦了,体内像是开了锅,面孔开始此起彼伏的鼓凸又凹陷;身体沉重的向下滑去,一层黏腻的皮肤粘在了树干上。忽然鼓胀的眼珠发生了爆炸,一股脓血激射而出。无心当即歪头一躲,同时掌心加了力气:``人都死了,尸身都被你毁了,你还不放过他吗?''

静夜之中,无心声若洪钟:``有冤报冤,有仇报仇。躲在洞里嚎丧有意思?一次收了二十多条人命,识相的话就该躲进坛子里偷着乐,还敢驱使了死人装神弄鬼?信不信我给你撒一把大盐,把你腌了晒干当咸菜吃?''

骂到这里,无心抬手一掌击向士兵的天灵盖,把纸符直压进了士兵的血肉之中。士兵痉挛着继续向下滑落,最后跌坐在地,伏在老树根上不动了。

顾大人松了口气,把纸符和手枪全部揣好:``师父,完事了?''

无心也下了树,扯着士兵一侧还算洁净的衣领,把尸首拖去洞旁空地。划燃一根火柴扔上去,皮肤表层的黏血油脂立刻烧成一片。无心知道此人其实已然魂飞魄散,方才全是洞中一股怨气支配了他的身体,所以往生咒也没有念。围着洞口走了一圈,他忽然想道:``如果让岳绮罗和洞里的坛子打一架,不知道是谁胜谁负。''

然后他忽然笑了,感觉自己的想法很有趣。可惜岳绮罗并非大傻瓜,未必自己下了圈套,她就一定会钻。弯腰捡起一根枯树枝点了火,他猛然回身掷向暗处。一团烟火腾起又熄灭,一个纸人化为灰烬。无心不知道山上到底还存着多少纸人,他怀疑岳绮罗并不珍惜这些不值钱的部下,反正来得容易,要多少有多少。

闭上眼睛原地转了一圈,他没有再发现新的纸人。林中此刻很洁净,只有几缕零碎的魂魄在洞口徘徊游荡,微弱的不成气候。忽然困惑的一皱眉头,他弯腰跳进了洞里去。

等到无心爬上地面之时,月牙和顾大人全赶过来了——先前在树上,来不及阻拦无心下洞,两人全都吓坏了。此刻一人抓住了无心的一条手臂,月牙的牙齿刚要接触空气,顾大人已经出了声:``你下去作死啊?''

无心立刻答道:``我没往深处去,我就是看看。''

月牙问道:``看见啥了?''

无心摇了摇头:``没啥。''

顾大人向前迈出了一步:``没啥就走!刚才队伍里领头的小子我认识,就是丁大头的部下。猪头山不算大,丁大头多派点人就能把山围住。趁着天没亮,咱们赶紧往外跑!''

无心拽着月牙跟上了顾大人:``洞里的金子还要不要了?''

顾大人把脑袋摇成了拨浪鼓:``不要了不要了,真不要了!''

无心走出没多远,就发现领头的顾大人步伐凌乱,东一头西一头的没有方向。顾大人自己也奇怪,一步一步慢慢的走,结果走着走着一回头,发现自己还是走出了弧线。

``怎么回事?''顾大人有些心慌:``这不是要闹鬼打墙吗?''

无心拉着顾大人停下脚步:``怕是那个鬼洞今夜吃开了胃口,要把山上的活物都引过去!''

月牙有了主意,让顾大人把纸符拿出来,一人身上贴一张。顾大人嗤之以鼻,认为女人就是见识浅:``纸符是贴鬼的,贴在人身上有什么用?''

月牙不和他一般见识:``那你说怎么办?反正在我们老家,说是如果男的碰上鬼打墙,脱裤子撒一泡尿就好了。''

顾大人一推无心:``尿!''

无心当着月牙和顾大人,没什么忌讳可讲,一弯腰就把裤子脱了。然而两人眼睁睁的等了片刻,他连个屁都没挤出来。顾大人看他耽误事,急得揉了揉小肚子:``妈的,我也没尿。月牙,你有没有?''

月牙啐了他一口,随即又道:``除了撒尿,还有个法子。你俩谁嘴更野?一路骂着往前走,也能把鬼骂跑了!''

无心提起裤子,对着顾大人一抬下巴:``骂!''

顾大人清了清喉咙,当即开骂,中气十足的日娘捣老子,一边骂一边抬头看星星低头吐口水。无心跟在后方,发现他果然是走了直线。月牙对顾大人则是肃然起敬,心想十个老娘们儿围成一圈,恐怕也骂不过顾大人一个人。

三人一步一探的向前走,兴许是黎明将至,夜色越发浓重如墨。月牙什么都看不清了,无心也闭了眼睛。顾大人对于猪头山太熟悉了,则是看不看都无所谓。估摸着前方就是林子边缘了,顾大人越发骂得气吞山河,语言十分牙碜。无心和月牙在后面偷偷发笑,笑着笑着忽听顾大人``嘎''的一声,声音竟是戛然而止。随即无心脚面一痛,正是顾大人后退一步,踩了个正着。

``师父!''顾大人像是被人捏了脖子,嗓门都细了:``看,看,坛子!''

无心睁眼一看,就见前方树下果然摆了个半米来高的坛子。林中本来已经黑到伸手不见五指了,坛子本身却是微微的放了光亮,映出坛口一颗微微垂下的女人头。一把将顾大人扯到身后,他上前一步正视了坛子。

下一秒,他轻声开了口:``不要怕,只是幻象。我们要走出去了,她舍不得而已。''

然后他一手拽了月牙,一手拽着顾大人,大踏步的就向前走去。而在三人经过之后,无心又面向前方说了一句:``不要回头!''

月牙不是好奇惹事的人,不让回头就不回头;顾大人吓得脖子都硬了,想回头也回不过去。深一脚浅一脚的走了一气,三人一起出了林子上了山路。无心仰头望天,发现天边隐隐现出了光芒,是天将要亮的光景,便把顾大人又推到前方带路。

三人一路小跑着下了山,猪嘴镇是不敢回了,只能再往远逃。猪头山下是个小三国的格局,文县虽然归了丁旅长,附近的长安县可是另有大军头驻扎。三人且走且商议,最后无心和顾大人决定先去长安县避避风头;而月牙无条件的跟着无心,只是惦记着家里,以及被她埋在地下的几百大洋。

丁旅士兵把猪头山围了两天,四周的村镇也都搜查过了,末了一无所获铩羽而归。军官站在九姨太面前,惊恐万状的描述了鬼洞情形,顺带着推脱了自己的责任。

九姨太正在心不在焉的吃午饭,半长的头发挽成双丫髻,乍一看很像观音大士身边的童女。粉红嘴唇撅起来吐出一块小小的骨头,她的眼睛在齐刘海下闪闪发亮。人活得久了,经历得多,就不会大惊小怪。山上居然有一处吃人不吐骨头的鬼洞,听起来很可怕,但是也合理,可以有,有就有了。鬼洞其实不过是另一种形式的``煞'',吞入魂魄,增长力量。可是如果没有魂魄让它吞,它也就只好原地不动的喝西北风。岳绮罗对于鬼洞兴趣不大,她心里想的是无心。几辈子没和人相好过了,她难得能看上谁。

稳稳当当的坐在桌前,她用童稚的小嗓子下了命令:``继续找,活要见人,死要见尸。''

先前没有这句话,军官还不大敢对顾大人开枪;如今得了包票,军官心里有了底。对着九姨太打了个立正,他兴致勃勃的离去了。

岳绮罗缓缓的舔着嘴唇,坐着不动。无心不怕拼命,但是她怕。所以她决定暂且躲在丁旅长身后。顾大人不过是个武夫,不值一提;月牙年轻丰满,皮肉紧绷,倒仿佛是很好吃的样子;至于无心——她想无心的味道一定不好,因为只有快生快死的肉体才鲜嫩。

岳绮罗感觉自己活得不开心,所以要吃点好的,穿点好的,作为弥补。如果开心的话,她就不吃人了。

房门忽然开了,丁旅长像根柱子似的,步态笨拙的挪了进来:``绮罗,见到老七了吗?''

岳绮罗微笑着摇了摇头,丁家七姨太也不见了。

\chapter{他乡遇佳人}

长安县的新县长是位又革命又文明的人物,把在街上大小便的百姓全都抓进了牢里,另有无数蓬头垢面的乞丐,也被巡警驱逐到了阴暗角落。大街上一干净,长安县看起来就比文县高级了许多,加之火车源源不断的从天津卫运来摩登元素,长安县便是好上加好,繁华极了。

无心等人在一处中等规模的旅店里落了脚。旅店是一座又大又破的两进院落,房间里面什物俱全,臭虫之类也不缺少。无心在住进来的当天夜里,一根火柴烧了窗外一个纸人。烧过之后天下太平,三人连着过了几天安静日子,一切都好,就是手上的金钱有限,眼看就要交不出房钱吃不起饭了。

午夜时分,顾大人独自坐在床上抽烟卷。金子化为泡影,想要东山再起,就得赤手空拳重打天下。隔壁睡着无心和月牙,哼哼唧唧的总有动静,让顾大人的心思不时的从事业转到女色。喝酒图醉,娶老婆图睡,顾大人想起月牙那敦敦实实的两个大屁股蛋子,认为无心很有眼光,是个务实的人。

最后一根烟卷抽到头,顾大人脱了裤子。唉声叹气的撸了一场,他射了一地精华,糊住了一只过路的蟑螂。隔壁还哼唧着,顾大人系好裤子出了门,旅店前院的门房里有伙计彻夜值更,兼卖烟卷和拉皮条。顾大人看不上伙计手里的货色,所以只想过去买包香烟。然而刚刚走到前院,他遇上了一位前来投宿的女客。借着大门口的灯光,顾大人就见对方梳着溜光的发髻,打着稀疏的刘海,脸上搽得粉红粉白,模样不说多美,但也算得上端正,只是眉尖微蹙,有点受气包的意思。大半夜往旅店跑的女人家,必是有个缘故在里面,尤其她还一脸倒霉相,手里空空的连个包袱都没有。

顾大人怀疑她是从家里偷跑出来的小媳妇,也许是受了公婆的气,也许是挨了丈夫的打。伙计把她往院内客房里领,顾大人就直着眼睛呆站着瞧。女客临到进门之前,忽然楚楚可怜的扭头对他溜了一眼。顾大人有日子没和女人对眼了,登时心中一喜,身上一酥。

买下香烟之后,顾大人点燃烟卷叼在嘴角,心猿意马的在院子里溜达了一圈,然而连只老鼠都没有勾引出来。停下脚步清了清喉咙,他长叹一声,心中暗道:``真想和娘们儿睡上一觉啊!''

顾大人不好贸然去敲陌生女客的房门,只能是悻悻的回到房中安歇。翌日清晨,顾大人偷空对无心说道:``你夜里差不多就得了,别没完没了,吵得老子都睡不安稳!''

月牙出去买包子了,无心抱着膝盖坐在床上,很坦然的仰头去看顾大人:``羡慕我?''

顾大人很不屑的翻了个白眼:``羡慕个屁!你当老子没见过女人?老子当初妻妾成群\ldots{}\ldots{}''

他话未说完,无心插了嘴:``现在光棍一条。''

顾大人登时被他堵的没了话。幸而月牙捧着热包子回来了,顾大人把包子当成挡箭牌,接二连三的往嘴里扔,水都不喝一口,噎得直打嗝。

无心想要往远了走,比如坐火车去天津北平。顾大人倒是不介意去天津北平,问题是没钱买车票,而且从长安县到天津北平,火车必定经过文县,太不安全。一天的光阴转眼过去,三人还是没有正经主意,顾大人出门进门,眼睛溜着院内动静。昨夜登门的女客一直没露面,连顿客饭都没叫过。顾大人回忆起她对自己溜出的一眼,越想越有滋味,末了他把牙一咬,心说十个女人九个肯,就怕男人嘴不稳。反正她身边也没有汉子,今夜我便前去试上一试,如果真能成就了好事,将来我发达了,就纳她做六姨太。

到了天黑,顾大人食不甘味的吃了六个大馒头。干巴巴的咽下最后一口,他抬起了头,忽然发现无心正在对着自己发笑。

顾大人咂了咂嘴,把月牙面前的一碗热水端起来,仰头喝了几大口,然后问道:``笑个屁啊?''

无心笑而不语,从他手里接过大碗,喝光了余下热水。月牙倒了满满一碗水,自己一口没喝着。捏着半个馒头转向无心,她也跟着问道:``笑啥呢?''

无心垂下眼帘,低声说道:``我看顾大人面犯桃花,脸上红扑扑的,还挺好看。''

月牙忍不住看了顾大人一眼,见他是有点面红耳赤的意思,就忍不住笑了。顾大人心怀鬼胎,此刻被无心轻轻戳了一下肚皮,不禁有些心虚:``光棍一条,哪来的桃花!我是热水喝多了。''

无心抓了月牙的手拍了拍:``其实我不会看相,我也是胡说的。''

顾大人吓得鬼胎几乎流产,站起来往远了走,声音越来越小:``要是真有桃花倒好了\ldots{}\ldots{}''

顾大人回了房间,漱漱口又梳梳头。等到天彻底黑透了,隔壁房里的无心和月牙也睡下了,他脱了身上的棉袄,精精神神的推门进院逛了一圈,随即大模大样的走到女客门前,抬手就敲:``哎,你怎么就睡了?起来起来,要烟不要?''

片刻的静默过后,房门开了。女客站在门口,抬头望向了顾大人。

顾大人立刻做惊愕状:``哟!抱歉抱歉,我敲错门了。''随即他要退不退的咧嘴一笑:``是不是打扰你休息了?''

房内没有开灯,幸而前院亮着电灯,光芒很足,所以后院也是黑的有限。女客直直的望着顾大人,粉脸忽然扭曲了一下,仿佛本是预备着要笑,可临时强行把笑容收了回去。表情不稳定,眼神却稳定,依旧像昨夜一样哀哀切切:``小石头。''

顾大人听了她的呼唤,从假惊变成了真惊:``你\ldots{}\ldots{}你是谁啊?''

女客的两边嘴角失控似的翘了起来,眼睛里面没有笑意,面孔笑的可是很足:``我是\ldots{}\ldots{}小春子。''

顾大人恍然大悟的一拍巴掌:``哎呀,是你啊!''

小石头是顾大人的乳名,小春子是小石头的小邻居。两人分开的时候,都是十多岁的年纪,郎有情妾有意的,不过情意也不算很深,眉来眼去罢了。顾大人很高兴,开口就问:``你嫁谁了?怎么一个人出来住店?''

小春子抬手扶住门框,极力的把脸扭到一旁,语气急促:``我嫁给了丁大头\ldots{}\ldots{}你走、你走\ldots{}\ldots{}''

顾大人看她态度不对,反倒不肯离去:``你怎么了?''

小春子的眼睛亮了一瞬,随即身体晃了一下,把脸转回了前方:``我没事。''她的声音渐渐变的轻柔:``前些年听你名声天摇地动的,我也不敢去高攀。现在见了面,你还认不认我是妹妹呢?''

顾大人一听有戏,登时裤裆支了帐篷:``我能不认我妹子吗?你告诉我,你怎么一个人跑出来了?''

小春子一侧身,向着房内一甩喷香的手帕:``我和丁大头闹崩了,不跟他过了。''

顾大人顺势迈步就进去了:``丁大头现在可是正红火的人,我都不是他的对手,你舍得不要他?''

小春子关掩了房门,屋内立时变成一片黯淡:``我不过是个七姨太,熬到老也只是个妾,有什么舍不得的?''

顾大人馋女色都要馋疯了,又想小春子是个妇人,什么都懂,自己也就没有必要藏着掖着,浪费光阴。一转身走到小春子面前,他伸手就把对方的双手攥住了:``我说,你要是没有依靠的话,就跟着我得了。咱俩也算青梅竹马,你说我还能辜负你吗?''

小春子的手冰凉黏湿,任凭顾大人紧握。顾大人嗅到了很浓郁的脂粉气息,太香了,香的都有点恶心人。手指忽然合拢回握住了顾大人,小春子的声音奇异的喑哑了:``走,快走\ldots{}\ldots{}''

顾大人看她对自己好一阵歹一阵的,不禁哭笑不得:``我走什么走,长安县又不是丁大头的地盘,你还怕有人踢门不成?''

小春子的手指渐渐松开了,顾大人在阴暗之中依稀看清了她的笑容:``你说得对,我才不怕。''

顾大人搂着小春子亲了一个嘴,亲完之后感觉小春子有点口臭,就转而又去亲了她的脸蛋。脸蛋也带了一点怪异的腥味,于是顾大人不敢亲了,带着小春子往床边走。小春子柔顺的仰在了床上,顾大人弯腰去脱她的衣裳,她一动不动,任凭他脱。

屋子里黑,顾大人没心思再说甜言蜜语,解开腰带压了上去,他屏住呼吸瞪了眼睛,活龙似的兴风作浪,把一张木床摇得吱嘎作响。一口气顶了几千下,他酣畅淋漓的喘出了声音。在极度的快活中,他仰起头,从喉咙里长长的``啊''了一声,仿佛把几个月的存货一次全激射出去了。

闭着眼睛享受了片刻余韵,顾大人畅心快意的低下了头,忽见一只苍白的手从床下伸了上来,``啪''的一声将一张纸符拍上了小春子的面孔。

随即是无心的脑袋探进了顾大人的视野。对着顾大人微微一笑,无心轻声问道:``舒服够了没有?够了就下来吧!''

小春子瞪大眼睛僵在床上,喉咙里开始咕噜噜作响。一侧鼻孔忽然伸出两根摇摆长须,正是一只尸虫挣扎着爬了出来。

\chapter{香消玉殒}

顾大人双手撑在枕头两边,直勾勾的瞪着下方的小春子,没有``抽身而出'',是命根子自然软缩成了一条鼻涕虫,随着温热的液体滑了出来。一滴黏稠的汗递到了小春子的鼻尖上,汗是冷的,小春子的身体也是冷的。冷,而且松弛沉重。腐臭气味顺着她的七窍,渐渐飘散出来。

尸虫终于挣脱出了鼻孔,飞快的向下爬进了小春子敞开的领口。小春子的体内发生了沸腾,咕咕噜噜痉挛抽搐。纸符贴在她的眉心上,她向上望着顾大人,一双眼睛越努越出,同时喉咙中发出了混杂不清的两种声音。

一种是柔媚娇嫩的,悲悲切切的哭叫哀鸣,另一种是低沉嘶哑的,断断续续的说:``小石头,走,走,走\ldots{}\ldots{}''

更多的细长触须从她的嘴角鼻孔耳朵中伸了出来,摇摇摆摆一探一探。顾大人仿佛元神归窍一般,骤然翻身滚下床去。无心取而代之的从床下爬出来,一根手指点在纸符上面:``说,是谁让你来的?''

两种声音还在此起彼伏,一个声音虚弱而又绝望:``九姨太\ldots{}\ldots{}是魔鬼,小石头,你快走——''

话未说完,另一个声音忽然挑高盖过了她,哭得人遍体生寒。无心丝毫不为所动,继续逼问:``九姨太是谁?''

哭声之中,小春子挣扎着答道:``九姨太\ldots{}\ldots{}名叫绮罗\ldots{}\ldots{}会吃人\ldots{}\ldots{}''

话到此处,她忽然猛一仰头,细长脖颈瞬间凸起无数小点。一处皮肤最先被里面的尸虫顶破了,裂口之处流出黑水,随即从颈向下爆发一般,体内尸虫将皮肤顶成千疮百孔。黑色触角最先伸出,小春子喉中``荷荷''两声,顾大人站在地上,就见小春子露出的皮肤上遍布尸虫触角,竟如生出一层黑色长毛一般!一颗眼珠子忽然骨碌碌的滚落下去,一只乌黑硕大的尸虫摇头摆尾,从她的眼窝里拱了出来。

无心一手依然摁着纸符,另一只手送到嘴边咬破指尖,对着小春子的身体猛然一挥。血点子横洒而出,小春子的皮肤立刻被蚀出了深深孔洞。体内的尸虫仿佛受了滚水浇淋一般缩了回去,开始在体内穿梭翻滚。而无心一边用一根手指压制着体内尸虫汹涌的小春子,一边回头看了顾大人一眼。

``不要怕。''无心面孔苍白,声音冷静:``她爱你。''

顾大人哆嗦了一下,满头短发是明显的竖了起来。

片刻过后,小春子不动了,尸虫也安静了。无心揭下纸符揉成一团,然后拉过床头的被子,弯腰盖住了小春子的脸。

转身对着顾大人一挥手,他轻声说道:``她走了,我们也走吧,万一惊动了人,就麻烦了。''

顾大人像木雕泥塑一般,不能说也不能动,是被无心推回了客房里。

旅店的生意马马虎虎,前院客房住满了,后院却是清静。无心点了桌上油灯,然后拎着水壶走去前院,向伙计要了一壶热水回来。兑了温水拧了毛巾,他上前想给顾大人擦擦手脸,然而顾大人退了一步,低声问道:``你早就看出她的问题了?''

无心单手托着毛巾,小声答道:``我没看出她的问题,我看出了你的问题。记不记得我今天说过你面犯桃花?''

顾大人点了点头:``记得。''

无心笑了一下:``桃花不假,可惜你印堂发黑,犯的是一朵阴桃花!''

顾大人问道:``既然看出来了,怎么不早告诉我?''

无心反问:``你不是想女人吗?''

顾大人沉着脸上前一步:``我想的是女人,不是死人!你他妈的不是个人,可我是!无心,我把你当兄弟看,可是你把我当猴子耍!你躲在床底下看我干一个死人!''

无心看出顾大人要发怒了,便想做出一番解释:``我躲在床下,是为了保护你。''

顾大人抡圆了胳膊,对着无心的脑袋狠狠扇去:``你懂个屁!她是小春子啊!''

无心一歪头,轻轻巧巧的躲过了顾大人的大耳光。而顾大人随着惯性一晃,站稳之后带了哭腔:``无心,你个老不死的,你狗屁都不懂!我他妈的就是要憋死了,我也不能去干死人;我他妈的就是真干死人,也不能去干小春子!我小时候要是不搬家,小春子现在可能就是我老婆了!''

无心退了一步,认为顾大人实在无须如此痛心疾首,因为嫁给他做老婆也没什么好。随手放下毛巾,他将一盆温水端过来放到了顾大人面前:``你要不要洗一洗?''

悲愤的顾大人受了提醒,回想起自己方才的所作所为,他``哇''的一声吐了一地。

顾大人用肥皂洗脸洗手洗屁股,洗了一盆又一盆。月牙受了无心的嘱咐,躺在房里没出来,就听隔壁开门关门的很热闹。

良久过后,她被无心叫去了顾大人房内。顾大人坐在床上,满身都是粗肥皂的气味;月牙仔细端详他,感觉一晚上不见,他竟像瘦了一圈似的,一个脑袋缩在棉袄领口,脖子都没了。

天气寒冷,房内又没烧炉子,所以无心带着月牙也上了床,守着棉被还能温暖一点。无心倚靠床头坐了,月牙袖着双手偎在他的身边;无心对着床尾的顾大人一招手,顾大人像只大号孤雁一样,犹豫了一下,末了也挪过去了。

无心抬起双手,一边揽着月牙,一边揽着顾大人。两个人都知道了他的底细,然而还依旧和他好,所以他决心要保护他们,要让他们都活到老,活到发苍苍齿动摇。

无心没提顾大人日了鬼,只说他是受了勾引才进了小春子的客房,而在他进房之前,自己先人一步的开窗户潜了进去,把他从恶鬼手中营救出来。月牙听到此处,忍不住埋怨顾大人:``就跟几辈子没见过女人似的,也不仔细想想,天上连馅饼都不掉,能平白给你掉个婆娘?''

顾大人垂着眼皮,一声不吭,和月牙一样把手揣进棉袄袖子里。他不是个易动感情的人,几乎就是铜皮铁骨狼心狗肺,然而想起小春子一声接一声的``走'',他难过了。很用力的清了清喉咙,他极力的找话来说,不敢深想:``怪不得丁大头不抓张小毛子专抓我呢,原来是有人给他吹了枕头风。''

无心对月牙解释道:``岳绮罗嫁给了丁大头做九姨太。她控制了七姨太——就是小春子的魂魄,让她成为行尸走肉追来长安县。''然后他转向顾大人又道:``活人的三魂七魄和身体附得很紧,不是轻易就能全被收走的。小春子的体内既有残余魂魄,又被岳绮罗另找冤魂附了上。冤魂戾气很重,本是占了上风;然而小春子大概是一直对你存了一缕牵念,所以相见之后,她竟是暂时镇住了冤魂,想要救你。''

顾大人吸了吸鼻子:``嗯。''

无心安抚似的拍了拍他的后背:``岳绮罗施在小春子身上的法术,已经被纸符破了。小春子魂飞魄散,从此世上再没有她。你放心,她不痛苦了。''

月牙叹了口气:``姓岳的怎么还没完了?一开始是拿纸人吓唬我们,现在可好,改派死人上阵了。善恶到头终有报,就没人能收拾她?''

无心想了一想:``控制魂魄,凭的是念力。纸人一旦远离了她,恐怕也就不会太听话,而且一个火星弹出去,就能把它烧光。换了尸首就不一样了,骨肉和纸毕竟不同,只是时间久了,免不了要腐烂。''

顾大人失魂落魄的答道:``原来鬼上身也不容易,怪不得都要修炼成煞。''

月牙表示赞同:``对呗,还是自己的东西用着顺手。''

无心拍着左右二人,慢慢的又道:``岳绮罗也许是得知了小春子和顾大人的渊源,所以才派了她来长安县。小春子连连的让顾大人走,可见她来意不善,是要伤害顾大人。而凭着岳绮罗的本领,没有必要和丁大头合作\ldots{}\ldots{}''

无心没再说下去,心想岳绮罗先前袭击过月牙,现在又袭击顾大人,显见是要让自己变成孤家寡人。其实变成孤家寡人也没什么,只是月牙已经和自己成了亲,离开自己也不好再嫁;顾大人又是个光杆司令,想当土匪都无山可上。

所以他不能让步,他对岳绮罗让了步,就对不起了月牙和顾大人。况且只有千年做贼的,没有千年防贼的,他想和月牙好好过上几十年的日子,不想天天提心吊胆。

最后,无心开了口:``天亮之后,我送你们去个安全地方。''

月牙和顾大人一起莫名其妙:``去哪儿?''

无心答道:``青云观。''

隔着中间的无心,月牙和顾大人大眼瞪小眼:``去青云观?人家能让咱们白住吗?''

无心很亲昵的和月牙贴了贴脸:``我有办法。等到安顿你们住下之后,我要去趟文县。放心,不会久,两三天就回来。''

\chapter{夜探}

天亮之后,无心付清房钱,坦坦然然的带着月牙和顾大人离开旅店。月牙倒也罢了,顾大人一步三回头,不住去望小春子的房门。后院已经隐隐弥漫开了尸臭,不过前院正有一辆收夜香的大粪车经过,大粪车顶风臭出十里地,伙计捏着鼻子皱着眉毛,也就彻底忽略了自家的异味。

无心一拽顾大人的袖子,不让他东张西望,免得惹人注意。离开旅店数了数钱,月牙走去买了十个菜包子,菜包子全有拳头大,顾大人吃了五个,月牙吃了三个,无心吃了一个半——他见月牙吃得舔嘴咂舌,仿佛是意犹未尽,就把剩下半个也给了她。

``我不怕饿。''他告诉月牙:``不吃也是一样的有力气。''

月牙不信,也不要。两人推推让让,结果一个失手,半个包子落在了地上。顾大人旁观至此,发出感慨:``妈了个蛋,不如给我!''

月牙和顾大人很想知道无心要去哪里,可是无心一路死活不说。三人出城上了山路,大半天后到达了青云山上的青云观。月牙虽然迁来直隶住了许久,可是最远只逛过文县附近山上的大庙。大庙已经算是金碧辉煌,庙里的和尚也都肥头大耳,十分富态;不料和青云观一比,她虽是没什么学问,可也觉出了大庙的俗。刚一经过牌楼,她就不由自主的扯了扯衣袖摸了摸头发,又特地用手背抹了抹嘴,想要做出庄重模样;顾大人一个脑袋也是四面八方的转:``哎哟,洞天福地啊!我先前怎么就没来过?''

无心踏着青石板路拾级而上,又微微侧身牵着月牙的手。深秋了,两边山中一派萧瑟风光,干燥的寒风穿林而过,吹得枯叶沙沙作响。一道小小山涧顺山而下,流出一点似有似无的水声。无心仰头向上望去,就见层林之中隐约显出雕梁画栋,正是山门之后的玉皇殿。

出尘子道长似乎是万万没想到无心还会再来。披着一件貂皮领子的黑大氅,他伸腿下了他的红木大罗汉床,大氅敞开来,露出里面一尘不染的雪白裤褂。

无心对他是相当的恭敬,拱手抱拳一鞠躬:``道长,我又来了。''

出尘子一头长发中分披下,黑亮的像一匹好缎子。眯着眼睛上下打量了无心,他眼角的鱼尾纹全藏在了长发下面,中间露出的面孔显得异常白嫩年轻:``你怎么又来了?''

无心挺直了腰,仿佛含羞带愧似的,对着出尘子低头一笑:``还不是因为你太师叔公——''

未等他把话说完,出尘子气得一晃脑袋,眼角眉梢全露了出来:``放狗屁!我哪有什么太师叔公?我太师叔公早在一百多年前就死过好几次了!''

无心笑微微的心平气和:``道长,你别急,听我把话说完。你太师叔公啊,在文县嫁人做九姨太了。''

出尘子后退一步,抬手一拍罗汉床上的小炕桌,怒发冲冠的叫道:``再说就给我滚出去!''

无心点了点头:``好,我到外面说去。''

出尘子龙行虎步的杀向前方,一把揪住了无心的衣领:``敢?!''

无心慢条斯理的抬起双手,轻轻一拍出尘子的肩膀,同时低声说道:``道长,你太师叔祖玩死人,玩得漂亮极了。''

出尘子瞪着他,不说话。

无心继续说了下去:``由着她玩下去,将来必出大乱,所以我要去趟文县,再看一看你太师祖的阵法。看见窗外站着的一男一女了吗?女人是我老婆,男人是我兄弟,我不能带着他们去文县冒险,所以想请你收留他们几日。我想凭你的道行,青云观里总不会闹鬼。''

出尘子松了手,一甩袖子背对了他:``闹鬼又当如何?''

无心绕到了他的面前:``修道的人,总是慈悲为怀,两条人命,我想你一定能护得住。''

出尘子抬眼看他:``你到底是什么人?''

无心双手合什:``道长,拜托了,你一天给他们三顿饭吃就行。''

出尘子一见到无心,就像落进了云里雾里,上不着天下不着地的悬起了心。太师叔祖是青云观内的秘密,他只把秘密传给了他的大弟子,因为将来待他羽化之后,大弟子就会是新一代的道观住持。秘密本来类似一个玄之又玄的故事,有趣而已,一文不值;可是当无心带来太师叔祖的消息之后,故事和现实衔接起来,就让出尘子隔三差五的做起了噩梦。

出尘子在青云观后找了两间小房,让月牙和顾大人住下。月牙和顾大人见识了道长飘飘欲仙的派头,都很景仰,老老实实的不敢妄言妄动。及至到了晚上,无心坐在出尘子的罗汉床上,细细讲述了岳绮罗的恶行。出尘子捧着一只古色古香的小手炉,听得脸上神色不定。而无心说到最后,隔着炕桌向他探过头去:``你的本事和岳绮罗相比,能差多少?''

出尘子听他终于收了``太师叔祖''四个字,不由得松了口气:``我太师祖和她不是一路,我们不能比。''

无心又问:``岳绮罗能把地下的魂魄召唤上来,你能吗?''

出尘子摇了摇头:``我只能把地上的魂魄镇压下去。''

无心恍然大悟的点头:``哦\ldots{}\ldots{}也不错,比我强。''

无心一夜没睡,因为回房之后对着月牙实话实说,承认自己是要去趟文县。

月牙当即表示不同意,又劝不服他,便跃跃欲试的想要撒泼。坐在床上扯散发髻,她想哭,没哭出来,于是下床去找了顾大人。顾大人披着棉袄进了房门,摩拳擦掌的放出豪言,说要打断无心的腿。无心抬脚踩上床沿,自己``啪''的一拍大腿:``来,打吧!''

月牙和顾大人刚柔并济的合了作,硬是没治住一个无心。午夜时分无心出发下山,月牙和顾大人跟在后方送出老远。月牙气得哭唧唧:``啥玩意儿啊,油盐不进的,驴脾气啊!''

顾大人跟着帮腔:``就是头驴!''

月牙又道:``我们跟你去吧,人多总比人少强啊!''

顾大人舔了舔嘴唇,没搭腔,因为真是不敢去文县,怕岳绮罗,也怕丁大头。

无心停下脚步,转身对着月牙嘿嘿一笑,又抬起右手微微一摇,做了个告别的手势。不等月牙再开口,他转向前方加快脚步,连跑带跳的消失在了夜色之中。

无心成了无牵无挂的一个人,行动起来反倒更利落。脚步不停的走到天亮,他进了长安县外的一家小饭馆里吃早饭,就听邻桌食客讲述县内大事——一家旅店夜里来了个女客,入住之后不吃不喝没动静,结果两天之后伙计忍不住去敲了门,没人答应;踹开门一瞧,女客早烂在床上了!

``死个女人不算太稀奇。''食客绘声绘色的讲述:``稀奇的是验过尸后,发现女客至少已经死了十天半个月——怪了吧?女客可是两天前自己过来的。''

馆子里面一片惊声。无心会了账,起身悄悄走了。

如此又走了大半天,无心经过了猪嘴镇,直奔文县城门。近来文县太平,城门从早到晚大敞四开。无心轻而易举的进了县城,混在人群里走向顾宅。

暮色之中,顾宅所在的一条胡同寂静无声,枯藤老树昏鸦俱全。无心慢慢的进了胡同,就感觉两边房屋全都没有人气。先前顾宅闹了几个月的鬼,也只是吓得左邻右舍搬走;如今顾宅不闹鬼也不闹人了,怎么反倒变得越发荒凉?

无心在两扇紧闭的黑漆大门前停了脚步。大门外面挂着黄铜大锁,锁上缀着点点斑斑的泥水痕迹,似乎已然经过了不少风雨。锁门是正常的,无心本来也没想过走大门。出了胡同绕到后方,无心决定爬墙进去。记得顾大人曾说宅子后面带有花园,无心现在对于顾宅的一切都很感兴趣。

花园的围墙不算高,无心赶在太阳落山之时翻了进去,落脚之处一片柔软,是荒草和落叶积了厚厚的一层。花木久不修剪,全都长得张牙舞爪,阴暗处不时发出窸窸窣窣的声响,是小活物受了惊动。一阵夜风而过,卷起漫天落叶。

无心经过几丛刺玫瑰,发现园子里不大干净。人不来,鬼就来了。

石子小径都被落叶覆盖了住,无心一路辨认着往前走。顺顺利利的到了园子门口,他抬头望去,却是停住了脚步。

院子门口摆着一具小小的棺材,木质漆黑,似乎里面只能容下幼童。

\chapter{偈语}

大凡一个人活着的时候阳气弱,死后必定阴气盛,所以无心站在棺材前方,一时之间不敢妄动。从尺寸来看,棺材显然是为孩童订制的。小鬼阴气重、执念轻,最易控制摆布;而棺材本身并不陈旧,可见它也是被人新近放到此处。

穿过棺材后方的大月亮门,向前再走几步拐一道弯,就能进入顾宅后院了。棺材挡门,乃是个阻拦的势子,拦的是谁,却不好说。无心想如今文县成了丁大头的地盘,而丁大头似乎也已经落入了岳绮罗的手中。岳绮罗在文县说一不二,满可以把整座顾宅划为禁区,何必还要在宅内多做手脚?如此看来,就不是拦,而是封闭。

要封闭的,自然就是棺材后方的区域。无心仔仔细细的观察了棺材,心想岳绮罗大概是依然顾忌着院中的水井,所以不许外人轻易靠近。在地下活活躺了一百多年,水井就算是她的重生之地了。

轻手轻脚的绕过棺材,无心迈步跨过了月亮门,同时后悔自己没有带几张纸符过来。纸符全在顾大人的棉袄暗兜里,竟然真有法力,可见出尘子并非浪得虚名。

然而未等走出几步,前方忽然响起了一串沉滞的脚步声音。无心向前一望,就见一个红衣小男孩跌跌撞撞的跑了出来。见到无心之后,小男孩停了脚步,不言不动。

无心继续前行,走到近前一瞧,就见小男孩脸色青灰,眼眶嘴角已经隐隐腐烂,原来不是活人,而是一具童尸。

一大一小对视片刻,小男孩忽然抬起一只小手,作势要抓无心的裤管:``大哥哥,你带我玩。''

无心低下头,就见小男孩的小手上皮肉破损,指骨关节全都白生生的露了出来,头上短发也是蓬乱。无心伸手拨开他的头发,就见他头顶心处孔洞赫然,是活着的时候被人钻开头骨注入了滚油。惨死的幼童,又经过了岳绮罗的炮制,阴气戾气全都重到极致,无心想他大概把自己误认成了他的同类,因为自己身上没有活人气。

无心把手指探入孔洞之中,勾着小男孩的头骨向上提。幼童身轻,被他直提向上。而他看着幼童的眼睛,开口问道:``是谁杀了你?''

小男孩乖乖的答道:``姐姐。''

无心又问:``饿不饿?''

小男孩不能点头,只很勉强的眨了眨眼睛,眨下了几根带着烂肉的睫毛:``饿。''

无心弯腰放下了他,就见小男孩站稳之后,猛然歪身一扑,捉住了墙角路过的一只大老鼠。把老鼠头塞进嘴里狠咬一口,小男孩吮奶似的开始吸血。

无心明白了——小鬼是扑着阳气去的,有活老鼠,杀活老鼠;有活人,就杀活人。

微微弯下腰去,无心问道:``你睡在哪里?''

小男孩把嘴张到极致,一侧嘴角撕裂开来。大老鼠的半个身子都被他吞入口中,一条细长尾巴抽搐着摇动不止。抬手一指月亮门外的小棺材,他已经腾不出嘴来说话。

无心点了点头。等到小鬼吸尽老鼠鲜血之后,他抬手咬破指尖,然后把手指伸向了小鬼。小鬼见了他指尖一点血红,立时张开血盆大口去吮。然而合拢嘴唇刚刚一嘬,小鬼立时有了反应——他的五内融化一般沸腾起来,七窍一起向外流出了脓血。

无心抽出手指,踢开小鬼继续前行。走过几步之后他忽然折返回来,拎着小鬼走出了月亮门。撕下小鬼身上的红衣裳,他就近找了一棵树,撕扯衣裳结成绳子,把小鬼绑在了树干上。他的鲜血正在腐蚀小鬼的皮囊,而等到黎明时分阳气上升,阳光自然会让小鬼魂飞魄散。

转身把小棺材也推开了盖子,无心伸手进去摸了一圈,没摸到什么,于是重新走进月亮门里去了。

无心进了顾宅后院,就见院内地上血迹斑斑,而通往前院的院门口赫然也横了一副小棺材。无心侧耳倾听,发现棺材里面传出了细微声响,仿佛有人在里面翻身。太阳刚刚下山,大概后门的小鬼先跑出来,前门的小鬼却是个慢性子。镇守后门的是个小男孩,按理来讲,前门值更的就该是个小女孩。对着小棺材迟疑了一下,无心忽然起了怀疑。太师祖善用阵法,太师叔祖也不该弱。小黑棺材摆得前一副后一副,会不会也是一种阵法?如果阵法被人破了,设阵之人是否会有知觉?

思及至此,无心没有过去惊动棺材。小鬼伤不了他,至多是给他捣乱,而且只能在夜间出没,天一亮就要躲回棺材里去。无心自认为可以在井中泡上一夜,横竖顾宅空荡,天亮后再上来也没关系。

转身走到院角井口,无心低头向内一瞧,发现井中的明月十分的近,却是井水涨了许多。就近在井边捡了一根结实的枯枝,他把身上的袄裤尽数脱掉,用腰带紧紧的系成了一个小衣裳卷。脖子上还挂着一只扁扁的小荷包,里面则是出尘子道长画出的黄符。

前方小黑棺材里的动静越发激烈了,棺材盖吱吱嘎嘎的出了声音,显见是里面的东西将要出来。无心抱着枯枝和衣裳踏上井台,不再迟疑,向下一跃落入井中。

双脚刚刚没入水中时,他奋力蹬住井壁止住了下落之势。抬手摸上青苔厚重的井壁,井壁也是用砖砌了的,年久失修,已经不甚平整。无心把枯枝狠狠插\textbar{}进一处砖缝中去,露出半截正好成了个木橛子。把衣裳包挂上去,把小荷包摘下来也挂上去,无心双手空空一身轻松,并拢双腿沉入水中。

井水很凉,无心入水之时连打了几个冷战。转着圈向下降到井底,他镇定了片刻,然后游向了坍塌石壁。大鱼似的越过石壁,他进入了密室。

石壁一破,密室自然也就谈不上密了。水中一片漆黑,无心缓缓游动,同时渐渐看清了室内情景。腥红棺材依然摆在正中央,棺材盖也依然是滑脱向后,铁链松松的捆着棺材,完全是个意思而已。井水随着他的游动而流,带的几张黄符上下沉浮。无心随手抓住一张仔细看了,发现符上图案都是相同的。

然后,他抬眼望向了三面墙壁。灰白墙壁上面符咒乌黑,无头无尾无始无终。他靠近过去细细的观察记忆,想要把它印在脑海里。对他来讲,符咒犹如天书一般,哪是容易记得住的?看着看着,他有些后悔,悔不该当初有什么忘什么。他是喜欢遗忘的,遗忘了,就可以重新再去认识一遍。道术之流他肯定是学过,两百年前或者三百年前;可是自从遇上玉儿之后,他就关了大门吃老本,一笔资产让他和她吃了几十年。玉儿死后,他钱也没了,本领也没了。

无心沿着墙壁缓缓游动,手指抚摸着黑色笔画,一点一点的记忆。其实整座密室便是一张大符,把岳绮罗彻底的封闭起来。可是石壁破碎了一面,大符就只剩下了四分之三。

四分之三,聊胜于无。无心不知道自己沿着密室转了多少圈。最后他抬手一推墙壁,伸展四肢浮在水中。闭上眼睛冥想片刻,他确定自己是把符咒图案尽数记牢了,才轻松的吁了口气。

他没有气,只从鼻孔里吁出了两道微弱水流。一个猛子向下扎去,他突发奇想,想要再研究研究正中央的棺材。

牵牵扯扯的拽下铁链,他仰面朝天的躺进棺材。后脑勺枕上沉重的玉石枕头,他伸出赤脚向上勾动棺盖,把自己封进了棺材里面。

棺盖严丝合缝的压了上来,无心在彻底的黑暗中抬起双手,心想岳绮罗就是这样躺了一百年。什么滋味,不能细想,因为一百年的黑暗寂寞孤独太可怕。

指尖忽然有了凹凸不平的触感,是左右两行深刻的字迹。无心轻轻摸索辨认,发现那是一句佛家偈语:``千江有水千江月,万里无云万里天''。

偈语写成对联的格式,两句中间夹着几笔潦草的图画。波浪线是水波纹,水上浮着一只潦草的鸭子——大概是鸭子。无心摸了又摸,始终不能确定,因为画得太简略了,也可能是鹅或者雁。

岳绮罗躺在棺材里面,应该不会有闲情逸致写写画画。无心笑了一下,心想这大概是太师祖的遗迹。太师祖怕太师叔祖躺在棺材里太无聊呢!

一对师兄弟,道不同就要斗,斗了就要分胜败。好不容易分出胜败了,败者痛苦,胜者也不舒服。没办法,无心想,几百年几千年,一直如此。

无心在井里翻江倒海,忘了时间。而文县丁宅内的岳绮罗,也是彻夜未眠。

最新式的留声机鸣唱一宿,几张片子翻来覆去的听。小小的她坐在大大的沙发椅里,两条腿垂下去,踩在一张古色古香的小脚踏上面。她的刘海长了,乌黑厚重的盖住了眉毛,黑压压的头发下面,一双眼睛皂白分明。用一把折扇轻轻打着手心,她盯着前方案上的两盏长明灯。

案面画了太极图,长明灯就位于阴阳鱼的鱼眼之处。两盏灯,其中一盏火苗闪烁。夜色浓重,黎明将至;火苗忽然暴跳起来,随即骤然熄灭。

岳绮罗站起了身,扔了扇子走出门去。门外两边站着卫士,就听她头也不回的说道:``备车,我要出门!''

\chapter{她的爱}

无心想要赶在黎明之时离开水井。黎明时分虽然天黑,然而阳气上升,逼得小鬼不能兴妖作怪。鬼不出来了,天寒地冻一片黑,人也不出来,可以随着他翻墙头满街走。如果时间不敷使用,无法赶在黎明之前爬上地面,那也没关系,大不了跑进花园子里等天黑。园子里很荒凉,即便到了白天,想必也是人鬼不至。

他盘算的很好,可是井下密室中没有月亮没有星星,他全神贯注的光顾着记忆符咒,也就忘记了时间的流逝。待到把棺材也翻过一遍了,他才忽然想起时间有限,不能由着自己翻江倒海的流连。游出密室来到井底,他仰头向上一望,不由得叫苦不迭——天都亮成青白色了!

双脚一蹬井底,他借力向上升去,一个脑袋``哗啦''一声露出水面了,随即传入耳中的,却是一阵金石摩擦之声。他立刻仰头向上望去,就见井上空中伸出四双手,把一只沉重的大铁罩扣上了井口!

铁罩是由铁条纵横交错焊成的,乍一看几乎像只无底的笼子,严丝合缝的覆下来,竟然连四四方方的井台也一起罩了住。无心知道坏了事,手足并用的撑着井壁向上爬,没有爬出多远,他的脑袋就见了天日。

四名士兵正要抬大条石压住铁罩落地的四边,冷不防井口忽然探出了一个水淋淋的脑袋,不禁都吓了一跳。吓归吓,当着九姨太的面,没一个人敢出声。而岳绮罗端端正正的站在井台前方,双手笼进袖子里,周身上下都是一丝不动,唯有一头厚重乌黑的头发随着冷风轻轻飘拂。

铁罩能比井口高出一个人头。无心双手抓住铁条,可以清楚的仰视岳绮罗。双方无言的对视片刻,天空越发明亮了,士兵也把条石安放好了。安放好后他们站到四角,恪守卫士职责,端着步枪注目井口。

岳绮罗微微一笑,细声细气的说道:``大哥,自投罗网啊!''

无心也开了口,声音有点嘶哑:``千江有水千江月,万里无云万里天。''

岳绮罗一眨眼睛,八风不动:``换一句吧。读了一百年,早读厌了!''

无心凝视着她的眼睛,看清了她右眼中的红点:``才一百年,就读厌了?''

岳绮罗向前走了两步,姿态与模样都是个小妹妹,要长成未长成,嫩的带了稚气:``你读了几百年?''

无心摇了摇头:``我不记得。''

晨风扬起岳绮罗的刘海,露出额头如玉:``不记得?难道开天辟地时就有了你?''

无心继续摇头:``我不记得。''

岳绮罗抬脚迈上铁罩,慢慢走到了无心上方蹲下。指尖一划无心的手指,她饶有兴味的低头看他:``来干什么?想找法子来对付我?''

无心仰起了脸:``我没找到。''

岳绮罗伸下一根手指,轻轻戳上无心的眉心:``你没找到法子,我却是找到了你。''

无心抬起双脚蹬着井壁,将身体赤条条的晾在了阳光下寒风中:``我不爱你。''

岳绮罗审视着无心的裸体,``嗤''的一声笑了出来:``日久生情。''

无心歪着脑袋看她:``日久生情?可我都不知道你是男是女。''

岳绮罗一屁股坐下去,银铃似的笑了一串,笑过之后她低头问无心:``要不要我脱了衣服验明正身?''

无心松开双手抱住膝盖,``扑通''一声沉入水中。

岳绮罗一怔,随即四脚着地跪趴在铁罩上,用小鸟的嗓音对着下方怒道:``什么意思?''

无心落入水中,感觉井水倒比空气更温暖些。沉到井底游进密室,他躺到棺材里,想不出逃生的方法。好在月牙和顾大人都有了着落,而且知道他不会死,多等一阵子大概也不会太着急。

过了不久,他依稀听到井口的铁罩被铿铿锵锵的敲响了。出了棺材浮出水面,他又看到了岳绮罗。

岳绮罗蹲在铁罩上面,面前放了一只大海碗。当着无心的面,她将一纸包白色粉末倒进了碗中。碗内满满盛着鲜肉,她用手指一边搅拌鲜肉粉末,一边对着无心问道:``你饿不饿?''

无心一跃而上,双手抓住了铁条:``我不吃人肉!''

岳绮罗的小手冻成通红:``不是人肉,是牛肉。''

然后她望向了无心:``加了砒霜,吃不吃?''

无心抬头张开了嘴,嘴唇棱角分明,牙齿很白,舌头很红。岳绮罗将一条牛肉拈起来喂给了他,他仿佛是饿了,嚼都不嚼,一伸脖子便咽了下去。咽下之后他仰起脸,又嗷嗷待哺似的张大了嘴。

隔着纵横铁条,岳绮罗把牛肉一条一条的扔进他的嘴里。待到扔空了一只大海碗后,她自己捻了捻手指:``没了。''

无心说道:``中午我想吃熟的。''

岳绮罗用两根手指摸了摸他的短头发,不知道怎样才能把他驯服,对于没有魂魄的活物,她真是束手无策。无心任她摸着,也并无和她硬碰硬的打算。

岳绮罗中午喂给了他许多油煎小虾,晚上则是把葱油饼撕成一块一块的往他嘴里送。无心吃过两张葱油饼后,问岳绮罗:``你要把我关到什么时候?''

院内的卫兵撤出去了,岳绮罗低头注视着他:``日久生情,所以要关得久一点。''

无心抬脚蹬着井壁,悬在井中轻轻的摇晃:``我已经对你生出感情了。''

岳绮罗一拍油腻腻的双手,仿佛是很欢喜。不料无心随即又道:``但在你长大之前,我是不会日你的。''

岳绮罗登时嗤之以鼻的哼了一声,随即像个半大丫头扑蚂蚱似的,跪趴下来凑近了无心。粉红色的薄嘴唇一张一合,她老气横秋的压低了小嗓门:``论做人,我男人做过女人也做过;论道行,我正道通晓邪道也通晓。凭我的身份和境界,会是贪图床笫之欢的人吗?笑话!''

无心不以为然的答道:``你的身份,无非就是个半人半妖的九姨太;你的境界,无非就是不择手段想要长生不死。我告诉你,我不说冰清玉洁,也算三贞九烈,说不日,就不日。但是你如果肯放我出去,我可以和你交个朋友。将来你老而不死,叫天天不应叫地地不灵的时候,可以来找我发发牢骚。''

岳绮罗还趴在铁罩上,拧着两道浓淡相宜的眉毛瞪无心:``你不想多问一问我的来历吗?''

无心有些累了,双手虽然还抓着铁条,可是身体开始慢慢的向下坠:``我无所不知,不必问了。''

然后他手指一松,想要回到水中,不料下落之时一屁股硌上了井壁突出的木头橛子。橛子上挂着的衣裳卷儿和小荷包都安然无恙,倒是无心发出一声惨叫,没有叫完就沉到井底去了。

无心被狠狠的硌了卵蛋,苦不堪言的捂了下身,在井底连打了几个滚,搅出了一个大漩涡。岳绮罗乐不可支的哈哈大笑,奶娃娃似的叽叽嘎嘎。

午夜时分,无心听得井上宁静了,便摇头摆尾的浮上水面,攀着井壁爬向上方。可是没爬多高,他便看到一个红衣小丫头站上铁罩,面无表情的低头看自己。

小丫头很丑,无心估量着她的前程,认为她即便不死,将来婚姻也成问题。忽然对着无心一咧嘴,她龇出满口油光水滑的黑牙,牙齿尖利,涎水滴滴答答的反射着月光。嘴很大,眼睛却小,眼梢斜吊着,瞳孔里除了凶光再无其它。

无心不理会,继续向上爬。爬到井口伸出头去,他环顾四周,发现士兵早没了,换了几个眉开眼笑的纸人值更。

咬破手指向着小丫头晃了晃,无心故意去逗对方。而小鬼嗜血,果然跪下来张嘴就咬。一口咬上指头粗的铁条,小鬼盯着一点鲜红不肯松口。而无心没有伤害她,单是饶有耐性的晃着手指,引得小鬼一口接一口的追逐啃咬。

咬到最后,小鬼无所收获,被一只活蹦乱跳的大老鼠吸引了走。无心腾出手来去摸铁罩,发现凭着小鬼的牙口,如果肯专心致志的咬上一夜,大概也能咬断一根铁条。可是自己鲜血有限,活气更是没有,勾引小鬼实在太难;井里也是可恨,不但没有鱼,甚至连条蚂蝗都不长。

翌日上午,岳绮罗又来了,挑了面条去喂无心。面条很热,烫得无心脸都红了。岳绮罗察觉到无心一直在观察自己,就沾沾自喜的问道:``看什么?''

无心答道:``你是个很漂亮的小姑娘,看不出你上几辈子做过男人。''

岳绮罗托着大碗,对他嘻嘻一笑:``投胎投胎,投的时候,看不见胎。投上了,出生了,才知道自己会有怎样的皮囊。皮囊不重要,灵魂才重要。''

无心点了点头:``可是我没有灵魂。''

岳绮罗用筷子搅着碗底面条,心想无心有着不灭的肉体,自己有着不灭的灵魂。如果自己的灵魂控制了无心的肉体,结果该有多美妙?

只是爱上肉体,算不算爱?应该也算。岳绮罗眯起眼睛,侧过脸去望白日青天,心想自己几辈子没有爱过人,如今又爱了。

\chapter{道不同}

无心一头扎进井水里,偷偷吐出口中一尾活泼泼的小鱼。一转身浮上去,他很灵活的攀爬向上,水淋淋的双手举起来,重新抓住了结实的铁条。

岳绮罗站在井台前方,系着黑底白梅花的缎子面长披风,一张小脸被狐皮领子团团的托出来,刘海剪短了,露出两道清清楚楚的眉毛。单手托着一只白中透青的瓷碗,她很满意的注视着无心,同时从瓷碗里捏起一尾摇头摆尾的小活鱼,对着铁罩轻巧掷去。无心张嘴去接,接了个空。小鱼擦着他的面颊滑入井中,无心哈哈笑了,对她大声说话:``再来,再来!''

岳绮罗看着他阴沉沉的白皮肤与黑幽幽的眉眼,觉得他很俊美。初冬的细雪飘落下来,无心已经在井中生活了三天,身体没有被冻僵,皮肤也没有被泡皱。岳绮罗爱死了他的身体,不能得到,相伴也好。

将碗中最后一条小鱼扔向前方,无心猛一仰头,用牙齿咬住了银白小鱼。随即低头嘬起嘴唇轻轻一吸,小鱼瞬间被他吞了下去。双手同时松开,他向下又一次坠入井中。

雪越下越大了,无心不肯再吃生食,要热菜热饭。吃饱喝足之后,他照例悬在铁罩下面,对着外面说道:``我爱你,放我出去吧,我很冷!''

岳绮罗站在雪中,双手揣在袖子里,人不动,只有头发随着寒风轻轻的飘:``你爱我什么?''

无心笑了,反问道:``你又爱我什么?''

岳绮罗静静的凝视着他:``爱你的身体。''

无心弓起身体,双脚向上一直蹬到了井口:``只有身体?''

岳绮罗突兀的一笑,眼睛眯成半月。笑容稍纵即逝,她随即恢复了平静:``谁的灵魂值得我爱?凭着我的智慧,看谁都是水晶琉璃。一眼看透,还爱什么?''

然后不甚情愿的翻了个白眼,她奶声奶气的哼道:``高处不胜寒,想必你也理解我的寂寞。''

无心轻轻笑了一声,忽然很想念月牙和顾大人,甚至包括出尘子道长。他的确是理解岳绮罗的寂寞,不过她是自作孽、不可活。

好在他怪物见得多了,也不差岳绮罗一个。岳绮罗不放他出来,大概是还没有想好如何控制住他;脚趾头蜷起来勾住井沿,他仰起头望天。万里长空,乌云密布;井水也许很快就要结冰了。

岳绮罗微微低了头,从刘海中抬眼看他;看着看着,她看到了铁条上的清晰齿痕。

大步流星的走上前去,她指着齿痕问道:``谁咬的?''

无心经过几夜的试验,已经对小鬼彻底失望,所以坦然答道:``棺材里的丑丫头。''

岳绮罗当即转身走向门前棺材,冷风席卷而来,吹起披风下摆,露出里面一身青色裤褂。不用旁人出手,她亲自推开棺盖,只见里面的小鬼仰面而卧,本来已经是个半腐烂的状态,如今受了稀薄阳光的照射,越发像被火灼一般,模样眼看着越发败坏,七窍都流出了黄汤绿水。抬手搭上漆黑的棺材盖,岳绮罗念念有词的画出一道符咒,最后一笔狠狠的抹出去,她闭上眼睛仰起脸来,声音又轻又急:``先杀恶鬼,后斩夜光,何神不服,何鬼敢当。太上老君急急如律令!''

抬手用力向上一挥衣袖,她猛的睁开了眼睛。附在小鬼身上的魂魄当初被她召之即来,如今又被她挥之即去。转身走回院子里,她命令四角的士兵:``棺材和人全部烧掉!''

然后她转向了井口:``大哥——''

无心已经无影无踪,井口的铁罩下面贴着一张黄符。黄符对于岳绮罗很有震慑作用,黄符一现,就表示无心要下去休息了。

无心浮在水中,陪伴他的是几条小银鱼。鱼嘴轻轻亲吻了他的耳垂和鼻尖,每天的伙食都不错,如果不是月牙和顾大人更有诱惑力,如果不是空气和水都越来越冷,也许他会安心的住下来。侧过脸抬起手,他眼看着小银鱼游过自己的指间。水流瞬间紊乱了一下,一条小鱼失了踪影;而无心的喉结缓缓滑动,是做了一次刹那间的捕猎。

几天之后,井水表面当真是结冰了。

无心吊在铁罩下面,双腿分开了蹬在井壁上,向下哗哗的撒尿,尿也是冰冷的。岳绮罗蹲在铁罩上,戴了一副雪白的兔毛耳套。眼看无心尿完了,她伸下一根手指,用力戳了无心的头顶心:``想不想出来?''

无心立刻抬了头:``想。''

岳绮罗起身走下铁罩,然后继续说道:``想出来,就先烧掉你的黄符!''

一名士兵划了火柴凑到铁罩近前。而无心并不反对,很顺从的取出黄符,当真是送到火苗上一燎。

大条石被搬开了,铁罩子也被掀起来了。岳绮罗怕无心伤人,向后退出老远;而在四支步枪的瞄准下,无心坐在井台上,慢条斯理的穿上了衣裤鞋袜。

岳绮罗远远的提防着他:``你现在对我是爱,还是恨?''

无心低头笑了一下,一边系纽扣一边答道:``凭着我的智慧,还会拘泥于爱恨吗?''

然后他抬眼望向岳绮罗:``接下来怎么办?你是关我,还是放我?''

岳绮罗皱起了眉头,发现自己对于无心是老虎吃天、无处下爪。无心似乎是真的无所谓爱恨,人太好摆布了,不是人的又太不好摆布了,岳绮罗正了正自己的耳套,一时不知如何是好。

``不关你,也不放你。''她最后开口答道:``留你住几天,怎么样?''

无心笑道:``恭敬不如从命,住就住。''

岳绮罗也笑了一下,右眼隐隐作痛。还没有告诉无心她已经盲了一眼,因为感觉没有必要。无心不会怜悯她瞎了右眼;她也犯不上自曝其短。

岳绮罗带着无心住进了顾宅前院。雪势越发急了,宅院内外阴风凄厉、魂魄遍布。房内燃了火炉,桌子正中央摆着一只瓷盆,里面咕嘟嘟的沸腾着一盆肉汤。岳绮罗和无心相对而坐,两人一起注视着盆中有鼻子有眼的小婴儿。

无心很平静的抄起一只大馒头,咬了一口慢慢咀嚼。而岳绮罗喝了一口滑腻的肉汤,不由自主的打了个冷战。

``吃人补人。''她轻声自语:``天寒地冻,我得补补。''

无心咽下馒头,反问她道:``怎么没有我吃的菜?你知道我不吃人。岳绮罗,你自己吃得满嘴流油,却让我嚼干馒头,可见你根本不爱我。''

岳绮罗一筷子伸进瓷盆,连汤带水的挑起一只圆滚滚的小脑袋。把热腾腾的小脑袋夹到自己碗里,煮烂了的皮肉零零落落,一颗熬成乳白的眼珠子半路掉下,一路滚过桌面掉到地上。一口气把小脑袋吮成空空荡荡的脑壳,她舔着嘴唇抬起头:``大哥,有的吃,为什么不吃?是人的,尚且对人敲骨吸髓;何况你根本就不是人。''

无心摇了摇头:``所以我和你过不到一起去。道不同,不相为谋。''

岳绮罗笑了:``你和谁能过到一起去?月牙?''

无心不搭她的话茬,生怕把她的注意力转移到月牙身上去。他一鼓作气吃了五个馒头,岳绮罗也吸吸溜溜的吃了整个婴儿。右眼的疼痛渐渐缓解了,她的体内又有了热气。忽然留意到了无心的目光,她没言语,单是微笑。

无心也在微笑,同时暗暗把舌尖伸到齿间。门外一定站着士兵,他一个人打得过岳绮罗,然而打不过四个顾大人似的小伙子。当然,如果一定要逃,办法还是有的,只是要么太危险,要么太痛苦。

还有一个太简单的法子,胜算几乎为零,不过可以试一下。无心手按桌沿站起了身,一言不发的走向门口。伸手推开两扇房门,他深深吸了一口寒冷空气,然后一步跨过门槛。

岳绮罗莫名其妙的看着他:``你干什么?''

无心把寒冷空气呼出去,另一只脚也站到了门外。背着双手经过两边全副武装的士兵,他回头对着房内的岳绮罗一点头:``雪很大。''

随即他转向前方,撒腿就跑。岳绮罗猛然起身赶了出来,随手夺过士兵手中的步枪,她拉动枪栓也不瞄准,对着无心的背影就扣动了扳机。一声枪响过后,无心被子弹向前轰了个跟头。然而一挺身爬起来,他已经拉开了顾宅的黑漆大门。

岳绮罗知道他不会安分,可是没想到他会公然逃跑。拔脚向前追了两步,她一边笨手笨脚的将子弹上膛,一边锐声喊道:``来人,给我追!活要见人死要见尸!''

``死要见尸''四个字一出来,士兵心里就有数了。四名青年蜂拥而出,岳绮罗站在院内,就听外面枪声响成一片,纵算无心能够飞天遁地,怕是也要被子弹打成筛子了!

\chapter{辗转}

枪声响彻了整条胡同,此起彼伏的不停。岳绮罗紧随其后的追出去,就见无心在前方路口拐了个弯,人影瞬间消失不见。她人小腿短,衣裳穿得又累赘,没跑几步就冒了汗。幸而士兵伶俐,一路追一路开枪。岳绮罗最后出了胡同,只听一名士兵扯着正在变声的哑嗓子,撕心裂肺的狂喊:``死了!打死了!''

岳绮罗猛然刹住脚步,下意识的抬手掩到了鼻端。空气中弥漫起了一股子说不清道不明的血腥味,而远处大街上趴伏着个一塌糊涂的人,正是无心。

岳绮罗并不怕血,然而无心的鲜血气味让她感到了窒息。手掌加上衣袖都无济于事,她明明白白的吸进了一股子又甜又腻又冷又腥的恶味。右眼针扎火燎的疼起来了,她连着退了几步,大声问道:``怎么回事?''

一名士兵端着步枪停在半路,余下三人跑上前去,用枪管翻动了地上的尸体。无心软绵绵的趴在街面上,身上不知中了多少粒子弹。脑壳是早破碎了,后背也被轰出了大洞;左腿从膝盖处断了开,两条手臂更是被打飞了皮肉,臂不成臂,手不成手。一个胆子大的弯了腰,伸手把他翻成了仰面朝天,然而面也没了,只留下了个完好的下巴;胸口红红白白的绽开来,红的是血,白的乍一看像棉袄里的棉花,仔细一瞧又不是,是嚼碎了咽进肚里的馒头。

三名士兵方才光顾着射击了,没料到乱枪会被人打成零零碎碎。有人发现了问题:``人都打烂了,怎么没血啊?''

此言一出,余下二人一怔,发现地上的确没有血流成河,只有黏黏腻腻的一小滩殷红,气味甜得恶心人。

在岳绮罗的命令下,四名士兵找来一只竹筐和一把铲子,把无心铲进了筐中。岳绮罗站在百米开外,心里不信无心会真的死了。既然没有魂魄,他的玄妙必然就在身体上,所以岳绮罗铲也要把他铲回去。铲回去封起来,倒要看他能有何种变化!

待到岳绮罗和士兵们一起撤退之后,街上重新恢复寂静。一条肮脏不堪的大野狗一路嗅着跑了过来,围着地上血迹转了一圈。

薄薄的一层血,已经被冻在了地面上。大野狗嗅过之后,连个肉渣子都没找到,便走到路边暗处沉下屁股,百无聊赖的拉了一坨狗屎。

拉过之后它垂了尾巴,似乎一时失了目标方向。而寒风吹过路边荒草,一只齐腕而断的手就忽隐忽现的向它逼近了。

食指中指迈着小步,拖着后方的整个手掌直奔野狗而去。忽然一把抓住狗尾巴,大野狗受了一惊,当即漫无目的的吠了一声,又吠一声。

两声吠过之后,那只手已经顺着尾巴攀上了它的后背。五指张开附在大野狗的皮肉上,污秽凌乱的狗毛遮住了它的行迹。

大野狗继续向前跑去,跑两步停下来,落水狗似的抖一抖,然后继续再跑。

大野狗在街上跑了一夜,凌晨时分停在了一户人家门口。天还没亮,院门已经开了,一个年轻小伙子睡眼惺忪的出来套马车,身后跟着个拎泔水桶的老太太。老太太把泔水往路边一泼,同时咳嗽气喘的嘱咐小伙子:``等在青云观里见了老东家,就想着提提换差事的话。老东家善良,兴许能答应。''

小伙子哈欠连天的满口答应;而大野狗则是在路旁尚未结冰的泔水里寻找剩饭吃。埋伏在狗毛里的手通了灵成了精,听见``青云观''三个字后,立刻开始不动声色的转了方向。

小伙子坐上大马车,一甩鞭子吆喝一声,全然没有注意到一只手扒在车窗窗口,顺着厚窗帘子就翻进去了。

无心没想到自己会``活''在了一只手上。夜里一枪打上手腕,他就感觉天旋地转。等到清醒过来之时,他发现自己变成了一只手。手是落在了路边的草丛里,手指很灵活,让他可以到处走。从一只手长成一个人,所需时间不会少;所以他打算先回青云观报声平安,然后再找个地方藏起来慢慢成长。但是一只手堂而皇之的在路上走,显然是不大合适,况且从文县到青云山路途遥远,恐怕路未走完,他已经不知变化成什么怪样子了。

无心摔在了马车座位上,食指轻轻叩着车座,他此刻疼倒不是很疼,只是有些犯愁,怕月牙会嫌弃自己。

大马车呱嗒呱嗒的走在大街上,速度很快。街上渐渐见了人,赶车的小伙子不住的遇见朋友,嘴里也有了话说。无心静静听着,得知小伙子的老东家家财万贯,一直住在青云观里修道。如今天冷了,春节也快到了,所以少东家支使小伙子跑一趟,去把老东家接回家来过节。马车顺顺利利的出了文县,沿着土路跑出一溜黄烟。无心被颠簸得蹦蹦跳跳,心想也许不到天黑,自己就能上青云山了。

傍晚时分,小伙子把大马车停在山门外,自己沿着山路往上跑。一个小道士背着一捆柴慢悠悠的跟在后面,柴捆里躲着个快要冻僵的无心。

柴禾被扔进了柴房里,小伙子自去寻找老东家,小道士自去吃晚饭睡大觉。柴房的破门开了一道缝,夜色之中,一根手指头鬼鬼祟祟的探了出来。

食指搭上了门槛,随即中指也跟上去了。手掌一使劲立了起来,食指中指迈开大步,一溜烟的就跑了。

凌晨时分,无心进了月牙和顾大人所住的小院。

他先跑去了月牙的门口。食指和无名指站立稳了,他伸出中指推了推门。

门锁的严实,于是他转而又跑去了隔壁的顾大人门前。月牙是个女人,夜里睡觉当然要关门闭户;顾大人却是满不在乎,横竖门是破门,锁不锁都无所谓,全是一样的不挡风。无心侧过手掌钻进大门缝里。屋里生了炉子,炉子加上顾大人,营造出来的空气正是暖融融臭烘烘。无心惬意的打了个冷战,然后就想要上炕。可是炕太高了,他无处攀爬,上不去。忽然感觉到了旁边就是顾大人的大棉鞋,无心索性爬进了鞋里,反正没鼻子,不怕熏得慌。

再说顾大人仰天长睡,直到天明时分,才被一泡尿憋醒。迷迷糊糊的一掀被子坐起来,他披上棉袄穿上棉裤,伸下双腿想要趿鞋出门。不料大脚丫子往棉鞋里一踩,他忽然感觉脚底下软中带硬的硌人。揉着眼睛低头一瞧,顾大人看到一根手指勾着鞋帮,正在奋力的向外爬。

顾大人把嘴张成瓢大,亮着嗓子眼打了个大哈欠,顺带着抬手抹下眼角一粒眼屎。感觉自己是清醒透了,他低头再看,发现一只苍白的手已经爬出了棉鞋。

第一缕阳光透过窗子,射在顾大人的脚丫子上。一团怒火忽然腾起,顾大人光脚下地,蹲下来抄起大棉鞋骂道:``好你个狗娘养的妖魔鬼怪,大白天的还敢来吓唬我!操!老子今天要不给你几分颜色,你就不知道马王爷有三只眼!''

话音落下,他一鞋底子就拍了下去,当场把无心拍扁在地。无心活动手指,还想在地面写字示意,可是顾大人怒发冲冠,片刻的机会都不给他,噼里啪啦的就只是拍。无心被他打得满屋逃窜,而顾大人拧着眉毛瞪着眼睛,一手一只大棉鞋,蹲在地上转圈追他。月牙刚起了床,蓬着一脑袋头发从茅厕里走出来,因听顾大人房内热闹,就凑到窗前向内张望:``顾大人,你干啥呢?屋里闹臭虫啦?''

顾大人头也不抬,两只手对无心围追堵截:``没事,我屋里来了个妖怪,今天我揍不死它我就不姓顾!''

月牙一听来了妖怪,也不避嫌了,推门就往里进。结果一只脚刚迈进去,便有一只手横窜过来,死死抓住了她的裤脚。她低头望去,正要尖叫,但就在要叫不叫之时,她弯下腰,忽然说道:``顾大人,别打,我看它怎么像是无心的手?''

顾大人双手套着大棉鞋,目瞪口呆的抬起了头:``师父的手?''

月牙没言语,试试探探的向下伸出了手,两只眼睛睁得特别大。而抓着裤脚的手仿佛有所感应,及至月牙的指尖快伸过来了,它不知怎样运的力量,竟然一跃而起。两只手瞬间交握了住,月牙转动大眼珠子,和顾大人对视了。

``无心啊\ldots{}\ldots{}''她开了口,声音打着颤:``是你吗?''

断手立刻抬起一根食指,在她手心里轻轻的划起圈来。

\chapter{无心的成长}

月牙屋里干净不臭,所以两人一手一起挪到了她的房中。月牙手忙脚乱的叠了棉被摆上炕桌,而无心的手就搭在她的肩膀上。肩膀下方便是斜襟纽扣,一根手指头跃跃欲试的往斜襟里探,因为里面更暖和,而且有两个香喷喷的大馒头。

顾大人把棉鞋穿在了脚上,手里换了一根擀面杖,随时预备着向月牙肩头来一下子:``我说,你确定这是师父的手?''

月牙忙得满头满脸都是长发,人就躲在头发里回答道:``他从头到脚都让我看八百遍了,我能不知道自己男人的手长啥样?''

话音落下,她沉重的叹了口气。而无心用小拇指勾住月牙的衣领,食指和拇指腾出来,对着顾大人作势一弹。

顾大人不由自主的也跟着叹了口气:``这怎么一次不如一次?上次只少了半个脑袋,这回可好,就剩一只右手了!''

月牙和顾大人盘腿上了炕,手则是被摆在了炕桌上。月牙把头发胡乱向后挽了个纂,心里也说不清是什么情绪。如果无心缺胳膊少腿的回来了,她肯定要又怕又疼的搭上许多涕泪;可是面对着桌子中间一只手,她总感觉自己是没睡醒。

顾大人也有梦游之感。盘腿坐在月牙的热炕头上,他连袜子都没穿,脚趾头下意识的动来动去。而无心的手趴在桌上,食指中指先是轮换着敲了敲桌面,感觉两人的目光都射向他一只手了,他才移动手指,开始在桌面上一笔一划的写字。月牙在很小的时候跟着她舅舅学过一点文化,大字勉强能认一箩筐,其中还夹杂着许多白字,所以无心直接写给顾大人看,断腕之处露出雪白的骨茬,也一并落在了顾大人的眼里。顾大人呆望了片刻,忽然扭头打了个大喷嚏;月牙倒是渐渐反应过来了,隔着桌子伸手一拍他:``你别走神,看看他写的都是啥!''

无心在桌子上长篇大论,末了提出要求,让顾大人把自己偷偷埋进土里。

月牙已经彻底认清了现实,想到无心遭了乱枪,一枪一个血窟窿,她果然是心疼的涕泪横流。听顾大人转述了无心的话,她拿起手帕一擤鼻涕,当即瓮声瓮气的表示反对:``不行!两间屋子还不够你长的?非得往地下钻?大冬天的,地都冻上了,你要活埋作死啊?''

顾大人愁眉苦脸的也是同样意见:``师父,不瞒你说,你现在这个模样,看着比上次利索不少。月牙不怕,我更不怕。只要你别耗子似的满地跑,养在屋里就养在屋里,我也不反对。''

无心等二人都说完了,继续写字,表示自己现在看起来是一只手,过两天就不一定长成什么德行了。

月牙不想再和他耍嘴皮子,直接泪眼婆娑的告诉他:``屋外是爷们儿做主,屋里是娘们儿做主。今天我就做主了,我那笸箩呢?''

不等人回答,月牙自己爬到炕角,把针线笸箩端了过来。针线被倒出去了,她又往笸箩里面垫了一层枕巾:``往后你就在这里面睡,等到长大些了,我再给你换个篮子。''

无心静了片刻,又写了起来,要到顾大人房里住。他很知道自己的成长过程,所以并不想让月牙亲眼目睹。月牙能够接受自己到这般地步,已经算是奇女子了,他想凡事都有个限度,不能因为月牙不怕,自己就无休止的扰她吓她。万一哪天月牙一甩袖子真不要自己了,自己可就傻眼了。

月牙不在乎他住到哪屋,只是坚决不肯把他埋进土里。顾大人掏了掏耳朵:``住我屋里\ldots{}\ldots{}行倒是行,不过你得老实点,我醒你醒,我睡你睡,而且不许满炕乱爬。''

协议达成,风平浪静。月牙烧热水自己洗了把脸,又拧毛巾擦了擦无心的手。擦手的时候顾大人凑上来了,很好奇的用手指去触断腕。月牙登时一转身隔开了他,急赤白脸的怒道:``你别弄他!''

顾大人绕到了她的面前,很认真的告诉她:``你看他那腕子里面,怎么不大对劲?''

月牙看了看手腕创口,发现骨头虽然依旧白生生,里面的红肉表面却像是结了一层透明薄膜,轻轻一捏手掌,手掌好像也厚了。

``可能是开始长肉了!''月牙抬眼去看顾大人:``你摸摸,手背都鼓溜了。''

顾大人想要和无心握握手,然而无心顺着月牙的手臂往上爬,一溜烟的又回了肩膀。月牙抬手拍了拍他,心想幸亏我没娘家,要不然女婿这个样,娘家还能让我跟他过下去吗?

月牙本来不大管顾大人的,因为顾大人是烂泥扶不上墙,把他收拾的再干净,一天不管也要回复原样;可是无心既然回来了,又是住在顾大人的屋里,她便放了心,有了闲精力去多干点活。把盛着无心的笸箩摆到顾大人的炕上,她一边扫地一边自言自语:``你得怎么长呢?先长胳膊再长身体?''

无心感觉此事一言难尽,要写也是千言万语,并且未必能写明白,所以趴在笸箩里就没回应。顾大人端着一碗热汤面上了炕,哧哧溜溜的吃出一头大汗;于是月牙拎着笤帚直起腰,又有了问题:``你连嘴都没有,咋吃饭呢?''

无心爬出笸箩,在炕上刷刷点点的写起来;顾大人直着眼睛看着,看到最后告诉月牙:``用水泡一泡他就行,他成人之前吃不了饭。''

月牙想了想:``水也不顶饿啊,熬点汤泡一泡呢?''

无心在炕上写了三个大字:``别放盐!''

顾大人受了无心的嘱咐,并没有向出尘子通报消息,怕老道闻信赶来降妖除魔,再把无心剁碎了。反正青云观产业庞大,只要住持发了话,其余道士并不在乎观里多了他们两个吃闲饭的外人。

到了下午,无心支使顾大人去寻一口大缸回来。顾大人嫌天气冷,不肯出门;月牙也说:``缸里又冷又硬的,哪有笸箩舒服?''说着她又找了一条枕巾搭在笸箩上:``再给你加条小被。''

无心没了办法,趁着自己还能活动五指,他爬到月牙身上,摸了摸脸蛋又摸了摸头发,亲热的了不得。月牙知道他的意思,趁着顾大人不注意,她把无心捂在了胸脯上。

入夜之后,月牙自去回房睡觉。顾大人上了炕,片刻之后也是鼾声如雷。笸箩摆在炕头,无心被枕巾盖住了,黑暗之中就见枕巾下面一膨一膨,像是活生生的一颗心脏再跳。

顾大人睡得很熟,梦里回到了两年前。两年前他杀伐征战,在猪头山下所向披靡。一路杀到天大亮,他睁开眼睛醒了过来。眼望着四周简陋的环境,他若有所思的翻了个身,满心都是怅然。

伸手把炕头的笸箩拽过来,他枕着胳膊问道:``师父,还睡着呢?''

枕巾下面没有动静,不是无心的行事作风。顾大人忽然怀疑他趁夜溜了,连忙掀开枕巾向内一探头。然而一瞧之下,他大惊失色,猛然坐了起来!

原来笸箩里面的手,已经手不成手。

屏住呼吸怔了一瞬,顾大人壮了胆子,把笸箩拉到近前细看,就见一块拳头大小的红肉赫然隆起,撑得手背皮肤四分五裂。纤细的指骨裸\textbar{}露出来,也被红肉挤得东倒西歪。肉是鲜红透亮的,表层似乎绷了一层薄膜。顾大人小心翼翼的伸手过去碰了红肉一下,软颤颤的只是嫩,并没有异样触感;俯身下去又嗅了嗅,隐隐的似乎有些甜腥,除了甜腥之外,也无其它异味。

顾大人也以为无心会长完胳膊长身体,万没想到一夜过后不但没有胳膊,甚至连手都失去了。端起笸箩凑到窗前,他迎着阳光细看;发现红肉其实不像肉,更像一胞血,不透明,可是隐隐的能透光。

顾大人不敢碰它,怕把它碰破了。轻手轻脚的放下笸箩,他穿上衣裤趿上棉鞋,连尿都没撒,直接奔去了隔壁月牙房中。做贼似的溜进去,他压低声音说道:``了不得,师父真变样啦!''

月牙吓了一跳:``变啥样了?''

顾大人向门一指:``你自己瞧瞧去吧!''

月牙见了笸箩里的东西,也发了傻。她没主意,顾大人也没主意。无可奈何之下,只好把日子照例过下去。一大碗肉汤晾得不冷不热了,月牙小心翼翼的要从笸箩里把无心捧出来,结果一捧之下,皮和骨头全落下去,就只有一块肉留在了她的手中。

把肉放进汤碗里,月牙从笸箩里捡起了一根手指。手指上的肉皮看起来干燥腐朽,骨头也是特别的轻,仿佛一捏就能碎。月牙咽了口唾沫,胆战心惊的真害怕了。

``你\ldots{}\ldots{}''她转向大碗,轻声问道:``你是无心吗?''

碗里的肉毫无反应,仿佛就只是一块怪模怪样的肉。

一天之中,无心没有继续变化。入夜之后,月牙想要把笸箩端到自己屋里去,然而顾大人存了好心,执意要把笸箩留下。

月牙一宿没睡好,知道自己嫁的不对劲,可是让她抛了无心另找汉子,她又实在是舍不得他。恍恍惚惚的过了一夜,翌日清晨她刚刚下炕打开房门,冷不防的就见顾大人从隔壁冲了出来,大惊失色的对她嚷道:``完了完了,师父变成蛆了!''

\chapter{千变万化}

月牙和顾大人并肩站在炕前,望着炕头的笸箩目瞪口呆。

昨天还是拳头大的一块红肉,一夜的工夫竟然抻成了一尺来长,一头浑圆一头尖细,鲜红的颜色也变淡了,看着正是粉粉嫩嫩的一条大蛆。小小的针线笸箩已经容不下它,尖细的尾巴伸出边沿,软软的搭在了棉被一角上。

最后,还是月牙打着结巴先开了口:``咋、咋长成这样了?''

顾大人端起笸箩掂了掂分量:``比昨天重了不少,至少增了一斤多。''

昨天它是块心脏大小的红肉,瞧着虽然怪异,但是还不可怕。如今红肉变成了软颤颤的一大条,可就有点瘆人了。顾大人迎着窗子光亮托起笸箩,两个人的脑袋凑在一起细细审视大蛆,就见它体内隐隐现出一条白线,从头延伸至尾,不知道是什么东西。

月牙奓着胆子伸出手去,轻轻的摸了它一下,摸完之后告诉顾大人:``还挺滑溜的。''

顾大人收回笸箩,低头嗅了一鼻子。龇牙咧嘴的转向月牙,他苦着脸说道:``不好闻。''

月牙也俯身把鼻尖凑了上去,长长的吸了一口气,她直起腰:``是不好闻,又有点甜又有点腥。''

顾大人问月牙:``他原来身上也是这味吗?''

月牙立刻摇了头:``不是不是,他原来没味。''

然后两人一起长叹一声。

无心的新形象虽然不大受看,但是月牙和顾大人都是经过了风浪的人,所以也不大惊小怪。月牙照例是收拾屋子烧水做饭,顾大人洗漱穿戴完毕了,奉了月牙的命令,把无心从笸箩里取出来,转移到一只大竹篮子里。

放好无心之后,顾大人低头盯着它又瞧了半天,越看越像蛆,末了就感觉浑身难受,并且恶心。把篮子轻轻的拎起来放到炕里,他把自己的棉被扯了过来。棉被经过了臭屁和臭脚丫子的彻夜熏陶,温度和气味全具备。顾大人用棉被把篮子严密盖住,正是眼不见心不烦。

到了下午,顾大人进了月牙的屋。人都有个爱美之心,月牙屋里干净,月牙本人也打扮的利落;顾大人坐在月牙的热炕头上,心里熨帖了许多。

月牙把篮子也拎过来了,篮子上面搭了一条枕巾,放在炕头。月牙一边做针线活,一边隔三差五的往篮子里扫一眼,希望能看到一点动静。然而大蛆怡然自得的躺在篮子里,一动不动。

针线活做久了,月牙放下针直起腰,抬头唤道:``顾大人,你说——''

顾大人正在发呆,冷不丁的受了惊动,立刻就是一哆嗦。月牙没想到自己会吓着了他,登时也闭了嘴。双方默然片刻,顾大人忽然苦笑了一下,问道:``你刚才叫我什么?''

月牙莫名其妙的看着他:``我叫你顾大人啊!''

顾大人扭头望向窗外:``没有兵没有马,没有枪没有钱,我他妈算什么大人!''

月牙眨巴眨巴眼睛,没领会意思:``叫惯了,你要是不乐意听,我往后改口不就行了?你说你让我叫你啥?''

顾大人知道月牙层次不高,但是身边没亲人,就她还算是个家里人了,心里有了话,只能对她说:``月牙,你知道我当初是什么样吧?''

月牙把针又拈起来了:``知道,你当初挺威风的,我见了你都不敢抬头说话。''

顾大人点了点头,随即一拧眉毛:``你放下针线,纳鞋底子着什么急?老实听我说话!''

月牙笑了,不和他一般见识:``行,行,你说吧,我听着呢。''

顾大人面无表情的看着他,同时说道:``月牙,我不能在道观里继续混下去了,我得出去打天下!''

月牙登时紧张了:``打天下?你单枪匹马的想打谁啊?刚消停了没几天,你又要兴风作浪了?''

顾大人一摆手:``不要头发长见识短,我当你是我亲妹子,才和你说心里话的!谁说打天下就非得动刀动枪?你当我除了张小毛子和丁大头,就不认识更高级的大人物了?我告诉你,算命的说我是武曲星下凡,此生必成大业,我住在道观里不活动,大业怎么成?''

月牙听他吹牛放炮,感觉挺有意思:``你就说你想干啥吧?''

顾大人舔了舔干燥开裂的嘴唇,郑重其事的说道:``我打算去趟天津,你也跟我去。正好师父没长大,还能省一张火车票。天津可是个大城市,你没去过吧?''

月牙摇了摇头:``我肯定没去过,连长安县我都是第一次来。''

顾大人踌躇满志的扬起头,望着窗外的蓝天白云:``本来我还想把散了的弟兄们召集起来,重新打回文县;可是经过了几个月的琢磨,我发现就算真把队伍拉起来了,我也不是丁大头的对手,而且文县里面还住着个妖怪,让我去我也不敢去。所以我打算到天津碰碰运气,大不了就空手回来呗,顶多是搭点路费,也不算什么。''

月牙对顾大人的前程毫无信心,不过倒是想起了另一件事:``咱们要是走远了,是不是妖魔鬼怪就追不上来了?''

顾大人抬手挠了挠头:``应该是吧!''

月牙瞟了篮子一眼:``也不知道无心愿不愿意去,再说就算省了他的火车票,咱俩也还是没盘缠啊!现在吃的用的,还都是人家道观里送的呢!''

顾大人不敢看篮子,直接一挥手:``管他愿不愿意呢,反正他现在也没说不愿意!至于盘缠,我下午就去找出尘子,看看能不能跟他借点钱。总之我得赶紧行动,要不然日子拖久了,谁知道师父又会变成什么样?万一过两天成了半人来高的一条大蛆,咱们可怎么把它往火车上带?''

月牙年纪轻,好奇心盛,依着她的心意,倒是愿意去天津开开眼界——当然,去也行,不去也行。而顾大人见她并不反对,就在吃过午饭之后,当真出门找出尘子去了。

顾大人出去了不过一个多小时,就带着两百多块钱回来了。喜笑颜开的进了月牙的屋,他真心实意的将出尘子赞美了一番:``人家那老道是真仗义,说拿钱就拿钱,还不让我还。我早就看他不是凡人,那大个子,那长头发,那气质,那派头,可惜出家当老道了,要不然也得是个大官!''

月牙看他吵吵闹闹的,不禁也来了精神:``他问没问起无心?''

顾大人高声大气的答道:``问了,我说我不知道。''

月牙有点激动,抬手摸了摸脑袋后面的圆髻,莫名的有些自惭形秽:``那咱们真去天津?你到了天津投奔谁啊?''

顾大人大喇喇的一挥手:``你别管,我又不是大傻×,心里能没数吗?''

到了晚上,月牙把无心捧出来,放在了一盆温暖的菜汤里。汤里没有放油,泡到汤冷之后,她把无心捞出来擦了擦,然后对顾大人说道:``你要是怕它,就把它放我屋里吧。我看了一天,现在都看惯了。''

顾大人犹豫了一下,有心答应,可是如果真答应了,就算是违了自己和无心的约定。伸手拎起篮子,他硬着头皮说道:``不用,我也看惯了。再说谁知道他明天早上又变成什么样了?变好看了还行,要是变得还不如蛆\ldots{}\ldots{}算了算了,还是我拎走它吧!明早我打头阵,好不好的我先看第一眼。''

因为说定了明天就下山到长安县上火车,所以月牙天一黑就上了炕,想要早睡早起,然而辗转反侧,却是睡不着觉。顾大人躺在臭被窝里思索天下大势,也是闹了失眠。两人全是直到午夜才睡,仿佛刚一闭眼便亮了天。

顾大人心里揣着大事,躺不住,一见窗户白了,就坐起来先去看篮子。篮子上照旧搭着一条枕巾,顾大人伸手捏住枕巾一角,一颗心在腔子里怦怦乱跳,不知道自己接下来会看到什么东西。

一咬牙一狠心,他猛的掀开了枕巾。低头向内一瞧,他睁大眼睛,忽然很想吐。

篮子里的蛆至少又长了大半尺,细尾巴不见了,从头到尾水灵灵的又粗又胖,并且不复昨日的光滑,粉嫩皮上坑坑洼洼,洼处生出尖刺刺的白毛,乍一看正是一条斑秃大毛毛虫!

顾大人理解了无心的隐忧,也承认此刻的无心实在是太不招人爱。伸手指试了试白毛的软硬,他见白毛并不扎手,便扯来一条不干不净的床单,皱鼻子瞪眼的把无心层层卷起来了。

顾大人没让月牙去看无心,只说``长得挺快,模样还跟昨天一样。''

月牙把头发梳得服服帖帖,衣裳穿得整整齐齐。接过顾大人送过来的床单卷子,她背上小包袱,意意思思的还问顾大人:``真走啊?''

顾大人意气风发的一晃脑袋:``走!''

\chapter{去天津}

出尘子身份高贵,并未亲自露面,但是命令弟子套了一辆大马车,送月牙和顾大人去长安县火车站。月牙挎着个小包袱,手里抱着床单卷子,卷子沉甸甸的挺有分量,可见无心夜里又长了不少。惶惶然的偷眼瞄着顾大人,她心里风一阵雨一阵的不踏实。进县城已经是开了眼界,可县城和镇上风光也差不许多,她纵是惊也惊得有限;天津卫就不一样了,在她心目中,天津卫几乎可以等同于外国。跟着个不着调的顾大人去外国,到底可行不可行呢?

月牙左思右想的还没得出答案,大马车已经把他们送到了火车站。

长安县的火车站,里外只有两间屋子,此刻天寒地冻又不靠年节,所以车站冷清,几乎没有旅客。顾大人自从出了青云观后,也是惴惴不安,生怕半路被鬼跟上。如今在车站里买了两张车票,他抓心挠肝的一边等车一边走来走去;后来估摸着火车快到了,他早早就带着月牙赶去了月台。

一列小火车轰隆隆的开过来,在长安县停了一分钟。一分钟后火车开动,月台上空荡荡,彻底没人了。

顾大人平时看着月牙挺体面的,模样挺好身段挺好,干别的不成,当媳妇是足够。然而如今在车厢里挤着坐下了,他才骤然发现月牙土头土脑的上不得台面。月牙占据了靠窗的位置,像刚被强盗劫过一场似的,缩着脖子端着肩膀,一脸茫然的睁着大眼睛,仿佛连东张西望的胆量都没有了;除此之外,两件行李也被她搂在胸前抱了个死紧,似乎随时预备着跳车逃跑。

顾大人用胳膊肘一杵她,低声问道:``原来没出过远门?''

月牙怔怔的扭头看了他一眼,声音轻的像蚊子叫:``没有。''

顾大人眼望前方清了清喉咙:``你放松点,坐火车你怕什么?''

月牙答道:``哦。''

然后她缩脖端腔像个猴似的,又往车窗外面望去了。

从长安县到天津卫,火车走四个钟头也就到了。前三个钟头月牙一直没敢乱动,第四个钟头她渐渐活泛了,见附近有旅客拿了冷馒头吃,就对顾大人说道:``咱们走得太急,连干粮都忘了带。''

顾大人正襟危坐:``你啊,就知道吃!''

月牙很惊讶:``哟,你转性啦?''

顾大人嗤之以鼻:``我转什么性,我一直也不馋!''

月牙又``哟''了一声,没再说话,心中暗笑,想顾大人开始装大人物了。

火车到站之后,月牙梦游似的跟着顾大人下火车出站台,一眼不眨的盯着顾大人的背影,生怕走丢了。一出车站,她登时有些眼晕——人太多了!

处处都是人,人人都说话,正好凑成个人声鼎沸,开锅似的没一处清静。月牙自从下了火车,不知怎的,嗓子还变细了,挣命似的在后方问道:``顾大人,咱们去哪儿啊?''

顾大人没听清楚,给了她一个侧影:``啊?''

然后没等她再重复,顾大人拦下一辆洋车,不由分说的把她推了上去。两人一起并肩坐好,车夫扶着车把一起身,月牙``忽悠''一下就向后仰过去了,吓得大叫一声。而顾大人对着车夫嚷了一个地名,随即无可奈何的对月牙急道:``叫什么叫,坐好!''

洋车的胶皮轮子跑在柏油路上,丝毫不颠,比坐马车舒服许多。月牙刚坐出一点意思了,洋车在一户大宅门前停住了。

顾大人下车付了钱,公然的上去敲门。大门一敲便开,月牙站在一旁,就听顾大人口气极大,劈面就是要见你家老爷。三言两语过后,对方居然真请他进去了。月牙被他安置进了门房里。瑟缩着坐在火炉边的椅子上,她一天没吃饭,肚子饿得咕咕乱叫。双手搂着床单卷子,她垂下头,忽然有点后悔,心想要是在青云观,这时候都该上炕睡觉了。

门房里面没人,她坐了许久,烤得双手双脚都暖烘烘。百无聊赖的抬手扒了扒床单卷子,她想看无心一眼,然而卷子上下两头都严密,想要扒开也不容易。月牙感觉床单卷子好像比早上又沉重了一点,就叹了口气,在心里默默的祈祷:``你可快点长吧,你长成人了,我就有依靠了。''

月牙在炉子边一直坐到了小半夜,才有个听差打扮的小伙子推门进来,说顾先生请她过去,到底过哪儿去,小伙子没说,月牙也没想着问。

又饿又渴又困的跟着小伙子走出门房,月牙顶着寒风往前走,沿途不是房子就是院子,她约摸着都走出一里多地了,还是不见头尾。末了到了一处灯火通明的屋前,屋门大开,里面散出腾腾的热气,热气成分复杂,又有酒气又有肉气,月牙吸了一口气,馋的垂涎三尺,直咽唾沫。

顾大人谈笑风生的走出门来,身边跟着个一团和气的大胖子。对着月牙一点头,顾大人又和胖子聊了十多分钟,然后才在几名听差的引领下,带着月牙走了。

一走又走出好几进大院子,出了后门还过了一条小街。最后听差把他二人送进一处小四合院里,又问:``顾先生,您还有什么吩咐吗?''

月牙抓紧时机,对着顾大人小声说道:``哎\ldots{}\ldots{}我饿了。''

顾大人恍然大悟:``我弟妹还没吃饭呢,外面有没有卖烧饼包子的?''

听差答应一声,调头出门,不过片刻的工夫,还真是买来了十个油盐烧饼。顾大人很阔绰的赏了他两块钱,又道:``我这儿用不着人伺候了,你们都回去吧!''

月牙一口气吃了五个干烧饼,又喝了半壶热水,肚里一有了食,她就来精神了:``顾大人,怎么着?咱们就住下了?''

顾大人巡视了几间屋子,发现屋内全都收拾得干干净净,便很满意:``可不就住下了?''

月牙很是惊讶:``白住?''

顾大人把床单卷子抱到了自己要住的东厢房里:``可不是白住?刚才那大胖子你看见了吧?这房子就是他的。当年他在文县外面遇了土匪,是我救了他一命。我当时没让他报答,现在落魄了来找他,他能不管我?他敢不管我?本来他是让我住他家里,但是我想咱们还带着师父,万一被人发现了,也不大好,对不对?''

月牙跟他进了东厢房:``你说得对。床单卷子呢?我再瞧他一眼,就睡觉去了。''

顾大人立刻挡在了床前:``别看了,要睡就赶紧去睡。临睡觉前看一眼蛆,有意思?''随即他挥动双手:``走吧走吧,我也要上床了!''

月牙都累极了,料想无心也不会有事,就当真回了西厢房。房内没有砌炕,摆着柔软的西式大床。月牙脱了衣裳往被窝里一钻,闭上眼睛往下一坠,直接就坠到睡眠里去了。

与此同时,顾大人也上了床。把床单卷子摆在床边,他有心打开,可是两只手都伸出去了,迟迟疑疑的却又缩了回来。

他害怕,不想看见两尺来长的斑秃毛毛虫。有床单卷着,看着还挺利落;如果没了床单——顾大人想象了一下,随即打了个冷战,酒都醒了。

伸手关了电灯,顾大人躺下也睡了。

天明时分,顾大人醒了过来。窗外天空还是鱼肚白,房内光线暗淡,看什么都是模模糊糊。顾大人侧身注视着床单卷子,就见卷子绷得很紧,显然里面的东西又长大了。

顾大人坐起了身,鼓足勇气扯过了床单卷子。一层一层的慢慢打开,最后隐隐的甜腥气息扑面而来,他低头望去,发现无心今天倒是没大变样,单是又长了大半尺,表面依旧坑洼不平,不但洼处的白毛越发长了,而且鼓凸地方也生出了浅浅的茸毛。

顾大人打开电灯,隔着床单托起了无心,凑近灯泡细细的看。茸毛浅淡,无心依旧是个半透明的样子,隐隐可见里面从头到尾藏着一条白线。身体长得快,白线却长得慢,模糊不清的嵌在肉中。

``师父。''顾大人忍不住开了口:``你到底是怎么个打算?眼看着也要长成一米来长了,你说你从头到脚,哪有一丝的人模样?你是想变虫子啊,还是想变蛇?''

他转身回到床前,用床单子把无心又裹起来了。

到了中午,月牙又要来看无心。顾大人把她推回西厢房,然后自己也跟着进去了。一本正经的坐在月牙面前,他发了话:``月牙,能不能别看师父了?''

月牙瞬间白了脸:``他咋了?''

顾大人知道她是误会了,连忙解释:``他没事,今天又长了大半尺。但是,真不好看,到底有多不好看,我不细说了,你自己想吧!''

月牙松了口气:``我胆大,不怕他。''

顾大人一摇头:``月牙,我比你大了十岁,也算你的大哥了,有些话,我为了你们好,是不得不说。你和我不一样,我和师父是兄弟,他长什么样我都不在乎,我又不跟他过日子。可是你和他一张床上睡觉,要是看多了\ldots{}\ldots{}我怕你以后犯恶心,不乐意和他睡一个被窝。''

月牙低头想了想,最后苦笑了一下:``我认命了,他爱啥样就啥样吧,我不在乎。''

顾大人沉吟着劝道:``你不懂,当初我可喜欢我家老五了,可是自打见了井里的女鬼之后,我一看老五披头散发的就受不了。再说师父和我也是一个意思,你就听我一句吧!''

月牙垂着脑袋,没说听,也没说不听,默然无语的摆弄起了手指头。

\chapter{人形}

岳绮罗站在一把椅子上,低着头往面前的缸里瞧。

缸里盛着一堆散碎皮骨,皮已经是干软的要烂成絮,骨头也是又松又脆,不禁碰触,一团乱糟糟毛茸茸的头皮搭在上层,上面摆着一只干瘪的眼球。

岳绮罗眼看着无心的肉体变成了一缸乌烟瘴气的垃圾,莫名其妙,无能为力。而丁大头旅长笑呵呵的站在门口,脸色惨白,傻笑得满脸都是干枯皱纹。缺魂少魄的人是不能久活的,他恐怕也撑不了多少天了。

岳绮罗抄起一根木棍,伸进缸里搅了搅,搅起一团烟尘,呛得她直咳嗽。

与此同时,顾大人也是站在房内一口大水缸前。月牙站在外面扫院子,扫得满院唰唰直响;而缸里腾出温暖的热气,是刚有温水注入进去。

几天的工夫,无心又变样了。

顾大人微微弯腰往缸里看,就见一条半人多长的粉红肉虫盘在水中,和前几日相比,肉虫身上的凹处更凹,凸处更凸,乍一看竟是疙疙瘩瘩的样子,饶是顾大人神经坚强,也有些忍受不住。每天早上都成了一道关,因为肉虫已然蠕蠕的会动,时常是顾大人一睁眼睛,就发现白毛已经刺到了自己的鼻端。

顾大人实在是扛不住了,夜里干脆就把无心放进缸里泡着;等到天亮了,自己精力足胆气壮了,再把它从缸里捞出来,放到床上抻直了晾一晾。然而无心似乎并不领情,顾大人一眼没看住,它就自动的要往黑暗闷热的臭被窝里钻。

顾大人拿了一条小毯子盖住缸口,然后推门对着月牙说道:``大晚上的扫什么院子,正落小雪呢,扫也是白扫。进屋听你的话匣子去吧,在外面冻着好受?''

月牙扶着大笤帚,手和脸都冻得通红:``他今天咋样了?''

顾大人挥了挥手:``好着呢,越长越快。''

月牙又问:``有人样了吗?''

顾大人顺口答道:``有一点了,你别着急。''

月牙回了西厢房,房里的小洋炉子烧得很旺,她叹了口气,真想过去看无心一眼,然而顾大人死活不让。顾大人的阻拦是一方面,另一方面,她自己心里也有点打鼓。顾大人没白比她多吃十年米饭,说的话都有理。真要是见了太可怕的景象,她也担心自己心里会生出一道坎,一辈子都过不去。现在她闭上眼睛想起无心,还是往昔的模样,白白的面孔黑黑的眉眼,偶尔也会穿插过一条粉红色的大蛆,不过大蛆不占上风,她总觉得大蛆和无心没什么关系。

屋里摆着一台手摇式的留声机,另备着一打唱片,都是京戏。月牙听了一段戏,无情无绪的又叹一声,只希望无心快点长。

顾大人在四合院里住得挺安逸,隔三差五会有大胖子登门,两人也是言谈甚欢。月牙躲在房内,就听他们在正房高谈阔论,句句都是老帅如何如何,仿佛是顾大人想要到老帅手下混饭吃,然而老帅一直在保定练兵,不定何时才能归来。而大胖子和老帅有点交情,届时愿意做个中间人,来为顾大人引一条路。

月牙对于顾大人的前程依旧是既无信心也无兴趣,一想到无心还没个人形,她心里就慌得要长草。

无心说他长生不死,可是眼见为实、耳听为虚。真能从一只手再长成一个人吗?要是长成别的东西了,怎么办?日子是过还是不过?过,怎么过?

月牙十分忧愁,又不好对着顾大人发牢骚,以至于饭量都减少了三分之一,一顿只吃一碗半白米饭加一个烧饼就饱了。

顾大人并没有一颗七窍玲珑之心,不曾留意到月牙的愁容。他到天津是专为攀高枝来的,高枝目前在保定,他一时攀不上,索性专心致志的蛰伏在小四合院里。闲着没事,他天天研究无心。起初无心变成了毛毛虫,他还以为对方接着会结茧化蛹,最后蛹破裂开来,里面出来一个新的无心。然而毛毛虫越长越大,似乎并没有吐丝的打算,顾大人就摸不清头脑了,不知道无心要走哪条道路成人。

下午时分,顾大人到月牙屋里听了一阵唱片,听够了就支使月牙去厨房蒸饭炒菜,自己则是回到房内,预备着把无心往缸里放。不料推门往里一进,他发现床上散开的棉被之中隆起一条,竟是无心完全钻进了自己的被窝里。

他嫌无心身上有股子怪味,故而登时皱了眉毛。关严房门之后,他大踏步的走上前去一掀棉被,正要骂上几句,然而放眼一瞧,他忽然发现了问题——随着凹凸日益明显,肉虫的线条渐渐有一点像人身了!

伸手一摸肉虫浑圆的上端,里面软中带硬,细细的从上往下看,他在一丛白毛之中发现了个小小的孔洞。手指试着捅了进去,浅浅的就只是软。

顾大人惊讶了,下意识的自言自语:``肚脐眼?''

随即他一转念,又起了怀疑:``不会是屁\textbar{}眼吧?''

抽出手指开了电灯,顾大人把大肉虫翻来覆去的细看。白毛长长短短的越发密了,肉也不复先前的细嫩透明。顾大人看不出详情来,就觉得肉虫微微的动,似乎还要往被窝里钻。

顾大人没声张,照例是把大肉虫放进了水缸里,然后洗手去吃晚饭。如此又过了四五天,这一晚他把大肉虫从头到尾的捏了一顿,最后确定肉里面是长出骨头了。

整条肉虫拎起来,已经快到顾大人的胸口,分着段的有粗有细,已经隐隐看出了脑袋脖子的形状。脖子下面还是圆滚滚的乱七八糟,白色茸毛脱落了一些,新生了一些,贴着粉红肉皮生长,至于尖刺的长毛,则是落一根少一根,不再增添。

顾大人依旧是装聋作哑,内心十分淡定,感觉自己将来无论见了什么怪物,都不会大惊小怪。把无心放回大水缸,他决定在接下来的几天内忘记对方,权当屋里什么活物都没有;否则天天对着一条肉虫左思右想,他都没有精力去筹划如何攀高枝了。

对于月牙,他则是实话实说:``看来师父是真没骗人,现在已经有骨头了,虽然不多,但是都挺硬。身上还多了个眼,不知道是肚脐眼还是屁\textbar{}眼,反正有了就比没有强,是吧?''

月牙高兴极了:``都有骨头了?''

顾大人一拍大腿:``我能骗你吗?不过还是挺难看的,所以你听我说就行了,不用看!''

月牙心里有了希望,手脚不停的干活,熬了一大锅肉汤晾好了,让顾大人端起倒给无心。顾大人依言倒了肉汤,然后盖住大缸,不闻不问。

倒了翌日下午,他忍不住好奇,又往缸里望了一眼。缸里的肉汤已经没了,肉虫随着成长,渐渐瘦出了骨骼的形状,枝枝杈杈的盘在缸里,黑黢黢的也看不清详情。顾大人把缸盖严,没太看清,也无意去看清。

转眼间,一个多月就过去了。月牙和顾大人终日守在四合院里,统一的都有些懒。顾大人不敢放月牙一个人出门,怕她走丢了;也不敢两人一起出门,因为不放心缸里的无心。眼看元旦都快到了,老帅没回来,无心也没成人,倒是大胖子派人送来了节日应用之物,又请顾大人前去喝酒打牌逛窑子。

顾大人心里有事,兜里没钱,所以不肯去,宁愿从早到晚的躺在床上睡大觉。白天睡足了,晚上接着睡,并没有闹失眠的危险。一天三顿饭倒是不耽误,吃饱喝足的上了床,睡得更香。

夜里睡得正温暖,他被一泡尿憋醒了。外面正飘着鹅毛大雪,他懒得往茅厕走,推门把肚子往外一腆,翘着家伙哗哗尿了一场,心想明天月牙起来扫院子,见了一摊冻尿必要骂人,不过骂就骂吧,明天再说,自己难道还能和个小娘们儿一般见识吗?

关上房门转过身,他睡眼惺忪的要摸黑上床,然而一步刚迈出去,他忽然听到了一声呻吟。

很轻,是软软的一声``嗯\ldots{}\ldots{}'',无心的声音!

他立刻扭头望向了屋角的大水缸——因为无心近来一直是半人半虫的没大变化,所以他都连着两天没往里看了,汤汤水水也没有倒。

连忙伸手开了电灯,他走过去掀开缸上盖着的小毯子。俯身向内一瞧,他就见缸中蜷缩着一个人形,上面的圆球类似脑袋,乱七八糟的长着白毛,从脖子往下凸出一溜圆珠子,仿佛就是脊梁骨。肩膀的形状还没现出来,可是身体两侧先前生着的肉包,经过了从肉疙瘩到肉瘤子的演变,如今变成细长弯折,已经是了手臂的雏形。

``师父?''顾大人小心翼翼的出了声:``你\ldots{}\ldots{}你是不是要活了?''

似是而非的人形微微颤抖着,一个脑袋垂下去,断断续续的又呻吟了一声。

顾大人向下伸出一只手,轻轻碰触了人形,却是一片冰凉。于是他又问道:``你冷了?''

收回手直起腰,顾大人走到床边坐下来,手忙脚乱的开始穿棉裤:``你等着,我烧热水去!''

\chapter{饥饿}

顾大人蹲在厨房里捅炉子,怎么捅也不起火苗,反倒是灌了满厨房的浓烟。他是不通家务的,越捅越糟,最后就惊天动地的一边咳嗽一边逃出来了。

啪啪的拍响了西厢房的窗户,他不得已的惊动了月牙。月牙睡得正酣,此时慌忙起身向外一瞧,只见玻璃窗上一层薄霜,窗外的院子模模糊糊,不是往昔的情景;而顾大人的脸贴在玻璃上,正在疯狂的向她吆喝。

月牙吓了一跳,以为家里失火了,连忙披了棉袄推门出去:``咋了?''

顾大人被烟呛的涕泪横流:``炉子是怎么回事?不起火只冒烟?''

月牙莫名其妙:``大半夜的你弄炉子干啥?饿啦?''

顾大人用大拇指向后一指:``是师父——师父正在打哆嗦,可能是冷了。你赶紧去烧过热水,给他泡一泡!''

月牙听闻此言,一拧身就奔厨房去了。

月牙顺利的生起了火,又把一大锅水坐在了炉子上:``他都能打哆嗦了?''

顾大人袖着双手站在一旁:``还会哼哼呢,夜里他要是不哼出声,我也不能想起来去看他。''

月牙立时扭头望向了他:``现在啥样了?''

顾大人沉吟着说道:``有点像人了\ldots{}\ldots{}''

月牙莫名的兴奋了:``让我看一眼呗!''

顾大人感到了为难:``想看啊?可是\ldots{}\ldots{}反正我提前告诉你一句,他虽然有点像人了,但还是一分像人,九分像怪物。你非要看,我也拦不住你,但是看完之后你不许哭不许闹。''

月牙一边伸手试着锅里的水温,一边忍不住笑道:``我比一般老爷们儿还胆大呢,还能怕他?''

话虽是这样说,但待到一锅水热到微微发烫之时,月牙心里还是虚虚的不踏实,并且在头脑中想象出了许多恐怖形象。顾大人力气大,把大铁锅从炉子上端起来往外走,她跟在后方,一步一心跳,自己算着日子,真有许久都没见过无心的面了。

顾大人走起路来龙行虎步,眼看快要到门口了,他脚步不停,同时下命令道:``月牙,给我开门去!''

月牙答应一声,正要往前跑,不料顾大人脚下一滑,只听惊天动地一声巨响,他在门前一泡结了冰的冻尿上摔了个仰面朝天,满满一锅温水全扣在了他的头上。月牙连忙一手拎锅一手扶人,好在顾大人皮糙肉厚,并不怕摔,一翻身就爬起来了。

顾大人满头满脸都是水,张口就想骂街,可是一句话没出口,他忽然想起尿是自己撒的,正是哑巴吃黄连,有苦说不出。而月牙看他没事,推门就往屋里走。顾大人甩了甩头上的水,苦着脸也跟进去了。

房内灯光明亮,月牙一只手伸向缸上的小毯子,犹犹豫豫的转向了顾大人:``我\ldots{}\ldots{}我看了啊!''

顾大人正要回答,哪知未等他把嘴张开,缸内忽然传出了声音,又似呻吟又似叹息,像无心,又比无心的嗓子更嫩一点:``嗯\ldots{}\ldots{}''

月牙像受了针刺一样,一把就将小毯子掀开了。探着脑袋向内望去,她不言不动的僵硬了姿态。而顾大人紧张的盯着她,生怕她吓出毛病来。

足足过了五六分钟,月牙终于抬起了头。长长的吁出一口气后,她对着顾大人笑了:``你老说他丑,吓得我都不敢细想他,现在一看,也不丑哇!''

顾大人睁大了眼睛:``不丑?''

月牙挽起了衣袖:``不就是只白毛猴儿吗?我也能养!顾大人你帮个忙,把他从缸里给我弄出来,往后我伺候他!''

顾大人张口结舌:``不是——你看清楚了吗?那叫白毛猴儿?你可别往他脸上贴金了!''

月牙不以为然的一摇头:``他这个模样,真比我想的漂亮多了。你过来瞧瞧,大脑袋小胳膊的,多齐全啊!''

顾大人上前一步,细看月牙的表情,发现她满脸都是真心实意,便暗暗的感叹,心想真是情人眼里出西施,月牙连美丑都不分了。

顾大人摩拳擦掌的鼓了勇气,弯腰向缸内伸出双手,托在了无心的腋下。慢慢的把它向上带起来,无心就在灯光之中显了全形。月牙睁大眼睛打量它的面孔,只见面颊和下巴已经有了形状,正中央也鼓起了隐隐的鼻梁,鼻梁下方是两个微不可见的细孔,兴许将来就是鼻孔。无心满脸都是一层一层贴肉皮的白毛,唯独眼窝很光滑的凹陷下去,薄薄的一层透明眼皮下面透出青晕,不知道里面是否生有眼珠。

从脖子往下,就是瘦骨嶙峋的身体,两条胳膊像是脱了毛的翅膀,蜷缩着紧贴在身体两侧,腕子尖尖的纠出一撮白毛,还没有手的影子;下身更是未脱虫胚,虽然依稀能看出胯骨的存在,可是往下还是一条虫尾。

月牙刚才看他的确是像个猴子,可是如今再瞧,又感觉他和猴子还是有点差距。顾大人见怪不怪,丝毫不嫌,拦腰把它抱到了床边放好。自己伸手捏了捏它的虫尾,顾大人看月牙脸色有点不对劲,就宽慰她道:``你来摸摸,它胯骨往下新长了两根长骨头,大概再过几天,尾巴就能分成两条腿了。''

月牙定了定神,然后说道:``顾大人,你把缸先挪我屋里去吧!''

顾大人一怔:``啊?''

月牙说道:``我真不怕,它原来像蛆的时候我都不怕,现在像人了,我反倒怕了?''

顾大人不能和月牙抢无心,月牙愿意照顾它,他还乐得清闲;不过作为月牙的老大哥,他真是不赞同月牙早早的就把无心弄过去。

无可奈何的搬动了大水缸,他摸黑干起了力气活。而月牙扯过顾大人的棉被把无心裹起来,像扛一袋米面似的,她扛着无心也走了。

顾大人把大水缸摆到了西厢房的角落里,然后自觉大功告成,抱着棉被回房睡觉,由着月牙重新劈柴烧水。到了翌日上午,他坐到月牙屋里嗑瓜子,就见月牙用两床棉被把无心团团包住,乍一看还以为她在床上发面。

``哈哈!''他快乐的吐了一地瓜子皮:``怎么样?''

月牙容光焕发的盘腿坐在床上:``可乖了!''

顾大人又笑了两声,心想鱼找鱼、虾找虾,老妖怪找傻丫头。

月牙有了事做,天天围着无心一个人转。顾大人落了清闲,继续等待老帅从保定归来。他的胖朋友派听差送来了几样绸缎,说是让他做衣裳穿。他没打算找裁缝,夹着料子直接进了西厢房:``月牙啊——''

月牙单腿跪在床上,转身扭头看他,右手捏着左手食指,指尖已经凝聚了鲜红的大血滴子。一眼看见顾大人手里的衣料,月牙登时亮了眼睛:``哟,啥料子啊?''

顾大人把绸缎往旁边桌上一放:``你手怎么了?''

月牙又气又笑:``那个小挨刀的,一宿的工夫就长出嘴了,刚才我把手伸进被窝里摸它,它冲着我手指头就是一口!''

顾大人挺好奇:``牙也有了?''

``有,可厉害了,跟刀子似的,一口就见了血。''

顾大人来了兴趣,上前将棉被一掀,随即兴高采烈的嚷道:``嚯!腿也有了!手也长出来了?''他捏起无心的手掌看了看:``幸好还没指甲,否则非得挠人不可!''

月牙忘了疼,凑上前去让顾大人看无心的脸:``你瞧,和原来是一模一样。等到白毛褪了,就更好看了。''

顾大人低头一看,发现面孔的模子的确是一如往昔,鼻梁高了直了,嘴唇也出了棱角,只是眼睛还没有睁,但是眼皮下面隐隐隆起,显见眼珠子也已经长完全了。

顾大人挺高兴,从上看到下,最后掰着无心的一条腿仰天长笑:``哈哈哈,鸡\textbar{}巴蛋都出来啦!''

月牙虽然是个成了亲的小妇人,然而听了他的笑语,脸上一红,还是感觉没法接话。正是尴尬之际,房内忽然起了声音:``饿。''

顾大人的笑声戛然而止,和月牙一起向下盯住了无心。无心的四肢缓缓蜷缩起来,懒洋洋的翻身背对了他们,同时又说一声:``饿。''

月牙轻声开了口:``无心,你饿了?想吃饭了?''

无心答道:``嗯。''

月牙尖叫着欢呼起来。俯身狠狠抱住无心,她在他的白毛脑袋上噼噼啪啪连亲了十几个嘴,又带着哭腔骂道:``小没良心的,饿了你就咬我啊?你等着,我给你做饭去,喂饱了我再收拾你!''

月牙心急火燎的煮了一盆面片汤,里面放了不少土豆和肉。把汤放到院子里晾温了,她端着汤盆进了房。

手托汤盆蹲在床前,她让无心自己凑过来吃。顾大人坐在一旁抽烟喝茶嗑瓜子,笑微微的看着无心把脑袋伸进盆里,不换气的连吃带喝。肚皮很快隆起来了,最后他用舌头舔净汤盆,猛然一口咬住了月牙的手。月牙吓了一跳,紧接着发现他不是真咬,只是牙齿轻轻一合,在吓唬人。

放下汤盆拧了一把毛巾,月牙托着他的脑袋给他擦脸。他的四肢细瘦蜷曲,中间鼓着个大肚皮,肚皮上面白毛稀疏,根根都是东倒西歪;一身的骨骼还没固定形状,肩膀塌着,脖子却是挺长。

顾大人看到此处,心有所感,忍不住向月牙问道:``你说,凭他现在的德行,世上也就咱俩看他顺眼吧?''

月牙虽然爱他,但是基本的理智还有,故而点头表示赞同:``是呗!''

\chapter{蜕变}

月牙站在床旁,一盆热水就放在面前的木凳子上。把衣领解开向内窝去,她披头散发的弯了腰,想要洗洗头发。窗外阳光照在大雪地上,亮堂的刺人眼睛,屋子里的洋炉子烧热了,玻璃上结了一层冰霜。

房门忽然一开,顾大人走了进来。顾大人冻得手脸干冷,乍一进门,迎头便是吸了一鼻子混合着香皂味的潮湿空气,又暖又香的带着水分,很富有一点女性的诱惑力,像是进了澡堂子的女宾部。月牙忙着洗头发,没遮没掩的现出了她的细腰大屁股,后衣领敞得大了,露出一小块粉白的脊梁,肉呼呼的带着一层细汗毛。

顾大人先看月牙,再看无心。无心趴在床边,肩膀胯骨已经长出形状了,身上的白毛却还没有褪尽,一双眼睛也还没有睁开,眼皮薄薄的,隐隐可见里面的大眼珠子。单从眼睛上看,他有点像个人胎。单手拿着一只小葫芦瓢,他舀了热水抬起来,准确无误的浇向了月牙的后脑勺。雪白的泡沫被冲下来,月牙舒服的吸了一口气:``对,再来一瓢!''

无心的细胳膊仿佛是很虚弱,颤巍巍的再来一瓢,手指上的短毛被打湿了,薄薄的指甲透了亮。顾大人上前几步夺过了瓢,一边浇水一边审视着月牙的身段,顺便说了话:``月牙,厨房里怎么什么都没有了?昨天不是还有一筐梨吗?''

月牙侧着脸用干毛巾擦头发:``唉,甭提了,全让他吃了!''

顾大人放下瓢转向无心,而无心虽然四肢细瘦,脊梁骨却是灵活有力。没等顾大人张开嘴,他已经像条大蛇似的游进了床角被窝里。背对着顾大人躺好了,他忽然意识到屁股还露在外面,就向内一拱,彻底消失在了顾大人的视野中。

月牙水淋淋的直起了腰,也是发牢骚:``饭量大得吓人,一个时辰就得喂一次,一次吃一盆。好在是不白吃,不信你摸摸他,骨头可结实了,胳膊腿儿也长肉了。''

月牙从早忙到晚,厨房里总烧着火。一天扫八遍床,每次都能扫出一大团白毛。好容易到了不做饭也不扫床的时候,她盘腿坐在床上,抓紧时间裁剪缝纫。顾大人拿回来的几样好绸缎,颜色新鲜的归她,颜色肃穆的归顾大人;顾大人说不准什么时候就要去见大人物,所以她得尽快给顾大人做几身体面衣裳出来。西装她不敢做,长袍马褂始终是一个老样子,她不用学习就会。而在她穿针引线之时,无心就爬出来枕上了她的大腿。

``又来缠我干啥?''她专心致志的比量着棉线的长短,同时轻声问道:``搭理你,你往被窝里钻;不搭理你,你又自己出来了。''

无心似乎是无法控制太精细的动作,比如说话,就说不利落,声音忽高忽低的不稳定:``我的样子\ldots{}\ldots{}吓到你\ldots{}\ldots{}''

月牙笑了:``哟,还挺疼人的哪?''

顾大人端着一笸箩红枣进来了,无心感觉出了他的身份,十分刺耳的尖叫了一声:``顾大人!''

顾大人吓得一哆嗦,当场把红枣颠出了三枚:``哎哟我的天,你他妈再鬼叫我掐死你!''

无心扯起棉被盖住了身体,改用柔和的男低音寒暄:``红枣甜不甜?''

顾大人把笸箩放到床边,然后弯腰去捡红枣:``可甜了。''

捡起三枚红枣直起腰,顾大人发现笸箩已经不知去向。月牙低头做着针线活,没声,然而笑得满脸通红,露出一口很齐整的牙齿。

顾大人立刻就明白了,对着月牙身边蠕动不止的一团棉被怒道:``你妈×,敢在老子面前吃独食!''

被窝下面出现一条缝,一只苍白的拳头伸出来,瞬间一松手又缩了回去。床上多了五枚干巴枣,枣上还纠缠着几根半长不短的白毛。

月牙忍无可忍,捏着针线笑得前仰后合。顾大人也气乐了。无心现在的动物性很重,非常之馋,所以顾大人决定不和他一般见识。

新年将至,顾大人心心念念盼望的老帅也终于从保定回了天津。顾大人的胖朋友登了门,进了上房和顾大人嘁嘁喳喳。月牙照例是缩在西厢房,扫过床后坐上去,拉着无心的一条胳膊仔细看:``比昨天又光溜不少。''

无心早上自己揉眼睛,揉着揉着竟然揉开了左眼的眼皮。眼珠子见了天日,是一种鲜润的黑白分明。一只眼睛紧盯着月牙,他忽然爬出被窝搂住了她的脖子,低声说道:``月牙,谢谢你。''

月牙摸索着拽起棉被裹住了他。无心太瘦了,外面加上一层棉被,抱起来才刚刚好。两人脸贴了脸,月牙抬手摸了他圆而坚硬的后脑勺,摸下一手的细软茸毛:``也得谢谢人家顾大人。''

无心点了点头,把尖削的下巴搭在了月牙的肩膀上:``嗯。''

月牙又说:``我看你好像一直都认识我。当初把你往床上一放,你就往我身边凑。''

无心答道:``我一直都清醒,只是不能动,能动了,又怕会吓到你。''

然后他力不能支似的弯下了腰,面孔正巧就贴在了月牙的胸脯上:``我还记得我们一起坐了火车。''

月牙抬手一拍他:``坏东西,今天刚穿的新衣裳,又被你蹭了一身毛。往后我可再不抱你了,抱你一次,我得浑身打扫半天!''

无心满不在乎的仰起头,对着月牙一撅嘴,见月牙还是在对着自己笑,他就像只爱撒娇的独眼龙一样,亲了月牙的嘴唇。

两人亲得有滋有味,无心披着棉被,挺身就要抱住月牙往下压,不料正是情浓之际,院内忽然响起一阵欢声笑语,却是顾大人送他的胖朋友走出来了。

顾大人兴致高昂,送走朋友之后便进了西厢房。月牙早有准备,推开无心之后又摸头发又擦嘴;而无心见顾大人走到床边了,并且穿着一身很漂亮的藏蓝长袍,便微笑着扑上去,张开双臂一把抱住了他:``顾大人,谢谢你。''

顾大人猝不及防的被他抱紧了,感觉还怪不好意思的。抬手一指无心的脑袋,他对月牙说道:``舌头比前几天利索多了,是不是?''

月牙没敢提醒顾大人注意无心的毛,顾大人也是早上刚穿的新衣裳,她怕顾大人脾气暴,再把无心揪起来揍一顿。

``是\ldots{}\ldots{}''她犹犹豫豫的答道:``声音也好听多了,前几天说着说着就要叫,让你骂了几次之后,就不叫了。''

顾大人拍了拍无心的后背:``看看,肩膀也长成了,脚趾头也挺齐全。好,算他度过了一大关,又成人了!月牙啊,你跟我出趟门。明天我要见人去了,光着脑袋不好看,你给我做参谋,我得趁早上街买顶帽子回来!''

随即他又低头问道:``师父,你要点什么不要?''

无心放开顾大人,赤条条的跪坐在床上。抬起左眼皮撩了顾大人一眼,他没说什么,只摇了摇头。

顾大人催促月牙穿鞋戴围巾,然后就很潇洒的出门去了。无心蹲在窗前,眼看他们锁好了院门,便伸腿下地,披着月牙的旧棉袄跑了一趟厨房,端回了一盆热水。

手掌蘸水打湿皮肤,他咬牙切齿的用力开搓,搓得白毛一卷一卷的脱落。往昔无人管他的时候,他通常会蠕进土中缓慢成长,及至成长完毕,身上白毛也自然的脱落净了;然而如今环境温暖,营养充足,他成长的速度竟是大大加快,以至于人长成形了,毛却还在。

漫长的洗过一场之后,他光溜溜的站在地上照镜子,并且想方设法的扒开了右眼。眉骨上面呈现了淡淡的青色,是眉毛将要生长出来。无心认为自己如今的模样还算对得起月牙和顾大人,前两个月,也真是难为他们了。

顾大人带着月牙进了帽子行,伙计满面笑容的迎上来招待,三言两语的交谈过后,伙计笑道:``您府上养狮子狗了吧?您等着,我给您掸一掸。''

顾大人一低头,这才发现自己满前襟都是白毛。

买下一顶厚呢子大礼帽,顾大人一出店铺就骂起了无心,月牙想要护短,可是太不占理,有话都说不出口。

顾大人戴着新帽子,月牙拎着两包点心,两人并肩往家里走。打开院门向内一进,两人都愣住了。

无心穿着顾大人的新长袍,站在院子里不知是要往哪屋去。转向院门一笑,他的皮肤白到透明,却又被寒风吹出了一片绯红。

月牙和顾大人都傻了眼,没想到自己只出去了小半天,无心竟然就彻底变成了个漂亮洁净的人模样。

最后,是月牙先笑了,笑得有点害羞,捧着点心不迈步;顾大人则是一拍巴掌,兴高采烈的大声笑道:``好你个老不死的,偷我的衣裳!''

无心看看月牙,再看看顾大人,不说话,得意洋洋的就只是笑。

正是一团喜气之时,一辆汽车响着喇叭开了过来。紧急刹在了院门口,车门一开,里面探出了一张气喘吁吁的大胖脸:``顾兄弟,你回来的正好,我来通知你一声,明天去不成了,老帅家里出事啦!''

顾大人立刻做了个向后转:``出事了?什么事?''

他的胖朋友一边喘一边回答:``他家小少爷生了邪病,快不行了!''

\chapter{不情不愿}

大胖子身为顾大人的挚友,到底也没弄清顾大人身边到底带了什么人。一个小媳妇,一来就能看见,说是顾大人的兄弟媳妇,可是兄弟在哪里,一直不知道;方才从汽车里伸出脑袋,大胖子依稀瞄见院子里好像多了个男人,不过一句话说完,连小媳妇带男人全没影了,就剩了个顾大人,连搀带抱的把他从车里搬了出来。

``苏先生。''顾大人把他往院子里请:``你别忙着走,给我细讲讲,怎么就去不得了?''

苏先生挪动两只穿着皮鞋的小胖脚,肉球一般温文尔雅的往院里滚:``唉,本来一切都说妥当了,可是老帅家的小少爷不知怎的就生了病,起初全以为只是伤风感冒,哪知一天重似一天,医院也进了,中药西药也都吃过了,可是全无效果。都说小少爷头些天曾经跑进花园子里玩过,花园子太大,不干净,兴许是撞客了,老帅就请了高僧老道过去做法驱邪,然而忙了好几天,还是不成。今天我往帅府里打了电话,听说小少爷虽然还有气,但是身体都冷了;你想老儿子素来是最招人疼,老帅眼看要保不住小少爷了,还能有心思提拔你吗?他根本就不能见闲客啊!''

顾大人推门请苏先生进了上房,然后若有所思的吆喝月牙沏茶。隔着一张桌子和苏先生相对落座,他等月牙送过热茶了,才迟疑着说道:``苏先生,不瞒你说,我倒是认识一位真有力量的法师,不是道听途说,是我亲眼见识过。问题是\ldots{}\ldots{}不知道能不能请动他。''

苏先生眼睛一亮,倒是笑了:``最好是能请动,而且要快请。你要是能救了老帅家的小少爷,老帅怎么着还不得给你个一官半职?法师在哪里?你可以坐我的汽车去。''

顾大人沉吟着笑了笑:``不用坐汽车,他人就在天津,我找他倒是容易之极,只是他肯不肯帮忙,我就不确定了。''

苏先生见他含含糊糊的不说明白话,猜出他可能是有难言之隐,所以也不追问,起身说道:``我还要去马总长家里打小牌,一旦你这边有了眉目,就可以去找我家的听差,他们总能知道我的下落。''

顾大人连连点头,恭而敬之的把他送出门去,推上汽车。及至苏先生的小汽车走远了;他转身往西厢房走,正赶上月牙换了旧衣裳推门出来。两人迎面相遇,月牙问他:``中午还是熬白菜,行不行?''

顾大人一把拽住了她,把她牵进了厨房里:``跟你说件事。''

月牙莫名其妙的抽回了手,弯腰从角落里抱起一棵大白菜:``啥事?说吧!''

顾大人压低声音答道:``我不是一直等着老帅回来吗?现在老帅回来了。''

月牙一边听一边撕去白菜外层的老叶子,没听明白。顾大人见她一脸懵懂,便继续说了下去:``老帅家的少爷好像是中了邪,马上就要嗝屁,你说我要是把他救活了,老帅还不得高看我好几眼?''

月牙伸手对着门外一指,声音也轻了:``你想让他去啊?''

顾大人又道:``月牙,我是什么人,你应该清楚。我要是发达了,能落下你们吗?我打算这就去跟他说,让他出手帮忙,他要是不愿意,我就吓唬吓唬他。你乖乖熬你的白菜,要是听见屋里有动静了,也别过去跟着瞎掺合,你放心,我不能真揍他!''

月牙想了想,一颗心悬起来不落地:``他刚长好\ldots{}\ldots{}我刚才看他耳朵眼里还有白毛呢,一身皮肉也嫩得像水豆腐似的,能出门吗?''

顾大人嗤之以鼻的一挥手:``我又不是让他卖肉去,毛不毛嫩不嫩的有什么关系?反正你别管,我有分寸。我要是攀上老帅了,将来有了钱,肯定亏待不了你们。做你的饭吧,白菜里面多切点五花肉,我可吃不了素!''

月牙知道顾大人利欲熏心,想当官发财都要想疯了,对自己和无心又一直挺仗义,所以也想让他高升一步;不过无心刚刚成人,到底有没有本事,自己也不知道。缓缓的切着大白菜,她竖起两只耳朵听动静。

顾大人大步流星的进了西厢房,见无心伸长双腿坐在大床上,正在吃月牙拎回来的甜点心。抬头望着顾大人,无心含着点心抿嘴一笑。

顾大人站在床前,将他细细的又打量了一番,发现和正常人相比,他还是有点小区别。首先眼眶太大,其次脸皮太薄,最后缺乏眉睫;显然,他还得再长几天才能完全恢复原样。

一屁股坐到床边,他笑面虎似的转向无心:``师父,我有件事,要麻烦你。''

无心嚼得面颊一鼓一鼓,太阳穴处可以看见皮下的血脉,血脉宁静,是若有若无的一线蓝:``说。''

顾大人把鞋脱了,近距离的凑到了无心身边,又抬手搂住了无心的肩膀,很亲热的笑道:``我来天津呢,是想巴结一位大人物。现在大人物的小儿子撞了邪祟,你能不能过去斩妖除魔,给我个机会攀高枝?''

无心缓缓咽下了口中的点心,皱起眉骨上的两抹青黑:``我\ldots{}\ldots{}我还没长好呢。''

顾大人正色说道:``谁说你没长好?你现在和人是一模一样,扒光了都看不出区别来!''

无心蜷起一条腿,扯了裤管给顾大人看:``我的毛还没有褪干净\ldots{}\ldots{}''

顾大人一瞪眼睛:``我又没让你真光着去!当初你上我家捉鬼的时候,我检查你有没有毛了吗?''

无心放下裤管伸直了腿:``我连眉毛都没生出来\ldots{}\ldots{}''

顾大人一摆手:``没事,怪模怪样的显得更神秘莫测!一般有大本事的人,都比较怪!''

无心把手臂环抱到胸前,畏寒似的瑟缩了:``顾大人,我不想去。我自从捉鬼就没落过好,先是丢了半个脑袋,后是剩了一只手。如今好容易又长全了,我打算转行去算吉凶或者看风水。要是再招惹来一个岳绮罗,好日子就又过不下去了!''

顾大人一听此言,登时急了:``好哇,当初你像条大蛆似的,人见人嫌,是谁天天照顾你?我告诉你,我就算是你重生的父母再造的爹娘,你不听我的,就是属核桃的欠打,属黄瓜的欠拍!''

无心满不在乎的答道:``我就不去。''然后把一整块点心全塞进了嘴里。

顾大人一挺身下了床,转过来伸手一指:``无心,你敢不去,我真抽你!''

无心无动于衷,一边咀嚼一边歪着脑袋掏耳朵,掏出一团白毛。

顾大人的手指变了方向,隔着房门瞄准了厨房:``不信是吧?我不揍你,我还不能揍月牙去?别跟我扯什么好男不跟女斗的屁话,我除了爹娘不打,对谁都能下手!''

无心把白毛弹到了地上:``你要是打了月牙,我就更不能去了。''

顾大人见他不吃硬的,当即决定改变战术:``无心,你真不去?!''

无心摇了摇头:``不去。''

顾大人瞬间变了语气,双手合十对着无心拜了一拜:``心哥,别跟我犯倔呀,别人的面子你不给,我的面子你还不认吗?''

无心噎着了,很徒劳的一边咽唾沫,一边抬眼看着顾大人。

顾大人见他不说话,只好进一步放低了身段:``心爷,我将来若是有了起色,还能亏待你们两口子吗?远的不说,等我得了好处,先给月牙买一副钻石坠子,怎么样?够大方吧?''

一整块点心被嚼得半烂不烂,堵在无心的喉咙口。他倒是憋不死,然而干张嘴发不出声。顾大人俯下身,抱着大拳头对他一拱一拱,语言越发甜美了:``心肝,发发慈悲吧,我顾某人也是老大不小的年纪了,再没有起色的话,一辈子不就耽误了?''

无心攥了拳头,一拳击向自己的胸膛。只听``咕噜''一声,点心下去了,他终于说出了话:``你也可以去找出尘子。''

顾大人听他口风松动了,心中登时一喜:``你懂个屁啊,肥水不流外人田!''

无心是真不想去,然而又不能不去。顾大人的前程当然是要紧的,他也希望顾大人能有个升腾。

月牙做好了饭菜,热气腾腾的摆上来。三人围着圆桌坐了,顾大人得了无心的答复,心中喜悦,忍不住挥着筷子高谈阔论,认为自己一肚皮雄材伟略,只要有了机会,就必能成就一番事业。

``区区丁大头和张小毛子就能把我打败了?''他边吃边说,洒得满桌都是白菜汤:``我顾玄武从小就不是怂货!''

月牙没吭声,从汤里捞肥肉片吃。无心扫了他一眼:``哦,原来你大名叫做玄武。''

顾大人洋洋得意:``怎么样?本来我叫顾石头,听着不够体面,所以当了司令之后,我就花了点钱,请县里的老先生给我起了个新名字!''

无心点了点头:``玄武\ldots{}\ldots{}就是乌龟嘛!''

顾大人张了嘴:``啊?''

随即他转向月牙:``月牙,玄武是乌龟吗?''

月牙立刻摇了头:``我连玄武俩字咋写都不知道。''

顾大人又面向了无心:``你别跟我开玩笑啊!''

无心往米饭里倒了半碗白菜汤:``爱信不信,乌龟背上趴条蛇,合起来就叫玄武。''

顾大人把筷子往桌上一拍:``操!老不死的还要了我一块大洋哪!''

无心吃了一口汤泡饭,随即转移了话题:``要去帅府,我得换身衣裳。法师就得有法师的样子,顾大人,你去给我弄套僧袍回来,颜色样式都不拘,但是料子一定要好。''

说到此处,他的大眼珠子在大眼眶里从左转到右,把顾大人和月牙尽收眼中:``人靠衣裳马靠鞍,要说装模作样,我可是个行家!''

\chapter{法师的派头}

顾大人吃了一肚子白菜猪肉大米饭,一抹嘴就起身出了门。傍晚时分他回了来,腋下夹了个小衣裳包,两道浓眉都上了霜。

寒风凛凛的进了西厢房,他把小衣裳包扔到了床上:``说好了,明天有汽车过来接我们去帅府。喏,你要的和尚袍子,看看合不合身,要是尺寸不对,赶紧让月牙再给你改一改!''

无心如今只有吃睡两桩大事,早早就钻进了被窝里。月牙正站在地上擦桌子,此刻就一边掀起围裙擦着湿手,一边走过来解开包袱看新鲜。

``哟!''她抖开一件黑色僧袍:``真是好料子,沉甸甸的厚实,不熨都没褶子。''然后她欢喜的对着无心招手:``过来试试,我有日子没见你扮和尚了!''

无心懒得动,蜷在被窝里半闭着眼睛答道:``不用试,一看就合适。''

月牙放下僧袍继续翻包袱:``呀,还有一件斗篷哪?''

顾大人抬手搓着眉毛上的霜:``他不是要装大法师吗?我得给他把好衣裳预备全了啊!就算不和出尘子比吧,也得比一般和尚强不是?''

无心虽然大致看着是个人样子了,其实细微处的生长并未停止,导致他无精打采的不是饿就是困。哈欠连天的坐在小板凳上,他把双手搭在膝盖上,两只眼睛闭得严丝合缝。月牙烧了一盆热水,打湿了毛巾为他用力擦腿,想要提前蹭下要脱未脱的白毛。顾大人握着镊子蹲在一旁,为他拔净耳朵眼里的白毛。无心被这两个人搞得心烦意乱,又没法挑剔,只好默默的忍受。身体越来越向前倾,最后他把下巴抵上月牙的肩膀,彻底睡着了。

顾大人低头对着镊子吹了一口气,吹下几根透明的茸毛:``你看他这个德行,真让我不放心。万一明天在帅府出了洋相,我可就没活路了!''

月牙扛着无心的脑袋不敢动:``就好像他愿意跟你去帅府似的,还不是你非逼着他去?''

顾大人叹了口气,心中七上八下的直打鼓。惴惴不安的回房睡了一夜,翌日清晨他早早起床,就见院子里贴地卷了一层炊烟,正是月牙已经在厨房里开了伙。

连忙穿戴洗漱了,他换上一身素净的长袍马褂,拿着厚呢子大礼帽走进院内:``月牙,做饭呢?''

厨房里传出了月牙的声音:``马上就熟,不耽误你们出门。无心也起来了,你进屋等着吧!''

顾大人捏着帽子进了西厢房,推门一步迈进去,正和无心打了个照面。无心已经穿上了僧袍,僧袍隐隐反射了阳光,随着瘦削身体的棱角垂出线条。僧袍乌黑,里衣雪白,衬得无心一张面孔洁净鲜嫩之极,薄薄的皮肤下面,甚至透出了青红血脉。一夜的工夫,他的眉毛也生出来了,大眼睛陷在眼窝里,带了一点阴森森的鬼气。

顾大人忽然看出了一个问题:``少了一串佛珠,昨天忘给你买了。''

无心走到桌前坐下,自顾自的给自己倒了一杯热茶:``没关系,少个一样两样也不算什么。顾大人,等到进了帅府,你没事别和我说话。''

顾大人一怔:``为什么?''

无心找出昨天吃剩的点心,亟不可待的就着热茶往嘴里送:``你听我的就是了!''

两人吃了月牙预备的米粥和馒头,院外响起了汽车喇叭,正是到了出发的时刻。月牙给无心掸了掸身上的馒头渣子,又展开斗篷给他穿上。无心一边受着月牙的伺候,一边叹了口气:``唉,一百来年没做过正经和尚了,经都不会念了。''

月牙在他后背拍了一巴掌:``听你说话就瘆得慌!早去早回,少惹闲事,别跟着顾大人胡闹,听见没有?''

无心乖乖的答应了,又把斗篷后面的风帽掀起来扣在了光头上。风帽大了一点,帽沿遮住了他的眉眼,顾大人也戴上了他的大礼帽,两人就一前一后的出了远门。

汽车果然停在了胡同里,一名戎装打扮的青年副官下了车,正在车旁来回踱步。忽然看到有人推门走出来了,他连忙上前几步问道:``请问您是顾先生吗?''

顾大人立刻一点头:``正是。''

副官又上下打量了顾大人身后的和尚,因为不见眉眼,只见嘴唇下巴,所以感觉对方神秘至极,一时竟是没敢贸然相问,只对顾大人又笑了一下。而顾大人当即会意,一派和气的又道:``他就是我所说的法师。''

副官连连点头,侧身伸手向汽车做了个``请''的动作:``两位请快上车吧,老帅一宿没睡,现在就等您二位了。''

顾大人不敢怠慢老帅身边的人,即便只是个副官。而无心则是一言不发,随着顾大人就钻进汽车里去了。

汽车一路驶出胡同,拐上平坦大街。无心扭头望着窗外风景,心中暗暗惊叹,没想到世界竟是变化如斯。而顾大人笑眯眯的同副官交谈不止,把帅府内的情况摸了个清清楚楚——现在府里已经开始给小少爷预备后事冲喜了,昨天国务总理从北京给老帅打来了长途电话,让他去长安县青云山青云观,找观里的住持道长来瞧一瞧;老帅不认识青云观里的住持道长,有点不大相信,所以就还没有真派人去。

顾大人知道出尘子是真有本领的,所以隔着斗篷一戳无心的大腿,意思是让他打起精神,千万别给出尘子登场的机会。无心扭头看了他一眼,依旧是不言语。

汽车跑得又稳又快,不过片刻的工夫,便在一处公馆门前停住了。无心远远望去,发现如今的房子和先前大不相同,全是洋灰砖石所砌,别有一种怪模怪样的巍峨。而顾大人伸着脖子从挡风玻璃向外望,遥遥就见公馆门前站了一大群人,其中一位体积不凡,正是自己的好朋友苏先生。

汽车停稳之后,前方副驾驶座上的副官先下了汽车,特地为顾大人和无心打开了后排车门。顾大人先落了地,因见苏先生都是一脸肃穆,所以立刻紧张起来。等到无心也出来了,他手足无措的走向人群,笑也不是不笑也不是,而苏先生善解人意,对着身边一位小个子军人说道:``老帅,他就是顾玄武。''

老帅看着不算很老,不过是五十来岁的模样,干巴瘦挺精神,蓄着德皇威廉式的翘胡子,眼睛不大,眼珠子却是犀利有光。顾大人生平第一次看见活的老帅,先前预备好的满腹寒暄瞬间全部化为乌有,话也不会说了,慌里慌张的就是一鞠躬,额头差点没顶到老帅的下腹。而老帅对他只一点头,随即就把目光射向了无心。

无心抬手向后推下风帽,双掌合十微微的一低头,声音低沉的诵了一声:``阿弥陀佛。''

老帅也合掌答了一句阿弥陀佛,然后便急切的说道:``法师,您先请进。''

无心不理旁人,一甩袖子向前走去,随着老帅率先进了大门。老帅的胡须疏于打理,一边翘着一边垂着,随着他的言语一颤一颤。仰脸望着无心,老帅一边描述爱子情形,一边暗暗的犯疑心,怎么看无心都不像一位得道高僧。不像高僧,可也不像江湖骗子,到底像什么,老帅也说不出来。

帅府院内是一大片空地,正中央砌着高大喷泉,冬季天寒,喷泉干涸,可见洋灰池子里面的高低水管。喷泉之后是一座大洋楼,窗子嵌着五彩玻璃,看着很是摩登。老帅唠唠叨叨的一直说,无心带听不听的欣赏洋楼,及至欣赏够了,他淡淡的问了一句:``令郎在哪里?''

老帅连忙向前一指:``就在楼内。''

无心背了双手,就感觉帅府之内魂魄骚动,阴气颇重,败坏了府中美丽的建筑。

老帅急得有些颠三倒四,东一句西一句的说道:``对了,还没请教师父的法号\ldots{}\ldots{}''

无心径直向楼内走去,同时头也不回的答道:``无心。''

无心和老帅步伐矫健,一马当先的进入楼内,后方的副官家人幕僚等等蜂拥跟上,统一的肃然安静。顾大人搀扶着球似的苏先生,落后一步,也极力的追了上去。高人一头的站在后方,他就见无心停在铺了波斯地毯的楼梯口,竟然抬手解下斗篷,坦然的交给了一旁的老帅。老帅显然也是愣了一下,不过随即接住斗篷,没敢出声。

无心仰头闭上眼睛,右手从左边袍袖里抽出了一条黑色布带。抻直布带向上蒙住双眼,他对着身边的老帅一挥手,轻声说道:``跟我来。''

然后他准确的踏上一级台阶,一步一步向上走去。老帅把他的斗篷搭在臂弯上,亦步亦趋的跟着上楼了。

楼下一大群人犹犹豫豫的不知该不该继续尾随。法师并没有一脚踏空滚下来,老帅也是避猫鼠一样大气不出。眼看他们一前一后的转了弯,苏先生扭头对着顾大人一挑大拇指:``真高人啊!''

顾大人捧着大礼帽,对着苏先生张口结舌,心想无心肆无忌惮的摆谱,万一救不活小少爷的命,会不会被老帅活嚼了?

\chapter{救人一命}

无心觅着魂魄的微光行走,如果楼内真有力量强大的鬼魂作祟,其余零散魂魄少不得要受影响。鬼魂作为阴气的源头,自然会吸引魂魄前去汇聚;而无心就凭着对魂魄流动的感知,一步一步的走上了二楼。

他不说话,老帅拿着斗篷跟在斜后方,惶惶然的也不敢说话,唯有胡须尖梢未上胶水,随着他的步伐一颤一颤。无心越是前行,他的眼睛越是睁大;及至无心在一扇房门前停住脚步了,他的小眼睛里透出光芒,干黄面孔上也有了喜色——房门背后,便是小少爷的卧室!

``法师高明!''老帅轻声细气的发出赞美,怕扰了法师的神通:``犬子在里面都躺了许多天了。''

无心没理他,背了双手定一定神,随即只听``咣''的一声,他一脚就把房门踹开了!

老帅吓了一跳,房里守着的两个大丫头也惊叫出了声音。靠墙摆着一张西式大铜床,床上躺着个十来岁的小男孩,倒是八风不动的很安静。无心杀气腾腾的大踏步走进去,停在床前微微低头,脑海中已经浮现出了鬼魂的形象。

大凡一个人成了鬼,既然没有修炼成煞,实体不存,似乎也就无所谓形象;不过鬼魂若是怨气极强,也能显出依稀的幻影;幻影多是它死时的模样,因为死时极痛苦,印象极深刻,痛苦深刻到它留恋着不肯魂飞魄散转世投胎,成了幻影也还停留在最难忘的一刻。

无心见过无数鬼魂,十成里有九成九都恐怖的没法看,可床上这一位却是百里挑一,整整齐齐的居然挺有人样,除了披头散发面色青紫之外,其余部位都算利落,五脏六腑也全揣在肚子里。若看衣着打扮,甚至称得上华丽富贵,一双三寸金莲套着红缎子睡鞋,小脚精致,鞋也漂亮。歪在床上摆了个悠然自得的姿势,她一下一下抚摸着小男孩的身体,同时翻起眼睛,望向了床前的无心。

无心没有解下眼上的黑布带子,可是已经和女鬼对视了。女鬼微微瑟缩了一下,无心则是开口说道:``走吧。''

女鬼不动,继续慈爱的轻拍小男孩。

无心又道:``冤有头债有主,别缠着孩子出气。孩子阳气弱,顶不了多久。''

女鬼终于有了回应:``我也是看着他长大的,很喜欢他。他若是死了,来世正好换个人家。给猪狗不如的老畜生做儿子,将来是要受报应的。''

无心发现女鬼的声音悠悠扬扬,仿佛先前曾是个唱曲的,带着一点动听的韵律:``人各有命,不干你事。想报复他就直说,不必另找借口。他一身杀气,你奈何不了他,所以在家里挑软柿子捏,对不对?''

女鬼的面孔隐隐有了变化,眉目之间缭绕了凶气:``我奈何不了他,还奈何不了旁人吗?''

无心摇了摇头:``少废话,赶紧滚,否则我让你魂飞魄散!''

女鬼仿佛听了笑话,当即仰天长笑:``哈哈哈哈哈\ldots{}\ldots{}''

无心把头伸到大床上方,忽然发现女鬼是个大嘴叉子,方才她一直低头说话,两边又垂着长发,居然遮住嘴角,伪装小嘴。暗暗的把一根手指送到嘴里,他狠心一咬,随即对着大笑女鬼弹出了一指头鲜血。女鬼正在哈哈,冷不防的受了袭击,大笑``哈嗷''一声戛然而止,幻影瞬间消失的无影无踪。

无心没有感受到新的魂魄,认定女鬼是逃走了。女鬼一走,房内阴气立时弱了许多。抬手扯下眼前的黑布带子,他趁着指尖鲜血尚未凝结,一指点上小男孩的眉心,神情肃穆的乱画一气,画得小男孩面如花猫。

收回手指吮了一下,他沉着脸转向老帅。老帅和两个大丫头方才见他对着虚空自言自语,因为多少是知道一点内情的,心里有鬼,所以全都惊恐的张大了嘴。无心刚刚赶走了一只大嘴女鬼,回头一看,迎面又是三张大嘴,不禁一皱眉头。

他一皱眉头,老帅和两个大丫头立刻就把嘴合上了。老帅上前一步,陪着小心问道:``法师,如何了?''

无心垂下眼帘,并不看人:``府上有鬼。''

老帅立刻瞪圆了眼睛:``啊?''

无心继续说道:``女鬼,缠住了令郎,现在已经被我驱了出去。我在令郎脸上划了保命的血符,请老帅速把令郎移到阳气重的地方休养,一个月内不许洗脸。''

老帅像个跳蚤似的,因为又惊又喜又怕,所以在无心身边左摇右晃:``请教法师,什么地方阳气最重呢?''

无心见卧室十分宽敞,便随口答道:``男人多的地方,阳气就重。如果老帅不愿挪动令郎,也可以在房内多添些人,最好是凶狠之徒,手上有人命的就更佳了。''

老帅连忙对着大丫头挥手:``快去找副官长,让他挑一帮不老实的小子过来!听见法师的吩咐了吧?原样告诉副官长,快!''

两个大丫头趁机一起逃走,而老帅回头再看小儿子,就感觉房内的气氛有所变化,虽然儿子一脸血,看着比先前还惨,可是阳光暖洋洋的照进来,儿子的小脸仿佛又透出血色了。

喜上眉梢的转向无心,他抱着斗篷正要开口,不料无心双手合什微微一躬,只给了他一个倨傲的侧影:``阿弥陀佛,令郎既然性命无虞,贫僧也就告辞了。''

话音落下,他昂起头,迈步走向门口。老帅慌忙追上了他,先是对着簇拥在楼梯口的一群人嚷了一嗓子,让他们赶紧去把家庭医生叫过来看护小少爷;随即转向无心笑道:``不不不,法师你可不能走。你救了犬子一命,我必要重谢才行。''

无心背对着他,抬起手轻轻一摆:``不必。出家人不贪财,若说救苦救难,凭我一人之力,也救不得许多。贫僧今日之所以肯来,全是受了顾玄武所托。老帅想谢,就谢他吧。''

说到此处,他忽然侧过面孔,轻声说道:``府上如今不干净,老帅夜间不要单独外出。''

老帅一把薅住了他的大袖子:``法师,神仙,帮人帮到底,送佛送到西,你先别走!''

无心被他拽得走不动,只好在楼梯口停了脚步。众目睽睽之下,他十分冷淡的对老帅说道:``今日贫僧累了,纵有邪祟要除,也是明日之事。一夜之内,料它也不能做出大乱。''

老帅不敢得罪他,立刻松了手。身后走廊响起一阵脚步声音,正是家庭医生们先副官长一步,从侧面楼梯跑上来了。而老帅灵机一动,有了主意,对着人群喊道:``顾玄武,你过来陪着法师先歇一歇,我还有好些话要和法师说呢!''

老帅谁也不理,单把无心和顾大人请到一间温暖的小客室内,笑眯眯的递烟奉茶,十分和蔼可亲。顾大人攥着两手汗,不敢和老帅平起平坐;无心则是坦然坐在了软沙发上,一屁股陷下去时,他不动声色的吓了一跳。

老帅闲话没说几句,一名副官忽然掀帘子进来了,说是小少爷已经有了知觉。老帅爱子心切,连忙起身上楼。客室里面落了清静,顾大人见无心一派坦然,就有点自惭形秽,低声说道:``哎,我是不是有点小家子气了?''

无心盯着茶几上的玻璃糖盘子:``你是打算投在人家手下做事的,姿态当然要低一点;我就不一样了,我不端起架子,他也未必把我放在眼里。顾大人,盘子里红红绿绿的都是什么?''

顾大人答道:``水果糖你都没见过啊?''

无心扭头向门口望了一眼,随即伸手抓了一把,飞快的藏到大袖子里去了。

顾大人见老帅迟迟不归,也从香烟筒子里抽出一根烟卷点燃了。烟是好烟,闻着都香;顾大人自从进了帅府就紧张,如今深吸一口长吁出去,他心里舒服了不少。

无心抽了抽鼻子:``哎,给我也尝一口!''

顾大人走到沙发近前,把烟卷送到无心面前。无心凑上去深吸了一口,然后颇为销魂的呼出烟雾:``好烟。''

顾大人低声说道:``等我发了财,好烟好糖要多少有多少,肯定亏待不了你。''

顾大人刚刚吸完一根烟卷,老帅就跑回来了,臂弯上还搭着无心的斗篷。

``法师!''老帅仿佛要哭,脸上纵起皱纹无数:``您是真高哇!我儿子睁眼睛了!''

无心微笑点头:``令郎本来也没有疾病,只是受了鬼魂纠缠。鬼魂一去,他自然就会慢慢恢复健康。''

老帅坐了下来,抱着斗篷说道:``可是您说鬼魂还在我家里\ldots{}\ldots{}怎么着才能永除后患呢?''

无心半闭了眼睛,沉默半晌之后问道:``老帅最近,有没有杀过女人?''

老帅看了顾大人一眼,顾大人很识相一鞠躬,马上蹑手蹑脚的走了出去。等到顾大人关好了房门,老帅长叹一声:``不瞒法师,我家的十二姨太是个骚娘们儿,妈的进了家门就不老实,专和副官们狗扯羊皮,还总和我吵。我前几个月一生气,让人把她给埋了。''

无心面无表情:``怎么埋的?''

老帅理直气壮的答道:``用棉被一裹,再拿绳子一捆,在花园里刨个坑就埋了。''

无心缓缓的点头:``哦\ldots{}\ldots{}活埋。''

老帅跟着点头:``对,是活埋。我年纪大了,脾气也好了,一般不爱动刀动枪。''

无心彻底闭了眼睛,心想和老帅一比,顾大人都是心慈面软的好人了。

``今天时辰不对。''他语气飘然的告诉老帅:``明晚我来试上一试。十二姨太的煞气很重,是否能够斩草除根,贫僧也不能够肯定。不过既然贫僧和顾玄武有缘,顾玄武又对老帅百般崇拜,所以贫僧必定勉力一试。''

老帅骂道:``好个臭婆娘,做了鬼还要和我捣乱。法师,明晚就明晚,如果真能铲除了臭娘们儿,您说怎么样,我就怎么样!''

无心站了起来,不再说话,只是微微一笑,又向老帅伸出了一只手。老帅福至心灵,立刻就把斗篷展开,亲自给他披上了。

无心带着顾大人坐上了帅府汽车,一路赶回家里。路上两人都不说话,及至在胡同口下汽车了,顾大人才发问道:``咱们怎么回来了?难得老帅高看你,你就应该留在帅府不走。''

无心戴上斗篷风帽,大踏步的往家里走:``你懂个屁!我留在帅府不走,月牙怎么办?再说我赤手空拳就能捉到鬼了?我不得做点准备吗?''

两人且说且走,敲开院门回到了家中。月牙很高兴:``哟,回来的还挺早!''

无心冻透了,一马当先的冲进了西厢房。在月牙的帮助下脱了斗篷,他对着床上一甩袖子,很快乐的笑道:``嘿嘿,月牙吃糖!''

\chapter{诡魂}

月牙下午从胡同口买了两条奇大的鲤鱼。鲤鱼已经冻成半死,被她摔在案板上刮鳞剖腹,洗刷干净之后丢进大锅中。及至一锅的鱼汤都熬白了,顾大人用抹布垫了锅耳朵,把大锅一路端进西厢房内的小炉子上。紧随其后的是无心,无心双手捧着一把碧绿的香菜末,顾大人把锅一放好,他就凑上去松了手,把香菜末尽数洒在沸腾的鱼汤里。

月牙殿后,将蒸好的大米饭运了进来。大刀阔斧的盛出三大碗饭,三个人或站或坐,围着大锅开始吃喝。

藉着炉膛里的一点火力,鱼汤始终是咕嘟咕嘟的滚热。顾大人吃得顺脖子淌汗,脑袋上面快冒热气。无心静悄悄的蹲在炉子旁边,伸下筷子一抄锅底,撅起了一截肥美的大鱼尾巴。连汤带水的把鱼尾巴夹到碗里,他一舌头伸出去,舔下了一层肉。

顾大人看见了,立刻有了话说:``月牙,你往后就等着吃亏吧!妈的他专吃独食!''

月牙津津有味的吮着一个大鱼头,咂得啧啧出声:``唉,我还能跟他抢嘴?''

无心面红耳赤的对着面前二人一笑,毫无诚意的表示羞愧,脸上还粘着一根大鱼刺。

三人大规模的吃了一顿晚饭,顾大人虽是一条身强力壮的好汉,可也撑得动不得,歪在床上直打饱嗝。无心鼓着大肚子,若有所思的用手指在墙上画来画去。月牙从厨房回来了,进门之后剥了一块水果糖含到嘴里:``剩了一锅底的鱼汤,明天揪点面片放下去,还能煮一大锅。''

月牙的思想比较简单,烧饭洗衣便是她的天职。从早到晚做足了一天的家务,她累归累,然而心安理得、心满意足;只可惜现在还没有一个固定的小家,将来有了家,家里也不会再有一群儿女,和她理想的生活总有差距。

月牙对于远景是心如明镜,所以不肯细致的深想。想也白想,她拧了一把热毛巾,知道自己是无论如何都离不开无心。把无心拽过来摁下脑袋,她给他擦净了一头热汗。无心垂下双手低下头,乍一看很乖,其实暗地里用大肚皮不住的向前去顶月牙。月牙站不住,连退几步之后给了他一巴掌,又气又笑的骂道:``欠揍啊?''

无心一弯腰,一言不发的扑在了月牙胸前,软绵绵的立不起推不开。月牙无可奈何的用脸蛋蹭了蹭他的光脑袋:``不要脸的,又跟我赖上了!''

顾大人在床上嗤嗤的笑,笑着笑着感觉有点不是味儿:``唉,等我有了着落,也该成个家了!''

月牙甩不开无心,惊讶的扭头望向顾大人:``你原来不是有五个小老婆吗?''

顾大人长叹一声:``一百个又怎么样?一看我失了势,妈的全跑没影了!''

顾大人不愿意回东厢房睡觉,认为留在西厢房更有意思。把藏在旧棉袄里的几张纸符取出来,他和无心从中拣出几张,预备着明天用来降妖除魔。无心从上房翻出一份纸笔,凭着记忆画了满篇古怪图案,又向顾大人问道:``怎样才能给出尘子送一封信?''

顾大人答道:``出尘子又没住在深山老林里,直接从邮局邮过去不就行了?''

无心恍然大悟,然后把画好的一篇纸折好递了出去:``明天你出趟门,把它寄去青云观。''

顾大人展开纸看了一遍:``什么玩意?''

无心笑道:``我也不知道是什么玩意,出尘子大概懂一点,让他钻研去吧!''

顾大人又道:``明晚到了帅府,你得多卖力气多摆架势,让老帅知道你不容易,顺带着也多记我一点功劳!''

无心不屑的一扬头,披着棉袄出了门。月牙撩了他一眼,以为他是去解手。不料等了半天也不见他回来。颇为担忧的转向顾大人,她开口说道:``你出去瞧瞧,天黑,他是不是掉坑里了?''

顾大人不情愿的推门往外走:``他眼神比野猫都好使,上个茅房还能掉坑里?''

片刻之后,顾大人带着一身寒气回来了:``月牙,你自己去厨房看看吧,你男人把剩下的鱼汤和饭拌了一锅,正吃着呢!''

无心夜以继日的大嚼,翌日又睡起了懒觉,直到中午才起。下床之后他被月牙逮了住,月牙比量了他的身材,发现一夜没留意,他竟然长高了大半寸,模样也有所变化,不但面颊丰润了许多,头上也生出了一层短短的黑发。

``呀!''月牙很惊喜的在他头皮上摸了一把:``上边的毛都长全啦?''

无心弯腰脱了裤子:``下边也长全了。''

月牙当即在他腿间轻轻拧了一把,无心向后一躲,紧接着就开始向月牙讪脸。两人滚在床上嘻嘻哈哈,越厮闹越亲热,越亲热越黏糊。而顾大人坐在东厢房,鹅似的抻长了脖子,还在望穿秋水的等着月牙喊他吃午饭。

冬季天短,不知不觉就暗了天色。无心刚刚换上僧袍斗篷,院外就响起了汽车喇叭。把纸符尽数揣进袖子里,无心正了正神色,然后跟着顾大人走了出去。

汽车一路开得风驰电掣,转眼间便到了帅府门前。老帅换了一身便装,照例是亲自出面迎接法师。无心下车走到了他的面前,并不询问小少爷的情形,只合掌一礼:``阿弥陀佛。''

暮色苍茫,老帅恭而敬之的回了礼:``法师,今天要请您大展神通了。''

无心轻声说道:``不敢当,贫僧并没有胜算。''

他越是呛着老帅说话,老帅越觉得他是真高人、有性格。抛下顾大人进了帅府,他发现府内空气森然,处处都有全副武装的士兵站岗。杀气能克阴气,无心缓步向前,因见老帅稳稳当当的跟着自己,并没有畏缩恐慌的表现,就知道对方的胆量真比顾大人高了好几级,也许老帅只是不知道克鬼之法,否则就亲自上阵了。

老帅客气起来是真客气,柔声细语的邀请法师先进楼内歇息一阵;无心没有和老帅拉家常的打算,所以干脆谢绝,直接请老帅引路,带领自己前往后花园埋尸地。

老帅一手按着腰间的盒子枪,且走且问:``法师,家里预备了黑驴蹄子黑狗血,下午还现抓了几只大公鸡。朱砂桃木剑之类的也都有,您用不用?''

无心摇了摇头,同时心中生出了奇异感觉——帅府太干净了,一路上连游魂散魄都没遇到,同昨日情形大不相同。若说十二姨太是一吓就跑,他不大信;况且十二姨太的尸骨还在花园子里,她一个新鬼,又能游荡出多远?

绕过大洋楼,无心跟着老帅又穿过了一片亭台楼阁,身后尾随着长长一条卫队。末了停在团团围起的一片残花败柳之间,卫队里出来几名士兵,不知从何处牵出了电线,开始往树枝上面架电灯。无心低头一瞧,见地上还露着白生生的树木根茬,可见脚下的空地是新开辟出来的,先前埋人的时候,此地大概还是草木葱茏的荒凉之处。

电灯通了电,方圆几十米内都被照了个通亮。副官长扛着铁锹走上前来,对着无心前方的地面一指,低声问道:``法师,现在就开挖吗?''

无心听他声音微颤,便猜出当初他是老帅命令的执行者,此刻必定心惊肉跳。不动声色的瞟了老帅一眼,他发现老帅盯着副官长的铁锹,神情倒是自若得很。

对着副官长一点头,无心随即后退了几步,让出了空地。

副官长带着几名卫士下了铁锹。时值寒冬,土地都冻硬实了,一锹铲下去,只能挖起一点浮土。副官长挖的不得力,命人找来了铁镐,帽子也摘了,领着头的猛刨一气。刨到了一定程度再换铁锹,副官长一锹插下去,拔\textbar{}出来时骤然向后一跳,黑土地上赫然出现了一角白色,正是挖到棉被了。

无心把双手揣进大袖子里,一直站在一旁闭目不语。十二姨太的鬼魂依旧没有出现,不出现,反倒是更糟糕,因为不好说她是蛰伏到了哪里。正是等待之时,前方忽然响起了金石之声,无心骤然睁开眼睛,就听一名卫士气喘吁吁的说道:``挖着石板了。''

无心狐疑的问道:``怎么会有石板?''

副官长自从挖出棉被之后,就退到人后不再动手:``法师,原来这地方是个挺大的坑,坑里还有口井。井的位置不当不正,也没有用,我们就把井填了盖了,上面堆土种了花木。''

无心听到此处,不禁苦笑了一下,心想当年出尘子他太师祖为了镇压岳绮罗,费了不少劲才把她埋到了水井旁边;你们可好,不挑不选,一埋一个准,直接把活人葬在了至阴之地,不闹鬼才叫怪了。

这时卫士们的动作渐渐变得细致,大土坑中显出了一个红缎子面的大棉被卷,用麻绳左一道右一道捆了个结实,棉被卷子的上端还露出了一把乱糟糟的长头发。

老帅不让人再挖下去,只问无心道:``法师,您看我往坑里浇一桶火油,直接点火烧了她行不行?''

无心点头答道:``尸骨是一定要毁掉的,只是\ldots{}\ldots{}''

他犹豫了一下,最后决定还是对老帅实话实说:``十二姨太的魂魄忽然消失不见,恐怕单是烧了尸骨,还不算完。''

老帅一扬眉毛:``没了?我家宅子大,您再仔细找找?''

无心望着卫士在副官长的指挥下往坑里倒火油,隐隐的感觉不对劲。十二姨太无故消失,不对劲;如此轻易的就挖出尸骨,似乎也不对劲。十二姨太无非是借了地下的阴气,所以修为高出一般的鬼魂,可毕竟是个新鬼,再高又能高到哪里去?

副官长站到坑边,从裤兜里摸出了一盒火柴。一股子阴气骤然上升,无心下意识的从袖子里挥出一张纸符,然而未等纸符出手,副官长的火柴已经扔下去了。

火苗``呼''的窜起一人多高,同时坑中传出一声撕心裂肺的女人惨呼。老帅愣了一下,随即高声吼道:``不对,不对,是九姨太的动静——小九,小九!''

此言一出,周遭的副官卫士们全变了脸色。而一坑烈火之中猛然坐起一人,作势要往坑上爬,但两条腿被裹在棉被里,周身也被火焰包围,又无人肯伸援手,她哪里能爬上来?张牙舞爪的向上伸了双手,她哑着嗓子嘶嚎一声:``老帅啊\ldots{}\ldots{}''

老帅看她已经烧成皮焦肉烂,索性拔出手枪,对着她的脑袋便扣动了扳机。一声枪响过后,坑中立刻恢复了安静,九姨太一身的筋肉都烧抽了,摆出正襟危坐的姿势蹿火苗子。而无心扯着老帅避开火焰的炙烤,又小声问道:``老帅平时是不是偏爱九姨太?''

老帅一脸的无动于衷,单是皱起了眉毛,似乎只想随便的闹点小脾气:``对,我家小九最招人疼。''

无心忽然想起自己半天都没有喘过气了,便深深的吸了一口气叹出来:``十二姨太方才上了九姨太的身——府上的女眷们危险了!''

\chapter{大开杀戒}

帅府内的所有女眷,从大太太到小丫头,包括年幼的少爷小姐们,全被集中在了大洋楼内的大客厅里。大客厅容不下这许多女流,于是范围扩大到了整层楼。老帅怕乱了人心,所以保守秘密,并不肯说出大集合的缘由,导致年轻的姨太太们惶然无措,生怕自己会重蹈十二姨太的覆辙——都说十二姨太是跟副官有了一腿,可谁知道到底是不是真有呢?反正老帅是不讲道理的,并且爱犯疑心病,万一又把矛头瞄准了谁,还不就是一枪的事?

楼内楼外都站了士兵,帅府大门也关起来了,因为大张旗鼓的捉鬼实在不是美事,顶好不要走露风声。顾大人找机会凑到了无心身边,压低声音问道:``怎么还越闹越大了?你到底行不行啊?''

无心轻声的告诉他:``你看你给我找的这份好差事,妈的我还没见过这么难缠的鬼!''

顾大人卯足力气瞪了他一眼:``我的前程全系在你身上了,你可别给我打退堂鼓!''

无心挥了挥他的大袖子:``别啰嗦了,和你说话有损我的庄严宝相。''

无心撵走顾大人之后,自己也是有些犯愁。他倒是不怕十二姨太,问题十二姨太躲在暗处,他想打都找不到对手。

老帅不但没能除去十二姨太,反而还搭上了一个心爱的小九,所以拧着两道眉毛坐在二楼书房里,对于无心有些失望。不过话说回来,无心起码还能看出鬼魂的来历,这就比先前请的几位半仙高出许多;于是老帅压着脾气,没敢挑剔无心。忽然从半开的门缝里看到了副官长的身影,老帅抬手打了个响指。副官长立刻推门进来:``老帅。''

老帅低声问道:``法师在干什么?''

副官长走到大写字台前,对着老帅俯身答道:``法师不让我们靠近他,正一个人在楼梯口打坐呢。''

老帅点了点头,不再言语,而副官长又问:``老帅,要不要我再派几名卫士给您守门站岗?''

老帅嗤之以鼻:``笑话!一个臭娘们儿还治住老子了不成?不用派,老子就等着她过来!''

副官长出了书房,从侧面的小楼梯下了一楼,结果刚露面就被大太太喊住了。家里的小少爷有了食欲,要吃点心;大太太让副官长跑一趟,到自己房里拿些高级糖果回来。副官长不敢违抗正房太太的旨意,带着两名卫士就走出了楼,直奔后方大太太所居的小院。

大太太人老珠黄,常年不受老帅的宠爱,然而在家中颇有权威,能够镇住她眼中所有的小狐狸精。往日帅府里处处灯火通明,大人笑小孩叫,直到深夜才能寂静;如今各院也都亮着灯,但是只剩几名壮年的男仆看家,往昔的快活空气是一丝都没有了。

副官长常给大太太当差,此刻轻车熟路的进了院子往屋里走。各房都通了暖气,掀起棉门帘子一步迈进去,扑面便是一股暖风。两名卫士平时难得到达内宅,如今也跟着挤进来了,可是没敢进里屋,单是站在外间东张西望。

副官长径自推门进了里屋,眼看床边的大梳妆台上摆着几只糖盘子,里面装着五颜六色的外国糖果,正是大太太所需要的。上前把几只糖盘子里的糖果汇总到一盘中,副官长忙中出乱,不慎把糖撒到了梳妆台下。弯腰捡起几枚糖果,他在起身之时顺势抬头,目光掠过了梳妆台上的大玻璃镜。

攥着糖果骤然僵了动作,他睁大眼睛望向镜中——他起来了,镜中人也起来了。镜中人披头散发,青紫的脸上带了一点狞笑,正是十二姨太!

大叫一声跌坐在地,副官长的声音随即断在了喉咙里。房内的电灯瞬间灭了,两名卫士慌忙要去里屋看个究竟,然而两个人四条腿绊在一起,统一的全扑在了房门上。乱七八糟的站稳了再去推门,房门竟是不知何时被锁上了,严丝合缝岿然不动。

电灯灭了又亮,亮了又灭,电流嘶嘶啦啦的起了噪音。两名卫士惊惧到了极点,浑身乱摸着想要拔枪。不料正在此时,忽有一人大步流星的冲进来,一掌生生击开了房门。两名卫士哭天抹泪的看清了,原来来者正是无心!

里屋的梳妆台前抽搐着一个长胳膊长腿的黑影子,看身材就是副官长。无心上前一把推开了他,同时对着镜面甩手拍出一张纸符。镜中的鬼影倏忽间消失了,只听一阵清脆的咔咔声响,两名卫士连滚带爬的冲进来,依稀就见大玻璃镜中央无端的出了裂缝,裂缝像有生命一般向四面八方延伸出去,最后``哗啦''一声,大玻璃镜四分五裂,散落了满台满地。

无心从碎玻璃中捡起纸符,不肯轻易的丢掉了它。低头再看倒在地上的副官长,他发现自己晚了一步,副官长将一把匕首捅进了自己的心窝里,一双眼睛快要瞪出眼眶,正是一副惊恐万状的死相。

电灯光芒渐渐恢复了稳定,鲜血顺着匕首上的血槽缓缓流出,很快便在地上积了一滩。两名卫士张着嘴傻了眼,而无心用纸符蘸了鲜血,俯身贴上了副官长的眉心。十二姨太装神弄鬼的本领相当高超,穿身附体的能耐也不小。副官长一旦死了,便有可能被十二姨太操纵成行尸走肉。所以在天亮下葬之前,须得先用符咒镇住他才行。

镜子碎了,可见里面的鬼魂多少还是受了符咒的影响。十二姨太有魂魄无肉身,想要痛快杀人,只能寻找人的弱点下手。副官长心虚胆寒,便被她钻了空子,想必在自杀之时,已经被她摄去了魂魄。十二姨太如此疯狂,想必是存了和生前仇敌同归于尽的心思;仇敌之中首要的便是老帅,其次副官长对老帅惟命是从,将她捆绑活埋;九姨太恃宠而骄,大概和她也有过节。九姨太和副官长已然先后横死了,接下来的又会是谁?

无心有些懵懂,没料到一个小鬼居然难缠至此。心中忽然灵光一现,他问一名卫士道:``这是谁的屋子?''

卫士带着哭腔答道:``是、是大太太的卧室!''

无心立刻转身向外走去,他想副官长可能根本就是做了大太太的替死鬼。如果十二姨太是追逐副官长而来,那自己在楼内必定能够有所知觉;如果十二姨太一直埋伏在此,她又如何预知来人必是副官长?按照常理来讲,屋内之人应该是大太太啊!

无心顶风冒雪一路疾行,独自穿过夜色走向了前方的大洋楼。鬼魂附体不是易事,即便附体成功,也未必能够自如的活动。况且活人体内本有魂魄,一山不容二虎,一具身体被两家的魂魄争夺,一个不慎,鬼魂便会被驱出躯壳。十二姨太方才虽然消失的及时,但是多少也受了损,应该不会再有能力蛊惑人心。但是她控制不了活人,还控制不了死人吗?

无心想起十二姨太下落不明的尸骸,不禁十分头痛,认为和顾宅井里的女煞相比,十二姨太别有一种难缠;幸而她不像女煞一样拥有一具灵活丑陋、不死不伤的躯体,否则就更不好办了。

在楼门口,无心遇上了老帅和顾大人。老帅已经记住了顾大人的名字,导致顾大人受宠若惊,红光满面。无心没理顾大人,直接对老帅说道:``副官长被十二姨太杀了!''

老帅一哆嗦,伸手就要去摸枪。冷不防一个小影子从他身边窜了出去,正是家里的小少爷。小少爷大难不死,瘦如小猴,可是顽劣至极,一边跑一边用他的小细嗓子骂骂咧咧,说是楼里面没意思,要让三哥开车载他出去玩。小少爷的地位自然是高的,老帅作势要去抓他,抓了个空,随即身后追出一名富态的妇人,正是大太太。

大太太摇摇摆摆的也伸着手,逮小鸡似的要逮小少爷。小少爷昨天养了一天,今日中午才下了地,东倒西歪的站都站不稳,此刻却是跑得极快,两条小腿迈了个乱七八糟,硬是不跌倒。忽然回头对着大太太哭了一声,他只嚷出一句``妈'',随即就像身不由己一般,一头扎进了路旁的灌木丛里。

无心早就怀疑小少爷不是好跑。凭着小少爷的体力,行走都是困难,怎么会有体力蹦跳吵闹?事到如今,他抬手一指大太太的背影,大声喝道:``站住!''

大太太也想站住,可是眼看着小少爷的小手在灌木丛外一闪,自己分明一把就能抓住,便忍不住加大了步伐,以为自己马上就能把孩子拽回来。

可是就在她伸手的一刹那间,喉间忽然起了一阵冰冷的痛。在后方骤起的惊呼声中,大太太疑惑的望向前方,看见了久违的十二姨太。

十二姨太穿着一身勒腰显屁股的绸缎衣裳,仍然是她最瞧不惯的浪样子,只是长发凌乱,周身沾了一层湿土,下嘴唇松松的垂下去,露出一排白牙,却是一侧嘴角被生生的撕扯开了。

大太太糊涂了,因为记得家里这只天字第一号的骚狐狸已经被老帅活埋。莫名其妙的低下头,她看到了一只残缺不全的手。手背手指都带着刀伤,皮肉翻开冻硬了,能看到里面的根根白骨。而完整无缺的食指中指插进了自己的咽喉里,弯曲着勾住了皮肉与气管。

手指忽然向外一扯,大太太的咽喉被十二姨太挖成了一个血洞。胖大的身躯颓然倒下,十二姨太向后一闪,隐没在了黑暗之中。

老帅和顾大人一起发了傻,一高一矮的站成了桩子,唯有无心撩起僧袍,撒腿追了上去。

\chapter{魂飞魄散}

无心一身宽袍大袖的打扮,跑起步来飘飘欲仙。高抬腿越过一片黑压压的灌木丛,顾大人和老帅就见他倏忽间失了踪影,真如神仙一般高来高去,踪影莫测。

无心坐在灌木丛后,则是咬紧牙关一声没吭——落地之时没站稳,他一屁股坐到了地面突起的树桩子上。说是树桩子,其实还没有孩子手腕粗,是棵被人伐断的小树,正戳中了他的屁股,让他疼得差点背过气去。前方依稀还可见到十二姨太的背影——借尸还魂果然不是容易事情,十二姨太的尸体在地下冻得太久了,行动起来东摇西摆,胳臂腿儿不听使唤,血淋淋的右手还不闲着,拉扯了小少爷的衣领向前拖行。小少爷失去了方才的精气神,脑袋四肢全垂下去了,很认命的在雪地上留下一道痕迹。

无心无暇忍痛,一咬牙又站起来,沿着地上痕迹继续追逐。顾大人和老帅冷不防的看他忽然又冒了出来,都是一愣,愣过之后回了神,顾大人虽然是赤手空拳,但也冲向了灌木丛;老帅不甘落后,一边高声吆喝部下,一边拔出手枪随上了顾大人。他不是为了去看新鲜,法师把十二姨太撕碎了他都不管;他惦记的是自己的小儿子!

然而他毕竟是有了一点年纪,冬天穿得厚重,又是个小个子。顾大人一抬腿就跨过了灌木丛;他也跟着跨,``嚓''的一声就把裤裆扯了,露出了里面的枣红色毛线裤子。顾大人闻声转身,殷勤忠诚宛如帅府内的三世家奴,立刻就伸手把老帅搀住了。老帅踮着脚翻过灌木丛,倒是感觉姓顾的小子够机灵,比家里用久了的副官还强。

十二姨太跌跌撞撞跑得很快,鬼魅一般直往前冲。无心如今一身嫩肉,薄雪下面全是枯枝碎石,硌得他两只脚一起作痛。一个踉跄扑倒在地,他顺势捡起一块拳头大的石头,对着十二姨太的背影用力一掷。掷的时候他没瞄准,然而石头却是正中了十二姨太的后脑勺。十二姨太向前一晃,脚步不停,继续往花木深处疾行。而无心趁着她短暂的一顿连迈了几大步,竟把双方之间的距离缩短了许多。眼看小少爷虽然瘫软,但是喘得厉害,无心便知道他身心无恙,不过是先前支配他的一股子邪劲消了,他如今又累又怕,连哭叫的力气和胆量都没有了。

十二姨太也察觉到了无心的靠近,猛然一回身面对了他,她把手合在小少爷的脖子上,口齿不清的发出了声音:``你敢过来,我就杀了他!''

无心立刻刹住了脚步:``你抢了小少爷当人质,莫非还想要挟老帅不成?''

十二姨太的下嘴唇随风轻摆,导致她说话说得挺费劲:``对!我要让他死!''

无心认为老帅基本没人性,死就死了,只是他的生死关系到顾大人的前程,所以自己还非保护他不可。双手揣进袖子里,他心想此刻出尘子要是在场就好了——岳绮罗在场,胜算更是高达十成。据说道术练到一定的境界,可以将一张纸符摆弄得出神入化,指哪打哪;可是凭着他现在的本领,即便把纸符叠成纸鸟了,顺风也扔不出多远去。

手指暗暗的将一张纸符搓成了细长的纸卷,他把纸卷夹在了指间。对着十二姨太诡谲一笑,他抬手解开斗篷,甩在了一旁的大雪地上。慢条斯理的卷起大袖子,他缓步向前走去。

十二姨太的脸上没有表情,但是语气有了变化:``干什么?''

无心骤然一个冲锋,远远的就对她挥起了拳头。而十二姨太本来威胁着要杀掉小少爷,可拳脚迎面而来,她虽是一具行尸走肉,不怕伤害,但下意识的也要躲闪。无心趁机一把扯住小少爷的衣襟,大喝一声奋然拎起,不由分说的向后一抛。小少爷在半空中划了一条弧线,病猫似的落了地,摔出``咭''的一声。

十二姨太莫名其妙的失了人质,气得一大片下嘴唇乱颤。扬起枯枝似的两只瘦手,她哑着嗓子低吼一声,直通通的就要去抓无心。无心舍不得咬破自己的舌尖,所以劈面只啐出了一大口唾沫:``心狠手辣的臭娘们儿,敢坏我的大事!今天要不打得你魂飞魄散,我都对不起我法师的名声!''

骂完之后他一歪头,躲开了十二姨太的两只手爪。又一拳直击向了她的面孔,在拳头将要触到鼻尖之时,他忽然一变手法,将暗藏的小纸卷塞进了十二姨太的鼻孔中。不料纸符卷得过于紧密,被他搓得如同一根火柴棍,十二姨太感觉有异,猛的一晃脑袋,竟将纸符晃了出来。无心的纸符全藏在袖子里,要拿虽然容易,但也需要腾出手来才行。十二姨太显然不会给他机会,直挺挺的向前抱住无心,她僵硬的向上一跳,同时把嘴张到极致,上面整整一排大白牙顺势而落,结结实实的啃在了无心的头皮上。无心痛彻心扉,登时急了,向上拼命击出一拳,正好斜着打中十二姨太的下巴。十二姨太已是一具死尸,无知无觉,并不怕他打。顺着力道退了几步,十二姨太抬头正视了无心,一个下巴受了打击,自作主张的歪向一侧,于是十二姨太的上下两排牙齿各自为政,再也没有相会的可能了。

十二姨太看出无心是个棘手人物,不除了他,便不能彻底的报仇雪恨;所以趁着自己的身体尚未腐朽,她便连下狠手,想要杀掉无心。无心无论如何都找不到机会再取纸符,无可奈何,只得专心还击。于是当老帅和顾大人赶到之时,前方一人一尸撕撕扯扯,正打得热闹。

老帅先从雪地上扶起了小儿子,见孩子不但有气,而且还会抽抽搭搭的小声哭泣,便放了心。抬眼再瞧战况,他一直把无心当成谪仙人看待,不料此刻仙人不仙,正和死而复生的十二姨太对扇耳光。顾大人摩拳擦掌的想要参战,可是手无寸铁,又不敢向老帅借用武器;好在他知道无心总不会有性命之虞,所以心里倒还有底。

老帅搂着小儿子,本意是要观战。可是见法师细皮嫩肉的不禁打,又痛恨十二姨太伤害儿子,便把小儿子推给顾大人,自己握了手枪趴下去。眼看无心把女鬼压在身下了,他伸手就是一枪,子弹贴地飞出去,立时轰飞了十二姨太半边脑袋!

冰碴子瞬间崩了无心一脸,全是十二姨太结了冻的脑浆。十二姨太死都死了,自然不会再吝惜尸体。半个脑袋向上一撞,她还想要推开身上的无心。老帅看她死而不僵,抬手又是一枪。十二姨太余下半个脑袋也粉碎了,冰渣子夹杂着碎骨头以及二十多颗牙,纷飞着又溅了无心一头一脸。无心气极了,恨不能先去把老帅痛打一顿。他正打算出声赶走老帅,哪知十二姨太抬起血肉模糊的脖腔子,对着他又是一拱。无心虽然知道她从血到肉都冻透了,但还是不愿硬碰硬。一个鹞子翻身滚到一旁,他忽然发现十二姨太的残躯瘫在地上开始痉挛,灵魂幻影似乎正要逃脱,然而几番挣扎过后,却是附在尸体上不能离去。

无心抬袖子一抹脸,俯身凑过去抓住十二姨太的肩头,轻轻的扳了她的身体。一张打着卷的纸符不知是染了什么黏液,险伶伶的粘在了她的后背上。

无心没想到自己的误打误撞加上老帅的误打误杀,竟然制住了十二姨太。随手抓起一把冰碴融化了,他不敢细看,直接将满掌黏腻抹上纸符,将其平平展展的粘上了十二姨太。

把十二姨太翻过来摆成俯趴姿态,纸符被寒风一吹,立刻被结结实实的冻住了,揭不下撕不掉。无心用雪擦了擦手,低头闭上了双眼,就感觉十二姨太的灵魂已经失了形状,光芒也越来越弱。显然,她抵不住符咒的力量,也许坚持不了多久,就要烟消云散了。

起身走去捡起斗篷披到身上,他恢复了先前的冷静,对着老帅和顾大人说道:``十二姨太已经被我降伏住了,现在请你们远离此处,我要做法散去她的魂魄,永绝后患。''

老帅一挺身爬起来,双掌合十对着无心拜了拜,也说不出别的话,只道:``厉害,厉害!''转头一看顾大人已经把小儿子抱起来了,他便不敢停留,领着顾大人向后撤退,顺带着把尾随而来的卫士也一并带走了。

无心在十二姨太的残躯旁边席地而坐,盘起双腿戴了风帽。垂下头闭了眼睛,他伸手拍了拍面前僵硬的躯干,心中说道:``我知道你是冤死鬼,可是相干不相干的人,你也杀了好几位,该出的气也出了大半,如今世间已经没有了你,你安心上路,到去处去吧!''

符咒专克煞气重的阴魂,在无心的脑海中,十二姨太已经成了一团灰蒙蒙的光。十二姨太的声音响起来,带着一点调子,是唱曲的人唱久了,再也改正不过来。

``我没有偷人\ldots{}\ldots{}''她哀哀切切的说:``是大太太陷害我。大太太出的主意,九姨太在老帅跟前煽风点火\ldots{}\ldots{}老帅打我骂我,用刀子割我\ldots{}\ldots{}我死得冤枉,我没有偷人\ldots{}\ldots{}''

无心入定一般默然垂头,只用心思和她应答:``我相信你。''

声音又响起来了,清凌凌的寒冷,像是初春的河水:``副官长调戏过我,我不理他,他得了报仇的机会,就亲自带人活埋了我。我并没有滥杀无辜,他们全是该死!''

无心问道:``十二姨太,九姨太是怎么死的?我想了又想,可是想不通。''

十二姨太的灵魂将要黯淡成了一片影子:``我附了她的身,直到你们浇火油时才放了她。''

无心在风帽的阴影中笑了一下:``你驱使她做了苦工,又要运出你的尸骨,又要躺下去补你的缺。''

十二姨太说道:``我知道你们必定要掘出我,所以不敢破坏地面,故意绕了远路去挖地洞。九姨太跪在雪里用双手刨土,皮肉磨破了,露出骨头;骨头一节一节的磨下去,骨髓流出来\ldots{}\ldots{}是她先害了我,她罪有应得。可惜直到最后魂魄归位,她才尝到了痛苦的滋味。''

天边现出了隐隐的霞光,是天快要亮了。无心知道十二姨太马上就要魂飞魄散,便对她说道:``人间素来不缺冤死鬼,你不是第一个,也不是最后一个。你没办法,我也没办法。如今你要走了,我念一段经送送你吧。往生咒我比较熟,地藏经就有点含糊,你想听哪一段?''

十二姨太最后答道:``都不爱听,来段大鼓书吧。''

话音落下,淡淡的影子消散成了薄雾,若隐若现的光芒向着四面八方一闪而逝。于是无心的回答就没了听众,只能留给自己听:``可是,我不会唱大鼓书啊。''

无心站了起来,屁股大腿全都冻得冰凉。世上再无十二姨太,可是他的事情还没完。一夜死了三个人,他得尽快处理好三人的后事,免得再生枝节。

\chapter{尽人事}

无心命人把十二姨太、副官长、大太太的尸首全抬到了大太阳下,没有用布单子苫盖,只让一队士兵围成一圈看守了他们。

花园的土坑里还盛着九姨太的骨头,也是应该一并处理掉的,不过无心现在饥寒交迫,实在是没有力量。老帅先让人把小少爷送去了医院做全身检查,然后专心致志的前来招待无心。

无心在帅府的客房里面脱了斗篷僧袍,坐进大浴缸里泡了个热水澡。舒舒服服的躺在热水里,他抬手抚摸着墙上贴着的白瓷片,心想自己不过是几年没下山,外面的世界竟然大变了模样,等到闲下来了,真该到处好好看看。

浴缸旁边的墙上支出铁丝络子,里面放着一块崭新的香皂。无心打出满身的泡沫,把自己洗了个喷香。披着丝绸浴袍走出去,他打了个哈欠,饿得肚子里叽里咕噜乱响。

重新将僧袍穿戴上了,他面无血色的推开房门,冷不防的和老帅打了个照面。老帅一宿没睡,可是精神百倍,先是对着无心一笑,随即开口说道:``法师,我想起个事儿。我家小子上车去医院时不老实,自己把脸上的血符擦掉了一半\ldots{}\ldots{}''

无心当即一摇头:``没关系,府上已经干净了,血符也没什么用处了。''

老帅双掌合十,五体投地的表示崇拜:``法师,您是真高!一夜死了三个大人,他个小崽子却是平安无事。全亏了您的血符保佑了!''

无心云淡风轻的笑了一下,心想自己当初本也没在小少爷的脸上留下许多鲜血,鲜血一干,更是少到了将近于无。十二姨太昨夜若是真想杀他,未必就一定杀不成。

他不爱搭理老帅,老帅却是挺爱和他说话。恭而敬之的把他请到餐厅,无心被他强行摁到了首席座位上。无心面无表情的扫了桌面一眼,就见一张大方桌上琳琅满目,各色饮食俱全,单是看一遍都能过瘾。

顾大人也被一名副官叫来了,扭扭捏捏的坐在下首。无心只对着他一点头,然后抄起筷子,自顾自的开始吃。依着无心的食欲,真恨不能直接端起桌子往嘴里倒;可是为了维护自己仙风道骨的形象,他一口一口吃得很有克制,刚刚到了三分之一饱,就把筷子放下了。而老帅终于除了大患,一口一个小笼包,吃喝之余谈笑风生,任谁也看不出他夜里刚死了两个太太和一名干将。

无心不饱不饿的出了门,带着老帅和顾大人往后花园里走。一路回到了十二姨太的埋尸地,无心从旁边的枯树上折下一根粗壮枝条当成手杖,弯腰去翻坑中的几块焦骨。一杖戳进了松软的土壁之中,无心低头问道:``老帅,想不想知道九姨太的死因?''

老帅正对着坑里的骨头唉声叹气,听闻此言,立刻答道:``法师,您快给我讲讲。''

无心跳进坑中,用树枝狠狠去捅坑壁,三下两下竟然捅出一个小小的洞口。正能容得一位苗条女子爬行出入。一边向老帅讲述了九姨太的来和十二姨太的走,无心一边继续研究洞口。末了他跳上地面,对老帅又做了一番吩咐。老帅如今对他是无不相从,立刻依言唤来家里卫士听差,沿着土洞的走向开挖。挖到最后,众人在花木林外的一条溪边发现了入口。

小溪本是用来点缀风景的,冬天花园里无所谓风景,小溪也早冻成了一条冰带。入口隐藏在一丛荒草之中,别说花园子里平常不来人,就算真有人经过,也不能留意到它。老帅弯腰从土里捡起一团兔子尾巴似的白毛,认出它是九姨太戴在头上的新式发饰;发饰尚存,斯人已逝,老帅眨了眨小眼睛,眼角闪烁了一点多愁善感的泪光。

胆大的卫士拣出了坑中的遗骨,整条土洞则是被卫士铲了泥土,结结实实的全填了上。把遗骨送到三具尸首旁边,无心建议老帅再来一把火,先把尸首烧尽了,然后再做丧事的打算;否则万一闹起借尸还魂,可不是玩的。

老帅围着尸首转了一圈,毫不犹豫的一口答应;无心冷眼旁观,见他不但不悲,仿佛还渐渐生出了一点喜色。

老帅察觉到了无心的注视,连忙正了正脸色,可是正了没多久,他又美起来了。俗话说男人的三大喜事是升官发财死老婆,老帅在升官发财一途,已经高到了极致,唯有同患难的正房老婆身体康健,总无去死的自觉。当然,大太太是拦不住他纳妾的,不过他一旦往家里讨了新人,免不得就要受她一顿聒噪。他念着对方老妻的身份,又不好对她施展拳脚。如今大太太一死,身边的女人再没有超过二十五岁的,放眼一望正是姹紫嫣红开遍,让老帅真是神清气爽。

火是在大雪地上燃起来的,三具尸首靠在一起,烧着烧着会猛然惊坐起来,是大梦初醒的样子。

帅府里闹过一场鬼之后,没人愿意靠近凶死的尸首了。所以留下来的人,还是无心。

无心孤零零的蹲在火堆旁,手里攥着一根拨火棍,同时轻声唱着一段经。生死也算是大事情,他总不希望一条生命来得孤单,走得也孤单。大人物出场退场都是要奏乐的,他没办法一个人奏出乐来,但是慢悠悠的唱上一段倒没问题。

细雪飘落下来,落在了他短短的黑头发上。他的声音沙哑苍凉,给他平添了千百岁的年纪。

一场恩怨落了幕,除了老帅安然无恙之外,其余四人全成了灰烬。帅府众人各归其位,老帅闹鬼还闹高兴了,嘻嘻哈哈的要重谢法师。然而无心只对他摇头一笑,轻声说道:``贫僧说过,前日之所以肯来帅府,并非为了惩恶扬善降妖除魔,而是为了我和顾玄武有点缘法。他让我来,我便来了。如今邪祟已除,贫僧完成了分内之事,也算了了一桩心愿。至于钱财,非我所图,所以心领足矣,无须接受。''

老帅眼看着法师飘然要走,连忙把顾大人叫了过来,又追着无心说道:``法师,话虽如此,可是我也没有让你白出力的道理。我——''

无心背对着老帅摆了摆手,头也不回的走向了大门:``老帅如果真有酬谢我的心意,把钱财施舍出去救苦救难,也是一样的。''

老帅有年头没见过不贪财的和尚了,几乎有些发愣。顾大人陪着小心说道:``老帅,您别多虑。法师不是假客气的人,他说不要钱,就真不要钱。''

无心硬着头皮冷着脸,拒老帅于千里之外。坐上帅府的汽车,他独自回了家。

他昂首挺胸的进了院门,院门一关他就软了,靠着门板咩咩的叫:``月牙,月牙,我回来了。''

月牙在房内听他颤声颤气的十分像羊,连忙推门迎了出来,只见无心向自己伸出了一只手,一双眼睛陷在眼眶里,昨天上午的好神采一丝都没有了。

月牙把他搀回房内,又给他脱了僧袍。无心瘫在床上,先是说累,后是说饿。月牙听得莫名其妙:``啥?你给他家忙了一宿,他连顿饭都不给你吃?''

无心懒得再提帅府情形,一味的呻吟不止。月牙给他端来了一大碗米粥,他起身喝了一口。月牙再喂他第二口时,他不喝了,哼哼唧唧的说道:``没滋味,放点糖。''

月牙饶有耐性的往米粥里拌了一勺砂糖,然后继续喂他:``顾大人呢?''

无心答道:``留在帅府拍马屁呢!''

月牙发现无心越来越缠人了。

大上午的,她有不少活要干,可是无心拉扯着她,死活不肯放手。月牙一横心,自己给自己放了假,上床躺到了无心旁边。两人也不说话,单是面对面的侧身对视。无心觉得月牙真好看,月牙也觉得无心真好看,双方统一的全看呆了。

良久过后,无心抬起一根手指,在月牙的脸蛋上划了一下。月牙也用手指一推他的鼻尖,给他推了个朝天的猪鼻子。

``哎。''月牙忽然开了口:``你说再过个二十年三十年,我成老婆子了,你咋办?''

无心不假思索的答道:``我伺候你。''

月牙一拧他的鼻尖:``你不嫌我?''

无心郑重其事的摇头:``只要你别半路嫌我就行。''

月牙继续试探他:``到时候我头发也花白了,脸上也有褶子了,牙也掉的差不多了,一说话就满嘴漏风,一干活就咳嗽气喘,你真不烦?''

无心双手捧着月牙的脸:``我不烦。将来你老了我不老,你别烦我给你招闲话,我就谢天谢地了。''

无心和月牙过了一天清静日子,到了晚上,无心站在厨房里,正在向月牙描述帅府里的新式大浴缸,院门一响,却是顾大人回来了。

顾大人已经吃过了晚饭,并且带着一点微醺的酒意。笑嘻嘻的倚着厨房门框,他对月牙说道:``师父算是给我长了脸,等我真得差事了,非得给你买副钻石坠子不可!''

然后他又转向了无心:``老帅明天中午订了一桌素斋,专要请你。你去不去?''

无心一摇头:``不去。''

顾大人得意洋洋的笑道:``我看你好像对老帅挺有意见。唉,男子汉大丈夫嘛,就得杀伐决断有魄力,死了太太就不过日子啦?反正老帅今天跟我聊了一下午,我俩倒是很投脾气。老帅说了,让我明早就到他身边当差去,凭着我的资历,他总不能让我当副官马弁吧?嘿嘿,他随便发一句话,我不就又是官了?''

翌日中午,无心并没有去帅府赴宴。

在他看来,老帅堪称恶棍之中的楚翘,实在是不能令人亲近。而自己已经尽了人事,顾大人能否东山再起,就看天命吧!

\chapter{得意的顾大人}

顾大人忙起来了,每天起的比大公鸡还早,睡的比夜猫子还晚。只要是清醒着,就必定不在家。无心和月牙日出而作、日落而息,竟然连着好几天都没和他打过照面。

月牙手里还有点钱,吃喝不成问题,并且还能吃香的喝辣的。无心自从生出一脑袋短头发之后,四肢渐渐结实,躯干渐渐苗条,彻底恢复了往昔的人样子,同时饭量也慢慢恢复了正常,不再心急火燎的终日觅食。两人既然吃饱喝足了,又是从早到晚的清闲,唯一的娱乐便放在了夜里床上。

无心喜爱月牙的一切,能整夜的把脸埋到对方胸前,仿佛是要溺死在两个大馒头之间。月牙汗津津的搂着他抱着他,从发梢到指尖,浑身上下软洋洋的舒服。休战时间没有持续多久,无心忽然仰起头,动手动脚的又爬了上来。只有累死的牛,没有耕坏的田;月牙才不怕他。

新年前夕的一天夜里,顾大人难得的早回来了。说是早,其实也早不到哪里去,天都已经黑透了,院子被下午的大雪苫盖住,漆黑天幕上撒了一把银亮的星星。

顾大人终于得了差事,并且换了一身威武的戎装。兴冲冲的走到西厢房门前,他先伸手推门,没推开,便转而去敲玻璃窗子:``哎,你俩睡了?''

房内是月牙开了口,声音又尖又细的打着颤:``我我我俩已已经睡睡睡着了\ldots{}\ldots{}''

顾大人弯腰往地上擤了一把鼻涕,然后抬头发起牢骚:``别跟我扯鸡·巴·蛋了,谁睡觉能睡得像闹猫似的?你们两口子真是天生一对地设一双,除了吃就是日,没一个是有大志向的!跟你们说啊,我今天有了好事,你俩但凡讲点感情,都该出来看看我的新军装;另外我还有三张戏票,明晚有空了,请你们去戏园子看戏。''

房内换了无心答话,答的一顿一顿,像是在大卖苦力:``顾大人,恭、恭喜你。我们明、明天再、再见吧!''

顾大人很扫兴,骂骂咧咧的自己回东厢房去了。

翌日清晨,月牙正在院子里扫雪,顾大人披着旧棉袄出来了,向月牙要热水洗漱。月牙伸手一指厨房,让他自己去拎热水壶,顾大人一边走一边瞄了她一眼,心想月牙的屁股越来越大了。

月牙跑去厨房熬米粥切咸菜的时候,无心也起来了。端着一小盆水蹲在厨房门口,他给月牙洗土豆。顾大人抓紧时间换上了新军服,及膝的大马靴也蹬上了,耀武扬威的在院子里溜达。月牙向外一眼瞧见了,大声笑道:``嗬!顾大人真漂亮!''

顾大人背着手站在了厨房门前,见小两口全都在望着自己发笑,心中便是十分满足:``废话,我能白忙活吗?老帅要是不给我一身新皮,我凭什么早出晚归的天天去恭维他?''然后他用下巴一指无心:``你啊,真是烂泥扶不上墙。老帅问了你好几次呢,你略微通点人情,可能就在老帅手下发大财了!我要是像你一样长生不老,不是吹牛,我现在早当上大皇帝了!啊不对,现在叫总统,我早当上大总统了!''

无心很仔细的搓去土豆上的泥土,两只手水淋淋的苍白:``没意思。''

顾大人向他弯下了腰:``什么没意思?''

无心答道:``当皇帝没意思。''

顾大人伸出一只手,在他头顶上``嘣''的弹了一指头:``你当过啊?''

无心停了手上动作,特地花了一分钟思索,末了低头继续洗土豆:``我好像是看别人当过。记不清楚了。''

顾大人直起了身,因为和无心谈不拢,所以把目光又转向了月牙:``月牙,你把咸菜用油炒一炒,要不然不好吃!''

月牙在围裙上擦着湿手,笑着问道:``顾大人,你昨晚是不是说今天要请我们去看戏?''

顾大人一点头:``是啊!''

月牙立刻把油瓶子拿了起来,自己小声笑道:``太好了。''

无心和月牙到底也没弄清顾大人当了什么官,只知道年后他就要被派出去带兵了。

顾大人给了月牙五十块钱,让她先用着过年;又说:``我现在手头还不宽裕,钻石坠子明年再买!''

月牙接了钱:``拉倒吧,我还真要你的钻石坠子?听说钻石可贵了,你有买钻石的钱,不如攒起来买间小房买块地。''

顾大人哭笑不得:``你啊,一辈子就是个老妈子的命。我想提拔你当阔太太,你可好,有福都不会享!''

月牙依然是笑,感觉顾大人的话都像天方夜谭:``我们老李家祖祖辈辈就没出过阔太太,再说现在的日子不是挺好?吃鱼肉穿绸缎,还想咋的?''

顾大人拿无心和月牙没办法,于是决定省省口舌,横竖凭着自己的本事,很能够养活一对胸无大志的穷鬼夫妻。到了晚上,他领着头把院门一锁,果然是带着无心和月牙去了戏园子。

园子里面乌烟瘴气,然而戏是真好。无心月牙全都看直了眼睛,嘴里没滋没味的嚼着顾大人买来的蜜饯。及至午夜散戏了,三个人顶着寒风往外走,月牙忽然又傻了眼,因为看见一辆汽车旁站着个妖娆女人,数九寒天的,居然光腿只穿了一层丝袜子。

她看,无心顺着她的目光也去看,看得十分持久,最后还是顾大人叫来两辆黄包车,把他俩全撵了上去。到家之后,顾大人端了一盆热水回到东厢房,脱了衣裤坐在床边烫脚。正是烫到销魂之际,房门一开,一身单衣的无心跑进来了。

顾大人莫名其妙:``有事?''

无心关了房门,然后走到小洋炉子旁边站住:``月牙把我撵出来了。''

顾大人登时来了精神:``为什么啊?''

无心显然是在害冷,拱肩缩背的走到床边坐了下来:``她说我看女人看丢了魂,还说我小白脸子不老实。''

顾大人的赤脚在水盆里踩出了浪花,摇头晃脑的发表意见:``嗨哟,你那个媳妇还要造反不成?男子汉大丈夫,三妻四妾都是天经地义,何况只是看一眼?我不是挑拨啊,如果换了我,早就一个大嘴巴抽过去,让她认得老子是谁了!''

无心抬腿滚到床里,扯了棉被向上一直盖到肩膀,同时轻声说道:``她不打我就不错了。''

顾大人侧过身,长长的伸出手臂戳了他一指头:``你是怎么个意思?还不走了?''

无心翻身背对了他:``先跟你挤一夜,明早再去哄她。''

一夜过后,顾大人一睁眼睛,就发现无心没了。

不以为然的打了个大哈欠,顾大人照例是披上棉袄出了门,前去厨房拎热水,结果刚刚到了门口,他就见月牙和无心在厨房里嬉皮笑脸的打情骂俏。原来无心凌晨便醒,贼似的潜回了西厢房。而月牙睡得昏天黑地,怒气早散了一干二净。无心预备的一番甜言蜜语尚未施展,两人在热被窝里就自动的抱在一起了。

顾大人出了门,晚上回了来。月牙炖了三条大猪尾巴,满院都是肉香。顾大人先是回房脱了军装卸了武装。等到月牙把一大锅猪尾巴端到上房桌上了,他才单手插着裤兜,颇为潇洒的出现在了人前。

掏出三张电影票扔到桌子上,他洋洋得意的垂下眼帘,瞄着锅里的猪尾巴说道:``告诉你俩,我今天改名了!从今往后,我大名就叫顾国强!''

无心率先坐到了桌前,笑吟吟的仰头看着他不说话。月牙一边盛饭一边问道:``不叫玄武了?

顾大人一挥手:``不许再提那个王八名字!国强可是老帅亲自起的,嘿嘿,老帅说我作为军人,应该把国家放在心上,叫国强正合适!''

月牙想起了玄武二字的由来与成本,不禁发出感叹:``早知道今天要改名,当初不如一直就叫顾石头,还能省一块大洋。''

顾大人不屑一顾的伸手一指她:``听你说话就是小家子气,真是个头发长见识短的小娘们儿——你找个盘子,把中间那条大猪尾巴给我盛出来。''

月牙一撇嘴:``真是大人物,吃猪尾巴都得抢条最大的。将来你要是当上大帅了,是不是还得改名字?''

顾大人一屁股坐下来,理直气壮的答道:``当然!国强虽然比玄武强,但是听着还不够雅。万一我有了当督军总长大总统的命,还不得换个更体面的名字?不过那是后话,眼下姑且不必提。你赶紧把筷子递给我,吃饱了我带你们看电影去!''

\chapter{快乐的新年}

月牙生平第一次当家作主,手里又有余钱,所以大手大脚的张罗置办了一切,兴高采烈的过了个肥年。顾大人也有了进项,虽然暂时还买不起钻石坠子,但是决定给月牙做一件新皮袍子。月牙万没想到自己竟然还能穿上皮子衣裳,兴奋的不知如何是好。等到成衣店把皮袍子做好送来了,她关上房门左一遍右一遍的试个不休。袍子紧随潮流,尺寸太合体了。月牙对着镜子照来照去,就见皮袍子明目张胆的勾勒出了身体曲线,显得大奶子大屁股,腰又太细,整个儿的像是葫芦成了精。

于是月牙就有些为难,小声的问无心:``裁缝也是的,恨不能把皮袍子做成紧贴身。你看看能不能穿出去?''

无心作为唯一的观众,伸手捏了捏皮袍子的松紧:``只要尺寸不错就行,现在街上不兴穿直筒棉袍子了。''

月牙又转了个圈:``瞅着不浪啊?''

无心很笃定的摇了头:``我看挺好。''

月牙一横心,决定效仿戏园子电影院里的摩登女性,也跟着展示一下曲线美:``你要是不管我,我就敢穿!''

大年三十的夜里,无心和顾大人蹲在院子里燃放烟花。顾大人自从改了名字之后,精气神都变化了,走起路来一步一响,是个意气风发的好模样。月牙捂着耳朵站在一旁,一直是连说带笑的看热闹,最后把嘴一闭,才发现牙齿舌头全冻成了冰凉。伸手捂住了嘴,她忽然想起了自己的娘家——当初要是不跑,现在就是在马家过年了。老头子是好伺候的?姨太太是好当的?

月牙扑闪着笑眼望着无心,无心穿着件单薄的小棉袄,正在很认真的和顾大人抢鞭炮。从相貌上看,他可以做顾大人的老弟;只要女人目光短浅一点,头脑糊涂一点,他就真是个最可人疼的好丈夫了。

顾大人喝了酒,醉得天下无敌,一屁股把无心拱出老远。回头一看无心跌坐在雪地上了,他大发慈悲,转身伸手又把无心拽了起来。无心坐了一屁股雪,自己不知道,还是月牙过去给他拍了拍裤子:``就知道闹!再闹都回屋吧,万一大除夕的你俩再打起来了,我可劝不开架!''

顾大人翻脸如翻书,毫无预兆的就和无心恢复了友谊。抬手揽住无心的肩膀,他对月牙发笑:``嘿嘿嘿!''

几分钟后,三个人蹲成了一圈。顾大人深深的吸了一口香烟,然后用橙红的烟头点燃了中央一管烟花的引线。五颜六色的小火星窜了出来,不高,可是五光十色的很持久。顾大人冻得耳朵鼻尖通红,很得意的问道:``漂亮吧?''

无心把双手揣进袖子里,眼睛一眨不眨的盯着烟花,黑眼珠特别的大,满眼都是流光溢彩的影子:``漂亮!''

月牙本是抬手在嘴边呵热气,呵着呵着不呵了,对着烟花笑出了个红彤彤的苹果脸儿:``是漂亮!''

细雪簌簌的飘落,落白了三个年轻的脑袋。待到最后一簇火星熄灭在了低空中,顾大人心满意足的长吁了一口气:``好,有点意思,没白花钱。''

月牙知道顾大人买烟花时专挑贵的下手,正是开口想要作答。不料无心忽然伸出双手,一边握住了月牙的手,另一边握住了顾大人的手。把两只手拉过来贴上了自己的面颊,他学着电影里的男主角,郑重其事的低声说道:``我爱你们。''

月牙无声的笑了,露出一口小白牙;顾大人愣了一下,随即对着月牙说道:``听见没有?电影没白看,学会发骚耍贱了!''

月牙不知道说什么才好,所以索性忍笑沉默;而无心不和顾大人一般见识,自得其乐的闭了眼睛,感受着二人掌心的温度。

月牙手软,顾大人手硬。无心爱死了他们,恨不能分别咬他们一口。月牙笑眯眯的始终是不言语,而顾大人对着无心望了半晌,末了抬眼和月牙一对眼光,伸舌头做了个鬼脸。

新年过得很顺,从初一到初五,就没发生过别扭事情,连顾大人的口齿都比平日甜美了许多。到了大年初六,顾大人慈眉善目的夸奖月牙:``月牙干别的不行,厨房里的手艺倒是真不错。''

月牙听他夸奖自己,先是笑,笑着笑着感觉不对味:``我干啥不行了?无心屁也不管,家里外头还不都是我一把抓?别说我还是个小媳妇,凭我的本事,老媳妇都赶不上!''

无心倚着门框站着,听闻此言,略微感觉有些不安,怀疑战火要烧到自己身上。结果顾大人果然把矛头转向了他:``师父,你说我是做大事的,从早忙到晚,不用提了;月牙才十七八,当家立计也不容易;就你是个闲人,你是不是也该干点什么?光天化日朗朗乾坤,你装什么闲云野鹤啊?''

无心的大眼睛在眼眶里左转右转,一脚门里一脚门外的站着,神情和姿态都很像一只落网的鸟:``你想让我干点什么?''

顾大人想了想,想不出该让无心干点什么。现在天下太平,老帅也很好伺候;他基本就是无忧无虑。

正是三人一起哑然之时,院门忽然被敲响了。无心和顾大人很坦然的原地不动,月牙则是不假思索的跑出去开大门。及至见了来客,月牙惊讶的``哟''了一声,原来门外胡同里停了一辆乌黑锃亮的新汽车,前后车门全都开了,两名小道士簇拥着一个披头散发的大个子,正是一身便装的出尘子道长。

月牙看出尘子和看活神仙也差不多,立刻连话都说不出了,张皇失措的去喊无心和顾大人。无心先出来了,满面春风的对着出尘子一点头:``道长,过年好。''

出尘子周身穿戴得华丽璀璨,乌黑长发就垂在貂皮褂子上面,褂子亮,头发也亮。淡然的一点头,他垂下眼帘低声诵道:``福生无量天尊,过年好。''

对着后方赶出来的顾大人一颤睫毛,出尘子算是打过了招呼,然后从袖子里抽出一只旧信封,直接送到了无心面前:``你寄给本道爷的,是什么玩意?''

无心听他语气有了变化,就知道老道可能是上门闹脾气来了。双手插到衣兜里,他没接信封,单是笑道:``令太师祖的遗迹,除了道长,我也无人可寄。''

出尘子一甩袖子:``百年之前的恩怨,与我无关!''

无心问道:``既然和你无关,你又何必要登门见我?''

出尘子叹了一声:``你当我愿意见你?若不是亲眼所见,我也没想到\ldots{}\ldots{}''

他是欲言又止,余音袅袅。无心追问道:``没想到什么?''

出尘子答道:``没想到\ldots{}\ldots{}她竟已与我近在咫尺了。''

无心把出尘子请进上房,让他慢慢的细讲。出尘子习惯成自然的盘腿坐在椅子上,摆了个打坐的姿势,唉声叹气的开了口:``我并没有亲眼见到她,可是我见到了一位姓丁的旅长。''

无心立刻问道:``文县的丁大头?''

出尘子点了点头:``是的,未料大年初一,我的道观里会迎来这样一位香客。''

无心微微向他探过了头:``丁旅长怎么了?''

出尘子放轻了声音:``他\ldots{}\ldots{}已经腐烂了。''

无心不动声色的点了点头,心想驱使一具尸体冒充活人,对于岳绮罗来讲,并不算是太难的事情。当然,她与傀儡之间也存在着一条无形的绳子,比如小春子走得太远,便会不再完全的听话。

出尘子继续说道:``丁旅长大年初一上山烧香,乃是多少年来定下的常例。烧过香后,时常还会在观里吃一顿素斋。贫道对他不算陌生,所以见了他的情形,十分心惊。''

无心端起了桌上的热茶,低头啜饮了一小口:``既然已经腐烂,想必再过些时日,丁旅长成了不堪的模样,文县的人马就不得不为他发丧了。只是岳绮罗失掉了丁旅长,接下来又要操纵谁呢?''

出尘子从鼻子里向外出冷气:``她操纵谁,我不关心。我只知道她纵是单枪匹马,神通也已远远的超过了我;如果再有了全副武装的军队,后果必定不堪设想。正所谓城门失火,殃及池鱼。城门在哪里姑且不谈;反正我是不想成为一条鱼。当然,贫道不是贪生怕死之徒,可青云观毕竟是太师祖的心血,如今传到我的手里,总不能让它毁于邪祟之手。''

无心认为出尘子忧患的很有道理。青云观是一片十分可观的大产业,天下道观何其多,可是能够穿貂皮坐汽车的住持道长,却是罕见。出尘子显然是打算一直舒舒服服的活到羽化登仙;让他像自己一样吃米粥咸菜炖猪尾巴,他肯定是不能愿意。

``令太师祖的符咒,我只抄写到了四分之三。''他对出尘子说道:``余下四分之一,道长能补出来吗?''

出尘子为了显示自己道行深厚,没好意思说自己为了研究符咒,闹了一个多月的失眠,并且毫无成果:``太师祖的符咒自成一派,想要补充,并不容易。''

无心笑着一点头:``道长加把力气吧!若能效仿令太师祖把她再封起来,是最好不过。''

出尘子又问:``还有其它方法吗?''

无心不再说话,单只是微笑。他看出尘子有点外强中干的意思,并且喜怒不定,所以有所保留,不肯实话实说。

\chapter{蛊惑}

正月十五的夜里,文县丁宅一片寂静,只有内宅深处的一间小院亮了电灯。

院中房屋是整整齐齐的三间,卧室客厅书房俱全。书房里面摆着一张很威武的大书案子,书案上面依次排列了笔墨纸砚。岳绮罗独自站在案前,背后白墙上挂着一副烟波浩渺的山水画,画上题了一句偈语,是她读厌了的两句:千江有水千江月,万里无云万里天。

她新近剪了头发,蓬蓬松松的打着齐刘海,像是从女子小学里走出来的半大姑娘。穿着一身绛红色绸缎裤褂,她微微侧身抬起右手,抄起毛笔蘸饱了墨,在面前的一张宣纸上写写画画。笔走龙蛇一气而下,最后一笔却是半途而止。重新审视了自己的作品,她发现自己又画了一张符。

灵魂虽然独立,可多少还是要受躯壳的影响。她老气横秋的叹了口气,然后从案角上的小玻璃碗里捏出一粒糖豆送进了口中。糖豆咯嘣脆,正适合她一口少年人的小白牙。一粒接一粒的吃起来,她感觉很寂寞。

她是不屑于和人相谈的,即便有心事,即便憋得慌。和``人''是没什么可说的,因为她认为自己超凡脱俗,已经不算人了。

无心的尸首在新年前夕彻底腐朽成了灰烬。当时子弹射得激烈,他的皮肉骨头被打飞了不少,导致岳绮罗没办法确认他是否真的彻底消失。无心显然也不是个真正的人,岳绮罗很想和他建立起一点感情,没料到他会说没就没。她想不通,感觉事情不应该是如此的简单;自己所见到的事实,也许并非事实。

房门一开,张显宗参谋长轻车熟路的走进来了。

张参谋长今年也就是三十来岁的年纪,看着不老不少,不丑不俊,乏善可陈,但也挑不出大毛病。走到书案前停下来,他微微俯下身,柔声问道:``绮罗,你怎么不吃晚饭?''

岳绮罗看了他一眼,感觉他好像爱上自己了。张显宗本来也算丁大头的心腹兄弟,不过后来的事实证明丁大头旅长是自作多情,因为张显宗在得知内幕真相之后,毫不犹豫的抛弃丁旅长,追随了岳绮罗。张参谋长没老婆没孩子,生平最爱小姑娘,逛窑子时专挑十三四的睡。岳绮罗倒是没和他谈过感情,不过他见了岳绮罗就双眼发直,是个从心眼里往外使劲的模样。

把桌上未完成的纸符揭起来放在一旁,岳绮罗压低了小女孩的童音,咕哝着答道:``我不饿。''

张显宗仔细端详着她的右眼,见眼珠上的红点子似乎有扩大的趋势,便问:``你最近身体不大好,要不要补一补?''

岳绮罗没有正面回答,另起话头问道:``丁旅长在哪里?''

张显宗轻声答道:``在外面站着呢。不冻不行了,我看饶是冻着,也支撑不了多少天了。''

岳绮罗又问:``你把事情办得怎么样了?''

张显宗诡谲一笑:``放心,一切尽在我的掌握之中。''

岳绮罗仰起头,长长的吁出了一口气:``好,可以筹备着给他发丧了!''

张显宗一点头:``是,我心里有数。''

岳绮罗往嘴里又丢了一颗糖豆,一边咀嚼一边含糊说道:``没事了,你可以下去了。''

张显宗答应一声,可是不动。于是岳绮罗从厚刘海下斜了他一眼:``你看我干什么?''

张显宗答道:``我看你好看。''

岳绮罗笑了,显出了薄薄的小嘴唇和单薄的小尖下巴:``不怕我?''

张显宗感觉自己像是聊斋里遇了女鬼狐狸精的书生,怕也认了,死也认了。至于岳绮罗到底是鬼是妖,他已经不甚在乎。豆蔻花开的小美人,是张参谋长眼中可遇不可求的尤物。

``我去想办法给你弄点好东西吃。''他着了魔似的说道:``你能让我取代旅座,我自然也要尽我所能的报答你。''

岳绮罗含着糖豆,不置可否的``嗯''了一声。

张显宗离去之后,岳绮罗在案上一沓字纸里面翻了翻,末了挑出一张巴掌大的小纸条。纸条上面用朱砂画了符咒。划根火柴点燃纸符,她念念有词的盯着火苗,及至将要烧到手指了,她将纸火猛然向外挥去。衣袖带动疾风,只见光焰最后一闪,随即和纸符一起化为乌有。

门外响起了沉重的脚步声,是两条腿在一步一步拖着走。丁旅长直挺挺的进来了,没有推门,是合身将门慢慢的顶开。人如其名,他的脑袋的确是大,因为院子里冷,屋子里热,所以他的大脑袋上立刻结了一层冰霜。脸皮本来已经烂得快要收拾不住,如今冻硬实了,又糊上一层霜,看起来正像是一座塑像,皮肤眼珠全是白的,是个没上颜色的坯子。

丁旅长是在一个多月前咽气的,咽气之前他已经类似一具行尸走肉。待到最后一缕魂魄也被驱逐出去,他彻底成了岳绮罗手中的傀儡。当时岳绮罗还没有和张显宗结成联盟,不能失了丁旅长做靠山,所以很小心的保护了他的身体,可是无论如何,一百多斤人肉冻了又化化了又冻,终究是保存不久。大年初一,她怕外界看出端倪,照老规矩安排丁旅长去了趟青云观。回来之后张显宗找到了她,说丁旅长真是不成了,烧香的时候一低头,差点把颗烂出脓血的眼珠子掉下去。

岳绮罗绕过书案,围着丁旅长转了一圈。她认定自己是要做大事的,所以需要吃点好的喝点好的,以及保护。丁旅长的使命已经完成,接下来的人选,就是张显宗了。

想起张显宗,她忍不住一撅嘴。张显宗对她太好了,让她简直有点不自在。

正月十六,丁旅长的死讯传出,死因说不清楚,仿佛是头天晚上一觉睡下去,第二天早上人就冷硬了。

丁旅长死的蹊跷,可是众人并不十分惊惶,因为他并不是横死的第一人。丁家的姨太太们死的死丢的丢,丁宅早在许久之前就成了凶地。丁旅长一死,部下众人虽然也哭也嚎,但是各有心思,全都怀了鬼胎。

文县里面暗潮汹涌,天津的老帅也立刻有了反应。直隶一带的几位小军头一直对他老人家不甚恭敬,他早就谋划着要一统直隶,只是对方兵强马壮,也都是硬骨头一类,并不能轻易啃动。如今丁旅长一完蛋,老帅就打算抓住机会,先对文县下手。打完文县再打长安县,把一溜繁华大县全攻下来了,他也就天下无敌了。

顾大人是被丁旅长从文县打出来的,此刻自然要受老帅的召见。与此同时,出尘子也回青云观了,带着无心。

出尘子在天津住了十来天,夜间在外国饭店下榻,白天坐汽车穿胡同找到无心,东一句西一句的闲谈。无心不说实话,他也不说实话,两人一团和气的互相敷衍。无心很有耐性,知道自己目前不是岳绮罗的对手,而岳绮罗又没有打上门来,所以根本不急。出尘子却是没有他的好涵养。临走之前,他忍无可忍,终于犹抱琵琶半遮面的吐了口风:``其实若想补齐符咒,也并非绝无可能。''

无心微笑着看他,对于下文是不问也不催,恨得出尘子瞪了他一眼:``贫道才疏学浅,不能领会太师祖所传道术之精华,所以先师羽化之前,曾经留下一份秘笈。也许从秘笈之中,能够窥出太师祖的\ldots{}\ldots{}''

话未说完,留了个尾巴。出尘子显然是难以措辞,沉吟片刻之后咂了咂嘴,仿佛刚刚吃了太师祖。

无心依旧不言语,伸手从桌上的盘子里抓了一把炒南瓜子,一粒一粒慢慢吃。出尘子是个成了精的老道,明明有求于他,却又拐弯抹角装模作样。所以无心按兵不动,倒要看看老道精还能发表出什么高论。

出尘子一狠心,把话继续说了下去:``但是想要拿出秘笈,非得进青云山不可。凭贫道一人之力,恐怕不足。''

无心吐出一片瓜子皮:``你有徒子徒孙无数,怎么会力量不足?''

出尘子摆了摆手:``好了,我不和你捉迷藏了。总而言之,我是希望你和我同去青云山中。先师的羽化之处乃是本派的大秘密,我不希望后人再去惊动先师。''

无心听出尘子说话前言不搭后语,心中就有些疑惑。但是岳绮罗是必定要除的,否则迟早都是祸患。如果出尘子真有办法,自己出手相助也是应该。

青云山不是遥远地方,位于长安县和天津卫之间。出尘子道长有汽车,所以干脆连火车都不必坐,随时可以出发。无心知道青云山不是大山,故而没把出尘子的话当回事。向月牙和顾大人道过别后,他和出尘子乘上汽车出了天津卫。汽车都开出百十里地了,他才发现了问题。

``不就是进山吗?''他问出尘子:``何必还非要到观里休整一夜?早去早回不好吗?''

出尘子当着随行的小徒弟,言简意赅的答道:``进山之后,还要入千佛洞。不提前做些准备,是不行的。''

无心眼睁睁的望着出尘子,从来没听说过青云山里还有千佛洞。

\chapter{洞中洞}

天下被称为千佛洞的风景胜地可是太多了,但无心在直隶混了几十年,从未听说过青云山里也有千佛洞。在他的印象中,青云山本来似乎只是一座荒山,如今山上除了青云观之外,也再无其它的建筑人家。此地的山连绵起伏,大多都不险峻,也谈不上壮丽,土产只有野菜野果和蘑菇,着实是不能吸引人去安居。

汽车疾驰在平直的土路上,因为连着几天不曾下雪,所以路面倒是好走;不过几个小时的工夫,一行人等便进入了青云山地界。出尘子带着无心在山门外下汽车换了轿子,到达观内之时,也才不过是下午时分。

无心住进了出尘子的小院,隐隐约约的感觉不对劲,很想向出尘子详细问一问千佛洞内的情形。然而出尘子关了房门不见天日。无心虽然看不到他,但也能听出他的忙碌——房内吱嘎乱响,显然他是一直在挪沉重箱笼。

天黑之后,连晚饭都吃过了,出尘子终于得了闲,立刻被无心逮了住。两人坐在红木大罗汉床上,无心捧着一杯热茶说道:``道长,你既然邀我来了,就该对我开诚布公。我想知道令先师作为一位有成的道长,为什么要把重要的秘笈藏进千佛洞?难道秘笈不是专要留给你的吗?''

出尘子刚喝了一大碗热汤,满头满脸的出汗。仰起头甩了甩一头乌黑长发,他用雪白的手帕轻轻一拭额角,同时飞快的斜瞟了无心一样:``因为\ldots{}\ldots{}先师羽化在了千佛洞。''

无心发现出尘子又要开始闪烁其词了,不禁有些不耐烦:``偌大的青云山难道没有好地方了不成?一个老道,非要死在千佛洞里?我怎么没听说过青云山里有千佛洞?你如果早说还要钻洞子,我未必会跟你来。''然后他抬手一拍身边的炕桌桌面:``你实话实说吧,千佛洞里到底有什么?为什么你一个人不敢进,非要找我做帮手?''

出尘子看他闹了脾气,不由得也跟着翻了个白眼:``我不知道!先师当时只嘱咐我,说我若是道行不够,就万万不要贸然进洞。非得有能从洞中全身而退的本领,才能领会秘笈中的奥秘。''

无心扭头盯着他:``你还没有回答洞里有什么!''

出尘子平日尊贵惯了,此刻见无心气色不善,就跃跃欲试的想要骂人:``他妈的我怎么知道?我只在洞口向内望过一眼!千佛洞千佛洞,洞里自然是有佛啰!''

无心继续盯着他:``不信你的徒子徒孙,信我?''

出尘子气的抬手一拢鬓发:``屁话!我怎么知道先师是如何羽化的?他老人家活着进洞再没出来,一旦有了万一之事,我又如何在徒子徒孙面前维护他们师祖的形象?''

无心恍然大悟的一点头,心里明白了。出尘子的师父大概是死的不清楚,万一洞里有着不可见人的秘密,曝露之后对出尘子和青云观都没有好处。而且观内多是修道之徒,苦修苦炼倒也罢了,万一窥见了精进之法,反倒容易闹出是非。

``秘笈是什么样子的?''无心缓和了声气,决定和出尘子讲和。

出尘子察觉出了他的善意,也跟着温柔了态度。抬手比出一本小册子的尺寸,他低声答道:``我最后见到秘笈时,先师还没有开始动笔去写,只预备出了一个本子。''

无心想了想,又问:``秘笈的封皮上,有什么记号吗?''

出尘子向他竖起两根手指:``上面有两个大字。''

``什么?''

出尘子一本正经的答道:``秘笈!''

无心当即``哦''了一声:``令先师还真是坦白。''

出尘子虽然说一句留两句,思前想后的不痛快,但是无心也大概明白了前因后果。一夜过后,他和出尘子凌晨起床。穿上顾大人淘汰给他的灰鼠皮袄,他一言不发的开始吃早饭。出尘子则是思虑周到,特地收拾出了一只雨布口袋,口袋里面装着饼干糖果,马灯水壶。用一根黑色缎带将长发绑成了个大马尾巴,他当着无心的面,从床下的箱子里翻出了手枪子弹。

无心不喝水,结结实实的往肚子里填米饭馒头。出尘子问他:``你要不要枪?''

无心摇了摇头:``我用刀。''

出尘子将一把勃朗宁藏在衣服里面,外面腰间又挎上了一只盒子炮。墙上挂着一柄宝光璀璨的短剑,被他伸手摘了下来。短剑出鞘,寒光凛然,竟然并非装饰用的样子货。出尘子叹了口气,用手帕将剑身擦拭了一番,然后站在炕桌前,挥剑切了半个馒头,没滋没味的咬了一口。

无心见了出尘子的种种准备,忽然有些紧张:``道长,不至于吧?''

出尘子没有食欲,吃过一口馒头就不吃了:``哼,但愿是我多虑。''

天还未亮,出尘子就领着无心出门了。

青云观倚着青云山,从道观后门溜出去,直接就进了山。要说人气,青云山比猪头山差得远,而且值此冬季凌晨,更是杳无人烟,连只野兽都不见。无心一边跟着出尘子连跑带跳,一边暗暗记忆了路线。末了出尘子骤然刹住了脚步,无心借着朦胧晨光向前一望,就见前方没了路,是直上直下的一段悬崖。两人往前又走了几步,这回看得更清楚了,原来悬崖并不高,更谈不上险,几乎就是一面比较陡的大土坡。

无心没有主意,所以询问出尘子:``怎么下去?''

出尘子沉吟着答道:``我当年是夏天来的,扯着树木藤蔓就爬下去了,很容易的。现在虽然没有藤蔓,但是树木还在,你我小心一点,总不会被它拦住。''

树木的确是在,歪歪斜斜的生长在土坡上,可是枝叶落尽,枝枝杈杈的让人不好攀附。出尘子率先蹲下伸出了一条长腿,蹬上了下方一根粗壮枯枝。试探着踩了踩,他也不在乎冰雪泥土了,身体贴着土坡慢慢的向下滑。末了他叉开双腿坐在了枯枝上。紧了紧口袋稳了稳心神,他既像猴子也像壁虎,一点一点的抓着树木往下蹭。

无心效仿了他,并且比他更灵活。有树枝抓树枝,没树枝就抓坡上的枯草根子,总而言之,不让自己下滑得太快。一番挣扎过后,两人灰头土脸的落了地,仰头再看大土坡,发现大概是因为阳光渐渐明亮的缘故,土坡看起来宛如一面直立的土墙,居然有了一点巍峨的气势。

无心搓了搓手上半融化的雪和土,转身去问出尘子:``然后怎么走?''

出尘子抬头看了看太阳,然后对着无心一挥手。无心立刻又跟上了他,两人沿着土坡一路前行。走着走着,却是见了一座小小的荒坟,看着就是个土馒头上面插了根木头牌子,牌子上面连字迹都模糊了。亏得是冬天来,如果夏天有了花草遮掩,简直会看不到它。

出尘子停住脚步,伸手对着坟头一指:``就是这里。''

无心没言语,等着出尘子作解释。而出尘子解下身上挎着的雨布口袋,从里面摸出一把崭新的小铲子。小铲子是花匠用来给花松土的,奇小无比,倒是够结实。出尘子蹲在坟前,开始挖土,一边挖一边说道:``坟是假坟,是我当年留下的记号。''

无心也蹲到了一旁:``假坟下面有什么?''

出尘子动作又狠又快,疯狂的挖土:``下面有一道天然的缝隙,容得下一个人出入。''

无心问道:``难道缝隙就是千佛洞的入口?''

出尘子摇了摇头:``不是。''

小坟头被他挖去一半,露出了齐平地面的一层铁板。无心看在眼中,不由得想起了猪头山中的鬼洞。而出尘子的智慧基本和顾大人不相上下,咬牙切齿的运力搬开铁板,他让无心看到了一条幽黑的缝隙。

缝隙不大,正能容得一个正常身量的成年人出入。大冬天的,地面全被冻成干硬,荒草残雪也都凝成一片;缝隙附近的泥土却是乌黑湿润,仿佛是大地受了伤。无心把手伸进缝隙,发现里面显然要比外界温暖潮湿。

出尘子心里有数,所以暂时还不惧怕,挎上油布口袋就要把腿往缝隙里伸,不料无心一把抓住了他:``道长,我有话说。''

出尘子抬头看他:``你反悔了?''

无心看着出尘子的眼睛说道:``道长,万一我在里面有了三长两短,你一定要把我的尸骨运回天津家中。记住了吗?''

出尘子点了点头,同时感觉无心居然比自己还要悲观:``没有问题。''

无心又指了指出尘子的鼻尖:``记住,千万别忘了。''

出尘子不以为然的答应一声,然后把脚探进了缝隙中:``别怕,里面是倾斜着的,也算个坡。你跟着我往里移动就行。''

无心背着出尘子出发时给他的短剑,眼看出尘子摇头摆尾的挤入缝隙里去了,便坐在地上,也把双腿伸进了缝隙中。缝隙里面总不会有大野兽,至多会藏着蛇;而他是不怕蛇的,有毒也不怕。

缝隙里面一片黑暗,无心躺在斜坡上,跟着出尘子一点一点的向下蹭。缝隙起初是狠狭窄的,上下几乎紧贴身,然而越向下越宽敞,最后无心坐了起来,凭着感觉追寻出尘子。蹭着蹭着出尘子不动了,随即下方起了一团昏黄火光,是出尘子把玻璃罩子的马灯取出来点亮了。

既然有了光,无心就可以使用眼睛了。扭头环顾四周,他发现周遭环境和地窖也差不多,只是斜下方黑黢黢的,依旧深不可测。出尘子换成了四脚着地的姿势,提着马灯往前爬。无心忍不住开了口:``道长,什么时候能到千佛洞?''

出尘子有些累了,气喘吁吁的答道:``快了。''

他说是快了,其实两人又爬了二十多分钟,土洞还是不见底。出尘子长胳膊长腿,爬得甚是辛苦;无心闭了眼睛,却是感觉十分惬意。他的每根汗毛都能感知四面洞壁的起伏,每一点起伏都可以让他轻而易举的借力爬行。游龙似的紧随着出尘子,他忽然一头碰了壁。出尘子回头怒道:``你撞我屁股干什么?''

无心连忙向后退了一尺:``怎么停了?''

出尘子蜷缩着坐起了身,怀里抱着马灯:``前头是大坑,我们得往下跳!''

马灯上的提手拴着细长的链子,出尘子先把马灯放下去了,然后自己纵身向下一跃。无心摸不清头脑,糊里糊涂的也跟着跳了下去。跳下去之后,才意识到自己进坑了。

坑能有个一人来深,还是土坑,大小比得过一间厢房。无心站在原地转了一圈,忽然问道:``道长,洞里挺暖和,也够湿润,而且不憋闷,怎么不生草木?''

出尘子拎起马灯:``我不知道,也懒得知道。如果不是为了秘笈,就算此地是金子砌的,我也不来!''

无心又道:``我们已经是在地下了,地下会有千佛洞?''

出尘子看了他一眼:``其实千佛洞是我杜撰出的名字,因为我当时在洞口的确是看到了佛像。长安县的县志上并没有千佛洞的记载,本地的山民也没在山中见过佛像。我怀疑洞子是许久之前挖掘的,不知为何半途而废,所以早早荒弃,传到如今,竟是无人知晓。''

无心发现出尘子其实比较无知,并且不思进取,简直不像岳绮罗一派的传人。而出尘子提起马灯沿着坑壁照了一圈,最后停在一处蹲了下去。无心跟过去一瞧,映入眼中的又是一个洞。

无心一屁股坐了下去:``道长,恕我直言,令先师的道行我不了解,可是钻洞的本事,的确高出了一般田鼠。''

出尘子也坐下了,从油布口袋里掏出了一包饼干:``不要妄言,再敢侮辱先师,当心本道爷揍你。我先吃几口,吃饱了好进洞。放心,这个洞子是口小肚大。当年贫道单枪匹马都敢往里闯,现在我们是两个人,你还怕了不成?''

\chapter{佛的笑}

出尘子坐在狗洞大小的黑洞旁,把马灯放到一旁地上照明,自己咔嚓咔嚓嚼了几十块油渍麻花的饼干,然后拧开水壶仰头灌了一大口水,呼噜呼噜的漱了一气,扭头``噗''的一声喷出老远。无心没想到出尘子嘴如水枪,如此有劲,几乎看呆;而出尘子是个讲究人,漱过口后又从怀里摸出一根细细的牙签,以手掩口开始剔牙。

等他重新收拾出一口大白牙之后,时间已经过去了半个多小时。他解开缎带重新系了长发,然后背好油布口袋,拎起马灯就预备钻洞。无心等得很不耐烦,如今见他终于有所行动了,连忙跪爬在了他的身后,又小声问道:``道长,钻过这一条洞,还要怎么走?''

出尘子的身量类似顾大人,肩宽背阔的,所以此刻极力缩了肩膀,想让自己的身段秀气一点:``如果贫道没记错,这条洞的尽头还是个坑。''

无心听在耳中,一时不知如何表态,只有三个字可以形容心情,但是又偏于不逊。舌头在嘴里动了动,他忍不住,还是说了出来:``他妈的!''

出尘子没有和他一般见识,在入洞之前又说了一句:``彼坑不同于此坑。''

无心立刻来了精神:``哪里不同?''

出尘子把脑袋伸进了洞口:``彼坑更大。''

洞通坑,坑藏洞,而且不见天日,怎么想都是危险地方。可无心随着出尘子入了洞,发现周遭除了潮湿之外,不但没有虫豸(念``zhì''),甚至连条冬眠的蛇都不见。出尘子爬着爬着,从怀里摸出一张纸符贴上了洞壁。无心听他隐隐的又喘起来了,忽然怀疑是空气有了变化,连忙向前问道:``道长,你感觉怎么样?''

出尘子头也不回的答道:``唉,累啊!''

无心又想起了一个问题:``道长,你今年贵庚啊?''

出尘子没理他。

无心看不出出尘子的岁数,如果出尘子有些年纪了,无心就打算帮他背包挎枪,减轻他的负担;但是他既然装聋作哑,无心懒得追问,正好省了力气。

这个洞子正如出尘子所描述的那样,口小肚大。出尘子爬行不久,便可放宽肩膀加大动作;再过了一段路途,他索性弯着腰站起来,拎着马灯向前一溜小跑。跑着跑着他停了脚步,却是脚下多了几级向上的石阶。

踏上石阶走出去,他昂首挺胸算是出了洞。无心跟在后方,一路走一路抚摸洞壁。洞壁本来都是湿土,可走着走着开始出现了层层岩石。无心心想千佛洞总不会是个土洞,石头一旦出现,可见千佛洞也应该是近了。

及至踩着石阶也出去了,他举目一望,不由得吃了一惊。原来他和出尘子是站在了险伶伶的一块大石头上,石头上方是黑漆漆的嶙峋穹顶,石头下方则是两人多深的大石坑。石坑底部坎坷不平,而正对面的石坑下方,正有一个深深的洞口。

出尘子伸手向洞一指:``那里就是千佛洞了!''

无心先不管千佛洞,只是上下左右的乱看。脚下的大石头凸出石壁,四面不靠,如何下到坑底就成了问题。直接跳下去,坑底不是软土,崴了脚扭了腿不是玩的;攀援下去,石壁上又光秃秃的不生草木,无处可以借力。

出尘子故技重施,还是拎着链子先把马灯放下去,链子不够长,马灯正好悬在了半空。只要有一点光明,出尘子就能摸索着扳住凸起石块,一点一点的爬下去。不过毕竟还是见老了,他记得自己当年爬得挺容易,如今却是笨手笨脚的很困难。

待到他落了地,无心把马灯收上去。用牙齿咬住马灯提手,他倒是比出尘子灵活许多。三下五除二的下到坑底,他把马灯交还给出尘子,然后径自就要往所谓的千佛洞口走去。出尘子连忙唤住了他:``慢着,不要莽撞!''

将一张纸符拍在石壁上,出尘子又要给无心也贴一张。无心摆了摆手:``道长,我不用。千佛洞你没进过,我也没进过,不知道里面是什么情形。纸符省着用吧!''

出尘子点了点头:``没有关系,我昨天把历年所画的纸符全都翻出来带上了,应该够用。''说完他俯下身,把手中的纸符放在了地上。

两人深一脚浅一脚的走向千佛洞。出尘子在洞口正前方拽住了无心,又把马灯递给他道:``瞧瞧,是不是有佛?''

无心伸长手臂送出马灯,借着玻璃罩子里的如豆之光,他向内望去,果然看到洞内左右分别立着一尊塑像。塑像兴许是不见天日、不受风雨的缘故,居然还保留着一层鲜艳的色彩。

无心提着马灯走上前去,近距离的仔细观察塑像。出尘子见他无所畏惧,就也跟了过来。塑像是位菩萨的形象,慈眉善目低垂眼帘,脸色粉白丰润,质地既细腻光滑,颜色也是又正又匀。出尘子第一次看清了菩萨的真容,心中就生出了许多感想,随口对无心说道:``不知是哪朝哪代的前人,竟然拥有如此精妙的技艺。如今的石匠,本领可是不行了!''

无心正在凝神留意洞内情形,所以只漫不经心的答了一声。菩萨像再老也老不过他,所以他没办法像出尘子一样欢喜赞叹。

出尘子欣赏够了两尊菩萨,然后一甩袖子,手里多了一柄小小的令旗。无心冷眼旁观,见他弯腰把旗杆往地面的石缝里插,就开口问道:``茅山道术?''

出尘子一摇头:``非也,本门博采众家之长,岂会拘泥于一派?''

无心把马灯放到了令旗前方:``道长,不必做法了,洞子深处我不敢说,可是百米之内一定安全,绝无邪祟。''

出尘子知道无心是有点本事,可是本事能有多大,他估量不出。拔出旗子拎起马灯,他深吸了一口气,然后把心一横,迈步踏入了洞内。

洞外一片乱石,洞内却是越走越平整。出尘子左顾右盼,忽然问道:``无心,你来瞧瞧,他们都是谁?''

无心望着石壁上的彩色图画,一边回忆一边猜测:``好像是十八罗汉\ldots{}\ldots{}''他用手指轻轻抚过连成一片的壁画:``没错,真是十八罗汉。''

出尘子笑了一下:``看来贫道很有先见之明,给它起名叫做千佛洞就对了!''

无心随着他慢慢向内走:``道长,令先师作为青云观的住持,在俗世里,也算是名利双全了,为何非要寻死?而且就算是活腻歪了,也没有死在千佛洞内的道理啊!''

出尘子放缓了脚步:``因为\ldots{}\ldots{}因为先师想要效仿太师叔祖。''

无心不动声色:``结果如何?''

出尘子低声答道:``结果\ldots{}\ldots{}结果先师走火入魔,不顾我的劝阻,一意孤行。''

无心不置可否的一笑:``要是把十八罗汉铲了,换成玉皇大帝元始天尊;或者是让令先师提前剃了头发去做和尚,就对劲了。''

两人边走边说,沿着洞内甬路走出老远。末了甬路一转,出尘子领先一步拐了弯,随即却是惊呼一声。无心瞬间赶上,就见甬路越发宽敞平坦了,两侧路边每隔几米就立着一尊一人来高的佛像。佛像紧靠洞壁,各有形象千姿百态;要说精致,不输于洞口两尊菩萨,而且也是颜色分明,乍一看仿佛是两队活人在夹道欢迎他们。

出尘子看在眼里,动在心中,暗想若是能把这佛像运出一样两样,用来送礼倒是真不错。黑暗中依稀看到佛像仿佛全是宝相庄严,并非狰狞的夜叉明王一类,他更满意了,因为佛菩萨更符合他的审美。

出尘子在看,无心也在看。起初的几尊佛像的确是华美飘逸,同洞口菩萨是一个风格。然而走出了几十步之后,无心忽然低低的起了疑声:``道长,看你身边的佛像!''

出尘子正提着马灯往前走,听闻此言,立刻原地转身。举起马灯一照佛像面容,他也愣了一下。

原来在不知不觉之间,佛像的容貌竟是有了变化。和方才几尊不同,出尘子就见眼前佛像虽然也是色彩鲜润,然而低垂的细长双眸却是向前睁开了,眼珠雪白,并未画出黑眼仁。

向前几步再照一尊佛像,佛像的眼睛越发睁大了,两边嘴角向上翘起,表情堪称欢畅。洞内前后都是无尽的幽黑,一点光芒托出上方诡异的佛脸,出尘子强定心神,转向无心答道:``我看见了。''

无心没有说话,抬手解下了背后的短剑。带着出尘子向前又走了一段路,他停下脚步,再次看向身边佛像。

佛像的面孔已经改成了青白颜色,神情冷酷,唯有一张嘴大笑咧开,显出一种意味深长的突兀。原地转了个圈,他发现左右的佛像全都在笑。

前方的路还未走完,不知到了最后,佛像会演变成什么恐怖样子。出尘子轻声开了口:``我看,洞子已经对我们做出警告了。''

他扭头望向了无心:``佛像的表情说明了一切。外面是平安的,所以佛像美丽;越往内走,越凶险。''

无心正视了他:``还走不走?不走的话我们马上出去,也来得及。''

出尘子叹了口气:``出去的话,秘笈怎么办?''

无心摇了摇头:``我当然是没办法。''

出尘子转向前方:``那还是继续走吧!''说完他一步迈出去,却是猛然绊了个踉跄。无心连忙扶住了他,两人低头一瞧,就见地上趴着一具小尸体,后脑勺上还拖着一根细细的小辫子,可见应该是个男孩。

无心拔出短剑,因见小男孩的两只小手还算饱满,仿佛并未十分腐烂,便把短剑贴地插到小男孩身下,想要把他翻过来。然而小心翼翼的试探了几次之后,却是不成功。最后他抽出短剑,直接伸手抓住小男孩的衣裳,强行把人拎了起来。

出尘子提着马灯一照,随即闭着眼睛扭开了头。不知道小男孩在地上趴了多少年了,脸上的皮肉竟然粘上了地面。无心一拎之下,小男孩的脸皮被生生撕下去了。

\chapter{一触即发}

看小男孩的穿戴打扮,绝不是近些年的人物,近些年的孩子们至少不会再留小辫子。出尘子不肯正视小男孩的面孔,单是提着马灯照亮;无心则是把小男孩放回地上,从头到脚的摸了一遍。一无所获的蹲稳当了,无心低头凝视着小男孩。看着看着,他伸手在小男孩的脸上抹了一指头。

血和外界是一个温度,他低头嗅了嗅,也是正常的血腥味道。千佛洞再怎么与世隔绝,其中的尸首也没有不腐的道理。可小男孩的确就是不腐——当然,也不是完全的不腐,然而烂得有限,除了脸皮与地面紧贴太久、不易分离之外,其余部分的皮肤都还堪称完好。

把小男孩的小手扯起来送到鼻端又嗅了嗅,隐隐的也有了臭味。无心把指尖的鲜血蹭到地上,然后站起身说道:``道长,你发现没有?洞里没活物。''

出尘子的目光避开了小男孩,深以为然的对着无心一点头:``不错。''

没活物,细皮嫩肉的一具小尸首摆在地上,连蛆虫都不生。无心又回忆了小男孩最初的姿势,发现对方仿佛正是在张牙舞爪的往外跑,跑着跑着一跤跌倒,跌倒之后就再也没爬起来。

他想得到,出尘子自然也想得到。两人对视一眼,然后心照不宣的一起向洞子深处望去。无心跨过小尸体继续向前,同时口中问道:``你师父好像是在洞里作了孽。''

师父不做脸,出尘子也无言回护。一边追赶无心一边环顾左右,他就见佛像的变化越来越大,虽然身姿还是庄严曼妙,然而面孔从诡笑渐渐转为狞笑,最后竟是大眼大嘴,如同鬼怪一般。出尘子常年的养尊处优,此刻就有点禁受不住。一甩袖子亮出令旗,他轻声向无心说道:``站住,前途凶险,让我测一测是否会有鬼魂作祟!''

无心宛如后脑勺生了眼睛,头也不回的低声斥道:``收回去!有没有魂魄,我比你先知道!''

出尘子伸手一指他的背影:``好哇,你敢呵斥本道爷!你——''

话没说完,出尘子忽然失了声音。姿态僵硬的伸着手臂,他发现前方的无心消失了。

出尘子虽然和无心言语不对付,但是也绝无把他丢在洞里的意思。举起马灯上下照耀了一番,他慌了神:``无心哪!无心?''

一只苍白的手忽然斜刺里伸出来,猛然抓住了他的手腕子。出尘子瞪圆了眼睛刚要叫,冷不防就听无心的声音响了起来:``该拐弯了,你拎着灯还瞧不清楚?''

出尘子无声的吁出一口长气,迈出一大步之后发现前方果然无路,真得向右拐弯了。无心先拐一步,站在原地等他,心想带他不如带顾大人。顾大人一旦被逼急了,还能散发出一种大杀四方的煞气,既能吓人也能吓鬼。出尘子倒是有道行,可惜道行稀松,胆量也稀松。

两人先前已经向右拐了一次,如今再拐一次,不知甬道究竟通往何处。出尘子眼看前方平坦清净,没有怪佛也没有尸首,拎着马灯就想上前;不料无心抬手一挡,低声问道:``道长,你看我们附近有没有能够镇压鬼魂的东西?''

出尘子怔了一下:``什么意思?''

无心闭上了眼睛:``一步之外,尽是鬼魂!''

出尘子听是鬼魂,反倒不甚怕了。高举马灯仔细检查了周遭,他最后对无心说道:``先师的确是布了锁魂的阵法。''

无心又问:``我能通过去,你能吗?''

出尘子当即从怀里摸出了一张黄符,``啪''的一声拍上了自己的眉心:``能。''

不知不觉之间,无心成了领头的人。闭上眼睛走在前方,他能够感受到身边的鬼魂如同气流,翻翻滚滚暗潮汹涌,分明是在极力冲击外来的活人。自己本无魂魄,不怕躯壳被夺;出尘子用一张黄符护在眉心,想必也是无恙。放缓脚步等待出尘子赶上来,无心突然想起了一件事情:``道长,阵法既是锁魂的,如果放了阳气重的活人进去,会有什么后果?''

出尘子想了又想,末了吹着面前垂下的黄符说道:``我不擅长布阵。''

然后两人一起扭头,又对了眼——一旦阵破,可不是闹着玩的!

心有灵犀一般,两人一起加快了速度,连跑带跳的往前冲。一口气跑到底,前方却是个左拐弯。无心知道出尘子全靠马灯照亮,不如自己灵敏,所以抓着他的手腕急转向左。面前不再是长长的甬道了,变成一间方方正正的石室。出尘子还未反应过来,无心先看清楚了——入口石壁上贴了几张纸符,而石室地上层层叠叠,堆满了尸首!

情况已经彻底明朗了,出尘子的师父想要自行悟出岳绮罗的本领,必然不能只是纸上谈兵。不知道他在洞内杀死了多少人,因为他需要魂魄,需要躯壳。岳绮罗在第一世时甚至自杀而死,出尘子的师父既然走火入魔不能自拔,恐怕下场也是不堪设想。尸首和魂魄被阵法分隔开来,想必也是有个缘由在里面。

无心不敢停留,拽着出尘子继续跑。出尘子刚刚看清现实,就身不由己的抬腿踩上了尸首。尸首类似外面甬道内的童尸,都是前清时代的形象,男女皆有,只不过全是俯趴着,所以看不到脸面。人在初死不久之时,身体会逐渐冷硬,然而过个一天半夜,又会慢慢恢复柔软。出尘子一脚接一脚的踏下去,总感觉自己要往下陷。无可奈何,他只好高抬腿大跨步,一颠一颠的向外窜。无心的手像铁钳一样,狠狠攥着他的腕子。两人一阵风似的掠过尸堆,末了无心在石室另一端的出口跳下去,左脚在一具尸首上绊了一下,正好将尸首踢翻过来。出尘子借着火光低头一瞧,心中又是一凛——是女尸,眼睛鼻子都被利刃旋下去了,白脸上留下三个鲜红的窟窿。死得越惨,怨气越重,鬼魂越凶。施加在女尸身上的虐杀,大概就是由此而来。

石室方方正正,出去之后碰了壁,原来还得向左拐弯。无心见前路又是一条甬道,便也顾不得许多,带着出尘子继续狂奔。

两人接连又拐了三个弯,路上平安无事。最后走投无路了——前方立了两扇铁门。

铁门关得严丝合缝,就和一般人家的院门差不多。出尘子走上前去,提了马灯上下的照,最后连个门把手都没找到。转身面对了无心,他皱起两道长眉唉声叹气:``先师的心意,真是莫测。''

无心也犯了难。甬道本来就够宽敞,铁门又是顶天立地,上下都没有可供翻越的空间。没有把手,也没有锁眼,让人撬都没法撬。出尘子摘了眉心的黄符塞进袖子里,和无心两个人在门前来回的走。正是无计可施之时,无心忽然在一扇门前停住了脚步。背起双手仰起了头,他忽然踹出一脚,只听一声铿锵之响,门扇竟是向内开了半寸。

出尘子目瞪口呆,原来铁门真就只是两扇铁门,其中并无玄机。而无心站稳之后点了点头:``令先师的心意,果然莫测。''

两人当即一起动手,拼命推动一扇铁门。铁门虽然也有门轴,但是门板沉重,上下又摩擦着石壁,所以很不灵活。等到门缝容得下一人侧身出入了,两人便鱼贯而进。进入之后,无心眼前一亮,发现了一片新的天地。

原来面前又是一间石室,四壁平整,很有房屋的意思。门口两侧忽然腾起了火苗,是出尘子发现了石壁上突出的油灯支架,灯里还存着半碗油。

有了油灯照明,无心的视野就更清晰了。前方的情景很像出尘子的会客室——正中央摆着一张木制罗汉床,床上盘腿坐着一位须发皆白的老道。老道身着黑袍,本是个打坐的姿势,然而一个脑袋歪得诡异,几乎快要侧枕到一边肩膀上。

出尘子微微弯了腰,蹑手蹑脚的走上前去细看老道的面容。无心也跟上去了,轻声问道:``是令先师吗?''

出尘子张着嘴回过头,颤巍巍的一点头:``是。''

然后他抬手掀起了老道的长发,就见对方的脖子上赫然一道刀伤,不但砍断筋脉,甚至连颈骨都几近断裂。鲜血早流尽了,白生生的骨茬翻出来,看着正是惊心动魄。

出尘子低头长出了一口气,抬起的手颓然落下;同时发现师父的道袍之所以看起来是黑色,乃是因为被鲜血浸染透了的缘故。

斯人已逝,而且逝了好几十年,导致出尘子哀而不伤,做不出哭天抢地的样子。无心则是把目光移向了罗汉床两边。两边和普通的居室一样,摆着桌椅橱柜书架子。无心走到书架前,发现架子上并无灰尘,书本也都排列得挺整齐。一本一本的抽出来看了封面,他开始寻找秘笈。

他行动了,出尘子也不闲着,伸手去摸罗汉床上铺着的被褥,又掀了师父的袍子下摆往里看。忽然``啊''了一声,他在床角的被褥下面翻出了一本薄薄的册子,口中说道:``找到了!''

无心连忙过去观看,就见册子封面上果然只有``秘笈''二字。出尘子翻开册子,里面并无文字,一页一页都是符咒的图案。一般的符咒多是画在长纸条上,册子上的符咒却是与众不同,无边无际的布满整页,一眼看去,如同乱麻一般。

抬眼望着出尘子,无心问道:``看得懂吗?''

出尘子沉吟着点头:``略懂。''

无心还要继续说话,不想话未出口,耳边却是响起了金石之声。觅声望去,他只见一条手臂直挺挺的从门缝中伸了进来,随即上方又拱进了一个女人脑袋,脸上赫然三个血窟窿!

出尘子登时一哆嗦,而无心瞬间明白了,立刻冲向铁门:``糟了,借尸还魂!''

未等他双手触到铁门,出尘子拔出手枪对准门缝,扣动扳机就是一枪。子弹力道极大,打得女尸向后一仰。而无心抓住机会推动铁门,竭尽全力的想要把门重新关严。哪知他一股子力气还没使完,旁边一扇铁门也有了动静。出尘子合身扑上来顶住铁门,苍白着脸问道:``借尸还魂?''

无心双脚蹬地,背靠铁门:``锁魂的阵法一定是出了问题,洞里的鬼魂现在自由了!''

外面不知来了什么东西,咚咚的在撞铁门。出尘子效仿无心,也用双脚蹬地借力。马尾巴辫子散开了,他一头一脸全是头发:``怎么办?''

无心简直快要忘了喘气:``不知道!''

\chapter{碰壁}

无心和出尘子都是身强力不亏的人,然而双拳难敌四手,好汉架不住一群狼。况且狼还有着趋利避害的本能,外面那一群行尸走肉却都是撞了南墙也不回头的家伙。无心背倚铁门,就觉得身后一股子推力十分强大,自己的双脚蹬在石板地上,身不由己的在一点一点往前蹭。

出尘子比他个子大分量重,这时倒是稍稍有了一点优势。抬起一只手摸进怀里,他咬牙切齿的开口说道:``无心,我们不能坐以待毙。我这里还有许多驱鬼的符咒,多少总会有点效验。''

无心立刻扭头望向了他:``怎么用?''

无尘子深吸一口气,合身猛的向后一顶,把已经被微微推动的铁门顶回了原位:``只要贴到外面门板上就行!''

无心登时拧起了眉毛。一门之隔便是凶残的活死人,谁有本事跑出去贴纸符?只怕是刚一露头,便被尸首们撕碎活嚼了!

可是不冒这个险也不行,行尸走肉们打起冲锋可是不含糊的,横竖已经全是死人,在铁门上撞成粉身碎骨也不在乎。对着出尘子一点头,无心决定和他分工协作,务必做成此事。

三言两语的交谈过后,出尘子咬破舌尖,连血带唾沫的在纸符后面舔了一大口。把纸符交给无心,他把贴身藏着的勃朗宁小手枪也拔出来了。忙里偷闲的重新扎好马尾巴辫子,他双手握枪,提起了精神。

无心的身体略略松了点劲,让一扇铁门缓缓开出缝隙。一转身面对了铁门门缝,他就见门外黑黢黢的全是人形,借着门内两边的油灯光明,他就见一张铁青的面孔直凑上来,鼻子是鼻子眼睛是眼睛的,很有几分人样子;不料在发现门缝不足以让脑袋探入之后,这位体面的活死人骤然张大了嘴,一口便咬上了铁门边缘。无心不敢耽搁,顺着门缝伸出左手臂,``啪''的一声就把纸符贴到了铁门外面。手背随即一痛,正是被其它尸首的指甲抓伤了皮肉。

无心连忙缩回了手,而出尘子抓住时机瞄准门缝,一枪先把咬铁门的活死人轰了个倒仰。接连扣动扳机又打出几枪,他虽然不能把外面的尸首再杀一遍,但是利用子弹的冲力,倒是把积极进门的几位全轰了个东倒西歪。无心收回手推动铁门,使出拼命的力气把铁门重新关好。倚着铁门等了片刻,他发现自己这一边果然是安静了,可是出尘子所顶住的另一扇铁门则是情形不妙,被外面的行尸走肉冲撞的咚咚直响。显见一张纸符还不够用,须得再加一张。

出尘子趁着舌尖伤口还新鲜,又取出一张纸符舔了一口。两人故技重施,无心故意把手背伤口渗出的淡淡鲜血抹开了,然后在出尘子的双枪掩护下,险伶伶的再次开了门缝,贴出纸符。果然,这次他没有再受伤害;出尘子也不含糊,枪枪不落空。眼看中弹的尸首摇摇晃晃的又要爬起来了,他和无心一起发力,大喝一声推动铁门,把两扇大门彻底关严实了。

门外渐渐的安静了,但也不是纯粹的没了声音。尸首们依旧在动,只是不再冲击铁门。出尘子松了口气,无心留意到了,也跟着发出一声叹息。

``不知道两道符咒能够抵挡多久。''出尘子把手枪重新掖回身上,一张脸上没有血色,但是眼睛亮了许多,仿佛经过一场战斗之后,精气神全上来了。

无心背着短剑,在石室里面兜了个圈子:``道长,我们一路走到这里,并没有看到岔路,难道千佛洞是个独眼洞子,有进无出?''

出尘子定了定神,明白了他的意思:``照理来讲,应该另有出去的通道。否则洞子成了一条死胡同,外面又困着尸首和鬼魂\ldots{}\ldots{}''

无心抽了抽鼻子:``道长,洞里的空气一直不算坏。''

出尘子连连点头:``是了,仅从这一点看,也不能真是死胡同。''

无心解下短剑,又把剑鞘递给了出尘子。两人分别握着剑与剑鞘,各自轻轻去敲墙壁地面,想要通过声音找出密道。石室的面积有限,几分钟后两人在罗汉床前会了面,全是一无所获。

无心蹲在地上,把短剑伸到罗汉床下又敲了敲,依然是声音沉闷。仰起脸望向床上的老道人,无心颇为沮丧的一屁股坐了下去:``道长,你和你师父倒是不大一样。''

出尘子彻底没了主意,跟着他也坐下去了:``我和师父的确是\ldots{}\ldots{}志趣不同。''

无心对着铁门一指:``你就不想练出几招呼风唤雨的法术?''

出尘子侧过脑袋,抬手扯下缎带,放开一头长发甩了甩:``我擅画符。''

无心微笑着看他:``如果你有志于学,我可以给你介绍个好师父。你太师叔祖——''

出尘子立刻很不耐烦的一挥手:``不要说了,用人性命修炼法术,想一想都令我感觉恶心!''

无心偷偷把左手藏在身后,因为手背上轻浅的伤口正在愈合:``道长倒真是个慈悲为怀的人!''

出尘子理直气壮的一昂头:``当然!我现在有电灯有电话,吃外国饭菜坐外国汽车,朋友不是总统就是总理,督军们见了我都一团和气。凭着本道爷如今的身份地位,怎么活都是风光无量,何必还要去研习什么法术?另外我是讲卫生的,让我守着尸体住山洞我会吐!''

无心早就看出尘子入世太深,不像是岳绮罗一派的人;如今听了他一番回答,更放心了。而出尘子把地上的马灯拎到近前,将事先揣进怀里的秘笈取出来一页一页的翻看。翻到最后他抬起头,急赤白脸的把秘笈往腿上一拍:``这哪里是一时三刻就能领悟透的?''

无心知道出尘子日子过得舒服,一定分外惜命。眼看出尘子所带的油布口袋就放在地上,他伸手将其拽过来,从里面掏出了饼干吃。一边吃一边说道:``道长,令先师是自杀吧?''

出尘子早在几十年前就认定师父是死了,所以现在虽是和师父的尸体同处一室,却也毫不动心:``应该是的。先师死状惨烈,大概也是为了刺激魂魄汇聚,以免消散。''

无心缓缓咀嚼着饼干,又问:``那令先师的魂魄,如今又在何处?''

出尘子自从进洞之后便是心慌意乱,此刻登时就被他问住了:``这个\ldots{}\ldots{}''

无心说道:``除了锁魂阵内,洞中其余地方都很干净,连零碎的魂魄都没有,可见令先师法术未成,大概死后便魂飞魄散了。''

出尘子没言语,心里认为师父也算倒霉催的。

无心咽下饼干,又拍了拍手上的饼干渣子:``道长,起来吧,我们再四处瞧瞧。怎么房里连饮食都没有?令先师当时已经辟谷成仙了?''

话音落下,他忽然意识到自己说了废话;而出尘子看了他一眼,脸上的血色又褪了一层。房内虽然没有五谷杂粮,房外却有一大群待宰的活人。出尘子的师父如果想要填饱肚子,倒也容易。

出尘子一贯仙气飘飘,没想到从太师叔祖到师父,接连着给自己丢人现眼。灰头土脸的站起来,他没敢搭茬,一边整理长发一边走去书架前。将架子上的书从头到尾翻了一气,最后找到了一本薄薄的大册子,翻开来却是一张地图。

将地图浏览了一遍,他忽然惊喜的出了声:``无心,你来看看,这是不是一条秘道?''

无心连忙起身走了过去。出尘子弯腰把地图铺在地上,一根手指点在下方:``两边画了小人,大概就是洞口的两尊菩萨。对不对?''

无心点了点头,然后顺着洞口向上看。地图画得十分简略,但是清楚明白。最后到了头,图上赫然标了铁门的记号,铁门上方画了个方块,想必指的就是这间石室。石室仿佛是千佛洞的最末端了,然而仔细看去,正对着铁门的墙壁上又用虚线描出了一道小门。两人一起抬头望去,发现若是小门当真存在,就必然开在了罗汉床的后方。罗汉床的床围高大,而且是紧贴着墙,方才竟是被他们忽略了。

两人心中立刻有了光亮,低头再看地图,却是立刻又傻了眼。原来虚线所示的小门后方,虽然也画出了两条线表示通道,然而纸张有限,通道就只延伸到了地图边缘。进入小门之后是生是死,竟然成了悬案!

出尘子看在师父的面子上,强忍着没有破口大骂;于是无心无声的翕动嘴唇,替他骂了。地图实在画得可恨,显然是在下笔之前根本没有考虑过布局,画到最后无处可画,也就算了。

无心存着一份希冀,还问出尘子:``会不会另有半张地图?''

出尘子把嘴撇得像鲶鱼似的:``先师的性格我最了解,素来是顾头不顾尾,而且没有长性,半途而废的事情做得多了!''

无心回头望向罗汉床上的老道人,心想知师莫若徒,看他的所作所为,也的确是个不着调的。对着出尘子喘了一口气,他转身走向罗汉床:``道长,咱们先把床搬开看看!如果真有暗门,再做打算!''

出尘子立刻跟上。一边走一边抬手摸了周身上下,确定纸符在,秘笈在,手枪也在。无心看了他的行动,立刻把短剑入鞘也背好,并且把油布口袋和马灯拎起来放在了角落里。

出尘子虽然对师父满腹怨言,但是师父毕竟是师父,不敢轻慢。无心知道他的顾忌,所以亲自动手,先是小心翼翼的把老道人抱起来放到地上,然后才和出尘子左右夹攻,使出吃奶的力气挪动了大罗汉床。

大罗汉床虽然沉重,但毕竟是木头制的,不会重得没边。出尘子常年养尊处优,到了动真格的时候,才看出他平日的保养并非无用功。无心的力气则是稍逊一筹,好在会使巧劲,摇摇晃晃的倒也不拖后腿。

两人拼死拼活的搬开了大罗汉床,床后的石壁显露出来,果然在半人高处有一道紧锁着的小铁门。小铁门方方正正,尺寸形状都类似于一张大棋盘,怎么看都不是给人走的。门上挂着个黄铜小锁,锁是老锁头,显然不足为惧。

无心先走到门前弯了腰,伸手去摸锁头。试探着拽了两下,他转身去问出尘子:``道长,会开锁吗?''

出尘子瞪了他一眼:``我又不是飞贼,怎么可能会开锁?''说完他从衣兜里摸出一根牙签,凑过来要去捅锁眼。无心见了,便嘱咐道:``道长,你先忙着,我再去到处找找,看看能不能找到钥匙!''

然后他当真起身找了一圈,连老道人的道袍都掀开了,可是连根钥匙的毛都没有找到。出尘子忽然``哎哟''一声,面如苦瓜的回头告诉无心:``牙签折在锁眼里了!''

既然文的不行,只好动武。因为不知门后到底是什么情形,所以出尘子先将一张纸符贴上门缝,然后给无心让出位置。无心拔出短剑,开始试着去砍小锁。砍过几下之后他有了准头,挥起短剑用力一斩,只听``铛''的一声,火花四溅,小铜锁已经落了地。

无心没有贸然开门。闭上眼睛靠近门缝,他静候良久,并没有感觉出异常的空气,才将剑尖插进门缝,轻巧的向外一撬。铁门带着纸符立刻开了,门轴略微有一点锈,发出了刺耳的吱嘎声音。无心率先探头向内一望,出尘子也赶上来了,跟着他一起窥视。

下一秒,两人扭头对视,面面相觑的全傻了眼。原来小铁门后只有一尺多长的空间,空间尽头,又是石壁!

\chapter{破壁而出}

无心张着嘴,解下短剑伸进小铁门内,轻轻的捅了捅石壁。出尘子弯腰站在一旁,不但也张着嘴,并且连带着伤的舌尖都露出来了。

捅过几下之后,无心扭头说道:``道长,真是石头。''

出尘子眨巴眨巴眼睛,好像没听明白似的:``啊?''

无心手上加了力气,用力去杵石壁。叮叮当当的声音闷闷的传出来,无心聚精会神的侧耳倾听,听着听着他开了口:``道长,你听出问题了吗?''

出尘子摇了摇头:``我\ldots{}\ldots{}我听不出来,你再敲几下!''

无心把整个脑袋都伸进了小门里。单听声音,仿佛石壁后面还有空间,但是无论如何,石壁必定很厚。他撤回了头,又把一条手臂伸进去仔细的摸了一遍,发现石壁四边似乎是有缝隙。

``应该会有活路。''他对出尘子说道:``道长,你能不能开枪打碎里面的石壁?''

出尘子推开了他,俯身向内看清了深度,又伸手进去推了推石壁,最后直起身摇了摇头:``不行,太危险了。子弹未必能够穿透石壁,反倒很容易引发跳弹伤人。''

无心正要说话,不料后方忽然有了响动。两人回头一瞧,发现声音来自门外,仿佛是有行尸走肉又要跃跃欲试的来冲撞了。

无心有点急了:``道长,真不成吗?''

出尘子出了一头的冷汗:``教我射击的人是总统府的侍卫官。他当时说得很清楚。子弹一旦在石头地上跳起来了,不一定就会伤到哪个方向的人,防不胜防啊!''

大铁门外传出``咚''的一声响,好像是有什么小东西敲到了门板上。无心连忙转身扑到门前,用后背顶住门缝,同时心中一动,突然有了主意。对着出尘子连招了招手,他把对方叫到近前,然后问道:``你有没有办法,能够封住自己的阳气?''

出尘子飞快的思索了一瞬:``我只能封住自己的魂魄,闭气能闭两分钟。''

无心一点头:``好,够了!等下我们放进一只活死人,他们的力气大,让他们帮我们打破石壁!''

一绺长发当即落下来遮住了出尘子的眼睛:``你做梦哪?''

无心一瞪眼睛:``去!把你的血涂在门里石壁上,多涂一点!''

出尘子望着无心,就见他的眼珠子忽然变得又黑又亮,在微微凹陷的眼眶里闪闪发光,竟然带了几分闪烁不定的妖相。一拍脑袋明白过来,他转身直奔小铁门前,挽起衣袖露出雪白的手臂,他用小匕首轻轻划破了皮肤,忍痛挤了鲜血往里面石壁上涂。活人血自然带着活人气,及至伤口被他挤得又红又紫了,他还意犹未尽的弯腰向洞内吐了两口唾沫。放下袖子转向无心,他开口问道:``然后呢?''

无心抬手轻轻一拍自己的眉心:``你做准备,站到我这边来。''

出尘子一边摸纸符,一边快步走向了他:``你怎么办?''

无心摆了摆手:``不必管我。我现在就要放尸首进来了,尸首只要一进门,你就立刻屏住呼吸,知不知道?''

出尘子把先前用过的黄符贴上额头,然后直挺挺的站在了门旁的角落里。无心手指摸到门缝,将一扇铁门慢慢扳开。外面的行尸虽然受了符咒的震慑,不敢硬闯,可是门缝一开,里面分明有着活人的气息,行尸们出于本能,便要往门缝里冲。

无心动作灵活,只容一具中等身量的男尸进入。男尸刚一进来,他便拼命关了铁门。出尘子见真家伙来了,立刻屏住呼吸一动不动。男尸原地转了个圈,额头眉目都算完好,从鼻梁往下却是像被镪水蚀过一般,皮肤破烂不堪,两排交错的牙齿齐根露出,连嘴唇都没有了。行尸本来也没有思想,完全是受附体鬼魂的支配。鬼魂若是完整,倒也罢了;可锁魂阵内的魂魄本来就是乱七八糟,如今胡乱附上尸身,也无思想智慧,全是凭着一股子戾气去追杀活物。无心缓缓后退,挡在了出尘子身前,而男尸终于找到了目标,拖着两条腿向前方小洞走去。

洞里涂了新鲜的人血,对他来讲,正是一种刺激。踉跄着走到洞前弯下腰,他把双手伸进洞内,拼命抓挠起了染血石壁。出尘子不敢分心多看,专心致志的调理内息,想要多支持一阵。而无心就听洞内传来了窸窸窣窣的细微声音,片刻过后,男尸开始伸头向内探入,然而宽阔的肩膀竟然卡在了洞口。无心就见男尸的肩膀渐渐变形,竟是男尸不知疼痛,强行挤了进去。一阵连续的闷响过后,男尸的肩膀益发深入,同时从洞口四面的缝隙中溢出了脑浆骨肉,显然是男尸已经将自己的脑袋撞碎了。

出尘子此时忍无可忍,伸手去拍无心的肩膀。无心没有回头,直接一指男尸:``道长,去把他镇住!''

收拾一具单枪匹马的男尸,对于出尘子来讲,还是不成问题的。他喘着粗气快步上前,将一张纸符贴上了男尸的屁股。男尸立刻就不动了,无心看得清楚,只见一团杂乱无章的光芒在男尸身上汇聚成团,越来越暗,正是魂魄已经受到了镇压。

在出尘子大喘特喘的空当里,无心扯腿把男尸拽了出来。男尸是真卖力气,头颅肩膀以及手臂都碎了。出尘子怕脏,不肯上前;于是无心挽起袖子,一边把里面的散碎骨肉往外收拾,一边去推尽头的石壁,末了发现石壁居然歪了,一侧已经有了半指宽的缝隙。把男尸扔到了出尘子的师父身边,无心让出尘子来看洞内情形,又说:``放心,没有血,骨头渣子和碎肉都被我捡出去了。''

出尘子听了他的安慰,差点没把胃里的饼干吐出来,随即表示绝对不看。

无心无可奈何,只好继续自行研究,又把短剑伸进缝隙里去扳去撬。堵住洞口的石壁,目前看来已是石板无疑,兴许是相当的厚,所以单凭无心一人之力,实在难以撼动。无心心疼出尘子的短剑,怕把剑损坏了,所以试探过后拔剑出来,开口说道:``还得再放一个进来,否则凭着我们的肉体凡胎,实在是打不通出路。''

出尘子闭目做了个深呼吸:``好,就再放一个!''

无心故技重施,又冒险放进一具女尸。女尸低着头往里冲,冲进来后无心和她打了个照面,才发现对方一张大白脸,脸上仨窟窿,还是个熟人。出尘子方才又往洞里放了血,此刻正直挺挺的站在角落里装死;无心身上根本没有人气,所以女尸觅着血腥,效仿男尸前辈,直奔前方小洞而去。女尸生着单薄的削肩,一头扎进去,咣咣咣的狠撞一顿。及至出尘子把一张纸符摁到她的屁股上时,她已经从头碎到胸口,彻底的不可收拾了。

尸体不知为何,体内全都没有淤血,所以倒是易于清理。无心把一条手臂伸进去又探了探,脸上显出了喜色,原来石板已经大大的移了位置,足以容得一个脑袋通过。

让出尘子把马灯拎过来,他打算亲自去瞧一瞧壁后风光。俯身先把马灯送进洞里,他由手臂而头颅,由头颅而肩膀,一点一点的向内挤去。出尘子很紧张的双手握拳站在一旁,眼看他的肩膀也蹭进去了,便忍不住出声问道:``看到什么了?''

无心的声音听起来很遥远,是一声粗哑的``哇''。出尘子听他语气有异,连忙预备出一张纸符握在手里:``怎么了?''

无心乌鸦似的大声叫道:``哇!屎!''

出尘子没听清楚,疑惑问道:``什么死了?''

无心摇头摆尾的退了出来,抬头对着出尘子说道:``不是死,是屎!''他伸手一指洞口:``外面下边砌着石台,台子上面摆了一排马桶,马桶里面全是屎!''

出尘子难以置信的看着无心:``怎么会有马桶和屎?你看准了吗?''

无心本是个百无禁忌的人,可是此刻一回想方才的所见所闻,还有点作呕:``不信你去尝尝?''

出尘子急得一甩袖子:``呸!你说是就是吧!可是除了马桶和屎,还有其它东西吗?''

无心一指洞口:``我说道长,这里面不会是你师父的茅房吧?''

出尘子快要被他质问的落下泪来,摊开双手反问道:``茅房要加一道石门一道铁门吗?''

他急,无心也不耐烦了:``令先师素来出人意表,什么事情干不出来?''

出尘子把嘴一张,刚要反驳,不料忽听一声巨响,两扇铁门竟然瞬间洞开!眼看外面乌压压一片行尸走肉狰狞而至,出尘子在一秒钟的恐慌过后,挺身而出挡在了无心前方。右手向前一甩,他第三次亮出了他的小令旗,同时左手背过去一推无心:``进洞,快逃!''

话音落下,他将令旗向下一掷,纤细旗杆竟然笔直的立在了石板地上。随即扬手把一沓纸符挥洒开来,他站立在纷纷落下的符雨中,生疏笨拙的结起手印:``临兵斗者、皆阵列前行!''

白色纸符在幽暗石室中飘飘摇摇,行尸走肉们像被施了定身法一般,统一僵住了姿态。与此同时,无心大喝一声,硬是将洞内石板又推开了几分,容得下他侧身钻出。伸手拎起马桶随便向两旁的无尽黑暗中扔了出去,他险伶伶的爬出去站在石台上,然后对着室内大声喊道:``道长,快来!''

出尘子的道术早已荒于嬉,如今镇住群尸,已经是快要累出尿。听到了无心的呼唤,他不管三七二十一,转身就往洞里钻。他是大个子,照理来说无论如何不可能通过如此小洞。可是人到了生死关头,往往能成不能之事。他缩肩弓背的向内硬挤,竟然真把脑袋和肩膀拱出了出口。无心已经又踢飞了几只马桶,落脚之处宽敞许多,此刻就抱住了他,不由分说的往外硬拽。

\chapter{道长好怕}

石台面积有限,无心把出尘子拉扯出来之后,两人便险伶伶的只有了立足之地。出尘子把手伸回洞内,对着四壁接连拍出纸符,想要再布一道阵法阻住行尸走肉。而无心环顾四周,就见一片虚空茫茫,洞内的光线射出来,竟被无边黑暗吸收了个一干二净。

到了此时,眼睛就没了用处。他站立稳了,开始伸手四处试探。洞口方方正正的带着边框,有些类似窗户。无心向上一摸,发现窗户上方却是个斜斜的石头坡。坡虽然很陡,但是石块起伏嶙峋,总有手抓脚蹬之处。至于坡有多高,通往何处,可就推测不出了。慢慢俯身蹲了下去,他小心翼翼的向下又伸了腿。腿伸到极致了,一只脚还是没着没落。忽然想起不远处还有几只马桶,他轻声对出尘子说道:``道长,你把马桶踢下去一只,我要听一听高度。''

出尘子会意,立刻斜着飞出一脚。马桶本来就是满满登登的,如今落得分外有劲,挟着风声快速下坠。无心侧耳倾听良久,末了却是没有听到马桶落地的声音。

他听,出尘子也在听。听到最后两人全出了冷汗。洞内传出了响动,是石室内的行尸走肉起了骚动,马上就要突破第一道阵法。无心知道自己再无时间迟疑,索性开口问道:``道长,你能不能看见我?''

出尘子紧贴洞口站直了,因为得知石台之下深不可测,所以恐慌的有些腿软:``看不清楚。''

无心抬手向上抠住了一块突起石头:``看不清楚也没关系,跟着我的声音往上爬!既然洞里是走不得了,我们就另找道路吧!''

话音落下,他抬脚踩上了洞口边沿,当真向上爬去。出尘子是走投无路了,并且全然没了主意。仰头看清了无心的动作,他立刻效仿,也跟着爬了上去。爬了没有多远,洞口的光芒便已完全消失。两人彻底陷入黑暗,无心每隔片刻便要出一次声,紧随在后的出尘子听见了,也连忙作出回应。两人像鸟似的一应一答,全凭着声音确认对方的方位。

爬了良久过后,坡势渐缓,两人心有灵犀的一起停了下来。无心坐起身向下伸出手,摸索着把出尘子拽到了自己身边:``休息一下吧,前边还不知道有多少路。''

出尘子没言语,盘腿坐稳了呼吸吐纳。无心知道他其实也是有点功夫的,所以并不打扰。

最后出尘子长长的出了一口气,感觉体力恢复了许多。从怀里摸出一盒火柴,他口中说道:``可惜,我的口袋落在石室里了。无心,你饿不饿?''

无心早作准备,出发前结结实实的填了一肚子干粮,所以此刻摇了摇头:``我还不饿,道长呢?''

出尘子叹了一声:``我有一点饿,可又没什么胃口,只想喝一杯冰镇酸梅汤。''

无心在暗中笑了一下,没接话茬,怕出尘子越说越渴,反倒受罪。

出尘子又摸着身下的石头说道:``石头下面,就是我们走过的千佛洞吧?''

无心思索着答道:``应该是。''

出尘子从盒里抽出一根火柴:``天下之大,无奇不有。今天我算是开了眼界。只是其中有许多不合情理之处,比如\ldots{}\ldots{}''

无心拍了他一巴掌:``比如你师父在洞外摆了一排马桶。难道洞外真是他的茅房?''

出尘子捏着火柴想了又想,末了得出结论:``也有可能,从洞口向外倾倒夜香,简直就像倒进万丈深渊里一样,绝不会有异味扰人。''

无心发现出尘子一派的思想都异于常人:``难道令先师也没有探明洞外情形?''

出尘子坦然的答道:``先师本来就不是个好管闲事的人,而且入洞来是要求精进的,未必会有闲心爬出去游山玩水吧!''

无心又问:``既然如此,为何最后又把洞口封起来了呢?''

出尘子经过一番慎重的思考,最后答道:``有两种可能。第一,先师修炼有成,已经无须饮食;第二——''

没等他说完,无心接了话:``要么是怕里面的东西出去,要么是怕外面的东西进来。''

出尘子摆弄着手里的火柴,虽然自己和无心想的一样,可是听无心说出来了,不由得心中一阵不安:``洞里面无非是鬼魂尸首,洞外又会有什么?''

无心沉默无语。出尘子在黑暗中坐久了,感觉有些憋闷窒息,则是忍不住划燃了一根火柴。火苗``嗤''的一声亮起来,无心忽然一哆嗦,猛然抬眼望向了出尘子。

出尘子对他早就看惯了,可此刻却是一惊,因为发现他的大黑眼睛忽明忽暗滴溜乱转,不是一个正常人应该有的样子。而无心直视着他,同时就感觉坡下起了窸窸窣窣的声音——或许有声音,或许没声音,但是总而言之,他感觉到了动静。

他知道自己没有活人气;有资格做猎物的,只有出尘子一人。要说石室内的行尸能够如此之快的突破阵法钻出小洞,似乎是不大可能;既然不是活死人们,来者又能是谁?又是为何而来?

忽然一口吹灭火苗,无心对出尘子急促的说了一声:``跑!''随即翻身趴上斜坡,快速向上爬去。出尘子莫名其妙的摸不清头脑,可是来不及多问,立刻随着他继续爬了起来。坡势越来越缓,两人没爬多远,便可以起身向上小跑。无心握住了出尘子的手,带着他头也不回的越跑越快。出尘子调整气息喘匀了,终于开口问道:``怎么了?''

无心头也不回的答道:``后面有东西!''

出尘子立刻把手伸进怀里要摸纸符,不料小腿忽然受了一夹。他只穿了一条薄薄的棉裤,无可抵御,正是惊惶无措之际,无心猛然俯身,也不知道是做了什么动作。出尘子就感觉小腿立时松快了,同时听见无心``呛啷''一声抽出背上短剑。石头地上随即起了火星子,他依稀看见无心正在弯腰狠砍着什么东西。

几剑过后,一只冰凉的手攥住了他:``道长把枪预备出来吧,后面追上来的是活物!''

出尘子一把就将腰间的盒子炮拔出来了:``什么活物?让它试试本道爷的枪法——''

话没说完,他被无心拽了个踉跄:``别吹了,快走!''

出尘子自从学会射击之后,购买了许多高级的外国枪支,隔三差五就在青云山里试枪打猎,所以听说后方追来的是活物,心中反倒有了底,认为活物总比鬼神好对付。大步流星的追上无心,他跑得十分安然,又道:``无心,小心脚下,用不用我再划根火柴给你照个亮?''

无心忙着跑,没有理他。而他见身后并无追兵,就把手枪插回腰间,直接摸出一根火柴在身上一划。火苗骤然窜起来了,他刚要继续说话,不料正见头顶闪过一条黑影。同时无心纵身一跃,竟是把那黑影扑在了地上。

火柴是用来点雪茄的长杆火柴,还算耐烧。出尘子笼着这一点小火苗赶上前去,就见无心一手摁在地上,一手挥起短剑。出尘子借着火光弯腰细看,随即打了个冷战——他终于看清了无心手下的活物!

活物不算大,两尺多长,形象十分类似于大壁虎,然而通体灰白,从首至尾光溜溜的,眼睛鼻孔一概没有;一张嘴大得惊人,口中布满了尖锐獠牙。怪物在无心的手下摇头摆尾,不住的张开大嘴向前空咬,火苗在熄灭之前暴涨了一瞬,把怪物口中的红色黏涎都照了个清清楚楚。

出尘子看了个心惊肉跳。地面又蹦起了火星子,是无心对着怪物猛砍了一气。怪物就厉害在了嘴上,身体没有鳞甲,皮肤又厚又滑又韧;徒手是杀不了它,利刃却能要它的命。无心估摸着自己已经剁烂了它的脑袋,就起身带了出尘子继续向前跑。

出尘子步大腿长,又受了怪物的惊吓,气运丹田跑得腾云驾雾,一边跑一边还有余力问话:``什么东西,为什么要追我们?''

无心刚想说怪物是觅着光来的,可是转念一想,又觉不对,因为自己和出尘子身边早就已经没了光。忽然想起了出尘子先前的所作所为,他恍然大悟的答道:``是血!道长,你身上有伤,怪物是追着血腥味来的!你师父之所以把他的大茅房封起来,想必也是因为洞里尸首有血腥气,招来了怪物!''

出尘子怪叫一声:``啊?我们该怎么办?''

无心加快脚步,要和出尘子齐头并进:``两个办法,一是让怪物吃掉你。''

出尘子立刻作答:``去你妈的,第二个办法呢?''

无心因为跑得太快,力量不敷分配,所以声音都变了腔调:``逃!''

此言一出,两人手拉着手,一溜烟的就冲向前方去了。

出尘子两眼一抹黑,完全不辨方向。无心倒是还能感知周遭环境,但路途崎岖,也只是跌跌撞撞的能跑直线而已。跑着跑着,出尘子的气息开始乱了;无心知道他最会使爆发力,搬大罗汉床时是把好手,长久奔波可就有点支持不住。忽然带着出尘子一拐弯,他按着出尘子的双肩说道:``蹲下往后退!''

出尘子一屁股就坐下来了,向后正好蹭进了一个石窝子里——一块巨石,一面不知为何陷了进去,成了个天然的掩体,容得下一个大胖子,或者两个小瘦子。出尘子是抱着膝盖坐进去了,低着头极力的调理气息。而无心四脚着地的在巨石旁趴伏下去,忽然抡剑一砍,他砍下了一只怪物的长吻。起身接连又是几剑,他不知道怪物是否同类相残,但是不管相不相残,他也无法收拾满地的血肉残躯了。

伸腿蹬了蹬石窝子里的出尘子,无心轻声问道:``道长,歇好了没有?''

出尘子抬起头,怏怏的发问:``无心,你说我们还得跑多远才能见天日?''

无心也是疲惫,所以懒得安慰他:``不知道。''

出尘子自己捶了捶小腿,语气幽怨的又问:``我觉得我们已经跑出很远了,怎么无边无际的总不到头呢?''

无心听他说话中气挺足,就起身拉扯了他:``道长,走吧,没有不到头的路,只是我们看不清而已。''

出尘子打起精神钻出石窝子,瞎子似的跟着无心继续往前走。走了没有几步,他就感觉脚下一空,大叫一声便坠了下去。而无心猝不及防,身子一歪也跟着摔了个倒栽葱。两人一前一后的着了陆,全跌在了冰凉坚硬的石板地上。

出尘子硌了尾巴骨,疼得当场落了一滴热泪。无心一挺身坐起来,伸手向前一摸,却是触到了另一只手。

另一只手,冷的硬的,石头雕的,手指纤细弯曲,栩栩如生。无心沿着石手向上摸去,最后从胳膊到肩膀,从肩膀到脑袋,他感觉自己是摸到了一尊塑像。

``道长。''他小声唤道:``我们好像掉回千佛洞里了。''

出尘子又喜又怕,连忙爬了起来,同时从口袋里取出一根长杆火柴。正是捏着火柴要划不划之时,他忽然感觉腰间一紧,仿佛是有什么东西搂住了他。

\chapter{金山}

出尘子冷不防的被人搂住了,不由得吓了一跳,手里的火柴顺势在无心后背上划燃了,他向前低头一瞧,登时吼了一声——一张血肉模糊的小脸正对他仰起来,看穿戴就是在洞内最先发现的小男孩!

小男孩显然也已经变成了行尸走肉,一张失了脸皮的面孔上可见层层鲜红筋肉。双眼的眼皮被撕掉了,两只眼珠突兀的鼓出多高。对着出尘子张开嘴,他一头扑上来就要咬。未等出尘子有所反应,眼前忽然寒光一闪,是无心一剑挥下来,削掉了小男孩的脑袋。小男孩没了头颅,可是双臂依然把出尘子箍了个死紧。出尘子生怕刀剑无眼伤了自己,连忙拍出一张纸符,正中小男孩的脖腔子。小男孩立时僵硬了动作,被出尘子一脚踹出老远。

随即趁着火苗未熄,两人看清了周遭环境。果然是回到千佛洞了,甬道两边的佛像正在夹道狞笑。远方隐隐响起了杂沓沉重的脚步声,仿佛正有大部队赶过来。无心和出尘子对视一眼,拔腿便想往出口跑,不料一步还未迈出去,头顶忽然噼里啪啦落下许多冰凉黏滑的东西,出尘子看得清楚,竟是上方暗处的怪物纷纷坠落下来。其中大的将近一米,小的也有一尺多长。出尘子只是受惊,并未真被怪物砸到;无心却是站在洞口正下方,怪物们全是先经了他的头顶,然后才落了地。

不等无心吩咐,出尘子拔出手枪,斜斜的扣动扳机射出子弹,又怕怪物不死,又怕跳弹伤人。而怪物受到攻击之后,发自本能的向洞内爬去。无心此时已经移了位置,一边抡剑去砍围攻出尘子的怪物,一边让出尘子边射击边撤退,万万不要被怪物咬到。出尘子见怪物口中尽是红色黏涎,一看就像富有剧毒,所以吓得双脚乱蹦,跳着后退。

上方怪物越落越多。出尘子退出老远之后又划了一根火柴,就见甬道地上一片此起彼伏的灰白后背,亮晶晶的蠕动不止。正是作呕之际,行尸走肉大部队可能是察觉到了活人味道,一路雄赳赳的开过来了。

行尸走肉虽然都已经死的有年头,可是因为洞内环境奇异,不甚腐烂,所以还有几分新鲜的血肉气味。嗜血的怪物们登时有了大方向,甬道地面起了灰白色的波浪,正是它们迎向了行尸走肉。出尘子见怪物与活尸狗咬狗打起来了,连忙蹲在地上摆出一溜纸符,口中念念有词的设起了阵:``众生多结冤,冤深难解结,一世结成冤,三世报不歇,吾今传妙法,解除诸冤业,闻诵志心听,冤家自散灭——哎哟!''

原来出尘子话音未落,忽然横空飞来一物,正掠过了他的头顶。东西``啪嗒''一声摔在地上了,他料得无心在后,应无大事,所以忍痛把阵设完。而后方的无心一把摁住飞来之物,却是一只小怪物叼着半条手臂,不知是被哪位活尸甩了过来。

无心一剑砍下了小怪物的脑袋,又将它四个爪子也剁掉了。怪物体内并无鲜血,创口倒是流出许多黏稠的清液。出尘子布阵完毕,起身做了个向后转:``我们快走!''

无心一手拎起小怪物的尾巴,一手攥住了出尘子的手腕,撒腿就跑。出尘子知道他是夜猫子的眼神,所以放心大胆的跟着他摸黑狂奔。拐了一个弯后,他忽然``咣''的一声撞上了一座石像,同时就听无心说道:``道长,我们出洞了!''

出尘子听闻此言,几乎快要喜极而泣:``继续跑,不要停!''

一旦出了洞,两个人心里有了盼头,累也不累了,一路只是向前疾驰。手足并用的穿过一片乱石地,两人一前一后的攀上前方石壁,最后爬上突在半空的一块大石。两人向下踏过几级石阶,进了来时所走的土洞。

因为没有马灯,所以出尘子全是摸索行事。无心的裤子很合身,索性就把裤腰带解下来,一端拴在了自己的脚踝上,另一端让出尘子攥着。洞子越走越窄小,他在前面爬,出尘子扯着腰带紧随其后,一颗心就提在喉咙口,无论如何不敢落后半步。

两人像穿山甲一样又在洞内爬行许久,最后感觉空气越来越凉,越来越干。无心率先爬上了地面,仰头只见夜空中悬着一轮冰盘似的大月亮,随即出尘子也把头伸出来了,呼哧呼哧先喘出了一片白雾。

无心把出尘子拽了上来,出尘子明明都要累瘫了,可还是强打精神用铁板和泥土重新掩埋了入口。待到他死去活来的忙碌完了,扭头向旁一看,却是发现无心正坐在荒草地上,低头摆弄着什么东西。

他站都站不起来了,四脚着地的爬过去看新鲜:``干什么呢?''

下一秒,他大惊失色的提高了调门:``你怎么把它带了出来?''

无心垂着脑袋轻声答道:``原来从没见过,在洞里又看不清楚,所以想要拿出来仔细瞧一瞧。不用怕,它的嘴巴爪子都被我剁掉了,现在就剩下了中间一段肉。''

他一边说,一边用手指拨弄面前地上的怪物身体。怪物成了一条灰白色的软肉,有筋无骨。双手托起怪物嗅了嗅,无心没闻出怪味来,只感觉略微有一点腥。见了月光冷风之后,怪物的身躯越来越软,无心把它放在地上,眼看着它软到不可收拾,最后化成了一摊半浊的浆子。

出尘子歇过了一口气,此时冷眼旁观,忍不住开口说道:``不要恶心人了。我们一天来也算是几次三番的死里逃生,趁着天还没有亮,赶快回观里休息吧!''

无心和出尘子趁着夜色,人不知鬼不觉的回了青云观。因为两人都是灰头土脸,所以出尘子不肯惊动旁人,只让值更的小道士去预备热水和夜宵。及至两只大浴桶摆在小小一间浴室里了,无心和出尘子像贼一样溜进来,无心倒也罢了,出尘子却是十分鬼祟,因为不愿意被徒子徒孙看到自己的土猴形象。

两人身上的气味都很复杂逼人,所以全很痛快的宽衣解带,想要沐浴涤荡一番。不料外衣一脱,地上却是叮叮当当响了一片。地面铺着雕花的石砖,能够摔出响的,必然也是坚硬东西。房内电灯明亮,所以无心蹲下去,立刻就捡起了五六块小石头。

两个人在石头洞石头山里摸爬滚打了将近一天一夜,衣服里面藏些碎石也是正常。出尘子懒得去瞧,弯腰一脱裤子,从裤腰里又滚出了几粒石子。看着无心蹲在地上专心致志的捡石头,出尘子把嘴一撇:``石头有什么玄妙吗?''

无心没有抬头,平淡的答道:``没有,我只是看一看。''

出尘子坐在热水里,伸手从附近的木架子上拿起一只小瓷瓶。将瓷瓶里装着的汁液浇在头上,他很惬意的抬起双手抓挠长发。无心也光屁股进了浴桶,手里依旧托着一把小石头。电灯光下,粗糙暗沉的小石头反射出了点点金光。不动声色的向水中一沉,他枕着桶沿闭了眼睛,心中暗道:``金矿石。''

忽然抽了抽鼻子,他睁开眼睛望向了出尘子:``你用了什么?''

出尘子还在洗他的头发:``是何首乌和皂角。''

无心``哗啦''一声从水里挺起了腰,把脑袋一直伸到了出尘子面前:``给我也来点。''

出尘子虽然感觉他的要求十分无稽,不过还是拿过小瓷瓶,往他头上倒了一点汁液。无心一直希望自己的头发可以再长一点,所以抬起一只手满头揉搓。出尘子很不屑的扫了他一眼,看他头发还没有狗毛长。

两人洗漱过后,换了一身洁净衣裳。无心暗暗揣好了金矿石,想要带回天津给顾大人。一团和气的吃过一顿丰盛夜宵,两人都不困,于是关了电灯,躺在大罗汉床上谈论今日的所见所闻。洞内的疑点谜团太多了,即便是牵强附会,也难以全部解释。眼看着窗外亮了天,出尘子还是满心疑惑。无心倒是安然,因为世上的不可思议之事太多了,想要凭着人力一一揭秘,根本不可能。

天大亮时,无心和出尘子一起入睡了。而在百里之外的文县,岳绮罗则是刚刚起床不久。

她穿着一身红衣红裤,领口袖口滚了白色的风毛,脚下趿拉着一双兔毛拖鞋。歪着脑袋站在窗边,她一手托着一只青花瓷的小碗,另一只手捏着小银勺子,从碗里舀出一勺白白嫩嫩软颤颤的物事。滚热的蒸汽熏红了她的嘴唇和鼻尖,她把嘴撅成了小花骨朵,凑到银勺边沿吸吸溜溜的去喝。

房门忽然一开,张显宗带着一身寒气走了进来。进门之后他望向了岳绮罗手中的小碗,直勾勾的一言不发。片刻过后,他终于开了口:``你怎么吃这个?''

岳绮罗抬眼皮撩了他一眼,用微哑的童声答道:``放心,是豆花。''

张显宗脱下了皮手套:``我知道是豆花。你怎么吃豆花?豆花能够补养身体吗?''

岳绮罗舀起一勺烫豆花,试试探探的又喝了下去:``没胃口,吃点清淡的更好。''

张显宗无可奈何的笑了一下,在她面前微微俯下身问道:``伤风好些了吗?''

岳绮罗答道:``伤风早好了,可是昨夜睡得不对,早上起来脖子疼。''

张显宗垂下眼帘,看她捏着勺子的小手。手掌是单薄白皙的,然而手指头带着稚气的肉感,笨笨的翘成了个小兰花,指甲粉红透明,短得让他心疼。他问不出她的来历,于是很笃定的当她是个小妖女。小,妖,女,三个字单拿出哪一个,都够让他心跳一阵的;三个字合起来凑成一个岳绮罗,让他心甘情愿的把她供到头顶上。

岳绮罗趴在床上,因为张显宗自告奋勇的要为她按摩脖子。床很平,她也很平,两平相遇,她在床上趴了个踏踏实实。一张脸侧过来,乌黑乱发中露出了一点小小的耳垂,白里透红,是初绽的花瓣。

张显宗坐在床边,用两只大手去捏她薄薄的肩膀和细细的脖子,同时口中说道:``有光兄弟昨天催促了我,我想事情拖了一个多礼拜,也该给他们一个答复了。''

岳绮罗从鼻子里往外哼出声音:``不就是他们在青云山发现了金矿吗?其实也无须多想,无论金矿由谁开采,都免不了要有一场战争。有光兄弟是日本人,当然可以隔岸观火,真要动刀动枪,还不是你们自相残杀?''

张显宗也知道其中的道理,本是不想去趟浑水的,可又舍不得金矿。思索之中走了神,他手上一时失控,捏得岳绮罗尖叫一声;两条腿翘起来,脚跟在张显宗的后背上连敲了一顿鼓。张显宗一回头,看到两只穿着洋纱花袜子的小脚乱摇乱晃,就忍不住笑着道了歉。又问:``我下不了决心,你替我做主吧!和日本人到底是合作,还是不合作?''

岳绮罗其实对于``人事''不是很感兴趣,并且感觉自己和人没什么可说的。不过如果手下没有了人,她就无法维持当下的好生活。所以居高临下忙里偷闲的思索了一瞬,她想钱总是越多越好,于是有口无心的答道:``随便你,想合作就合作吧。''

\chapter{家园}

有光兄弟是两个人,哥哥叫有光勉,弟弟叫有光淳。兄弟两个来到中国也有好些年了,哥哥的身份是大商人,弟弟的身份是旅行家。两人满中国的来回走,一边走一边交中国朋友,勘中国矿藏。有许多人都说他们是间谍,不过并没有十分确凿的例子;有光兄弟自己也满不在乎,反正无论中国人说什么,他们都一概不承认。

青云山的名气很大,但是从地理位置的角度来看,的确还是偏僻,距离长安县和文县都有一段距离。自从得知了青云山中兴许藏着一座金矿,他们立刻来了精神。因为长安县内的大军头对日本人素来不大友善,所以他们立刻登了文县新贵张显宗的门,以着一家大商社的名义,要和张显宗联合开矿。如果张显宗无意合作,他们会马上跑去长安县另寻伙伴;如果张显宗有意合作,金矿一旦真实存在,长安县内的人物少不得也要出场,从他们的手中抢一杯羹。总而言之,舍不得孩子套不来狼,想要分金子,就得卖命。好在据有光兄弟说,日本的技术人员在秘密勘探之后,认为青云山金矿的含金量也许会是相当之高。

张显宗在定了主意之后,虽然前途未卜,但好像放下了一桩心事似的,没来由的感到一阵轻松。带了几色鲜艳绸缎去了丁宅,他没别的事,就想见岳绮罗一面。岳绮罗的身体不是很好,让他一直有点悬心。她要吃人,他就供着她,反正她小小一点肠胃,吃也吃不了许多。供养着岳绮罗,像供养着一个秘密的小神仙。他很愿意去做她的信徒,不为别的,就为她是个阴森森的美丽小姑娘。阴森森的豆蔻花开,阴森森的二月年华,矛盾而又调和,让他失了神入了迷。

进入丁宅之后,他轻车熟路的直接进了后方的小院。丁宅的人都快死绝了,也只有岳绮罗敢在凶宅继续住下去。小院内外都很安静,仿佛快要落春雪了,天空阴的厉害。他推开房门走进去,房内一片冷清,黯沉如水。天光从玻璃窗中射进来,深深浅浅的投了满室阴影。

岳绮罗摆了个弥勒佛的姿势,歪坐在一张靠墙的长沙发上;似乎是刚刚午睡醒来,一头齐耳短发乱成无法无天。一手撑在沙发上,一手搭在膝盖上,她抬眼望向张显宗,脸很白,眼睛很黑,薄薄的嘴唇透出淡淡的水粉颜色。

张显宗笑了一下,向她一托手上的玻璃匣子。匣子里面一层层的叠了绸缎,有桃红有柳绿,有鹅黄有天蓝,每一样的尺寸都不大,因为岳绮罗是个小人儿,从头到脚的做上一身,也用不了许多料子。

``好不好看?''张显宗问道:``春天到了,该添新衣裳了。''

岳绮罗本来正在发呆,此刻怔怔的盯住了玻璃匣子,直过好半天才有了回应:``好看。''

然后她伸手向前一指:``绿的我不要,你给我换一件雨过天青的。''

张显宗很有耐心的点头:``好,我记住了,换一件雨过天青的。''

他把玻璃匣子放到一旁的桌上,走上前去蹲在了岳绮罗面前,仰起脸笑问:``怎么一个人坐在屋子里?闷不闷?''

岳绮罗倒是不闷,因为方才一直在发呆,不知不觉就消磨了时间。微微低头正视了下方的张显宗,她想他是凡夫俗子,死了,就没了。她不爱他,可是他爱她。

忽然对着张显宗微微一笑,她伸手从沙发缝隙里摸出一盒火柴:``多谢你来瞧我,我变个戏法给你看吧!''

说着她抬手在虚空中画了一道符,随即划燃一根火柴向上一扔。火苗幽幽的燃烧在了半空中,随着她的指尖起伏旋转,是一颗灵活的小流星。短暂的光明过后,她利落的打了个响指,附在火柴上的魂魄立时消散,只余一缕灰烬无声落下。

``好不好玩?''她兴高采烈的问张显宗。

张显宗认真的点头:``好玩。''

岳绮罗慢慢收敛了笑容,感觉自己的幸福和本领不甚匹配。百无聊赖的咂了咂嘴,她伸手一拍张显宗的肩膀:``我牙齿有些疼。''

张显宗立刻提起了心:``哪颗?''

岳绮罗张大了嘴巴,用手指向里面一指:``啊!''

张显宗探头望去,就见她生着两排整整齐齐的小白牙,里面有一颗白中透出隐隐的一点黑,似乎是蛀了,不过他不是医生,也不能确定。

文县城内有座小教堂,教堂里驻扎着一名老掉牙的西洋神父,神父除了传教之外,同时也担任西医一职,而且医术还颇高明。张显宗领着岳绮罗去了教堂,要请神父为她看一看牙齿。经过神父的诊视,他得知岳绮罗的牙齿的确是处在了危险之中,大概是冬天糖豆吃太多了的缘故。

牙齿虽然要坏,但还没坏到值得修补的程度,所以张显宗和岳绮罗在心中有数之后,就坐上汽车回了家。一路上岳绮罗暗暗用舌尖舔着她的坏牙齿,心想一旦它坏到不可救药了,自己就拔掉它,换颗金牙。而张显宗坐在一旁,先是不动声色的抱着胳膊看风景,看着看着伸出一只手,试试探探的握住了岳绮罗的手。

岳绮罗全神贯注的舔牙,随他去握。对于张显宗,她并不讨厌,她只是不喜欢。

开矿是件大事情,动工之前要做无数的准备,打通无数的关节。所以日子风平浪静的过下去,外人并不知晓内情。

文县太平,长安县也太平。只要不打仗,两处就都是繁华的好地方。无心在青云观内住了三天,其间不见天日,从早到晚的只和出尘子谈论山中怪洞。洞中的怪物姑且不提,行尸走肉都有来历,也不奇怪;怪的是洞子本身。出尘子认为凭着先师的力量,绝不能够不声不响的挖出大山洞。师父或许是偶然间进了山洞,发现洞中的种种古怪;至于山洞的由来,恐怕他老人家也是不知道。

``千佛洞''三个字的称呼,显然也不适于山洞了,因为洞中并没有真正的佛,只有一些类佛的诡异塑像。塑像是怎么来的,两人想破了头,也还是想不出个眉目。

出尘子的思想向来是条理分明的,如今方寸大乱,就不让无心离开,要他陪着自己一起苦思冥想。无心倒是不在乎苦思冥想,问题是他很想家。连着四天没回去了,他想家想得要命。

于是他不顾出尘子的挽留,在第五天清晨起了个绝早,乘坐青云观的汽车上了路,下午就进天津卫了。

兴高采烈的下汽车进胡同,他停在自家院门前,先把双手插进口袋里,上下将院门打量了一通。院门后面就住着月牙和顾大人了,他忽然有点激动。

伸手轻轻一推院门,院门顺势而开。隔着玻璃窗子,他看见月牙拿着鸡毛掸子,正在房里忙碌。忍无可忍的快步走去推开房门,他大喊一声:``月牙!''

月牙系着围裙,一条腿跪在椅子上,正在掸柜子上的灰尘。冷不防听到了他的声音,她立刻抬头望向门口,随即惊喜的叫道:``呀!''

无心不等月牙多说,张开手臂就迎过去了。月牙攥着鸡毛掸子下了椅子,不假思索的扑上来和他抱了个满怀。两人的手臂全勒紧了,无心低下头,鼻端都是月牙的气味,让他想起了好饭好菜热被窝,想起了一切温馨温暖甚至热烈的好生活。猛的抱起月牙转了一圈,他忽然很想和月牙搂着睡一觉。

两人抱够了,月牙推开无心,用鸡毛掸子在他身上抽了一下:``你不是说过一两天就回来吗?这都过了几个一两天了?不回来也不给个信,让我在家瞎惦记,你个不长心的!''

无心笑嘻嘻的从衣兜里摸出一只洋酒瓶子。酒瓶子不大,比他的巴掌略长,方方正正的挺好看,里面盛着大半瓶颜色深浓的汁水。把洋酒瓶子递给月牙,他开口说道:``给你的。''

月牙接了瓶子:``啥玩意儿?是酒?我也不喝酒啊,你留着给顾大人吧!''

无心答道:``不是酒,是用来洗头发的。青云观那老道你也瞧见了吧?他就用这东西洗,我看着不错,昨天向他要了一点。东西是他按照秘方熬出来的,不好盛放,他给我找了个空酒瓶子,结果大小还真合适。''

月牙拧开酒瓶盖子,低头凑到瓶口一嗅,然后抬头对着无心笑道:``有点苦气,也有点香。我这就烧水洗一次,看看咋样。''

然后她把盖子拧好了,将酒瓶珍而重之的放在橱柜上面,然后一路欢天喜地的扭出去烧水。洗过之后晾干头发,她坐在床上梳头,无心抱着膝盖蹲在一旁。天空晴朗,两人全都披了一身的阳光。

月牙让他摸自己的头发:``滑不滑?''

无心摸了:``滑。''

月牙是很容易快乐的,头发洗得又顺又滑,就足以让她心满意足的高兴一阵子。将长发在脑后盘成一个圆髻了,她往床下伸了腿,要去买肉买菜。她闲不住,无心也跟着跑前跑后。一把大锁挂在院门上,无心拎着菜篮子,跟她一起往胡同口去了。

顾大人在天擦黑时回了家,一进院子就是一愣,因为发现厨房灯火通明,月牙摆着大场面煎炒烹炸,旁边站着个游手好闲的无心。院子里弥漫了带着葱花味的油烟,让顾大人立刻就饿成了心急火燎。

``哟,回来了?''他没进房,直接就奔了厨房:``你怎么才回来啊?不是说就走一两天吗?这他妈是几个一两天了?我告诉你啊,你没事可别出去野跑了,你不在家你媳妇就不正经做饭,天天给我熬萝卜切咸菜,吃得老子嘴里淡出鸟。''

顾大人话音落下,又伸手一指月牙:``说你呢,你还偷着笑。妈的不是亲媳妇就是不行,就知道哄你男人,一点都不孝敬我。''

月牙忙着切菜,不肯回击。而无心则是把顾大人拽去了东厢房:``我给你带了几样东西,你看看有没有用。''

顾大人进了房,摘帽子脱衣裳:``青云山能有什么好东西?''

无心向顾大人伸出一只手,掌心托着几枚灰扑扑的小石头。

顾大人看见之后,登时哭笑不得:``什么破玩意儿,你给我带了一把石头回来?''

无心一扬下巴:``你仔细瞧。''

顾大人莫名其妙的拿起一颗小石头,当真是放到灯光下缓缓转动着细看。看到最后他抬头问无心:``石头上撒金粉了?''

无心答道:``是青云山里的金矿石。''

顾大人登时严肃了表情:``青云山里有金矿?''

无心摇了摇头:``我也不能肯定,我只有这些金矿石,而且是从地下带出来的。''

顾大人掏了掏耳朵:``我没听明白,你再说一遍。你钻地下去了?''

无心把金矿石的来龙去脉简单讲了一遍。顾大人听得目瞪口呆,最后他低头看向手中的金矿石,一双眼睛射出了喜悦的光。

\chapter{大好前程}

钱权二字乃是顾大人人生道路上的明灯,骤然得知了青云山里可能藏着金矿,他登时心乱如麻的亢奋起来。恨不能立时插翅飞去青云山,把整座山全都搬到自家院子里来。

然而辗转反侧的度过一夜之后,他的头脑渐渐降温,理智也重新占据了上风。凭着他如今的势力,莫说是发现了一个也许有也许无的金矿,就算眼前真摆上一座大金山了,他单枪匹马,也是守不住。既然独占不成,那跟着分几分红利也是妙的,于是他把所有的金矿石都装进一只布口袋里,攥着口袋就奔帅府去了。

他抢不到的好处,也不会白白让给别人。他要先把这份没主的大礼送给老帅,一旦将它搞成了国家大事,蠢蠢欲动的小军头们就没机会暗里私吞了。而自己随在老帅的屁股后面,怎么着还不能得点金末子金粒子?

顾大人日夜奔波,并且还带上了他的胖朋友苏先生。苏先生是个有知识的人,在老帅面前也是很有分量的幕僚。而老帅本来就预备着要和小军头们打一仗,如今一听青云山有金矿,更是中了下怀——他若是强占了金矿,免不得要起事端,一旦起了事端,老帅就师出有名了。

于是不过三天的工夫,一支勘探队伍便启程去了青云山。队伍成员都是在国外专修过矿业的留学生,据说水平是相当之高,只要是去了实地,就必定能带个结果回来。

顾大人为了事业不眠不休,这天好容易得了闲,大下午的想要回家睡觉,不料刚一进院,就听见月牙在西厢房呜呜的哭。他以为是小两口打起来了,连忙走到玻璃窗前向内望,结果只见月牙蓬着一头乱发坐在床上,而无心俯身托着一条毛巾,正在为她撩起头发擦脸。

伸出手指一弹玻璃,顾大人随即推门进了房:``你俩怎么了?月牙,他揍你了?''

月牙接过毛巾捂在脸上,抽抽搭搭的说不出话;无心苦笑着直起腰,轻声答道:``上午带她出去烫头发,烫完回来一照镜子,就哭了。''

顾大人登时笑出了声,一边笑一边后退一步,仔细端详月牙的新发型:``狮子狗似的,不过也不值得哭啊,现在街上的娘们儿不都这个德行?看习惯就好了。''

月牙在毛巾后面哽咽出了声:``你懂啥啊?''

到了傍晚,月牙照例出来做饭,顾大人才发现月牙的确哭得有理。她原来的长头发,又黑又密的一大把,现在被剪得只剩一尺多长不到两尺,松松散散的披在肩头,发梢全被烫成焦黄。发髻是挽不成了,小辫也编不得,并且大概是头发太厚的缘故,满脑袋都是卷子,蓬得一个脑袋有两个大。

月牙感觉自己现在这幅模样,和妖怪也差不多了,又恨自己当时烫完便走,也没细看;结果不但毁了头发,还饶上不少的钱。哭丧着脸熬了一锅老萝卜,她喂猪似的打发了无心和顾大人的晚饭。

入夜之后,她唉声叹气的上了床。无心把安慰的话也说尽了,这时无话可说,就躺在被窝里伸手抱她,又探头凑上去想要亲她。月牙没心思,把头一扭,于是无心的脸就陷在了她的蓬头中。无心在她的头发里蹭了蹭,忽然感觉面孔很温暖,并且全是月牙的气味。踏踏实实的躺稳当了,他一头扎在月牙的头发里睡着了。

月牙起初没当回事,又过了几夜之后,才发现无心养成了新癖好,专把脸往自己的头发里拱。她没想到自己的新发型还把无心哄舒服了,不禁哭笑不得。夜里两人钻了被窝,她小声笑问无心:``你不嫌我丑啊?''

无心伸出一条手臂让她枕着,听了问话,他沉默了片刻,末了答道:``月牙,你知道,我只怕你会不要我。''

然后他低头把脸埋到了月牙的胸脯间。而月牙细想了他的话,忽然眼眶一热,无心既是她的丈夫,也是她的儿女了。只要她活着,她就得陪伴着他,拉扯着他。

赶在自己落泪之前,她在他后背上用力拍了一巴掌:``没个爷们儿样!你看谁家男人天天害怕被媳妇踹了?''

无心没回答,把脸深深的往月牙胸口埋。月牙搂着他抱着他,忽然又恨了他,恨他不老不死,恨自己没了,他将来又会再找别人——贱兮兮的,可怜巴巴的,讨好卖乖的,像怕自己一样,怕那个新娘们儿不要他。

月牙越想越是不忿,最后暗暗伸手在他手臂上狠拧了一把,拧过之后,他却是一动不动,无声无息。

月牙等了半天,忍不住问道:``疼不疼?''

无心声音很闷的答道:``疼。''

``疼咋不叫?''

无心抬起了头,在窗外透进的浅淡月光中去看月牙,两只眼睛一眨不眨:``我怕你生气。''

月牙像个老姐姐似的摸了摸他的短头发,心里很后悔方才的一掐,同时决定以后再也不欺负他了。

月牙多愁善感的浮想了一宿,翌日早晨起了床,总像心里有愧似的,不但把洗脸水一直端到了无心面前,甚至对顾大人都温柔了许多。家里的女人一露了好脸色,无心和顾大人立刻松了一口气,都有了雨过天晴之感。顾大人端着一海碗打卤面,开始挑三拣四:``月牙,卤子淡了啊!''

月牙用小勺子舀了一勺盐,从厨房一路小跑着进了上房,把盐撒进盛卤子的大碗里,又说:``拌一拌。''

顾大人伸舌头一舔自己筷子上的酱汁,然后理直气壮的伸了筷子去搅卤子。月牙一时没拦住,一边转身往厨房走一边嘀咕:``你倒是换双新筷子啊!''

顾大人不以为然,当即反驳:``一家的人,穷讲究什么?''然后扭头去问无心:``你嫌我吗?''

无心饿了,正在狼吞虎咽的往嘴里捞面条。鼓着腮帮子看了顾大人一眼,他满嘴流油的无暇回答,只摇了摇头。

顾大人洋洋得意,又对无心说道:``师父,告诉你啊,老帅这回兴许能给我放个旅长。''

无心把空碗放在桌上,因为实在是匀不出舌头来说话,所以只对着顾大人一拱手,表示恭喜。不等咽下口中的面条,他起身又给自己盛了一大碗。月牙回了来,正赶上了个话尾巴,倒是诚心实意的挺高兴:``顾大人,咋的,你升官了?''

顾大人沾沾自喜的一笑:``那是当然。等到委任状一下来,我就是先头部队!''然后他对无心说道:``老帅已经派人去看明白了,说是真有金矿,但是不大。如果要开矿的话,影响不到青云观,不怕观里的老道干涉。摆在眼前的金子,傻子才不要。所以老帅要派我先去青云山,你跟我一起走吧,再把月牙也带上。放心,我是领着大部队走,你俩都吃不了苦!真要是交了火,也有地方安置你们。''

有些内幕,顾大人和无心知道,但是月牙不知道。无心迟疑了一下,随即说道:``你找出尘子也是一样的。他上次是措手不及,如果提前做足了准备,再加上你们的协助,应该不会有问题。况且光天化日下开挖,就算真有什么,也闹不出大祸来。''

顾大人摸着下巴,有些为难。近一年的风浪都是和无心一起闯过来的,忽然让他单独一人去做大事,他心里还空落落的不踏实了。

无心看出他的心事,便又补了一句:``反正青云山也很近,你先带兵过去,我和月牙留下来再等一等。如果真用得上我了,随时给我送个信就行。''

月牙没有多问,猜出顾大人所顾忌的肯定是些鬼神之事。平白无故的挖大山,能不考虑考虑山神老爷的意思吗?

七天之后,顾大人接了老帅发下来的委任状,走马上任成了顾旅长,彻底恢复了往昔的大人身份。他乐坏了,在外面一路绷着面孔,回到家后关了院门,才爆发似的哈哈大笑起来。

然后他把无心和月牙全部叫进了上房。无心和月牙都向他热烈祝贺之后,他还意犹未尽。抬腿一马靴踩到椅子上,他拍着大腿开始向面前的两口子展望未来,顺便许了许多大愿。月牙的钻石坠子也有着落了,说是等到他从青云山回来了,就一定给她买。

无心坐在一旁,胳膊肘拄在桌面上,托着下巴笑而不语。月牙站在一旁,一边嗑瓜子一边做听众。如此闹到晚饭时分,顾大人真是饿了,才宣布散会。

三个人肥吃海喝的快活了一晚上,翌日上午,顾大人率领队伍,当真是出发了。

\chapter{误入山林}

顾大人又有兵了。

因为他先前就有些大名声,资历很可以服众,如今又是老帅眼前的红人,所以队伍上下没有敢向他挑战的刺头。他耀武扬威的把军队开到青云山,先把富有金矿的半面山围住了,然后自己提了几样华而不实的礼物以及老帅的亲笔信,前往青云观拜访了出尘子。

出尘子听闻自家后山居然有金矿,不禁大吃一惊。不过他的思路很类似顾大人,一想到有金矿也轮不到自己独占,他索性做了个顺水人情,表示青云观对于开矿之事是不闻不问不干涉。至于山中地下的玄妙,出尘子想了又想,却是不知当讲不当讲——毕竟是没影的事情,一旦说了,没有证据,倒像是他有意作梗;可若不说,万一真挖出了灾祸,不知道军中失火,会不会殃及青云观里的池鱼。

出尘子是精于人事的,在达官贵人面前,一张嘴素来极有分寸。顾大人虽然算不得多么达贵,但是前途未可限量,而且身后还有一位老帅做靠山,所以出尘子沉吟良久,最后却是问道:``无心来了吗?''

顾大人对于出尘子的印象很好,笑呵呵的答道:``他没来。来了也没事做,我就让他留在天津了。''

出尘子垂下眼帘,决定还是静观形势,不要妄言。

因为开矿的机械器具都没有运到,有技术的工人也未招募齐全,所以青云山上除了士兵之外,依旧就只有勘探小队在活动。顾大人对于矿务完全不通,唯一的任务就是坐等对头打上门来,所以并不亲自进山,只在山脚下借用了青云观的一片房屋,又派副官去长安县的大窑子里接回了几名花枝招展的妓女,终日饮酒作乐,十分快活。

他一快活,文县的张显宗就不快活了,有心带兵杀过去,又没有十分的胜算。心事重重的站在一棵老树下,他仰起头对着岳绮罗勉强微笑。

老树发了新芽,枯枝上生出点点鹅黄,近看没什么好的,远看倒是春意盎然。岳绮罗穿着一身桃红衣裳,大喇喇的分开双腿骑在一股子粗枝上。季节一变,她的心境也随之有了变化,像一般十几岁的少女一样,生出了一点伤春悲秋的情绪。人一伤悲,脾气自然也就好不到哪里去;她本来不打算理睬张显宗,可是张显宗静静的站在树下,不说话也不离开,她默然良久,最后忍不住斜了他一眼:``有事?''

张显宗把她当成了个带着神性的小偶像,有了心事而又茫然无措之时,就很愿意向她倾诉一番。移下目光盯住了她的一只脚,他低声说道:``出了一点麻烦,青云山被人占住了。''

岳绮罗随着他的视线,也低头望向了自己脚上的绣花鞋:``谁?''

张显宗答道:``顾玄武,现在改名叫顾国强了。''

岳绮罗一听到顾玄武三个字,就想起了无心。无心是她心中的谜,世间的一切都令她感觉索然无味,除了道术,以及无心。对着张显宗张开双臂,她俯身向下一扑,直接落进了对方的怀里。而未等张显宗将她抱稳,她已经像条小鱼似的,从他的臂弯中下滑落地。

很久没有出门见天日了,岳绮罗忽然起了兴致。脚趾头在绣花鞋里动了动,她决定亲自出门去会一会顾大人。因为顾大人是无心的老朋友,也是张显宗的新敌人。万一能够通过顾大人打听到了无心的生死,万一无心当真活着,万一自己找到了无心,万一无心心回意转爱上了自己,自己岂不就是可以活得更快乐了?

如果以上的``万一''全不成立,她就宰了顾大人,为张显宗除去眼中钉。反正跟着张显宗也不坏,张显宗在她面前,时常温柔的让她坐立不安。

岳绮罗定下主意之后,也没有和张显宗商量。入夜之后她径自出了丁宅。宅子门口站着卫兵,对待她素来是毕恭毕敬;听说她要出门,连忙张罗着要去呼唤卫士随行。岳绮罗说道:``不必惊动他们了,我自己走。''

卫兵知道她是带着一点神秘性的,不敢阻拦,立刻又问:``您是坐马车,还是坐汽车?''

岳绮罗略微思忖了一下,随即答道:``全不用,你给我牵一匹小马过来。''

卫兵领命去牵马,可是挑来选去,军马全都高大威武,不合岳绮罗的意。后来卫兵福至心灵,弄来了一头小毛驴。毛驴背上鞍辔齐全,正是一头时常出城、见过世面走过长路的好驴。

把一根小鞭子双手送到岳绮罗面前,卫兵还问:``用不用再去通知参谋长一声?''

岳绮罗摇了摇头,然后轻轻巧巧的飞身上驴。伸手摘下驴脖子上挂着的小铜铃铛,她一甩皮鞭,毛驴登时就善解人意的跑上路了。

岳绮罗走的是小路,毛驴耐力好,在崎岖路上又是特别的灵活,反倒走得比马更快。天色将明未明之时进了长安县,她随便找了一家小旅店住下,足足的睡了一天。到了傍晚时分,她和毛驴全歇足了吃饱了,便又一起上了路,直奔青云山而去。

出发之前,她研究过地图。如今估摸着距离青云观还有五六里地远了,她把毛驴拴在了路边的野林子里,开始徒步前行。忽然念念有词的一甩袖子,前方多了两个探路的纸人,飘飘摇摇的给她打前锋。夜色越来越浓重,天空疏疏朗朗的点缀着几个银星星,一弯白月亮勾着几缕云。岳绮罗的体力一直是马马虎虎,初春的夜又是寒冷如冬。她把两只手揣进袖子里,吸着鼻子顶着寒风往前走。走着走着,纸人不动了,似乎是前方有无形的屏障阻挡了它们。岳绮罗心中一动,知道自己已然进入青云观的地界了。

据她所知,顾大人的军队全驻扎在了青云观后方的山麓一带,并没有进山,也没有骚扰道观。一挥手指挥纸人转了方向,她开始往后山走,结果刚刚走了不远,她便看到了成片的帐篷。夜深了,士兵也都睡了,帐篷之间偶然有火光闪动,是小队举着火把在巡逻。

军队大营的阳气杀气都很重,纸人一旦离她远了,就像失去力量一般,摇摇欲坠的要倒。岳绮罗索性收了它们,想要亲自设法潜入军营。只要让她见了顾大人的面,只要顾大人是个人,她就有办法了。

岳绮罗攥着手帕,一边擤鼻涕一边在黑暗中来回的走,同时忍着一个大喷嚏。军营周围三步一岗五步一哨,没有明显的破绽,于是她决定换个方向进攻,先从青云观与军营之间的一条小山路上进山,然后从山上往山下走。军营总不会四周全是固若金汤,都知道山里没有人,想必朝着大山的方向,便是军营外围的最薄弱处。

她打好了算盘,开始摸着黑踏上了坎坷山路。她记得在许多许多年前,自己仿佛是登过一次青云山,那时候青云山还不叫青云山,青云山上自然也没有青云观。自己进山是干什么来着?不记得了。山里是什么样子?也不记得了。

她深一脚浅一脚的走,走着走着停了脚步,发现前方路上现出了一个大坑。月光之下,坑周的泥土还很蓬松,显然是个新坑。岳绮罗怀疑新坑和开矿有关,刚想小心翼翼的绕过去,不料一脚踩在地上,却是泥泞的一滑,让她险些跌了一跤。

踉跄着站住了,她低头一看地面,就见地上亮晶晶的漫开一摊白浊液体,方才被自己踩了一脚,液体和泥土混成了泥。莫名其妙的蹲下来,岳绮罗没看出液体的成分。眼皮向上一撩,她忽然又发现了新玩意!

就在液体之中,还浸了几块尖锐的骨头,以及一副奇大的利齿。

岳绮罗从衣兜里摸出一张小小的人形纸片,随手向外一挥。一个白脸笑眼的纸人立刻站到了她的身边。岳绮罗站起身,后退一步说道:``把它给我捡起来!''

纸人能够领会她的命令,果然弯腰伸手,将一副大牙捧起来托到了她的面前。月光之下,岳绮罗看得清楚,就见牙床将近有人头大小,利齿尖端闪着寒光,齿缝之中竟然还有鲜红黏涎反射月光。可见黏涎是新鲜的——身体都没了,只剩了一副牙齿,齿间的黏涎怎么可能还会新鲜?

岳绮罗莫名其妙,抽动着鼻尖凑过去一嗅,感觉微微有点腥,倒是没有十分恶臭的异味。走回地上一滩液体跟前,她低头又想细看。不料就在此时,坑中忽然窜出一条白亮亮的怪物,张开大嘴直奔了她的脑袋。岳绮罗心中一惊,瞬间仰头向后一躲,同时就听``啪嗒''一声,正是怪物结结实实的拍在了地上。一边后退一边望去,她就见怪物足有一米来长,通体灰白,头部光秃秃的扁扁长长,一张大嘴十分醒目。眼看怪物对着自己又龇出了大牙,她情知不妙,转身就想逃,不料怪物纵身一扑,一嘴叼上了她的小腿。岳绮罗惊叫一声扑倒在地,回身一看,却见自己的小腿虽然陷在了怪物口中,可怪物瘫在地上,并未发力,长大的身体眼看着失了形状,软软的竟然化成了水。

岳绮罗一下子全明白了。小腿隐隐的有些刺痛,不知道怪物的尖牙有没有刺破裤子伤到皮肉。千辛万苦的撬开牙关收回小腿,她也来不及细看,爬起来就要往山下跑。可是一步迈出去,她``咕咚''一声跪下来,受了伤的腿竟是不能使力。

她慌了神,心想万一坑里再爬上来一只怪物,无论它死得有多么快,恐怕自己都难逃一劫。扶着身边的小树站起来,她对着纸人后背一扑。只听``咔嚓''一声,纸人真成了纸人,被她压了个四分五裂。

岳绮罗摔了个大马趴,真是急了。右手指尖在地上快速划出一道符,她用力一拍地面,同时轻声叫道:``生!''

附近地面立刻缓缓隆起一个土包。土皮四分五裂,一具很有年头的野狗尸骸破土而出,腐烂得只剩了一身骨架。岳绮罗见状,气得一挥手。附在野狗身上的魂魄立时消散,骨头在地上散成了一堆。

岳绮罗换了右手,继续在地上画符,想要召唤出得力的阴兵来救自己下山。手掌狠狠一拍地面,树下土中却是拱出了一名士兵。

士兵穿着一身血衣,胸前弹孔清晰可见,不知是死于战争,还是死于军法;不过身躯还算完好,两只眼珠一起向上翻着,一张嘴张得很大,仿佛是临死之前还在呐喊。岳绮罗没心思再挑拣,爬起来蹦上士兵的后背。而士兵在她的操纵下,就拖着两条腿一步一顿的往山下走去了。

岳绮罗搂着僵冷尸首的脖子,一颗心狂跳不止。小腿越来越疼,让她心慌意乱的忍不住想:``我这么漂亮,不会被毒死吧?''

想着想着,她落了一滴泪,不是怕死,是舍不得自己的好皮囊。

\chapter{蚀骨之毒}

岳绮罗骑着一具行尸跑了五六里地,然后换乘毛驴往文县赶。路上她的腿越来越疼,疼到毛驴一颠,她的心也随之一颠。

天亮天又黑,她终于进了文县,见到了坐卧不宁的张显宗——张显宗一直在等她回来。

她本来是不把张显宗放在眼里的,任凭张显宗把自己从驴背上抱下来,她依旧只当对方是个不值一提的凡夫俗子。可是等到张显宗把她送到房内、心急火燎的蹲下来去掀她的裤管时,她心中一动,忽然想道:``除了他,还有谁能这样待我?''

张显宗没有留意到她的若有所思,接着方才的话急问道:``到底是被什么东西咬了?这么大的牙印,怎么可能是壁虎?''

岳绮罗懒得看他,感觉他一点也不好看,没什么可看的,然而说出话来,语气中却是带了一点委屈:``我不知道那是什么东西,只是有点像壁虎,但是比壁虎大得多。''

张显宗把她里外的裤子一层一层卷起,卷到最后剩下一层紧贴小腿的长筒羊毛袜。张显宗握着她的脚踝仔细审视了她的袜筒,却是并未发现齿痕。

``好像是没咬透。''张显宗松了一口气:``我给你脱了袜子再看看。''

羊毛袜子脱下来,露出了红肿滚烫的脚踝。岳绮罗把赤脚蹬在了张显宗的怀里,脚心贴上军装一粒冰冷的铜扣。一只粗糙的巴掌握住了她纤细的小腿,她不动声色的抬眼去看他——看他,看不起他。

迎着她的目光抬起头,张显宗笑了:``不怕,只是扭伤了关节,贴两剂膏药就能好。''

岳绮罗一翘嘴角,也笑了。笑容一闪而逝,她其实没什么可笑的。

右眼一跳一跳的隐隐胀痛,无须照镜子,她知道自己眼中的一点血色正在扩散蔓延。直直的望着张显宗,她轻声说道:``我饿了。''

岳绮罗伸长双腿坐在床上,右脚脚踝已经贴了膏药。远处忽然起了一声枪响,不知是谁成了张显宗的枪下鬼。张显宗很能为她找人。死囚牢里的,街上流浪的,路边被人买被人卖的\ldots{}\ldots{}他手里总是不缺活人。

房门一开,张显宗端着个小碗走了进来。屋子里立刻起了复杂的腥气,岳绮罗从他手中接过小碗。翘起小兰花指捏住小勺子,她低着头,忽然说道:``我会保护你。''

张显宗一愣,随即又笑了:``好,谢谢你。''

他始终看岳绮罗都是个小小的妖女。而岳绮罗有时候自居为少女,看他是位体贴的大哥;有时候翻尸倒骨的把前世今生叠加起来,又老气横秋的看他还小。小,而且没有英豪的资质,怎么看怎么都是个太普通的男人,能够在文县当个小军阀,已经是到头了。

岳绮罗在怪物口中死里逃生,虚惊一场。张显宗听了她的讲述,一时不知如何是好,索性按兵不动。与此同时,顾大人在青云山下花天酒地,十分快乐,每天晚上都有一场吹拉弹唱,房内男男女女载歌载舞。及至歌舞毕了,便开始捉对寻欢。又因房子处在青云观内,从来没有听说庙观里闹鬼怪的,所以他分外安心,无所畏惧。

工人器械都还没影,勘探队伍自成一派,除了满山挖坑不干别的,军队也没有敌人可打,顾大人只能是玩。这晚他痛饮了一场烈酒,喝到最后扔了杯子就睡。勤务兵们生拉活拽的把他扯到了卧室床上去,而他御用的一个小妓女,名叫梅香的,趁此机会就向旅部的一名参谋飞起了眼风。参谋是个小白脸子,是梅香理想中的美男子;两人你看我我看你,看着看着就一起离了席,勾勾搭搭的不知所踪。

顾大人醉透了,呼噜打得震天响,乍一听宛如火车过山洞,轰隆隆的一声接一声,隔着一道门一座院都听得到。勤务兵一听他这个动静,就知道他已经睡得雷打不动;两名卫兵在门口冻得拱肩缩背,见勤务兵溜了,于是双方一合计,也悄悄钻进旁边一间小门房里烤火去了。

长夜漫漫,两名卫兵在小炉子上烤红薯,烤得聚精会神。而顾大人的呼噜响到极致,一口气忽然哽在了喉间。几秒钟的清静过后,他像匹马似的打了响鼻,把自己给震醒了。

屋内的炉子烧得很旺,顾大人只感觉自己满腔烈火,燥热的恨不能一个猛子扎进水缸里去。伸手向旁一摸,他没摸到女人,就睡眼惺忪的自己爬了起来,想要去找水喝。不料一脚伸到床下,他眨了眨眼睛,发现地上扑了个人影子。

他以为自己是睡迷糊了,特地抬手揉掉眼角一粒眼屎。睁眼再瞧,地上的人影子清楚了,看身形正是梅香!

梅香仿佛是进门时在门槛子上绊到了,一个大马趴就再没起来。顾大人挺诧异,出声唤道:``梅香?晕啦?''

然后他不情不愿的下床趿拉了棉拖鞋,先走到桌旁端起大茶杯,咕咚咕咚灌了一肚子冷茶。放下茶杯转向梅香,他对妓女是谈不到怜香惜玉的,伸脚就要去踢:``哎,至于吗?醒醒!''

然而他的棉拖鞋骤然停在了半空,因为在依稀的晨光之中,他看到了梅香空空瘪瘪的下半身。斗篷还在,裤子也在,甚至鞋袜都在,一股脑儿的浸在一摊不辨颜色的液体中,只有其中的肉体不在!

短暂的愣怔过后,顾大人抬手猛然拍向电灯开关,随即转身走到床前,从枕头下面抽出了一把手枪。哗啦一声将子弹上了膛,他单手套了棉手套,弯腰蹲在梅香面前,一把抓起她后脑勺上的大发髻。梅香顺着他的力道抬了头,一双眼珠将要瞪出眼眶,嘴巴张到极致,不知是要痛哭还是要惊呼。顾大人小心翼翼的试了试她的鼻息,发现梅香已经是面目狰狞的彻底死去了。

顾大人看出梅香不是好死,手一松放了对方的脑袋,他急急的起身,从屋角的箱子里翻出一件旧棉袄穿了上。棉袄还是月牙的针线,里面藏着两张纸符。当初无心从出尘子那里要来许多纸符,结果经过几次三番的使用过后,如今就只剩了两张。他不能像月牙似的,把护身符装进小荷包里挂在脖子上,于是索性让她将纸符缝进了棉袄的暗兜里面。系好纽扣之后,他把军裤和及膝的大马靴也穿上了。一脚把梅香踢翻过来,他不再看她的狰狞死相,只去研究她的□。□没了长斗篷的遮掩,薄薄的绸裤下面已经显出了腿骨的形状。顾大人随手拿过一只鸡毛掸子,弯腰用掸子长柄掀开了湿淋淋的裤管向内瞧,结果就见骨头水汪汪白生生的,并非是被野兽啃了,也不是被人用刀刮了,一身的血肉竟像是自己化了。

地上的尸水越来越多,顾大人只是一沉吟的工夫,梅香就连胯骨也塌了下去。顾大人见状不妙,一大步越过尸首跳到门外,同时抽了抽鼻子,发现尸水半透明的几乎不带血色,微微的有点腥,倒也谈不上很臭。凭着他的见识,自然知道梅香既不会是生了怪病,也不该是中了剧毒,到底怎么回事,恐怕又是谜团。

门房里的卫兵见旅座房内亮了电灯,连忙含着滚热的烤红薯跑了出来,抱着步枪重回岗位。不想还未等他们站稳,一名副官策马而来,下马之后也不讲明来意,直接就扯着嗓子大嚷道:``旅座,旅座,您醒了吗?营里\ldots{}\ldots{}出了点事,想请旅座过去瞧瞧啊!''

不过半分钟的工夫,副官就见顾大人戎装整齐,大步流星的走出来了。

顾大人和副官骑马前进,片刻之后就到了军营。副官且行且道:``不知道是在哪里咬的,王参谋自己都说不清楚,反正觉出疼的时候,已经被那东西一口咬住了。王参谋吓坏了,赶紧往回跑,可是跑着跑着就坏了事。现在\ldots{}\ldots{}旅座自己看吧,王参谋的腿都不行了。''

顾大人心里略略有了点数。下了骏马一扔缰绳,他一边往帐篷走,一边问道:``军医怎么说?''

副官紧赶慢赶,累得直喘:``军医说不是毒蛇,因为那东西嘴太大,咱们这地方就长不出那么大的蛇。但到底是什么,也不知道。军医给王参谋上了点蛇药,可是什么用处都没有。''

话音落下,副官眼尖,一伸手为顾大人撩起了面前的帐篷帘子。帐篷里面也吊了电灯,顾大人弯腰进去一看,登时一皱眉头。

王参谋的小白脸子彻底白成了纸,长条条的仰卧在一条躺椅上,不用细看,也知道他是出气多进气少。裤子已经被扒掉了,两条细长的白腿就搭在椅子上。一条腿还是正常好腿,另一条腿却是从小腿中间开始溃烂。白生生的腿骨露出来,骨上干净的连一丝血筋都无。上下两端的皮肉不见鲜血,反而是滴滴答答的流下黄水,椅子下面已经湿了一片。

帐篷里面围着几名与王参谋交好的军官,以及一名最有资格的老军医。见顾大人来了,众人连忙起立,而顾大人背着双手,直接问军医道:``他怎么不喊疼?''

军医的神情很像是在梦游,并且直打结巴:``报、报告旅座,王参谋好像是没、没有很疼。''

顾大人又问:``小王是在哪里被咬的?''

王参谋气若游丝,显然不能说话,于是旁边一名军官答道:``报告旅座,小王刚才说是在山里被咬的,还说咬他的东西挺大,像四脚蛇。''

顾大人沉默下来,心里明白了——小王和梅香跑到山里私通,不慎遇了怪物咬人。小王必是抛了梅香先逃了,而梅香受了重伤,又想活命,只能跑回自己房里求救。

梅香和小白脸偷情,顾大人并不吃醋,因为梅香又不是他的姨太太,两人无非是露水姻缘,说不定哪天就一拍两散了;梅香和小白脸因为偷情而死,顾大人也不怜悯。问题是他俩并非好死。至于所谓的四脚蛇,他和无心当初的描述一对照,立刻就知道了它的来历。但单是知道还不行,若是由着它肆意咬人,自己的军队非被它吓散了不可。

帐篷内的众人束手无措,眼看着小王烂到了肚破肠穿。最后实在是看不下去了,又探出小王已经咽了气,几名胆大的军官便用一块厚帆布把他裹起来,深深的挖坑埋掉了。

顾大人下了封口令,不许在场之人妄言。天明之后他回了自己的屋子,推门进去一瞧,发现梅香已然成了一具雪白的骷髅。

顾大人胆子大,光天化日之下更是胆大包天。用火钳子把骨头一根一根夹到一床棉被里,他包了个白骨包袱,想要去找出尘子设法。不料未等他出发,勘探队的队长来了。

队长是个斯文强壮的大个子,戴着眼镜,人很和气,想请顾大人派出一辆军用卡车,运送一尊佛像到天津去。

顾大人没听明白:``什么佛像?你们还兼收古董哇?''

队长立刻笑道:``非也非也,是一名队员偶然间挖到的,哎呀,非常美丽,可惜鄙人不通历史,不能鉴别出它的年代。我们想把它尽快送去天津,请几位老先生来看一看。如果真是罕有的宝贝,那我们也算是幸运之至了。''

\chapter{我来了}

顾大人和出尘子各自守着个蒲团相对而坐,面前摆着几根骨头,以及一副凝结着红渍的利齿。骨头是梅香的遗骸,利齿则是顾大人在上山之前,副官赶着送过来的。说是他们几个在天亮之后进了山,结果顺着脚步痕迹走到一处干燥了的土坑前,旁的没发现,只发现了孤零零的一副牙。军医一看牙骨的尺寸,就知道大家是找到凶手了。

凭着出尘子的智慧和口才,满可以把大牙安到三皇五帝身上去,并且能够把谎圆得天衣无缝,任谁都要赞叹他的有理有据。只要他愿意,他可以为一切未解之谜安排答案。可顾大人是无心的朋友,看在无心的面子上,出尘子不大好意思用虚话来敷衍他。但如果不说虚话说实话,出尘子就得承认自己对怪物束手无策。而他在近十多年里一直保持着无所不能的仙人形象,让他承认自己无能为力,如同迎面抽了他一个大嘴巴。

在验出骨上无毒之后,出尘子心乱如麻的开动脑筋,不知自己是应该继续向顾大人展示华丽一面,还是老老实实的袒露朴实本质。思来想去的叹了一声,最后他没头没脑的问道:``无心还没有来?''

顾大人发现出尘子只要一见自己,必定问起无心,就忍不住笑了:``他还在天津呢。他不愿意来,我也不强求他。''

出尘子自从在千佛洞内历过险后,如果身边没有无心,他简直都不愿再回忆起地下经历。不动声色的撩了顾大人一眼,他开口又问:``顾旅长,你知道无心的来历吗?''

顾大人立刻打起了精神,十分谨慎的答道:``他\ldots{}\ldots{}他就是个走江湖的呗,去年我家里不干净,有东西闹事害人,请他过去禳治了一次。后来\ldots{}\ldots{}后来我们就认识了。''

出尘子点了点头,又道:``近来夜里不要让人进山,尤其是不要靠近深坑水潭。青云山地下的玄机,恐怕不是凭着人力可以探明的。顾旅长,若让我说,放弃金矿方为上策。否则山麓一旦开挖,谁知道会放出多少怪物来?就算它们见光即化,可是防不胜防\ldots{}\ldots{}''

顾大人笑了一下:``道长,您说的都对。问题是老帅不发话,我们也做不了主啊。''

出尘子最通人事,当然了解顾大人的苦衷,于是最后又道:``如果要挖,一定要选在白天动工。一旦挖出了尸骸,立刻就地焚烧。''

顾大人在出尘子面前唯唯诺诺的答应了,离开道观回了军营,不料进了营门之后,发现卡车停在空场上,勘探队的队长像熊似的上车下车,正在指挥士兵将一只用木条钉成的长箱子往车上运。除了长箱子之外,地上还摆着两只方方正正的小木箱,顾大人走近一瞧,就见一口箱中放着一个菩萨脑袋,脑袋花里胡哨的,乍一看能吓人一跳;另一口箱子里则是放着两只手,连着半截小臂,也是色彩斑斓,不过雕工真好,连指甲都是饱满端正。队长见他来了,就跳下卡车,一路唉声叹气的走过来:``糟糕,真糟糕。''

顾大人问他:``怎么了?''

队长双手叉腰:``佛像出土之后,颜色立刻就变了\ldots{}\ldots{}''他伸手去指箱中的菩萨脑袋:``眨眼的工夫,竟然面目全非!''

顾大人莫名其妙的又问:``怎么就只有一个脑袋一双手?刨碎啦?''

队长摇了摇头:``不知道,我们就只挖到了这些,兴许是先前有人发掘过这里。可是据我所知,青云山上并没有什么古迹,即便是青云观,也是在近百年内修建的——真是奇哉怪也。''

顾大人没敢多言语,心想多一事不如少一事,我是来打仗的,不是来开矿的,更不是来挖老佛像的。师父和出尘子都替我探过一遍路了,明知道地下很邪,我何必还要跟着凑热闹?反正我在山下给金矿看门护院,山里爱有什么就有什么吧!

顾大人想的挺好,然而事与愿违,一夜过后,营里又出了事——一个帐篷里面睡了四名士兵,早上其中一人醒过来,发现三名同伴不见了,取而代之的是三具骷髅,统一的躺在行军床上。

消息扩散开来,营中立时大哗,偏偏昨天又有人从浅土里挖出一具很新鲜的尸首,尸首竟然还是前清的打扮。尸首见到天日之后,很快腐烂出了臭气,被人一把火烧了个干净。流言有了种子,众人在山下闲得泼烦,所以土壤也具备。不过两日的工夫,竟然开始出现了逃兵。顾大人慌了神,连忙去请出尘子设法,又将一封信寄了出去,催促无心过来帮忙。

信件从青云山出发,一天之后便进了天津卫,且被邮递员顺着门缝塞进了院内。院子里面没有人,因为无心带着月牙去北京了。

初春时节到了北京,无心和月牙先是遇了几天的大风。等到风平了,天空一碧如洗,接连着倒是有了几日的好天气。两人的精力与体力都充足,可逛的地方逛全了,可吃的风味也吃遍了,尤其令人快乐的是他们在照相馆里拍了好几张照片。当然,天津也有照相馆,可他们在天津就没想到过要合影。

两个人都是生平第一次照相,都坐在回天津的火车上了,无心还忍不住把照片拿出来看。照片上的两个人肩并肩,在照相师傅的指挥下歪着脑袋,也是头挨头。月牙起初怎么也笑不出来,后来好容易笑了,被照相师傅一按快门捕捉到了表情。他看,月牙凑过来也跟着看,同时小声说道:``是不是笑得太大了?''

无心摇头:``没有,笑得正好。''

月牙又道:``你看我是不是烫完头发就显老了?我咋瞅我像你大姐呢?''

无心认认真真的扭头端详了月牙,最后答道:``没有的事。''

月牙不看自己了,专心去看无心。无心是深眼窝直鼻梁,平时偶尔会显出一点阴森森的怪相,没想到上了照片却好。月牙拿过照片,用手把自己挡住了,只露无心一个人:``你看你,跟电影里的人似的。''

无心抬手一摩头顶:``可惜我的头发长不长,否则梳个分头就更好了。''

月牙笑道:``分头是肯定梳不成了,要不然回家给你做身洋衣裳穿?我到成衣店都问过了,连手工带料子,有二十块钱就足够。二十块钱咱们有啊!衣裳做好了,再给你买一双皮鞋,一顶礼帽,一根文明棍。还缺啥?对了,还缺一副黑眼镜。''

无心想象了月牙给自己设计的新形象,不知为何感觉十分可笑,下意识的就想把脸往月牙的头发里拱。月牙连忙推了他一把,咬牙切齿的低声骂道:``别不要脸,人都看着呢。''

月牙说到做到,下火车后直接带着无心去了成衣店,量尺寸选料子交定金。然后两人回了家,结果大门一开,就见地上躺着一封信。

无心撕开信封一瞧内容,发现信是三天前寄到的,自己已经是耽误了时候。连忙拟了一封回信出门邮寄了,他回来后对着月牙问道:``怎么办?顾大人让我们去呢!''

月牙在厨房切开了一个心里美的蔫萝卜,挑了一片比较水灵的递给无心:``我就知道他饶不了咱俩。去就去吧,他又出啥事了?''

无心嚼着萝卜答道:``没大事,我应该是能有办法。''然后他把咬过一口的萝卜送到月牙嘴边:``甜的。''

月牙咬了一小口,感觉的确是很甜,于是一推他的手:``你吃吧,我不吃了。吃多了爱放屁。''

无心听她仿佛是真不想吃,就咔嚓咔嚓的把萝卜全吃掉了。

两人在家又过了一日,到了第三天上午,一辆汽车开过来,把两口子一起接去了青云山。

顾大人先前天天和无心月牙混在一起,如今一旦分开久了,竟然满心思念。背着双手站在山路边上,顾大人一边迎风等待汽车,一边暗暗纳罕,没想到自己居然多愁善感,还会思念。

及至一辆小汽车当真一溜尘烟的开过来了,他登时不由自主的打了个立正。汽车刹住开了门,无心和月牙络绎钻出来,顾大人高兴极了,先对无心行了个拥抱礼,啪啪的在他后背上连拍了几大巴掌:``老不死的,总算来了!''然后一双眼睛转向月牙,野调无腔的又嚷:``月牙,几天不见你可是胖了啊,好家伙,粗腿大屁股的!''

顾大人的玩笑显然是不中听,不过月牙素来不和他一般见识,所以只道:``胖了咋的?胖了富态!往后不许你再叫他老不死的,他看着还是个小伙呢!''

顾大人心里痛快极了,豪气干云的一拍无心肩膀,随即继续和月牙斗嘴:``月牙,带上你家小伙跟我走吧。你家小伙太嫩,你可看住了,仔细他让狼叼去。''

无心双手插在衣兜里,一言不发的只是笑。而顾大人和月牙闹了一气,最后转向无心说道:``别偷着美了,我最近有点头疼的事,你得给我帮忙。还有青云观那老道总念叨你,你想着上山瞧瞧他去。另外我问你,你说现在人心已经不稳了,流言蜚语全传的有鼻子有眼,我怎么办?''

无心不假思索的答道:``做法事。''

顾大人一瞪眼睛:``屁话,营里又不是闹鬼,做法事有个□用?''

无心理直气壮的答道:``做法事给人看啊!流言蜚语是半真半假,法事也是半真半假,正好相生相克,是个对子。''

顾大人咽了口唾沫:``那做完法事呢?''

无心对着顾大人眨巴眨巴眼睛:``我不是来了吗?难道我是来玩的?''

\chapter{深入虎穴}

顾大人派人撒网,漫山遍野的抓黑狗。青云山附近几乎没有像样的村庄,村庄里也都以黄狗居多,所以为了抓住几只没有杂毛的纯粹黑狗,小兵们很是费了一把子好力气。

``黑狗能吃怪物?''他问无心。

无心正在一位副官的教导下练习射击,听了顾大人的问话,他把手枪交还给副官,然后带着顾大人一边往远处走,一边低声答道:``我总觉得那怪物有点邪,所以想要预备几样辟邪的东西,在它身上试一试。单是黑狗血还不够,我还想多带几样。''

顾大人睁大了眼睛看他:``还想要什么?带到哪里去?''

无心对于后一个问题避而不答,只说:``有没有童子尿?''

顾大人一耸肩膀:``没问题啊,队伍里有不少半大孩子,十个里面总有一个是童子吧?''

无心迟疑着又问:``有没有老童子?''

顾大人掏着耳朵问道:``多老?''

无心思索着答道:``三十岁往上。''

顾大人当即``哈''的笑了一声:``三十多岁的童子?你还是让我给你找一条三十多岁的黑狗吧!''

此事就此放下不提,顾大人预备出了一份厚礼,带着无心去了青云观,要请出尘子到军营做一场法事驱邪。大概是一同出生入死过一次的缘故,出尘子对无心生出了一种隔世相见的亲切感。看在无心的面子上,他竟然连住持道长的大架子都没摆,一口就答应了顾大人的请求。顾大人放了心,开始凑趣聊闲话,说着说着,无心忽然开了口:``道长,有句话,不知当问不当问。''

出尘子飘飘欲仙的微笑点头:``可以问,贫道事事都可与人言,并无忌讳。''

无心看了顾大人一眼,然后向出尘子微微探头,郑重其事的说道:``道长,你是童子吗?''

出尘子一愣,瞪着无心半天没说出话。而顾大人啼笑皆非,连忙圆场:``师父你胡说了啊,人家是修道的出家人,肯定是——''顾大人想要琢磨出个文雅的词来赞美出尘子,想了又想,末了福至心灵,一拍巴掌:``肯定是三贞九烈、冰清玉洁啊!''

无心不以为然的一摆手:``道长又不是全真派,不讲那些死戒律。''然后他变戏法似的从衣裳里面摸出一只军用水壶,转向出尘子又道:``道长,实不相瞒,我想弄点法力高强的童子尿。你要是童子的话,给我尿一壶如何?''

出尘子的白脸上青一阵红一阵的,一只手就搭在旁边的炕桌上。顾大人见他总不言语,不禁脸色一正:``道长,您\ldots{}\ldots{}不是童子?''

出尘子嘴角一抽,随即抬手重重拍向桌面,气急败坏的大声怒道:``妈的,粗俗,给我滚出去!''

一秒钟的停顿过后,他又吼了一句:``水壶留下!''

无心双手把水壶放在炕桌上,然后笑微微的点头哈腰,恭而敬之的扯着顾大人退出去了。

无心领着顾大人下山回应,顾大人一路上唠唠叨叨,怀疑无心得罪了出尘子。无心满不在乎:``唉,要得罪早得罪了,还差今天一句话?''

结果到了傍晚时分,果然有一位器宇轩昂的大道士送来了一只沉甸甸的水壶,以及一只大食盒。自从梅香化在房内之后,顾大人就悄无声息的搬进了军营里住。大道士走后,月牙先进来了:``哟,啥啊?''

无心揭开食盒盖子一看,当即笑了:``是点心。''

点心很精美,全用模子扣成了梅花形状。月牙刚拿起一块要吃,顾大人也进来了:``哟,哪里来的?''

无心笑道:``出尘子刚才派他的徒弟来,给我送了一盒子点心和一壶尿。''

月牙嚼着点心回味丈夫的话,想着想着就有点咽不下去了,并且感觉房内臊气烘烘。

无心经过了两日的筹备,第三天的夜里,他带上三只水壶以及一只放在厚棉套子里的玻璃瓶,领着顾大人以及顾大人的心腹军官进山去了。

水壶里分别装着黑狗血、童子尿和火油,玻璃瓶里则是按照顾大人的主意,盛了一瓶子镪水。一行人翻山越岭,最后到达了出尘子所布置的假坟前。刨开假坟掀开铁板,顾大人和部下守在入口旁边,将随行带来的两口铁皮箱子敞开放好,而无心重走旧路,向下进入了斜洞之中。

此时月明星稀,夜风已经不算寒冷。顾大人在洞边地上拢了一堆火,席地而坐静静等待无心。隔三差五的摸出一只怀表看看时辰,他第一次感觉时间过得太快,而无心怎么还不上来?

自从无心入洞之后,部下军官全都抱着膝盖,神情肃穆的一言不发。顾大人暗暗忍住了一个哈欠,忽然想道:``如果无心不出来了,我怎么办?''

他虽然胆子不小,可也不敢贸然下洞,于是又想:``月牙非哭死不可。''

顾大人和军官们叼起了烟卷,一口一口慢慢的抽,大张嘴的铁箱子旁边也摆着个水壶,壶里盛着黑狗血。烟草的气息弥漫开来,军官们仿佛受到了一点刺激似的,慢慢的也活泛了。有人问顾大人:``旅座,黑狗血真能打鬼?''

顾大人沉吟着答道:``能是能,但是力量不大,大概也就是能把鬼吓一跳吧!''

有人又问:``旅座,你说咱们周围会不会有鬼?''

顾大人对他一摆手:``别他妈妖言惑众扰乱军心,老子做旅长的都不怕,你们几个穷鬼怕个屁?真要是来了鬼,本旅长第一个上!''

军官立刻竖起了大拇指恭维:``旅座霸气!''

顾大人咬着烟卷,正要继续发出豪言壮语,不料身边洞中忽然窸窸窣窣的起了响动。他立刻来了精神。``呸''的一声把烟头吐到火里,他率先起身走到洞口,就见黑土之中赫然扒着一双惨白的手,正是无心要上来了。

他登时松了一口气,伸手握住两只白手,他一边往上拽,一边问道:``怎么着?白跑了一趟?''

无心很重,他轻描淡写的一拽,竟然没拽动。双手握紧了,顾大人正要再次使劲,可是就在将要发力之际,他忽然感觉不对劲。猛然低头向下望去,他就见自己手中的白手骨节分明,皮肤丰润,软软腻腻的带着水分。而在他的记忆中,无心的手可是单单薄薄的挺秀气!

与此同时,洞中缓缓伸出了一个血肉模糊的圆脑袋。火光之下顾大人看得分明,就见对方如同火海里爬出的活鬼一般,皮肉丝丝缕缕的或鲜红或焦黑,两只眼球骨碌碌的鼓凸着,鼻子只剩两眼孔洞,一口乱七八糟的牙齿尽数暴露在外。一股子焦臭之气扑鼻而来,顾大人大叫一声,发现对方居然衣着齐整,从脖子往下还有成片的苍白皮肤存留。

一瞬间的惊惧愣怔过后,旁边的军官有了反应,怪叫着拔枪抡刀,可是顾大人的双手被活鬼死死攥住,让人不敢轻易上前,只怕误伤了旅座。而顾大人弯腰抬腿,想要一脚把它踹回洞内,不料它骤然向上一窜,几乎把个恐怖的脑袋撞上了顾大人的面孔。顾大人立刻仰头一躲,随即后退几步,竟是把它带出了洞。

一名军官拎起水壶,哆哆嗦嗦的先把一壶狗血淋向了活鬼。活鬼受到了袭击,果然身体僵了一下。两名军官左右夹击挥起砍刀,硬生生的砍断了活鬼的两条手臂。顾大人匆匆甩开两只鬼手,随即拔出手枪,不打脑袋,专打关节。一连串枪响过后,他一边换弹匣一边后退;而他的部下有样学样,立刻上前补枪。活鬼的四肢全被打断,瘫在地上动不得,其余人等抄起砍刀,因为都吓得要发疯,所以分外狠辣,一顿寒光将活鬼剁成了肉泥。

最后,众人气喘吁吁的围着一地骨肉站住了,其中一人试试探探的出了声音:``我们\ldots{}\ldots{}刚杀了个什么?''

顾大人此刻虽然也不确定到底死了个什么,不过因为见多识广,所以能够不假思索的编出答案:``杀了个煞!知道什么是煞吗?告诉你们,就是恶鬼修炼成了人形!''

军官们一起打了个寒战:``我们\ldots{}\ldots{}这么厉害吗?''

话音未落,洞口突然起了咣咣两声巨响。众人慌忙一起回头,就见两只大铁箱子竟然一起合了上!

随即在两口铁箱之间,一个黑影游动而出。

顾大人像被钉在了地上似的,拖不动腿抬不动脚,只能颤颤巍巍的唤道:``是无心吗?''

黑影很冷静的做出了回答:``顾大人,扶我一把,累死我了。''

按照事先的安排。军官们一拥而上,用铁链捆紧铁箱,然后几人合力把铁箱往山外抬。箱子里面一直有动静,时而是扑通扑通的跳跃顶撞,时而是吱吱呀呀的抓挠啃咬。顾大人的心腹,再孬也比一般人胆子壮,抬着箱子一路疾行,风似的掠地而过。

顾大人跟在一旁,背着无心。无心显然是真累了,下巴搭在他的肩膀上,一丝气息都没有,脑袋随着他的步伐左右摇晃。

天亮之前众人回了军营。两口铁箱子被送进一间营房里去,营房的窗户上面蒙了一层厚毡子,四边用钉子钉在了窗框上,遮得一丝光都不漏。

无心没有受伤,只是疲惫不堪,像一条垂死的大蛇一样盘在床上。月牙等了他一宿,如今见顾大人把他全须全尾的背进屋了,连忙热了昨晚的剩饭剩菜给他吃。待他吃饱喝足之后,顾大人见他身上的水壶玻璃瓶全没了,便开口说道:``夜里在你露面之前,洞里钻出个妖怪,哎哟我操,可他妈吓人了,但是本领一般,让我打了个稀碎。''

无心半闭着眼睛答道:``我知道是谁,不是妖怪,是一具行尸走肉。''

顾大人想了想:``不对啊,你上次不是说老道布了个什么阵,把一大帮活死人全封在洞里了吗?''

无心摆了摆手:``出尘子的道行,不能不信,也不能全信。一百具尸首能被他封住八十具,就算好样的了。我进了洞后没往深处走,直接就登高想要从洞顶往上爬,不料惊动了一具尸首。我没时间理它,直接泼了它一头镪水,没想到它虽然不再追我,但是偷着跑出洞了。''

顾大人压低声音问道:``你又进怪物堆里去了?''

无心彻底闭了眼睛:``我试过了,一般的东西全伤不了它,镪水都没有用,好像它只怕日月星三光。想要把它杀尽也不可能,太多了。''

顾大人用手指从上往下,用力杵到床上:``既然怕光,我就把山挖开,晒死它们如何?''

无心犹豫着摇了头:``洞子的上方好像是一层石壳子,想要挖开,怕是不容易。''

然后他坐了起来,伸腿下床:``我抓了两条活的回来,现在就去研究研究它们。顾大人,你要替我守好房门,千万不要透光进房。''

无心进了放置铁箱的黑屋子,房门一关,门缝里都塞了毡子,屋内竟然黑到伸手不见五指。

一队士兵围住房屋,不许闲人靠近。与此同时,遥遥的传来鼓乐声音,原来法事就定在今日,此刻天光刚亮,出尘子梳洗打扮穿了法袍,在徒子徒孙们的簇拥下坐上一顶华丽大轿,前呼后拥的下山来了。

\chapter{战火}

法事做得太漂亮了。

顾旅驻扎在青云山下的先遣部队,开大会似的排列大队,鸦雀无声的举目瞻仰高台上的出尘子。各级军官提前就向部下宣扬过了出尘子的高贵身份——活神仙,国务总理见了他都毕恭毕敬。

出尘子身穿绣花法袍,人在高处,下面看不清法袍花样,就见一片金碧辉煌。出尘子本人也体面,宽肩膀大个子,一动不动都显威仪,披发跣足的舞起一把七星剑,下方静得只余风声。有个小兵忍不住发出一声咳嗽,当即被身后的班长兜头扇了一巴掌。

顾大人带着月牙远远站了,也伸着脖子看得发呆,两人一时入神,全把无心给忘了。

无心软绵绵的躺在黑屋子里,脑袋枕在怪物的脊背上。还是累,非得睡足一天才能行。怪物并不肯攻击他,大概是根本没把他当成活物,以为他是石头,不能吃。

屋角还趴着一只怪物,脑袋被他切下来了,基本可以算作死掉,但是因为没有光,所以皮肉并未融化。无心割了一块肉,自己低头嗅了嗅,又张嘴咬了一口。肉微腥,像是没熬好的鱼冻。他又拿了肉去喂活着的另一只,另一只闭着大嘴,显然完全没有要吃的欲望。于是他一歪身躺下来,枕着怪物自己吃。

中午他出来了一趟,怀里抱着半截怪物身体。守门的士兵依言为他捉来一窝狗崽子。无心拎起一只小白狗,用融化的肉汁从头到脚涂抹了它。狗崽子送到了怪物面前,怪物还是无动于衷。

于是无心再次出门,挑了一只花狗,灌了满狗嘴的肉汁。等到小花狗喝饱了,无心把它送去怪物面前。小花狗叫都没叫一声,``咔嚓''一声就被怪物整个活嚼了。无心立刻抽手后退,险些在怪物的尖牙上刺破手指。

无心认为自己是摸清了怪物的习性。法术还未结束,他已经把怪物晒成了两壶汁水。及至法事结束了,顾大人和月牙来看无心,结果就见房门大开,几名士兵正在拆卸钉在窗户上的厚毡子。顾大人开口一问,得知无心是去了溪边。

此刻冰消雪融、春暖花开,山里是不缺少小溪的。在一条小溪旁边,顾大人和月牙看到了无心——无心蹲在水岸,正在用一把刷马的刷子用力刷着什么。

两人蹑手蹑脚的走近了,探头一看,月牙吓得惊呼一声,顾大人则是当即问道:``你干什么呢?''

无心一手握着刷子,一手将一副白森森的利齿摁在水中,仰着头笑问:``顾大人,你看它好不好看?我很快就能把它刷干净了,刷干净了送给你。''

顾大人莫名其妙的看了月牙一样,然后问道:``送给我一副牙?''

无心低下头,继续用刷子拼命刷洗齿缝中的干涸血涎:``我觉得它很凶,可以用来给你镇宅。''

顾大人没理他,直接对月牙说道:``你管管他,知道他是好心,可要是由着他发神经,兴许过两天他就要往家里收尸首了!''

月牙也弯腰在他后背上打了一巴掌:``你赶紧把它扔了,看着恶不恶心吓不吓人?你看谁家用一口牙镇宅了?''

顾大人和月牙强行没收了无心的利齿和刷子。利齿和刷子被扔进了溪水里,顾大人和月牙一左一右握了无心的手,左右夹攻的把他押回了营房。两人都对他的行为深恶痛绝,顾大人发出恐吓,说无心如果再敢做出类似行为,就把他的爪子剁掉;月牙立刻发话:``你别吓唬他!''然后一扯无心的手臂:``听见没有?再也不许你往家里带怪东西,否则我先挠死你。''

无心一片好心,结果不但没有落到半句好话,反而还被妻友分别威胁了一通,不禁啼笑皆非,一边点头一边走路:``嗯,嗯,我再也不敢啦!''

出尘子做完法事之后,没有即刻离去。在营房内换了一身便服,他斥退身边徒弟,舒舒服服的坐下了端起了一杯热茶。刚刚气定神闲的啜饮了一口,无心无声无息的走上前来,把一只军用水壶放到了他的手边桌上。

出尘子一愣:``干什么?又想向我要尿?''

无心在一旁陪着坐下了:``不是要,是给。一点小礼物,有防身的用处,请道长收下吧。''

出尘子轻轻嗅着茶水氤氲的香:``礼物?是什么?''

无心答道:``是地下怪物化成的汁水。''

出尘子一口热茶当即喷出,喷了无心一头一脸。无心抬袖子一抹脸,继续把话说完:``道长,如果将来你偶然遇了怪物,只要把汁水涂在头脸身上,应该就可以逃过一劫了。''

出尘子简直不愿触碰水壶,非常勉强的向无心道了谢。然后叫来一名不明真相的小徒弟,让小徒弟捧了水壶。

待到出尘子坐上大轿返回道观之后,顾大人随着无心回了卧室。营房里没有床,砌着火炕。月牙坐在炕里,正在嗤嗤的纳鞋底子,而无心和顾大人也上了炕。无心对顾大人说道:``开金矿是可以的,不过很危险,最好是不要开。''

顾大人捏着一根烟卷,在炕沿上轻轻的磕,磕到最后他把烟叼在嘴上,``嚓''的一声划了一根火柴。捧着火苗凑上烟卷,他从浓眉下面向无心射出两道目光:``只要别把烂摊子砸在我的手里,哪怕山里藏着一条活龙,我都不管!''

无心转身拉过放在炕上的点心盒子,从里面拈出一只蜜饯枣子送到月牙唇边。等到月牙先吃一个了,他才又拈一个扔进自己嘴里。而顾大人继续说道:``我想好了,我要打仗!''

两道白烟从他的鼻孔中呼出来,是两条带着力度的小白龙:``想要暂时和青云山脱离关系,唯一的道路就是开战。我宁可上战场,也不想再和怪物打交道。反正金矿我发现了,我也交给老帅了;从此我上大路往远走,谁愿意来开矿,谁就来。谁死了谁活了,和我也没关系!妈的老子是军人,不是矿工。烫手的山芋别往老子怀里扔,老子才不接!''

顾大人一边说话,一边咬牙切齿,满脸都是恶狠狠的缺德相,坏模样全露出来了。月牙正在专心致志的穿针引线,无心则是靠在月牙身边,不置可否的吮着一枚蜜枣。吮着吮着,他被月牙推了一下:``你离我远点,我做活呢,别扎着你。''

无心没有动,歪着脑袋望着月牙笑眯眯。月牙扭头看了他一眼,忍不住也笑了:``看啥啊?说你呢!不怕挨扎啊?''

无心伸手搂住她的细腰,晃着脑袋就要往她怀里滚。月牙连忙把拿针的右手高高举起来了,用未完工的鞋底子轻轻打他的后脑勺:``看你的烦人劲儿,你还想不想穿新鞋了?''

顾大人踌躇满志,不料忽然失了听众。眼看月牙雷声大雨点小,对无心作势要打,其实打一下揉三揉,力气轻的都不如一阵风。四脚着地的爬过去抢过鞋底子,只听``啪''的一声,他对着无心的后脖颈来了一下狠的:``你俩是怎么回事?我说完了吗?我还没说完呢,你俩等会儿再骚!''

然后他盘腿坐回原位,双手搭在膝盖上,烟卷叼在嘴角上:``趁着形势没恶化,我得赶紧脱身。反正迟早得开战,我就先迈一步了!师父不能走,留下来给我做帮手;月牙你怎么着?你要是害怕,我就送你回天津去。''

月牙不假思索的摇了头:``我也不走,无心在哪儿我在哪儿。你们打仗的时候,我就找个地方猫着,不打仗了,我给你们做饭。放心吧,我胆不小。我小时候还和我舅舅进山打过狐狸呢。''

顾大人一抬手:``你胆大我知道。你胆子要是不大,早让他吓死了。''

无心立刻向顾大人使了个眼色,不许顾大人多说,怕把月牙说得起了心事,会不要自己。顾大人会意,也知道他找个女人不容易,所以立刻闭了嘴。

顾大人一旦暗暗下了决心,便立刻开始了行动。长安县一直天下太平,不是他的目标,他的目标是文县——因为他是被人从文县撵出来的!

派出队伍小打小闹的挑衅了几次,张显宗果然如他所愿的开了火。战事一起,顾大人立刻发出急电,让驻扎在天津城外的顾旅主力全部开来前线。老帅也不提金矿的事情了,只是密切关注战情。

无心和月牙随着队伍离开了青云山,一起驻扎在了距离文县有八十里远的一处小村庄里。两人都和官兵们保持着距离,因为官兵们见了女人,虽然明知道不能碰,可两只眼睛还是要生出钩子。无心怕月牙吓着,恨不能生出两只翅膀包围住她。月牙除了无心谁也看不上,所以等闲也不出门,一门心思做她的鞋,另外就是早晚三顿饭。

战事很快进入了僵持阶段,顾大人有后盾,底气足;张显宗却是只有文县一处大本营。抢矿的事情自然是早就不想了,有光兄弟见势不妙,也脚底抹油一起逃之夭夭。张显宗独自站在司令部里,对着半面墙的大地图若有所思。早春三月,青黄不接,再扛下去,城里就要闹饥荒了。他不能坐以待毙——为了岳绮罗,他也不能束手就擒。

半软半硬的指挥鞭点上地图,一路从文县移动到了顾旅的总指挥部。张显宗面无表情的盯着总指挥部,同时用指挥鞭一下一下的戳。

最后,指挥部上乌云盖顶,被他戳出了一团浅淡的黑。忽然把指挥鞭向后一扔,他转身大踏步的向外走去。

傍晚时分,岳绮罗策马而来,下马之后里外走了一圈,揪住一名军官问道:``参谋长到哪里去了?''

她个子矮,伸手抓着军官的衣领。军官比她高了两个脑袋,可是乖乖的俯下身,因为感到了莫名的恐怖:``参谋长亲自带兵出去了。''

岳绮罗用一双清澈分明的大眼睛看他:``去哪里了?''

军官抬手掩口,嘁嘁喳喳的对她耳语了几句。岳绮罗点头放下了手,一颗心渐渐的向上提。忽然抬头又望向军官,她开口问道:``大部队出发了吗?''

军官答道:``马上出发。''

岳绮罗抬起双手,手指插进了满头乌发。双手缓缓向后拢去,半短的黑亮头发滑过指缝,纷纷散乱。她的小脑袋成了一朵心事重重的、黑色的花。

最后像下了某种决心似的,她忽然说道:``我也去!''

\chapter{偷袭}

春日的凌晨,张显宗带着一支小队伍,悄悄靠近了顾旅指挥部所在的唐各庄。

对于一场偷袭而言,凌晨比午夜更合适。凌晨时分,人睡得最沉最熟,支持了一夜的卫兵们也疲惫了,都在拄着步枪打盹。村子里的公鸡还没有开始打鸣,张显宗没入黎明前的黑暗,一步一步的进入了唐各庄地界。

根据侦察兵事先提供的情报,他开始寻找村中最为高大坚固的房屋。身后的百十来人全都屏住了呼吸,双手紧紧的握了步枪,不肯发出半丝异响。悬着一颗心走入村中巷道,周遭除了偶尔的狗叫便是连绵的风声,一切都很顺利,前方出现了一名士兵的影子,正靠着半截土墙犯迷糊,依稀听到脚步声音了,士兵打着哈欠说道:``口令!''

没有口令,只有一把刀抹上了他的脖子;鲜血喷出红色的扇面,激射到了半截土墙上。

张显宗的队伍继续前进。在下一个巷道口,他们又遇上了士兵。士兵倒是比先头的死鬼有精神,大声嚷道:``口令!''

张显宗等人并不知晓顾旅的口令,所以低着头继续往前走。士兵``哗啷''一声拉了枪栓,声音提高了一个调门:``口令!''

张显宗抬手一枪,当场打碎了士兵的脑袋。枪声一起,四方的家犬都有了知觉,而张显宗向后一挥手,小队伍加快速度,直奔前方的砖石院落而去。据他所知,唐各庄中的驻军并不多,顾旅的士兵都在前线上!

天边现出了鱼肚白,鸡鸣狗吠伴着枪声此起彼伏。顾大人猛然坐了起来,眼睛还没睁开,下意识的伸手先从枕下摸出了手枪。光着屁股一步蹿出被窝,他先从玻璃窗子向外看,就见卫兵端着步枪正在往院外跑,便连忙转身去找衣裤往身上套,同时口中高声吼道:``无心,月牙!快醒醒,出事了!''

无心和月牙睡在隔壁,早在顾大人开口之前,也一起被枪声惊醒了。月牙还没醒透,愣头愣脑的拥着棉被发呆;无心却是伶俐,一掀被窝作出了回应:``知道!已经醒了!''

无心的声音一起,月牙的神魂立刻归了位。把衣裳裤子劈头盖脸的全扔向了无心,她强忍着不哆嗦,怕吓着谁似的小声说道:``快穿上。穿好了咱们往院子后面躲,后面通着庄稼地呢!''

无心一边往两只脚往裤子里蹬,一边说道:``傻丫头,现在庄稼地里又没庄稼,光秃秃的去了也白去!''

月牙的手指头快要忙出花来,一鼓作气扣上了一长串扣子:``哎呀,可不是!''

无心穿了鞋,拽着月牙的手就往外跑,出了房门之后,两人正好和顾大人打了个照面。顾大人无暇多说,只大声喊道:``妈的是偷袭!你俩别添乱,快往后走!''

想要往后走,也得先经过前方的院子。无心把顾大人和月牙全拦在身后,第一个露面走了出去。结果他的眼睛刚刚见了天日,一名卫兵在前方的院门口猛一抽搐,正是已经中弹身亡。顾大人大骂一声,推开无心举起手枪,一路扣着扳机向外走。而无心紧紧攥住了月牙的手,想要带她尽快冲出院门——方方正正一座院,如果不出门,就得翻墙,可是翻墙更危险,因为人在高处,目标明显。可是未等他迈出步子,忽有一人冲了进来,对着顾大人迎头一枪,正是张显宗!

在月牙的惊叫声中,无心纵身一跃,在硬生生的撞开顾大人同时,腰间被子弹开了个小小的血洞。顾大人猝不及防的跌倒在地,一头撞上了院角的大水缸,而无心趁着张显宗还没做出反应,几大步跑过去想要夺枪。可是一夺不成,二夺也不成。月牙跑去扶起了顾大人,顾大人头上没伤,然而愣眉愣眼的坐着直晃,竟然是被撞迷糊了!

张显宗不能再放仇人逃生,一边呼唤部下士兵支援,一边疯狂的想要甩脱无心。无心握住了他的右腕,正在想方设法的要掰开他的手指缴枪。他没法开枪,身上又没带军刀,急得只能拼命捶打无心。一队士兵交战着经过了院门口,子弹在空中带着尖啸穿梭,有人似乎想要进院支援张显宗,可是被子弹封锁了道路,咫尺的距离,竟然就是不能经过!

顾旅的援兵还没有赶来,张显宗的援兵也在不远的路上。唐各庄里有限的士兵厮杀成了一团,人人都有对手,想做逃兵都不可得。张显宗无法收回右手,索性不加瞄准又扣了扳机。子弹打在砖墙上,红砖碎屑簌簌的向下落进了月牙的头发里。月牙瞬间竖起了一身的汗毛,冷汗顺着鬓角往下流。弯腰扯住顾大人的一条手臂,她使出吃奶的力气要把人往屋里拖。屋子里虽然没退路,可毕竟墙厚,足够人支撑一阵子。顾大人受了惊动,像是清醒了一些似的,从鼻子里哼出一声,嘴里咕哝道:``妈了个×的。''

然后他把枪又拿起来了,想要射击,但是两眼发花,手也哆嗦。与此同时,无心和张显宗已经厮打到了院角。院角堆着一座小小的柴禾垛,无心一脚踏上柴禾,随即一跃而起,竟然是窜上了张显宗的肩膀。双腿夹住对方的脖子,他一弯腰,正好紧紧搂住了张显宗的脑袋。张显宗的面孔埋在他的胸腹之间,眼前一片漆黑,什么都看不见了。发了狂似的转身冲向院墙,他一下接一下的往墙上撞,想要把无心撞下来。而无心的后背接二连三的磕在坚硬的墙壁上,有心扭断对方的脖子,可是腰间枪伤疼得厉害,让他几乎使不上劲。

月牙蹲在门口,见无心腰侧已经漫出了小小的一块血迹,就急得使劲推搡顾大人。而张显宗感觉箍在自己脖子脑袋上的大腿手臂似乎松了些许,越发咬紧牙关使出全力。双脚发力冲向前方,他大喝一声,竭尽全力的顶向了院墙。无心闭上眼睛,绷紧身体想要扛过撞击。不料就在后背将要触到墙壁之时,院内忽然起了一声枪响!

张显宗立刻僵住了动作,无心抬头望去,就见月牙双手握着顾大人的佩枪,正战战兢兢的站在自己面前。枪口缭绕着似有似无的青烟,月牙的手指就勾在了扳机上。

院子里面静了一瞬,随即张显宗身体一歪,带着无心倒了下去。

无心立刻松开手脚爬了起来,而张显宗姿态扭曲的趴在地上,后背已经被轰出了一个血窟窿。枪和枪是不一样的,顾大人的盒子炮,威力和重量都只比步枪差一点。月牙也是个有力气的小女人,可是抄起顾大人的手枪跑过来射击时,她是抡起胳膊使足了劲,才勉强把枪端平了的。

一枪开过,月牙的腿都硬了,站在原地动弹不得。双手被枪坠得慢慢下沉,可还紧握着枪柄不放。无心把张显宗翻成仰面朝天,发现他大睁着双眼,是个死不瞑目的模样。

正当此时,一名副官气喘吁吁的冲了进来:``旅座,咱们的人和敌人在村外交火了!战况不明,您先撤吧!''

顾大人扶着门框站起来,心里越来越清楚了,天旋地转的一点头:``好,撤!''

顾大人骑着高头大马都跑出村了,才彻底恢复了神智。他难以置信的问无心:``什么?月牙把张显宗毙了?''

无心趴在马背上,点头``嗯''了一声。

顾大人立刻扭头去看月牙:``你个小娘们儿,够厉害啊!还会开枪?''

月牙一张脸红成滚烫,虽然对张显宗是不得不杀,但人命毕竟是人命。她脸上热,身上凉,抬起手满脸的抹泪,带着哭腔答道:``啊,我小时候跟我舅舅进山打过狐狸,用过汉阳造。''

顾大人长长的伸出手臂,在她肩膀上拍了一下:``别哭,哭什么啊?你不杀他他就杀你,开枪开得好,早就看你不是一般的娘们儿。''

然后他又转向了无心:``你总趴着干什么?''

不等无心回答,月牙哭道:``你是啥脑袋啊?他给你挡了一枪,你都忘啦?''

顾大人抬手摸着头顶青包,恍然大悟。

顾大人带着部下亲信成功突围,因为知道张显宗已经死了,所以心满意足的弃了唐各庄,另寻安全地方落脚。而村庄外的一场混战结束,前来接应支援的张旅队伍,终于在一场厮杀之后进入了唐各庄。

有士兵在一处院落里发出了单枪匹马的惊叫:``参谋长!参谋长让人打死了!''

一个小小的身影花蝴蝶似的飘了进来,岳绮罗一指头捺上了士兵的眉心。士兵怔了一下,登时仰面朝天的倒了下去。而岳绮罗随即蹲在张显宗身边,伸手一试,发现他的鼻端隐隐似乎还有一丝热气。

三下五除二扯开了他的军服,岳绮罗蘸着他的鲜血,在他胸前画起了符。而张显宗大睁着眼睛望向天空,仿佛有所感应似的,在岳绮罗的身边呼出了最后一口气。

张旅的士兵占领了唐各庄,可他们很快发现占领毫无意义。唐各庄孤零零的位于顾旅后方,顾旅随时可能反扑,届时他们逃都逃得艰难,因为此地距离文县大本营实在是太远了。

军官们在村内搜查了一气,没有任何成绩。忽然意识到参谋长一直不曾露面,有人慌张了,开始满村子呼唤张显宗。正是混乱之时,张显宗出现了。

张显宗浑身是血,破烂的军服之中,可见里面缠裹着衬衫撕成的绷带。一步一晃的走到军官面前,他没有多说,直接下了撤退命令。

因为参谋长受了伤,所以在岳绮罗的授意下,士兵理直气壮的从村里抢了一辆大马车。岳绮罗扶着张显宗钻进车内,张显宗坐下之后,就不动了。

鲜血还在源源不断的向外渗,岳绮罗伸手摸了摸他的脸,面孔已经冰凉,皮肤也在失去弹性。张显宗想要眨一眨眼睛,可是眼皮已经不听他的使唤。

马车上了路,在辘辘的车轮行进声中,他轻声问道:``绮罗,我真的死了吗?''

岳绮罗正襟危坐的面对了他:``放心,无论死活,我都会保护你!''

张显宗望着他,渐渐僵硬的面孔上露出了绝望神情:``我不想死\ldots{}\ldots{}''

岳绮罗清清楚楚的答道:``不想死,就不死!''

\chapter{活死人}

张显宗站在岳绮罗的面前,血迹斑斑的军装上衣已经脱掉了,层层缠裹的肮脏绷带也解开了,胸腹间是手掌大的创口,鲜血流尽,可以看见皮下薄薄一层黄色的脂肪,以及青紫斑斓的混乱内脏。

呼吸的欲望消失了,一切欲望都消失了,他甚至感觉不到了痛苦。缓缓抬起一只僵冷的手,他仿佛看到了一块阴暗的尸斑,然而凝神望去,却又没有了。窗外风和日丽,鸟语花香,他扭头凝视着大好的一派明媚春光,失去光泽的眼睛忽然蒙上了一层冰冷的泪。

``绮罗。''他声音喑哑的开了口:``我是变成丁大头了吗?''

岳绮罗不屑于为任何人动心,可是静静的望着张显宗,她的右眼毫无预兆的刺痛了。埋伏在眼内的血点开始有了扩散的趋势,她忍着痛不动声色,只答出一个字:``是。''

张显宗高高大大的站在春光中,青灰色的面孔上面流露出一丝苦笑:``我想活。''

然后他转向了岳绮罗:``可是,也许我死了更好。''

岳绮罗在他面前岿然而立。双手揣在袖子里,她用单薄的小嗓子说道:``张显宗,我会保护你的灵魂。''

然后她从袖子里抽出一条手帕,走上前去仰起了头,举手为他拭去了面颊上的泪光。

张显宗微微垂下了头,不想让她太费力气。没想到她也会如此的善待他,可惜他已经死了,她善待的不是活人,是尸首。

岳绮罗掩人耳目的运来净水,然后斥退仆人关严房门,又派卫兵防守在外。高高挽起两只衣袖,她露出了两条雪白的细胳膊。握着剪刀剪开了张显宗的胸腹,她掏出了他的五脏六腑。

毛巾蘸水擦去血渍,她又在他的腔子里涂了一层烈酒。张显宗仰卧在地上,看她像个小丫头似的从棉被里扯了大团的棉絮往自己腔子里塞,像在填她的布娃娃。他心里清楚,自己真的还是死了好;可是眼看着岳绮罗全神贯注的炮制着自己,他又感觉到了荣幸。为什么会爱岳绮罗?他说不清楚;为什么爱她爱到宁愿万劫不复?还是不清楚。他活了三十多岁,已经知道世上有好些事,永远都找不出前因后果。

``毕竟是自己的身体,好用。''岳绮罗在满室的腥臭中,轻描淡写的说道:``将来真是坏到用不得了,我会再给你找一具新的来。''

张显宗看她穿针引线,密密缝起了自己前胸后背的创口:``好,到时我要换个年轻好看的皮囊。''

岳绮罗眯起了疼痛的右眼,捏着钢针的手指翘成了一朵笨拙的兰花:``肤浅!''

她认为张显宗是个最平常不过的凡夫俗子,根本没有资格臭美。

门窗关得很严,房内的臭气并没有浓烈的扩散出去。天黑之后卫兵撤走了,张显宗拎着一只铁桶出
了门。

他把自己的脏腑埋在了丁宅后方的一棵老树下。幸好天暖了,土化了冻,让他可以很轻易的挖出深坑。将一桶柔软的物事稀里哗啦的倒进坑里,张显宗感觉自己是在梦游。没有偷袭,没有死亡,等到自己梦醒了,就又是新的一天。

各种感官都不敏锐了,寄居的感觉则是渐渐强烈。他拎着空桶往回走,腿不是自己的,然而听自己的话。一步一步迈出去,步伐僵硬得让他随时可能跌倒。铁桶一晃一晃磕打着他的膝盖,他不知道疼。

墙头露出了两双人眼睛,他也没留意到。及至他走远了,两双眼睛一起下降。两名军官佝偻着腰,战战兢兢的一起跳了下来。给他们充作垫脚石的勤务兵起了身,十分警惕的东张西望。

一名军官抱着胳膊,畏寒似的轻声问道:``你看见没?''

另一名军官是同样的姿势:``我看见了。''

午夜时分,墙头又起了动静。两名军官夹着小铁铲子翻墙过来,开挖树下的新土。

一个时辰过后,坑被原样填了上。两名军官直着眼睛翻墙出去,出去之后就站不住了,被勤务兵背着往远跑。腿软,舌头却硬,一句话也说不出来,顺着鼻孔往外呼冷气。都是跑过战场的人,人身上的零碎还能不认识吗?作为前旅长丁大头的亲随,他们不傻,心里有数。凭着参谋长的一身血,能下了马车直接走路?还一气走出老远?不对劲,肯定不对劲!

但是两人趴在勤务兵的背上,互相对了眼色,同时心有灵犀,统一把嘴闭了个死紧。

翌日上午,张显宗一身戎装,出现在了司令部内。

他的脸色很不好看,手上加了一根手杖,走起路来略有些摇晃。有人嗅到了异味,陪笑问道:``参座喝酒啦?''

张显宗神情木然的点了点头,颈骨一节一节的运动:``是,喝酒了。''

有人又问:``参谋长,您的身体没事吧?''

张显宗答道:``皮肉伤,无碍。''

他不肯示弱,因为江山不稳,所以在身体尚能支撑之时,他万万不敢露出破绽。忽然又很不想死了,因为他手里有权有兵。他想也许绮罗会有办法保住自己的肉身,也许自己在某一天清晨醒来,会真的重生。

在司令部里露过面后,他又回到了岳绮罗面前。现在他能很自如的调动口舌了,所以把昨日之事如实的讲述了一遍。

``开枪的人是个小媳妇。''他告诉岳绮罗:``顾玄武身边有个古怪的小白脸,先是替他挡了一枪,然后没事人似的冲上来夺我的枪。如果没有他捣乱,我也不会被个女人打中。''

岳绮罗一愣:``古怪的小白脸?是什么模样?''

张显宗下意识的摇头:``我没留意,只记得他是白脸,眼睛很大。''

岳绮罗又问:``你确定你一枪打中了他?''

张显宗答道:``我确定。''

岳绮罗双手攥成了小拳头,她没有确凿的证据,可认定了古怪的小白脸就是无心!她就知道无心不会死,可是死不死的又和她有什么关系?他又不爱她。

肯开枪去救无心的小媳妇,想必也就是月牙了。月牙抢了她爱的,杀了爱她的。她本来懒得和月牙一般见识,但是此刻,她想月牙真是欺人太甚。右眼一阵一阵的开始胀痛,她生气了。

顾大人离了唐各庄,来到了距离唐各庄约有二十里地的李各庄。条理分明的安顿好了,他调兵遣将,开始筹划报仇反扑。忙过一天之后,傍晚他进了临时征用的砖瓦房里,发现月牙正在心事重重的包饺子。

月牙死活也想不起自己是怎么开的枪了。她就只记得张显宗带着无心往墙上撞,撞得她脊梁骨跟着生疼。院子里没有帮手,谁也指望不上,于是她拎起枪跑了上去。枪很沉,沉得不像枪,像一块铁疙瘩,出乎了她的意料。枪都响过了,她还举着枪不放,心里怔怔的,只想着枪沉,沉死了。

顾大人知道她是受了惊,可是不知道怎么安慰她才好。转身进了东屋,他在炕上又看到了无心。无心的腰上被子弹穿了个挺整齐的孔洞。血是早就不流了,顾大人掀了他的衣裳细看,就见孔洞中堵着个粉红的肉瘤子,根据经验,肉瘤子大概会越长越大,最后把孔洞填满。无心不死,可是很容易害疼,此刻长长的趴在炕上,他连睁眼说话的精气神都没了。

大恩不言谢,何况是救命之恩。顾大人和他不耍嘴,只在他后背上拍了拍。一歪身在炕沿上坐下了,他心中生出了好奇:``我说师父,你有腰子吗?''

无心翻了他一眼,没说话。

顾大人继续追问:``心肝脾肺呢?''

未等无心回答,月牙端着一盆热气腾腾的煮饺子进来了。顾大人很有眼色的摆上炕桌,而无心就向后退到了角落里。月牙给他盛了一碗饺子放在枕边,让他趴在炕上慢慢的吃;自己则和顾大人隔着炕桌相对落座。吃着吃着,月牙感觉有手指头在戳自己的后腰,回头一看,是无心伸来了一只空碗。

顾大人清了清喉咙,开口说道:``月牙,别往心里去。你救你男人是天经地义,没什么可后悔的!''

月牙一边往碗里盛饺子,一边答道:``我没后悔,我就是心里不舒服。''

顾大人给自己剥了两瓣大蒜:``睡一觉就好了,别当回事!''

月牙低低的``嗯''了一声,转身把满满一碗饺子给了无心。窗外起了风,吹得窗棂直响。月牙不动声色的向外瞟了一眼,怀疑是张显宗的鬼魂来找自己算账。不过念头一转,她收回了目光,心想你要害我男人,我自然就要杀你。如果再有下次,我也还是一样。

正如顾大人所说,月牙枕着无心的手臂睡了一夜之后,仿佛就像过了心里一道坎似的,又恢复了往日的性情。盘腿坐在炕上,她手里总有针线活可做,做得太细致了,一个鞋底子让她纳了个没完没了。

如此过了三天,她终于做成了一只鞋。无心站在炕上穿了,来回走了几步,然后说道:``月牙,鞋小。''

隔着一层鞋面,月牙用手指摁着他的脚趾头:``不怕小,越穿越大。''

无心刚要说话,不料窗户上被人弹出``咚''的一声。顾大人的笑脸在窗外一晃,随即大踏步的转身走进了屋内:``嘿嘿,出了一件挺好的怪事!''

无心坐下来脱鞋:``什么怪事?还挺好?''

顾大人答道:``挺好,但是也挺吓人。''

无心知道他在等着自己发问,于是笑着看他,故意不问。顾大人沉默片刻,见无心和月牙串通一气,一起装哑巴,便忍不住开了口:``张显宗,不是被月牙一枪毙了吗?原来他没死,还活着。''

月牙听闻此言,心里倒是一轻松,因为卸下了一桩人命官司。无心则是不置可否,等着顾大人说下去。

顾大人洋洋得意的笑道:``虽然他没死,但是他带兵回去之后没过一两天,不知道是怎么回事,文县就闹起了内讧。具体详情我不清楚,反正现在老子不发一枪一弹,姑且坐山观狗斗。等到他们打疲了,恐怕不用老子出兵,他们自己就主动降了!哈哈哈!''

\chapter{夜色逼人}

张显宗穿着一身便装,搂着岳绮罗策马飞奔,沿着文县城外的土路向荒凉处疾行。马是军马,又有力量又通人性,跟他很久了,可是此刻跑得不安稳,总像是预备着要尥蹶子,甩下背上的两个人。

岳绮罗知道其中的原因,畜生的感觉往往会比人更敏锐,而张显宗已经被自己炮制成了非人非鬼的行尸走肉。军马怕了。

迎面即便是有夜风吹拂,腥臭气息也依旧缭绕不散。张显宗没有赶上好时候,如果把时间换到冬天,他不会这么快就被人看出破绽。天气一日热似一日,他可以遮住一切,唯独遮不住气味。流言仿佛瞬间就爆发起来了——当初丁大头做活死人的时候,已经引起了部下军官们的疑心;疑心存到如今,全发作在了他的身上。

自从掌握军权开始,他就成了某些老家伙的眼中钉。丁大头留下的队伍,凭什么就全归了他?即便他是个活人,也有被人谋杀的危险;何况他现在死了,更不会被宿敌们容留。军队在恐怖与疯狂的气氛中四分五裂,他成了所有人眼中的妖魔鬼怪。

丁宅被烧成了火海,房梁木架在火焰中哔哔啵啵的爆裂崩塌——他们要烧死他和岳绮罗,而岳绮罗本领再大,也还没到撒豆成兵的程度,也还不能同时抵抗成百上千的人马。

所以,他们得逃。

张显宗一手揽着怀中的岳绮罗,一手紧紧握了缰绳。手指黏腻的渗出了脓水,掌心的血肉蹭上了粗糙的缰绳。指尖已经磨出了白骨,他在温暖的春夜中疾驰而过,一边求生,一边腐烂。

最后,在一片无边无际的荒原上,张显宗勒住了马。

他翻身下马,又伸手抱下了岳绮罗。天是一匹漆黑的金丝绒,看起来博大而又柔软。银白的月光照耀了荒原上的一棵树,岳绮罗坐在树下,刘海乱七八糟的掠上去,露出了如玉的额头。

张显宗没有靠近她,只在不远处的一座小丘上坐了,坐在下风向,因为不想熏到她。侧耳倾听着她浅淡的呼吸声音,他忽然忍不住开了口:``绮罗\ldots{}\ldots{}''

他背对着岳绮罗,去问前方无尽的黑暗:``如果我没有死,如果我一直对你好,你会不会\ldots{}\ldots{}会不会对我有一点点爱?''

岳绮罗抬眼望向了他的背影,随即移开目光,清晰而沉重的冷笑了一声——你算个什么东西,也配和我谈爱?

笑很冷,心也很冷。一挺身站了起来,她走到了张显宗身后。弯腰一拍他的头顶,她开口说道:``趁着天黑,我们继续上路。''

张显宗现在已经类似了鬼魅,阳光会让他感觉很不舒服。

顾大人的指挥部一天换一个村庄,随着前线的推进而推进。此刻他距离文县只有四十里地。文县内的军队乱成了一锅粥,正在和他联络着要投降。投降当然是可以的,顾大人放心大胆的给了敌人时间,是战是降全随着他们的意思。降也接受,战也奉陪。

月牙跟着军队走,无论走到哪里,都是照样负责她的老活计。一天不把三顿饭做足了,她就感觉心里空落落的,仿佛失了身份和地位。无心已经换上了新鞋,她又预备着给顾大人也做一双。顾大人的大脚丫子很费鞋,无论是多么结实体面的好皮鞋,最后都能让他穿成两条又扁又长的臭咸鱼。所以月牙动了心思,想要在鞋面鞋底都多加几层,专为对付顾大人大铁锉似的脚后跟和长了牙似的脚趾头。

月牙费了死力气,天天纳鞋底纳得咬牙切齿。晚上屋里点了油灯,顾大人和无心坐在炕上玩纸牌,她不加入,恶狠狠的用大钢针往鞋底里戳,把线绳拉的嗤嗤直响:``给顾大人做一只鞋的工夫,够我给无心做一双了。''

无心的伤早好了,很快乐的攥着一把纸牌说道:``费你的闲劲!白天忙一天,晚上也不知道歇一歇。你不给他做,他还就光脚了不成?''

顾大人一纸牌抽上了他的脑袋:``没人味的东西!怎么着?你媳妇给我做鞋,你还不乐意了?''

月牙实在是累得手疼,又因为猜测明天恐怕又要搬家,所以爬到炕里打开包袱,把针线缠在鞋底上往包袱里放。包袱里没什么正经东西,只有几件衣物,以及两只小荷包。荷包里掖着黄符,当初是顾大人和无心戴过的,现在两个人都不戴了,被她一起卷进了衣物里。系好包袱放回原位,她伸腿下炕穿了鞋,出门进了院子。

院外站着两名东张西望的小卫兵,月牙看在眼里,感觉十分安全。院角用栅栏和碎砖围起了一个臭气熏天的小茅房,她走进去解了裤子蹲下来,捂着鼻子想要撒尿。然而刚刚哗哗哗的开了闸,她忽然生出了一种被窥视的感觉。茅房四处漏风,她猛然回头,却是并未看到异常。

手里攥着一小块草纸,她蹲在坑上定了定神,脊背还是毛毛的发寒。眼角余光忽然瞥到黑影闪过,她立刻通过一处缝隙向外望去,却是依然一无所获。

想到院外还有卫兵,她壮了胆子,嘀嘀咕咕的骂道:``臭不要脸的,头上长疮脚下流脓的缺德货,不怕瞎了你的狗眼,回家看你妈去!''

系好裤子走出茅房,外面的卫兵忽然起了喧哗,月牙赶去一瞧,却是两只野猫在墙头上飞檐走壁的打架,卫兵怕它们扰了旅座的清静,所以上蹿下跳的在撵猫。月牙松了口气,心想自己原来是把野猫给骂了。

她回到房内之时,顾大人和无心的牌局还在进行。她站在地上揉了揉小肚子,身上一阵一阵的冷,总像是没尿干净,还想再去一趟茅房。转身向门口迈了一步,她想起了茅房里似有似无的动静,又有些瘆得慌。

``无心啊。''她开口说道:``你跟我出去一趟呗。外面闹猫闹得怪吓人的,我有点害怕。''

无心正在全神贯注的看牌,听了她的话,才把目光从纸牌上移了开。抬眼向月牙一望,他看到了月牙身上依稀笼罩了一层带着微光的黑气。

不动声色的放下纸牌,他一边往炕下伸腿,一边开口说道:``野猫叫春是够难听的,我先出去瞧瞧。等我把猫全赶走了,你再出去。''

月牙答应一声,小肚子不舒服,说不清自己到底有尿没尿。等到无心披着一件小夹袄出门了,顾大人笑嘻嘻的伸手一掀他的纸牌,月牙见状,倒是暂时转移了注意力:``还带偷看的哪?''

顾大人竖起手指对她``嘘''了一声:``别吵,我就看一眼。''

无心一直认为身边环境挺干净,没想到月牙偶然摸黑出去了一趟,竟然就会被几缕零碎魂魄缠了上。零碎魂魄无知无识,等闲不会缠人,如今缠了,就必定有个缘故在里面。

他进院之后作势要打猫,弯腰从靠墙的地上捡起了一根粗木棍。一路若无其事的走出去,他发现魂魄的流动带了方向。有人在附近控制了它们,它们成了暗器。

无心忽然想起了文县的内讧,想起了下落不明的岳绮罗和张显宗。不知觉倒也罢了,既然对于他们的行踪有所知觉,就决不能轻易的放了他们。因为开枪打伤张显宗的人是月牙,而他们现在一无所有,想必会更加穷凶极恶。

春天正是闹猫的时节,无心一路上拆散了许多对野猫鸳鸯,看着是在打猫,其实是在沿着魂魄流动的方向走。忽然身边``嗤啦''一声响,他停下脚步低头看,发现是自己的衣裳被一丛低矮灌木刮破了一道。

他在黑暗中低头弯腰,费了不少的力气,才把挂在灌木尖上的衣角扯了下来。追着一群野猫又跑了几步路,他忽然发现魂魄光芒渐渐变得浅淡稀疏,方才的线索无端的中断了。

他停了脚步,因为一时摸不清头脑,所以拎着木棒向后转。不料未等他踏上归路,一个黑影忽然斜刺里急冲出来,带着雷霆之势猛撞向他,当场把他压在了地上。未等他反抗,黑影已经反剪了他的双手,力气极大,几乎扭断了他的关节。

他立刻就乖乖不动了,极力回头去瞧来人。朗朗月光之下,他看到了一张恐怖的人脸——眼眶鼻翼都糜烂成了黑红两色,一只眼珠凸出眼眶,另一只眼珠上面则是生了一层白霉。恶臭的气味从他七窍中飘散开来,他的喉结已经露出了白骨黑洞,他是张显宗!

一双布满尘泥的肮脏绣花鞋缓缓走近了,无心向上转动眼珠,仰视了岳绮罗的双眼。

岳绮罗看起来像一只肮脏的布娃娃,可是神色很平静。单单薄薄的伫立在夜幕下,她对着无心点了点头,嘴角忽然一抽搐,是似笑非笑,似哭非哭,百感交集,哭笑不得。

``张显宗。''她发出了声音,声音单调而又甜美,是一杯水,加了糖又加了冰:``砍下他的四肢!否则他很会跑,会让谁都捉不住他!''

张显宗当即腾出一只手,从腰间抽出了一柄军刀。而无心没有挣扎,只问:``你为什么要抓我?''

岳绮罗答道:``没人想要抓你,我只想要月牙的命。''

在张显宗挥起砍刀之前,无心抢着又道:``别砍,我们做个交易!''随即他奋力转向张显宗:``和你有关!''

岳绮罗一抬手,止住了张显宗的动作:``什么交易?''

无心的眼睛陷在了阴影中,心中的主意迅速有了雏形。为什么要杀月牙?因为月牙杀了张显宗。为什么要把张显宗制成行尸走肉,即便化成了一具腐尸还不抛弃?因为对于岳绮罗来讲,张显宗与众不同,很重要。

乌黑的眼珠在暗中转过一轮,无心开口说道:``你饶月牙一命,我会设法保住张显宗的身体!''

岳绮罗笑了一下:``身体,我要多少有多少。''

无心不再说话了,让她自己去想。她的确有无数办法去安顿张显宗的魂魄,可张显宗的躯壳是独一无二的,如果躯壳换了,他还完全是他吗?

况且操纵旁人的身体也并不容易,他的灵魂,天生就只适合他的身体。

无心不说话,张显宗也不说话。岳绮罗沉默半晌,开口又问:``你有什么办法?''

无心的半张面孔都陷在了泥土里:``我带你们去青云山。''

岳绮罗疑惑的看他:``青云山?''

无心放轻了声音:``青云山中有一处秘洞,可保尸身不腐。''

岳绮罗微微一点头:``我只知道前一阵子都在风传青云山里有怪物。''

无心答道:``不是怪物,是行尸走肉。洞里尸身不腐,灵魂不散,忽然受了军队的惊动,你知道会有什么后果。''

岳绮罗若有所思的俯视着他,想把他和张显宗合二为一,可是做不到。

非不为也,实不能也。

\chapter{圈套}

月牙和顾大人坐在房内等待无心,左等不回,右等不回,月牙就有点着急了,趴在窗前向外张望:``跑哪儿去了?是撵猫去了还是让猫撵了?''

顾大人攥着一把好牌,也是有些不耐烦。把纸牌往炕上一放,他穿鞋出去推门喊道:``师父!师父呢?''

院门口的卫兵做了应答:``报告旅座,师父拎着棍子出去了!''

顾大人转回屋披上一件夹袄,嘴里骂道:``真他妈没有正经,撵猫还能撵出失踪案子,我找找他去!外面黑漆漆的,他是不是跟谁家大姑娘小媳妇扯上皮了?''

月牙虽然不信无心能去和女人扯皮,不过话从顾大人嘴里一说出来,她听在耳中,就有点坐不住。紧赶慢赶的跟着顾大人进了院子,她和一名卫兵一起往外面幽深的巷道里走。卫兵提着个硕大的纸灯笼,把脚下地面倒是照了个通亮。

穿过几条巷子之后,顾大人一无所获,月牙扯着嗓子大叫无心,也是全然没有回应。三人眼看再走就要出村,只得悻悻的往回返。不料就在将要进门之际,顾大人忽然发现了问题。

他一把夺过卫兵的大灯笼,弯腰往地上细照。近来常落春雨,土地松软潮湿。他就见一条新鲜的深痕划在地上,从门口开始向外一路延伸。

他来了精神,沿着深痕转身前行,一路拐了几个弯,最后却是停在了一丛灌木之前。灌木下面扔着一根结实的木棒,而灌木上挂着一片灰色细布。村里人都用土布,士兵们又全穿军装,所以月牙低低的``呀''了一声,认出灰色细布是从无心的外衣上撕下来的。

顾大人也意识到了,扯下灰布展开一瞧,布上却又并无字迹。把布片递给月牙,他问:``是不是?''

月牙一摸布料就确定了,带着哭腔轻声答道:``是。顾大人,咋回事啊?''

顾大人摇了摇头,同时心里七上八下的安慰她道:``你别慌,别闹。反正知道他肯定是死不了,兴许是半路有了别的事,他来不及告诉我们,直接就跑去办事了。''

月牙知道无心是死不了,可是不能因为他死不了,就由着他平白无故的无影无踪。六神无主的随着顾大人回了房,顾大人先派出一队士兵出村搜查寻找无心,然后自己闷闷的收拾起了纸牌,也是一脸的困惑和不安。

凌晨时分,无心带着岳绮罗和张显宗进入了青云山地界。

张显宗握着手枪,枪口一直抵在无心的后背上。岳绮罗跟在一旁,路上始终也没有多说。待到脚下道路渐渐变得崎岖,张显宗拴好了马,然后摘下马灯交给了岳绮罗。岳绮罗察觉到自己是真往深山里走了,才开口问道:``你所说的秘洞,到底在什么地方?''

无心背对着她问道:``不叫我大哥了?''

岳绮罗望着他的背影,想象自己和他情投意合,手拉着手在山路上走。

毫不动情的想象过后,她自顾自的说道:``我不想晒太阳!''

无心忽然停了脚步,转身面向了她:``我们还没有达成协定。我送你们进洞,你们放过月牙。''

岳绮罗冷笑一声:``谁知道你的洞到底是真是假?入洞不腐,可是出了洞呢?你以为你是救了我们?''

无心答道:``你很聪明,入洞之后自己想办法吧。在你想出办法之前,他至少可以完好的等待。''

然后他向前继续走去:``洞里没有食物和水,你们自己准备。''

张显宗带着水壶和一包干粮,饮食虽然全被腐臭的气息浸染透了,但是聊胜于无。回头看了岳绮罗一眼,他心里很愧疚。他知道其实没有自己,岳绮罗也是一样的活。岳绮罗不给自己好脸色,可是为了自己,她风餐露宿的成了一只肮脏的小鬼,他都死了,她还保护他。

在一处小小的坟头前,无心停了脚步,开始蹲下刨土。最后搬开土下的铁板,他让岳绮罗来看:``入口。''

岳绮罗蹲在洞边,很谨慎的向下伸手。洞内的空气微微的有点暖湿,她俯身探头进去吸了几口气,空气也很干净。

她抬起了头,目光撩过了面前的无心。无心歪着脑袋,目光经过眼角射向洞内,黑眼珠在马灯的照耀下忽明忽暗,让人联想起了一只妖。

察觉到了岳绮罗的注视,无心一转眼珠望向了她,正色说道:``洞里很危险,到处都是厉鬼和行尸。上次进来的时候我只会逃,这次有了你,想必可以安全一点了。''

岳绮罗缓缓捏碎了一块黑土:``现在才知道我的本领了?''

无心一笑:``随便你有没有本领,我又不想拜师学艺。''

然后他继续问道:``我打头,谁殿后?中间的把灯拿稳了,里面可是伸手不见五指!''

不等旁人回答,他把外衣脱了,只留一身单衣单裤。动动肩膀扭扭脖子,他率先下洞去了。

张显宗殿了后,手里攥着枪,随时预备着保护前方的岳绮罗。岳绮罗手里的马灯散发着昏黄的光芒,正好可以照亮前后环境。无心虽然脱了外衣,可是因为长胳膊长腿,所以在洞里还是显得有些笨。他笨,张显宗就更笨了,倒是岳绮罗更灵活些。

三人爬了许久,最后终于到达了千佛洞。无心停在洞口不走了,回头告诉岳绮罗:``我第一次来时,洞里存了许多尸首,看模样还是前清时进去的。洞里本来被人设了阵法,把魂魄和尸首分了开;但是阵法被我冲破了,魂魄附在了尸首上,见了活物就杀。我和他都不怕,你怎么办?''

岳绮罗不屑一顾的把马灯交给了张显宗,然后大踏步的直接进入洞内。结果刚刚走出几步,迎面就直挺挺的来了一具活死人。在马灯的照耀下,活死人果然还保留着柔软的皮肤和浓密的头发,看辫子和身量,生前正是个壮年男人。不知道他是怎么死的,大概也是虐杀,因为他的嘴唇牙齿全都被挖去了,鼻子下方是个四四方方的血洞,洞口隐隐的还在收缩,仿佛是要吞噬什么。

无心立刻又停了,张显宗则是猛然举起了手枪。唯有岳绮罗面无表情。迎着活死人走上前去,她一边念念有词一边抬手虚空画符,最后对着前方一挥衣袖。活死人动作一顿,随即仰面朝天的倒了下去,却是附身的魂魄已被岳绮罗引了出来。

让张显宗拎过马灯,她蹲下仔细查看了尸首。看过之后,她承认无心所言非虚。

起身继续向前走去,沿途开始出现了断肢碎骨,以及零落肉块。肉块是灰白色的,无心记得曾经有一批怪物落入洞内,想必此地便是当时战场的一部分。怪物能吃人,行尸走肉们也不是吃素的,战况必定十分恐怖惨烈。

岳绮罗踢开了一只断手:``怎么回事?''

无心走在她的身边:``内讧。''

岳绮罗一笑:``它们也会内讧?''

无心摇了摇头:``不清楚。我只来过一次,是被它们一路打出去的,它们和张显宗不一样,它们都疯了。''

岳绮罗答道:``不要拿张显宗和他们打比。''

无心知道自己的话不好听,故意又问:``你为什么不找个人?''

岳绮罗立刻扭头望向了他:``什么意思?''

无心满不在乎的微笑:``找个活人,会老会死会喘气的。''

后脑勺一凉,是张显宗用枪口顶上了他的头皮。而无心混不在意,继续说道:``否则,你和他就只能做一对洞中夫妻了。''

岳绮罗听了他置身事外的悠闲语气,心中忽然腾起了一团怒火:``你讽刺我们会成洞中夫妻,就不害怕你和月牙会人鬼殊途吗?''

无心答道:``如果我和月牙人鬼殊途,我会让你永生永世不得自由、不见天日!''

岳绮罗真生气了:``笑话!''

前方又来了一具东倒西歪的骷髅,于是无心停了脚步:``希望是个笑话。''

岳绮罗气得连符都没有画,一言不发的向前方结了个手印。只听``哗啦''一声,骷髅直接散碎成了一堆骨头。

无心如愿以偿的把岳绮罗和张显宗全得罪了。带着二人拐了个弯,岳绮罗看到了甬道两边的怪异佛像。

她也不知道佛像的来历,甚至没有兴趣多看佛像一眼,因为无所畏惧。而无心站在一尊佛像身边,抬手向上一指:``我们得往上爬。上面还有一个洞,是一条直的捷径,可以通到尽头,并且还有灯油。''

洞内漆黑,灯油自然是十分重要。岳绮罗完全不怕无心作乱。洞内尽是流窜的魂魄,让她几乎感到了亲切。

况且,还有张显宗。

洞顶满是嶙峋怪石,深深浅浅的阴影之中,的确可见一个向上的入口。无心踩着佛像向上攀爬,他上去了,岳绮罗也上去了,最后是张显宗。三人带着一盏马灯站住了,遥望四周,却是只见一片无边无际的黑暗。

岳绮罗从未见过如此的情境,不禁出了声音:``怎么——''

无心未等她把话问完,迈开大步向前就走。岳绮罗连忙带着张显宗跟上了他,依稀就感觉他是转了几个弯。鼻端忽然嗅到一丝隐隐的腥气,岳绮罗开口问道:``你很认路?''

无心答道:``来过一次,多少记得。''然后他一伸手:``马灯给我。''

没人问他要马灯干什么,因为此地处处需要光亮。张显宗把马灯给了他,就见他仿佛是有所迟疑似的,试探着向后退了一步。

随即就听``哗啦''一声脆响。玻璃灯罩狠狠撞击到石头地上,马灯瞬间熄灭,三人陷入了永夜般的黑暗之中。张显宗下意识的觅着声音扑了上去,然而一阵疾风掠过他的指尖,他扑了个空!

\chapter{道长又好怕}

无心沿着来路向回疾奔。四周窸窸窣窣的响动越来越近了,遮盖住了他的脚步声音。马灯被他摔成粉碎,三个人都陷入了无边的黑暗之中,岳绮罗和张显宗看不见他,他却是能够感知到周遭一切的动静。一颗子弹掠过了他的头皮,是张显宗对着虚空开了枪。

脚下忽然一空,他在坠落时蜷了身体,走兽一样无声无息的四脚落地。一大步越过横在地上的零碎残尸,他头也不回的向洞口冲去。来时的笨拙消失了,他攀爬跳跃着跑过石地钻入土洞。身上衣裤单薄的可以忽略不计,他闭了眼睛趴伏下去,像一条长蛇一样向前游动。洞壁的每一处起伏曲折仿佛都在他的掌握之内,他的手肘膝盖全都灵活到了不可思议的程度。爬出一条土洞,再入一条土洞。他长条条的身体几乎就是在土壤里钻。忽然一个打挺仰起了头,他睁眼看到了明烈的阳光!

爬出土洞上了地面,他先把铁板搬回原位,又把泥土重新铺好。起身在附近转了一圈,他咬牙切齿的搬来一块大石,压在了乱七八糟的假坟上面。

靠着大石头坐下来,他感觉到了疲惫。赶了一夜的长路,又钻了许久的洞子,他实在是饥渴到了力不能支的地步。现在让他直接回家,是不可能的了,他捡起入洞时丢在外面的外衣,也没有穿,单是一甩搭上肩膀,然后勉为其难的做了个起立,决定先去青云观要顿饭吃。

在青云观的山门外面,无心迎面遇到了出尘子。不过他很识相的退到一旁,并没有贸然呼唤,因为出尘子头戴纯阳巾,身穿青道袍,在一大群华服弟子的簇拥下,正在飘飘然的送客人。客人只有一位,生得金发碧眼高鼻梁,却是西洋人士。无心站在石板路外的荒野地里,就见出尘子和西洋人你一言我一语,是个谈笑风生的模样。一位教书先生似的青年跟在一旁,显然就是中间的通译了。

出尘子出了山门之后,也看到了无心,不过无暇理他。而无心饶有兴味的微笑旁观,只见他天仙下凡似的伸出一只手,和西洋人行了个轻描淡写的握手礼,同时施施然的说道:``三克油喂你妈吃,古德拜密斯特劳伦斯。''

等到西洋人和通译一起坐上轿子下山去了,无心才土猴似的凑到了出尘子面前:``道长,几天不见,你和西洋人交上朋友了?''

出尘子微微一笑:``一个英国记者而已,慕我青云观的名声而来。至于朋友二字,哈哈,贫道素来豁达,四海之内皆兄弟也\ldots{}\ldots{}''

未等陶醉完毕,出尘子忽然意识到了自己身边的人乃是无心,便立刻把笑容一收:``你来干什么?''

无心心平气和的答道:``我饿了,又正好路过青云观,所以想来吃顿饭。''

出尘子立刻松了一口气:``哦\ldots{}\ldots{}''

``哦''过一声之后,他在心中暗叹:``福生无量天尊。原来只是吃饭,让本道爷白白吓了一跳!''

无心进了出尘子的房间,有汤有水的吃了顿饱饭。而出尘子解下头巾散开长发,沾沾自喜的回味着西洋人对自己的恭维。正是得意之际,他忽听无心问道:``道长,令先师的秘笈,你研究的如何了?''

出尘子还沉浸在喜悦中不能自拔,梦呓似的轻声答道:``成绩是有了一点,不过,一点而已。''

无心看了他的德行,忽然对他的本领十分怀疑。低头又往嘴里扒了一口米饭,他察言观色的瞄着出尘子,同时说道:``道长,岳绮罗此刻正在千佛洞里。''

出尘子猛然扭头望向了他:``什么?''

无心垂下眼帘,夹了一筷子菜放进碗里,似乎是有话难说:``她\ldots{}\ldots{}她昨夜找我的麻烦,我没办法,只好把她骗进了千佛洞。''

出尘子抬手一拢下垂的额发:``现在呢?''

无心答道:``现在她要么是被怪物吃了,要么是在四处寻找出口。所以我想请道长效仿令太师祖,画一道符把洞封住。''

出尘子脸色都变了:``你是说,她如今距离我青云观,不过几里地远?''

无心一点头:``道长有秘笈在手,画一道符,应该不为难吧?''

出尘子一拍桌子,瞪着眼睛骂道:``不为难个屁!你当贫道是天生奇才,一学就会一点就通吗?直告诉你吧,秘笈我根本就没看懂,再给我一个月,我兴许能把太师祖留下的符咒补全!''

然后他站起了身,一推窗子向外喊道:``长风皓月宇清!''

三个白白净净的小道士应声进了院子,而出尘子一口气下了一串命令:``长风快去让人预备轿子汽车,皓月收拾行李,宇清去找你师父,就说师祖要去一趟天津,观内事务由他管理!''

三个小道士立刻跑了两个,剩下一个问道:``师祖,您是长住还是短住啊?薄衣裳带不带?''

出尘子不耐烦的一挥手:``带,全带!''随即他转身回到墙边书架前,抽出一本薄薄的册子往怀里一揣,无心看得清楚,正是``秘笈''。

他没想到自己一句话,把出尘子吓得炸了庙。放下碗筷起了立,他试探着问道:``道长,你要出门?''

出尘子动作极快的打开两本书,将夹在其中的存折抽出来一并揣到身上:``一想到岳绮罗就在山中,并且与道观只有咫尺之遥,贫道便不由得有些惶恐。横竖做学问在哪里都是一样,我决定去天津或者北京用功一个月,必要把符咒补完方罢!''

然后他一甩大袖子,转身就往外走,走到门口时一回头,他匆匆的又道:``我的大弟子认识你,你想住可以住下,不必客气拘礼。''

随即门口青影一翻,出尘子龙行虎步,鼓着风跑了。

无心早知道出尘子有点不靠谱,没想到几日不见,竟然不靠谱如斯。回到桌前端起饭碗,他心事重重的吃了余下小半碗饭。

吃饱之后,他叹了口气,心想接下来怎么办呢?

无心又回了一趟山中,发现入口的巨石泥土都无变化,显见洞口一直是个封锁的状态。他不能入洞去确认岳绮罗的死活,可是不确认的话,又不放心。围着大石头左转右转,他认为自己上午逃得十分成功,堪称毫无痕迹。如果再次进洞,也许反倒会弄巧成拙。

无心思来想去的守着大石头,足足耗了半天光阴,末了自己一拍脑袋,心想出尘子他太师祖的道术再高,也只镇压了她一百多年。可见世上从来没有万全之策,如今岳绮罗在千佛洞内凶多吉少、九死一生,也就可以算是自己成功了。自己一味的悬着心不回家,也是无用。

思及至此,他起身拍了拍身上的尘土,穿上外衣下山去了。

无心走了小半天才到家,然而家里就只有几名士兵留守。原来顾旅今早开拔,往文县去了。

士兵赶着马车带上无心去追大部队,结果一直追进了文县城里。县里的守军彻底投降了,顾大人在外面流浪了小一年,如今终于耀武扬威的杀了回来。

在一处空空荡荡的大瓦房里,无心见到了月牙。月牙惶惶然的翻着上嘴唇,见到他后只说了一句:``哎呀妈呀,可回来了。''

无心探头对她左看右看:``嘴怎么了?''

月牙像脱了力似的,一屁股坐在了椅子上:``你跑哪儿去了?你不知道家里人惦记你啊?''

无心看清楚了,发现一夜不见,月牙的上嘴唇左右各鼓出一只大火泡。

月牙一宿没睡,如今总算把他盼回来了,也不多说,先出门买了几个热烧饼,就着凉开水吃进肚里。肚子里有了热烧饼垫底,她恢复了精气神,正好顾大人也意气风发的回了来。两人一起振作了精神,开始此起彼伏的数落无心,说他人如其名,真没长心,不怕家里人急出病来,又问他一天一夜死到哪里去了,未等他作答,两人又统一的表示想要揍他一顿。

无心知道自己在月牙和顾大人心中有分量,可是没想到分量如此之重。他把眼睛睁大了,彻底露出了圆溜溜的黑眼珠。抱着膝盖蹲在床上,他像一只慌乱而又俊俏的猴子,看看月牙,再看看顾大人,最后才讲述了自己一夜一日的所作所为。

及至话音落下,他得到了热烈的赞美和抚慰。他并没感觉自己受了冤枉,可月牙和顾大人全都愧疚了。顾大人扯了他的手臂往下拽:``走,咱们下馆子去吃顿好的!月牙,你给他换身干净衣裳!''

没等无心说话,月牙托着一条热毛巾过来了,挖耳朵拧鼻子的给他擦脸:``好,往后再也没人给咱们捣乱了。''

无心受宠若惊的仰头任她擦着,没想到自己能让两个人一起欢欣鼓舞。眼睛瞟着月牙的上嘴唇,他开口问道:``你们以为我去哪里了?''

月牙对着顾大人一撇嘴:``他说你摸寡妇炕上去了。''

顾大人一摊双手:``我不是想逗你玩吗?要不然你唉声叹气的不睡觉,我不管你?''

然后他用手指一指无心的鼻尖:``全怪你。''

月牙很心疼的在无心的脑袋上的摸了一把,顺势打开了顾大人的手。三个人里数她的年纪最小,可是不知怎的,她活成了无心和顾大人的老姐姐。顾大人被她打了一下,笑嘻嘻的毫不在意。而无心正要依偎向她,不料她转身去洗毛巾;无心靠了个空,当即歪倒跌了个四脚朝天。

\chapter{生活与前途}

顾大人带着无心和月牙住进了他当初的司令部。司令部本来就是一处民宅,曾在炮火中受过损毁,修缮之后始终是不及先前体面。但顾大人报仇似的非住此地不可,因为他当初就是从司令部里逃出去的。

按照计划,他至少得在文县耽搁一个月,一个月后看情形,如果长安县里的军头不识时务,他就带兵一路杀过去。而在等待期间,他无所事事,终日花天酒地的消磨光阴。无心和月牙则是关起门来过日子,月牙从来不生病,如今一股火全发在火泡上了,天天翻着上嘴唇操持家计,性情倒是安静了许多,因为嘴唇疼痛,不便唠叨。

顾大人连着玩了五六天,最后在一个阳光明媚的午后,他回了司令部。推开院门往里一走,他就见月牙和无心坐在树荫下,正在摆弄一地的烟叶子。烟叶子是顾大人带回来的,沉甸甸的一大捆,是来自吉林的上等关东烟。顾大人对一切东西都不上心,随手把烟叶子往上房一扔,从此就不再管;月牙看不下去了,趁着天晴把烟叶子拎出来,一片一片的摊开了晒。听见院门有了响动,两个人一起扭头来看。而顾大人扶着门框站住了,就见月牙把头发挽成了个勉勉强强的小圆髻;几缕弯曲碎发垂在鬓边,眼睛水汪汪,脸蛋红扑扑;无心则是带了一点傻相,微微张开了棱角分明的嘴唇,像是被顾大人吓了一跳。

顾大人笑了,感觉小夫妻两个很般配,都是漂亮人。和去年此时相比,月牙显然是胖了,也长开了,褪了青黄不接的丫头相,成了个很饱满的小娘们儿。揉着肚子慢慢走上前去,他开口问道:``晒烟呢?''

月牙嘴唇上的火泡已经干瘪了,结出了一片厚厚的血痂:``再不晒就要长白毛了!好好的烟叶子,就让它在屋里潮着?''

顾大人悻悻的打了个哈欠,转移话题诉苦道:``我肚子疼。''

无心握着一把剪刀,正在月牙的指挥下剪笸箩里的碎烟叶子,一边剪一边问道:``吃坏了?''

顾大人摇了摇头:``应该是夜里凉着了。''

月牙嗤笑了一声。顾大人连着好几夜都没在家里住,自然是跑去了窑子里落脚。而月牙作为一个颇硬气的小媳妇,对顾大人的行径是相当的不赞同。利利落落的把烟叶子全翻了个身,她开口说道:``你也三十来岁了,就不能正正经经成个家?你跟你媳妇睡觉,你媳妇准保不能让你凉着!''

无心立刻点头附和:``没错,月牙天天夜里给我盖被。''

顾大人捂着肚子说道:``我不是得挑个好的吗?告诉你们,凭我现在的身份,我要娶就娶个大家闺秀!''

月牙低头说道:``你可饶了大家闺秀吧!吃饭打嗝睡觉放屁,臭脚丫子熏死蚊子,大家闺秀能跟你过到一起去?''

话音落下,无心很及时的笑了一声。笑声未落,他被顾大人打了一巴掌。顾大人讲理讲不过月牙,于是转移方向开火:``笑个屁呀!''

月牙又道:``肚子疼也没事,往肚脐眼里抹点烟油子就好了。''

无心和月牙都没有抽烟的瘾,倒是顾大人除了烟卷之外,偶尔也抽两口小烟袋。顾大人在艳阳之下撩起上衣鼓起肚皮,而无心找来小烟袋,抠出烟油涂向了他的肚脐。顾大人是结结实实的精壮身材,腹部硬邦邦的能显出一块块腱子肉,从肚脐眼往下生出一溜浓重汗毛,打着卷儿根根见肉,一直延伸到松松的裤腰里去。月牙看惯了无心,如今偶然向顾大人撩了一眼,便不由得心中暗笑,认为顾大人皮糙毛重,像头野猪。

无心给顾大人涂过烟油之后,坐回了小板凳上,继续闭着眼睛剪烟叶。月牙往树影下挪了挪,刚想呼唤无心也过来,可是抬眼一瞧,就见阳光透过枝叶,撒了他一头一脸的深浅光斑。他心不在焉的一下一下合着剪子,脸上神情静谧极了。

月牙看出了神,直到顾大人扛着一把大躺椅走了过来。把椅子往树下一放,他一屁股坐下去,随即也留意到了无心。伸手轻轻推了月牙一下,他露出个坏笑,弯腰脱了脚上一只皮鞋,随即把鞋缓缓的凑向了无心的鼻端。

无心什么都知道,可是装成不知道的样子,想让顾大人阴谋得逞。阴谋得逞了,顾大人很得意,会笑;月牙看了个小热闹,也会笑。

皮鞋越凑越近了,他忍无可忍的睁开眼睛猛然一躲,同时露出了受惊吓的表情。顾大人果然哈哈大笑了,月牙也笑道:``傻东西,困了就回屋睡去!要不然顾大人还得撩你。''

无心和烟叶一起晒着太阳,的确是生出了睡意,不过留恋着不肯离开。而顾大人从他面前的小笸箩里捏了一撮烟叶塞进小烟袋锅中,点燃之后吸了一口,随即很销魂的长叹一声:``真是好烟。''

月牙起身从房里取出一只布口袋,让无心把笸箩里的碎烟叶子往口袋里倒:``我们要是不把它收拾出来,你也不把它当好烟。抽吧,够你抽一年的了。''

无心欠身伸手,挑了几片干燥的烟叶子,握着剪刀想要继续将其剪碎。月牙夺了他的剪刀:``不剪了,累手。''

顾大人在窑子里混了几天,混到如今回了来,不知怎的,和无心月牙会特别的亲。大下午的,人家小夫妻两个上炕睡午觉,他也跟着上炕了。房内弥漫着一股子香甜辛辣的烟叶子味,无心躺在中间,侧身面对着月牙;顾大人躺在他的身后,当仁不让的占据了大半铺炕,并且把呼噜打得震天响。

月牙被顾大人吵得睡不着,扯了无心的一只手仔细看。无心握久了剪刀,手指上硌出了一道道红痕。月牙轻轻揉搓着他的手指,心想出了文县再走几十里地,就到平镇了。自己的娘家就在平镇,跑出来了小一年,不知道家里成了什么样子。要说回去瞧上一眼,其实也行。私奔的姑娘只要嫁得好,回家也是有脸的。当然,自己的家真是不值一回,虽然还有个亲爹,但是把大姑娘卖给债主老头子当小妾的行径,一般的后爹都做不出。

月牙思来想去的,不知该不该回娘家。翻身面对了熟睡着的无心,她看了又看,最后从鼻子里呼出了一口气——算了,不回去了。家里人多眼杂,又没有善意,犯不上让他们对无心品头论足。

傍晚时分,月牙系着围裙在厨房里煎炒烹炸;无心一趟一趟的把烟叶子运回房内,然后独自守着个小笸箩把烟叶剪碎。人人都不闲着,唯有顾大人像个大爷似的躺在炕上。枕着双手仰面朝天,他翘起了二郎腿,咂着嘴喊道:``月牙,给我倒杯水!''

厨房里的生菜刚下了锅,``嗤啦''一声响中,月牙依稀答了一句,也不知道答的是什么。顾大人口干舌燥的等了半天,屁也没有等来一个,于是又开了口:``师父,给我倒杯水,我都睡渴了。''

无心没说什么,起身去将一杯冷茶端到炕边。顾大人晕头转向的坐起来,喝过茶后又道:``你把烟袋拿过来,我抽袋烟提提精神。''

无心把茶杯放回原位,果然又找出了烟袋。填好烟叶子点着了火,他坐在炕头靠着墙,自己吸了一口。顾大人看他喷云吐雾的挺舒服,不由得盘起双腿一拍膝盖:``哎,是我要抽烟,不是让你抽。''

无心躲在烟雾后面,理直气壮的答道:``可我也没说要伺候你啊!''

顾大人一晃脑袋:``那现在我也想抽,怎么办?''

无心挥了挥手:``你屋里有烟卷,自己拿去!''

顾大人抬手一指他:``老不死的,我看出来了,你就能对女的使劲。在月牙跟前你贱的没边,恨不得摇着尾巴给人家舔屁股;我支使你干点活,你就跟我装大尾巴狼。''

无心守着一笸箩碎烟叶,抽完一袋再装一袋。深以为然的点了点头,他对顾大人笑道:``是,我的确是这样的人。''

顾大人四脚着地的爬过去,一把夺过了小烟袋:``重色轻友,什么玩意!''

无心听了这句评语,却是很高兴的笑了:``重色轻友?''

顾大人吸了一口烟,莫名其妙的看着他:``你美什么啊?以为我夸你呢?''

无心心花怒放的下了炕头。重色轻友,说明他有色可以重,也有友可以轻。这四个字让他越品味越愉快,于是他幸福得坐不住,决定去厨房给月牙打下手。顾大人叼着烟袋怔怔看他,没想到自己把他损了一顿,他反倒欢天喜地的活泼了。

吃过晚饭后,顾大人出门去军部转了一圈,回家后发现无心和月牙坐在炕上,又剪起了烟叶子。房内电灯通亮,月牙嘴里嚼着柿饼,无心则是呆呆的望着摊在炕上的一本薄册子。

顾大人凑过去一瞧,发现册子上印的是风水学问。月牙说道:``看书呢,天天晚上看半天,说是以后要改行给人看风水。''

无心显然看得十分乏味,一双眼睛半睁着望向书页,半晌不眨一下。顾大人嗤之以鼻:``扯鸡·巴·蛋!等我把仗打完了,直接给他安排个差事不就行了?''

月牙笑道:``拉倒吧,你说他能干啥?你让他写写算算还是打打杀杀?''说完她伸腿一蹬无心:``不爱看就算了,一晚上都没见你翻过一页!''

无心伸手把书一合:``没意思,是不爱看。''

顾大人伸手去扳他的肩膀:``给我当个副官怎么样?''

无心打了个小小的哈欠:``当副官不就是伺候人吗?我不愿意伺候男人。''

月牙当即又蹬了他一脚:``你想伺候哪个女的?''

无心眯着眼睛对她一笑:``你。''

顾大人拍了拍无心的脑袋:``别骚了,我知道你的意思。你是不是怕和人走近了,被人看出问题?''

无心一低头:``对。''

顾大人张开蒲扇似的大巴掌,罩在无心的头顶捏了捏,然后扭头对着月牙说道:``由着他吧!反正你俩花销有限,就算他什么都不干,我白养着你们也养得起。''

月牙并不想吃顾大人的白饭,所以思索着说道:``要不然,种地也行。原来在老家的时候,我家除了开油坊之外,也种好几亩地呢。种庄稼嘛,肯下力气就有粮食收,不比他和鬼鬼神神打交道强?''

无心比较懒,既不愿意伺候人,也不想在土地上卖苦力。所以抬手揉了揉眼睛,他把他的风水册子又翻开了。

\chapter{猪嘴镇}

因为文县实在是天下太平,周边地区也无战事,于是月牙想要去一趟猪嘴镇。当初无心从顾大人手里要来一千大洋,租房子过日子花了一些,还剩好几百,被她装进瓦罐埋在了地下,本来算作是家中的宝藏,非到紧要关头不肯取用的,然而后来遇了变故,三人离开猪嘴镇后就再没回去过。如无意外的话,她想,瓦罐应该还在地下。

几百大洋的财产,放在哪里都不是小数目,而猪嘴镇又不偏僻,即便是步行前往也不算远。顾大人在文县住腻了,听说月牙和无心要去猪嘴镇,他欣然同意,并且亲自带了一队士兵,要给他俩做保镖。

顾大人重走去年的逃亡之路,心中别有一番得意。沾沾自喜的骑在高头大马上,他沿途伸手指指点点:``看见前面的路口没有?我当时要是在那里拐了弯,就到不了猪嘴镇,也见不着你们了!''

无心和月牙合乘了一匹马。听闻此言,无心开口说道:``有缘千里来相会。''

顾大人一点头:``没错,咱们是有点缘分。阴差阳错的见了一次又一次。''

月牙靠在无心怀里,看着路边的野花迎风摇曳。碧蓝色的天空下,一只金黄蜂子掠过她的鼻尖。把手轻轻搭在无心握着缰绳的手背上,她笑道:``挺好,往后你俩也别生分。''

顾大人立刻笑了:``放心,我和他打不起来。''然后他看了无心一眼,继续说道:``真打起来也没事,他打不过我,我打不死他。''

马走得慢,无心坐烦了,自作主张的飞身下马,把月牙和顾大人全吓了一跳。顾大人正要大骂,不料月牙像个小晚娘似的,凶巴巴的先发了吼声:``干啥去?''

无心仰脸对着月牙微笑:``我给你牵马。''

无心说要给月牙牵马,其实牵着牵着就松了手。蹲在路边采了一大把迎春花,他走回月牙身边,把花插在了马辔头上。月牙一直追逐着他的身影,看不够似的看。而他牵着缰绳向前行走,仿佛是察觉到了她的目光,忽然回头一笑。

春日明烈的阳光照耀了他的头脸,他笑出了一口很好看的雪白牙齿,看起来有种天真无邪的动人。月牙也跟着笑了,一边笑,一边把他深深的印进眼中、刻到心里。她想:``他多好啊!''

无心心满意足的扭开了脸,伸手又要去拉顾大人的缰绳。顾大人立刻一挥手:``去,我不用你给我牵马!''

月牙也俯身打了他一巴掌:``你就不能上来歇歇你的狗腿?在家里顶数你最懒,出来倒勤快了!你看你摘的这些花,招来多少蜜蜂?你趁早给我上来,要不然我和顾大人走了,没人管你!''

无心乖乖上了马,感觉月牙和顾大人都没什么情趣。

一行人到了猪嘴镇,先前租住过的房子还锁着大门,显然里面没来新房客。月牙贴着宅院的后墙根往下挖,从深处挖出一只破瓦罐。瓦罐沉甸甸的,里面正是大洋。

虽然大洋是月牙当初亲手埋下去的,不过半年之后挖掘出来,总像是失而复得,十分庆幸。三人到镇子中心的饭馆里去吃了顿迟来的午饭,本打算吃饱喝足之后就回文县,不料菜未上完,外面却是阴了天。顾大人走到雅间窗前向外一望:``哎哟,是不是要下雨啊?''

无心和月牙也不确定,三人正要看天说话,雨丝飘下来了。

顾大人回到县里也没急事,所以索性坐稳当了,慢悠悠的连吃带喝,顺便等着雨停。然而春雨下得绵长,天色也是越来越暗。

月牙坐得久了,又吃得腹中饱胀,就想起身活动活动。饭馆是大馆子,上下两层楼。她一挑帘子出了二楼雅间,沿着满地油污的长廊往楼梯走。走着走着,她忽然直着眼睛停了脚步。

抬手捂住胸脯,她张了张嘴,随即``嘎''的打了个饱嗝。此嗝十分响亮,月牙虽然不是文雅仕女,可也比不得顾大人的粗豪。闭嘴之后红了脸,她向左右瞟出两眼,就见今日楼上客人不多,雅间之内都很安静,想必无人领略自己的饱嗝,便加快脚步,做贼心虚的赶紧离去了。

与此同时,她身后的雅间门帘倏忽一动,一双惨白的小手将伸未伸,无声的停顿在了半空中。

月牙到了楼下,见顾大人的小兵们围了一张大圆桌,正在欢天喜地的连吃带喝。二十来岁的青年人,肠胃全是无底洞,而且又有长官付账,所以一个个狼吞虎咽,不住的让伙计加菜。月牙走到门口往外看,就见街上湿漉漉的,空气经了小雨的洗涤,像是更透明了。

门口的柜台后面坐着年轻的老板娘,是个非常伶俐的小媳妇,见月牙站着望天,就很亲热的向她搭话,且把柜台上的一盘椒盐花生推过去,要和她边吃边聊。月牙难得能遇上个同龄的女伴,又知道顾大人必在楼上谈论他的军政大事,十分无聊,就守着柜台和老板娘唠了许久。后来她约莫着时间差不多了,便向老板娘告了辞,准备上楼回雅间去。

椒盐花生是老板娘亲自炒的,里面加了几根小红辣椒。月牙一边咀嚼一边上楼,嚼着嚼着就感觉嗓子里不痛快,仿佛是被干辣椒皮呛着了。抬手扶了墙,她一路咳嗽着往上走,及至进了二楼走廊,她面红耳赤,鼻涕眼泪全流出来了。停下脚步清了半天的喉咙,直到感觉嗓子里不再火烧火燎的难过了,她才继续迈步往前。走着走着,她忽然又停了脚步。

走廊狭长,只在尽头有两桌客人,在雅间里面偶尔发出谈笑之声。月牙无缘无故的打了个冷战,一只手依旧扶着墙,另一只手则是伸进了衣兜里摸摸索索。似乎是有阴寒气流拂过了她的后颈,油污的雅间门帘无声的动了,惨白的小手又缓缓的伸了出来。阴暗之中,小手稚气未脱,手背上凝结了鲜红的血痂,光秃秃的指甲破烂肮脏。

这时,月牙的手从衣兜里抽出来了,手中多了一条薄如蝉翼的破旧手帕。

手帕被她捂上了鼻子,在小手将要触及到她的发髻之时,她猛一低头,惊天动地的擤了一把鼻涕。随即手帕被她向后一掷,正好打在了小手上。

小手一惊,登时停在半路。而月牙抬起头继续迈步,低声自言自语道:``哎呀妈呀,难受死了。''

月牙刚回雅间,就听窗外楼下一阵喧哗。片刻之后门帘一挑,一个胖子挤入雅间,却是本镇的镇长。镇长和顾大人有点拐弯抹角的亲戚关系,论交情是非常的浅薄,几乎等同于无。但顾大人东山再起,不但攀附了老帅,而且占领了文县,导致镇长重打算盘,决定和顾大人再叙一叙旧。听闻顾大人驾临猪嘴镇了,镇长慌忙赶来,生怕自己步伐迟缓,会放走一位好亲戚。

既然把顾大人堵在雅间里了,镇长谈笑风生,就绝不肯再让他轻易的走;亲戚辈分也全论起来了,口口声声都是你嫂子如何如何,你侄子如何如何。顾大人含笑听着,态度是不冷不热;听到最后,他接受了镇长的邀请,决定到镇长的官邸中住上一夜,因为雨水不停,道路必定十分泥泞。几十里路走下来,可是让人有点受不了。

镇长作为本镇首富,拥有一套格局混乱的大宅院,安置着他的太太小妾以及众多儿女。顾大人进了客厅和镇长闲话,镇长见他对无心和月牙十分关怀,便腾出一间上好的房屋,请他们进去安歇。

房屋可能是位姨太太的卧室,里面收拾得花红柳绿挺热闹,并且带着一股子隐隐约约的脂粉香。月牙捧着一杯热茶坐下了,有点不自在:``今天就住在这儿了?''

无心答道:``管它呢。住就住,正好让你少做几顿饭,也清闲一天。''

月牙笑着看他,怎么看怎么好,恨不得咬他一口。

入夜之后,无心和月牙早早上床,缩在热被窝里嘁嘁喳喳的说话。顾大人却是和镇长坐在前厅,觥筹交错的痛饮不止。顾大人喝高兴了,嘻嘻哈哈的开出许多空头支票;而镇长本来和他不熟,不大了解他的性情,所以此刻也听不出他言语的真假。糊里糊涂的闹过一场之后,镇长离席撒尿,换了镇长的小姨太太上场,娇声嫩气的要和顾大人划拳。

小姨太太颇有姿色,顾大人也是器宇轩昂,两人划得眉来眼去,不知不觉就过了许久。最后还是顾大人先有了知觉:``我大哥怎么还不回来?''

小姨太太不甚情愿的打发了身边仆人去找镇长。结果半晌之后仆人回了来,却是答道:``老爷在院子里摔了一跤,摔得腿疼,刚被人扶回您的房里去了。''

小姨太太立刻一拍桌子:``真是的,兄弟还坐在这里呢,他怎么说走就走,连个屁都不放?''

镇长素来是个一团和气的性格,面对小姨太太就更是和蔼之至。仆人知道小姨太太比镇长厉害得多,所以不敢多说,只是陪笑。

镇长走就走了,小姨太太兴致高昂,还要和顾大人继续喝酒划拳。倒是顾大人认为小姨太太虽然眉目姣好,但也谈不上如何美艳,可勾搭可不勾搭;而且按照亲戚辈分来论,镇长毕竟算是自己的大哥,自己犯不上和大哥的姨太太狗扯羊皮。笑嘻嘻的搪塞几句,他推辞酒醉,也离席了。

小姨太太十分扫兴,气冲冲的回了房,迎面就见床帐低垂,帐下垂着一只粗腿。重手重脚的关上房门,她坐在梳妆台前,一边卸妆一边抱怨:``你好大一个镇长,一点礼数都不讲。我要是不派人去找,人家顾旅长还得继续等你呢!摔跤是摔了你的腿,又不是摔了你的嘴,你连支使丫头通报一声的力气都没有了?''

把一只发卡丢到梳妆台上,小姨太太对着面前的大圆镜一撅嘴,正要继续埋怨。不料就在将要开口之时,她忽然愣了一下。

通过大圆镜子,她看到自己的床帐微微有了波动;而自己那胖墩墩的镇长夫君,无声无息的从帐子后面露出了一只眼睛。

\chapter{疑局}

顾大人擦了脸漱了口洗了脚,自我感觉十分卫生。舒舒服服的钻进被窝,他很惬意的伸直了双腿,同时就听隔壁传来低低的说笑声音,是无心和月牙还没有睡。

被窝微凉,顾大人打了个撕心裂肺的大哈欠,忽然认为月牙说的也是有理——应该讨个正经八百的太太了,娶妻娶德、娶妾娶色,不太美也可以,但是一定得要好人家的姑娘。自己在当家立计的方面已经是不高明,再弄个不靠谱的傻媳妇进家门,日子更过不得了。

顾大人酒量不错,虽然断断续续的喝了一晚上,但此刻只是微醺,迷迷糊糊的不闹不吐。正是昏昏欲睡之际,房门忽然被人敲响了。

顾大人刚把被窝焐热了,绝没有下地开门的意思,只不耐烦的问道:``谁啊?''

门外响起了小姨太太的声音,低低的带着哭腔:``顾旅长,是我,你快开门哪,我院里出事了。''

顾大人一掀棉被坐了起来,心想莫非她是想色诱我?如果真是色诱,可别怪老子将计就计。披了上衣下了床,他走去开了房门:``小嫂子,有事你该去找大哥啊!''

然后他看到了月色中的小姨太太。小姨太太披头散发,身上就系了一件斗篷,一条白胳膊露在外面,皮肤上赫然显出几道红痕。一把抓住顾大人的手臂,小姨太太急促的说道:``兄弟,救命啊,你大哥疯了!''

顾大人登时一愣:``疯了?''

小姨太太见神见鬼的放轻了声音,自己伸了胳膊让顾大人看:``不是说他摔了一跤吗?我刚才回屋见了他,哪知道他就像鬼上身似的,对我又咬又挠。家里上下就数我能降服住他,现在他连我都敢打了,还有谁能管他?兄弟你跟我走,我的丫头已经去找太太了,到时候大家一起上,倒要看看他是怎么回事?''

小姨太太的眼睛被凌乱长发遮了住,瞧不清神情,就只能听见她惶惑的声音:``怎么看也不像是醉了,吓死我了!''

顾大人见状,不能把她推出去不管,只好转身敲了敲隔壁窗户:``师父,忙吗?不忙就起来一趟,外面出了点事,你跟我过去瞧瞧!''

无心刚和月牙``忙''过一场,此刻正窃窃私语的说话,对外面的动静全没留意。忽然听到了顾大人的呼唤,无心``唉''了一声,很不情愿的告诉月牙:``你先睡,不知道顾大人又在闹什么。''

月牙累极了,一动不动的答道:``去吧,把衣服穿好了,夜里风凉。''

无心对于镇长没什么感情,所以穿得挺细致。末了推开房门一步迈出去,他和顾大人打了个照面:``怎么回事?''

顾大人正要让小姨太太说话,不料未等他开口,小姨太太忽然转身跑向院门,迎头正遇上了一名气喘吁吁的仆人。仆人停了脚步,大声说道:``哎哟,您怎么跑这儿来了?大太太正找您呢!''

小姨太太只做了一瞬间的停留,随即继续向外跑去。而顾大人叫住了仆人:``镇长出什么事了?''

仆人做了个深呼吸,然后哭笑不得的答道:``回长官的话,我们老爷把衣裳全脱了,正在院子里打滚骂人呢!''

顾大人和无心对视一眼,知道镇长可能是黑夜里撞着脏东西了。

顾大人让仆人领路,带着无心穿过几重院落,末了到了小姨太太的院内。小姨太太的院子很精致,靠边摆着花花草草,中间是光溜溜的空地。一群仆人明火执仗的站成一圈,照出中间一个光屁股大胖子在胡叫乱骂。一个富富态态的妇人扶着小丫头站在人前,打着哆嗦也在骂人,而所骂的对象,却是不知何时挤进去的小姨太太。小姨太太依然是披头散发,显然是被大太太骂老实了,缩在斗篷里一声不出。

大太太没了主意,让仆人去拽老爷,可是仆人一旦靠近,必定会被老爷抓咬厮打。眼看丈夫丢人现眼至此,她又气又怕,索性对着小姨太太发了火,满嘴骚狐狸臭□的乱骂,一口咬定``就是你魇了老爷''。

顾大人既然来了,自然不好袖手旁观。束手无策的摸了摸脑袋,他问无心:``你看出问题了吗?''

无心一直站在他的身后,此刻轻声答道:``鬼上身,不是大事。''

顾大人侧身给他让出了路:``那你还不快去治一治?''

无心迟疑着没有迈步:``顾大人,我想不通。这鬼哪天不能上身,非要赶在今晚?照理来讲,官兵所到之处阳气杀气都重,不是阴魂作祟的好时机啊!而且一般鬼魂是没有力量上活人身的,既然能上,这鬼魂就必定有来历,有所图。可是你看,镇长一味的只是发疯,连小姨太太都能安全逃出去,可见他没有杀机,倒像是\ldots{}\ldots{}''

顾大人抬眼望向了他,心中也是一凛:``倒像是什么?''

无心翕动嘴唇,声音低得类似耳语:``倒像是在故意捣乱。''

顾大人也随之压低了声音:``可这捣乱的目的是什么?''

无心正要回答,哪知就在此刻,人群中的镇长忽然直起了身,一头撞向了顾大人。顾大人猝不及防,当场被他撞了个跟头。

无心知道顾大人身强力壮,和谁打架都吃不了亏,所以后退一步并不出手,只是留意周遭情形。正值此刻,小姨太太拢着斗篷跑了过来,仿佛是要和仆人一起合作营救顾大人。而顾大人被镇长压了个四脚朝天,气运丹田一蹬腿,大喝一声踹中了镇长的胸膛。镇长顺着力道向后一仰,泰山压顶似的拍向了地面。众人慌乱散开,其中五姨太后退一步,踉跄着正是靠住了无心。无心垂眼一瞧,忽然在五姨太的头顶发现了一点银光。

一根粗长的钢针,在丝丝缕缕的黑发之中露出了尾端,反射了灯光。无心终于恍然大悟——原来镇长真的只是一面挡箭牌!

小姨太太钢针入脑,如今已然是一具行尸走肉。无心来不及多说,正要反剪住她的双手,可是就在他将要动作之际,一直惊恐的小姨太太忽然稳稳的回过了身。裹在身上的斗篷被风吹开了,藏在里面的右手举起一把匕首,一刀扎向了无心的眼睛!

无心当即歪头一躲,同时抬起右手,让刀尖掠过了掌心。趁着小姨太太未收回手,他一掌拍上了对方的面孔。伤口迸出的点点鲜血尽数涂在了她的脸上,小姨太太一声哀嚎,随即倒在地上抽搐成了一团。围观的仆人吓傻了,只见小姨太太仰卧在地,仿佛被浇了滚油一般痉挛不止,双手十指狠狠抓着地面,似乎周身的关节都要断裂错位。

片刻过后,小姨太太安静了;镇长方才倒在一旁,如今也安静了。

大太太最先神魂归位。她颤巍巍的走上前来,首先去看镇长。镇长大睁着眼睛,气息已无。

伺候小姨太太的老妈子也凑上来了,心惊胆战的想要拂开对方脸上的乱发。然而在看清乱发下的面孔之后,老妈子吓得一屁股坐在了地上。

小姨太太也是死不瞑目,黑眼珠向上翻起,嘴角却是微翘,居然还带着笑意。

院内奇异的安静了,无心望着地上暴死的二人,心中越来越慌。控制镇长和小姨太太的鬼魂到底是要干什么?只是为了害命吗?可是人早已死了,何必还要借尸还魂的演一场闹剧?想要借刀杀人?杀谁?杀顾大人?杀自己?

无心忽然打了个冷战,抬头对顾大人喊道:``月牙!''

随即他扭头就跑。而顾大人怔了一下,一言不发的立刻追了上去。

\chapter{人间苦}

全宅子的人都跑去瞧镇长了,其余院落就变得寂寞空落。无心和顾大人一前一后冲向所住的小院。在进院的一瞬间,连殿后的顾大人都嗅到了隐隐的血腥气。而无心猛然刹住脚步,俯身从地下捡起了一只小荷包。

荷包上的细带子断裂了,荷包口收得却紧,是月牙永远贴身挂在脖子上的小物件。隔着一层薄薄的布料,可以捏出里面折好的黄符。细带子是湿的,浸的不是鲜血,而是脓水,散发出腐臭味道。顾大人抽抽鼻子,知道是不好了!

而在他开口之前,无心疾冲向了房门。

房门是虚掩着的,推开门是迎面一片温暖的漆黑。汩汩流淌的鲜血浸润了微凉的春夜,棉被从床上拖到地下,而月牙被一柄钢刀穿透胸口钉在床上,一身的单衣被血染红了,红的像她去年为自己缝纫出的嫁衣。

她还清醒着,可是不呻吟。一口热气存在胸间,她要等着他回来。

无心站在了床边,俯身唤道:``月牙?''

他的声音轻而颤,是又惊又痛又绝望。伸手抚上她的面颊,触及之处一片湿热。刀子割了她的脸,她是受了酷刑。

月牙忍着不死,等了又等,终于等回了他。本来前一个时辰两人还亲亲热热的分享着一个被窝,没想到只是一刻钟的工夫,她一生一世的日子就化为了乌有。她知道自己是不成了,她甚至都感觉不出了疼。

``是岳绮罗。''她开了口,声音很轻,然而很稳:``她跑出来了,带着个骨头架子。''

在回光返照的平静中,她定定的凝视着无心。要说的话太多了,约好了是过一生一世,现在提前没了一个,另一个怎么办?

所以她不能停,她得趁着气息还足,把话说完:``我不求你给我报仇,你要是打不过她,就赶紧往远了跑。''

无心答道:``嗯,我记住了。''

顾大人的脚步声缓缓近了,黑暗中能听到他呼哧呼哧的喘息声音,是怒不可遏、欲哭无泪的光景。一只大手伸到月牙胸前,他想拔刀,可是一旦拔刀,月牙必定立死。

月牙听出了他的动静,于是又开了口:``顾大人\ldots{}\ldots{}''

顾大人闷声闷气的答道:``啊,月牙,你放心吧,我肯定给你风光大葬。祸害你的妖怪娘们儿,我也饶不了她。''

月牙扯动嘴角微笑了:``顾大人\ldots{}\ldots{}你对我俩一直挺好\ldots{}\ldots{}''她的声音越来越弱:``以后我没了,你替我顾念着他\ldots{}\ldots{}他没啥正经本事,将来要是穷了,你想着给他口饭吃\ldots{}\ldots{}''

顾大人的声音又粗又哑:``月牙,我向你保证。有我一口稀的,就有他一口干的。我还能养不起一个他吗?我有兵有钱有地盘,养他就像玩似的!''

月牙点了点头,然后把目光又转向了无心:``咋不点灯呢?点灯,我再看你一眼。''

``嚓''的一声,火苗窜起,是顾大人划燃了火柴。烛台上的蜡烛一根一根的亮了,月牙的面孔渐渐显现在了光明中,血痕交织,狰狞纵横。眼睁睁的望着无心,她气息一颤,一滴血泪顺着眼角滑落。

``咱俩才过了一年\ldots{}\ldots{}''她的声音越发轻了:``往后\ldots{}\ldots{}你一个人\ldots{}\ldots{}咋办啊\ldots{}\ldots{}''

她只有一双眼睛依然洁净明亮,一眨不眨的盯着无心:``无心,我跟你\ldots{}\ldots{}没过够\ldots{}\ldots{}''

无心一言不发的凝视着她,有透明的液体在他眼中汇聚成滴,悬在睫毛上,粘稠而又沉重,是他的泪。

``月牙。''他轻声说道:``我也没过够。''

月牙笑了:``以后\ldots{}\ldots{}我不伺候你啦\ldots{}\ldots{}你自己好好活吧\ldots{}\ldots{}''

然后她缓缓的眨了一下眼睛,望着无心又看了半晌。

最后,她慢慢闭了眼睛。口鼻逸出浅浅的一声叹息,带着她短暂一生中所有的苦乐与留恋:``没过够啊\ldots{}\ldots{}''

无心仰起了头,已然凝固的透明泪珠坠落下去。微弱的光芒在他眼前流动闪烁,是月牙的魂魄脱离躯壳,挽不回,留不住。

顾大人的卫队包围了小院,不许闲杂人等靠近。无心端了热水关了房门,要为月牙擦身;顾大人独自靠墙站在门外,不歇气的一根接一根抽烟。不敢歇,眼泪与哭泣就堵在他的喉咙里,他得用一口一口的烟雾把它们压住。

房内又加了一副烛台,烛光几乎可以媲美电灯。无心拧了一把毛巾,去给月牙擦脸。两人做了一年的夫妻,全是月牙照顾他,月牙把家里的活全干了。

月牙死得惨,周身的关节竟然都被捏碎了,所以临死前想要摸摸无心都不能够。无心很细致的为她擦去身上的血渍,没过够,两个人,在一起,都没过够。

无心经过了无数次的生离死别,可每次的主角对他来讲,都是独一无二。让他彻底忘记一个人,也许只要一天,也许需要一百年。

无心给月牙换了一身干净衣裳。顾大人命人套马车,拉着月牙回了文县。夜色深沉,他和无心并肩坐在车里,顾大人问他:``你媳妇让人给弄死了,你怎么想的?''

无心答道:``我想报仇。''

顾大人又问:``有计划了吗?''

无心摇了摇头:``正在想。''

顾大人抽了一夜的烟,此刻下意识的又要去摸烟盒:``想明白了就说话,我有人有枪!''

无心``嗯''了一声。

月牙没娘家没儿女,天气又热,所以葬礼没法办得太复杂隆重,三天之后就出了殡。三天里无心一直守在灵堂里。搬了个小板凳坐在月牙身边,他闭着眼睛歪着脑袋,用面颊去贴月牙的手背。月牙身上苫了一层白布单子,静静的躺在灵床上。家里没了她,立刻就不像家了。顾大人不知跑到了哪里去,只有一个小勤务兵会一天三顿来送饭菜。厨房里清锅冷灶的,从早静到晚。无心把月牙的针线笸箩端到面前,笸箩里面扔着一只未完工的大布鞋。月牙总不闲着,做不完的饭菜,做不完的针线;饭菜做得快,针线做得慢,说要给顾大人做一双鞋,直到现在还没做成。无心捡起布鞋看了看,知道自己又是一个人了。

顾大人再好,不是月牙。顾大人有他自己的事业,将来还会有他自己的家庭,有他孙男娣女一大群热热闹闹的亲人。而他无论在何处活久了,都会活成众人眼中的谜团。顾大人对他再有感情,也没法向亲人们解释他所有的谜。

可月牙就不一样了。

他是月牙的唯一,月牙是他的唯一。月牙不必为他的存在辩白,反正他们只为对方负责。你们看不惯我们,我们就走。

无心弯下腰,把笸箩里的碎布头一片一片的整理好。月牙从来不肯轻易扔掉任何破烂,仿佛预备攒出个千秋万世的基业来。无心攥着一大把五颜六色的布条,忽然自言自语的开了口。

他说:``我想你。''

在月牙下葬的当天,顾大人风尘仆仆的回来了。

他赶在盖棺之前进了门,进门之后大喝一声:``慢着!''

然后他大步流星的挤到了棺材旁边,从军装口袋里掏出一只金丝绒小盒子。盒子打开了递给无心,他对着棺材里的月牙一歪头:``你给她戴上。''

无心接过了小盒子。盒子里垫着紫红色的绒里子,上面摆着一副钻石耳坠。耳坠子亮晶晶的,像两滴泪,也像两抹闪烁的泪光。

在棺材旁边弯下了腰,无心伸手摘了月牙耳朵上的小金耳环,为她把钻石坠子换了上。两个人都知道月牙如果活着,一定不会让顾大人花钱买钻石。她有了金的,已经非常知足了。

顾大人把月牙葬在了文县城外。

葬礼结束之后,顾大人和无心还停留着没有走。顾大人问道:``你不是会念经吗?怎么没给月牙念上一段?''

无心摇了摇头:``因为我根本就不想让她走。''

顾大人又问:``接下来怎么办?''

无心说道:``我要等岳绮罗。''

顾大人没听明白:``等岳绮罗?她把你媳妇都杀了,还不得早早就逃了?''

无心又对墓碑望了一眼,随即迈步向前走去:``她不怕死,不会逃。''

顾大人追上了他:``你要在哪儿等啊?不会是在家里等吧?''

无心低声答道:``我要去猪头山。''

\chapter{三种心思}

无心坐在老树高高的枝杈上,前方就是天边火红的晚霞。太红了,像一场大火,摧枯拉朽的烧过了整条地平线。一只乌鸦在空中留下了一个漆黑的剪影,``哇''的一声兴高采烈,大概是因为白昼结束了,它也要回家歇着去了。

无心手里捏着半个干馒头,想月牙如果还活着,晚饭也该摆上桌了。开饭之前是最热闹的,月牙一趟一趟的往房里搬运饭菜和碗筷,同时扯着嗓子呼唤他和顾大人。他和顾大人都饿了,但是偏在吃饭之前都有事做,非得让月牙三催四请。月牙气得唠唠叨叨,先骂无心:``把你那破书放下,大白天的不见你翻,天黑你倒用上功了!''然后再嚷顾大人:``你说你从下午就吵着饿,饿到现在饭菜都好了,你咋还钻茅房里不出来了?''

他跟着凑趣:``可能是饿得厉害,已经在里面吃上了!''

月牙笑出了声音,同时顾大人走出茅房,气吞山河的发出了质问:``谁他妈又拿我开心呢?''

无心想着想着,就忍不住笑了。

家里没了月牙,就不成了家。前些天忙着办丧事,乱七八糟的倒也把日子混了过去;及至丧事结束、日子清净了,他和顾大人才发现他们没有家了。

勤务兵从馆子里买回饭菜送进上房,他和顾大人相对而座,没滋没味的填饱肚皮。太冷清了,太荒凉了,能让人吃出叹息,吃出眼泪。

无心和顾大人都不说话,都知道为期一年的好日子,结束了。

无心上了猪头山,该去的迟早要去,该来的迟早要来。一年的光阴成了黄粱一梦,他独自坐在老树枝杈上,把余下半个干馒头塞进了嘴里。旧日的空气渐渐包围了他,月牙的死,把他打回了原形。

他的原形,就是永恒与孤独。

恐怖的永恒,永恒的孤独。永生的人,也有自己的轮回。

咽下馒头又拍了拍手上的渣滓,无心向后依靠上了一根枝杈。暖屋子热被窝都不再有了,他从怀里摸出一张小小的照片。照片上的他和月牙欢天喜地,肩膀挨着肩膀,脑袋抵着脑袋。月牙说他比自己照得好,如果梳起小分头,会像电影明星;月牙还说以后每年都去照一张合影,一张一张攒起来,倒要看看自己咋变成个老太太的。

可是他们只有一年的光阴,他们的合影,也只有一张。照片上的月牙笑成了个圆圆满满的苹果脸,以至于她看到照片后有些懊悔,忍不住问:``我是不是笑大了?''

无心盯着月牙的眼睛看,又想起自己似人非人的时候,因为肚子饿,曾经把月牙的手指头咬出了血。然而月牙还挺高兴,因为他长出牙齿了,知道吃东西了。

无心把照片揣回怀里,心中没有风也没有雨,空空荡荡一望无际,什么都没有了。

顾大人奉了无心的命令,把自己的心腹副官派去了火车站,让他去天津寻找出尘子。出尘子或许不在天津,不过没有关系,反正他是个有名的人物,只要想找,肯定能有法子找到。

然后他搬到了窑子里住。家里没了月牙,又跑了无心,如今简直成了他的禁区。他没法回去睡觉,因为触目之处全刺眼睛。三个人在一起出生入死的混了一年,他第一次发现自己的生活中竟然处处都是月牙和无心。

枕着双臂躺在软床高枕上,他没有和身边的妓女玩笑,而是沉沉的想起了心事。

他在想无心和猪头山。无心说要等岳绮罗来找他,所以要去猪头山等待。顾大人起初以为他是怕给自己惹麻烦,所以故意想要远离自己,然而三言五语的追问过后,他又感觉无心仿佛别有主意,只是不说。

这让他有点不痛快,认为无心和自己不亲了,不过还是骂骂咧咧的发表了意见:``你不知道猪头山上有鬼啊?到哪儿等不是等?这一带别的没有,山有的是!青云山,小黑山,妃子岭\ldots{}\ldots{}你上哪座山不行,非得去猪头山?我告诉你,我现在一提猪头山就吓得腿肚子转筋,山上到底有什么,当初咱们三个可是亲眼见过的,我不信你一点也不怕!''

然而无心不听话,也不解释。

于是顾大人换了策略,又问:``那你打算在山上住多久?山上要什么没什么,如今野菜都老了,也打不到正经动物,你在山上喝风屙屁?''

无心对着他笑了笑,还是要去。

顾大人气得一挥手:``滚你的蛋!''

等到无心当真滚蛋了,顾大人把这件事从头到尾的回忆了一遍,怎么咂摸怎么不是味。猪头山上除了有个鬼洞之外,其余地方再无奇异,和周遭所有的山岭一样。无心死活非上猪头山不可,也许就是为了那个鬼洞。自己当初带他进过一次鬼洞,差点没被鬼手拽进洞壁里去,现在还是噩梦的源泉;逃上地面之后,无心闹了脾气,因为洞里太危险,他也怕被鬼手缠住。听无心的意思,似乎是凡人被鬼手抓住之后,无非就是一死;而他既死不成,又逃不出,岂不是陷进了活地狱里?

顾大人犯了疑心病:``他不会是要在鬼洞里面做文章吧?''

自从月牙死后,无心一直是闷闷的,未见得多悲伤,倒像是若有所思。顾大人看了他鬼气森森的阴郁样子,几乎有些怕。如果无心一夜之间变了妖或者吃了人,他都不会太惊讶。

鬼洞里能做出的文章,无非是把岳绮罗诱进去喂鬼。可是话说回来,岳绮罗前脚断了气,后脚就能转世投胎。活上十来年,又是个新的岳绮罗。无心早就说过岳绮罗不能杀,杀了之后更麻烦;可见他是别有心肠。但到底是什么用意,顾大人思来想去,可真是猜不透了。

顾大人想亲自去趟猪头山,把无心拎回来拷问一番,不说就揍,打服了算。然而无心早在上山之前嘱咐过他,万万不许他进山寻找自己。顾大人见识过了月牙的惨死,不能为了好奇心搭上性命,所以在去与不去之间,他长吁短叹的犹豫不决,实在是拿不准主意。

顾大人在妓院里辗转反侧,不能入眠。与此同时,无心却是在树上入睡了。

除了顾大人之外,岳绮罗也在失眠,陪着她的人,还是张显宗。

岳绮罗坐在猪头山中的密林里,仰起头可以可见漫天星辰。张显宗远远的躺在一丛荒草里,因为自惭形秽。

没人知道他们是如何逃出千佛洞的,连他们自己都不能详尽的描述。半边身体上的腐肉都被怪物的尖爪利齿撕扯掉了,绿油油的草叶穿过了他的肋骨,肋骨不干净,上面还存留着丝丝缕缕的血肉。

左臂也没有了,原来肉体真是脆弱之极,能够腐朽到不可收拾的地步。前几天他还能用左手扯下月牙颈上的荷包——荷包里有黄符,会伤害岳绮罗,但是他不怕。

可在接下来的几天里,他也不知道是怎么回事,左臂的骨头零落分解,最后竟是一节一节的自行脱落尽了。

失了左臂,他也不心疼,因为他活够了。

忽然,岳绮罗开了口:``你为什么不听我的话?''

她的声音有点嘶哑,带着怒气:``当时为什么要躲开?''

今天下午,在他们进入猪头山之前,岳绮罗给他找到了一具新的身体,是个十六七岁的半大孩子,挑着扁担立在山路上,魂魄已经被岳绮罗勾了出去。类似的试验,岳绮罗已经做过一次,然而失败了,因为张显宗的力量似乎越来越弱,已经不能控制完全陌生的身体。

她不甘心,还要再试,然而张显宗避开了。

猛然扭头望向张显宗的方向,她提高了调门,恶狠狠的说道:``你到我面前来!''

张显宗缓缓坐起了身。明亮月光洒了他一头一脸,把他曝露出来的头骨镀成银白色。他的面孔已经近似骷髅,仅在腮部还存留着一点皮肉。行尸走肉是见不得天日的,只有他敢在大太阳下走,一方面是因为岳绮罗法术高明,能保护他;另一方面,则是他在拼命。

他没有命了,可是依然在拼。他的灵魂已经很虚弱,他心里明白,他甚至能够预感到自己终有一天会无可挽回的魂飞魄散。

窸窸窣窣的起身爬到了岳绮罗面前,他让她看,希望她看到恶心看到吐,看到永生不想再看。这样他会走得更安心,不再留恋不再妄想。

然而岳绮罗目光森冷的凝视着他,神情并无波澜。

她也快要支持不住了,右眼上的血点已经扩散成了红斑。支持不住了会怎样?她不知道,不过至多就是一死,而她并不怕死。

把手伸向张显宗的面孔,她从他空洞的左眼眶中捏出一条蠕动的蛆虫。左眼珠是昨夜脱落的,他只是一低头,它就无牵无挂的落在地上,溃败的砸出一摊脓水。

``你坚强一点好不好?''岳绮罗弹开蛆虫,肮脏的小脸上没有表情:``他们把我们害成了这个样子,难道就算了吗?月牙已经死了,接下来就是无心!世上无难事、只怕有心人。无心的身体是永远不死的,我要想办法把它抢过来给你!''

张显宗轻轻动了动右手,一截指骨脱离关节,静静的留在了草地上。他无法露出笑容了,心中只有无尽的疲惫与悲苦,以及一点意外的小幸福:``绮罗,谢谢你。可是\ldots{}\ldots{}''

未等他把话说完,一个白影飘然而至,是附了魂魄的纸人靠近了,双手掐着一只小小的灰兔。岳绮罗扬手接过半死不活的兔子,低头一口咬上了兔子的咽喉。小灰兔在她手中微弱的抽搐着,而她捧着兔子仰起头,像是捧着一只水壶,闭上眼睛汩汩的吸血。

她好饿。饿了,就压制不住右眼中的毒。她不怕死,可生死毕竟是件大事情,如果能活,还是活着更好。

\chapter{无依}

虽然张显宗已经腐朽到了不大能动的程度,可是岳绮罗自能驱使身边一切魂魄,并不缺少喽啰。大白天的,她双手捧起一只肥田鼠,仰起头几口吸尽了鲜血。指尖捅进死鼠的伤口里转了转,她转身在张显宗的身上画起了符。

张显宗委顿在树荫下,情形类似一具最糟糕的腐尸。肉体溃败着,魂魄的光芒也越来越弱,所以岳绮罗须得在他身上一道一道的加符,极力想要锁住他的魂魄,不让他在大太阳下魂飞魄散。

张显宗的喉咙已经烂穿了,让他不能再发出声音。右眼的眼珠深深陷进眼窝,无法转动了,可是还能依稀看到岳绮罗。岳绮罗越来越脏了,头发乱蓬蓬,脸上横七竖八的抹着血痕,看起来正是一个最凄惨的小叫花子。

可怜,真可怜。她杀人吃人,张显宗认为不算什么;她杀不到人吃不到人了,张显宗悲哀的望着她,就感觉她太可怜。

岳绮罗画完最后一笔血符,然后摘下一片草叶擦了擦指尖。抱着膝盖席地而坐,她忽然托着腮揉了揉,低声咕哝道:``牙疼。''

张显宗无能为力的瘫在阴影之中,心里想:``她牙疼了。''

岳绮罗漫无目的的坐了一天,傍晚时分她又饿了,于是砸烂了田鼠头,吮吸到了有限的一点点脑髓。用沾染着红白黏液的手指从怀里摸出三张纸片,她漠然的向外一甩。还是没有找到无心,可是据她所知,无心就在猪头山中。

夕阳将落未落,她的身边幻化出了三个纸人,替她四处游荡,一边寻找无心一边打猎。抠出田鼠眼珠也塞进嘴里,她的舌头和眼珠打了架,滑溜溜的没有立刻下咽。百无聊赖的四处张望了一番,她最后仿佛痛下了决心似的,一口咬爆了口中的眼珠。

与此同时,不远处的草丛中腾起一团无根的火焰。她猛然抬头,就见火光一闪即逝,瞬间照亮了无心的身形。月黑风高,无心站在随风摇曳的野草之中,鬼魅一般无声无息。

踏破铁鞋无觅处,得来全不费工夫。岳绮罗并没有起身,双手向下垂到地面,她现在和无心已经无话可说。其实根本就不曾有过什么爱情,她想,自己只不过是对他好奇。几辈子了,一切都在变,只有好奇心不变。如果不是因为好奇,她当初就不会把心血和生命全耗在道术上,后来更不会把自己修炼成了妖魔。

指尖轻轻的动了,她不动声色的开始画符:``我知道你一定在山里。''

无心抬起右手,露出了一柄雪亮的短刀。左手掌心缓缓抚过刀刃,他在疼痛中骤然冲向了岳绮罗。而岳绮罗看清了滴血的短刀,登时勃然变色。放弃了手下尚未完成的符咒,她起身对着无心一甩衣袖。可是未等纸人出手,无心的刀已经逼近了她的眉心。可是就在寒光将要劈下之时,一道黑影斜刺里冲出来,硬生生的替她挡了一刀。与此同时,白色纸人幻化成形,岳绮罗在一刹那的犹豫之后,扭头就跑。

纸人是不足畏惧的,一把火便能把它们化为灰烬。而地上的张显宗抽搐成了一团肮脏的骨肉。刀刃上浸染了无心的鲜血,破了岳绮罗施加给他的所有符咒。黯淡的魂魄忽然明亮了,回光返照之后,便是一场痛苦的魂飞魄散。

无心低下头,饶有耐性的等待张显宗彻底死亡。他知道张显宗会为岳绮罗挡刀,就像月牙会为自己开枪一样;岳绮罗杀不得,张显宗还杀不得吗?

一个一个来,谁也错不过,谁也逃不脱。他什么都没有,唯有时间无限。

无心烧掉了张显宗的骸骨。火苗微弱,在夜风中微微的颤抖,像一颗垂死的星星坠落在地。岳绮罗藏在不远处的一小片密林里,左眼死死的盯着火光。右眼一胀一胀的剧痛了,痛到牵扯了她的心脏。

火光熄灭之后,山林归于漆黑寂静。岳绮罗坐在一棵老树下,无声的翕动了嘴唇:``张显宗。''

她以手托腮,不带感情的发出声音:``张显宗,我牙疼。''

向后靠向老树树干,她继续自言自语:``这辈子没活好,很糟糕。''

无心沿着山路走,一直走到了鬼洞附近。随便找了一棵树爬上去,他察觉到周遭游荡着无数鬼魂,全是岳绮罗的耳目,自己可以守株待兔了。

除了他和顾大人,恐怕再也没有人会想到树下竟然藏着一处洞口。从树上向下看,是匀匀的一片绿草,地下本来还有一块方方正正的石板,被他前几天掘了出来,抬到了十米开外的一道土沟里。石板太重了,记得当初他和顾大人合力才能掀动;可是如今他单枪匹马,却也搬运成功了。

石板没有了,改用细树枝横七竖八的搭出骨架,上面盖一层席子,再盖一层草皮,能禁得住一只大号的野狗踩踏。

无心像一条蟒蛇一样,长长的趴在了枝干上,怔怔的望向地面。

``如果我在里面陷了一百年,两百年,三百年。''他想:``那它算不算是我的坟墓?''

然后他摇了头。坟墓是安静的所在,他充其量只算是堕进了地狱。

可是,他随即又想:``没关系,我不急。''

世间没有了月牙,他永恒的流放就又开始了。

凌晨时分,无心被一阵响动惊醒了。

他依然趴在树枝上,睁开眼睛望向下方,他看到了地上一片波浪起伏,不是野兽,是十几名行尸走肉的脊背。它们四脚着地的往前走,大多都还保留着肮脏恶臭的衣裳,是军装,因为几个月前刚刚开过战,山下是条过兵的道路,炮火不断,不会缺少尸首。

行尸的目标,显然就是他所栖息的大树。而无心抬眼向前,看到了行尸后方的岳绮罗。借着稀薄黯淡的晨光,他看到岳绮罗也在仰脸凝视自己。

岳绮罗变样子了。

她曾经稚嫩白皙的小脸,现在已经在血痕下面呈现出了衰败的青灰色。凌乱的齐眉刘海下,她的右眼不再黑白分明,而是通体转成了血红颜色。

``知道我要干什么吗?''她出了声音。

无心缠在树枝上,一双眼睛陷在了阴影里:``杀我?''

岳绮罗笑了一下:``非也,是让你重生给我看。''

无心把下巴抵上了粗糙的树皮,眼中光芒一转。天光越来越明亮了,可他的瞳孔依然黑得如夜:``一个意思,没有区别。''

岳绮罗把双手揣进了袖子里:``你我之间,谈生谈死都没意义。''

行尸缓缓靠近了大树,显然,它们异于同类,竟然已经不怕阳光。姿态僵硬的直立了身体,它们作势开始爬树。爬是不容易的,可是只要想爬,叠罗汉都上得来。

无心知道自己落入行尸群中,必定会被撕咬成为碎片。对着岳绮罗又瞟一眼,他心里有了数,顺便紧了紧系在背上的短刀。

岳绮罗仰着头,等着看一场好戏。等到无心杀光这一批行尸,她会再召一批,让他杀个够。不是会杀吗?不是会把张显宗烧成灰烬吗?很好,让他杀,倒要看看他有多少鲜血,多少力量!

果然,随着行尸的逼近,树枝上的无心爬起来了。

他险伶伶的蹲在树枝上,一只手抬起来,握住了后方的刀柄。树枝一颤一颤,快要禁不住他的重量,而一只行尸已经上了枝杈,正在东倒西歪的向他爬行。可就在腐烂的手掌搭上树枝的一瞬间,无心忽然纵身向外飞跃出去。借着树枝的弹力,他从天而降,直扑岳绮罗!

岳绮罗当即后退一步,正要有所反应;不料无心下落之后就地一滚,随即一跃而起,瞬间冲到了她的面前。张开双臂抱起了她,无心向后一仰,合身砸向起伏草地。只听``喀嚓''一声,草地豁然开裂,两个人已然相拥着坠入了深洞之中。

\chapter{归于黑暗}

落地之后向内一滚,无心和岳绮罗就一起没入黑暗中了。

岳绮罗挣扎着伸出双手,想要扒住洞壁;然而一个小姑娘的身体根本敌不过无心的力量,她的指尖在地上留下一道一道的抓痕,指甲生生翻开了,她怒不可遏的大吼了一声,鲜红的右眼珠随之爆裂,浓稠的血浆直迸溅到了无心的面孔上。

无心不为所动,拖拽着她往深处走。她知道不好了,洞中一定是别有玄机。血淋淋的手指划上无心的眉心,她不间断的画出一道道符咒,想要镇住对方。

可是,没有用。

右眼眶中汩汩的流出鲜血,洞中的血腥气越来越浓了。情急之下,她起了同归于尽的心思,一指抠向无心的眼睛。而无心仰头一躲,却是个很惜命的样子。

他不想让岳绮罗被自己的血毒死,他要让对方活。大踏步的连拐了几个弯,一块泥土从天而降,碎在了他的头顶上。

如他所料,这座地洞已经和洞中的女鬼化为了一体。一切进入其内的活物,都会把它惊动,被它吞噬。去年它吞下了几十名年轻的士兵,如今岳绮罗的鲜血洒了一路,它又要开斋了!

还未到达地洞尽头,洞内如同发生了地震一般,洞壁已经开始簌簌的落下泥土。一条血肉模糊的手臂骤然突破泥土伸了出来,在无心的颈后抓了个空。岳绮罗万没料到洞内会是此情此景,惊恐之余却是大声笑了:``无心,要和我一起死吗?''

她奶声奶气的大笑回荡在洞中,是一串尖利的叽叽咯咯。一条手臂横伸出来抓住了她的细手腕,带着千钧之力向内缩入。她猝不及防的顺着力道伸出了手。可在手指没入洞壁的一刹那间,她骤然长声惨叫起来。另一只手不知从哪里摸出一张纸符狠狠掷去,薄薄的纸符飞刀一般切断了鬼手,而她强行把手抽回,手掌鲜血淋漓,从指尖到掌心如同浸过镪水,皮肤肌肉全被蚀去,只剩鲜红的掌骨带着筋脉。单手握住伤手手腕,她似乎明白了,似乎又不明白——她是不怕死的,难道无心不知道她不怕死吗?

冷不丁的打了个激灵,她猛然扭头怒视了无心。而与此同时,无心已经在黑暗中下了手。两只手掌搡了她的后背,她猝不及防的一个踉跄,合身便栽向了洞壁。

可是在向前扑倒的一刹那间,她回手用力扯住了无心的衣袖。未受伤的好手显出了从未有过的灵活,手指顺着衣袖攀上小臂,她把毕生的力量全用在了手上。在无心扬手拔刀之前,她锐声叫道:``一起走吧!''

在拉扯无心的同时,她的额头已经触到了泥土。泥土温暖松软,似乎每一粒土壤都带着獠牙利齿,撕咬着送到口中的每一寸血肉骨皮。而无心站立不稳,在她发出哀嚎的下一秒,侧身也撞向了洞壁。一只鬼手已经掐向了他的脖子,他的肩膀陷入泥土,刺骨的疼痛让他向后猛的一纵,然而还是晚了,肩膀上衣物皮肉全脱落了,几乎没有血,直接露出了白生生的骨头。

他被鬼手扼住了脖子,身边又无处可以借力挣脱。一只皮破肉烂的小手忽然伸到了他的面前,他发现岳绮罗正在一边奋力后退,一边高举了一只皮破肉烂的手,要在洞壁上画出符咒。无心不知道她的符咒会有何等效应,他只知道不能让她再反抗下去了,否则她失血过多,真的会死。不能让她死在外面,死在外面就是前功尽弃!

拔刀砍断了纠缠自己的鬼手,无心走到岳绮罗身后,对着她的后背就是狠狠一推。岳绮罗本来就是垂死挣扎,如今受了偷袭,越发体力不支。在俯冲向前的一瞬间,她使出最后的力气抬脚一蹬洞壁。仰面朝天的摔倒在地,她在被鬼手抓住双腿的同时,回身也死死抱住了无心的大腿。鬼手拖着她往泥土中拽,而她牙关咬得咯咯直响,在自下而上的吞没之中抬头瞪视了无心。无心握着短刀,满可以立刻砍下她的手臂,可是不能砍,因为怕她太早的死!

对面的洞壁也伸出了鬼手,招招摇摇的一大片。无心握住一只鬼手,想要借力蹬开岳绮罗,然而洞内狭窄,根本容不下他横躺。岳绮罗的双臂像铁一样箍住了他的大腿,他的双脚随着她的胸口一起陷入了泥土中。

纠缠着岳绮罗的鬼手忽然瑟缩了一下,连带着岳绮罗也发生了痉挛;他知道是自己的血流出来了,可是吞噬与吸收依然在进行,岳绮罗忽然抬起头,对着无心恐慌的惨叫了一声。

一声过后,她被一只鬼手捂住嘴,彻底摁入泥土之中。

而无心抡起了刀,一刀砍向了自己的大腿。

他怕疼,一直怕。刀是普通的刀,不算很锋利,也不算很结实。无心的脸上没有表情,一刀接一刀的砍下去,直到砍断了自己的大腿骨!

刀刃卷了一处,然而他的酷刑还没有完。另一条腿已经陷到了膝盖,他一边勉强固定了身体,一边抡起钝刀,继续剁下。类似哭泣的哽咽在洞中回荡,骨头太硬了,刀刃又太软了。鬼手从四面八方逼近,他走投无路的低下了头,双手托起骨断筋折的大腿,用牙齿去咬开最后相连的一点皮肉。

他疼极了,疼到浑身哆嗦,疼到让他想起了曾经受过的一场又一场非刑。握住短刀向前爬去,他扔下的两条腿被鬼手迅速瓜分了,尽数消失在了洞壁泥土中。

岳绮罗没了,他的腿也没了,他自己成了鬼手的下一个目标。洞穴深处传出了隐隐的哭泣声音,哀哀的带着得意。无心没回头,发狂一般拼命的向前爬行。他很会爬,一只手挥起钝刀乱刺乱砍,他调动了一条手臂和两条残腿,在粗糙起伏的地面上摸爬滚打。眼看前方就是最后一道弯了,他一刀挥出去斩断拦路的鬼手,可是在他收刀之前,洞壁忽然冲出一个皮肉斑斓的脑袋,定睛一看,竟然是岳绮罗!

岳绮罗的脸皮头发全被蚀去了,一只左眼却是还在。狞笑着一口咬向无心,她沦为了洞内众多鬼手中的一只。无心无暇躲闪,索性用刀一挡,让她正是咬在了刀身上。仿佛有股力量在后方控制着她,她身不由己的咬着短刀向后缩回了泥土中。而无心趁着空当继续前行,拼死拼活的拐过了弯。

拐过了弯,就安全了。

无心手无寸铁的继续向前爬,爬着爬着,眼前微微的有了光亮。恍恍惚惚的抬起了头,他想起上次自己和顾大人慌里慌张的往外逃,逃到最后向前看,就看到月牙站在一束阳光下。

缓缓的眨了眨眼睛,他看到阳光还在啊,月牙哪儿去了?

在连绵的剧痛中,他停了动作趴伏下去,闭上眼睛集中了精神。洞里真干净,什么都没有。活着的,死了的,全没有。

于是他继续爬行。

眼前越来越亮了,耳中甚至听到了依稀的人声。他怔了怔,以为自己是产生了幻觉。在洞的尽头仰起了脸,他向上看到了一小块碧蓝的天。

一块带着草根的泥土落下来,随之探下的是一个大脑袋。背着万丈阳光,顾大人和无心打了个毫无预兆的照面。

顾大人愣了三秒钟,然后粗声大气的骂出了两个字:``我操!''

随即他的大脑袋消失了。无心就听上方响起了他的号令:``全体向后转!小马你别转,你把装子弹的木箱子搬过来一个!''

木箱子先顾大人一步落入洞中,准确的砸中了无心的脑袋。随即顾大人也跳下来了,跳得顾前不顾后,两只穿着大皮靴的脚一起降落在了无心的后背上。

木箱子不算小,顾大人把无心抱起来塞进箱子里,又悄声问道:``腿呢?''

无心歪着脑袋,极力的蜷成一团:``不要了。''

顾大人听了``不要了''三字,``咣''的一声就把箱盖合上了。

装子弹的木箱子,做工自然不会细致。无心透过一道缝隙向外张望,就见漫山遍野全是士兵,士兵之中又夹杂了一群服饰华丽的道士。顾大人费了大力气把木箱运上地面,然后从土沟里找到石板,依着原样盖好洞口,上面又铺了一层土。

木箱被人抬上一辆小马车,也没人敢问顾大人箱中内容。马车顺着山路往下走了,箱盖重得像有千斤,因为有顾大人一屁股坐在了箱子上。

在从缝隙透进的一线阳光中,无心疲惫不堪的闭上了眼睛。

\chapter{一梦}

顾大人走进文县家里时,正遇上一名小道士站在东厢房外,和房内的无心一应一答。房门是锁着的,因为他怕外人冒冒失失的闯了进去。

小道士神色俨然,穿得也是格外体面。忙里偷闲的对着顾大人一施礼,他同时就听房内问道:``你师祖为什么不回来?''

小道士理直气壮的答道:``师祖说了,他好害怕。''

然后房内的声音换了对象:``顾大人?''

顾大人站在院子里,摘了军帽满头抹汗:``啊,是我。''

无心说道:``顾大人,你进来。''

顾大人开了门上的锁,一闪身钻进房内。片刻之后他溜出来了,向小道士递出了一封信:``他给你师祖的信,一定得送到了。''

小道士立刻接了信往怀里揣:``好嘞,我下午赶火车回北京,晚上就能见到师祖。''

打发走了小道士之后,顾大人又回了东厢房。无心光着屁股趴在被窝里,一边肩膀晾在外面,本来是露出了白骨的,然而经过一天一夜的休养,白骨上面已然生出了一层粉红色的肉膜。顾大人忙得很,长安县的军头决定投到老帅麾下,于是很有保留的投了降。而他作为老帅的全权代表,当然不能藏起来不管事。

一屁股坐在床边,他挺费劲的弯腰脱马靴,床上摆着一张黄灿灿的大纸,上面用朱砂画了个乱七八糟,是出尘子特地派徒孙从北京送过来的,说是无心一定用得上。结果他带兵上山之后,才发现无心凭着一己之力,已然大功告成。

天气热,顾大人穿着大马靴奔波良久,如今大脚丫子见了凉空气,惬意的无法言喻。很自觉的把两只脚伸远了,他在无心身边躺了下去。龇牙咧嘴的抻了个懒腰,他又打了个气吞山河的大哈欠。

``怎么样?''他开口问道:``还疼不疼了?''

无心慢慢的把黄纸折好,塞进一只大信封里:``好多了,不妨事。''

顾大人仰面朝天的枕着双臂,扭头对他笑了一下:``说说吧,怎么回事?昨天把你弄回来之后,一直没抽出时间和你说话。''

无心侧身躺好了,面对着顾大人说道:``我把岳绮罗拖进了鬼洞里,我逃了出来,她留下了。''

顾大人眨巴眨巴眼睛:``不对啊,你不是说不能杀她吗?''

无心问道:``顾大人,你记不记得我们去年冬天最后一次经过鬼洞?当时是有丁大头的士兵来追杀我们,我们从猪嘴镇一直逃进了猪头山。''

顾大人想了想,随即一点头:``记得,我和月牙在树上蹲了半天,看着那帮小兵接二连三的下洞,下去的基本就都没上来。不是还有个闹诈尸的吗?让你抓住烧了,烧完之后你还跳进了洞,我和月牙在树上来不及拦你,急得我俩一边下树一边骂\ldots{}\ldots{}''

无心没有顺着顾大人的话头追忆往昔,只又问:``你猜我当时为什么进洞?''

顾大人摇了摇头:``有话直说!''

无心翻了个身,也向上面对了天花板:``那一夜连着死了许多人,可是我发现洞里洞外都很干净,尸首没有,魂魄也没有。可见\ldots{}\ldots{}''

顾大人略略的明白了:``那地方是有进无出,就算她有转世的本领,不得自由也是白搭,对不对?''

无心点了点头:``没错。我虽然不知道其中的道理是什么,但是洞里的确吸收了许多冤魂,这很奇怪,也很可怕。所以,我给出尘子写了一封信。''

顾大人看着他:``给老道写信干什么?''

无心叹息一声:``让老道来善后吧!或许可以把洞口永远堵死,上面再修座塔压住——他也不是完全的浪得虚名,应该总比我懂得多。让他考量着做吧,以后的事情,我不再管了。''

顾大人跟着叹息:``对,不管了。俩腿都没了,也够卖力气了。''

话音落下,无心没有回应。房内寂静,院里也寂静。无心透过玻璃窗子向外望,能看到半开半掩的厨房门。

顾大人今非昔比,没有时间天天守着无心,可是又不能让外人见了真相。命令卫兵牢牢的把守了院门,他每天早上都会把一天的饭菜端进房内,马桶也摆在床边。然后一把锁头扣住房门,屋子里就剩下了无心一个人。无心坐在床上,怔怔的去看对面的西厢房,看够了,再去看斜前方的厨房。厨房里的灶台上还摆着一只长柄铁勺,是月牙常用的,去猪嘴镇的前一晚摆在那里,从此再也没人动过。

天黑之后,顾大人通常会带着一份热饭热菜回来。无心在成长的阶段里总是胃口惊人,顾大人叼着烟卷靠墙站着,看他捧着海碗埋头大嚼,就不由得想起了天津岁月。那时候他和月牙心惊胆战的怀着希望,一天一天的把个怪物养成了人形。一颗心忽然不可思议的柔软了,他不假思索的开了口:``别成天愁眉苦脸的了,等你长齐全了,我再给你找个媳妇。老子有钱有势,别说你模样还不赖,就算你长成狗头蛤蟆眼了,我照样能给你弄个黄花大姑娘!''

无心对着海碗笑了一下:``万一将来她发现我不对劲了,怎么办?''

顾大人蛮横的嗤之以鼻:``怎么办?继续过呗,敢闹事就往死了揍!嫁太监的还有呢,你不比太监强?没事,你放心吧,真出乱子了,我替你做主!她敢不服,我烧了她的娘家!''

无心听到这里,发现顾大人的坏劲又上来了。顾大人不出头也就罢了,一旦出人头地,将来必定不少作孽。无心素来不喜欢坏人,可是对于顾大人,他只感觉无可奈何。

顾大人的主意,当然是馊主意,无心当个乐子听,听过也就算了。每个人都有自己的姻缘生死,他不能因为失去了自己的月牙,就出手去抢别人的月牙。

顾大人收拾了碗筷,因为懒,所以带着一身汗臭上了床。马桶还是摆在了床尾,他告诉无心:``夜里要是想撒尿了,就推我。使劲推,我睡觉沉。''

展开一床棉被躺下去,他关了电灯,在黑暗中又道:``师父,真的,人只要活着,就得向前看。月牙没了,我心里也难受,可是难受有什么用?难受她也活不了啊!月牙临走的时候嘱咐过我,让我照顾着你,这话我永远记得,我骗谁也不能骗她。现在仇也报了,你也没什么牵挂了,往后就跟着我吧。你应该看得出来,凭我的本领和志气,绝对不是平地卧的角色,养活一个你,肯定不成问题。''

无心笑了笑,没言语。他当然相信顾大人的诺言,可惜,顾大人再好,不是月牙。顾大人将来有妻有妾有儿有女,无须久,只要过上十年二十年,顾大人就无法向亲人们解释他的存在了。

他身上的破绽太多,比如,他不会老。

``顾大人。''他突然说了话:``你知道我为什么不做正经营生,专在鬼神身上挣饭吃吗?''

顾大人立刻答道:``我看你就是个懒蛋,根本没有上进的心思!''

无心继续说道:``我是想让人怕我,远离我。''

顾大人在朦朦胧胧的夜色中看了他一眼:``别胡说八道了,赶紧睡吧。''

无心又道:``自从玉儿死后,就再也没有人善待过我。我没想到会同时遇到月牙和你。这一百来年,我的运气还真是不错。''

顾大人心中涌出了一股子悲凉,当即翻身背对了无心:``行了行了,听你说话都瘆得慌。''

无心不说话了,悄悄从怀里取出他和月牙的合影。把照片摆在顾大人的后脑勺前,他们三个人,还是在一起。

一个月后,无心恢复了人样子。

在一个花红柳绿的五月清晨,他换了一身利利落落的单薄裤褂,说是要去青云观看望出尘子。出尘子新近从北京回来了,似乎是听从了无心在信中的建议,当真要去猪头山修塔。

顾大人睡懒觉睡得睡眼朦胧,蓬着头发光着膀子眯着眼睛,坐在床上一边挠大腿一边问道:``去青云观?行啊,让小马开汽车送你去吧!''

然后他伸脚下床,想要去趟茅房。不料无心站在门口,拦住了他的去路。

顾大人不挠大腿了,改摸下巴上的青胡子茬。无心定定的看他,他莫名其妙,也看无心。无心的眼睛是特别的黑,黑而幽深,是要把他的影子印刻吸收。

顾大人和他对了半天的眼,渐渐的醒透了,不由得抬手揉去眼角的眼屎:``看什么呢?你不是要走吗?''

无心收回目光,忽然张开双臂拥抱了他。手臂紧紧箍住他的赤裸上身,顾大人猝不及防,险些被他勒断了气,并且有点不好意思:``哎,哎,干嘛呀?大早上的别挡道,我还憋着尿呢!''

无心抬手拂乱了他油腻粗硬的短头发,随即松手后退一步。

看不够似的看着顾大人,他微笑说道:``可能要在青云观住上几天,你一个人在家,多保重。''

顾大人不以为然的一挥手:``滚吧!住个三五天就回来,咱们下个礼拜可能就要回天津了。''

在清凉的晨风中,无心对着顾大人点头一笑,然后转身走向了院门。

五天之后,顾大人派小马去青云观接无心,然而小马开着空汽车回了来,站在他面前说道:``观里的出尘子道长说,无心师父只在观里住了一夜,四天前就下山走了。''

顾大人听闻此言,不知怎的,浑身汗毛竖起了一层。撒开人马布下天罗地网,他开始四处寻找无心,然而人仰马翻的找了大半个月后,却是一无所获。

顾大人独自坐在院子里,顶着烈日骄阳发呆。忽然打了一个冷战,他怀疑自己是做了一年的大梦,梦里有个月牙,还有个无心。现在,梦醒了。

顾大人再次和无心相遇,是在十年之后。

那时他已经改名叫做顾庆宣,半俗半雅的,正好符合他越来越高的身份。人无千日好,花无百日红。因为专权和贪婪,他终于在过完四十整寿之后,被他的敌人们联合起来赶下台去了。

顾大人想得开,不犯愁,下台之后住进了天津租界里,领着一大家子继续过阔日子。在一个阳光明媚的午后,他带着两个儿子去逛百货公司,两个儿子全很像他,是儿童的年纪,少年的身量,别别扭扭的都不听话,一路把他扯了个东倒西歪。他本来就是个高大的坯子,如今又发了福,站在街上像个巨大的不倒翁,一手一个的拽着儿子,嘴里气得骂骂咧咧。眼角余光忽然仿佛瞥到了什么,他猛的回头,依稀看到了一个熟悉的背影。正要定睛细看,两个儿子又闹起来了:``爸爸你带我们去吃冰激凌,要不然我们都不走了!''

顾大人一头大汗的转向两个儿子:``吃你妈了个×!再闹就把你们两个小子撕了喂鹰!''

大儿子不怕他,继续耍赖:``不吃也行,你给我十块钱,我自己去吃!''

顾大人又回了一次头,心想:``我看见谁了?''

他也不知道自己是看见了谁,于是在两个儿子的胁迫下,像座大山似的继续前进了。

无心站在街角,隔着人潮去望顾大人的背影。

顾大人老了,胖了,有了一点老太爷的意思。从报纸上读到了顾大人的坏消息,他放心不下,所以特地赶来天津,想要偷偷看他一眼。

还好,顾大人虽然在仕途上受了挫折,然而精气神都足,并不是一蹶不振的颓丧模样。顾大人的儿子也很好,看起来活蹦乱跳,也许长大之后会比顾大人更有出息。

转身背对了顾大人的方向,无心沿着马路向前走去。阳光暖融融的洒了他一头一脸,在金黄色的幻觉之中,他看到年轻的顾大人在小四合院里抽烟望天,月牙则是系着围裙走出厨房,没说话,只对他粲然一笑。

面颊绯红,眼神明亮。她笑得真美,是他记忆中一朵不凋零的花。

\begin{quote}
作者有话要说:
\end{quote}

\begin{quote}
无心和月牙、顾大人的故事,到此就结束了。
\end{quote}

\begin{quote}
接下来我打算休息几天。几天之后,我或许是继续再写一个无心的故事;也或许是完结本文,另开一个新坑。
\end{quote}

\begin{quote}
感谢大家对本文的喜爱与支持,非常感谢O(∩\_∩)O\textasciitilde{}
\end{quote}

\part{抗战时期}

\chapter{设法过冬}

一九四三年秋,上海。

无心在一座无名荒山里度过了整个夏季,因为荒山里人少食多。在长达三个月的时间里,他吃了很多田鼠与蝙蝠,唯一一次遇到不幸,是睡觉的时候被野猪啃了一口。

夏季结束之后,山里的天气渐渐变得不适宜人居,于是他拎着一只帆布旅行袋下了山。有车坐车,有船坐船,他糊里糊涂的到了上海。抗日战争打了六年,战况很不分明,到处都不太平,倒是大都会里更安全。在一间小小的公寓里面,无心找到了容身之处。

一套公寓共有三间房屋,分别出租给了三位落魄的单身汉。一位是个小犹太,没有国籍;一位是个老白俄,没有祖国;无心作为第三位,没有财产。

去年他也曾经挣到过一大笔款子,可是他的人生无边无际,简直无法计划经营,所以采取了今朝有酒今朝醉的活法。如今将仅有的一点余钱交到房东手里,他拿着钥匙进了自己的小房间。一丝不苟的关上房门,他慢慢坐在吱嘎作响的铁架子床上,终于是一无所有了。

房里有个小洋炉子,炉膛里面挺干净,显然是三季没用过了,就等着入冬。无心虽然在山里混了许久,但是并未和现实社会脱节。战事日益激烈,煤炭一天一个价钱,凭着他的资本,连饭都吃不上,怎会有钱买煤?

无心一想起自己的衣食住行,就恨不得钻进地下,效仿蟒蛇冬眠。一动不动的坐在床上,他没有呼吸也没有表情,甚至心中都没有心事。怔怔的望着前方白墙,他百无聊赖的消耗着无尽时光。

木雕泥塑似的从下午坐到翌日晚上,最后还是难耐的饥饿催动了他。他懒洋洋的站起身,心想单是坐着也不成,还是得行动,还是得设法过冬。

摸黑走过去打开电灯,他把一只手举到了小灯泡前。长久的忍饥挨饿让他消瘦了,然而皮肉并未干枯松懈,而是渐渐硬化,似乎要与骨骼融为一体。在灯光下,他单薄的手掌呈现出了蜡质的半透明。缓缓的把另一只手也抬起来,他往墙壁上投了个手影。影子大鹏展翅,是只雄鹰。自得其乐的笑了一下,他又双手合作,映出了一只模模糊糊的狗头。

然后把手伸进怀中,他摸出了一张纸符。轻轻一拍电灯开关,他在骤然降临的黑暗中捏住纸符两端,``嚓''的一声撕成两半。一股子寒气随着破裂声音窜上他的鼻端,他的小喽啰在黑暗中幻化出了影子。

小喽啰看起来只有八九岁大,做着白衬衫背带裤的小学生打扮,衬衫很白,所以显得胸前一滩鲜血很红,一侧的耳朵脖子也是血肉模糊,永不愈合。

他叫小健,放学的路上不听话,跑到大马路上跳舞给保姆看,结果一辆电车刹车不及,当场把他碾死。大千世界,无奇不有。他也算是一奇,死后竟成了个漂泊无依的小鬼,并且结结实实魂魄不散。作恶的本事他没有;恶作剧的主意却是层出不穷。一个礼拜之前,他竭尽全力的搬运了一点火苗,想要去吓无心一跳,结果反被无心当成试验品练了手。无心花了十年时间学画符,成绩相当之差,但还是把他封在了一张纸符里。

七天之中,无心忙着找房安身,只能忙里偷闲的偶尔放他出来,当他是个小朋友。小健很不愿意被他关押,可还是立刻就认他做了大哥,因为无心看得见他,能和他说话。自从他被电车轮子碾过之后,已经连着两年没人理睬他了。

将一只血迹斑斑的小手拍向无心的大腿,小健仰起头笑嘻嘻:``大哥哥,你有房子住了?''

小手只是一个凄惨的影子,还停留在横死时的模样。畅通无阻的掠过了无心的身体,只留下一抹似有似无的寒意。

无心转身走到了小窗户前,推开窗扇探出脑袋。窗下是一条繁华的小街,油炸臭豆腐的味道一直向上冲到三楼,冲进了他的鼻端。

小街对面矗立着一座巍峨的大厦,从无心的角度望出去,可以看到无数灯火通明的后阳台。大厦里面也是公寓房子,不过价值极高,非得阔人才有资本入住。有女仆站在阳台里面淘米择菜,也有老爷少爷坐在阳台上读报喝茶。无心嗅着空气中似有似无的饭香,忽然起了劫富济贫的心思。

当然,凭着他的本领,去打劫肯定是不成。扭头看了看飘在自己肩上的小健,他心中像开水冒泡似的,咕嘟咕嘟的起了坏主意。弯腰从墙角捡起前任租客留下的空酒瓶,他把酒瓶横放在窗台上一转。酒瓶原地转过几圈之后,细长的瓶嘴向窗外定了方向。无心顺着瓶嘴一瞧,正看到了一面紧挨着后阳台的大玻璃窗,窗子没有拉拢窗帘,可见里面灯光辉煌,正是一户很富足的人家。

无心点了点头,心想:``就是它吧!''

与此同时,对面楼中享受着辉煌灯光的马家姐弟,莫名的一起打了个冷战。

马家姐弟是一对龙凤胎,当初他们的母亲怀孕之时,有经验的老妈妈看了她的形容举止,都认定腹中该是一对双生女。不料其中一位比较狡猾,居然在胎里男扮女装。马老爷偶然灵感发作,提前为女儿们拟出了一对野心勃勃的名字。及至孩子出世,真相大白,他一时失落,索性将错就错;于是女婴理直气壮,大名叫做赛维,是要赛过英国女王维多利亚;男婴含羞带愧,大名叫做胜伊,是要胜过英国女王伊利莎白。

马家在北京城中也算大户,成员十分复杂。赛维和胜伊因为是同胞的姐弟,所以在大家庭中分外亲近。时光易逝,转眼间他们进入了青春发育的时期,虽然生活优渥、营养充足,但是统一消瘦的如同野狗一般。赛维升入比利时女中,成绩介于平凡与糟糕之间,唯一的事业是舞动着两条细胳膊打排球,没有男朋友,只有女朋友。而胜伊尽管体态几乎类似豆芽,却有一颗早熟又骚动的心灵,常年在各大女校门口徘徊。可惜凭着他小鸡崽子似的风采,根本不能打动少女的芳心。以至于他在女校周边踏破铁鞋,不但一点罗曼司都不曾发生,反倒落下了个不甚光彩的外号,人称马浪蹄子。

这样一对无人问津的姐弟,浑浑噩噩的混到中学毕业。从此无所事事,越发游手好闲。在家里混了一年半载,他们合谋向父亲敲了一大笔钱,以探望姑母为名离开北京,跑来了上海。

此刻坐在吊灯下的羊毛地毯上,赛维正在和胜伊算账。两人在上海肆无忌惮的挥霍了一阵子,如今闹起了经济危机。赛维自认为比胜伊更有头脑,于是想要和他分家,从此各花各的,谁先空了手,谁就回北京去。反正公寓房子是租了半年整,足够他们住了。

赛维剪着齐耳的短发,头发先前是烫过的,剪过之后还可以看到焦黄的发梢。穿着长裤盘腿而坐,当着自家兄弟,她大模大样的低头数钱。马家的孩子说起来是成长在锦绣丛中,其实一个个见钱眼开,所受竞争的激烈程度,大概一般的孤儿院也望尘莫及。双目炯炯有神的盯着钞票,她嘴里一五一十的念念有词;胜伊伸着脖子,睁大眼睛去看她快速捻动的手指。

一时数清了数目,赛维俯身拿起铅笔,在白纸簿子上记下了一笔。记完之后她叹了口气:``娘在信里说,爸爸上个月给老四买了一件银狐斗篷。''

老四是指马家的四小姐,和他们不是一个娘,并且十年如一日的为敌。马老爷给四女儿花大钱,赛维和胜伊都嫉妒得眼红,并且全忘了自己也曾向父亲要过巨款,否则怎么可能如此舒适的跑来上海过生活?

赛维把钞票分成两部分,想要继续说话,不料在她开口之前,头顶的吊灯忽然一闪。两人一起抬了头,就听上方响起了嘶嘶啦啦的电流声音。而灯光稳定了不过几秒钟,随着声音又开始闪烁了。

赛维和胜伊全都没有生活的常识,不知道吊灯是犯了什么毛病,扬着脑袋就只是看。结果在短暂的黑暗之中,他们一起瞥到了屋角的小小人影!

猛然扭头望过去,随着电灯恢复明亮,人影却又消失无踪。赛维攥着一沓子钞票,张着嘴转向了胜伊。胜伊伸长了他的细脖子,一双黑眼睛睁得又圆又大:``姐,我们是不是\ldots{}\ldots{}看见了什么?''

赛维向角落中又看一眼,角落空空荡荡,干干净净。

抬手揉了揉眼睛,她对胜伊问道:``我们眼花了?''

然后两人一起点头,承认自己的确是眼花。赛维恋恋不舍的攥着钞票,盘算着想要从胜伊的份里克扣一些。胜伊则是向她伸出了手:``姐,钱——''

话音未落,吊灯骤然全灭!

胜伊的手停在半路,同时就觉头顶寒气一闪。伴着电流的噪音,一圈灯泡此起彼伏的亮了又灭,灭了又亮。每当黑暗笼罩之时,就会有小孩子的身影在他们的视野边缘掠过。赛维和胜伊惊声尖叫抱作一团,一起趴倒在地。侧过头去面对了沙发四条短腿,他们猛的一抖,就见沙发下面影影绰绰的,现出了一个小孩子的下半张脸——尖尖的下巴,稚嫩的脸蛋,可惜一侧面颊血肉模糊,甚至露出了苍白的骨头。柔软的嘴角微微一翘,鬼脸向他们笑了。

赛维和胜伊怔了一瞬,随即发出了惊天动地的怪叫。一只灯泡在叫声中自动爆裂,``啪''的一声,碎玻璃渣四散飞溅,全落在了两个人的短头发上。

午夜时分,小健穿过玻璃窗子飘回了家。无心没有睡,正蹲在地上整理他的招牌幌子。小健围着他转了一圈,得意洋洋的开口笑道:``他们家里有一个大哥哥,还有一个大姐姐,现在正哭着呢。''

无心不置可否的一挑眉毛:``嗯。''

小健又道:``他们家里,满地都是钞票。''

无心抬头看着小健,笑了一下。

小健落在了他的头顶上:``大哥哥,我看你不大喜欢我。''

无心终于出了声音:``你要是个人,我就喜欢你了。''

他把破旧的布幌子折叠起来,继续说道:``我很久都没有和人交过朋友了,真想找个活人说说话;不说话,让我摸一下也好。等我弄到了钱,我想养一条狗。小健,你要黑狗还是白狗?''

小健听了他的实话,心里有一点难过,低声说道:``花狗。''

无心一本正经的点了点头:``好,等我买够了粮食和煤,就养一条小花狗。''

\chapter{阴谋诡计}

无心起了个大早,洗漱过后穿戴整齐。房内墙上粘着一面缺了角的玻璃镜,他对着镜子左照右照。阳光还没有照进他的小房间,所以小健飘在镜子前,也想跟着他一起照一照。然而他看了半天,镜中就只有一个无心。

他很亲昵的抱住了无心的大腿,童言无忌:``大哥哥,你看起来像只妖怪。''

无心如今饿得皮肤蜡白,双目凹陷,的确是带了一点阴森森的妖气。咬着手指向下望着小健,他恨不能把自己吃掉。小健仰脸迎着无心的目光,随着阳光的强烈,他的影子越来越淡——毕竟只是一个小鬼,虽然莫名其妙的有点力量,但是力量终归有限。

无心对他实在是没什么感情,所以不假思索的尽说实话:``唉,你要是活的该多好。如果你是活的,我可以做你的父亲。''

小健也不是自愿去死的,所以听了他的话,幼小心灵一阵悲凉。而无心很惋惜的俯视着他,两道眉毛蹙起来,是真心实意的在遗憾。

在把小健审视成一团灰扑扑的悲哀光团之后,无心夹起他那卷成一卷的布幌子,没心没肺的出门走了。

他所居的公寓位于三楼,夹着幌子刚刚下到二楼,无心就觉得身上寒冷,几乎有些不能忍耐。一转身返了回去,他决定换身衣裳。身上的一件僧袍,穿过若干年了,飘飘然的薄如蝉翼,唯一的作用是遮羞。平日扮成和尚模样,比较适宜他求生存;不过今天他目的明确,似乎暂且抛弃僧人身份也没关系。

掏出钥匙开了房门,他在旅行袋里掏出一身半新不旧的裤褂换了上,顺便还在褂子口袋里摸出了几张零碎钞票。再次迈步出了门,他一鼓作气的跑下楼,在开始他的大事业之前,先在一处小摊子前买了一串臭豆腐干。臭豆腐干上面淋淋沥沥的涂了许多辣椒酱,无心一边走一边小心翼翼的吃,染得嘴唇舌头都鲜红。末了穿过小街绕过大厦,他在大厦前门所对的马路边上坐下了。蔑绳上面还穿着两块臭豆腐干,他不忙着吃,先把自己那一面没有骨头的幌子摊在了身边地上,表明自己是个算命运看风水兼降妖除魔的全才。

然后他继续吃臭豆腐干,吃得路人掩鼻子过。而马家姐弟忍着臭气,不动声色的围着他转了一圈,末了远远的停在了他的身后。

赛维与胜伊都是一宿未睡,脸上统一的生出了几个红疙瘩,两人本来就瘦,平日举止潇洒,还可算作弱柳扶风;如今一切风度全没有了,他们端着肩膀抻着脖子,像一对营养不良的乌龟,惶惶然的盯着无心的背影瞧。无心穿着单衣单裤,也是瘦极了,隔着一层衣裳,可以看到线条清晰的肩胛骨,骨头凸出来,像是一对翅膀的遗迹。

胜伊用胳膊肘一杵赛维,触到了赛维的肋骨:``姐,你看见没有?他说自己会捉鬼。''

赛维潦草的裹了一件薄薄的皮夹克,抬手摸了摸脸上的痘子:``看是看见了,不过他怎么一副惨相,像个要饭的花子?''

胜伊轻声说道:``高人都是深藏不露的。''

赛维不以为然的摇头,感觉对方太年轻了,就算深藏不露,也得有的藏才行。依着她的主意,她打算去向姑母求援。姑母是个老太太,必定能有主意;不过老太太又太热心了,一旦招惹上,就不能轻易甩脱,他们十七八岁,耐不下性子和老太太打交道。

胜伊又问:``姐,到底要不要他?不要就走吧,我快被臭豆腐熏死了。''

赛维想走,可是在她迈步之前,远方的无心忽然回头望向了他们。他的面孔很白,眉眼很黑,嘴唇很红,脸上还蹭了一抹辣椒酱。面无表情的咽下最后一口臭豆腐干,他背对着初升的朝阳与喧嚣的大路,向马家姐弟招了招手。

胜伊是个有意见没主意的人,一胳膊肘又杵向了赛维的肋下:``姐,你看,他叫我们过去呢!''

赛维不能确定,迎着无心的目光,她抬手一指自己。无心点了点头,随即向她微笑了。

无心今天收拾得挺干净,虽然脸上有辣椒酱,但依然可以归到美男子一类。赛维见他的笑容颇为动人,两只脚便闹了自治,自动的开始前进。胜伊连忙跟了上,口中一路嘀嘀咕咕:``我就说试试他,你还不听。你看他就在楼下坐着,不试白不试。如果他是个混饭吃的骗子,随便花两个钱把他打发了就是,也不麻烦。对不对?你就非得去找姑母,姑母是能轻易找的吗?老太太一来精神,谁能打发得了?''

赛维根本没理他。迈着细腿一路快走,像只急性子的鹭鸶,三步两步就停在了无心面前。胜伊追逐而来,和赛维成夹攻之势,把无心围在了中间。无心坐井观天似的抬起了头,直接说道:``我有句话想对二位讲,可又不知当讲不当讲。''

赛维舔了舔干燥的嘴唇,正在酝酿答案,不料胜伊开口就道:``讲吧!我们听着呢!''

无心微笑说道:``我看二位印堂发黑、一脸晦气,是个噩运当头的表现。''

胜伊一拍大腿:``哎呀,噩极了呀!''然后他抬头去看赛维:``姐,姐,你听见没?我就说他靠谱,你还不信。''

赛维平时难得能遇到美男子,即便美男子是个坐路边吃臭豆腐干的疑似叫花子,也让她生出了一点小小的心思,极力想要显出一点内秀。然而胜伊聒噪不止,让她憋了满腔的内涵不得释放。心烦意乱的扫了胜伊一眼,她不置可否的继续沉默。

胜伊蹲到了无心的面前,兴致勃勃的继续问:``那你再瞧瞧,我们是走了什么噩运?''

无心几乎从他们身上嗅到了小健的味道,所以胸有成竹的笑道:``大概是府上不干净吧?''

胜伊几乎大惊失色了,抬手去拍赛维的小腿:``姐,姐,真神了啊!''然后他又问无心:``你脏不脏?要是没有虱子跳蚤的话,我就带你到我们家里去一趟。你把鬼给我们除了,我们必定重谢你!''

无心卷起布幌子夹到腋下,然后站起来对着马家姐弟说道:``我不脏,绝对没有虱子跳蚤。''

为了拉住两位主顾,他还特地对着胜伊拉了拉衣袖扯了扯衣领,让他看自己的手臂和脖子。胜伊当即询问赛维:``姐,他算卫生吧?''

赛维被胜伊吵得头疼,所以不假思索的答道:``嗯,还挺白的。''

话一出口,她后了悔,因为感觉自己格调太低。半晌没说话,甫一开口,就是失言。

无心随着马家姐弟走入大厦,乘坐电梯上了六层。公寓房子里面有个女仆,每天早来早走,负责洒扫烹饪,只在后阳台和厨房徘徊,等闲不肯轻易露面。光天化日之下,自然不会闹鬼;所以三言两语的交谈过后,无心应邀在客厅坐下,等待天黑。

吊灯的碎灯泡被卸下来了,沙发上面的碎玻璃渣也被清扫干净了,羊毛地毯一时不好办,索性撤了下去。胜伊把无心当成了救世主,手舞足蹈的向他讲述自己的惊魂夜,无心喝着热橘子水倾听。不知道胜伊早起吃了什么,口鼻中热烘烘的呼出甜酸气;赛维坐在一旁,每隔一分钟就换一个姿势,也是一刻都不安静。无心处在包围之中,感觉很快乐,于是就一直笑眯眯,自称是个孤独的和尚,因为寺庙毁于战火,所以才一路流浪漂泊。

赛维对于他的身份没有兴趣,因为无论他是僧人还是神棍,和她都不是一个阶级,牵扯不到姻缘。不过毕竟他是个男子,自己是个姑娘;人总有个要好的心思,她自知不很美,所以格外想要利用智慧一鸣惊人,给对方留下个惊鸿一瞥的印象。问题是她的智慧也很有限,真是要了命了!

无心在马家公寓里混过了大半天,其间吃了一顿午饭一顿晚饭,并且还有精致的下午茶可以享用。天不黑,鬼不来,于是三个人在大玻璃窗前席地而坐,打起了小扑克。打着打着,赛维见无心总是输,就耍了一点小心计,故意藏牌调牌,想要让他赢上几局,不料手法太差,刚一行动就败露了,被胜伊捉了个正着。

赛维登时恼羞成怒,学着马老爷的口吻,老气横秋的骂道:``混账东西,竟敢犯上!''

胜伊把扑克牌往地上一扣:``你也无非是比我年长了一分多钟而已,算什么上!''

赛维见他胆敢抵抗,登时露出本相:``好你个马浪蹄子,还敢和我嘴硬!''

胜伊一听``马浪蹄子''四个字,登时被她戳中了内心痛处,本是盘腿坐着的,此刻双手撑地蹲了起来,跃跃欲试的想和赛维斗殴一场。

他们姐弟都不是省油的灯,从小又最亲近,免不得相爱相杀,时常对打,但是打过就算,绝不结仇。无心不了解内情,没想到偌大的人了还会动手,就想去劝解一番。而赛维沉默了将近一天,此刻也是憋得够呛。跪起来脱了身上的皮夹克,她露出了里面的粉衬衫。有条不紊的解开袖扣向上挽起,她露出了细细的手腕子。

两张相似面孔对视了,虎视眈眈的全不肯退让。无心正要挤上前去把他二人隔开,不料就在他将动未动之际,一阵寒风忽然掠过了三人的头顶。原来太阳刚刚沉下了地平线,虽然天边还有些许微光,但是阳气退散阴气上升,已经算是入了夜。

吊灯自从爆掉一只灯泡之后,就没敢再开,客厅全凭着门旁一盏壁灯照亮。壁灯本是个装饰品,亮度十分有限。无心顺着寒风的方向扭过了头,就见小健影影绰绰的附在灯旁,正在对着自己做鬼脸。

在马家姐弟互相对峙的空当里,无心对着小健一挤眼睛。小健当即会意,摇头摆尾的飘过了壁灯罩子。灯光骤然一闪,随即彻底熄灭。

客厅里面安静了一瞬。小健很欢喜的经过马家姐弟,若隐若现的躲进了曳地窗帘后面。随之而起的是两声嚎叫,马家姐弟自动化干戈为玉帛,像两头暴烈的小马似的,一起扑进了无心的怀里。无心下意识的张开双臂,猝不及防的拥抱了他们。

两人都是瘦,细条条的不够他一抱。两个脑袋拱在他的胸前,散发着隔夜的生发油味、淡香水味、雪花膏味。三合一的香味混合了肉体的汗气和热量,成分十分复杂,可因为是年轻人,别有一种洁净新鲜,所以复杂归复杂,并不让无心感到污秽。很久没有结结实实的抱过谁了,无心的双臂微微加了力气,感觉自己像是中了奖券。

``不要怕!''他搂着怀里一对魂飞魄散的姐弟:``我看到它了!''

然后他适可而止的松了手,起身过去一抖窗帘。小健探究似的从上方垂下了一个脑袋。赛维与胜伊看得清清楚楚,登时又嚎一声。与此同时,无心已经向上使了眼色。小健会意,一转身就穿过玻璃窗,消失在了夜空中。

无心转向瘫在地上的两姐弟,背过双手正色说道:``它逃了!''

赛维打着结巴问道:``逃逃逃了?还还回来吗?''

无心摇了摇头:``只要有我在,它就不敢回来!''

胜伊也开了口:``要要要是你不不不在呢?''

无心想了想,随即答道:``要不然,你们搬家吧!''

赛维和胜伊异口同声的说道:``没没没钱哪!''

无心叹息一声:``哎呀,小鬼最是难缠,想要把它消灭,不好办啊!''

赛维和胜伊听他口风活动,分明是个漫天要价的意思,反倒放下了心,预备和他认认真真的讨价还价。不料未等他们开口,隔壁的电话忽然铃声大作,吓得他们一起打了个激灵。

铃声响得很急,接二连三的不停歇。赛维和胜伊爬了起来,想要去接电话,可是又没胆子。面面相觑的僵持了片刻,最后还是赛维跑去隔壁,抄起听筒``喂''了一声。胜伊竖着耳朵,却又并没听到下文。

至多是过了一分钟,赛维失魂落魄的走了出来。扶着墙壁站定了,她轻声说道:``胜伊,是大哥从天津打来的长途电话。''

胜伊莫名其妙:``他又有什么事?''

赛维答道:``娘没了。''

胜伊眨巴眨巴眼睛,仿佛是没听懂。于是赛维把话重复了一遍:``他说,娘生了急病,今早没了。''

她口中的``娘'',指的是他们的亲生母亲,马家二姨太。作为一名母亲,二姨太乏善可陈,并不能成为儿女眼中的榜样;可母亲毕竟是母亲,所以胜伊一听,也僵在了当地。

``不可能。''他气息微弱的说:``娘的身体一直都好,怎么会忽然病死?不可能。''

然后两人抬起袖子一抹眼睛,一起嘤嘤的哭了。

\chapter{遗信}

赛维和胜伊并肩跪坐在地板上,双手捧着脸低头啜泣。两人上身都是衬衫打扮,显出了相似的薄肩膀和细脖子,细脖子挑着个圆脑袋,挑不动了似的一颤一颤。

无心盘腿坐在对面,不知道如何宽慰才好,身上也没有手帕一类,只有两只巴掌,可是往谁的脸上抹拭都不合适。及至姐弟二人整齐划一的吸着鼻子抬起头了,他才抓住机会问道:``哪里有毛巾?''

赛维和胜伊一起伸手指了个方向。无心走过去推开门,就见内中四壁贴着白瓷砖,正是一间现代化的卫生间。走进去扯下两条柔软毛巾,小健忽然从门缝里伸出了脑袋,对着无心一歪头,他把血淋淋的半边脖子露了出来:``他们怎么了?''

无心对他一挥手,把声音压到了最低:``今天夜里不要闹了,他们刚刚死了娘。''

小健了然的一点头,把脑袋缩回了门缝。

赛维和胜伊都不说话,捧着毛巾靠着墙壁,四条细腿乱七八糟的伸长了,让无心觉得身边到处都是腿。

他们哭一阵,歇一阵,后来还互相依偎着打了个盹儿。真正清醒之时,已是凌晨时分。赛维强撑着起身去了厨房,从冰箱里找出一瓶浓浓的橘子汁。忽然回头望向身后,她朦朦胧胧的看到了无心。

无心一脚门里一脚门外,很认真的问她:``要干什么?我帮你。''

赛维的各方面都是高不成低不就,又是一直在女校里面读书,异性的朋友几乎没有。无心对她有了一点好意,她立刻就感觉出了。把冰凉的玻璃瓶子放在菜台上,她极力想要把红肿的眼睛睁大,鼻音浓重的答道:``我想兑一点热橘子水喝。''

无心把厨房翻了个底朝天,终于找到了暖水壶。兑了三玻璃杯热气腾腾的橘子水,他用托盘端着往客厅里走。赛维哽咽着跟在他的身边,忽然把阶级问题忘记了,只感觉他很好。

三人还是围坐在了地上,一人捧着一杯滚热的橘子水。胜伊无声的啜饮了几口,元气略略恢复了一些。望着窗外天边泛出的鱼肚白,他哑着嗓子问道:``姐,大哥还在天津吗?''

赛维点了点头:``他说他马上就回北京。爸爸上个月去了日本,家里没人主事。''

胜伊眨巴着干涩的眼睛:``等到天大亮了,我们直接去火车站吧!''

然后他转向无心:``谢谢你,陪了我们一夜。''

无心摇头笑了笑,知道自己的生财之路断绝了,不过也没什么可抱怨的,和对方的丧母之痛相比,自己的饥寒虽然紧迫,但是也算不得太大的问题。

赛维忽然开了口:``无心师父,你若是愿意的话,我们买票的时候可以带你一张。''

胜伊惊讶的扭头看她,而她自顾自的继续说道:``反正你在上海也是漂泊无依,如果到了北京,兴许更好找活路呢。''

随即她又转向了胜伊:``现在南北都一样。就算上海更好玩,可没有钱不也是白搭?''

胜伊没见过赛维对哪个男人特别关怀过,如今可是破天荒头一遭。但是脑筋转了一圈,他又感觉不可能。虽然他们姐弟俩是互相的低看,但是他想赛维再怎么没人要,也不至于爱上一个穷困潦倒的和尚兼神棍。

无心只是微笑,心中有些迟疑。要说走,当然容易,至多是浪费了两个月的房租罢了;可是真去北方吗?真去北方大概也不错,上次到北京天津还是在十年前,后来一路向南,想再回去,然而炮火连天,就难了。

外面的大世界渐渐苏醒,楼下的大街上开始有吃食担子络绎经过。赛维喝过橘子水后,打算去收拾行装北上。不料她刚刚扶墙起身,就听房门被人咚咚敲响了。

一天来一趟的女仆是有钥匙的,当然不必敲门。赛维和胜伊又对视一眼,随即走去开了房门。原来敲门人是大厦里的杂役,送来了一封刚刚到达的加急快信。赛维接信关门,一边低头看信封一边转过了身,走过几步之后,忽然停了。

苍白着一张脸抬起头,她目光散乱的小声说道:``奇怪。''

胜伊仰脸看她:``怎么了?谁来的信?''

赛维站在原地,手有点抖:``是\ldots{}\ldots{}是娘。''

胜伊一听,也愣了。原来马家二姨太的学问十分有限,大字认不了一箩筐,连唱本都看不明白,一辈子没有正经提过笔,一百年和人通一次信,向来是劳驾账房里的老先生代笔。所以姑且不提信中内容,单说写信行为的本身,便已是罕见之极。再看信封上的字迹,歪歪扭扭缺胳膊少腿,肯定不是老先生的作品,倒像是二姨太的亲笔——马家姐弟也曾偶然见过母亲的账本,上面一笔一笔记着的乱账,就和信封上的字迹一模一样,拙劣得可笑。

赛维撕了封口,从里面抽出一张信笺展开来,就见上面笔画漆黑,不是用毛笔写的,也不是用钢笔写的。用指尖蹭了一下,蹭出一抹子黑色,竟然是画眉用的眉笔。二姨太没有写过亲笔信,生平第一次写,里面全是前言不搭后语的白话。姐弟二人凑上去一起读了一遍,末了面面相觑的抬起了头,互相大眼瞪小眼。

二姨太在信里做了两桩抱怨,一是大少爷和老爷吵得很凶,险些又动了枪;二是她最近闹了奇异的心病,夜里一闭眼就是噩梦连连。请了个明白人解了解梦,结果都是很不好的兆头。最后她做了嘱咐,让一对儿女先不要急着回家,因为自己的心脏总是怦怦乱跳,想要静养,可是家里太不安静,如果可能的话,她还想去上海和儿女一起过秋天呢。

三件事情,让二姨太写了个颠三倒四;末尾她又强调了一句:``不要回家,钱不够用,娘贴补给你们。''

拿着信坐回地板,马家姐弟全都心神不定的傻了眼——第一,二姨太居然亲自给他们写信;第二,二姨太居然会闹睡眠问题;第三,二姨太居然没有催促他们回家;第四,二姨太居然主动要给他们钱。

末了,是胜伊先开了口:``大哥又回家了?''

赛维看了看信,信上落款连个日期都没有写,只能从信封邮戳上推测发信日期:``大概是在爸爸出国前回去的。''

胜伊咬牙骂道:``死瘸子,到了哪里都是鸡犬不宁!''

赛维立刻伸手拍了他一下,似乎是怪他当着无心口无遮拦。及至把胜伊拍哑巴了,她想了想,反倒忍不住作了解释:``我们的大哥,腿脚有些不方便。爸爸年轻的时候脾气暴躁,有天喝醉了发酒疯,开枪打伤了他。''

无心了然的点了点头,没说什么。

赛维又道:``我们娘\ldots{}\ldots{}身体素来都是很康健的。''

此言非虚,二姨太基本可以算作心宽体胖,人生的唯一事业是取悦马老爷,至高成就则是一举产下了一对活泼泼的龙凤胎。生下一双儿女之后,她自觉地位有了保障,绝不会受到驱逐和冷遇了,便放心大胆的开始发福,终日唯一的运动就是打麻将牌。横竖马老爷也无意再临幸她了,她索性玩完了吃,吃完了睡,由于胖,所以张着嘴打着酣,一旦入睡,雷打不动。儿女和私房钱是她的护身符,她很不赞成两个孩子一起远行,若是她说话算话而一双儿女又肯听话,她定然要把赛维和胜伊关在家里。两个孩子关不住,手里的体己可是关得住的。二姨太很是有点小积蓄,永远不动,因为在大家庭里没有安全感,一旦马老爷完了,马家散了,她还可以买所小房,继续过她胖太太的好日子。

胜伊拿过信笺又读一遍,读过之后低声咕哝道:``是不是人之将死其言也善?娘怎么像转了性似的?''

赛维立刻瞪了他:``别胡说八道!难道娘是早知道自己要走吗?娘是担心我们——''

胜伊止住了她后半句话:``我说的转性,是指娘亲笔给我们写信。你看信里的话,都是家里确实发生的事情,没什么可瞒人的嘛!再说娘的性子你还不知道?连天津她都感觉是远在天涯海角,她会无端的来上海?她舍得她的小房小院小牌桌?''

赛维眨巴眨巴眼睛,听了胜伊的话,她不知怎的,脊梁骨忽然要冒凉气。小鬼神秘不可知,很可怕;信上疑点众多,也透出了一点恐怖的意味。扭头再去看胜伊手中的信笺,雪白纸上,笔画黑到刺目。二姨太虽然是个半文盲,可是精通化妆,总不应该用一支眉笔写信。除非\ldots{}\ldots{}

赛维看了无心一眼,见他静静的坐在一旁,像一尊磐石,心里就安定了一点,仿佛他是自己姐弟的保护神。把玻璃杯里余下的一点橘子水喝了,她垂下脑袋思索良久,最后抬头说道:``胜伊,娘是不是心里有话,可是又不知道怎么说,怎么写。于是\ldots{}\ldots{}''

胜伊鼓着两只肿眼泡看她:``什么?''

赛维垂下眼帘,慢慢的答道:``是不是娘有了什么异常的感觉,但是她又没有证据,所以只能在信上写出当时发生的实事?她不让我们回去,是不是因为发现家里要出什么事情?她偷偷的给我们写信,是不是因为有人盯着她,不许她写?眉笔很软的,写过几个字,笔头就磨平了,非得再削尖了才能用。娘就算一时找不到好笔,随便用支描花样子的铅笔头也比它强。娘又不傻,为什么非要磨损眉笔写信?''

胜伊缓缓的点头:``姐,你比我想得周全。''

赛维和胜伊本来打算清早就出发的,可是接了信后,越想越是糊涂,便耽搁在了房内。至于无心,因为并没有受到驱逐,所以厚着脸皮守在姐弟二人身边,晒着太阳听人说话。及至吃过了午饭,胜伊认为单是胡思乱想也没有用,于是打起精神,还是想要去买火车票回家。然而未等他们出发,邮差又送来了今天的第二封信。

信上字迹丑陋,依旧是二姨太的亲笔。赛维撕开封口取出信笺,发现信笺上就只有三个黑字:别回家!

\chapter{大家族}

二姨太是很明确的不让两个孩子回家,可是两个孩子即便及时接到了两封信,又怎能当真依言不回家奔丧?马家从来就不是个祥和的大家庭,于是赛维坐在沙发上思索良久,最后抬头对胜伊说道:``家是一定要回的,否则别说对不起娘,就从礼数上看,也不像话。不过娘虽然不管事,但是脑子一直不糊涂,绝不会无缘无故的写信阻止我们回家。家里兴许是出了什么不为人知的秘密事故,我们出来了几个月,一直没和家里联系,当然也就一无所知。总而言之,回家之后我们找个借口,全住到娘的院里,一旦有了什么变化,两个人总强过一个人。''

胜伊的思想素来没有赛维细致,不过两人从小一起长大,仿佛有所感应似的,一听就点了头。

赛维又转向了站在一旁的无心,嘴唇欲言又止的动了一下。说老实话,她此刻有点心惊肉跳,胜伊也不是个有主意的,她很需要一位帮手。可是和无心也不过刚认识了一天一夜而已,以交情论,似乎还不该和对方太亲近。

她犹犹豫豫的看着无心,胜伊有所知觉,也把目光移向了他。姐弟二人全都是微微的驼着背蹙着眉,一脸可怜相的注视着他。无心迎着二人的目光,同时迟疑着说道:``如果二位用得上我,尽管开口就是。''随即他又笑了一下:``反正我是个无牵无挂的闲人。''

此言一出,马家姐弟一起松了口气。他们是没人可以指望依靠的,如今突然多了个伴,也好。

此刻并不是交通繁忙的季节,不到傍晚,三个人已经进了火车包厢。包厢是大包厢,上下共有四张床。三张床用来睡人,一张床用来放行李。无心只有一个帆布旅行袋,轻飘飘的不算分量。马甲姐弟却是各有一只硕大沉重的皮箱。赛维和胜伊换了素净衣裳,并肩坐在小床上,仰头看着无心爬上爬下安放行李。无心的动作很利落,脸上没有什么表情,纯粹只是在干活。等到把行李全安置好了,他又拎起暖壶,走去车厢尽头打热水。

入夜之后,三个人各就各位的躺好了,无心睡在胜伊上方的空床上。胸前微微的有点凉,是贴身藏着一张纸符,符里封着小健。虽然他说话不大中听,但小健还是不想离开他。宁愿随着他到处走。

包厢里很安静,三个人都是无声无息。赛维侧身躺着,偷眼去看斜上方的无心。无心平平地仰卧在床上,胸膛一起一伏。赛维看惯了胜伊,如今见无心比胜伊处处都大一号,就很感好奇;丧母之痛渐渐淡化了,反正马家就没有过母慈子孝的情况,他们和二姨太已经算是亲密,但是平日母亲不管儿女不听,感情也是深的有限。

``凭着他的穷法,可真是不成。''赛维随着火车的颠簸,一板一眼的思考:``除非学习五姑姑脱离家庭。不过五姑姑养了十年的五姑父,最后五姑父还不是攀上富贵人家跑了?听说五姑姑现在活得很凄惨,所以我还不能学她。''

夜色深重,她双目炯炯的不能闭眼,念头一会儿一变:``能不能托人给他找个小职位呢?五姑父是彻底的浪荡子弟,他和五姑父还不一样。五姑父在家横草不拈竖草不动,他比五姑父勤劳多了。''

随着火车的颠簸和前进,她想得越来越远:``他竟然穷到了穿破袜子的地步。等到了北京,我无论如何都要给他买一身新衣新鞋。''

赛维浮想联翩,忘了时间。对面的胜伊和衣而卧,却是早就睡了。胜伊连着受了几日几夜的精神折磨,如今上方多了一位私人保镖,让他很有安全感,睡得格外踏实。

无心静静的闭着眼睛,不睡装睡。他知道赛维在偷看自己,不过并不动心,不是因为赛维不好,赛维作为一个干干净净顺顺溜溜的大姑娘,没什么不好的。但是,没有可能和他配成一对。

他享受不到做人的好处,却又处处受着人的规矩。对于赛维的窥视,他只有斩截利落的四个字:高攀不起。

旅途通畅,无心和马家姐弟躲在包厢里,似乎也没有做出几场讨论,便进了北京地界。下了火车坐上洋车,他们一路走大街穿小巷,最后钻进了一条大胡同里。马家虽然人多事多,但不是``诗书传家久''的家族,马老爷的父亲在晚年发了家,家业传给马老爷,经过几十年的经营,越发充实扩大。及至日本人来了,马老爷见风使舵,依旧立于不败之地。否则凭着当今世道的艰难,一般的汉奸都未必有资本供着儿女们吃喝玩乐。马家的孩子们也知道父亲有着大汉奸的名声,不过看在钱的面子上,没人敢向马老爷提出异议。唯一敢和马老爷对战的是大少爷,但是大少爷常年住在天津,纵算父子双方斗志昂扬,可是掐架的机会也难找。

赛维带着胜伊领头走,路上还是一派平静。哪知刚一进家门,脸上就显出了哭相。把行李全交给门房里的仆人,他们先对无心使了个眼色,然后嚎啕一声,一路哭天抢地的往后院跑。无心进了院门,正在瞻仰迎面一座洋楼,冷不防听了他们大爆炸似的哭声,几乎吓了一跳。随着二人一路向前小跑,他经过了几重大门,几丛花木,最后进了一处很精致的小院落里。赛维和胜伊一边哭一边四面八方的乱看,口中``娘啊娘啊''的乱叫。一个老妈子从房里迎出来,是二姨太使唤惯了的人,如今见姐弟二人回来了,就垂着泪请他们进房。

赛维和胜伊对母亲的屋子当然是最熟悉,此刻又是怀着心思,所以虽是抽抽搭搭,两只眼睛却不闲着。可是未等他们进入里间卧室,外面忽然有个丫头叫道:``二小姐三少爷,大少爷来了。''

赛维对胜伊一挑眉毛,然后独自转身走了出去。无心还没来得及进房,如今站在门口,就见院角的月亮门外青袍一闪,转出了一位面色苍白的中年男子。

赛维眼泛泪光,倚着门框哭道:``大哥,娘现在停在了哪里?到底是生了什么急病?''

马家大少爷拄着一根黑漆手杖,站稳之后喟叹一声,仿佛对妹妹弟弟也没什么亲爱之情,只言简意赅的答道:``医生做了检查,说是心肌梗死。''

然后他把眼珠转向了赛维身边的无心。无心和他打了个照面,发现大少爷生得浓眉大眼,鼻梁挺拔,身姿也算潇洒,唯一的美中不足,便是鼻尖略略有点鹰钩,给他添了几分阴鸷颜色。抛去年龄不论,单看面貌的话,他显然是比赛维和胜伊都更能漂亮。

``这位是——''大少爷开了口,话说半截就不说了,只对着无心微微一点头。

赛维抢着答道:``他是胜伊在上海结识的好朋友,这一路我们什么都做不成了,全靠他来照顾我们。''

话音落下,胜伊也哭天抹泪的走了出来,鼻音浓重的唤了一声``大哥'',然后呜呜的又开始哭。大少爷似乎是生出了一点同情心,唉声叹气的走上前来,对着无心又一点头,然后伸手说道:``多谢关照,请问先生高姓大名?''

无心和他握了握手,低声答道:``我从小在寺庙里长大,法名是无心二字。''

大少爷答道:``哦\ldots{}\ldots{}无心师父目前还是出家人的身份吗?''

无心微一摇头,笑而不语,不说是,也不说不是。大少爷没有得到明确回答,又不好追问,于是自我介绍道:``敝姓马,马英豪。''

无心依旧是笑,笑得带了一点傻气。

马英豪松了手,让赛维和胜伊去前面楼内的灵堂中去看二姨太,语气温和,不带情绪。又说:``妈一直守在灵堂里。''

所谓``妈''者,乃是马老爷前些年娶进门的正房太太。正房太太比姨太太们还年轻,今年不过三十多岁,当初如果不是娘家败落,也不会嫁给马老爷做填房。家里的孩子没有一个是她生的,可是按照规矩,都得喊她一声妈。马老爷对她不冷不热,她自己活得也是不冷不热。

赛维和胜伊哭丧着脸,要跟马英豪走了,两人临走前回头看了无心一眼,然后又支使老妈子给无心倒茶。

无心不动声色的进了房。等到老妈子奉茶完毕退出去了,他从怀里摸出纸符。扯住纸符一撕两半,他对着虚空中淡淡的影子轻声说道:``去,跟上他们!''

小健亲昵的在他颈间绕了一圈,然后一闪而逝。

不过半晌的工夫,小健回来了,是一团寒冷的光,就附在他的肩膀上。他端着一杯热茶慢慢喝,同时听到小健在自己耳边嘻嘻笑道:``屋子里面好多人,大姐姐和大哥哥换了白袍子,哭得像狗叫一样。床上的胖婆婆好丑喔,头发里面还有根钉。''

\chapter{灵堂}

无心坐在房内,一杯接一杯的喝茶。到了傍晚时分,房门一开,披麻戴孝的胜伊踉跄着走了进来。无心见状,随手拿起一只茶杯,倒了一杯热茶直送到他手里。而他捧着热茶一屁股坐下来,先是长长的吁出一口气,然后哑着嗓子说道:``累死我了。''

未等他话音落下,赛维也东倒西歪的回来了,无心一看桌面,发现两只茶杯都被占用,再看赛维,赛维嘴唇干枯泛白,显然比胜伊更需要茶。

无心素来善待女人超过男人,此刻略一思忖,又见胜伊捧着茶杯无意要喝,便轻轻巧巧的一伸手,从他手中夺了茶杯送向赛维:``节哀顺变,坐下歇歇吧。''

赛维一来很看得上无心,二来并不嫌弃胜伊,所以不假思索的就接了茶杯。靠着桌沿站稳了,她低下头,尖着嘴巴一边吹热气一边啜饮。而胜伊诧异的抬头望向无心:``不是给我的吗?怎么还带往回抢的?''

然后他又转向了赛维:``姐,你不要领他的情。''

赛维充耳不闻,扯着乌鸦似的嗓门让老妈子预备晚饭。

马宅有个大厨房,总供合家的饮食,从早到晚不断火。老妈子见二小姐三少爷是要留在二姨太的院里了,以为他们是有缅怀之意,心里倒是很乐意。而赛维和胜伊在进中学之后就平分了一处大院子,院中也有两个小丫头负责杂务。此刻小丫头们就和老妈子合力,用大食盒从厨房运了饭菜回来。

胜伊还记着一杯茶的仇,在饭桌上瞄着无心:``你到底还是不是和尚了?又向我姐献殷勤,又吃肉!''

说完这话,他后脖颈上凉了一下。他一激灵,当即扭头打出一个大喷嚏,险些把饭粒呛进气管。无心连忙伸手为他拍了拍后背,又对着他的上方轻声说道:``别闹。''

小健蹲在胜伊的头顶上,很不忿的分争道:``他挤兑你呢!''

无心笑了:``闹着玩,不算挤兑。你自己玩去,离他远点。阴阳相克,当心伤了他也害了你。''

然后他好脾气的挥了挥手:``去吧去吧,听我的话。''

小健喜欢他,总预备着向他献媚,不料他永远不领情,气得一阵风似的就冲进了墙壁里。而赛维咬着筷子尖,直着眼睛去看无心,同时含糊问道:``你在和谁说话?''

无心答道:``小淘气鬼,已经走了。''

胜伊放下碗筷,当即抱着肩膀缩成一团,扬着脑袋四处乱看。而赛维心中一动,随即又问:``无心师父,你既然能够看见小鬼,可见人的确是有灵魂的。我们的娘\ldots{}\ldots{}''

未等她把话说完,无心直接摇了头:``屋子里很干净,我没有看到令堂。''

胜伊拉着椅子,挪到了无心身边坐住。而赛维又道:``屋子里没有,去灵堂看一看呢?''

无心点了点头:``好。''

胜伊开了口:``可是姐,什么时候去看呀?''

赛维答道:``一会儿就去!我们自己的娘,我们想怎么看就怎么看,谁管得着?哪个敢嚼舌头,我一巴掌拍死他!''

胜伊把自己的碗筷也挪到面前了,又对无心说道:``我姐不是吹牛。原来在女校排球队里,她有个外号,叫做奔雷手,一巴掌能拍死一条哈巴狗。''

赛维继续装没听见。弟弟的言谈举止全都不得人心,专挑她的老底来揭。

无心笑了笑,也不好把话接下去。

三个人吃饱喝足,赛维和胜伊虽然下午在灵堂里百般做作,累了个死去活来,但是年纪轻,吃点喝点便恢复了元气。赛维嫌无心穿戴寒碜,带他去了一趟胜伊的房间。胜伊是位爱美的青年,新衣无数,可惜都不合无心的尺寸,只有一条带有背带的帆布工人裤,是胜伊图新鲜置办的,宽大无匹,可以装进两个胜伊,或者一个半无心。赛维让他穿,他就穿,虽然从来没穿过。

他在房内换衣服,房外的胜伊悄声说道:``姐,他好像很听我们的话。我们把他留下来吧!''

赛维故意反问:``留他干什么?''

胜伊答道:``让他陪着我们、保护我们啊!反正他一无所有,我们养活着他,他还有什么不满意的?''

赛维一听他是要把无心当狗养,登时心里生了气,想要找出辛辣词语教训教训他,可是``浪蹄子''三字还未出口,前面房门一开,无心笑模笑样的走出来了。结实粗糙的工人裤穿在他身上,倒是很有一点款式,上身背带下面是胜伊的旧衬衫,衬衫的肩膀有点窄,所以领口的纽扣就没有系,露出一小块干干净净的白皮肤。

赛维看着他,没有说话,大脑则变成了一台转疯了的留声机。先想``他比我白'',再想``怎样才能让爸爸给他找个差事'',接着想``或许做生意也不错'',最后想``结婚之后一定要离开北京,否则会被他们嘲笑''。

及至胜伊一胳膊肘杵上她的肋骨,她已经想到了如何贴钱成家。找个流浪汉似的丈夫,当然不是光彩事情,所以免不了还要和家里人进行战斗。正在措辞骂人之时,她忽然听到了胜伊的声音:``姐,你发什么呆呢?走不走哇?''

赛维意犹未尽的终止了幻想,其实根本没有要和无心结婚的打算,不过不知怎的,她时常会失控似的对着无心浮想联翩。

马宅房屋众多,灵堂就设在了宅子前部的一座空楼里。二姨太毕竟是个姨太太,虽然有了一点年纪,还有一对儿女可以撑腰,但姨太太一辈子都是姨太太,一对儿女也还是未长大的吃货,故而丧事不会如何隆重。

按照规矩,三天入殓,所以二姨太已经进了棺材,不过因为亲生儿女还未见最后一面,所以棺盖倾斜着留了缝隙,是等赛维和胜伊回来再看亲娘一眼。而阴阳先生择定时辰,明早就要正式合棺了。

赛维和胜伊离了灵堂,还能若无其事的说笑两句;如今回了来,心中悚然,哀痛的情绪就又占了上风。马家不和睦,又是夜晚,只有一名老仆昏昏欲睡的守着。赛维和胜伊把他打发走了,然后茫茫然的站成了一排。

无心围着棺材缓缓绕了一圈,最后停在了棺头的缝隙前。赛维和胜伊看了他的行动,知道必有缘故;而无心把衬衫袖子挽到肘际,双手扶住棺材两角,俯身把双眼凑上了缝隙。

棺材内当然是一片漆黑,漆黑之中,躺着个艳妆华服、面目狰狞的二姨太。二姨太的眼睛没有闭紧,可是黑眼珠已然翻了上去,所以上下鲜红的眼睑之间,赫然露出了一线惨白。

即便是横死的人,死相也不该如此怪异。无心想了一想,随即直起腰转向了姐弟二人:``你们见过令堂了没有?''

赛维和胜伊并肩站立,一起点头,赛维又低声说道:``就看了一眼\ldots{}\ldots{}没敢多看。''

无心知道他们虽然顽劣惫懒,但毕竟还是年少。对着他们又笑一下,他轻声说道:``有我在,不要怕。''

然后他垂下眼帘,将右手慢慢伸进了缝隙之中。他的手掌很薄,手臂像白蛇一样蜿蜒而入。指尖划过了二姨太的头发,他微微蹙起眉头,轻声唤道:``小健!''

小健从缝隙里露出一只眼睛:``你又用得上我了?''

无心说道:``我怎么找不到?''

眼睛消失了,他的指尖有了知觉。随着一抹凉意慢慢移动,最后他在二姨太头顶心中停了指尖。厚重油腻的头发下面,有了一点若隐若现的小小尖端。他低声说道:``小健,胡说八道,哪里有钉子?''

指甲钳住了坚硬尖端,他咬牙切齿的向外抽拔:``分明是一根针!''

小健正要反驳,然而却是忽然向后一缩:``有人来了!''

无心猛然收回了手,一弯腰拎起了供桌下的小油壶。同时灵堂门口黑影一闪,马英豪毫无预兆的出现了。

赛维和胜伊全吓了一跳,可是吓归吓,并不失措。两人训练有素的转向门口,一起悻悻的唤道:``大哥。''

马英豪换了一身黑袍,衣裳黑,头发眉眼也黑。拄着手杖慢慢走了进来,他平淡的说道:``在为二姨娘守灵?''

赛维点了点头,仿佛一身的骨骼要散架子:``大哥,往后我们就成没娘的孩子了。''

马英豪停在棺尾,移动眼珠扫视了灵堂环境,口中答道:``你和老三都很有孝心,如果二姨娘在天有灵,也该欣慰了。''

然后他把目光转向了无心:``师父也来了?''

无心简短的答道:``我是没事做的闲人,正好可以陪伴他们。''

话音落下,他转身背对了马英豪,提起小油壶,往长明灯里添油。而赛维保持着悻悻的状态,半死不活的问道:``大哥怎么也来了?娘的丧事全依靠你张罗,已经够累得慌了,夜里还不好好休息?''

马英豪答道:``我怕仆人偷懒,既然你们都在,我也就放心了。''

话说到此,他转身作势要走,可是在临走之前,却又说道:``有没有手电筒?''

赛维和胜伊对视一眼,随即答道:``没有手电筒,有灯笼。''

马英豪一点头,转而注视了无心:``师父既然是个闲人,可否提着灯笼送我一程?''

无心方才一直提着小油壶,此刻放下油壶,他答道:``当然可以。''

然后他点了一只沉重的白灯笼,绕过棺材走向了马英豪。马英豪不再看他,拄着手杖径自向外走去。

目送着无心的背影出了灵堂,胜伊低低的嘀咕道:``你看大哥阴阳怪气的死样子!''

赛维没言语,因为发现无心站过的地面上,留下了一道一道的油迹,分明是用灯油浇出了潦草的字。走近了弯腰仔细一看,她轻轻念出了声:``发内有针。''

然后伸脚抹乱了字迹,她莫名其妙的对胜伊又重复了一遍:``发内有针?发?头发?谁的头发?''

胜伊立刻望向了棺材缝隙:``姐,刚才他不是伸手在摸娘的头?''

赛维知道胜伊胆子小,所以直接挽起袖子,壮了胆子把手往棺材里伸。哪知未等伸到深处,就在二姨太的头顶上摸到了一根突出半寸的钢针。咬牙捏住针尾,赛维运足力量猛然一拔,长针立时被她彻底抽离。

可是还未等她把针取出看清,棺材里面忽然传出一声沉重的叹息。腐臭气味从缝隙中弥散开来,她清楚感觉到母亲的脑袋向下一沉,是彻底脱力放松的表现。

与此同时,无心已经护送马英豪穿过了两重院子。马英豪走得很慢,一边走一边盘问无心的来历。步速慢,语速也慢,一切都是慢条斯理。无心挑着灯笼,问一答一,内容还是老一套。眼看快到大少爷的院里了,远方忽然隐隐起了嘈杂混乱的人声。无心和马英豪一起觅声望去,却见灵堂方向红光冲天,竟是失了火的光景!

\chapter{疑团}

马宅是座老宅子,灵堂所在的小楼,已经有超过二十年的历史,因为陈旧,所以早就空置不用,只是因为楼下有个宽敞的大厅,所以如今才打扫布置了,专为停放二姨太。大火是从楼上烧起来的,火苗顺着电线窜,眨眼的工夫就蔓延到了楼下,把灵堂围成了火海。大半夜的,万籁俱寂,除了赛维和胜伊再没别人;赛维和胜伊没有坐以待毙的道理,但是也不具备抢救棺材的力量。撩着孝袍子逃出小楼,他们站稳之后一回头,就见楼门已经被大火封死了。

两人都傻了眼,其中赛维算是一位运动家,虽然心中恐慌,但是两条细腿还能支撑身体;胜伊则是成了一束瑟瑟发抖的麻杆,撑着一身孝袍子单是发抖。而赶在惊动仆人之前,无心已经像阵风似的,越过两道灌木以及一大片草坪,抄近路跑回来了。

他虽然回了来,但也无济于事,只能是给姐弟二人一点精神上的安慰。胜伊本来是依靠着赛维的,如今见了无心,当场倒戈,用一只汗湿的凉手紧紧扯住了他的裤子背带,又低声唤道:``姐,姐,你也过来。''

赛维和胜伊一起站到了无心身边,与此同时,仆人也呼号着来了。人来了还没有用,因为消防队救火会迟迟不到。火场乱成人场,马英豪方才被无心抛在了半路,如今带着几个随从也到了。赛维不等他问,直接跑上前去哭道:``大哥,怎么办?怎么办?娘抢不出来了!''

马英豪显然也是头大如斗。安抚似的拍了拍二妹的肩膀,他手舞足蹈的开始做指挥。而赛维趁乱退下,带着胜伊和无心悄悄撤退了。

他们回到了二姨太的小院,未等进门,迎面却是来了一队莺莺燕燕。走进了一瞧,原来是几个俏皮小丫头簇拥着一位苗苗条条的小姐。小姐穿得素净,看年纪也就是十六七岁,瓜子脸,丹凤眼,倒是有几分妩媚的风采。对着赛维一蹙眉头,她开口说道:``二姐三哥,怎么了?我听说你们又遭遇了不幸?''

赛维轻轻一叹:``是呀是呀,我好不幸呀,刚刚没了娘,灵堂里又走了水。哪像四妹无忧无虑,多么幸福。''

四小姐顿了一下,面不改色的又道:``看了二姐三哥的不幸,我做妹妹的又怎么幸福的起来呢?''

赛维挑着小脖子,细着嗓子``唉''了一声:``四妹你可别乱讲。你肯陪着我们不幸,我们没有意见,可是举头三尺有神明,万一真连累了五姨娘可怎么办?做人子女的,孝字当头,可不能有口无心的胡说哟!''

她说完了,后方的胜伊又轻飘飘的加了一句:``四妹不怕的,四妹年纪还小,童言无忌嘛!''

赛维立刻接道:``哟,四妹,看你三哥多偏向你。''

然后她转身向院内走去,胜伊迈步跟上,头也不回的又留了一句:``四妹,天黑三哥就不留你进屋坐了。要看大火可得快点去,等到水龙架好了,仔细喷湿了你的衣裳。''

马四小姐本是为了看笑话出门的,不料话只说了两句,反倒被一对龙凤胎狠狠挤兑了一场。咬牙咽下一口恶气,她就觉眼前一黑,仿佛有个影子追在胜伊身后似的。未等看清,胜伊已走远了。

黑影是无心,他悄无声息的跟着胜伊进了房。院门关上了,房门也关上了。赛维不忙着脱孝袍子,而是先对无心伸出了一只紧握的拳头:``你瞧。''

拳头一松,一枚铁针落到了无心手中。铁针能有巴掌长,带着一层晦暗的锈色,一端尖锐,另一端浑圆。无心捏着铁针迎了电灯看,没有看出眉目。忽然嗅到了小健的气息,他开口问道:``今天怎么很自觉,直接就躲了起来?''

小健远远的悬在窗帘后方:``我怕你的针。''

无心怔了一下:``你怕它?为什么?''

小健答道:``不知道,反正就是怕。''

赛维和胜伊听不见小健的话,但对他的自言自语也是习以为常,并不惊讶。等他沉默了,赛维说道:``无心,还有一件事情我要告诉你。把针拔下来的时候,我听到棺材里有人叹气\ldots{}\ldots{}就是娘的声音。''

胜伊随即也开了口:``只有一声,我们想看又不敢看。结果后来就着火了\ldots{}\ldots{}''

无心思索了片刻,末了却是问道:``灵堂里的火,是怎么来的?''

赛维摇了摇头,低声说道:``怎么来的,我们不知道。照理来讲,不该失火;不过电线老化也是有的\ldots{}\ldots{}不好说啊!''

无心又问:``如果我说是有人故意纵火,你们想一想,目的会是什么?''

赛维想得多,一时无话可答;胜伊的头脑相对简单,倒是立刻有了答案:``烧死我们?''

赛维立刻摇了头:``不对不对,凭着我们的身手,怎么可能等着火来烧?灵堂又不关大门,难道放火的人不知道我们会逃?再说了,本来也不该我们去守灵,我们不是临时决定去的吗?''

无心轻声又问:``你们能逃,谁不能逃?''

赛维望向无心,声音也轻成了耳语:``都能逃\ldots{}\ldots{}只有娘不能逃。''

胜伊出了一身冷汗,慢慢脱了孝袍子:``娘已经过世了,难道还能被人再杀一遍不成?''

无心继续问道:``如果对方是要把令堂化为灰烬,化灰的目的又是什么?''

胜伊不敢想了,一步一步挪到无心身边,拖了椅子坐下。赛维也开始去解孝袍子:``人成了灰\ldots{}\ldots{}我们就看不到她了。''

无心对她一晃铁针。

赛维恍然大悟:``火烧起来,天下大乱,也不会有人发现娘的头里插着针了!''

胜伊轻声说道:``明早就要盖棺呢,盖了棺不也是一样的不会有人发现?''

赛维把孝袍子堆在一把空椅子上,露出里面带着花边的青色衬衫:``倒也是。''

无心盯着手里的铁针:``盖了棺,遗体还在;烧掉了,就什么都没有了。''然后他向前微微探头,一双大黑眼睛透了亮光:``你们知不知道借尸还魂?''

赛维和胜伊一起打了个冷战:``知——不知道。''

无心把声音压到了最低:``人死之后,灵魂不散,就成了鬼。若是鬼的力量足够大,可以附回到尸体上,操纵控制尸体,能活动,能说话,乍一看好像活人。''

然后他把针一竖:``如果只是为了掩盖它,不用放火,派个人偷偷把它取出来就行。''

赛维难以置信的瞪了他:``你是说我们的娘\ldots{}\ldots{}变成了鬼?''

无心继续摇头:``变成鬼倒好办了,起码不会伤害你们。''然后他又是一亮铁针:``也许,有人对令堂施用了邪术!''

房内静了一瞬,随即胜伊福至心灵,效仿无心进行了思考:``姐,你说如果我们二房倒了霉,谁最高兴?''

问完之后,他抬手轻轻一拍嘴唇,感觉自己是说了废话。马家除了乱七八糟的成员不算,真正儿女只有五人。将来分家产,也是五个人,少了哪一个都能省一份金钱。二房人多,两个孩子,如果全军覆没,余下三人自然都有好处。但是父亲身体如今还很硬朗,若说对方是为了家产下毒手,未免太早了一点。

赛维冷笑一声:``哼,都有嫌疑!老大不用提了,根本就是和家里有仇;老四不用提了,恨不能吃了我们;老五虽然年纪小,可是八姨娘比猴子还精,仗着有一张好脸子,可没少欺负娘。''话到此处,她将一只瘦骨嶙峋的小拳头捶上桌面:``远的先不要提,只说眼前——一会儿可怎么睡?''

胜伊立刻答道:``我和无心一起睡。''

赛维是个大姑娘,自然知道自己不能和他们挤做一床。略略思忖了一下,她摆出大姐的派头,不由分说的做了安排:``我去睡娘的卧室,你们不许走,就睡到卧室外面去。''

胜伊茫茫然的看她:``姐,外面没床。''

赛维立起眉毛:``不是有张罗汉床,还有个小沙发吗?将就着吧!''

胜伊基本不是赛维的对手。卧室的确连着一间小小的屋子,是二姨太吸鸦片烟的场所。他在罗汉床上铺了被褥,也不洗漱,脱了鞋就往床上滚。无心没有思考出下文,索性也挤上去了。

赛维进了卧室,心想一墙之隔躺着两个大男人,总算是够安全。要来热水擦了把脸,她坐在梳妆台前梳了梳头发,心想明早必定还是不得安宁,此刻得歇且歇,娘没有了,胜伊又不是个硬气的青年,自己再不振作,还不让人生吞活剥了?

思及至此,她也不打算脱衣。抬手关了房内电灯,她半睁着眼睛预备上床。然而就在转身坐到床沿的一瞬间,她忽然一愣,感觉自己是瞥到了什么。

慢慢扭头望向梳妆镜子,她看到镜中游移着一团微弱的光。浑身肌肉骤然一紧,她猛的站起了身,下意识的攥了拳头,对着镜中光芒先啐一口,随即恶声恶气的叫道:``什么东西?少来作怪!我可不怕你!''

\chapter{窥视}

赛维惊恐无措,因为听人讲老故事,都说鬼怕恶人,于是退无可退,索性站在地上开始叫骂。卧室内外只有一墙之隔,她一出声,外间立刻就有了知觉。

她是不防备胜伊的,房门虚掩了,并没有锁。所以未等她话音落下,房门被人``咚''的一声撞了开,正是无心和胜伊一起冲了进来。胜伊身上还缠着一条毛毯,两只脚一路乱绊,刚一进门就摔了个狗吃屎。无心穿着衬衫裤衩,打着赤脚挡在了赛维面前。张开双臂做了个护卫的姿态,他向前定睛一看,随即却是松了一口气。

一步一步走到梳妆台前,他对着玻璃镜子弯下了腰。从衬衫胸前的口袋里摸出铁针,他用针尖轻轻去刺镜中的光团。针尖触到冷硬平滑的镜面,当然不能够深入,然而光团宛如自有生命一般,竟然随着他的一戳,闪闪烁烁的熄灭了。

若有所思的捏着针直起腰,无心回头对着赛维和胜伊一笑:``没事了。''

赛维在叫骂了一句之后,就下意识的屏住了呼吸,直到此刻才透了气:``怎么会有光?''

无心笑着摇了摇头:``不用细想,一缕残魂而已,自保都不能够,自然也不会害人。至于它是怎么来的,我还要再想一想。不过一般人是看不到它的,一旦见到了,说明你们阳气不足,不是个健康走运的时候。从今往后,万事都要小心为好。''

胜伊抱着毛毯,凑到了赛维身边:``姐,我不出去睡了。咱们三个谁也别走,一起混到天亮吧!''

二姨太的床,算是一张双人床。赛维和胜伊东倒西歪的蜷缩着躺下了,无心坐在一旁充当守夜人。独自坐在夜色之中,他聚精会神的玩弄着手里的铁针。方才镜中的一缕魂,不知道是不是二姨太的,总之是受了铁针的吸引,此刻还幽幽的附在针上,在无心眼中,是一抹挺好看的光。小健从门缝里挤进了一个血淋淋的小脑袋,因为怕针,所以不敢靠近,只怔怔的看着他。看了一会儿,见他不理人,就索然无味的飘走了。

无心对着一根针思索良久,最后心里隐隐的有了点数。转头再去看身边的一对姐弟,他发现姐弟两个都已经入睡了。窗外的月光洒在床上,深浅光影勾勒了二人的相貌——平平的眉毛,内双的眼皮,很干净秀气的单薄脸儿,因为瘦,所以看着仿佛是还没长开,有一点青黄不接的幼稚相。经过几日的交往,无心知道他们两个绝不幼稚,小小青年的躯壳里驻扎着泼辣少奶奶的灵魂;若谈情操和志向,他们或许没有;若谈小心眼和小手段,他们都算人才一流。一样米养百样人,他们姐弟也算其中一类。不过无心寂寞极了,能够和他们两位厮混一阵,已经感觉十分荣幸和快乐。

天还没亮,赛维就先醒了。醒了之后坐起身,她朦胧着一双睡眼去看无心:``你一直没睡?''

无心扭头看她:``还早呢,接着睡吧!''

赛维摇摇头,伸腿下床,摸索着去穿拖鞋:``不睡了,不知道今天还要出什么幺蛾子。原来有娘的时候,虽然娘还不如我们机灵,但总像是有主心骨;现在娘没了,爹又不在家,我们不提防是不行的。''

她正色说过了一篇话,然后就出门去叫丫头送热水。一番洗漱过后,三个人都干净了,胜伊又让老妈子预备早餐。早餐是西洋式的蛋糕、牛奶、咖啡。赛维和胜伊显然是对于饮食兴趣不大,一双大鸟似的相对而坐,浅啄几口就算饱了。胜伊见无心能吃能喝,忽然起了一点玩心,把自己的蛋糕碟子推向了他:``喏,我只吃了一口,你要不要?''

赛维对无心生出了一点回护的心思,此刻见胜伊一脸笑嘻嘻的贱相,就开口斥道:``你少欺负人,谁要吃你的剩蛋糕?''

无心微微一笑,倒是脾气很好:``没关系,如果你们不爱吃,就都留给我。''

赛维没言语,自顾自的想:``胜伊什么都好,就是狗眼看人低。将来我若真是和他结了婚,恐怕胜伊都要笑我。没人要的浪蹄子,竟敢笑我,混账,欠揍!''

她想着想着就攥了拳头,正想找碴和胜伊火拼一场,不料外间忽然起了问候声音。扭头向窗外一看,却是马太太来了。马太太穿着一身灰哔叽袍子,生得头发乌黑,面孔圆润,一双皂白分明的大眼睛,几乎还带着一点姑娘的青春气。总而言之,算是一位美丽的少妇。

无心不等人吩咐,拿起碟子里的蛋糕就走,一直撤退到了卧室里去。而马太太被小丫头引进房内,对二人苦笑着一点头:``我那屋子,离前头太远,早上才听说夜里走了水。你们爸爸不在家,我又是个没主意的,就苦了你们两个孩子了。往后你们算是大人了,要知道自己照顾自己。如果有了困难,就直接找我去。''

说完这话,她带着一点愁容,惨淡而又端庄的起身离开。赛维领着头,一直把她送出院门;结果转身刚一回屋,就听胜伊对着无心嚼舌头:``我们这位妈,和老大\ldots{}\ldots{}''

赛维听他口无遮拦,肆意宣扬家丑,立刻喝止。然而停顿了一秒钟后,她心痒难耐,做了进一步的解释:``所以你看她虽然不老不丑,但是爸爸早就不理她了。若要人不知、除非己莫为。现在怎么样?大哥搬去了天津住,对她也淡了。''

胜伊点了点头:``对,死瘸子没良心的。''然后对着赛维一挤眼:``她也真是憋疯了,瘸子都要。''

然后一对姐弟嘻嘻而笑,虽然还没结婚,可是因为早熟,所以咂摸着马太太的烦恼,感觉格外有意思。胜伊一边笑,一边端起咖啡杯,翘着兰花指捏着小勺子,像个居心叵测的小娘们儿似的搅了搅咖啡,然后仰头一饮而尽。

不等外人催请,姐弟两人穿上孝袍子,在微明的天光中赶去火场废墟。无心独自留在房中,把门窗都关掩好了,然后继续对着手中的铁针发呆。

铁针上的残魂已经散了,可见它虽然带有一点力量,但是力量不强。人的头骨最硬,把它插进二姨太的头顶心里,必定不会容易。据说二姨太是在清早起床后自称不适,一口气没上来,就此去了西天;经过了医生的验尸,也认定的确是她的心脏出了问题。如果其中没有谎言的成分,铁针就必定是死后才插进去的。马家是个各顾各的大家族,真想对二姨太的尸体动手脚,想必并不会很难。

无心越想越是清楚,末了把针贴身藏好了,他起身开始在卧室内四处走动。赛维和胜伊不知为何,是特别的信任他。二姨太的梳妆台下一排小抽屉,全没上锁。他拉开一只一看,就见里面乱糟糟的放着绢花头饰,珠子链子。东西不算多么贵重,但也都是值钱的,他连着拉开几只,心想还是再等一等吧,否则私自翻检,有做贼的嫌疑。

关了抽屉直起身,他发现梳妆台的镜子前还摆着一只半旧的化妆品盒子,盒子里面盛放了许多杂物。他随手掀开盒盖,就见里面扔着几管口红,一只粉扑,和几根七长八短的眉笔。眉笔都是高级货,笔芯又软又黑。其中有两根最醒目,因为全被削成了小手指长,并且削得乱七八糟,绝不会是丫头的作品,怕是二姨太亲自削的,而且削的时候,并不是心平气和。

无心饶有兴味的审视着眉笔,看过眉笔之后,发现镜子下方的缝隙里并不干净,凝结着白色的粉渍、黑色的笔芯碎屑、红色的胭脂末子。而一道黑迹划过宽宽的镜框,显然也是眉笔所留。

无心伸手摸了一下,蹭得手指一道黑。仆人虽然工作马虎,可是每天都会进来四处抹拭一番,可见黑迹很新,也许是二姨太太在临死前留下的——人一死,照例的洒扫自然会中断,上下全为了二姨太忙做一团,还有谁能想到继续清洁房屋?

黑迹画在了镜子右侧,于是无心下意识的向右望了一眼。右边是靠墙的大床,并无异常。无心走去坐到床边,心想二姨太也真是要人命,连句明白话都不给儿女留。

然后他抬头面对了前方的玻璃窗,却是吓了一跳。玻璃窗前左右垂了窗帘,窗帘中间露出缝隙,缝隙之后,赫然贴着一只眼睛。

一挺身站起来,他上前几步,双手扯着窗帘用力一分。窗外的面孔露了全貌——原来是个十二三岁的小男孩,西装革履的打扮着,若从相貌论,平头正脸,眉目倒是类似赛维姐弟。老气横秋的瞪了无心片刻,他忽然扭头就跑。而无心一转身出了卧室,找到了老妈子问道:``刚来的小孩子是谁?''

老妈子也带有马家风格,背后从来不说人的好话:``是五少爷,小鬼似的不声不响,他要是不跑,我都不知道他来了。不怪老爷不疼他,好好的少爷家,干什么成天贼头贼脑的?''

无心点头,又回房去了。

据他所知,二姨太平日除了打小牌攒体己之外,就是在自己的小院里高卧享福,把自己养的富富态态,以至于马老爷很善待她,看她是个敦厚有福的人。二姨太死前行动异常,应该也疯不到远处去。卧室里面是很值得搜查的,但是他不能单独行动,要等姐弟两个回来了再计议。

他定下主意,不再停留,出门绕到房后,找了个犄角旮旯坐下了。天光大亮,小健不知躲去了哪里,他竖着耳朵,总感觉五少爷不会无故窥视。

果然,不过一个时辰的工夫,他听见了四小姐的声音:``哟,张妈,瞧见俊杰了吗?''

俊杰大概就是五少爷的名字,因为老妈子立刻答道:``五少爷刚来跑了一圈,早就走啦。''

四小姐又道:``前头乱得很,我进去坐着歇歇。听说三哥带了个朋友回来,新鲜,三哥去了一趟上海,还学会交际了!张妈,屋里有生人吗?有的话,我就不进去了。''

老妈子当即作了回答:``四小姐请进吧,不用看。三少爷的朋友刚出去了。''

四小姐无端的在房内坐了半个多小时,末了告辞离去。

无心一直没敢露面。他虽是个孤独漂泊的人,但是大家庭里的斗争,他是明白的。大概在二姨太死亡之前,暗潮就已经开始有了汹涌的趋势,如今既然他和赛维姐弟有缘相识,他就要保护他们两个不受伤害。

\chapter{秘密}

胜伊下午先回了来,脸上花里胡哨的带着黑灰。他们凌晨赶去灵堂之时,二姨太已经被人挑拣进了一只大铁盘子里,零零碎碎的,一共能有大小十几块焦黑的骨头。马英豪彻夜未眠,英俊的面孔看起来有点垮塌,拄着手杖站在废墟上,他半闭着眼睛摇摇晃晃。

兴许是同性相斥的缘故,塞维特别看不上四小姐,胜伊也是见了大少爷就烦。赛维还去敷衍做作,他索性呆着面孔傻站。新棺材运来了,照理说今天是出殡的日子,遗骨被装进棺材里,马家也无所谓孝悌门风,大少爷做主,该出殡,还是出殡。

胜伊的悲痛已经被城里城外的奔波疲惫抵消了。擦了把脸换了套西装,他把臂上的黑纱整理好了,然后也不理人,只在卧室外间的罗汉床上一坐。坐着坐着,他迟缓的撩了无心一眼,心里倒像是有所依靠似的,略微安定了一点。无心还是工人裤白衬衫的打扮,静静的站在一旁,并不肯出言搅扰他。

片刻之后,赛维也回来了,形象之狼狈,类似方才的胜伊。她走去浴室对自己痛加涤荡,一小时后才复又出现。把湿漉漉的短发掖到耳后,她热孝在身,不好化妆,可是完全不修饰的话,她气色不好,又是一张薄薄的黄脸。从理智上讲,她一点儿也没有和无心谈恋爱的打算,可同时很希望对方倾倒在自己的石榴裙下。犹犹豫豫的往脸上抹了一点雪花膏,她自觉着颇为清秀白净了,才算满意。

无心见他们二人到齐了,便低声向他们讲述了自己的计划。两人且听且点头,松弛了的神经重新恢复了紧绷。吃过一餐晚饭之后,房内电灯通亮,三个人既不休息,也不行动,而是围坐在罗汉床上打扑克。偶尔有老妈子小丫头出入往来,他们也毫不介意。扑克打到十一二点,赛维又让人端来了夜宵。三人吃饱喝足之后,才作势是要各自休息了。

他们不睡,仆人也不能睡;熬到午夜,全困得东倒西歪。好容易得了休息,登时就各归各房作鸟兽散。而赛维拉了窗帘锁了房门,又把电灯一关。窗外空中高悬着一轮银白色的大月亮,月光透过窗帘,倒是照得房内影影绰绰。

胜伊先动了手,在墙角一处玻璃橱前蹲下了,小心翼翼的拉出下层抽屉。赛维则是赤脚上了床,从头到尾细细的摸索褥子底下。

胜伊的嘴没有赛维伶俐,干起细致活,却是一双巧手。搜查过玻璃橱后,他转而蹲在了梳妆台前,无声无息的把小抽屉整个拉出来放在了地上。翻着翻着,他忽然轻声开了口:``娘的东西,被人动过了。''

赛维登时抬头看他:``怎么?''

胜伊举起一只金灿灿的小蝴蝶:``夹头发的小夹子,和绢花混在了一起。''

无心低头去看,就见地上一排三只小抽屉,里面全是乱糟糟的花红柳绿,毫无秩序可言。而赛维则是恍然大悟,低声对无心解释道:``小夹子是镀金的,应该和珠子放在一起。''

原来二姨太有个特点,就是很爱自作主张的为物品分类,分了类,就要各归各类。一类的东西邋里邋遢混在一起,看不出整洁,但是她就感觉顺眼舒服。

胜伊继续翻检,赛维继续满床爬,无心又望向了梳妆镜框上的黑迹。伸手摸了摸镜子后,他没摸出什么,于是下意识的又向右侧望去。胜伊和赛维忙着,也无暇去注意他。

良久过后,赛维把被褥都快捏熟了。一无所获的跪坐着,她叹了口气,刚要说话,不料床下忽然传出``笃''的一声。

她吓了一跳,胜伊也停了动作。随即床下又起了低低的敲击声音,和敲击一起响起来的,是无心的声音:``床板下面,有东西!''

赛维连忙跳下了床,蹲在地上一掀曳地的床单,很惊讶的发现无心不知何时钻了进去,此刻正长条条的躺在黑暗中。

床是铁架子床,铺着木头床板,床板上又放了弹簧垫子。无心从床板与铁架之间的缝隙中,抽出了一张折好的白纸。

顶着头上一缕灰尘爬出来,他把白纸对着姐弟一晃。而赛维手快,一把夺过了展开,胜伊伸头一瞧,紧接着却是一愣:``什么东西?''

赛维把纸递给了无心,无心看过,也是莫名其妙——纸片本身只有巴掌大,上面寥寥几笔,依稀画出了一座小山,山上有个亭子,亭子中央又画了个很重的圈。除此之外,再无其它。

无心看了又看,实在是摸不清头脑。赛维也嘀咕道:``画的是哪里呢?''

胜伊答道:``反正娘多少年没出过城了,如果真是写实画,也不会远。''

赛维夺过纸片又看了看,然后对着面前二人竖起一根手指,见神见鬼的轻声说道:``我知道了!的确不远,我们走到画上的地方,也要不了几十分钟。''

不等二人发问,她诡谲一笑,又一抖手中的纸片:``它不就是我们家的后花园吗?''

马宅的后花园,也有几十年的历史了,和马宅一样,都是马老爷之父的成绩。赛维和胜伊对于祖父,印象都不深刻,只知道祖父白手起家,很是厉害。后花园的面积,抵得上一个小公园,里面风景全是人工堆砌,倒也有山有水,有花有林。此刻虽然入了秋,但园内景致还是颇有看头;只是马家人都看惯了,看不出美来,甚至会懒得去。

赛维和胜伊再迷茫,也看出问题了。三人挤到床上,开始嘁嘁喳喳的谈话。赛维说道:``肯定是娘画的,看看,用的还是眉笔。''

胜伊思忖着说道:``是不是娘出了什么事,提前想要逃,没逃成?她不许我们回家,是不是因为家里不太平?''

赛维垂下了头:``我们家能有什么大事?无非就是内战罢了。''她把纸片往床上一放:``除非是亭子出了问题,我们家要闹分裂,内战变成外战。''

胜伊冷笑一声:``瘸子不是已经分裂出去了吗?''

赛维答道:``你当五姨娘八姨娘是老实的?别看老四老五年纪小,也都诡着呢!爸爸是个火药桶的脾气,我都懒得瞧他,五姨娘八姨娘能和他真有感情?''

姐弟两个把家中上下批判了一场,批判过后,毫无结论。无心由着他们说,等他们说过瘾了,才把话题转向正途。马英豪在家,总像是家里有个主人;于是他们决定等马英豪回天津之后,便去花园亭子里实地的侦查一番。

如此过了两天,马英豪见家中平定,果然就要回天津去。弟弟妹妹们对他都有几分顾忌,听说他要走,纷纷表示好走不送。

马家早在祖父一辈,就和日本人有交情。马老爷是日本人的官,马英豪也是吃日本人的饭,并且是各吃各的,不是一派。抗日战争进行了六年,越打越是不分胜负,马老爷趁机得了滔天的权势;马英豪比不得父亲的本领,但在天津也很吃得开。

乘坐汽车离北京到天津,他在一个明媚的秋日下午回了家。天津的马公馆,是一处平淡无奇的小洋楼,位置和样式都过分的平淡了,简直不称他的财富和身份。

五年前大少奶奶和他离了婚,所以家中如今就是他一条光棍。他拖着从小瘸到大的右腿,一步一晃的走入楼内。

在小客厅里坐下来喘了几口气,他喝了一杯热茶,然后拄着手杖站起身,楼内没有正经仆人,此刻跟在他身边的,是个用久了的半老头子。老头子跟了他几步,见他始终是没吩咐,就也退下了。

马英豪一边走,一边从裤兜里摸出一串白铜钥匙。在走廊尽头的一扇小门前停了脚步,他低下头,找出一枚钥匙开了房门。

开门进房之后,房门随即就又被关上了,``咔哒''一声,暗锁合了个严丝合缝。伸手一扯门旁的灯绳,天花板上垂下的电灯泡立刻放了光明。房间应该本是间储藏室,连窗户都没有,但是也没有杂物,只靠墙摆着一只硕大无朋的大玻璃缸。细铁管子穿透天花板,沿着墙角从二楼走下来,拐着弯的探入玻璃缸内,是一套颇为丑陋的自动换水装置。

房内弥漫着憋闷的咸腥气息,因为半面墙大的玻璃缸中蓄满海水。十几条斑斓海蛇游曳其中,姿态是极度的灵活。

马英豪自己不灵活,所以很愿意欣赏海蛇的灵活。定定的望着大玻璃缸,他足足发了半个多小时的呆。玻璃缸的正中央竖起一丛乱七八糟的钢管,充当陆地。一条海蛇孤立无援的盘在上面,昂着尖细的小脑袋,倒是和他对视了一阵。

马英豪不是玩物丧志的人,看够了他的宠物之后,他转身走到玻璃缸对面的墙角。墙角地面上铺着一米见方的铁板,一边带着合页,像是地窖的铁门,门边还带着把手和锁头。他俯身打开锁头,然后握紧把手,用力把小铁门掀了开来。

铁门之下,黑洞洞的深不可测。阴凉的空气扑上来,带着霉味,直冲鼻子。马英豪慢慢蹲稳当了,伸手进去在门边摸摸索索,终于摸到电灯开关一摁,地下立刻隐隐有了微光。

轻车熟路的伸下一条腿去,他踩住了下面一级一级的铁制楼梯。身体随着步伐缓缓向下沉入,原来下方正是一层地下室。

地下室的正中央地面上,依然是盖着一层铁板。然而和上一层铁门不同,这层铁板虽然也是合页锁头俱全,但是面积更大,而且铁板上面开了个两尺见方的整齐风口。风口焊着一排粗实铁条,让人想起监狱。

手杖重重的杵上脚下铁板,发出一声闷响。马英豪静立不动,就听下方的空间里由远及近,起了一串铃铛声响。恶臭污秽的气息越来越重了,他摸出一条手帕,忍无可忍的掩了口鼻。

藉着微弱的灯光,他垂下眼帘,就见一张苍白肮脏的面孔缓缓升近风口。面孔微微偏着,乱发之中,露出一只蔚蓝的眼睛。

\chapter{白琉璃}

马英豪一手用手帕堵着口鼻,一手把手杖伸进风口的铁栅栏里。手杖一端拨开门下面孔上的乱发,他闷声闷气的问道:``有结果了吗?''

幽闭空间中似乎响起了隐隐的毒蛇吐信之声,嘶嘶的似有似无,不走耳朵,沿着人的汗毛孔往里钻,一直刺激到神经上去。蔚蓝的眼睛隐没进了黑暗,另一只眼睛露在了昏暗光中——大概本来也该是蔚蓝色的,然而瞳孔里面生了一层雾蒙蒙的白膜,至于到底瞎没瞎,马英豪就不知道了。

马英豪不知道,旁人也是一样的不知道。他是马英豪的日本朋友从西康带回来的。

马英豪有很多日本朋友,其中有一位名叫小柳治的军官,不过三十来岁的年纪,和他已经有了超过十年的友谊。小柳治在几年之前,曾经秘密潜入过西康。在西康,他从一群秃鹰口中救下了一个奄奄一息的怪人。

怪人看起来似乎还是青年的面貌,有一种病态的苍白和肮脏。裹着层层动物毛皮蜷在一片空场上,他看起来简直就是一座臃肿的尸堆。秃鹰嗅到了死亡的气息,张开翅膀盘旋在上空,而他微微低着头,从纠结的长发中露出了很清秀的尖下巴与薄嘴唇。

他的怪异形象,还不足以让负有重任的小柳治出手相救;小柳治之所以在他身边停了脚步,是因为听见他在用日本话喃喃自语,一岁如何如何,两岁如何如何,仿佛是在讲述谁的生平。

小柳治以为自己是遇见了落难的同胞,于是决定救他一命,带他离开西康,不料返程刚刚走到一半,小柳治就把肠子给悔青了。

怪人很少说话,并且永远裹着他的兽皮。兽皮的边缘还带着干黏的紫黑血肉,可见根本没有经过硝制,似乎是从野兽身上活剥下来之后,就被他直接披到了身上。兽皮下面偶尔可见他的衣裳——是一件看不出本质的藏袍,之所以看不出本质,并不是因为料子异常,而是因为肮脏。

没有人能够摆布得了他,他把得到的一切食物都藏进了他的兽皮下面,所以甚至没有人见他吃过喝过。小柳治渐渐发现他会说好几种语言,包括中国话,很可能只是个杂种,和自己的祖国毫无关系。小柳治想要把他抛弃,在动手的前一天夜里,他照例忍着嫌恶去和怪人搭讪,怪人缩在他的长发与毛皮里,却是意外的说了一句中国话。

他说:``我是白琉璃。''

小柳治登时大惊失色——白琉璃是西康地区近五年来,最恶名昭彰的巫师。他仿佛是从天而降,作恶多端之后又无端消失。在传说中,他已经死了。

小柳治不知道自己应该如何处置一个活魔鬼,于是白琉璃在到达天津后不久,就被投入了一间最隐蔽的监狱里。

谁也不肯接收他,他成了没人管理的怪物,直到马英豪听说了他的存在。使用了一点小小的手段,马英豪把他运到了自家。

对于一切异类,马英豪都很感兴趣;况且白琉璃并非只是简单的异类而已。而白琉璃还挺讲道理,吃着他的,喝着他的,也就真听他的。马英豪已经暗暗养了他一年,但是确定他不会伤害自己,还是在一个月之前。

弯腰打开锁头,马英豪掀开铁门,下方又有几级铁梯。他险伶伶的走下去,同时忍着越发浓重的恶臭说道:``我不想再等了,还有,你的铁针丢了。''

角落里盘踞着一团黑影,依稀发出轻轻的铃铛声。铃铛是马英豪亲自系在白琉璃脖子上的,因为地下室灯光昏暗,他时常看不出对方的所在,声音利于他的寻觅。本来没有在地下室再挖地下室的道理,但是白琉璃需要,白琉璃的眼睛,浑浊的加上清澈的,已经全不能见光了。巫术的反噬几乎彻底摧毁了他,他牺牲了他儿子的性命使自己苟延残喘,直到获救。

他很爱他的儿子,他的儿子一直被他藏在怀里。蜷缩在潮湿的地下室一角,他闭着眼睛垂下头,硬着舌头说道:``是的,丢了,我知道。''

马英豪已经渐渐习惯了此地的空气,所以放下了手中的手帕:``一切都是按照计划来进行的,可是很奇怪,事后我没能找到铁针。时间我算得很准确,绝没有差错。''

白琉璃的右臂软软垂在一侧,低头答道:``有人提前拔了针,散出了一魂一魄。''

马英豪皱起了眉毛:``魂魄不全,怎么办?''

白琉璃抬起左手,摸进怀里:``我试一下。''

然后他掏出了一只小小的人皮鼓,摆在了地上。左手指尖轻轻一叩鼓面,发出``怦''的一声,竟然类似心跳。随着鼓声响起,他的右臂猛然一颤,仿佛皮肉中没有骨骼,而是藏了活物。

马英豪并未畏惧。用雪白的手帕重新堵住口鼻,他冷静的观看白琉璃做法。

白琉璃是墙角里最肮脏最污秽的一堆,只有不断在鼓面跳跃的手指,表明一堆皮子里面有个活人。鼓声时急时缓,他的右臂也随之剧烈的抽搐痉挛。忽然神情痛苦的一仰头,他抬起右臂狠狠抽向墙壁。掩在胸前的兽皮松开了,一样东西骨碌碌的滚出来老远。马英豪不动声色的向下扫了一眼,然后立刻权当不见。

东西能有一尺多长,是具死婴。尸首经过了特殊的炮制,没有腐烂,也没有干枯。在上方透下来的电灯光中,它周身逸出鲜红的雾气,小小的面孔上,一双眼睛鼓凸着紧闭了,口鼻却是受了损毁,被人用黑线缝成了扭曲的一团,像个粗制滥造的娃娃。

正当此时,白琉璃已经停了动作。左手捏住右手中指,一根铁针从指甲缝中慢慢伸出。随着铁针一起出来的,是滴滴答答的黑血。

``我看到了\ldots{}\ldots{}''他哑着嗓子,竭尽全力的要逼出铁针:``看到了花,树,山,河。''

马英豪睁大了眼睛:``花树山河?那是什么地方?''

铁针彻底离开了白琉璃的指尖,针尖还带着丝丝缕缕的血肉。白琉璃答道:``我不知道。''

马英豪不耐烦的出了一口气:``你说过你能读魂!''

白琉璃把铁针横送到唇边,从头至尾的舔了一遍:``她的魂不全,少了一魂一魄,我也没有办法。''

马英豪一挥手杖:``废物!你本来说你能拘到她的灵魂,结果怎么样?她直接被你吓死了,还要我去给她收尸!你又说你能把她的灵魂引来,可是他妈的半路又丢了一魂一魄!花树山河花树山河,天下之大,到处都有花树山河,你给我的答案,有意义吗?''

白琉璃匍匐在地上,在低低的铃铛声中爬向马英豪。伸手抱过地上的婴尸,他慢慢后退,同时把婴尸揣回了怀中。

而马英豪单手叉腰,翻着白眼,心中暗想:``花树山河?二姨太大门不出二门不迈,怎么会看到花树山河?家里有花树山河吗?还真有,后花园子里,花,树,山,河,都有。''

收回目光望向白琉璃,他毫无预兆的转移了话题:``你需要什么吗?''

白琉璃双手抱在胸前,抱的是兽皮下面的婴尸:``我要盐。还有,去找我的针。''

马英豪若有所思的点了点头,然后忽然对他一笑:``辛苦你了。''

黑暗中起了铃铛响,是白琉璃缩回了角落。

马英豪向上回到人间,花了两个小时沐浴更衣。若有所思的走到电话机前,他将一只手搭上话筒,想了又想之后,他抄起话筒,要通了长途电话。

电话连到了北京马宅,听筒中响起了娇滴滴的女子声音。马英豪清了清喉咙,唤了一声:``八姨娘,我是英豪。''

八姨娘立刻就笑了,语气柔和之极。而马英豪继续问道:``俊杰在吗?他让我为他买几本图画书,我要问问他要求的程度。''

不出片刻,听筒里面变了声音,马俊杰清清楚楚的``喂''了一声:``大哥。''

马英豪笑道:``俊杰,要不要到天津玩两天?大哥招待你。''

马俊杰的声音低了些许,然而依旧清晰:``你们大人的事情,不要找我。我该说的都说过了,以后你不要再问,我也不想再提!''

马英豪问道:``俊杰,你以为二姨娘的死,和我有关系?''

马俊杰加重了语气:``我什么都不知道!''

然后``咔哒''一声,电话被挂断了。

\chapter{第二个人}

无心花了整整一天的时间画符,画了个人仰马翻乱七八糟。纸符高高摞起了一大叠,其中没有几张是真有效验的。画符至少要讲个心无旁骛一气呵成,可是无心的心灵像是一片空场地,四面八方的风随便过,他即便经过了十年的练习,也依然还是``定''不住。

胜伊坐在外间,算是卫士;赛维在屋里陪着他,看他一张一张画个不休,哪一张都是笔画流畅,像一幅画。他画的时候,她坐在角落里不敢出声;等到他唉声叹气的放下笔了,她才随之透过了一口气。看着无心做神棍勾当,她心里有些不舒服;不过做神棍总比一无所能稍强,她和无心一样,思绪在脑子乱窜:``反正现在只要认字,就没有办不了的公务。哪个衙门比较肥呢?交通还是财政?''

无心凝神静气的忙碌一天,忙得毫无成绩,不禁有些沮丧。垂着头把笔墨纸砚都规规矩矩的收拾好了,他对着玻璃窗,用一条手帕慢慢的擦头上热汗。而赛维轻手轻脚的走到近前,看他刚刚端起茶杯喝了一口冷茶,就鼓足勇气伸出手去,将一片薄薄的花生糖送到了他的嘴边。

无心愣了一下,并且转动眼珠看了她一眼,随即立刻张嘴衔住了糖,也没有笑,单是非常认真的用舌头把大片糖卷进了嘴里,嚼得面颊一鼓一鼓。赛维一手端着个糖盘子,见他把嘴里的糖咽下去了,便伸手又喂一片。无心垂下眼帘,先是将糖咬下一角,然后歪着脑袋找好角度,把余下大半片也一口吞下。嘴唇柔软的蹭过了赛维的指尖,赛维一哆嗦,感觉无心像一只驯良的野兽——非常的野,也非常的驯良。

房内很安静,空气中弥漫着花生糖的香甜气味。赛维一片一片的喂无心吃糖,喂多少吃多少。双方都不说话,仿佛已经心有灵犀。无心忽然抬眼正视了她,抿着满嘴的糖笑了一下,笑得很温柔,又有点讨好卖乖的意思,像个贱兮兮的小男孩,几乎带了一点可怜相。

赛维面无表情的看着他,浑身的血都冲进了脑子里,脸上红彤彤的发烧,手脚却是冷得将要颤抖。``不行了,不行了。''她迷乱的想:``他神棍就神棍吧!我倒贴就倒贴吧!横竖我贴得起,从今往后我再也不乱花钱了,我要攒钱做大事\ldots{}\ldots{}''

房门一开,胜伊进来了。

房内幽闭甜蜜的空气立时流通出去,赛维的头脑有所降温,然而一颗心还是在腔子里上下奔突,大跳不止。胜伊为了免得有人偷听,故意没关门,只压低声音问道:``无心,画完了没有?不是说今夜就去吗?我等了好些天,可要等不及了!''

无心若无其事的从桌上拿起两道纸符:``你和赛维一人一道,贴身贴在胸前就好。''

然后他伸舌头舔了舔嘴角的糖渣子,没有再看赛维。赛维的心思,他都知道;可还是原来的四个字:高攀不起。

赛维不是一只可以随着他到处走的孤雁,赛维身后牵牵扯扯一大家子人呢,人多眼杂嘴也杂,万一有个心明眼亮的看出了他的破绽,他受害,赛维一定也要受害。

胜伊接过了符,因见赛维还端着糖盘子,就暂且没有给她,继续低声说道:``你们听说了没有?八姨娘连着两三天没见人影了。''

此言一出,赛维不禁莫名其妙:``八姨娘不见了?她又没有娘家,能去哪里?俊杰都十二三岁了,她总不会还生别的心思吧?''

胜伊对她竖起一根手指,``嘘''了一声:``小声点,吵什么?外头都听见了。我猜她就是私奔了。她刚三十出头,要是真有相好的肯要她,不比她在家里守活寡强?''

赛维摆了摆手:``你别嚼舌头了,我们自己的娘都死的不明不白,还有闲心去管俊杰的娘?晚上我们都要多吃一点,否则到了夜里没力气,可就糟糕了。''

话音落下,院中忽然起了轻轻的脚步声。随即房门一开,进来的人却是马俊杰。

马俊杰虽然是个孩子,但是穿戴的比大人还要一丝不苟,一身小西装堪称笔挺,脚上皮鞋也没有半点灰尘。小游魂似的登堂入室,他站在里间门前,静静的仰头看人:``二姐三哥,你们见到我娘了吗?''

二姐三哥被他注视得很不舒服,立刻一起摇头,又装成懵懂天真的样子说道:``八姨娘从来不到我们院里来呀,怎么,你找不到她了?''

马俊杰抬手扶着门框,没言语,扭头仔细看了看自己的指甲,然后小声说道:``你们还是回上海的好。''

他的手很白,是个半大孩子的形状,骨骼纤细,巴掌薄薄的:``如果你们真去上海,把我也带上吧。我长到这么大,还没有出过北京城。''

赛维笑问道:``你光顾着玩,不上学读书啦?''

马俊杰放下了手:``我们家的人,还要靠着学问吃饭吗?''

然后他转身就走了。

胜伊看了男人就烦,包括马俊杰这个小男人,只感觉无心还算顺眼。马俊杰前脚一走,他后脚就嘀咕上了:``什么东西,鬼头鬼脑!怪不得连八姨娘都不疼他,我看他根本就是让个老鬼上身了。''

赛维无言的又摆了摆手,希望胜伊把嘴闭上。马俊杰的怪性子,也不是一天两天了,而她一直对这位小五弟毫无兴趣。

三人吃过晚饭,静等天黑。黑夜当然是不利于出行,然而花匠近来正忙着给花园里的花木剪枝,正好全聚集在了山上亭子周围,从早到晚人来人往,让他们没法肆无忌惮的寻觅勘探。依着无心的意思,是自己单独行动,让姐弟二人留在房里等待;依着赛维的意思,是她和无心同去,胜伊既无力量又无智慧,留下看家;胜伊直接啐了他们二位满脸花,表示从此以后,无论做什么事情,都必须三个人一起行动。

待到夜色浓了,赛维领头翻窗户出了屋子,无心和胜伊紧紧跟上。天虽然黑,但是还没到入睡的时候,所以他们一路走得躲躲闪闪,生怕被人瞧见,直到进了花园地界,才松了口气。

三人穿的全是橡胶底子的网球鞋,走起路来轻便利落。赛维眼神好,依旧是做领路人,无心跟住了她,同时伸手拉扯着身后的胜伊。胜伊一无所长,只好提了个手电筒。花园白天或许还有几分可看的景致,然而到了夜里,花木随风微微摇曳,一丛一丛深深浅浅,如同鬼影一般,让人只觉阴寒。片刻过后,无心听到了隐隐的水流声音,而前方的赛维轻声说道:``快到河边了,桥是坏的,我们是绕远路走过河,还是划小船抄近路?''

胜伊答道:``还是划船吧,划船的话,一下子就过去了。绕远路,至少得绕一里多地。''

两人一问一答,说话间已经到了河边。无心放眼望去,就见前方一条湍急小河,也就十多米宽,河对岸是高低的岩石,岩石往上一路斜坡,正是一座小山;而在山顶,果然有着一座小亭。夜色朦胧,看不出美;但是无心做了一番想象,认为如果到了好季节好天气,河水翠山小凉亭,再配上周遭的花花草草,的确是一幅毫无特色的美景。

河虽然不宽,但是也足够顺流泛舟,所以小河两岸也拴了几只小木船。赛维跳跃着靠近河边,因为平日时常来玩,所以轻车熟路的解开一只小船,又对着无心和胜伊招手。及至全体都上船了,她也无需帮忙,自己扳动木浆,便将小船划进了水中。

无心坐在船尾,先是一直不言不动。忽然抬手摸进胸前的衬衫口袋,他抽出了一直随身携带的铁针。弯腰把铁针探入水中,他发现河水似乎蕴藏了吸引力,在把铁针往水里吸。

他捏住铁针直起腰,用针尖刺破了指尖。将一点鲜血涂抹到铁针上,他向水中伸手又试了一次。果然,吸引力消失了,铁针随着小船的方向,在河水中乘风破浪。

无心收回铁针,随即摁了摁裤兜,裤兜里装着几张用来画符的黄纸。抬眼望向前方的赛维和胜伊,他没有说话,因为不想吓坏他们,自乱阵脚——马家如今真成凶宅了,凡是阴气重的地方,比如临水之地,全都汇聚了邪气。邪气是哪里来的,他说不清,总之,和铁针是同源。

赛维三划两划,便靠了岸。上船之时岸边平整,下船之时就困难了,因为为了美观,岸边巨石是个错落的形态,很不好落脚。三人蹦蹦跳跳的一路往山上跑,因为都很兴奋,所以仿佛也只是三步两步的工夫,便一起到达了亭子前。

亭子虽然陈旧,但却是一处精致的建筑,并非四根柱子八面来风的结构,四面都有活动的雕镂槅子,槅子背面还糊了一层薄纱,人在其中坐着,外界影影绰绰的看不真。夏天亭子顶损坏了,往下掉落砖石,马老爷来不及派人修理就出了国,所以家里管事的索性把亭子锁起,免得人进去了遇危险。赛维很了解家里的情形,提前在兜里藏了一把小钳子,预备使用蛮力,直接把锁扭开。然而掏出钳子围着亭子绕了一圈,她发现已经有人捷足先登,扭开了一个锁头。

没了锁头的钳制,槅扇自然是一推就开。赛维犹犹豫豫的抬起了手,作势要推:``是不是花匠白天进去休息了?''

无心上前一步挡在了她的身前。慢慢推开槅扇,他率先走了进去,只见亭子里除了四周有座位,中间有石桌之外,再无其它摆设。赛维随之进入,原地转了一圈,轻声说道:``也没有什么呀!''

胜伊提着手电筒,没敢开,因为现在还不需要光:``有什么才叫怪了呢。我们从小到大,来过无数次,哪次看出什么了?''

赛维抬手抓了抓头发:``娘到底是什么意思?真是的,有事情还瞒着我们!''

胜伊刚要回答,不料无心忽然抽鼻子嗅了嗅,随即一把抢过了他的手电筒。在他推动手电筒开关之时,三人上方忽然起了``咭''的一声。像是陈旧的门轴活动,也像是秋虫鸣叫。

光柱骤然向上打去,三个人仰起了头,就见黑幽幽的亭子檐下,探出一张惨白的面孔,正是失踪了几日的八姨太!

八姨太穿着一身花纹斑斓的长旗袍,身姿扭曲的盘绞在亭内梁柱上,如同蟒蛇。烫过的头发披散开了,她咧着嘴做了个笑脸,一双眼睛却是黑油油的反了光,居然不见白眼珠。低头面对着下方三人,她忽然又低而尖锐的鸣叫了几声,声音怪异,绝不是人能够发出的!

而在赛维和胜伊发出惊叫之前,无心猛然出手,把他俩全推出了亭子:``快跑!''

\chapter{水里逃生}

赛维其实都没看清楚八姨太的容貌详情,可也无须看清,单凭八姨太凌空下探的姿势,就足以把她吓成魂飞魄散了。顺着无心的一推迈出亭子,她耳边听得``快跑''二字,立刻不假思索的撒开了腿。

她腿长,尽管道路崎岖,但是她一窜一窜的跳着跑,全然不在乎脚下的起伏。跑出几步之后一回头,她又吓了一跳,原来胜伊紧随在后,因为过于惊愕,所以把嘴张了老大,像要咬谁一口似的。张着大嘴跳过一丛长草,胜伊忽然意识到了赛维的注视,不禁一个激灵,恢复神智,嘴也合上了,带着哭腔问道:``姐,我们往哪里跑啊?''

赛维见他无恙,放下了心,一边继续往河边狂奔,一边又用眼角余光去找无心。脚下忽然一个踉跄,她一个大马趴摔在地上。未等她痛叫出声,胜伊弯下腰使出吃奶力气,已经把她硬拽了起来。而她抬手捂着下巴,眼中流出了两行热泪——下巴磕在石头上了!

石头前方就是小河,小船也没有拴,孤零零的飘在岸边。赛维正要继续逃,不料身边的胜伊骤然怪叫一声:``鬼呀!''

她下意识的回了头,登时发根痒痒的竖起了一片。无心正在跑向自己一方,八姨太跟在他的后面,竟然如蛇一般趴在地上,快速的游动追逐。而无心抬头见姐弟二人全在岸边吓傻了眼,就急得大声吼道:``别等我,快上船!''然后回身一脚,他狠狠的踢中了八姨太的额头。

八姨太顺着力道一歪脑袋,无心看得清楚,就见她白皙的脖子显露出来,竟然是横绽开了一道细细的裂缝。缝中无血无肉,只露一线黑色。八姨太一晃肩头,一条手臂如同软鞭似的甩了出来,径直抽向无心的脚踝。无心向后一跳,避开手臂之后转身继续飞跑。

赛维和胜伊像被魇住了似的,思想和行为全停顿了,眼睁睁的看着无心冲向了自己。正是迷茫之时,赛维忽觉身体一飘,头脑瞬间清醒了,她发现自己是被无心拦腰抱了起来。一阵腾云驾雾之后,她``咣当''一声着了陆,却是被无心从岸上扔进了船里。

忍着疼痛爬起来,她眼前一花、脚下一震,正是胜伊也从天而降砸进了小船。姐弟两个全被摔聪明了,赛维有力气,转向前方抓住双桨,而胜伊跪在船尾,对着岸上的无心伸出了手,急得乱叫:``快来快来,抓我的手!快呀!''

无心不理会,一步跳进了河边浅水里。回头眼见八姨太又追上来了,他俯下身,用力把船推向前方。借着他的力量,小船立刻滑入深水,而他纵身一扑,将上半身扑上了船尾。胜伊发疯一般扯了他的衣领衣袖,不由分说的往船上狠拖。三下五除二的,居然立刻把他拽上了船。

未等无心坐稳,他哭唧唧的开了口:``下水了,她也下水了!她怎么了?她发精神病了?''

紧接着,前方的赛维也咬牙切齿的开了口:``他妈的!怎么划不动?''

无心把胜伊推向了赛维,同时说道:``她不是八姨太!''

赛维颤抖着扯了高音:``鬼?''

无心跪在了船尾,双手扶着船帮,目不转睛的盯着水面:``不是鬼,不要怕,当她是条蛇好了!''

赛维和胜伊各握了一支船桨,咬牙切齿的使劲划水。水中莫名的藏了阻力,他们费了十分的力气,却是只能前进一分。而无心从裤兜里摸出一张被水浸了半截的黄纸,咬破指尖画起血符。水面已经浮现出了一头黑发,是八姨太在觅着活人气息追逐。距离小船越来越近了,她忽然从水中一仰头,一张笑咧着的嘴骤然张大。嘴角皮肤撕裂开了,眼鼻五官也变形了,然而她的嘴继续扩张,最后竟成了个四方形状的口器。口腔之中色呈乌黑,密密麻麻的生着尖锐倒刺。苗条身体随着水流蜿蜒游动,她真的变成了一条怪蛇。赛维和胜伊偶然回头看了个正着,两人并没有尖叫,只打嗝似的在喉咙里``呃''了一声,随即如同上满发条一般,几乎把手中的船桨摇飞了。

无心依旧四脚着地的跪伏在船尾,一手撑地,一手拿住了血符。人真是不逼迫不成器,他费了一天的笔墨,成绩不如他方才的随手一画。血符在他手中生了寒气,眼看八姨太越来越近了,他忽然出手一掷。纸符平平的破空而出,竟像带有刃锋一样,斜斜的切进了八姨太的额头!

非虫非兽的``咭咭''声又响起来了,正是八姨太所发。无心知道自己画符的本事是带有抽疯性质的,时灵时不灵,所以抬手又从胸前抽出了铁针。偷眼扫向后方,他见赛维姐弟还在拼命和沉重水流作斗争,便放了心。忍痛握紧铁针,他一针戳进了自己的脖子里。虎视眈眈的盯紧水中怪物,他随时预备着拔针。

水中的八姨太仿佛十分痛苦,翻江倒海摇头晃脑,颈部的裂缝随着动作加深扩大,蔓延得四分五裂。身边忽然有了动静,无心扭头一瞧,却是赛维气喘吁吁的挤了过来:``怎么办?桨断了——''

她显然是恐慌到了极致,一张脸青白不定的没了人色。然而未等她把话说完,水中的八姨太猛一挥头,竟然颈部齐根断裂,把个头颅甩向了前方。赛维一双眼睛正望着无心,依稀感觉是有个黑球飞过来了,她的脑筋还未转过弯,双手却是不由自主的抱拳互握,以着垫球的手法向上一挺身。只听一声闷响,她把八姨太的脑袋当成排球,直接回击到了十米开外的水中。

远方溅起一朵大水花,近处水面则是暂时恢复了平静。她愣头愣脑的问无心:``我刚打着什么了?''

无心没敢说实话,扯着手臂把她往自己身后推:``船桨断就断了,你们坐在船上,千万不要乱动!''

此言一出,船尾水面``唿''的翻卷出一道黑浪。无头的八姨太在水中打了个挺,脖腔子里伸出一只油黑锃亮的尖脑袋,尖脑袋乍一看类似水蛇,然而对着船上活人一昂首,它张开了满是倒刺的四方大口,决计不是水蛇的构造!眼看它要冲向小船了,无心迎着它纵身一跃,竟是投入了水中。侧身避开了它的大嘴,无心手足并用抱住了它的身体,不让它继续冲击小船。一只手拔出深插在脖子里的铁针,他一针扎入了怪物滑腻的皮肤。

铁针本来就是一样邪恶的器物,此刻被他血肉浸染久了,会有何等效用,他也不能预料。随着铁针刺入,八姨太的身体开始在他怀中激烈的抽搐,而怪物极力的扭动脑袋,想要去咬无心。无心左右躲闪,深知一旦被对方衔住了,不但皮肉要被倒刺全部刮掉,恐怕连骨头都不能幸免。

他躲闪得机灵,怪物似乎也是个有智慧的,随着他的躲闪挣扎不止,一个水蛇似的怪头越探越长,仿佛后方连着的也是一条蛇身,正要从八姨太的身体中钻出。无心见它不败,索性拔出铁针,将铁针伸进自己口中,让针尖从舌根一路划到舌尖。用沾染了鲜血的铁针再一次扎中怪物,他同时发现怪物居然生了一双人眼。

怪物痛苦不堪,然而硬是不死。口中的血腥气越来越淡了,趁着舌面伤口还在,无心无计可施,索性横下了心,一口向下咬中了怪物的头顶。而在赛维和胜伊的惊呼声中。怪物猛一打挺,随即一条滑溜溜的细长身体滑出八姨太的脖腔子,彻底露出了本来面目,也看不出它到底是个什么,正是介于蛇和虫之间。

它显然是十分狡猾,卷缠着无心往深水里钻。然而无心并不在乎水陆的分别。除了一身帆布工人裤浸水之后有些累赘之外,他在水中并不比怪物笨拙。因为身上再无武器,所以他一针接一针的狠戳怪物双眼,同时死活不肯松口。突然猛一扭头,他用牙齿撕扯下了对方头顶的一块皮肉。黑血在水中迅速弥漫开来,他把铁针插在腿上,然后双手扒住怪物的伤口,奋力撕扯向了两边。怪物显然是疼到发狂了,翻腾盘旋着想要甩开无心,可是无心用双腿紧紧夹住它的身体。寒冷腥臭的黑血遮挡住了无心的视线,他把所有的力气都运到了手上,生生的在怪物头上挖出了一个血洞。松开双腿向上一浮,他拔出腿上的铁针,在怪物的身体上飞快画出了一道符。最后一笔向上一挑,他踩着怪物的尾巴,借力凫向小船。

``哗啦''一声水响,他在船尾露出了头。仰头忽见赛维和胜伊正直勾勾的睁了眼睛在看自己,他怔了一下,随即开始呼哧呼哧的喘。

赛维和胜伊显然是吓丢了魂,望着无心愣了足有半分钟,然后也没说话,一起出手把他拽上了船。两人的手都是出奇的有劲,像钳子似的钳住了他。他都在船上坐稳了,两人还不放手。

无心喘得很累,所以正好趁机不喘了,对着二人说道:``别怕,怪物已经被我杀掉了。''

把话说完,他背过身面对河面,凝神又向水中观察了片刻。凭着两只眼睛看,当然是看不出什么,他只是做了个凝视的姿态。水中的邪气淡了许多,散是不会散,因为死的只是一只喽啰,幕后的人在哪里,他不知道。

河水恢复了往日的平缓,赛维和胜伊费尽力量,总算是利用一根船桨横渡小河。三人互相搀扶着上了岸,一路不肯多言,像贼似的潜回了小院。

院里的老妈子和丫头都早睡觉了,朦胧中忽听房内起了热闹,但是少爷小姐不叫,她们乐得躺着装睡。而她们不露面,也正合了少爷小姐的心意。

无心一身腥臭,得到了最先沐浴的权力。他知道赛维和胜伊都是很讲卫生的,所以用香皂满头满脸的涂抹,刷牙齿的时候,也特地把舌头抻出来一起刷了刷。舌头上面一道长长的红色伤口,被牙膏泡沫刺激的很痛,他忍着痛,一丝不苟的漱口。

一个小时之后,赛维和胜伊也洗干净了,又亲自提暖壶倒开水,沏了三杯热茶。无心又没了衣裤可穿,只好套上了胜伊的睡衣。睡衣本来就是宽松的衣物,对于尺寸要求并不严格;而无心更是无所谓,如果赛维和胜伊不介意,让他光屁股也是没问题的。

赛维和胜伊也换了睡衣,并且裹了一件睡袍,仿佛穿得越多越安全。分踞左右守住无心,两人默不作声的喝完一杯热茶,心中有着无数的问题,一时简直不知从何说起。

\chapter{野火春风}

赛维和胜伊包围无心,坐成了个左右夹攻之势。一杯热茶下了肚,他们身体温暖,腹中熨帖,回首方才的惊魂记,简直如同噩梦。

胜伊抱着肩膀,看看赛维,又看看无心,两只眼睛睁得很大,是茫茫然无所依的模样。虽然他只比赛维年幼了一分多钟,不过从小到大,他的气焰总比赛维低上许多,一旦遇了困难,就要依靠赛维做主,所以如今虽然已经成了十八岁的青年,但是摇摇晃晃的,还得找个人来依附。赛维距离他稍微远了一点,他若想去投奔,就必要在床上挪动。大床铺着弹簧垫子,软颤颤的也不便于挪,于是他就近取材,一言不发的蹭到了无心身边。

他不动,赛维也不动;他动了,赛维拨动着心中的小算盘,不着痕迹的也挨上了无心。无心知道他俩全受了大惊下,有心张开双臂搂抱他们,可是犹豫着又没敢动,因为胜伊可以搂,赛维不能搂。赛维是个大姑娘。

胜伊彻底的崇拜了无心,小声问道:``你在河里\ldots{}\ldots{}把八姨娘杀死了?''

赛维立刻伸长手臂拍了他一下:``别胡说八道,谁杀她了?没人杀她!''

胜伊自知失言,立刻抬手掩了嘴。而无心思索着说道:``要说你们的八姨娘,还真不是死在了人的手里。''

胜伊恍然大悟,伸手一拍无心的手臂,又望着赛维嘁嘁喳喳:``啊,我知道了!姐,是不是花园里面有怪虫?你记不记得百科全书里面写的,有种虫子能钻进人的肛·门里吃肠子,一直把人吃空——''

赛维不耐烦的一挥手,粗着喉咙怒道:``你还能不能让他把话说完?''随即她转向无心,做出求学的姿态,三分诚恳七分天真的问道:``那么,到底是怎么回事呢?''

无心且不答话,闭上眼睛沉默片刻,及至确定屋内屋外真是一片清净了,才低声说道:``你们听没听说过`蛊术'?''

话音落下,他见胜伊把手揣进了睡袍袖子里,赛维的手倒是按在了床上,就用指尖在她的手背上一笔一划写出蛊术二字。赛维点了点头,因为太好奇,所以忘记了伪装女学生:```蛊'字我是认识的,可蛊术又是什么术?''

无心想了一想,忽然感觉八姨太的死因,是桩一言难尽的事情:``总而言之,是种巫术。一旦中了蛊,或死或生,全凭施术人的手段。依我看,八姨太就是中了蛊。''

赛维试探着问道:``中了蛊\ldots{}\ldots{}人就变成大水蛇了?''

无心摇了摇头:``非也,是蛊虫在她体内生长,吃空了她。我们所见的八姨太,其实只是一只裹着人皮的怪虫。''

胜伊抬手抓了抓短发:``八姨太\ldots{}\ldots{}是怎么吃下一条大虫子的?''

无心被他问笑了:``不是不是,也许怪虫在进入八姨太体内之前,只不过是一点粉末。八姨太无意之中吸进一点粉末,总不会有知觉,对不对?可粉末遇了血肉,就要变形了!''

赛维惊讶的张了嘴:``有点像中毒啊!''

无心微微的歪了脑袋,想要用睡衣领子遮住脖子上的针孔:``你们说八姨太是两三天前失踪的,失踪之前并无异状,可见她是新中的蛊。而蛊虫又是长到如此之大,两三天的时间都算是少的,可见中蛊和失踪,发生的时间即便不是同时,也该相近。''

赛维深以为然:``可是,她怎么就中了蛊呢?''

无心沉吟了片刻,末了低声说道:``我猜,八姨太和令堂,是死在了同一人的手里。''

赛维和胜伊立刻全变了脸色:``我娘也是中蛊?''

无心一摇头:``不,令堂的死,或许和蛊毒没有关系。但是令堂头内的铁针,却和水中的怪虫有点相似的气息。应该是施术的人把两种巫术混在了一起使用。现在我只想一个问题——八姨太会是在哪里中的蛊毒?''

赛维答道:``应该不是在家里,在家里中了毒,她还不得去医院?''

胜伊随即接道:``我看就是在花园里。''

赛维立刻表示反对:``白天花园里全是花匠,也没见谁肚里生出大水蛇了!''

胜伊来了精神,开始辩论:``哦,八姨娘在外面中了蛊,还坚持跑到花园里等死,她疯啦?还是她肚里的大水蛇想看风景,裹着她的皮自己跑去了花园?''

无心最后做了总结陈词:``有一种蛊,是用阴魂的邪气催动蛊虫,蛊虫的性子,就类似鬼。河水属阴,利于蛊虫的隐藏;白天它蛰伏着不动;一到夜里,阳气散尽,它就活了。下蛊的人将它布放好了,一旦有人冲了它的布阵,就必定中毒。''

赛维和胜伊相视一眼,脸上立时退了血色,异口同声的喃喃说道:``八姨娘\ldots{}\ldots{}夜里去花园了?''

然后他们立刻联想到了自身——自己不也是夜里去了花园?

无心拍了拍他们的膝盖:``没事,若是你们也中了蛊,就像八姨太一样直接失踪了,蛊毒凶猛至极,还能让你们活着回来吗?''

赛维打了结巴:``谁谁谁下的蛊蛊毒害人呢?花园子里到到底有有什么?''

无心压低声音说道:``花园的秘密,令堂知道,八姨太可能也知道。还有没有第三个人,我们暂时猜测不出,所以姑且按兵不动的看吧!对方要用邪术对付你们全家,可见花园里的秘密不一般,而且他的仇恨也是十分之深。''

赛维和胜伊一起开动了脑筋想仇家,想了片刻,忽然发现自家仇家很多,自己老子的名声也一直不好,做过许多缺德事情,前些年还遭过一次暗杀。

无心不再多说,伸腿下床走去外间。片刻之后,他端着一杯水回来了。单腿跪到床上,他对着面前二人说道:``虽然你们的肚子里肯定不会长出虫蛇,但我还是不大放心。你们把它喝了,喝了就绝对安全了。''

胜伊先爬到了床边,跪起身探头一瞧,就见杯中是大半杯红水,因为水热,所以还散发出一股子又甜又腥的蒸汽。甜和腥凑在一起,虽然不是好滋味,但也不该让人不能忍受;但是无心杯中的饮料就是甜腥得令人感到恶心,甜不是好甜,腥不是好腥。

胜伊当即一咧嘴,捏着鼻子问道:``什么东西?''

无心坦然的答道:``水里面搀了我的血。我的血\ldots{}\ldots{}很好,哪怕你真中了蛊,喝一口也能解毒。''

胜伊连连后退:``我、我不想喝。''

赛维四脚着地的爬到无心身边,跪起来接过茶杯,仰起头就喝了一大口,差点没烫出眼泪。屏住气息转向胜伊,她缓缓呼出了一口气,口鼻之中的甜腥差点让她当场呕吐。勉强定了定神,她凶神恶煞的斥道:``快来喝!''

胜伊抗命不从,结果被无心拽过来从后方抱住了,伸手强行捏开了他的嘴。赛维的手脚很利落,把余下半杯血水尽数倒入胜伊口中。胜伊咕咚咕咚几口咽下,想要吐,然而赛维放下茶杯捂住了他的嘴,无心禁锢着他也不松手。两人合作摆布他一个,直过了十分钟才给他自由。而他干呕几声,恶心劲过去,也就不吐了。

赛维想要看看无心放血的伤口,然而无心遮遮掩掩,并不让看。电灯一关,卧室陷入黑暗。三人凑在一张大床上,不敢拆分。把两床被子全展开了,也没有人正经盖被,三个人偎做一堆,糊里糊涂的就闭了眼睛。

赛维累狠了,连个噩梦都没有做,再一睁眼就到了天光微亮的凌晨。清醒之后她没有动,细胳膊细腿缩在软腾腾的棉被里,感觉十分温暖舒适。及至打出一个哈欠了,她才发现自己是个半躺半坐的姿势,结结实实的全靠在了无心胸前。

翻着眼睛向上望去,她见无心还在熟睡,歪着身子压住了胜伊,胜伊团成一只球,埋头挤在了床角落里。胜伊的姿势不对劲,气息不畅,睡得呼哧呼哧;无心则是喘得有一搭没一搭,胸膛半天起伏一下,仿佛随时预备着断气。

赛维没有多想,保持着原样不肯动,心旷神怡的睁大眼睛往窗外望,望了没有几分钟,她忽然一挑眉毛,把注意力全集中在了左手心里。

有一条半软半硬的东西,隔着一层薄薄的丝绸,热烘烘的贴上了她的左手心。她缓缓的垂下眼帘,隔着一层棉被去看自己左手的位置。头脑里骤然发生了大爆炸,她发现自己竟然把左手搭上了无心的裤裆!

左手,连同左臂,登时就僵硬了。她惊慌失措的闭了眼睛想要装睡,同时在心中发出了大感叹:``天哪,原来\ldots{}\ldots{}这么大!''

未等她感叹完毕,手下的东西忽然跳了一下;无心随之一动,鼻子里还哼了一声。

赛维当即紧闭双眼,做睡死状。

她睡了,无心却是醒了,然而睡眼惺忪,醒得不透。他先掀开了身上的棉被,然后对着被里风光愣了一下,随即轻轻握住赛维的手腕,把她的左手抬起来放到了一旁。

轻手轻脚的挪下床去,他摇摇晃晃的出去撒尿。而赛维偷偷在被窝里右手摸左手。左手的手心像是被一条烙铁烙过了,灼热的一线从腕子开始延伸,一直向下经过中指,正是一段很可观的长度。赛维对于男女之事,一直只是通过爱情小说纸上谈兵,如今终于见识了真家伙,不禁心跳如鼓,并且满头满脸的发烧。耳边传来踢踢踏踏的脚步声,是无心趿着拖鞋回来了。

赛维缩在棉被里,一动都不敢动。而无心在床边伸展身体躺下了,很舒服的伸了个懒腰,两条腿不慎伸过了界,隔着棉被蹬上了赛维的小腿。他很自觉,双脚立刻转移了方向;而赛维等着他再蹬一下,等来等去等了个空,就在被窝里暗暗叹息:``男大当婚、女大当嫁。看来,我真是长大了。''

下一秒,她的叹息换了主题:``真吓人,那么长!''

赶在老妈子丫头进房伺候之前,三个人都起了床。赛维谨记了按兵不动的战术,若无其事的支使仆人去成衣店。三天前,她把无心的尺寸送了去,只不过是做几套普通衣裳,三天时间,又是马家的买卖,怎么着也该完工了。

赛维和胜伊都坐在房内没出门。一个小时之后,仆人带着新衣回来了,顺便还报告了一条新消息:``咱们家的花匠,在河边发现了半截旗袍后襟,都说像是八姨太的衣裳。五少爷倒是奇怪,不哭不闹,听了好像没听见似的,让他去瞧瞧,他瞧过了也不言语。''

胜伊过去接了新衣,为了掩饰脸色,所以故意忙着审视新衣料子;赛维手里攥着一把尺子,已经若有所思摆弄了一早晨,此刻不摆弄了,蹙着眉毛摇头叹气:``我们家里近些天来,真是没法说,糟糕事情全赶在一起了!''

然后她摸了几张钞票扔给仆人,把仆人高高兴兴的打发走了。

\chapter{虚惊一场}

无心换上了赛维给他订做的新西装,西装料子非常的好,是绸缎庄子不知从哪里偷运的英国细呢,市面上有钱都没处买的,非得马家厉害的二小姐才能要到。褐色细呢在阳光下,反射出隐隐的紫光,配着里面的白绸衬衫,看起来是特别的绅士派。胜伊是位爱美的青年,在卧室里面一边指导无心穿西装,一边暗暗的有些嫉妒,因为褐色呢子不适宜做女装,如果没有无心的话,赛维一定会把好料子让给他的。

``你们昨晚上一起欺负我!''他对着无心嘀嘀咕咕:``我看你根本就不是什么和尚,你是个巫师。''

然后他顿了顿,很幽怨的又加一句:``坏巫师!''

无心低头系好了腰间皮带,随即抬头对他一笑,轻声说道:``别生气啦,我也是一片好意!''

胜伊蹙着两道平平的眉毛,因为对无心还是有些崇拜和依赖,所以也就不计较了。

无心穿戴整齐了,推开卧室房门往外走。赛维正盘腿坐在外间的罗汉床上发呆,此刻闻声望向了他,不禁呆了一呆。而无心笑着一点头:``西装很好,多谢你。''

说话之时,他也走到罗汉床前坐下了。自己低头看了看脚上的新皮鞋,皮鞋锃亮的能照人影。看过之后抬起头,他对着赛维又是一笑,笑得没有什么意味,仿佛就只是在高兴。

褐色西装与天蓝领带,是个鲜明的对比;白皙皮肤与乌浓眉目,又是一个鲜明的对比。赛维对无心注视了片刻,只感觉他俊美得刺眼,并且把自己衬托的面貌模糊。不置可否的把脸转向窗外,她无声的吁出一口气,然后心中暗道:``倒贴也值了。凭着他的好模样,我要是不倒贴,也未必拿得住他!幸好我马二小姐倒贴得起,不在乎白养个丈夫。回头得去整理整理我的银行折子了,现在银行也不靠谱,说冻结就冻结。盛世古董、乱世黄金,改天和胜伊商量商量,把娘的体己钱取出来买金子得了。胜伊要是不同意怎么办?哼,他懂个屁,敢不同意,打不死他\ldots{}\ldots{}''

赛维的脑子里像是在过大兵,乱哄哄的不消停。忽然又瞟了无心一眼,她见无心脱了皮鞋,已经跪坐在了床上。刚穿的新裤子,就往床上跪,非把裤子膝盖顶出两个大包不可。有好衣裳也穿不出好样,于是赛维心中又想:``真不讲究,需要教育。''

赛维满心都是一个无心,想得太过于入神了,以至于半天没搭理无心。胜伊自己出去溜达了一圈,末了带着一身凉气回了来,进门就道:``八姨娘找到了!''

赛维看他说起话来不管不顾,嗓门还不小,就急得向他使眼色。胜伊满不在乎的摇了摇头,自顾自的继续说道:``是花匠老陈的儿子找到的!老陈他们在山上干活,他儿子在河边钓鱼,结果勾出一具尸首!''

他越说越可怕,引得外面的仆人都跑了过来。赛维见状,立刻做出难以置信的惊愕表情,无心则是悄悄的躲进了卧室。胜伊对着周围听众,继续绘声绘色的讲述:``你们可千万别去看热闹,哎哟吓死人了。八姨太的脑袋没了,腔子里面的五脏六腑也被鱼吃空了,就剩了一层皮,像个皮袋子似的。俊杰刚被人叫去了,都说不该让他看,怕吓坏了他,但不让看也不行啊,八姨娘毕竟是他亲娘不是?''

听众得了这样一个可怕的消息,全都面目失色,并且联想起了二姨太的猝死,心头不禁全蒙了阴云。而赛维当众问道:``花园子里是谁看着呢?家里接二连三的出坏事,爸爸又不回来,唉\ldots{}\ldots{}''她站起来一跺脚:``你也别光顾着传消息了,现在家里顶数我们两个是姐姐哥哥,再怎么恐怖,我们也得去瞧瞧啊!该报警得报警,该调查得调查,好好的八姨娘,难道就糊里糊涂的让她没了不成?''

姐弟二人一唱一和,果然驱散众人,一前一后的出了去。及至走出院门了,赛维见旁边没人,才轻声说道:``大哥不在家,数我们说了算!我们不能守在屋子里坐以待毙。一旦让我抓了把柄查出凶手,他不杀我,我还要杀他呢!''

胜伊心悦诚服的跟了上:``姐,我早说过,你就是块巾帼英雄的料。你说得对,死瘸子不在家,我们就算老大,我们也站出去管管事,不能全凭着人家在暗处摆布我们。''

赛维感觉胜伊说话特别没有水平,所以只一摆手,示意他闭嘴。

两人快步赶到花园河边,就见河边围了一圈壮年家丁,家丁之中摆着一副担架,担架上面苫了白布,白布下面有所起伏。其中花匠老陈是个有年纪的人,见赛维和胜伊来了,就苦着脸向他们一弯腰,低声唤道:``二小姐,三少爷。''

赛维伸手一指担架,正色问道:``是\ldots{}\ldots{}八姨娘?''

老陈答道:``二小姐三少爷也都知道了?五少爷亲自认过了,说真是八姨太。''

八姨太是个花蝴蝶似的人物,衣饰一贯花里胡哨、与众不同,饶是没了脑袋,也依旧存有特征。赛维拿出管家人的气派,走到担架前蹲下来,不等旁人说话,径自掀开白布向内一瞧。瞧过一眼之后,她拧着两道眉毛起身退了一步:``俊杰呢?''

老陈在一旁答道:``五少爷回去了。''

赛维想问他俊杰哭没哭,但是将问未问之时,又把话忍了回去,因为感觉问得不对劲,不如不问。正当此时,马家的管家颠颠跑来了,气喘吁吁的想要派人去给大少爷打长途电话,赛维立刻说道:``找他干什么?若是操办葬礼,当然是需要他来主持;可八姨娘死得不明不白,怎能随便就埋葬了事?她可是生儿育女的人,不是一般的姨娘。我看报警也不大好,毕竟八姨娘死得怪异,传扬出去,对我们马家也是不利;不如想办法保存了她的尸体,等爸爸回来再做定夺吧!''

管家一直知道二房的孩子不是省油的灯,也承认八姨太的确是死的太蹊跷。照理来讲,大少爷作为家中长子,自己不能不通告他一声;但是话说回来,大少爷和老爷乃是一对死敌,让大少爷为二姨太主持葬礼,或许没问题,横竖二姨太死得光明正大,葬礼也是光明正大;但八姨太就不同了,八姨太不是好死呀!

赛维就是不让八姨太下葬。谁想埋八姨太,将来谁就去向马老爷做交待。大家都是下人,谁敢负这种不清不楚的责任?于是一番安排过后,八姨太被运去医院,冷冻住了。

保护了八姨太的尸体,到底能有什么作用,赛维也不知道。但她想既然凶手上次烧掉了娘的尸体,可见尸体对凶手来讲,绝不会有好处,以至于让对方必定除之而后快。凶手想毁灭尸体,自己便故意保护尸体。对方连自己的娘都杀了,自己还不敢唱出几声反调吗?

赛维和胜伊在花园内耽搁许久,最后见场面全被自己控制住了,才满意的打道回府。不料刚一进院,就听丫头禀告,说是五少爷来了。

马俊杰素来性情孤介,而且年纪又小,和哥哥姐姐们都谈不拢,平日一贯独来独往。但今时不同往日,赛维独自快步走去上房,就见马俊杰端端正正的坐在一架沙发椅上,眼泡红肿,分明是哭过一场。

赛维见了他的模样,不禁想起了自己的娘。在他面前的长沙发上坐下来,她叹了一声:``俊杰,二姐不说空话安慰你了。我们都没有了娘,爸爸又是做大事的人,不会细致的关怀我们,往后的冷暖,全靠我们自己疼爱自己。可我们越是悲痛,越要振作。否则我们的娘到了天上,惦念着我们,也不得安息啊。''

马俊杰翻了她一眼,随即却是哑着嗓子低声问道:``二姐,你说我娘是怎么死的?''

赛维听他肆无忌惮的说``死'',语言一点儿也不柔和,就感觉有些刺耳:``我不知道。''

马俊杰欲言又止的张了张嘴,紧接着一挺身站了起来:``家里有鬼,大家都小心着吧!''

然后他绕开面前的小茶几,迈步就走了出去。赛维回头盯着他的背影,心中暗想:``小小年纪装神弄鬼,真是不讨人爱!''

赛维知道凶手躲在暗处,所以想要把家中一潭深水搅浑。要遭殃,大家一起遭,谁也别想逃。心事重重的回了厢房,她在卧室外面的小房间里,看到了无心。

无心穿得漂漂亮亮,然而姿态并不漂亮,正大喇喇的蹲在地上整理他的破旅行袋。用一张白纸仔仔细细的包好铁针,他显然是想要把针收藏起来。赛维在他面前,扭扭捏捏的也蹲下了。无心抬眼看她,又小声问道:``没事吧?''

赛维摇头答道:``没事。''

然后她垂下眼帘,忽然发现帆布袋的夹层口袋里,露出了相片的一角。下意识的伸出手,她飞快的抽出相片定睛去看——看过之后,她登时就面红耳赤了!

相片乃是无心和一个女人的合影,两人肩膀挨着肩膀,脑袋碰着脑袋,笑眯眯的别提多么甜蜜了!赛维明知道自己和无心之间既无表白,也无承诺,根本就是没有关系;可是心头翻出一股子酸醋,她简直要暴怒了。

暴怒归暴怒,暴怒在心里,还没有波及到神情。把照片向无心一递,她开口问道:``你不是一直做和尚吗?怎么还和年轻女人一起照相?''

无心扫了相片一眼,仿佛是被她问怔了,迟疑着没有回答。赛维在心中冷笑一声,故意追问:``照片上的人,是你吧?''

无心缓缓的一点头,声音犹犹豫豫拖得很长:``是我\ldots{}\ldots{}的\ldots{}\ldots{}爹!''

赛维大吃一惊:``啊?''

然后她低头再仔细看照片,心里登时透了光明——照片已经旧到泛黄,周围也都磨出了毛边,要看历史,至少也得有二十年了。

不由自主的翘起嘴角,她笑着收回了照片,饶有兴致的细看:``哈!你和令尊也太相像了,简直就是一个人嘛!旁边的女士,一定就是令堂了,对不对?''

无心继续用报纸包裹铁针,同时点了点头:``嗯。''

赛维方才忽然极愤慨,此刻又忽然极欢喜,捏着照片看个不够:``无心,令堂年轻的时候很美呢,可是你一点儿也不像她!''

无心低头把裹好的铁针放进帆布袋里:``嗯。''

赛维笑着看照片,照片上的女人穿着大镶大滚的老式衣裳,没型没款的,全靠一张面孔显露姿色,脸是丰润的苹果脸,笑得欢天喜地,倒是过去照片里少见的神情。在赛维的印象中,父母年轻的时代真是太久远了,家中存有的旧照片里,人物统一都是木讷呆板的神情,大概是因为当时难得照相,太过紧张。

旧照片不是很清楚,赛维看一眼照片,再看一眼无心,看得心花怒放。原来只是虚惊一场,真真吓死她了。

\chapter{夜影}

八姨太进了医院的冷冻柜,也不知道是算死算活。照理来讲,连她的亲生儿子都确定了她的身份,似乎也就没有什么疑问;可她毕竟死得怪异,又没了脑袋,马俊杰的辨认是否是百分之百的可靠,便藏了一个隐隐约约的问号。赛维通过了马老爷的秘书,想要联系到远在日本的父亲,可是几封急电发出去,只得来一封内容漠然的回信,仿佛马老爷正在日本忙大事,公务缠身,已经顾不上几个姨太太的死活了。

老爷对于姨太太,都是不讲感情;家里除了马俊杰,旁人自然是更不动心。转眼间又过了风平浪静的十几天,这天早上胜伊起了床,一眼看到站在地上的无心,登时气得叫了一声:``谁让你把头发剃了?''

无心站在床前,脖子上搭着一条白毛巾,头上脸上全都热气腾腾的,青白头皮被剃刀刮得光溜溜。扭头对着胜伊一笑,他拽了毛巾满头满脸的擦水珠子:``剃了舒服。''

胜伊如今和他住在卧室对面的西厢房里,因为胆子小,所以时常和他挤做一床。气急败坏的一捶床,他伸腿下去找拖鞋:``我让姐来瞧瞧你!昨天还说你的头发不大长呢,今天可好,你索性剃成光头了!秃头秃脑的好看吗?''

胜伊把赛维找了过来,赛维怀着鬼胎,当场问道:``你还存着要去当和尚的心思吗?''

无心歪着脑袋,很细致的擦着脖子:``我是从小剃惯了,不剃难受。当什么和尚,我到哪儿当和尚去?''

赛维听闻此言,心中立时放下一块大石。和她一起暗暗松了口气的,是无心。

三人相处也有一个多月了,他天天过得提心吊胆,梦里都怕自己忘记呼吸。赛维和胜伊昨天都说他的头发太短,一个多月了,怎么就不长呢?

他无话可答,并且知道再过一个月,头发的长度也还是不会有变化。头发的长短当然只是极小的事,不过他的异常也就体现在小处,时间长了,总要露出马脚。

头发的公案告一段落,赛维自去梳洗打扮,然后也不带人,自己挎着只锃亮饱满的漆皮包乘车出门,直到天黑方归,漆皮包被她夹在腋下,竟然是快要胀开的光景。

当着胜伊和无心的面,她把门窗都关严了,然后打开皮包,从里面一扎一扎的取出美钞。美钞全都崭新整齐,她故意要让无心看清,表示自己虽然没有十分的姿色,却有十分的资产,就算瞧在钞票的面子上,你也不能不高看我一眼。

胜伊傻了眼:``姐,你从哪里换来的?现在北京城里还有美钞?''

赛维一挑眉毛:``你没朋友,我也没朋友吗?苏太太在牌桌上赌疯了,把战前积攒的美钞当金子卖,我就买了。日美不管怎么打,美国至多是不赢,总不会亡国。我告诉你,在大后方,美钞比金子还值钱呢!''

话音落下,她得意的瞄了无心一眼。无心坐在不远处的一把椅子上,胳膊肘支了桌面,正在托着下巴旁观微笑,也不问她,也不夸她。她等了良久,看他始终是个哑巴,就忍无可忍的向他问道:``怎么样?我还算有点办法门路吧?''

无心点了点头,笑容虽然是至真至诚,不过总像是隔着距离,有点事不关己的意思,见了美钞,眼睛也不放光。

赛维不禁有点失望,心想难道我有钱也不能打动你吗?况且我不只是有钱,论知识我是中学毕业,论年纪我是十七八岁,论相貌我也不丑陋,你为什么不像胜伊一样凑到我身边来呢?

思及至此,她又重重的看了无心一眼。不知怎的,心中一阵沮丧气苦,像受了天大委屈似的,简直将要落下眼泪。而无心一直是倚着桌边稳坐,忽然见赛维变了神情,便一转身面对了她。两只胳膊肘架在膝盖上,他俯身探向了她,没说话,只用一双乌溜溜的大眼睛看她,是个探究而又关切的姿态。

但是赛维无话可说,只勉强一笑,随口找了话道:``我也去剪了头发。''

她的确是在理发店剪掉了焦黄的发梢,把头发收拾得乌黑柔顺。女为悦己者容,可是她也不知道无心能否看出自己的心意。

``你可别逼急了我。''她低下头,望着美钞想道:``逼急了我,我可什么都敢说,什么都敢做!有钱的男子可以三妻四妾,我的实力,并不比男子差什么。我只要你一个,你不同意也不行!''

赛维很细致的收好了美钞,正要起身玩笑几句,不料无心忽然轻声说道:``我想再去花园看一看。''

赛维登时瞪了眼睛:``去花园?多么危险,不要去了!''

无心起身笑道:``我一个人去,你们在家等我。真有危险,我不会逃吗?''

然后他就开始预备换身粗糙衣裳出门。赛维左拦拦不住,右拦也拦不住,想要跟着他去,他又坚决不许。而在赛维气得青筋迸出之时,他自顾自的真溜了。

走过一遍的路,他只要肯认真记忆,便能记得丝毫不差。自从二姨太八姨太死亡之后,马宅上下人心惶惶,无须天黑,天色一暗就都各归各位,全不敢乱窜。无心提着百般的小心,一路穿花拂柳的往后方走。走着走着,他脖子上忽然凉阴阴的有了感觉,扭头一看,他和小健正好贴了个脸。

他的耳朵穿过了小健的幻影。转向前方继续前行,他压低声音问道:``我还以为你被人收了。''

小健像个骄矜的小儿子似的,用双腿夹住了他的脖子。血淋淋的小拳头举起来,他用力去捶无心的脑袋:``你还认识我吗?我不见了,也没见你找我!''

无心笑了一声:``小鬼难缠。''

小健的拳头也是幻影,他在人间,永远都是没着没落。他想和无心在一起,可无心是明显的对他没兴趣。他打算惩罚无心一下,又没有惩罚对方的力量。正在他愤慨之时,无心忽然放缓了脚步,因为前方花木黑影层层叠叠,已经到了花园地界。

鞋底踏过枯草,碾出细微的声响,几只垂死的秋虫还在暗中鸣唱。天空斜吊着一勾白森森的弯月,无心闭上眼睛,感觉四周并不太平。

步伐越来越轻了,他试探着往园子里走。小健不知何时又消失了,只在他的后脖颈上留下一抹哀伤的寒意。踏上石板铺就的小径,他无声无息的直奔河边。然而在距离河边还有三五米远的地方,他猛然刹住了步子。

前方,在紧挨河边的一丛花木之侧,刚刚闪过了一个黑影。黑影是个中等身量,一闪而逝,看不出男女,无心只听到窸窸窣窣的一串脚步声响,想要之时,河边已经恢复了平静。

无心蹲在一丛灌木后方,静等许久,末了感觉河边的确是再无活物了,才四脚着地的趴伏下去,贴着花木丛向前爬去。爬了没有多远,他抽抽鼻子,却是嗅到了一股子奇异的血腥味。

血腥味淡极了,而且并不纯粹,显然是从河边飘过来的。河水里面即便是存了臭鱼烂虾,气味也不会如此的古怪;无心伏在地面思索片刻,末了慢慢向后退却,不肯继续前行。

耳边响起了小健的声音:``大哥哥,有个穿黑衣服的人,刚刚跑到林子里去了。''

无心点了点头,随即轻声说道:``小健,你玩归玩,可是夜里千万不要靠近小河。有人对河水动了手脚,我怕你会受害。''

小健听了他的话,感觉他对自己似乎还有一点好意,就唧唧哝哝的在他耳边说道:``我在外面逛了好久,去撞大肚子女人。我想钻进她们的肚皮里去,做她们的小孩。可是,都不成功。''

无心退到了一定的程度,才站起了身:``也别强求,顺其自然吧!''

小健搂住了他的大腿:``可是你又不喜欢我,嫌我是鬼\ldots{}\ldots{}''

无心转身往回走:``我自己已经是活得半人半鬼,再喜欢鬼,岂不是更不成人了?''

小健忽然向上一窜,在他的颈侧消失无踪,只把声音送进了无心的耳中:``大哥哥,你身边有人。''

无心扭头一瞧,就见路边一棵枝叶萧索的矮桃树中,竟然当真坐了一个静静的黑影。迎着无心的目光,黑影发出了老气横秋的童声:``是我。''

无心不动声色的一点头:``原来是五少爷。''

马俊杰很巧妙的藏在了桃树枝杈中,一动不动的和桃树融为一体:``你是谁的人?''

无心不假思索的答道:``我是你二姐三哥的朋友。''

马俊杰在暗中又问:``你来花园干什么?''

无心一笑:``没有五少爷来得早。''

然后他眼见四周无人,大踏步走到桃树前,伸手一把扯住了马俊杰的西装领子。马俊杰还未惊叫出声,已然被他不由分说的拖拽落地。马俊杰毕竟是个孩子,根本不是无心的对手;而无心拦腰抱起了他,一溜烟的跑出了花园地界。他不得自由,两条腿乱踢乱蹬,又要抬手去打无心的头脸,然而在他动手之前,一张惨白的小面孔忽然凌空探到了他的面前,正是小健对他做了个鬼脸。

马俊杰倒吸了一口冷气,当场收回双手捂住了嘴,闷声闷气的尖叫了一嗓子。

无心找了个僻静角落,把马俊杰放在了地上。双手紧紧握住他的肩膀,无心蹲下来,看着他的眼睛正色说道:``我告诉你,河水里面被人下了蛊,你再敢夜里往花园里跑,当心下场会和你娘一样!还有,赛维胜伊都不是坏人,他们没了娘,还不知道该找谁报仇去呢,你根本无须鬼鬼祟祟的盯着他们!''

马俊杰张了张嘴,欲言又止的没出声。

无心握住他一只手,扯起了他往前走。小健看他比自己大不许多,也是个孩子,就讪讪的飘在一旁,又特地把比较完好的一侧面孔显露给他。马俊杰果然不住的瞟他,心想自己今夜是真见鬼了。

及至距离花园足够远了,无心松开了手,对他说道:``你回去吧,以后如果有了麻烦,可以找我。''

马俊杰翻了他一眼,随即撒腿就跑,瞬间没了影子。无心知道他是个沉默的小人精,所以也不担心,若有所思的往家里走。走着走着,他对身边的小健说道:``去跟上他,小心一点,别伤害他。''

小健终于得了一桩任务,立刻欢喜的答应一声,在夜空中散了影子。而无心快步走回所居的小院,哪知刚一进门,胜伊就迎了上来,低声说道:``你总算回来了。不让你去,你非去,我姐生气啦!''

\chapter{恋爱关系}

无心先是换了一身干净衣裳,然后蹑手蹑脚的穿过院子,推开了东厢房的房门。东厢房卧室是一间半的格局,电灯通亮。赛维还穿着白天的洋装皮鞋,一头乌黑短发胡乱掖到耳后,脸倒是洗过了,不施脂粉,皮肤透着一点营养不足的黄色,倒是很光滑细腻,能够反射灯光。独自坐在罗汉床边,她沉着脸低头翻阅一本杂志。身旁床上摆了一架红木小炕桌,桌上是一壶咖啡,一碟子奶油蛋糕,大概就是她的夜宵了。

胜伊从来不是姐姐的对手,所以干脆上床睡觉,不来涉险。而无心见赛维冷着一张单薄的小黄脸,对自己视而不见,真是动了大气的模样,就陪了百分的小心,走到罗汉床前深深的弯下腰,轻声说道:``我回来了。''

赛维没理会,神情硬得像雕塑,充耳不闻的翻过了一页书。

无心在女人面前、尤其是年轻可爱的女人面前,一贯没有脾气。赛维生气,他不生气。有心伸手碰赛维一下,可他又犹犹豫豫的不敢出手,毕竟人家是大姑娘,和自己又没什么亲密关系,自己说碰就碰,有可能招来一个大嘴巴。

绕到赛维另一边,他把腰弯得越发深了:``你是在担心我吗?我只是去看一看,没有冒险。''

赛维面如铁板,就恨他不听自己的话。当然他没有对她听话的义务,但是赛维对他另有一番一厢情愿的高要求,他不听话,她就生气。

无心好些年没和女人亲近过了,此刻不禁有些手足无措。两只眼睛紧盯着赛维,他慢慢的蹲了下去,口中喃喃的又道:``胜伊说你刚才发了脾气\ldots{}\ldots{}''

他蹲稳当了,仰着脸去看赛维:``别生气了,我向你赔礼。''

赛维又翻一页杂志,心里有主意得很,就是不理他。

无心静静的蹲在她的腿边,缓缓的把头垂下了,半晌不言语。屋内寂静久了,赛维忍不住斜瞟了他一眼,不料他就像头顶心长眼睛了似的,立刻抬头迎了她的目光。欲言又止的抿了抿嘴,他浅浅的吸了一口气,然后抓住时机微笑道:``我错了,对不起。''

赛维硬着心肠,把目光收回到了杂志上,同时就瞥见无心站起了身,端起咖啡壶,轻手轻脚的往空杯子里倒了大半杯温咖啡。无声的放下咖啡壶,他把杯子往赛维一边推了推,又道:``夜深了,是不是该睡了?''

赛维合上杂志,用眼皮一撩无心:``知道我要睡了,还给我倒咖啡?''

无心听她总算开了腔,就知道她的怒气至少是开始消散了。隔着一张小炕桌,他也静静的坐在了床沿上,只听赛维继续说道:``我知道,我也没有什么资格对你发火。''

虽然她是气话,但是话中蕴藏着的意味感情就复杂了。无心扭头看了她一眼,然后笑着转向前方,垂下眼帘对着地面说道:``你有。要说没有,也是我没有。''

赛维心中一动,立刻转向了他:``你没有什么?''

无心给了她一个含羞带愧的微笑侧影:``我什么都没有,你是知道的。''

赛维很不好惹的翻了个白眼:``随便你有没有,我又不要你什么!''

无心扭头正视了她,看了片刻,最后却是苦笑着低了头,又叹息了一声。赛维的心意,说到此处,已经是极端的明了,可是他的秘密,又该如何出口呢?

赛维看了他的行为,也摸不清他是愿意还是不愿意。难得今夜有了机会,她索性紧逼一步,把话挑明:``无心,你吞吞吐吐的,到底是什么意思?男子汉大丈夫,有话就说。我并非心口不一的人,希望你也坦诚痛快一点。直说了吧,我和你很投脾气,愿意与你建立一份长远的感情,你呢?''

无心没想到她忽然采取了单刀直入的方法,不由得有些懵:``我\ldots{}\ldots{}''

赛维伸手拍了拍身边:``你过来坐,我们又不是开会谈判,隔着桌子干什么?''

无心站起来,乖乖的走到了她的身边坐下。赛维的一只手就搭在腿上,他微微歪着头,伸手想要去握一下,可是手都伸到半路了,却又迟疑着停顿了:``赛维,我对你是\ldots{}\ldots{}高攀不起。''

赛维一把抓住了他的手:``我现在和你谈的是感情问题,不是阶级问题。''

无心握着赛维的手,赛维的手瘦瘦的,皮肤很软,骨头很硬。两人的手指相扣,是个纠缠不清的样子。

``赛维\ldots{}\ldots{}''他凝望着两人交握的手,同时轻声开了口:``感情方面,我没有任何问题。可是感情之外的方面,我们也不能完全不考虑。说句实话,你并不了解我。''

然后他抬眼望向赛维:``你肯爱我,我真是受宠若惊。等到大家平安度过眼下的风波之后,我会把我的故事讲给你听。听过之后,你再做决定。''

赛维的脑子里忽然拉起了警铃:``你有什么故事?是遭了通缉?还是结过婚了?''

无心立刻摇头笑道:``不是不是,我没犯法,也没结婚。''

赛维当即松了一口气,心想他的故事,大概就是一个``穷''字。念头忽然一转,她又起了疑心:``你是在搪塞我吗?你实话实说,如果你不喜欢我,我也不会强人所难!''

无心揉搓着赛维的手,心中茫茫然的,不知道自己还能有几次机会和她亲近。抓起她的手贴上自己的面颊,他低声说道:``赛维,我是你的。只要你肯要我,我就是你的。将来或许有一天,你会怕我躲我。赛维,不用怕也不用躲,你不要我,我就离开。''

赛维歪着脑袋凝望了他,两只眼睛透出了光彩:``你说什么?你再说一遍,你是谁的?''

无心在淡淡的雪花膏香气中,正视着她答道:``我是你的。''

赛维听清楚了,竟比听到``我爱你''三个字还要满足。心花怒放的粲然一笑,她像不知道怎样才好了似的,单只是笑。无心也笑了,笑得不甚踏实,因为感觉赛维和自己根本没有结合的希望。结合了,是长的美梦;不结合,是短的美梦;无心不敢多想,总之赛维此刻是爱他的——有一个女人,爱上他了。

赛维在爱情上取得了阶段性成功,十分狂喜,立刻感到了饥饿。在房间里点起火酒炉子,她想要煮一点米粥吃。无心不劳她发号施令,直接就自动的点火倒水,出去取米。不过片刻的工夫,火酒炉子上的小锅里咕嘟出声了,炕桌上也摆了四个小菜碟子。赛维盘腿坐在罗汉床上,心满意足的笑道:``干嘛呀?我是不要男朋友伺候的。''

说到``男朋友''三字,她像饮了一口蜜一样,满嘴甘甜,一直美到了心里去。无心也笑了,只盼将来真相大白,她不要恨自己是个骗子。

赛维像只欢天喜地的鸟,叽叽喳喳的嚷着饿,可是啄了几口热粥就饱了。两人再纠缠就算彻夜了,于情于理都该各自回屋休息。赛维遂了心愿,打着哈欠回了卧室。无心横穿小院进了西厢房,东西厢房格局相同,西厢房外面的半间屋子里摆着沙发茶几。无心摸着黑进了屋,见沙发上光溜溜的没放被褥,就决定进里间去和胜伊挤一宿。

他上床时惊动了胜伊,胜伊厌烦天下一切男性,唯独对他不嫌,迷迷糊糊的问道:``她好了吗?''

无心小声答道:``好了。''

胜伊翻身背对了他,含含混混的又问:``没打你吧?她打人可疼了。''

无心梦游似的躺下去,扯过半边被子盖住了身体:``没打,睡吧!''

胜伊打了个呼噜,重新坠入梦乡。无心辗转反侧,却是难眠。他是喜欢女人,可是从来没有打过赛维的主意。睁着眼睛发了许久的呆,最后他往被窝里一缩,决定不想了。反正赛维肯喜欢他,哪怕只喜欢一天,也是他的幸运。

无心睡得晚,醒得却早。昨夜他心中惶恐,似乎根本谈不上悲喜;大清早的回首往事,他回过了味,胸膛像是迎风敞开了,五脏六腑满是光明清凉。外间有人出出入入,是老妈子送了热水进房。他不管熟睡的胜伊,径自下去洗漱穿戴。最后推门一步迈出去,他抬头一怔,随即就笑了。

原来赛维和他心有灵犀,也是正推开了房门。她已然经过了一番修饰,头发不但一丝不乱,面孔上也施了脂粉。含着笑容向前走到院中,她把腰背挺得溜直,像朵小桃花似的抿嘴一笑:``早呀!''

无心上下打量了她一番,在印象中,他总觉得她像是带了一点病容,没想到经过了香粉胭脂的武装,她也是个白里透红的苗条大姑娘。忽然快步跑向了对面的东厢房,转眼的工夫他出来了,手臂上搭着赛维的呢子大衣。把大衣展开披到赛维肩上,他又绕到了她的面前,伸手为她拢着大衣前襟:``冷。''

赛维一直没有男朋友,男朋友的爱护,自然就更没享受过。清晨的秋风,凉如深水,可是她从心眼里向外散发着热量,想要说话,又不知道说什么才好,于是失控似的就只是笑。笑着笑着,她眼珠一转,忽然不笑了。

弯腰从院子地上捡起一块小石子,她扬手用力掷向西厢房的玻璃窗。窗子后面贴着一张蓬头垢面的脸,正是惊讶的胜伊。隔着玻璃受到了一次震慑,胜伊当即后退一步,而赛维站在院内,扬着脑袋大声道:``你姐我就站在外面,要看出来看,鬼头鬼脑的干什么?''

房内的胜伊乱窜了一圈,末了找到大衣裹到身上。趿拉着兔毛拖鞋跑去外间,他推开房门伸出脑袋,继续警惕的审视赛维和无心。赛维已经把大衣穿利落了,公然挽住无心的手臂,她对胜伊说道:``我们已经建立了恋爱的关系,一会儿要出去找家广东馆子吃早茶。你呢,最好就不要跟着我们了,我会给你带芋头糕回来,好不好?''

胜伊听闻此言,几乎愤怒了:``凭什么?我是你亲弟弟,你要他不要我?等我十分钟,我也要去!''

赛维和胜伊从出生到如今,一直是不拆伙;如今忽然听说赛维要和无心恋爱了,胜伊若有所失,同时恨起了无心。及至他们到了馆子,胜伊冷眼旁观,就见无心端起茶壶,自然而然的为赛维洗涮杯碗,还不时偷眼看她。赛维涂了个亮晶晶的红嘴唇,一排白牙齿始终晾在外面笑嘻嘻。也不是浓情蜜意的模样,倒像是刚刚得了大胜利,洋洋得意。

胜伊深深的吸了一口气,含着一点眼泪望向窗外,感觉自己是孤苦伶仃了。

\chapter{敲山震虎}

胜伊别别扭扭,虽然不敢和赛维正面抗衡,但是已经暗暗的把矛头对准了无心。用牙齿啃了一丁点芋头糕的边角,他饱了,开始斜着眼睛去看无心。三人是围成了一个``品''字形落座,无心正是坐在他的旁边。察觉到了他的目光,无心一边慢慢咀嚼,一边疑惑的抬眼回望向他,又带着上扬的调子,向他询问似的``嗯?''了一声。

胜伊冷笑着转向窗外,不言不语。无心看出了他的异样,放下筷子轻轻一拍他的手臂,结果他像被热水泼了一样,猛然一拧肩头,又对着外面风景说道:``姐,照理我该向你们道喜,可又怕我道了也是白道。你想爸爸能同意你嫁给个穷困潦倒的和尚吗?他身上穿的戴的,还都是我们给他置办的呢!你若是真跟了他,你的婚姻,就不是下嫁两个字可以说完的了。你把五姑的教训全忘记了?''

他说话时,无心就怔怔的看着他,嘴里还含着一点糕饼,面颊微微的鼓着。赛维两只耳朵对着胜伊,一双眼睛瞄着无心,越看越爱。及至胜伊话音落下了,她露出了和弟弟一模一样的冷笑:``你把我说成傻瓜了。难道我真能直通通的就跑到爸爸面前,说要嫁给无心吗?我自然是有我的主意,你等着瞧吧!''

胜伊无所谓似的一耸肩膀,从鼻子里笑出一声:``哼。''

三人中的两人吃饱喝足,出了馆子。家里的汽车一直等在门外,胜伊把双手插在西装口袋里,站在后排车门前仰头望天。车内的汽车夫跃跃欲试的回头看他,不知道自己要不要下车为他开门。

及至无心和赛维也从后方赶上来了,胜伊还像根刺似的戳在地上,一动不动。无心伸手为他拉开了车门,没说话,只笑了一下。

胜伊翻了个白眼,随即爱答不理的钻进车里。赛维在一旁看得清清楚楚,当即翻了个同样的白眼,心想你没人要,我可有人要。难道我见了可意的男人不找,天天照镜子似的看你吗?

三人坐上汽车,无心居中。忽见赛维没戴手套,一只手缩在袖子里,另一只手就撂在大腿上。他下意识的握起了她的手,心中依旧是没有生出天长地久的奢望,又想此刻自己每多关怀她一次,将来真相大白,恐怕自己就要多挨一个大嘴巴。大姑娘的手是能随便握的吗?不过有的握就是幸运,握一次算一次。将来算起总账,她爱怎么着就怎么着吧!自己在大问题上规矩一点,别耽误她以后的婚姻,也就是了。

无心盘算定了,便把赛维的手揣进自己的口袋。赛维状似无意的望向前方,一颗心在暗地里怦怦乱跳,同时听见无心询问胜伊:``你冷不冷?''

胜伊像只受了惊的鸡崽子一样,急赤白脸的将两只膀子乱扇一通,满车里都是他来无影去无踪的手:``哎呀别管我别管我,离我远点,一边儿呆着去!''

赛维没有动,心里想着对弟政策:``我是揍他呢,还是不揍他?''

胜伊半路下了汽车,说要找朋友玩去。赛维先还不理会,及至到了家,忽然发现胜伊居然随身携带着支票本子,登时吓得魂飞魄散,生怕胜伊被人诳去赌场,输尽二人身家。

她把无心留在家里,慌里慌张的独自出去找弟弟。无心独自留在赛维房中,这里坐坐,那里坐坐,忽然自己笑了,笑过之后翻出他的破旅行袋,找出了他仅有的一张小照片。眼看院内寂静,他捏着照片坐在窗前,在阳光下面细看。

二十年前得到照片时,感觉它真清楚,真奇妙,竟然能把两个人的面貌收在一张小纸片上,并且是活灵活现。说好每年都要拍一张合影的,倒要看看一个小女人是怎样一点一点的老去;而纵算是女人老了,照片上的影子也依旧年轻。

可是,他们只有一年的光阴,月牙死在了十九岁的好年华,永远不老。

手中的照片已经渐渐变得模糊,仿佛他与照片之间,隔着二十年的岁月风尘。时间剥夺他的一切,他是永恒的一无所有。

无心盯着照片看了许久,想起了许多热气腾腾的往事。对他来讲,往事也是珍贵的。他的人生是无涯荒野,十年之中,未必会有一件事情值得记忆。

旁边窗台上摆着一瓶蔻丹,是赛维用过的。蔻丹红得热烈,和照片形成了一个刺目的对比,陈旧的更陈旧,新鲜的更新鲜。

无心看看蔻丹,看看照片,诸如此类的对比看得多了,所以他并不动容,只叹了口气。

起身把照片收好,他坐回窗前,拿起蔻丹摆弄着玩。通红的小玻璃瓶子带着一点芬芳,无心拧开了上面的金属瓶盖,瓶盖里面伸出一根小刷子,浸染着淋漓粘稠的指甲油,油的气味很刺鼻,幸而他此刻可以肆无忌惮的不呼吸。

正在他自娱自乐的做研究时,院内忽然来了客人。他隔着玻璃窗向外望,就见来者裹着一件簇新的长披风,袅袅婷婷如入无人之境,正是马家的四小姐。二小姐三少爷不在家,丫头们乐得躲在屋子里偷懒,院子里空空荡荡,于是四小姐手里捏着几张花花绿绿的票子,站在院内娇声叫道:``三哥,在吗?我来给你送几张义务戏票。''

然后不等人回答,她一扭头,忽然发现了东厢房内的无心。马家上下各自为政,如今敌对势力范围内忽然出现了新面孔,她就下死劲的盯着他看了好几眼,随即径自转弯,迈步上前推开了房门。

抖着手里的票子一挑里间门帘,她是不怕男人的,站在门口直接问道:``哟,你是二姐三哥的朋友?''

无心知道马家的关系很复杂,所以不想和四小姐生出任何联系。迟钝而又阴沉的扫了对方一眼,他垂下眼帘,默然无语的将一刷子蔻丹涂抹在了手背上。手很白,蔻丹很红,看着有点触目惊心。

四小姐愣了一下,又问:``我三哥呢?''

无心自顾自的拧好玻璃瓶子,然后开始对着手背上的指甲油吹气。吹着吹着,他忽然笑了一声,然而脸上又没笑容。眼中光影一闪,他的黑眼珠在微微凹陷的眼窝里骨碌碌的转动了,是过分的明亮和灵活,一下子转向四小姐,然后就定住了。

指甲油在皮肤上干结了,他一边缓缓去抠,一边对着四小姐又笑一声,神情和举止全都不带人气。四小姐捏着票子后退一步,感觉自己是见了妖魔鬼怪——至少也是个疯子。

退了一步,再退一步,四小姐骤然转身跑出了东厢房。无心装疯卖傻吓跑了四小姐,心里暂时也没有事,就饶有兴味的继续去抠手背上的蔻丹。哪知安静了没有几分钟,院子里又起了脚步声音。他转向玻璃窗子,很意外的看到了马英豪。

马英豪是西装打扮,头上歪戴着一顶礼帽,不是要卖俏,而是真戴歪了,腾不出手去扶正。拄着手杖站在院子中央,他先喘了一阵,然后才环顾四周喊道:``二妹,老三,我来了!''

二妹老三都不在,他只唤出了一名平头正脸的老妈子。老妈子当然不是他的目标,于是在一眼瞧见窗边的无心之后,他对着玻璃窗一挥手,然后一边整理礼帽,一边点头笑了一下。

隔着一层玻璃,无心点头一回礼,然后漠然低头,继续去抠手背上的蔻丹——蔻丹凝在了皮肤纹理中,除不去了。

而马英豪拖起右腿,自作主张的进了东厢房。一看房内的情形,他就知道一直是有人住的,而外间的罗汉床上扔着几件女衣,可见所住之人,应该是赛维。赛维从来不是一盏省油的灯,无心却可以公然在赛维的卧室内高坐。马英豪一边脱下手上的皮手套,一边感觉其中有戏。

摇摇晃晃的坐在了无心对面,他记得无心并不是个无礼的人。然而无心只对他又一点头,显然是无意和他攀谈。

马英豪摘下礼帽,把皮手套放进了帽子里:``许久不见,无心师父是旧貌换新颜了。''

无心抬头答道:``赛维和胜伊很可怜我,给我饭吃,给我衣穿。他们真是天字第一号的大好人。''

马英豪微笑了:``是的,不过他们肯供养无心师父,可见师父你也是有过人之处。''

无心很认真的盯着他看:``哦,我还没有告诉过你。大少爷,我已经还俗了,以后你叫我无心就好。''

马英豪一挑眉毛:``还俗?为什么?''

无心答道:``我做和尚,无非是想到庙里讨生活。现在有活路了,何必还要守戒律当和尚?我决定从此就跟着二小姐三少爷了,他们正好少个跟班,我做别的不成,当跟班是绝对没有问题。对不对?''

然后他拉着椅子向前挪了挪,几乎要把脑袋伸到大少爷的眼皮底下。非常诚恳的对着大少爷的眼睛,他正色又问:``大少爷,你的意见呢?''

马英豪想了一想,随即答道:``二妹和老三也还是小孩子,家里有仆人伺候也就是了,哪里还需要跟班?我看你的新职业,并不是长久之计。''

无心郑重其事的对他摇头:``没有关系,混一天,算一天。''

马英豪沉吟着笑了:``也是。''

无心又问:``大少爷要回来住几天?''

马英豪心平气和的答道:``关于二姨娘的丧事,我打算向二妹交待一下账目明细,等到父亲回来了,二妹也可以独自去向他做汇报。另外听说八姨娘失踪了,有人在花园河里捞上一具尸体,很像八姨娘。我打算去医院瞧一瞧,另外也看看五弟。五弟年纪还小,没了娘可真不行。''

无心说道:``听说府上大太太没有子嗣,五少爷年纪小,可以让大太太来抚养嘛!''

马英豪做了个哑然失笑的表情:``这个\ldots{}\ldots{}总要双方愿意才行。''

然后他顿了顿,笑容渐渐收敛了:``而且我在大太太面前毕竟是个晚辈,也没有资格指手画脚。''

无心淡淡的答道:``没错。事不关己的话,指手画脚是不大对劲。''

马英豪静静听着,感觉他每一句话都来得别有用心。一个来历不明的人,而又别有用心,并且表明了要追随二妹三弟,他到底是什么意思?

伸手贴在温暖的窗玻璃上,马英豪笑道:``大白天的,怎么不出去走走?''

无心全神贯注的搓着手上蔻丹:``府上人多,我是个外人,总不好跑到别人的院子里叨扰。倒是听说花园里菊花开得很好,可我胆子小,不敢去。''

马英豪把目光转向了他:``是因为八姨娘的缘故吗?不过光天化日之下,想必不会有事。''

无心摇了摇头,闲闲的又道:``光天化日之下,鬼怪照样横行,只是你我看不到而已。''

马英豪饶有兴味的问他:``哦?谁看得到?''

无心往手背上啐了口唾沫,然后继续搓:``鬼怪自己看得到。''

马英豪在无心面前,有点坐不住。

他一团和气的告辞走了,一出院门就变了颜色。而无心先是吓跑了四小姐,又说走了大少爷。独自把手背搓得通红,他终于除去了皮肤上的红色蔻丹。

他也不知道作怪的人到底是谁,所以敲山震虎。隐患未除,持久的安逸就要不得。

\chapter{杀蛊}

因为赛维总也不回来,所以无心只好坐在窗前自娱自乐。

他发现蔻丹是很有趣的东西,可以用它在自己的手背上画出一道一道鲜红的符。他放心大胆的停止了呼吸,低下头慢慢的描画,画完了再撅起嘴轻轻的将其吹干。及至指甲油当真凝结了,他再很细致的去把它一点一点抠下来搓下来,最后搞得手背通红,像被人狠狠挠破了皮肉。

到了下午,赛维把胜伊扯回了家。两人已经言归于好,赛维在脖子上添了一条新纱巾,胜伊的脑袋上也多了一顶新猎帽。带着凉气进入东厢房,他把一只五颜六色的大纸盒子放到炕桌上,又对着里间嚷道:``隔着窗户就看到你啦!喏,给你带了日本点心吃。哼,你还有功了!''

无心搓着手,笑微微的走了出来,问他:``你不生我的气了?''

胜伊正要扬头回答,忽然见他手背有异,连忙拉起他的手细看了一番,又伸了冰凉的鼻尖去嗅。赛维正好推门进了来,见状便是笑道:``你可真是前倨后恭到了极点,上午还要欺负他呢,现在就改行吻手礼了?''

胜伊把无心的手向下一掼:``呸,他玩你的蔻丹!''

赛维看他把蔻丹往手背上乱涂乱画,分明是在祸害东西,但是并不着恼,只和胜伊拌嘴:``你不是也用过我的雪花膏?''

胜伊存着一腔求偶的热情,极力修饰自己,从少年时代起就依赖上了生发油和雪花膏。一屁股坐在罗汉床上,他挑起两条平淡的眉毛,预备转移话题:``瘸子真是豁出去了,大白天的就往妈院里进。怎么着,他还要把爸爸顶下去不成?''

赛维解下纱巾,一双手隐隐的做痒,忍不住用冰冷纱巾一拂无心的脖子,同时口中说道:``闲事莫管,他俩爱怎样就怎样好了,横竖闹大发了,还有爸爸呢。我倒是没想到,五姨娘居然不声不响的搬去庵里住了。老四一张破嘴,居然替她娘瞒了个紧。哼,养儿育女的姨娘已经没了两个,就剩五姨娘一人活得好好的,她逃到庵里,就脱嫌疑了?等爸爸回家断案吧!''

胜伊从兜里摸出两张花花绿绿的票子:``老四刚才在大门口,还给了我几张义务戏票。就是明天,在西单牌楼,戏码可是够硬的。姐,去不去看热闹?''

赛维摇了摇头:``我现在是越来越不爱抛头露面了。上半年咱们去参加游艺会,下汽车之后,学生们都不用好眼神看我们。反正现在我们家是\ldots{}\ldots{}''

她犹疑着措辞,感觉怎样批评都不大合适:``我们家是\ldots{}\ldots{}''

后面的话始终是没说出来,胜伊点了点头,心中了然。他们姐弟虽是既不做官、也不作恶;但爸爸是大汉奸,他们也脱不了干系。他们尽管吃得好穿得好,有大把的钱花,可一生的名誉,已经是糟了。先前年纪小,还不在意;如今越来越大,他们偶尔被人狠狠的瞅上几眼,心里也知道别扭。

``再说吧。''胜伊把票子放在桌上:``反正大戏也不是今晚开演。''

赛维站在地上,默然片刻,然后把外面的大衣也脱了:``真的,把嘴都闭上吧。大哥不说一会儿还要过来和我说话吗?万一我们说着说着,他忽然进来了,才叫可怕。''

正当此时,院子里忽然响起了马英豪的声音:``二妹,回来了吗?''

赛维和胜伊一起吓了一跳,还是无心摆了摆手,轻声说道:``别怕,我看着呢,他是刚来。''

赛维和胜伊跑去上房,和马英豪做了一番长谈。无心独自坐在东厢房,把马家的事情翻来覆去思索一遍,越想越是糊涂,仿佛人人都有嫌疑。依着他的意思,就该让赛维和胜伊离家出走,远离是非之地。可是他也知道姐弟二人一定都不会走,当然是为了马家的钱。马老爷的手似乎是挺松,他们不去勒索,钱就让别人要去了。他们纵算时时刻刻紧盯了,竞争也还是十分激烈。马英豪是嫡长子,本来是必占上风无疑,可他偏偏又和马老爷是一对仇家。嫡长子一自立门户,马家留下一群庶出的孩子,孰胜孰负,委实难料。

良久过后,马英豪告辞走了。赛维一直送他到院门外,胜伊有一搭没一搭跟在后方,跟着跟着拐了弯,一推门进了东厢房。把炕桌上的票子拿起来又看了看,他对着无心一笑:``其实我挺想去的,唱压轴的我认识,我想去给人家捧捧场呢。我姐要是不去,你陪我去呀?''

无心一口一个的吃小点心:``看戏还用人陪?什么时候?''

胜伊对他扬了扬戏票:``明天晚上。''

无心答道:``明天晚上,你和赛维去看戏,我留下来看家。赛维要是不愿意,我帮你劝她。''

胜伊狐疑的看着他:``家有什么可看的?再说看家有丫头呢,也用不上你啊。你是不是\ldots{}\ldots{}''

无心一点头:``是,我打算再去花园一趟。上次没看出什么来,我得再看一次。我劝赛维去看戏,你劝赛维别管我,我们合作,好不好?''

胜伊立刻点了头,又道:``合作是没问题,但你一定得小心。''

无心和胜伊串通好了,当晚无话。到了翌日白天,马英豪出发返回天津,胜伊则是围着赛维游说不止,终于劝得她动了心。无心则是另找借口,表示自己不爱看戏,宁愿留在家里睡觉。

赛维没有多想,只以为胜伊是好热闹,又想他刚刚拈酸吃醋生了一场闷气,便温柔了态度,天没黑就张罗汽车,和他一起出门前往西单。

无心吃饱喝足,及至天黑透了,他也悄悄溜出了院门。轻车熟路的走向花园,他半路经过了八姨娘的后院。八姨娘没了,院内的主人就剩下了马俊杰一个人。玻璃窗户没拉窗帘,无心遥遥的向内张望,就见屋内床上躺着马俊杰,姿态是伸胳膊伸腿,显然已经入睡。一个老妈子站在床前,为他牵扯棉被盖住了手脚,然后转身走到门口,关了电灯拉上房门。屋子里面黑黢黢的没了动静,无心也不能长久的去看马俊杰睡觉,于是蹑手蹑脚的要继续走。

可就在将走未走之时,他忽然感觉房内有了动静。

单凭两只眼睛看,是看不出什么的。好在屋子里外都是一样的黑,无心人在窗外,总不会轻易暴露行迹。隔着窗子静静的望向屋内,他依稀感觉床上被子一掀,马俊杰直挺挺的坐起身了!

然后他很利落的穿戴整齐。走到窗前打开插销,他缓缓推开窗扇踩上窗台,一侧身就跳出了房。落地之后挺直了腰,他一抬头,正好和一丛玫瑰树旁的无心打了个照面。

无心不知道对方又在搞什么鬼,所以迟疑着没说话。而马俊杰怔了一下,随即却是大踏步走上前去。在无心面前停住脚步,他仰头又看了无心一眼,紧接着张开双臂,一把抱住了他:``大哥哥。''

他的脑袋正到无心的心口,隔着衣裳用脸蛋蹭了蹭无心的胸膛,他声音很轻的说道:``大哥哥,我是小健,现在你喜欢我了吧?''

无心大吃一惊,连忙握着他的肩膀俯下了身:``怎么着?你把马俊杰给弄死了?''

小健用手指头一点自己的脑袋,沾沾自喜的小声说道:``我没有害人。白天是他,夜里是我。嘻嘻,他还不知道呢!''

无心早就看小健是只异常的小鬼,没想到他真有点鬼运,投胎不成,就借了一具活人躯壳,并且还借成功了。看他举止灵活自如,一般有道行的鬼煞,都没有他的本领。

小健又道:``昨天夜里,不知怎么回事,我只是扑了他一下,结果就上了他的身。今夜我又试了一次,还是成功。你来得正好,你不来,我也要去找你。''然后他向无心伸出了一只手:``大哥哥,你摸摸我,我是热的。他比我大多了,可是我如果不死的话,长到今天,是不是也像他一样大了?''

无心握住了他的手,有点为难:``小健,我现在想去花园,明夜再来找你玩。''

小健脚下没根似的,习惯性的又向他一扑:``我也去!''

无心对待小健,总有一种无可奈何的情绪。他对小健毫无兴趣,可是小健很依恋他,他对小健理睬不是,不理睬也不是,所以只能糊涂着来。此刻他领着小健,糊里糊涂的,真往花园去了。

小健把身体控制得很好,轻轻巧巧的又跑又跳。两人蹲在河边一丛花木之后隐藏了,小健拱在无心的怀里,极力的想要和他贴贴脸,又因为自己终于借来了一具身体,所以炫耀似的总让无心摸摸自己。无心心不在焉的搂着他,从花木枝叶之间向远眺望。亭子里面一定是大有玄机,说是财宝或许未必准确,说是宝贝总该无误。自家的宝贝,按理说不必藏成一团谜案,除非宝贝本身也有问题。

忽然,他的手臂紧了一紧。原来河岸远远的走来了一个苗条黑影。上次只是一眼之缘,看不清楚,如今看清楚了,就见对方穿着一身合体的袄裤,正是个平常女人的身姿。女人沿着河边快走,走着走着转了方向,站上了岸边一块凸进水中的大石。一扬手将样东西扔进河里,东西不大,砸出一朵小水花。然后女人下了大石,转身沿着来路返回去了。

而就在她转身的一刹那间,无心看得清楚,原来对方不是旁人,正是马家的大太太!

等到马家大太太走得远了,无心一拍小健的肩膀,轻声说道:``你去给我把风,我要看看她到底扔了什么。''

小健一声不吭,四脚着地的往前小跑,一路连滚带爬的先到了河边。左右望了一望,他缩在大石之旁,回身对着无心招了招手。无心赶了过去,眼看河面已经恢复平静,他连忙脱了鞋袜衣裤。趟进水中走了几步,他俯身向前一冲,无声无息的没入了水中。

秋夜的河水,自然是很凉。无心不肯弄出大声响,小心翼翼下潜到了河底。在大太太站过的大石附近,他看到了水中悬浮着一只半开的纸包。

纸包似乎是被胶封过了,如今浸了水,便一点一点的软烂绽开。纸包的内容不知是什么,沉甸甸的仿佛很软,随着和缓的水流缓缓下沉,一直落到了河底的砂石地上。

无心没看明白,想要游过去捡纸包。可还未等他作势前进,砂石地下忽然起了变化。只见几道黑影破土而出,闪电一般直奔纸包。无心见它们细条条的类似鳗鱼或者水蛇,连忙向后退了一米,与此同时,纸包在怪鱼的头顶彻底破裂,里面漏出一团鲜红的蠕虫。蠕虫不过是手指的长度,头尾纠缠不清,乍一看竟是一团毛茸茸的物事。随着怪鱼的冲击吞噬,蠕虫四散开来,虽然大部分都被怪鱼东一口西一口的捕捉吃掉,可是总有几条漏网之虫,随着暗流飘到了无心面前。无心一伸手抓住了它,触感十分粗糙,送到眼前细看,他登时摇了摇头——此虫只有手指一半的粗细,不但麻麻癞癞柔软不平,从头至尾还生了无数短短的细足,方才所谓毛茸茸者,便是细足乱动的效果。无心捏着虫子两端,将其一扯两半,虫身中立刻涌出红血。无心愣了一愣,随即丢开虫子,一转身窜出老远。而一条怪鱼马上补了他的缺,一口吃了两段虫子。可惜未等怪鱼消化,一只手从天而降抓住了它的脑袋。它的身体立刻如蛇般一卷,一圈一圈缠满了无心的拳头手臂。无心满不在乎的调转方向,直接游向了岸边浅滩。

无心上岸之后,光着屁股直奔花木丛。小健见状也不犹豫,抱了他的衣服紧紧跟上。两人找了个僻静地方坐稳当了,小健见无心从右手到肘际,被一条黑亮亮的蛇缠住了,就伸手要碰。无心立刻侧身一躲:``别碰,有毒!''

小健吓了一跳,随即想起自己的身体属于借用,一旦毁坏,就算造了一条人命的孽。他不动了,不但不动,甚至还向后挪了挪:``什么东西?是蛇吧?''

无心的确是按照抓蛇的法子来抓怪鱼的,鱼脑袋就被他攥在手里。从他的虎口看,可以看到怪鱼的正面——怪鱼的脑袋还小,类似水蛇,生着一双狭长的人眼,然而没有白眼仁。对着无心极力长大了嘴,嘴是四方形的,口腔之中生满了倒刺。

无心心里有了数,继续攥着怪鱼不松手。而怪鱼用身体绞拧着他的手臂,松一阵紧一阵,不出三五分钟的工夫,它忽然脱力一般彻底脱落,成了一条软垂的黑绳子。

无心松了手,自己抓起一把枯叶擦了擦手,口中自言自语道:``脏。''

小健用一根树枝去拨怪鱼:``不是蛇?到底是什么?''

无心答道:``有人在河水里放了蛊,偶尔会有小鱼中毒,蛊虫寄居在鱼的体内,很快就会长出形状。鱼的大小有限,容不下它,它就钻出鱼身自找活路了。''

小健吃惊的张大了嘴:``哇,如果让它继续长下去,会不会长得像河一样大呀?''

无心摇了摇头:``不会的,有人在用诱饵杀它们。它们的作用只是夜里成为路障,毒死一切过河的活人。没人需要它们长大,它们长大了,对任何人都没有好处。''

小健又问:``谁干的?又有谁想夜里过河?为什么呢?''

无心想了又想,没有回答,只觉不可思议。

\chapter{揣测}

赛维和胜伊到家之时,无心刚刚洗完了澡。姐弟二人凑了一晚的热闹,戏楼里热,两人都是面颊绯红,是个极端兴奋的样子。见了一身香皂芬芳的无心,胜伊抽着鼻子笑道:``你这卫生可是讲得莫名其妙,大半夜的洗什么澡?''

无心托着毛巾,一边歪着脑袋擦耳朵,一边低声答道:``别提了,今晚真是摸了两样脏东西。''

胜伊脱了大衣,自己抬手捧着火热的脸蛋,很活泼的一步蹦到了无心面前:``抓狗屎了?''

无心摇了摇头:``和狗屎还不是一路的脏。''

然后他走到了赛维身边,也没别的事,单是想陪她站一站。赛维嗅着他身上暖烘烘的香气,忽然很想和他行个拥抱礼。可这不是件先下手为强的事情,他不主动,自己当着胜伊,也不好强求。欲言又止的抬眼看着无心一笑,她没说话,只下意识的咬了咬嘴唇。

无心又道:``我有话对你们讲,不过不着急,反正晚上有时间。''

他既然说了这话,赛维和胜伊自然就要好奇。两人匆匆忙忙的洗漱更衣了,然后一起进了东厢房里间。三人围坐在大床上,无心把今夜见闻原原本本讲述了一遍,只把小健剔了出去。而胜伊听到``大太太''三字之后,隔着棉被一拍大腿:``原来是她?!''

赛维向他摆了摆手:``别吵,仔细让人听见!''

随即她转向无心:``你继续说,然后呢?''

无心答道:``大太太投进河里的虫子,其实不能算是真正的虫,因为它是人用邪术培养出的,培养出了它,也无非是要把它当成一味毒药来使,把它放到自然中,它是活不成的。''

赛维听到这里,也惊讶了:``虫子\ldots{}\ldots{}还能凭空造出来么?''

无心一皱眉头:``所以说我今天是碰了脏东西。如果我没记错,那虫子是在人身之中生长成形的。''

赛维也跟着皱了眉头:``寄生虫吗?''

无心犹豫了一下,似乎不知道该不该说:``是把一个人捆绑好了,将虫卵送到他的耳道里,然后封住他周身的孔窍,只留鼻子呼吸。虫子长得快,只要几天的工夫,就会遍布人的体内,自行咬破皮肤钻出来了。''

赛维审视着他:``你\ldots{}\ldots{}你怎么懂得这些事情?''

无心睁大眼睛望着她,不假思索的答道:``我听说的。我还知道很多,可是我绝对没有干过。''

赛维盯着他道:``不用解释,我相信你。''

无心做了个深呼吸:``真正厉害的蛊,都是认主人的。大太太既然能治它,自然会和它有些渊源。''

胜伊小声说道:``妈——太太她平时挺老实的呀。别人不理她,她也不理别人。要说和她有关系的,也就是死瘸子了。瘸子和爸爸有仇,和我们娘没仇哇,难道是\ldots{}\ldots{}''

赛维扭头向窗外看了一眼,然后向前挪了挪,把声音压到了最低:``家里有什么秘密,谁都可能不知道,但是爸爸一定知道,没错吧?''

无心和胜伊一起点了头。

赛维自己也跟着点了点头:``秘密,应该就在花园亭子里。到底是什么,我们不知道,但是如今除了爸爸之外,娘也应该知道,否则她不会有预感似的给我们写信,也不会在床底留下一张小画片。''

无心轻轻一拍赛维的手臂:``令堂头中的铁针,是一种摄魂的法术,能把人的魂魄镇到一处,好的巫师能通灵,可以和魂魄交流。''

赛维垂下眼帘,沉默片刻之后又道:``有人想要知道秘密,不能去问爸爸,只好去问我们的娘。既然是秘密,娘对我们都不说,当然更不会对外人讲。所以对方不肯甘心,即便娘没了,他还要拘住娘的魂魄继续拷问。''随即她转向无心:``我推测的,有没有理?''

无心点了点头:``继续。''

赛维听了他这声斩截利落的回答,感觉很对脾气,于是接着说道:``这个人,不管他是谁,总之他应该是知道秘密的存在,但不知道秘密的内容。秘密在亭子里,而他并不想让别人靠近亭子,所以在河水里下了蛊毒。对不对?''

胜伊答道:``对!''

赛维又道:``无心说河水里的蛊,夜里才会有效。而八姨娘中了蛊,说明什么?''

不等旁人回答,她自顾自的给了答案:``说明八姨娘夜里去了花园,而且,是她独身一人去的!所以她中了招,都没人跑回来通风报信。可八姨娘夜里去花园干什么?一是偷情,二是探秘。''

胜伊摇头答道:``不会是偷情。旅馆饭店处处有地方,咱们家的人演不出夜会后花园的戏。''

赛维的眼睛里透出了亮光:``如果是探秘,可见八姨娘也知道秘密的存在,知不知道秘密的内容呢?就不好说了。但她绝不会是那个幕后黑手,因为放蛊和做法的,应该是同一个人,她不该着了自己的道呀!那么我们想想,家里还有谁像鬼似的,有知道秘密的可能?''

胜伊当即答道:``俊杰?''

赛维想起了马俊杰所说过的一些怪话,不由得笃定说道:``俊杰虽然鬼头鬼脑的,但不是胡说八道的孩子。你们想想八姨娘死后他的反应,哪里是个儿子的态度?好像早就认定八姨娘是要死一样。''

胜伊沉吟着说道:``看来家里除了我们,和这事有关系的,就是俊杰和大太太了。俊杰还小,可以不算嫌疑犯。那么,就剩下大太太了。大太太到底是怎样的人,我真拿不准。不过她如果要找外援,就只能去找大哥\ldots{}\ldots{}''

说到这里,他不言语了。马英豪和这个家,是不讲感情的;如果这个家里真藏了宝藏,他必定会毫不留情的抢夺搜刮。他和马老爷之间的仇,多少年了,简直说不完。

``姐。''他忽然抬眼望向了赛维:``你敢不敢和我去找爸爸?我们把来龙去脉都告诉他。''

赛维垂头,瞄着无心的手:``爸爸那脾气,阴晴不定的,谁知道他识不识好歹呀。万一他当我们是搬弄是非,我们反倒有了罪过。''

姐弟两个暂时没了主意,不过马老爷不知何日归国,所以倒也不急于让他们拿出主意。三人统一的怀疑了马英豪,可又没有证据,连指控的话都说不出。再说马英豪是什么样的人,家中上下都看在眼里,如今平白无故的就说马英豪施巫术害人性命,恐怕马英豪安然无恙,倒是他们两个要被强行送去医院精神科。

最后,还是赛维抬手在鼻端扇了扇:``行了行了,我们心里有数就好。死瘸子有心眼,难道我们就是傻的?看他也未必比我们知道得多,大家见机行事,将来死的还不定是谁呢?他有坏招数又怎么样?我们有无心!''

话音落下,她不等旁人附和,先在心里暗暗的佩服了自己的勇敢坚决,并且惋惜自己不是男人,否则随着爸爸入了仕途,必有大大的前程。

伸手又去一拍胜伊的大腿,她盯着弟弟的眼睛说道:``明天你去衙门,去问机要秘书,爸爸到底什么时候返回。我去找俊杰,看看那小崽子到底心里藏了什么事情。''

转头望向无心,她认真的说道:``你还是看家。''

话说到此,也就可以告一段落。无心跟着胜伊要回房休息,可是人都走到门口了,他忽然感觉自己一走了之也不大像话。回头看了赛维一眼,他总记着自己的身份——她爱他,所以他已经把自己送给她了。

赛维站在地上,到底要看他怎么走。他回了头,正中她的下怀。胜伊也回头望了望,但是很识趣,一言不发的继续走了。

房内只剩了他们两个人。无心对着赛维微笑,笑着笑着,他试试探探的张开了双臂。胸膛瞬间受到了柔软的冲击,是赛维扑到了他的怀里。合拢双臂拥抱了赛维,赛维太瘦了,让他的手臂不敢太用力。还是生分,还是有隔膜,他愿意为赛维做任何事,但总感觉自己和赛维不会是一家人。瘦瘦的赛维硌在他的胸前,他低下头去看她的睫毛鼻梁,她的睫毛在颤,气息也乱。

``我爱你。''赛维低声的说,两条芦柴棒似的胳膊箍住了他的腰。

无心喃喃说道:``我知道,我是你的。''

然后他后退一步,不着痕迹的推开了赛维。不能让赛维离他太近了,因为他胸中一片死寂,没有心跳。

赛维见他仿佛有些畏缩,便猜测他今晚不会有勇气吻自己了。但是也没关系,来日方长,反正他是她的。

两人就此分开,各自休息。到了第二天,赛维亲自出马,把马俊杰强行拎到了自己房内。无心怀着鬼胎,在暗处偷窥马俊杰的一举一动,马俊杰的精神很足,一如既往的沉着小脸,是个小阴谋家的模样。

赛维对他没客气,``咣''的一声摔上房门,她摆出大姐的派头,一屁股坐在沙发椅上,盯着马俊杰的眼睛问道:``说吧,你心里到底藏着什么事?你没本事给你娘报仇,我可有。''

马俊杰万没料到赛维会开门见山的如此说话,不禁怔了一下,但是把嘴闭紧了,站在她面前一言不发。

赛维凝视着他,决定诈他一诈:``我告诉你,真相,我已经查出大部分了!杀人的不在家,在家的不杀人,对不对?别人我不管,反正我马赛维不是好惹的,谁也别想在我手里讨了便宜去!大不了鱼死网破,死了我也不做糊涂鬼。他有人,我没人吗?笑话!我要是没人,也不能安然无恙的活到如今。马俊杰,你放清醒点,你亲娘都让他弄死了,你还缩头乌龟似的装什么孙子啊?别说你十二三岁,不是小孩子了,你就是二三岁,良心志气总该有吧?''

马俊杰定定的望着她,良久过后,他终于出了声音:``我可以说,但是你有了好处,不要忘记我。我没了娘,爸爸又不喜欢我,以后我不知道自己会怎么样。''

赛维当即点头:``没问题。二姐从来都不是小气鬼!''

马俊杰自己搬了一把椅子,坐在了赛维面前:``那时候,爸爸还没有出发去日本\ldots{}\ldots{}''

\chapter{真相}

马俊杰在赛维面前正襟危坐,绷着一张面孔说话。原来他平时的行踪一贯类似游魂,专爱乱钻乱躲。一天他溜到了马老爷所居洋楼的顶层阁楼里,正在自得其乐的翻检旧物,不料阁楼下面忽然来了人,他伏在楼板上听声音,听出来人正是爸爸和二姨娘。

他屏住呼吸,起了偷听的兴致。然而听到最后,他的呼吸无声,一颗心却是将要跳出喉咙。因为马老爷向二姨太交待了一桩秘密:后花园的亭子下面有机关,机关后面,藏着宝贝。

宝贝还是马老爷的父亲从关外发掘出来的,发掘之时,就赔上了几十条人命;及至把宝贝分批运到京城,又是一路的鲜血。人命关天,赔了人命也要挖也要运,可见宝贝的价值。

宝贝到了家之后,马老爷的父亲亲自主持重修了后花园,河边的小山是后堆出来的,山上的亭子就是暗门。

二姨太是个很容易知足的人,骤然听到这般惊天内幕,反倒吓得手足无措,宁愿自己没有听过。而马老爷继续解释,说自己这一趟去日本,路上兴许会有危险,平安归来倒也罢了,一旦遇险,就把这桩秘密传给家里的龙凤胎——老大已经是他的死敌了,老四是个小姑娘,老五是个小孩子,只有老二老三年纪大,心眼足。但是秘密传归传,不能破,因为宝贝带着邪性,一旦让它见了天日,反倒要伤人。所以马家其实是拥着火炭受冻,明知道小山肚子里揣着巨大财富,却只是知道而已,无路使用。二姨太是个老实头,马老爷对家里人观察了一辈子,最后就感觉她心宽体胖,是个可以信赖的,所以在临行之前,就把心里话对她说了。

``等爸爸和二姨娘走后,我悄悄逃回了家里。''马俊杰低声说道:``全家上下,顶数我们这一房最穷,所以我也想取一点财宝给娘。''

赛维看着他:``你告诉八姨娘了?''

马俊杰犹豫了一下,最后一点头:``是,我告诉娘了。娘听了之后,就像疯了似的,睡也睡不好,吃也吃不下。但是我们势单力孤,根本不可能去挖山运宝。所以,我就打算再找个帮手。''

赛维立刻问道:``谁?''

马俊杰叹了口气:``我一开始想去找四姐,可是四姐她们和我们也差不多,都是没本事的。于是,我就\ldots{}\ldots{}我就找了大哥。''

赛维,因为太紧张,所以反倒笑了一下:``大哥怎么说?''

马俊杰小声答道:``大哥愿意和我们合作,还给了娘三条小黄鱼。娘见了金子,就更疯了。''

赛维回想往事,不记得八姨娘有过异常的举动,想必她也是忍得辛苦,暗暗的疯。

``后来\ldots{}\ldots{}''马俊杰开始吞吞吐吐了:``我也不知道是怎么回事,二姨娘就发急病死了。我很害怕,让娘不要再和大哥合作,娘也害怕,真的不再理睬大哥。可是她放不下山里的宝贝,我早就看出她想要单独干,又拦不住她,结果她也\ldots{}\ldots{}''

马俊杰摇了摇头,脸上一点孩童的稚气都没有,是位老气横秋的少年。

赛维问他:``今天你说的这些话,敢不敢随着我到爸爸面前,再说一遍?''

马俊杰答道:``不敢。''

赛维一愣:``你不想给你娘报仇了?''

马俊杰神情冷漠的答道:``娘财迷心窍,死就死了,我也没有办法。在我心中,爸爸也和疯子差不多,如果我说了实话,恐怕他第一个就要惩罚我;就算他放了我,大哥也饶不了我。总之我把实情全告诉你了,你们爱怎样就怎样吧,我什么都不要了,只想活着。''

赛维早就感觉五弟的性情偏于阴柔,如今一看,真是毫无刚性,心中就很鄙视。但在脸上做出和颜悦色,她压低声音说道:``你今天所讲的,二姐会完全保密。你年纪小,怕事,也是正常。放心,二姐不会和个老弟弟玩心术,将来无论家里怎样,二姐都会尽量的维护你。二姐三哥是一个娘肚子里出来的,齐心协力,未必就一定不是大哥的对手。你等着瞧吧!''

马俊杰垂头沉吟片刻,忽然又道:``宝贝是爷爷在关外的什么兴安岭里发现的,说是当初为了抢宝贝,爷爷带着人打了好多仗。当地的萨满在宝贝上施了咒,也可能是下了毒\ldots{}\ldots{}爸爸也说不清楚,总之宝贝不能见天日。见了天日,就要发生坏事。''

赛维一听,心想宝贝成了鬼了。

赛维把马俊杰打发走,临走时又告诉他``有事就来找二姐''。马俊杰一脸未老先衰的惨相,心不在焉的答应一声,显然是无论有事没事,他都谁也不想找。

马俊杰前脚刚走,后脚胜伊就回来了。甫一进门,他便大声疾呼:``爸爸后天就能回北京!''

赛维踩着门槛,向他和无心招手:``你们过来,我有话说。''

赛维把马俊杰的话,原原本本复述了一遍,听得胜伊瞠目结舌,又低声笑道:``爷爷也是够坏的,明知道家里全是饿死鬼,还偏在大家眼前吊起一块肥肉。不过话说回来,真不能取吗?要是有毒,我们戴副手套,不碰它也就是了嘛!''

赛维同样爱财,若是大家都得不着也就罢了,一想到马英豪对宝贝虎视眈眈,还害死了自己的娘,她就牙痒痒的想要咬谁一口。

赛维姐弟怀恨在心,不能罢休。马英豪人在天津,也有心事。这几天,天津似乎比北京更冷似的,他披着一件沉重的军大衣,在他的密室中一坐能坐小半天。

对着前方的大玻璃缸,他看水蛇蜿蜒游动,形象灵活而又恐怖。新仇旧恨在他心中来回的翻腾,他缓缓摩挲着自己的右腿,天一冷,旧伤就犯了,整条腿都是又酸又痛,并且闹起独立,不听他的调动。

他讨厌自己的伤腿,想要变成一条水蛇。

密室中的空气潮湿微咸,带着一点海的腥味。探入水中的铁管中忽然传出呼噜噜的空响,仿佛一位巨人在咳嗽气喘;随即一团泥鳅从铁管口涌动而出,是蛇们的晚餐。一名老仆人住在楼上的空屋里,专门负责伺候他的蛇。换水,喂蛇,捞出死蛇,补充活蛇。老仆人问他:``为什么不换几条好鱼来养呢?''

他说:``蛇更漂亮。''

马英豪轻轻的咳了一声,把身上的大衣紧了紧。他想父亲将要回来了,回来了才好。一场战争,没有硝烟也就罢了,居然连对手都在千里之外,真是让人感觉乏味。他要为自己的右腿报仇,为自己的亲娘报仇,还要为谁?是了,也加上佩华一个吧。佩华在他的冷宫中苦度时光,难道不该有仇恨吗?

佩华是他的继母,他的爱人。他逼她为自己做事,不情愿也得做。他想自己其实是为了救她,但她不知道。

马英豪凝视着他的宠物们争夺泥鳅,宠物们很快就要被处死了,因为他的好朋友小柳治,为他新弄到了几条更斑斓美丽的海蛇。旧的不去,新的不来。

马英豪戴上一副消过毒的口罩,像名战地医生似的,裹着军大衣下到地下二层,去见白琉璃。

站在恶臭的地下室里,他依稀只能看到黑暗角落里有个人影。忽然从大衣口袋里掏出手电筒,他拨动开关照向了对方。一照之后,光芒立收,因为他只是想确定人影的身份。

白琉璃看起来是臃肿的一大堆,乱发下面露出了清秀的尖下巴。臂弯里躺着他的死儿子,他的右手鲜红淋漓,是刚刚抓碎了一大把毒虫——用来杀蛊的毒虫。

把毒虫的汁液慢慢涂抹到婴尸上,铃铛随着他的动作微微作响。马英豪冷眼旁观,看他像个疯女人;同时听到他在用古怪语言低吟浅唱,又的确是男人的声音。他的身边黑黢黢的躺着一团物事,是具千疮百孔的尸体。忽然``噗嗤''一声低低响起,一股子鲜血窜起老高,正是一只毒虫摇头摆尾,突破了尸体的皮肤。而白琉璃看也不看,直接把它抓住,揉碎在了怀中的婴尸身上。

马英豪看了他一年,对他的一举一动都看惯了,只是从未看清过他的面貌,甚至很少见他起立。他是个臭不可闻的妖魔,视污秽与阴寒为力量的源泉;马英豪即便对他敬而远之,可还是时常发起冲动,想要像刷马一样把他摁倒水里,狠狠刷洗一通。

``家里来了个麻烦。''他躲在口罩后面,闷声闷气的说道:``不知道老二老三是从哪里弄来的人,带着三分鬼气,而且仿佛无所不知。''

白琉璃把赤红的婴尸藏进怀里,然后轻声说道:``是不是麻烦,我看一眼就知道了。''

马英豪摇头叹气:``不能够。他从来不离老二老三。即便我把你带到北京家里,你也未必有机会和他见面。''

白琉璃不言语了,摸索着从身后翻出一只铁皮罐子,自顾自的从尸体身上挖出毒虫,一条一条的往罐子里扔。扔着扔着,他忽然一舔血肉淋漓的手指,开口说道:``我只做我能做的,不是万能。如果没有新的命令,你就走吧。''

马英豪用手杖轻轻敲打了地面:``我留下,又碍了你什么事?''

白琉璃轻言细语:``好,那你就留下。''

然后他从尸体上慢吞吞的拧下一截小臂,撕了烂肉往嘴里塞。

马英豪不为所动,继续用手杖敲击地面,暗想事成之后,自己会让小柳治运来一架火焰喷射器,把眼前这个怪物烧成灰烬;然后再往地下室内注入水泥,让他的灰烬永不见天。

粘稠的血浆顺着白琉璃的嘴角流下来,毫无预兆的,他抬起头,对着马英豪笑了一声。马英豪一哆嗦,脸上神情不变,只是敲地的节奏略微有些乱了。

\chapter{马老爷}

在一个阳光明媚的午后,马老爷回家了。

马老爷大名叫做马浩然,今年不过是五十多岁的年纪,对于一名政客来讲,正是壮年,绝不算老。赛维和胜伊提前筹划清楚了,如今做出欢天喜地的面孔前去迎接,同行的自然还有四小姐马天娇,五少爷马俊杰。

无心不着痕迹的混在人群里,在远处一闪而过。在看清马老爷的面目之后,他理解了为什么赛维和胜伊最受偏爱——马老爷也是个瘦骨伶仃的身材,一脑袋紧贴头皮的自然卷,五官周正而又平淡,和赛维胜伊站在一起,正是等高的三根大刺。他们之间的关系,无须介绍,一望便知是如假包换的一家人。

赛维和胜伊先迎上去了,随后四小姐也迎上去了,五少爷死死板板,却是站在人群中不动步。其余未生养的年轻姨太太们站在外围,喜气洋洋的连说带笑。马老爷像是落进了脂粉堆里,在莺莺燕燕的包围下向前缓缓移动。

晚饭之后,胜伊独自回了小院,进门之后满世界的喊无心。把无心从东厢房里喊出来了,他随即又把对方推回了房内:``快快快,洗脸换衣服,你吃什么吃了一嘴黑?赶紧把牙齿也刷一刷!我姐向爸爸提过你了,爸爸要瞧瞧你呢!''

无心十分惊讶:``啊?''

胜伊拼命的把他往浴室里搡:``等到见了我爸爸,只说你做和尚的一段就够了,可千万别提你在上海当过神棍!还有啊,我和我姐是在街上遇到你的,大家闲聊几句,就成了朋友。记住了吗?''

无心被他催得晕头转向,手忙脚乱的刷牙,又喷着满嘴白沫,弯腰对着水池问道:``你又愿意认我做姐夫了?''

胜伊恨铁不成钢的叹息:``嗐!女大不中留,她要是非嫁人不可,索性嫁给你算了。你再不怎么样,也比外人强呀!''

无心刷了牙,洗了脸,还用梳子在头上划了几下。对着胜伊站稳当了,他提裤子系腰带,胜伊则是微微仰头,为他打了个饱满的领带结。两人分工协作,不过几分钟的工夫,就西装革履的一起奔出门去了。

胜伊带着无心走去马宅前头,进了马老爷常住的洋楼。虽然还是秋天,但是楼内已经烧起了暖气,进门便是暖风扑面。马老爷换了一身藏蓝缎子的长袍,扬着一张小干脸坐在长沙发上,倒是挺和气,笑模笑样的打量无心。赛维坐在他的身边,尽管眉目和他类似,然而比他新鲜滋润了好几十年。

四小姐五少爷以及姨太太们都退下了,大客厅里面堪称清静。马老爷让无心在对面坐下了,慢条斯理的询问他的来历。他按照胜伊的吩咐,清清楚楚的作了回答,脸上始终带着一点笑模样。赛维远看了他,越看越喜,等到马老爷和他的对话告一段落了,她便接了话头说道:``要说他有什么,他什么都没有,孤身一人逃到外乡,能保性命就是幸运了;可要问他没什么,他身体健康,性情温和,要知识有知识,要思想有思想;一个人最重要的几要素,他也是丝毫不缺少。爸爸,您瞧,我不是在胡闹。如果只是为了一时的玩乐,我大可以找个浮华子弟作伴。但是人在年轻的时候,应该每一步都朝着正确的方向走,我想凭着我们家的家世,并不需要攀高枝嫁女儿;既然物质问题不是问题,我就要寻找一位精神上与我契合的好伴侣。''

马老爷笑了,一张干巴巴的单薄面孔刮得溜光,一点须根都不显,乍一看不像马老爷,倒像马老太太。让女儿嫁个刚还俗的穷和尚,当然是很不像话;不过依着他的心思,他也真不想让赛维外嫁。即便没有无心,他也打算给二女儿招个上门女婿。家里的孩子都不成器,他很想培养几名得力干将,帮助自己对抗天津的长子。

他很后悔,当初应该一枪打死马英豪。

马老爷抬手摸了摸自己短短的一头卷毛,眼皮一撩,又看了无心一眼,末了又笑了,一边笑一边把眼珠转向赛维,眼波流转,很有一点徐娘半老的风致。无心因他是赛维和胜伊的父亲,所以正襟危坐,万万不敢发笑。胜伊坐在一边,垂着眼帘走了神,怀疑自己之所以对男人深恶痛绝,乃是受了父亲的影响。父亲作为一个男人,一举一动全不对劲,他看在眼里,厌在心里,由此及彼,也就嫌恶了全体男人。

``我知道你们年轻人,都是先恋爱,恋爱到了一定的程度,才肯结婚。''马老爷摸着自己的卷毛开了口,微微有点公鸭嗓,还是很像马老太太:``爸爸并不是老古董,当年也是摩登过的。我先摩登,你们后摩登。再说你也真是大姑娘了,哈哈!''他对着前方空气又一点头,用标准的伦敦音温柔说道:``Women
are meant to be loved。''

胜伊,因为听懂了,所以咽了口唾沫,认为当爹的完全没有必要和女儿谈论爱情问题。赛维则是像只鸟儿一样,叽喳笑道:``爸爸,不许你再说了!''

马老爷在婚姻之事上,没有吐露半点口风,只用一句英文把话题岔开。赛维不让他说了,他正好也不想说。他很明白赛维的心意,女人照样可以色迷心窍,比如当初他的五妹,如今他的女儿。现在这个年头,比较文明自由,老二要恋爱,就让她去恋爱;真到了谈婚论嫁的时候,自己自有办法控制她。

胜伊不知道父亲接下来还会有什么惊人之语,单是看马老爷翘着兰花指捏勺子搅咖啡,就已经有些承受不住。而赛维知道他不堪大用,于是三言两语的,把他和无心全支走了。客厅里彻底变得空荡,她把脸一板,忽然低声说道:``爸爸,我有重要的事情和你讲。我知道你旅途辛苦,可是不讲不行。我们到你书房里去,好不好?''

马老爷对着女儿张了嘴,做了个天真表情,同时站起了身。

在马老爷的小书房里,赛维把马俊杰彻头彻尾的出卖了。

马老爷坐在大写字台后面,一边听,一边若有所思的给自己点了一根雪茄。等到赛维说完了前因后果,他夹着雪茄,歪着脑袋呼出一口烟雾,然后抬眼望着赛维说道:``二姑娘呀,你的话,爸爸全相信。''

然后他咬着雪茄深吸了一口:``可是俊杰的话呢,爸爸就不很信了。''

赛维侧身靠着写字台的边沿,忽然有些懵:``爸爸,你认为俊杰是在撒谎?''

马老爷沉吟片刻,末了垂下了头,盯着雪茄的火头突兀一笑:``赛维,爸爸是把你当成儿子看待的,不会想你长大了,嫁人了,就和我马家无关了。马家的秘密,你不问,我迟早也是要告诉你的。你们的娘,本质不错,养出的儿女,也不错。爸爸一直高看你和胜伊,你们体会到了吗?''

赛维立刻点了头:``当然,娘都说我们只和爸爸亲,不和她亲呢!''

马老爷斜着身体,把左胳膊肘支在了沙发椅的扶手上。右手伸长了,将雪茄架在玻璃烟灰缸上。人老了,精神就渐渐有了软弱的倾向,他发现自己永远活成孤家寡人也不成;好的儿女,还是要拉拢到手下的。

``我在临去日本之前,的确是和你们的娘说了些私话。''他把右手搭在写字台上,小拇指蓄了半长的指甲,此刻就在台面上轻轻的叩:``问题是,我只说家里藏了宝贝,后面的话,我当时可没有说呀!''

赛维下意识的伸长了脖子,两只耳朵也有竖起来的趋势。

马老爷微微皱起两道平平的眉毛:``我当天晚上去了你们娘的屋子,又对她补充了后面的话。总而言之,话的内容,是没有错。可俊杰总不会两次都藏在旁边吧?''

然后他对赛维竖起了一根手指,做了一个警示的手势:``此乃问题之一。''

赛维有些茫然了:``那\ldots{}\ldots{}俊杰又是从哪里听来的消息呢?''

马老爷一耸肩膀:``知道秘密的人,马家只有我和你们的娘,我不说,还有谁能说?''

赛维难以置信的反问:``娘?''

随即她结结巴巴的想要为娘辩护:``也许是俊杰听了片言只语,出去学舌,结果坏人因此威胁了娘,娘不得已才说出了实情。爸爸,我忘记告诉你了,娘在临去世前,曾经给我们写了两封信,全都写得前言不搭后语,她还说在家无聊,想要到上海和我们一起住一阵子。''

马老爷并没有和死人算账的打算,所以只点了点头:``不管内情如何,总而言之,我的秘密被你们的娘公开化了。俊杰那一房是知道的,还有谁也知道?不好说!''

赛维默然无语,没敢提自己三人曾经夜探花园,险些送命;也没敢提大太太的杀蛊行径,因为不想把无心拖下水。

马老爷继续说道:``你们的娘又不傻,当然不会主动去说,所以肯定是俊杰那个东西坏了事。你们的娘,老实讲,没什么城府和心术,是个厚道的人,怎么是那帮人的对手?必定是着了人家的道,把一切都全盘交待了。那帮人会是谁?其中一个肯定是老八,俊杰是她儿子嘛!''

马老爷说到这里,一拉身前抽屉,抽出了一张白纸和一支钢笔。把白纸摊在写字台上,他拧开笔帽,在纸上写了个``八''字,同时口中喃喃说道:``老八一个人不能成事,所以就得找帮手。找谁呢?有你们大哥一个。老五说是跑到庵里去住了?很好,可能也有她。她们成年的谋划着我的钱,有了机会,还能放过?''

话音落下,马老爷猛然抬头,见神见鬼的压低了声音:``赛维,我告诉你,不要看她们和我过了一辈子,她们都是我的敌人哪!''

赛维苦着一张脸,怎么回答都不对,所以依然不出声。

马老爷抬手摸了摸自己的卷毛,又道:``俊杰那孩子,本质有问题。以后无论他说了什么,你都要打个折扣来听。''

赛维从鼻孔里呼出凉气:``我是一片好心待他,怕他受了伤害,没想到他真话假话掺和着骗我。我想抽他大嘴巴呢!''

马老爷摆摆手:``改天再抽,不要急。''

赛维又道:``爸爸,八姨娘怎么看也不会是溺水而死,河里肯定有古怪,或许藏着吃人的妖怪。你夜里千万不要去花园。''

马老爷点了点头,伸手拿起雪茄,顺便又扫了赛维一眼。家里的老二的确是比一般的孩子强,但还是年轻幼稚。如果是个男孩子就好了,如果是个男孩子,便可以代替自己当家了。可女生外向,谁知道她将来和谁一条心?

马老爷的思想素来是天马行空没有轨迹,一边思索家中疑案,一边考虑给二女儿招个上门女婿,两条思路齐头并进,各想各的。末了他又吸一口雪茄,喷云吐雾的说道:``俊杰的话,无论真假,全部推翻。所有的人都有嫌疑,宁可错杀一千,不可放过一个。''

然后他站起来:``好了,你回去休息吧!''

赛维攥着拳头往后面院子里走,半路好几次想要拐弯,去把马俊杰痛捶一顿。勉强控制自己走了直线,她走着走着,忽然想通了:``俊杰会骗我,孰知爸爸就不会骗我呢?有没有宝贝我不管,反正坏事别找我,好事也别丢下我。只要不让我吃亏,我管你们做什么乱呢!''

\chapter{傻子}

马老爷并没有去找小儿子的晦气,因为已经不把小儿子当成儿子看待了。只是因为小儿子没了娘,不好将他逐出家门;否则他会让八姨娘带着她的崽子一起滚蛋。

``真有诅咒吗?''他成夜的不睡觉,坐在书房里沉沉的思索:``按照科学的观点来看,父亲的话当然是无稽之谈。不过父亲并不是胡言乱语的人——真有诅咒吗?''

马老爷因为一直富有,所以从来没打过家中宝贝的主意;可是此刻他心中活动了,不是为了钱,纯粹只是好奇。但对于玄而又玄之事,他是宁可信其有,不可信其无。让他亲自进入山内藏宝库,他是绝不肯、也不敢的。

马老爷摸着自己光溜溜的下巴,想天想地,想到最后,想出了一声冷笑。

与此同时,远在百里之外的天津,马英豪裹着半新不旧的军大衣坐在密室里,对着他斑斓缤纷的新宠物也在冷笑。密室中冷腥的海水气味越发凝重了,来自南太平洋的海蛇在水中扭绞成了一团。

两小时后,他接到了来自北京的长途电话。电话那边的说话人是马宅管家,语气疲惫而又茫然,让大少爷明天早早回家,因为老爷有重要的事情,要向晚辈们宣布。

马英豪一团和气的答应了,然后放下电话,开始出神。

马英豪凌晨出发,在中午之前就到了北京。他进入马老爷的客厅时,下面的四个弟弟妹妹都已经到场了。对着马老爷一点头,他不冷不热的唤道:``爸爸。''

马老爷端坐在沙发上,脸上似笑非笑,笼罩着一层不甚温暖的假春风:``英豪。''

然后两人再无其它话可说,马英豪在角落里的沙发椅上坐下了,顺便不动声色的环顾了旁人面貌。赛维和胜伊照例是并肩落座,脸上没有什么表情;马天娇坐在侧面的短沙发上,专心致志的低头去望自己的漆皮鞋尖;马俊杰弯着腰,几乎就是委顿在了大沙发里,看起来是特别的幼小。门外忽然由远及近的响起了脚步声音,浓妆艳抹的五姨太走了进来,表情有些怯,而马天娇立刻就向她招了手:``娘,你怎么才到呀?''

五姨太试试探探的笑了:``我刚回来嘛,到你七姨娘院里说话去了。''

然后她走到马老爷身边坐下,很殷勤的从烟筒里抽出一根香烟,自己先叼在嘴上点燃了,深吸一口之后送到了马老爷面前。马老爷抿着薄嘴唇,老而俏皮的莞尔一笑。一手接过香烟,另一只手摸着脸,马老爷心事重重,同时感觉自己皮肤挺好。

未等他自恋完毕,门外人影一现,却是大太太佩华。佩华算是这家里的黑人,常年不见天日的,此刻不施脂粉,打扮得不显山不露水。她进门时,因为毕竟身份还在,所以孩子们无论情不情愿,都要喊她一声妈,只有马英豪不言不动。佩华低着头,微微的笑了笑,没答出什么,搭讪着也在角落坐下了。

厅内众人表面上虽然自然,其实内心七上八下,都是临时被马老爷召集来的。马家素来是独裁统治,从来没开过家族会议。而与会成员一会儿增加一个,到底都有谁,也是令人难以预料。

马老爷知道所有人都在胡思乱想,所以慢慢的吸烟,由着大家想,等人们把心全想乱了,他才在烟灰缸里摁熄烟头,开口说道:``人到齐了,我们是一家人,当然不必讲虚套,现在,我也就直入主题了。''

听闻此言,孩子们面面相觑,心里登时有了计较——家里有分量的人,可不都是到齐了?除了儿女们不提,佩华既然没有被休,名义上就还是马家的正房夫人;五姨太虽然是个姨太太,但是生了四小姐,是孩子的娘,当然也不同于一般姨娘。

马老爷扯着单调干燥的公鸭嗓,自顾自的继续说道:``本来,今天到场的人,还该有赛维胜伊的娘,和俊杰的娘。但是人各有命,她们先走一步,错过了啊!''

用手掌抹平了长袍上的皱纹,他慢悠悠的继续说话:``我离家几个月,回来之后,听到许多流言。与其让旁人胡说八道,不如我来戳破这一层纸,也免得你们装神弄鬼,做出种种不堪的举动,败我家风,损我名誉。''

话说到这里,房内各人的神情就开始千变万化了,但是万变不离其宗,面部肌肉都在勉强绷紧,是个遮遮掩掩的紧张样子。

马老爷手不闲着,一下一下的摸着自己的大腿,眼皮也垂下去,不肯正视儿女妻妾们的眼睛:``我们马家,是有一点秘密。上一辈曾经在关外谋过生活,机缘巧合,就弄到了一批财宝。财宝是什么?不好说,因为我没有亲眼见过,听你们的爷爷讲,无非也就是些古董金玉之类,值钱一定是值钱的,但也仅仅只是值钱而已。''

轻轻一拍自己的大腿,他把搭在腿上的袍襟抹了个溜平:``为什么我对这一批宝贝是从来不提也不动?因为我不缺钱,我不靠着祖宗吃饭!我想把上一辈的遗产存住了,将来留给你们这帮没出息的混蛋,免得你们有朝一日吃不上饭,会流落街头挨饿受冻!''

两道平淡眉毛跳了几跳,马老爷西洋化的一耸肩膀:``可是,似乎你们并不能理解我的苦心。也好,我索性开诚布公,迟早都是你们的,我又何必多做隐瞒,还惹得你们猜忌怀恨?''

然后他一挺身站起来了,对着客厅大门一挥袖子:``走走走,我带你们去花园!''

马老爷拎着一根手杖打前锋,儿女妻妾紧随其后,因为全是心怀鬼胎,所以一路走得目不斜视,互相连眼神都不肯交汇。及至到了花园河边,众人举目远眺,却是一起傻了眼——对岸山上的凉亭,不知何时竟然被拆了顶,四周的雕镂槅子也全没了,原本很精致的一处凉亭,如今就只剩了四根柱子,以及中间一张固定不动的石桌。

马天娇忍不住``啊''了一声,随即被五姨娘狠狠拽了一把。一行人分乘三只小船,三摇两摇到了对岸山上。这回走到亭子近处,只见四周脚印凌乱,正是施工不久的迹象。另有一架梯子倒在地上,不知是丢弃不用,还是忘记带走。

马老爷迈步进了亭子。背过双手挺直腰身,他在寒凉的空气中做了个深呼吸,然后用手杖一敲亭子地面:``我们家的宝藏,就在我的脚下!''

此言一出,鸦雀无声。

马老爷又道:``胜伊,把梯子扶起来。''

胜伊答应一声,与赛维合力扶起梯子。马老爷不再多说,将手杖往地上一扔,紧接着亲自动手,把梯子搭到了亭柱上。一撩袍子登上一步,他因为瘦,登高上远的时候反倒占了便宜。十分轻灵的爬到了顶,他把右手探进了柱子里。

赛维和胜伊在下面给他扶着梯子,见了他的举动,登时一怔,赛维抬手敲了敲柱子,声音沉闷,却又不像中空。而上面的马老爷只把右手向下伸了一尺,歪着脑袋翻着白眼,用力做了个上扳的动作。众人只听脚下``咯噔''一声,而马老爷明显的松了口气,自己点了点头,似乎也是出于意外。

下了梯子换位置,他从余下三根柱子顶端伸进手,或推或扳。原来柱子上半截才是空的,里面有套机关。机关一被触动,水泥铺就的地面下方,就有声音作响。最后马老爷下了梯子,对着中央石桌审视良久,末了开口说道:``来人,把它搬开。''

话音落下,众人面面相觑。原来石桌并不是精雕细刻的产物,看起来就是一块颇有意趣的大顽石,只是上方磨出了镜子一般的桌面,想要推动这么一块大石,非得力士不可。

马老爷并不是糊涂虫,他让人搬,自然就有道理。所以孩子们在短暂的沉默过后,一言不发的一起上阵,连马俊杰都出了手。一大群人咬牙切齿去推大石,最后只听``咕咚''一声,竟然真把大石推倒了。

接下来,又是一阵寂静。因为先前石桌所占之处暴露出来,竟是一处黑洞洞的入口。

马老爷捡起手杖,好整以暇的走了过来。十分好奇的弯腰对着洞口看了又看,他也是生平第一次开眼。洞口四四方方,在半人来深的地方凿出斜坡,一路向下。斜坡尽头的风光,自然是看不到;就连斜坡本身的情形,除非亲自下去,否则也是不得而知。马老爷想起了父亲对自己的千叮万嘱,当即意犹未尽的直起了身。

后退一步伸出手杖,他指着洞口说道:``我还不老,你们也没有大到可以自立门户,所以里面的东西,在分家之前,不许你们随意取用。可是,我做爸爸的,也没有让儿女看到好东西干着急的道理,所以从今开始,每年我允许一房派一个人下去,拿一样宝贝上来。''

不动声色的环顾了四周面孔,马老爷轻声问道:``谁想第一个下去,现在就可以了。''

赛维和胜伊盯着洞口,心里急得快要伸出手,真想入洞看个究竟;但是他们很懂``枪打出头鸟''的道理,尤其是在自家,万万不能盲目出头。况且宝贝能不能碰,还是一件未解的疑案。

马俊杰也直了眼睛,恨恨的瞪着洞口,同时又感觉可笑——自己的娘,死得可笑。

佩华站在一旁,偷眼观察着马英豪的脸色。

马英豪不动声色,想下去,但是不敢下去。

五姨娘用皮鞋的细高跟轻轻磕着地面,看看这个看看那个,是个欲言又止的样子。而马天娇沉吟片刻,忽然用轻快的声音说道:``大哥年纪最大,大哥第一个下去吧!''

马英豪摆了摆手:``我是有职业有进项的人,经济上很宽松,不急。''

马天娇又转向了赛维:``二姐三哥呢?大哥不下去,你们下去呀!''

赛维摇了摇头:``我们两个都怕黑,不敢下。''

马天娇犹犹豫豫的又看旁人,不料佩华忽然开了口:``如果我也有资格的话,我想第一个下去。''

马英豪飞快的横了她一眼,眼神凌厉;马天娇则是着了急,没想到还真有不客气的。而佩华接收到了马英豪的暗示,不由自主的后退一步:``要不然\ldots{}\ldots{}还是请五姨太第一个下去吧。''

五姨太心乱如麻的对着佩华一笑,又抬头去看马老爷。马老爷依旧刮着满脸的假春风,显然是没意见。

``我下去?''五姨太有点不好意思了:``你们都不下,那我就做第一人。宝贝什么的倒是其次\ldots{}\ldots{}''她讪讪的笑:``我是想见见老太爷的大手笔\ldots{}\ldots{}''

她含羞带笑的,跃跃欲试的就要往洞口走。而马天娇见她穿着一双高跟皮鞋,走平地都是风摆荷叶似的不稳定,又觉得娘平时笨手笨脚,就一扯她的袖子:``还是我下去吧,我比你伶俐呢!''

母女两个是一家,谁下去都是一样。于是五姨娘停了脚步,抱愧似的一边点头一边笑,心想你们尽管装模作样去吧,我们娘儿俩可是要发点小财了!

马天娇穿着一双平底皮鞋,行动起来十分利落。洞口狭小,也非得她那种苗条的身材才出入灵活。一大步跳进半人来深的小洞里,她也不听五姨太的嘱咐,弓腰缩背的佝偻了,径自踏上了向下的斜坡。地上的人只听她叫了一声:``真黑啊!''

马老爷弯下了腰,大声说道:``天娇,如果感觉气闷了,就马上往回返!''

马天娇没理会。

直过了十多分钟,地下忽然传出一声金石撞击之响。赛维站得略近,就见马天娇捧着个破鼎钻出来了。直起腰露出头,她辫发散乱,面色苍白,但是笑嘻嘻的,将手中破鼎往地面上一放,口中说道:``我可没敢往里走,太黑了,比夜还黑。''

马老爷脸上没有笑模样,并且后退了一大步:``里面是什么样子?''

马天娇拉住五姨太的手,连滚带爬的上了地面:``爸爸,我看不清,反正随手摸到一样东西,就赶紧出来了。''

然后她笑吟吟的把小锅似的鼎抱在了怀里:``爸爸,你不来瞧瞧?说好了,它可归我喽!''

马老爷远远一望,就见那鼎铜锈斑斓,像个大铜疙瘩似的,凭着自己的学问,万万看不出价值。忽然又想起了父亲的叮嘱,他下意识的连连摇头:``不必,我也不大会看。明天你和你娘去找个懂行的人鉴定鉴定吧,看它是不是件真正古物。''

五姨太和马天娇虽然没有大见识,但也知道古董的珍贵。五姨太像抱孩子似的抱着鼎,虽然感觉沉重之极,但是舍不得松手。马天娇又伸手托了它的底,也不知是哪里来的力量,丝毫不觉疲劳。

余下众人竭尽全力,把石桌扶起来推回原位。马老爷也上了梯子,在四根柱子里面动了机关。居高临下的俯视下方,他见四个孩子加上佩华,全在偷眼窥视马天娇母女,一个个神情复杂阴沉,绝非羡慕颜色。

最后把目光转向五姨太和马天娇,马老爷不动声色的想:``我当我家里全是狐狸,没想到还真有两个傻子。''

\chapter{各种下场}

五姨太和马天娇母女两个捧着铜鼎,一路力大无穷的往花园外走。其余众人远远的跟在后方,心怀鬼胎,统一的不肯靠近她们。她们也不在乎,仰着白脸喜笑颜开,两口白牙在外面晾了一路。

及至出了花园,她们开始嘻嘻的笑出了声,腿脚可是很有劲,轮流抱着大铜疙瘩前进,步伐一致的越走越快,谁也不等了,一溜烟的就没了影。马老爷也不吭声,走着走着忽然拐了弯,直奔宅子前头自己的洋楼。马俊杰察觉出马英豪在看自己,故作不知,撒腿就跑。赛维则是暗暗一扯胜伊的袖子,然后回头笑道:``大哥,我们今天算是开了眼界。下次开眼就得等明年了。现在我们心满意足,要回院里去了。你什么时候回天津?要是没事的话,就在家里多住几天得了。''

马英豪皮笑肉不笑:``天津事多,我抽不开身。''

赛维又道:``过几天我们要是有钱有闲了,兴许还去天津叨扰你呢。''然后她对着佩华也挥了挥手:``我和胜伊真走了,回头见。''

赛维和胜伊回了院里,向无心原原本本的讲述了今日见闻。三人合计一番,也没得出结果。如此过了一夜,翌日清晨,胜伊出门去找朋友玩,不料没走几步,便看到了五姨太母女。

两人全都打扮得花枝招展,一张脸是异常的白,仿佛是彻夜未眠,失了血色。她们自己也有所意识,为了补救,故意抹了一层鲜红的胭脂,把自己打扮得像个待烧的纸人。见了胜伊,两人一起微笑招呼,笑得很大,嘴角失控似的往两边咧。胜伊吓了一跳,问道:``五姨娘,四妹,你们起大早干什么去?''

马天娇呵呵笑道:``找个明白人,帮我们看看昨天运上来的古董呀!''

胜伊停了脚步,给她们让路:``哦,那请先走吧。''

母女二人不再言语,笑模笑样的走了。

当天下午,胜伊回了来,无巧不成书,又遇到了五姨太母女。两人还穿着早上那一身鲜艳服装,脸上的胭脂粉有点褪色,显出苍白的皮肤本质。胜伊停了脚步,含笑问道:``五姨娘,四妹,找明白人看过古董了吗?''

母女二人依然一脸欢畅,面对胜伊的提问,却是没有答复,笑微微的自顾自走过去了。

胜伊莫名其妙进了院子,对赛维和无心说道:``我看老四和她娘快要美疯了。''

赛维坐在一把硬木椅子上,开口答道:``刚才我和无心在外面,也见了她们一次。''

然后她抬眼望向胜伊,犹豫着问道:``你说,诅咒什么的\ldots{}\ldots{}难道真存在吗?''

胜伊立刻转向无心:``真存在吗?''

无心靠着桌沿半站半坐,笑眯眯的不言语。赛维则是做了个深呼吸:``我已经问过他了,他说真存在。''

无心点了点头:``等着瞧吧。''

随即他笑了一下:``其实也没什么。诅咒而已,不犯它的忌讳不就得了?''

无心的话说得轻飘飘,和没说也差不多。赛维和胜伊怀揣着一颗蠢蠢欲动的惊恐心灵,数着钟点熬过一夜。他们姐弟二人的特点,就是吃得少睡得少,深夜闭眼,天亮即醒。胜伊还是偎在无心身边,因为贪恋着热被窝,所以一时还不肯起。打着哈欠伸直了腿,他不慎蹬上了无心的赤脚。无心没有反应,他心里却是一别扭,因为无心毕竟是个男人。

要是女人,他就不别扭了,问题是又没有正经女人肯和他睡觉。

赛维是孤家寡人,早早的披着衣裳下床洗漱。然而未等她涂匀脸上的香粉,遥远方向忽然传来一声尖叫,吓得她一粉扑拍到了眼睛上。扔了粉扑猛然起身,她绷紧的神经忽然有了断裂的趋势,使性子似的做狮子吼:``大清早的,谁在外面鬼叫?吓死人不偿命吗?''

然后她气冲冲的转身出门,想要探个究竟。结果刚一出院门,迎面就见一个小丫头踉跄奔来,正是五姨太院里的人。看到二小姐气势汹汹的站在路上,小丫头当场哭叫道:``二小姐救命啊!死人啦,发疯啦!全白啦!''

赛维脸色一变,当即拔腿跑向了五姨太的小院。

五姨太的院子,格局比较类似四合院,两间厢房,是五姨太和四小姐的卧室。此刻院内站了几个面无人色的老妈子,另有几个小丫头瑟瑟发抖的抱作一团。见赛维来了,一个略镇定些的老妈子哆嗦着说道:``二小姐,了不得。我们四小姐夜里死了!''

赛维正要往马天娇所居的卧室里走,可是刚迈了一步,她又迟疑着停顿了。她对于一切歪门邪道都不了解,宁愿把诅咒解释成某种传染病。可如果真是传染病的话,她此刻便应该立刻逃跑。隔着玻璃窗向房内望去,她看到床帐半掩,上面果然直挺挺的躺着马天娇。忽然下意识的上前一步,她骇然问道:``四妹怎么变了样子?''

一个小丫头锐声哭叫道:``四小姐昨晚说今天要出门,让我早早进房叫她起床。可我刚才一进屋子,就见四小姐躺在床上笑,不但脸皮煞白的,头发眉毛也白了\ldots{}\ldots{}我叫她,她不应;我去推她,她、她已经冷硬了\ldots{}\ldots{}''

赛维回头又问:``五姨娘呢?''

一个老妈子颤巍巍的指向后方厢房:``五姨太还活着\ldots{}\ldots{}可是不认人了。''

赛维六神无主,心想自己可不往烂泥里走,便打算找出借口撤退,不料她刚一转身,只听身后的玻璃窗子``哐''的一声。在众人的惊呼声中,她回头一瞧,登时汗毛竖起——丫头口中已经冷硬了的马天娇,此刻竟是直挺挺的贴在玻璃窗后,披散开来的白发之中显出面容,她还在笑!

隔着窗子和院内众人对峙了一瞬,她侧了身,是个要出门的光景。院内骤然爆发出一波惨嚎,随即在赛维的带领下,老妈子小丫头甩开大步,争先恐后的全冲出了院门。哪知还没跑出十米远,前方有人快步走来,正是刚刚接到消息的马老爷。赛维张着嘴,还要向父亲汇报情况,不料马老爷停了脚步定睛一瞧,随即握刀似的握紧手杖,一转身也跑了,且跑且喊:``来人哪!''

赛维听他把嗓子都喊破了,心中诧异,忍不住回头又瞧一眼,只见在人后不远处,马天娇一步一步踉跄着走,居然走得很快。深吸一口寒冷空气,她张大嘴巴,发出了比马老爷还要雄浑的声音:``来人哪!''

话音落下,无心和胜伊全来了。无心一边跑一边揉眼睛,胜伊满头翘着乱发。对着前方一大队狂飙的人马愣了一瞬,无心随即看清了后方独行的马天娇。

拨开人群挤到马天娇面前,他没言语,脚下直接使了个绊子,让马天娇当场摔了个仰面朝天。

众人以为他是个傻大胆,尤其没想到他居然如此对待邪祟。而无心蹲下又试了试马天娇的鼻息,见是真没气了,就回头大声说道:``别怕,不是诈尸。''

赛维带着哭腔嚷道:``她还会动呢!''

无心答道:``她只是没死透,现在好了,彻底死了。''

赛维远远站着,继续高声叫道:``真死了吗?不会再诈尸吧?''

无心很笃定的摇头:``不会不会,绝对不会。至多是今天夜里诈一回,现在还早着呢!''

此言一出,马老爷伸手一扯赛维的袖子,气喘吁吁的低声问道:``你的朋友,是不是脑子里缺根弦?''

赛维心乱如麻的做出辩护:``他\ldots{}\ldots{}见多识广,所以\ldots{}\ldots{}镇定!''

马老爷遥遥的伸手一指他:``他那叫镇定?我怎么听他是在胡言乱语?''

赛维实在不想揭露无心的身份,所以十分为难的看了父亲一眼,随即转移话题道:``爸爸,你看,四妹真不动了!''

马老爷一甩袖子,突破了老妈子小丫头的屏障,大踏步走上前去。在距离四女儿两米远处站住了,他伸长脖子看了一眼,看得心中一寒。而无心仰起了脸,忽然对他轻声说道:``午时三刻生一把火,烧了她。她是凶死的人,恐怕夜里要闹。''

马老爷打了个冷战,低头正视了无心:``你到底是什么人?''

无心笑了一下:``我原来做过和尚走过江湖,见得多了,所以懂得一点。''

马老爷神气不定的沉默了片刻,忽然又问:``死因到底是什么?我知道绝对不是急病。''

无心平静的一摇头。

于是马老爷立刻换了问法,声音也低到了极致:``怎样破解呢?''

无心想了想,末了答道:``解铃还须系铃人。''

马老爷紧盯着他:``可若是系铃人已经死了呢?''

无心又摇了头:``世上从来不缺无解的题目。系铃人活着,问题一定能解;系铃人死了,一切就都不确定了。''

马老爷说道:``如果,我想试一试呢?''

无心站起了身:``我不是巫师,无能为力。''

马老爷刚要说话,五姨太悄无声息的走出来了。她只穿着薄薄一层睡衣,手里却还捧着那只铜鼎。马老爷见她疯头疯脑,不由得向她伸出了手,眼看指尖就要触到铜鼎,无心骤然摁下他的手臂,同时低声说道:``不要碰她!''

听闻此言,马老爷当即横起手杖,摆了个防御的姿态:``怎么着?她还能伤人吗?''

无心夺过他的手杖,一杖敲到了五姨太的后脑勺上。五姨太一声不吭,当场晕倒在地,手中铜鼎骨碌碌滚出老远。

把手杖交还给了马老爷,无心说道:``不祥的东西,还请尽快毁了吧!''

马老爷握着手杖,心中翻江倒海,念头层出不穷。原来一切都是真的,他想,原来不可思议的恐怖,就埋伏在他的身边,埋伏了几十年。

一双眼睛死盯着无心,他认定对方不是凡人。可未等他说出下文,他的管家忽然带着一群听差狂奔而来,发疯似的疾呼道:``老爷,不好了,外面来了一队日本兵,封锁了咱们的前后大门!''

马老爷难以置信的看着管家:``日本兵封锁我的家?!''

管家生生的又喘出了一句话:``还有大少爷!大少爷也来了!''

\chapter{半人}

马老爷不能站在原地束手就擒,他无意再管死女儿和疯姨太,一眼盯住前方的赛维,他拖着手杖开步走,在经过赛维身边之时轻声说道:``见机行事!''

赛维低低的``嗯''了一声,然后对着胜伊和无心使了个眼色,也不多说,拔腿就跑向了自住的小院。胜伊见状,连忙要拉无心跟上,不料无心侧身一躲,随即挥手做了个驱赶的动作。胜伊怔了一下,可又来不及问,只好糊涂着先追赛维去了。

待到胜伊走出一段距离了,无心才迈步赶了上去。他不敢让旁人随便触碰自己,因为自己刚刚摸过了马天娇。如果马天娇凶死的原因是诅咒,那么为何五姨太没有下洞,却也失了神志?难道诅咒还带有传染性不成?

他不远不近的追踪着赛维和胜伊,跟着他们进了小院。赛维虽然一直自诩精明,可是此刻也失了措。在院子中央静站了足有一分多钟,她的头脑渐渐恢复了清醒,一转身便冲进了东厢房。胜伊也跟进去了,进门之后就见赛维打开靠墙的大立柜,正将一只皮箱往层层衣服下面隐藏。皮箱沉甸甸的挺有分量,里面正是一扎一扎崭新挺括的美钞。不等赛维吩咐,胜伊福至心灵,直接奔向了梳妆台。翻出二姨太的首饰盒子,他迅速拣出最珍贵的几样小玩意儿,快手快脚的全揣进了贴身口袋里。两人的动作堪称训练有素,仿佛上辈子被抄过几次家似的。

将一枚大钻戒套到手指上,胜伊终于腾出口舌说话了:``姐,怎么回事?大哥带日本兵包围了家,难道还要和爸爸正式开战不成?''

赛维是无法把皮箱随身携带的,所以索性把它藏到大立柜里,取个出其不意的巧:``只要别往家里开炮,我管他呢!''忽然一眼看到了窗外的无心,赛维急得冒了火,高声喝道:``都什么时候了,你还有闲心洗脸?''

无心的确是把裸露在外的皮肤全用香皂痛洗了一遍,并且还换了一身干净衣服,算是消毒的意思。可惜房内二人不能体会他的好意,不但赛维气得高声大叫,胜伊也得了一个提醒:``呀!我还没有刷牙洗脸梳头呢!''

可是未等他往浴室里进,院子外面跑来了管家。管家平日养尊处优,今天一早上,把今年一年的路都跑满了。喘着粗气进了院,他敲着窗子说道:``二小姐,三少爷,请快到前头楼里去吧!''

赛维``哗啦''一声,把整扇窗户全打开了:``大哥到底是怎么回事?日本人是来干什么的?''

管家上气不接下气的摆摆手:``大少爷拿枪指了老爷的脑袋,老爷没服软,日本人不言语,现在前头正僵持着呢!''

赛维又问:``让我们去干什么?当和事老吗?''

管家累得声音都变了:``是大少爷让找的,都得去,我先通知您,然后顺路就去叫五少爷!''

话音落下,管家撩起长袍,调头便走。而赛维六神无主的回头和胜伊对视一眼,胜伊问道:``姐,去不去呀?''

赛维惶恐的反问:``不去行吗?大哥都对爸爸动了枪\ldots{}\ldots{}家里今天是要出事啊!''

无心的声音忽然在窗外响了:``我陪你们去。''

赛维忧虑的探头向外看他,一刹那间,忽然生出了一个和眼前情形毫不相干的念头:``他的头发怎么不见长?''

念头像只小鸟,在她心上没做停留,轻描淡写的掠了过去。而胜伊抓紧时间漱了漱口,又用冷水洗了把脸。

赛维像只领头羊似的,带着胜伊和无心往前头走。不去是不行的,虽然平时大家都是一团和气,但和气是假和气。马英豪心里没有他们,正如他们心里没有马英豪一样。平日吃饱喝足到也罢了,一旦闹起饥荒,马家把大门一关,自家人就能互相嚼了。

三人走到半路,迎面正看到前方一条斜路上走出了马俊杰。赛维现在见了他就烦,冷着脸不理不睬。而他驻足扭头,向二姐三哥望了一眼,然后默然无语的后退一步,等到他们走近了,便自动汇入了队伍。

四个人齐齐整整的走到了宅子前头,就见马老爷所居的洋楼门口,站了一大队全副武装的日本兵。本来他们都是不怕日本人的,因为父亲就是在吃日本人的饭,而且吃到了很高的阶级;可是此刻想到日本兵和日本兵也不都是一派,马英豪带来的日本兵,大概不会惯着马家上下。脚步略顿了顿,赛维依旧是打前锋,平静着面孔昂首进楼了。

四个人进客厅时,正好赶上马老爷在咆哮:``我并没有犯法,为什么要被限制自由?八十川少将是我的学生,稻叶大将是我的同学。你也无非就是马英豪的朋友罢了,难道我没有朋友吗?''

一名戎装打扮的日本军官在马老爷面前打了个立正,似笑非笑的紧闭着嘴,显然是听得懂一切中国话,但是不打算作答。而马英豪拄着手杖站在军官身边,眼看弟弟妹妹们都来了,他缓缓的举起了手枪,瞄准了赛维的眉心:``老爷子,不要敬酒不吃吃罚酒。''

马老爷回头一瞧,登时把眉毛一拧——他是从不心疼人命的,可赛维总像是与众不同。如果马英豪此刻瞄准的是马俊杰,他或许还可以继续不在乎。

忽然狠狠一跺脚,他咬牙切齿的锐声叫道:``天娇早上刚刚死了,难道你还不知道其中的利害吗?''

马英豪依旧瞄准着赛维,同时轻声答道:``我已经收到了四妹的死讯,还听说四妹死得离奇。很好,这让我们对洞里的宝贝更感兴趣了。''

然后,他意外的发现了站在人后的无心。无心正处在客厅角落里,无声无息的盯着赛维看。仿佛意识到了马英豪的注视,他抬眼回望,随即又垂下头,缓慢的,公然的,走到了赛维身后。

马英豪收回了目光,心里有点不舒服。怪人怪物他都见过许多,但是无心让他感觉格外异样。他说不出对方到底怪在哪里,但是他和白琉璃都能相处,看无心却能看得迷惘。

马老爷见马英豪始终举枪不放,心里又怕他当真毙了赛维。在脑子里把前因后果又梳理了一遍,他暗自点了点头,随即毫无预兆的改了口风:``好,好,你要宝贝,我就给你。但是我有条件!''

话到此处,他转向了日本军官,改用日本话说道:``小柳先生,我的四女儿,因为接触到了其中的一只古鼎,已经在今天早上离奇的死掉了。我可以打开地道,但是我和我的儿女,绝不会亲自进洞,你必须要保证我们的人身安全。''

日本军官——小柳治——当即一点头:``我保证,没有问题。''

马老爷长叹一声:``走吧!''

马老爷、小柳治、马英豪三个人齐头并进,后面跟着赛维等人。出楼门时,两名日本兵已经把马天娇的尸首抬到了楼前。小柳治和马英豪过去一瞧,只见马天娇喜笑颜开的翻着白眼望天,皮肤惨白,肌肉僵硬,两边嘴角扯开了,几乎快要咧到耳根。

两名新观众登时勃然变色,抬头互相对视了一眼。听闻终归是听闻,非得亲眼见了,才能受到震慑。可震慑又终归只是震慑,比不得洞中宝贝的诱惑。尤其是在看到了后方一名士兵抱来的古鼎之后,震慑就更加不值一提了。小柳治仅凭直觉,就知道自己和马英豪是要做出大事了。

马老爷很认命的走向花园,沿途无话可说。而他的管家趁机躲在楼内,想要向外打出电话求援,可是抄起听筒之后,才发现公馆电话线已经被切断了。

一行人等不要随从,在日本兵的簇拥下到了花园。顺顺利利的渡过小河之后,马老爷仿照前天的举动,登高上远,调动了四根柱子内的机关。而两名粗粗壮壮的士兵领命上前,在马英豪的指挥下推翻石桌。地面洞口见了天日,还是老样子。

马老爷很自觉的站远了,小柳治虽然左一眼右一眼的一路打量古鼎,可是心有提防,只是看,绝不摸。此刻他和马英豪在距离洞口一米远处站住了,心有灵犀的还是不敢靠近,只把脖子尽量伸长,看到洞口方方正正,四壁不知是石砌还是水泥,竖井似的垂直向下,能有个半人多深。而到了下方,又在洞壁上开了个矮矮的斜洞,看斜洞的尺寸,略微高大一点的身材,都钻不进去。

小柳治若有所思的抬起头,审视了前方马家的一群瘦子,感觉此洞简直就是为他们量身定做。而马家的瘦子们察觉到了他的目光,不禁一起悚然。

马英豪像有读心术似的,专挑带有刺激性的话来讲:``洞子太小,一般的人也钻不进去。俊杰,你试一试。''

马俊杰随手抱住了一棵大树,紧张的身体都硬了,从牙关中挤出回答:``不!''

马英豪随即望向了赛维。家里就剩下二妹还算是个清醒明白的人,但是他并不想让马老爷再有一个好继承人。

于是他笑了一下:``二妹三弟呢?俊杰太小,下去之后也不堪大用,你们倒是更合适一点。''

赛维和胜伊全变了脸色:``下去就是个死,我们才不下去!''

马老爷也用手杖一杵地面:``不是说好要保证我们的人身安全吗?''

小柳治也改讲了中国话:``不是你们,是你。''

随即他举起一只带着白手套的手,轻轻巧巧的在半空中一挥。几名日本兵立刻上前想要拉扯赛维。赛维刚要叫骂,却听身后的无心说了话:``我下去。''

赛维猛然回头:``不行!''

无心没理会,迈步绕过了她和胜伊,径直走到了马英豪面前:``不要为难赛维和胜伊,我替他们下洞。''

马英豪饶有兴味的看着他:``你不怕死?''

无心脱了西装上衣,遥遥的扔向了胜伊,然后又问马英豪:``有没有手电筒?里面一定很黑。''

小柳治回头吩咐了身后的士兵,很快就真有人送上了手电筒。无心接过手电筒,摁动开关试了试光,随即转身走向洞口。马英豪上前一步,怀疑他根本就无法进洞,不料他跳入竖井之后四脚着地弯了腰,像条大蛇似的一拱,三扭两扭的就消失在了斜洞之中。

地面的赛维和胜伊全白了脸,因为怀疑无心会有去无回,所以一起喘得鼻孔翕动,又痛又恨。与此同时,无心已经沿着斜洞,向下爬出了老远。

斜坡坚固平整,起初空间逼仄,越往下深入越是宽敞。因为是倾斜向下,所以让人感觉不出自己所在的深度。忽然前方豁然开朗,他发现自己已经到达了一间宽敞的石室里。

连滚带爬的起了身,他用手电筒照耀四周,就见石室四四方方,四周靠墙摆了大小箱笼,箱笼上面又放着各种奇形怪状的器皿。马天娇所抱的古鼎,显然便是其中之一。

无心来了兴致,试图从中找出几样熟悉物品。真是想不起自己的来历了,连自己的年纪都算不出。他伸手拿起一只小陶盆,心中忽然迷迷茫茫的想道:``现在的粗瓷大碗都比它强,可当初还拿它当好东西呢\ldots{}\ldots{}''

思绪到此就中断了,他也不知道自己用没用过类似的器物。至于器物的真假,他也还是不确定。随手放下小陶盆,他席地而坐了,用手电筒的光柱扫射全室。箱笼整齐,倒也罢了,箱笼上面的各尊物品形态各异,却是在墙壁上投出各种离奇的影子。

无心出了一会儿神,莫名的生出了恐怖感。不是因为影子狰狞,而是因为孤独。守着满室的古老东西,他真怕时光倒流,自己要随着它们重新再活一场。一跃而起站稳了,他向前走了几步,忽然发现和洞口相对着的墙壁上,还有一扇小铁门。铁门没有锁,门轴甚至都没大生锈,推过几下便开了。他晃着手电筒迈进一步,就见里面还是一间石室。

石室很平常,和外间相比并无不同,然而空空荡荡,只在角落里摆了一口细长的棺材。对着棺材愣了愣,无心轻轻走上前去,发现棺材也不是严丝合缝,起码棺盖是松动的。

他把手电筒咬在嘴里,双手用力去推棺盖。在低沉的摩擦声响之中,他垂头一看,不禁吃了一惊。

棺材里的确是有人,人也的确是死人,并且死得不能再死,已经成了干尸。

问题是,尸体只有左侧一半!

\chapter{离散}

无心咬着手电筒,因为嘴巴张得太久了,所以口水顺着嘴角往下流。借着手电筒的光芒望向棺内干尸,他一吸口水,同时心想:``好刀功!''

的确是好刀功,从头至脚切得齐齐整整,连中间的胸椎骨都被平均劈开。他明白了棺材为何造成细长——凭着外面狭窄的入口,正常的棺材是难以进入的,恐怕当初的人也只是拖进了木板,到达石室之后才把棺材拼装成形。而半具干尸又能需要多大的空间?大概用窄木板拼成棺材样子,也就足以容纳他了。

思及至此,无心又特意摸了摸棺材板子——的确不是古老的木料,甚至料子都不算好,是最平常的板子。

把棺材盖彻底推开,他握着手电筒,将干尸彻彻底底的照耀审视了一番。干尸已经抽缩得快没人样,身上不着寸缕,从下身仅存的一只睾丸来看,绝对是个男人。无心垂头对他出了半天的神,忽然一笑。他的记忆力虽然坏,但还没有坏到一塌糊涂的地步。棺材里的阵势,他在很久很久之前,曾经见识过。

干尸的半只头颅,不知是用什么东西填充了,乍一看像是盛了一瓢干泥。干泥之中活跃着一点微弱的光,是干尸的魂魄,被镇在了尸首上。当然,魂魄不全,因为还有另外半具尸体。另外半具尸体在哪里?不好说。

同时无心也放了心。原来马天娇真的只是死于诅咒。没有毒,也没有什么传染病。五姨太受了影响,大概是因为马天娇带出的古鼎刚见天日,就被她捧到怀里的缘故。

室内的一切宝贝全受了诅咒,从它们见了天日开始,诅咒就发作了。

无心完全没把外间石室里的东西当成宝贝看,一些老得看不出岁数的陶器,一些锈迹斑斓的铜器,箱笼里还有什么?想必也都是老东西。在无心的眼中,它们加起来还抵不上一只崭新的铝锅。但是放在一般人的眼里,它们是国宝,牵扯着诸如``人类历史''之类的大题目。

肚子里咕噜噜的鸣叫出声,无心想起自己还没有吃早饭。

在无心研究干尸之时,地面上一片寂静。小柳治站在一棵小柳树下,两只眼睛各自为政,一边盯着士兵手中的古鼎,一边盯着洞口。马老爷尽量的远离了洞口,一张干巴巴的脸上没有表情。赛维和胜伊并肩而立,一动不动的望着洞口。马俊杰神情漠然,还抱着大树。

众人虽然形态各异,但是所思考的内容,却是差不多统一。人人都在暗自计算着时间,无心可是在里面停留太久了。

马英豪拄着手杖,无声无息的缓缓走动。无心不出来,他心里很焦急。事态已经够复杂了,如果地洞还能要人性命,对于他和小柳治来讲,就更是雪上加霜。围着洞口转了一圈,他向对岸远眺了片刻,随即无情无绪的轻叹一声,顺便往洞中扫了一眼。

一眼之间,他猝不及防的吓了一跳。不知何时,无心竟然已经从斜洞中伸出了脑袋。此刻他正抱着肩膀仰卧在下,只把一张苍白的面孔对了青天。一双眼睛倏忽间转向了上方的马英豪,他开口说道:``里面的情景,我看清楚了。''

他一出声,四周立时围上了一圈脑袋。马英豪开口问道:``里面是什么情景?''

无心平静的答道:``里面一共有两间屋子,第一间靠墙摆了一圈破烂,比如它——''

话到这里,他藏在斜洞里的身体有了动作,右手向上送出了一只绿莹莹的铜爵。

马英豪和小柳治的眼睛登时一亮,但是谁也不敢向下伸手去接。

无心缩回了手,只听隐隐的一声响动,仿佛是他把铜爵扔回了暗道:``第二间是空屋,里面只摆了一具棺材。棺材里面的东西,倒是比外间的破烂更有意思,我也带出来了。''

话音落下,他扭开了头,两只手似乎是在斜洞里使劲拖拽着什么。一丛干焦的毛发忽然冲出了洞口,随即是半张扭曲的人脸,像方才的无心一样仰面朝天,和上方众人打了个照面。

马老爷眼神很好,看了个清清楚楚,当场一屁股坐倒在地。赛维和胜伊一起怪叫一声,连着退了几大步。小柳治几乎把眼珠瞪出眼眶,连马英豪都倒吸了一口冷气:``什么东西?''

无心抬手搭上干尸的一侧肩膀,费力的把他又摁了下去:``应该是个萨满。守护洞中宝物的萨满!''

马英豪居高临下的用手杖指了他,正色问道:``到底是怎么回事?''

无心仰面朝天的没有动,是个事不关己的态度:``没什么,一种巫术而已。萨满法师用自己的性命施下了毒咒,专为守护洞里的老宝贝。''

马英豪早就看他可疑,如今看了他的反应,越发坐实了自己的猜测。飞快的瞟了赛维胜伊一眼,他对着洞中的无心低声说道:``你给我出来!''

无心歪着脑袋看他:``要不要顺便给你带出一两样?比如破陶盆锈酒杯?''

马英豪冷笑一声:``你想置我于死地吗?''

无心轻声嘀咕:``你好聪明。''

随即他晃着肩膀,像条长蛇一样从斜洞中一点一点游动向上。两只手扒上地面,他借力一纵,很灵活的跳回了人间。转身对着赛维笑了一下,他开口说道:``我没事。''

赛维面无表情的呆望着他,怀疑他会像马天娇一样,至多再有两天的寿命。她的目光又贪婪又悲怆,一言不发,心中暗想:``我会给你报仇的!''

无心向她走近了一步,微微弯腰去看她的眼睛:``赛维,我真的没事。''

赛维点了点头,声音哽在喉咙里,一句话也说不出,只能在心中作出答复:``你放心,我拼了性命也要给你报仇!''

因为她始终是不出声,所以无心只好转向了胜伊,微笑说道:``我饿了。''

胜伊惨白着一张脸,恨恨的转向马英豪说道:``你已经把人逼到死路了,现在让他吃顿饱饭,总可以吧?''

然后他对着无心又道:``无心,我们朋友一场,我一辈子也忘不了你。''

无心越对姐弟两个和蔼可亲,姐弟两个越是苦大仇深如丧考妣。他饿得心慌意乱,简直快要笑不下去。无计可施的咽了口唾沫,他连气都喘不动了。

只有马英豪若有所思的盯着无心,认为他可能真的``没事''。

地洞被一队标枪似的日本兵围住了,其余人等暂时离了花园。

他们回了马老爷所居的洋楼。赛维本来就是单薄的小脸,此刻一张脸越发紧绷,仿佛已经不能流露表情。

她都不敢再看无心,看一眼,心脏就被狠剜一刀。仆人从厨房运来了饮食,一样一样摆满了长条餐桌。谁也吃不下,甚至连餐厅都不肯进,于是她让无心坐了首席,自己和胜伊分别陪在两边。无心见自己面前摆着一屉热气腾腾的小包子,当即伸手抓了一个,抓完之后他左右看了赛维和胜伊:``你们怎么不吃?''

随即他忽然有点怯:``是嫌我脏吗?''

他把一屉包子全端起来了:``要不然,我出去吃?''

赛维一直绷着脸,绷到此刻她气息一颤,抬手猛的一拍桌面,走腔变调的怒道:``屁话,谁嫌你了?吃你的吧!吃还堵不住你的嘴!''

胜伊隔着桌子向她一挥手:``姐,你干嘛啊?你别骂他!''

赛维把脸一扭,``哇''的就哭了。

无心先把包子塞进嘴里,然后伸手一拍赛维的肩膀:``你以为我是在骗你吗?我没有说谎,我真没事。''

包子存在他的嘴里,撑鼓了他的一边面颊。见神见鬼的压低声音,他对着赛维和胜伊低声说道:``我会法术,我不怕诅咒。''

赛维咧着嘴转向了他,泪眼朦胧的收了嚎啕:``真的?''

无心一本正经的对他们说道:``你们记住,我是不会死的。''

赛维和胜伊怔怔的看他,感觉他不像是在撒谎,但是他的话又不合道理和逻辑。而他捏起一只包子又塞进嘴里,开始在二人的注视下大嚼。

无心凭着一己之力,吃了半张桌子的食物。马英豪走进来时,赛维正在夺他手里的大汤勺,生怕他活活撑死。而无心之所以能吃能喝,只是想要增长力气,保护姐弟二人。

马英豪停在门口,没有深入。颇为探究的盯着无心,他开口问道:``诅咒,如何破解?''

无心站在桌边,失控似的对他打了个饱嗝。

马英豪不动声色:``再问一次,诅咒,如何破解?''

无心摇了摇头。

马英豪一笑:``就知道你不会老实。''

随即他用手杖一敲房门。立刻有几名日本士兵一拥而入,反剪了无心的双臂。赛维和胜伊同时起身怒道:``大哥,你到底想怎么样?''

马英豪平平淡淡的答道:``借用一下你的朋友,如果他好用,我就不再找你们的晦气了。''

话音落下,他率先转身离开;而餐厅内的日本兵亮出手铐,咔嚓一声锁了无心的双手,一路推搡着他往外走。无心在临出门前,抢着又对姐弟二人说了一句:``记住,我不会死!''

赛维追着日本兵出了餐厅,连跑带跳的往楼上冲。楼上马老爷的书房里有枪,她今天一定要给马英豪来一枪!

胜伊没了主意,茫茫然的跟着日本兵往外走,眼看他们把无心押进了楼下的一辆小汽车里。马老爷则是把赛维堵在了楼梯上,死活不让她感情用事。而马俊杰独自蜷缩在角落里,只觉身上一阵一阵的冷,像有一股子寒风把自己吹成了透心凉,简直凉到了眩晕的程度。

小健正在暗处反复的扑向他。小健需要一具身体去救无心,非常需要。可是光天化日之下,他的力量微弱,抢不过马俊杰。

\chapter{宠物}

马英豪和小柳治在汽车里达成了共识——无论真相如何,他们都要把事情向上报告给军部了。

汽车队伍疾驰在通往天津的大路上,上午出发,晚上才到。汽车队伍分成两拨,小柳治一派不作停留,直接赶往稻叶大将官邸;马英豪一派则是直奔自家。

汽车络绎开进天津马公馆的院子里,日本兵把无心从车里押进楼内。马英豪奔波一天,右腿隐隐作痛。进门之后先吃了一片止痛药,他端着一杯热茶走到了无心面前,一边慢慢的喝,一边上下的打量对方。

无心的双手依旧是被手铐锁在背后,两名日本兵虎视眈眈的站在两旁,分别握住了他一条臂膀,两人静静的对视片刻,马英豪仰头喝尽杯中残茶,缓缓咀嚼着口中的茶叶渣子,他发现无心的眼睛很特别——黑眼珠太大了,微微陷在眼眶里,倏忽一转,快如闪电。

``请你到我家来。''他开了口:``谈一谈诅咒的事情。''

无心轻声答道:``我有要求。''

马英豪一挑眉毛:``说。''

无心说道:``我要撒尿。''

马英豪的脸上显出失望神情。对着两名日本兵说了一句日本话,他端着茶杯转身走到桌边,拎起茶壶又倒一杯。

两名日本兵没有为无心卸下手铐,而是一路跟他进了马公馆内的卫生间。无心毫不客气的连拉带尿,一切都由日本兵伺候着。而日本兵虽然属于战争机器,但也具有人的情绪。二人站在抽水马桶两侧,统一的皱着眉头,是有苦说不出的模样。

良久之后,无心回到了马英豪面前。马英豪看他脸上隐隐的带着点笑意,显然是很舒服,就忍不住好奇,又问一句:``还有要求吗?''

无心点了点头:``我\ldots{}\ldots{}饿了。''

马英豪一笑:``如果你我是萍水相逢,我此刻一定好好招待你。''

无心摇了摇头:``不必,家常便饭就可以。''

马英豪再次挑起眉毛,发现对方不傻装傻,把话全拧着说。既然如此,他只好单独直入的挑明正题:``如果你肯和我合作,荣华富贵还不是唾手可得吗?''

无心认真的正视了他:``大少爷,我无能为力。''

马英豪垂下眼帘,望着手中半杯热茶笑了:``无能为力?无能为力,就意味着没有价值。无心,你既没有价值,我又留你何用?''

马英豪不喜欢打持久战。他活了三十来年,一直处于备战状态,如今终于正式开战,他真想痛痛快快的速战速决。对于不听话的无心,他自有一套刑罚。当然不是深牢大狱里的老一套,他可没有耐性去做行刑人。

他把无心带进了他的密室里。让人扒下了无心的衣裤,他用手杖轻轻一杵半面墙大的玻璃缸,缸中新换了水,水位高出了他的头顶。几条海蛇在其中穿梭游曳,在电灯的照耀下,它们显得分外绚丽。

扭头望向无心,他轻描淡写的说道:``你现在唯一的用处,就是充当食物。''

随即他微微一笑:``不合作的代价。''

下一秒,无心腕子上的手铐被解开了,他被人高高举起,直接扔进了玻璃缸中。

扑通一声落了水,他在水中仰起头,就见一面铁丝网从天而降,罩在了玻璃缸上。而玻璃缸的边缘镶着一圈铁箍,铁箍每隔一段便有铁环突出,几把锁头挂上去,便能把铁丝网固定在玻璃缸上了。

马英豪等着无心服软求饶,所以并没有即刻上锁。然而隔着一层厚厚的有机玻璃,他只见无心缓缓下沉,没有恐慌,没有挣扎,只有几串银亮亮的细碎气泡,从他的耳孔鼻孔中逸出。

苍白修长的身体落到缸底,剧毒的海蛇们似乎没有当他是个活物,纷纷在他的腋下与腿弯之间穿梭,姿态是一如既往的灵动。

马英豪彻底愣住了,几乎以为自己是出现了幻觉。而无心在水中把脸转向了他,抬手拍上了玻璃缸壁。歪着脑袋继续探头,他的鼻尖在玻璃上贴出一个小平面。

海蛇的尾巴在他头顶盘旋扭绞,他向上一转眼珠,做了个天真好奇的表情,然后继续向前凝视了马英豪。

马英豪与他对视片刻,忽然爆发似的大吼一声:``上锁!快,上锁!''

无心双手全贴在了玻璃上,仰头去看几名半老仆人踮脚伸手,很费力的把铁丝网锁在了玻璃缸顶。玻璃缸太高了,仆人们虽然都算是高个子,但还是有人需要踩着小板凳借力。如果他猛窜上去,或许还能突破铁丝网逃脱,可是日本兵站在门口,他们全副武装,举枪就能把他也打成一张网。

于是无心就没有动。他自己倒是不怕什么,只是有点惦念北京的赛维和胜伊,并且真饿。

马英豪的手有一点抖,连带着手杖都软了,点在地上虚虚直晃,不能完全取代他的右腿。东倒西歪的出了密室,他心中狂乱的想:``怎么回事?''

随即他告诉自己:``水性好,一定是他水性好。老二老三是从哪里弄来的他?他到底是个什么东西?''

马英豪让仆人给自己拧了一把热毛巾,满头满脸的狠擦。擦过之后眨巴眨巴眼睛,他认定自己是太疲惫了,累糊涂了。于是他饭也不吃,一头倒在沙发上,闭了眼睛就想睡。身体沉重到了极致,反倒是轻飘了,他长长的呼出了一口气,只感觉自己虚弱至极,竟然一动都不能动。

仆人都消失了,客厅黑暗如同深水。忽然外面走廊响起了脚步声,有人来了。

他依旧是不能动,只能极力睁大一双眼睛。潮湿微咸的海水气味弥漫开,毫无预兆的,一只冰凉的手落在了他的咽喉间。一双乌溜溜的大眼睛显现在了他的眼前,是无心的眼睛。眼睛大极了,黑到不见了眼白,在暗中骨碌碌的乱转,像鸟,像蛇。

``我饿了。''他清楚的听到了这三个字,是无心说出的,看不见嘴,但是听得到话。

没有呼吸,没有热气,只有血腥味道直冲他的鼻端,让他很笃定的预感到了一口利齿的逼近。惊恐万状的大叫一声,他一挺身坐起来,眼前放了光明,原来方才只是一个梦。而搭在脖子上的冷毛巾落到腿上,是噩梦的始作俑者。

客厅里面的确是早没有人了,墙角的座钟倒是尽忠职守,在静夜中敲响了十二点整。马英豪摸过手杖,冷汗涔涔的起了身。单身汉的日子是不好过,他想,等到将来事情彻底完结了,自己应该把佩华接过来。两个都是苦命人,应该互相怜惜,况且她性情柔和,应该不会干涉自己的嗜好,比如养蛇。自己不抽大烟不嫖女人,养几条蛇,实在不算过分。

他一边想,一边出门进了走廊。慢条斯理的走向尽头密室,他且行且嗅,下意识的害怕梦境成真。最后摸出白铜钥匙,他打开房门,房内自然是伸手不见五指的,于是他蹲下来,在下方隐秘处摁了电灯开关。

玻璃缸旁亮起了一串小小的电灯泡,不足以照亮整间屋子,但是烘托出了一缸流光溢彩的水。玻璃缸正中竖起了一丛钢管,上面盘满了海蛇,水中就显得空荡了,只悬浮着一个无心。

骤然而来的光芒惊动了无心,他在水中灵活的转了个身,直勾勾的向外盯着马英豪。而马英豪看了他方才的动作,感觉他既像人又像蛇,在水中的样子,也很美。

玻璃缸再大,也大得有限,尤其无心生得长胳膊长腿,在里面就不能自如的游。马英豪仔细寻找着他的鳃,没有找到。而无心把一只手拍上玻璃,对着他张嘴说了一句话。

马英豪听不见他的声音,但是很好奇的抬起左手。隔着一层玻璃,他印向了无心的手掌,同时忍不住微笑了——即便无心当真再没有利用价值了,他也不打算要了对方的性命。他会制造一只更大的玻璃缸来容纳他,他看起来不是比任何海蛇都更有趣么?

无心收回了手,抬起双脚蹬上了玻璃缸壁。双手捂上腹部,他在水中做了个口型,正是一个``饿''字。

马英豪摇了摇头,无心是个不听话的,所以他准备杀一杀他的性子。他要饿出他的顺从与实话,如果饥饿都不能驯服他,马英豪想,自己只好行不得已之事,从赛维和胜伊中挑出一个带到此处,放点血给他看。

无心没有如愿,一挺身在水中做了个后翻。脑袋从水底向上钻出,他把鼻尖又贴上了玻璃。

马英豪越是细致的观察他,越感觉他不是人。隔着玻璃,他用手指轻轻一点无心的鼻尖,心态很奇妙的发生了变化,把无心和他的海蛇们归于一类了。

但还是不肯给他食物。海蛇们是美丽无邪的,而他并不无邪。马英豪知道他一定藏着一肚子秘密,只是不肯说。

\chapter{好奇}

马英豪无端生出了一种``神魂颠倒''的感觉。于是他及时离开密室,上楼睡觉去了。他是凭着脑力做事业的,需要充足的睡眠和清醒的头脑。天亮之后小柳治一定会带来稻叶大将的指示,而凭着他对稻叶大将的了解,大将对于宝藏和诅咒,必会抱有天大的兴趣。

他脱了衣服,泡了个短暂的热水澡,然后上床盖好羽绒被子。一切准备都做齐全了,可他还是只睡了几个小时。天还未亮,他就又睁了眼睛。

魔怔了似的,他不由自主的下了床,想要再去观察无心。

他一板一眼的穿戴整齐了,然后像游魂似的推了门往楼下走,没有开电灯,因为是自己的家,住了好些年了,闭着眼睛都不会走错一步。脚下一深一浅的走着,他的脑筋也在转动。眼看距离密室越来越近,他不由自主的生出了兴奋感觉,像小孩子将要拆开一份礼物,或是吃到一份美食。

将白铜钥匙插进锁眼,他在开门的时候,快乐的几乎要战栗。房门开了,咸腥空气扑面而来,潮湿寒冷的带了重量。他不舍得去开上方电灯,因为灯光自上而下的笼统倾泻,显示不出缸中海水的清澈剔透。他时常只打开玻璃缸下的一串小电灯泡。有限的一点点光明被水吸收,他的大玻璃缸暖洋洋的发了光,会变成一块巨大的黄水晶。

此刻,他弯下腰摁动了开关。大玻璃缸果然瞬间明亮了,可是并没有黄水晶!

他看到了一大缸血水,淡红的微透明,水中悬浮着丝丝缕缕的杂质。血腥味道越发重了,血水之中,是苍白的无心在半躺半坐。双手握住一条黑蓝相间的海蛇,他衔住了海蛇的头,正在专心致志的吮吸。浓重的红色从他的嘴角向外蔓延流动,是血。

扭头望向外面的马英豪,他赤条条的沉在血水之中,像母体中一具奇异的胎,非常平静,非常自然;张开嘴吐出海蛇的头,海蛇其实已经没有了头,头被他用牙齿咬掉了。

他咬死了缸中所有的海蛇,自给自足的喝饱了蛇血。残缺不全的死蛇们长条条的脱了节,胡乱绕在他的小腿和脚踝上。

马英豪的宠物们在几小时内灭绝,后来者居上,他现在只剩下了一个无心。而无心扔下手中的死蛇,忽然一跃而起,竟然向上一直窜出了水面。头顶随即撞上了铁丝网,他仿佛是猝不及防,当即四脚朝天的又沉了下来。抱住脑袋蜷起双腿,他吃痛的在水中翻滚了几圈,顺手抓起了一条死蛇。伸长双腿一蹬缸底,他举起双臂再次向上浮去。

手指穿透网眼吊住了身体,他仰起头,一个脑袋露出了水面。另一只手把死蛇也贴上铁丝网,他对着下方的马英豪说道:``给你。''

铁丝网的网眼太细密了,蛇身根本无法通过。所以马英豪可以好整以暇的反问:``为什么要给我一条死蛇?''

无心舔了舔嘴唇,嘴唇很红:``你把它蒸熟了给我吃。''

马英豪哑然失笑,随即轻声说道:``人到底是比蛇有趣。''

无心常年不会大喜大悲,即便是被马英豪锁在一缸冰冷的血水里了,他也并不恐慌愤怒,只是肠胃不舒服,想要吃点温热的饮食。他知道马英豪不会善罢甘休,其实他不说,是为了所有人好,但是自作孽不可活,眼看着有人偏要往死路里走,他也没办法。

马英豪没有接受他的死蛇,拄着手杖自顾自的离去了。他索然无味的松手向下沉去,不能总在水里泡着了,他想,他得设法逃生。

可还没等他想出眉目,房门一开,马英豪拎着一串小钥匙又回来了。伸手开了房内电灯,他用手杖从角落中拨出一只小板凳,然后站在玻璃缸前,饶有兴味的审视着他。

无心和他对视片刻,忽然捞起一条死蛇,作势又要向上浮出水面。马英豪微笑着摇头摆手:``不必不必,如果你肯和我合作,难道还怕我没有东西给你吃吗?''

无心依稀能够听到他的声音,但是不肯回答。

马英豪知道小柳治在天亮之后一定会来,而他并不想和任何人分享无心。小柳治如果知道了真相,也许就会把无心送去军部的秘密研究所里,而他又怎能和军部抗衡?

所以赶在小柳治到来之前,他得放出无心。横竖是放,不如顺便讲讲条件。很可惜,他想,老二老三先捡到了他,他就成了老二老三的人;如果当初在上海遇到他的是自己,自己现在就无需使用种种招数逼供了。他真的只是个无庙可归的落魄和尚吗?显然不是,要么是老二老三联合起来欺骗自己;要么就是老二老三也受了他的骗。

无心站在了水中,一手向前扶着玻璃缸壁,一手攥着半条斑斓死蛇,表情有点茫然,仿佛随时预备着向上窜。忽然抡起死蛇轻轻一抽玻璃,他垂下头做了个深吸气的动作。当然没有空气让他吸,但他的腹部的确是凹陷了,苍白皮肤下显露出根根肋骨的形状,可见他肚子里真是没了食。

抬手拍拍自己的瘪肚皮,他歪着脑袋望向马英豪,一切尽在不言中,还是要吃要喝。

马英豪笑了,一边笑一边踩上小板凳,很费劲的去开锁。

当最后一枚小锁头也被除下后,不用马英豪再出手,无心自己向上一头顶起铁丝网,双手扒住了玻璃缸沿。身体贴上滑溜溜的缸壁,他蜿蜒蠕动着向上攀爬。皮肤摩擦玻璃,发出刺耳声音,马英豪眼看他越爬越高,末了将一条水淋淋的长腿从缸内甩出来,他已经趴在了窄窄的缸沿上。

不动声色的斜出一眼,无心见马英豪正在下方眼睁睁的注视自己。马英豪让他在海水中吃了一夜苦头,他不由自主的生出了坏主意。

他打算从天而降,把马英豪砸个七荤八素,不是为了逃跑,而是为了报复。再次把眼珠瞟向对方,他骤然做了个失手的势子,张牙舞爪的从缸沿翻落而下,一屁股拍向了马英豪的头脸。马英豪当他无所不能,正在欣赏他的灵动体态,不料他竟然也会失误。下意识的后退了一步,马英豪连叫都没有叫出一声,只觉眼前一黑,已然被他砸了个仰面朝天。

在熬过后脑勺的剧痛之后,马英豪睁开眼睛愣了一下,随即扬起双手,恶狠狠的把骑在自己脸上的无心推出老远。无心软绵绵的不反抗,紧闭双眼蜷缩成了一团。而马英豪爬起来站稳了,一边用袖子抹脸,一边怒问:``你是怎么回事?''

无心哼哼的不说话,因为马英豪的鹰钩鼻子硌了他的蛋。他弄巧成拙,此刻疼得发昏。

马英豪随即拉开房门,伸手向外一指:``自己出去!只要你肯乖乖的听话,我自然不会亏待了你!''

无心长长的呻吟了一声,感觉自己的蛋都要碎了。哭丧着脸爬起来,他扶着墙慢慢的往外走,心中很想要一点温柔的关怀,可惜他如今仅有的好朋友,赛维和胜伊,都远在百里之外的北京;而且即便他们全在身边,恐怕也不会做出关怀的举动。

马英豪不给他衣服穿,怕他打扮的有人样了,会动心作怪,伺机逃窜。把他带到一楼的小餐厅里,他先让无心光着屁股坐在椅子上,然后自己靠着桌子站稳了,居高临下的问道:``说吧,有什么说什么。说清楚了,就让你吃饭。''

无心望着桌上的饭菜,饭是白米粥和热烧饼,菜只有一盘香肠,显然,此地的伙食比不上北京马宅。

伸手抓向烧饼,他心不在焉的打太极:``说什么?''

手伸到半路,被马英豪握住手腕又送了回去:``如果再明知故问的话,我就把你送给日本人。让日本人好好的研究你,看你到底是个什么东西。''

无心翻了他一眼,仿佛不甚情愿似的,低声说道:``我也不是百分之百的懂,说就说,反正我对府上的宝藏毫无兴趣,只希望我说过之后,你可以放我走。''

马英豪盯着他细看,始终怀疑他生了鳃:``不要讨价还价,我和你没有仇,对赛维和胜伊也没意见。只要你们肯如我的意,我自然不会伤害你们。''

无心点了点头,对着热烧饼开了口:``诅咒是可以破解的。''

然后趁着马英豪不防备,他一把抓过了烧饼:``只要能找到另一半干尸。''

马英豪紧盯着他:``什么意思?''

无心咬了一大口热烧饼,三嚼两嚼的咽了:``一种巫术,萨满法师发出诅咒之后,让人把自己活劈成两半,炮制成干尸。法师惨死时的痛苦和怨气,可以让诅咒永存。''

马英豪微微皱起了眉头:``另一半干尸在哪里?''

无心摇头答道:``另一半干尸,应该就在萨满法师的惨死之地。''

然后他把手中的烧饼撕成两半,对着马英豪重新一拼:``萨满法师的三魂七魄分别附在两半干尸上。只要把两半干尸拼成一具,萨满法师的灵魂就复活了。''

马英豪不以为然的一点头:``听起来是很恐怖。''

无心将一半烧饼填进嘴里,同时摇头:``不恐怖。等到法师的灵魂复活,你们找个有道行的高人,让法师魂飞魄散就可以了。法师一旦魂飞魄散,他所施加的诅咒自然也就失效。到时候洞里的破铜烂铁,你们想怎么运,就怎么运,绝对不会再出人命。''

马英豪舔了舔嘴唇,因为是受过科学教育的,所以总感觉自己一本正经的和无心谈论神鬼之事,有些荒唐:``你的话是真是假,我会找人帮我判断。''

无心没理他,捧着瓷碗喝大米粥,又把盘子端起来,用筷子将切好的香肠往嘴里拨。而马英豪若有所思的上下打量着他,看着看着,忽然说道:``你真像人,简直和人一模一样。''

无心听了,很不高兴,感觉自己是被马英豪揭了短。

正当此时,仆人在门口禀告道:``大少爷,小柳先生来了。''

\chapter{旧相识}

仆人刚刚禀告完毕,小柳治已经自作主张的走进了餐厅。一眼看清餐桌后面赤条条的无心,他把目光转向马英豪,颇为诧异的``哦?''了一声。

马英豪转身面对了他,用日本话低声说道:``我刚刚问出了一点眉目,你呢?''

小柳治答道:``古鼎已经被秘密送去了满洲,稻叶大将对此抱有极大兴趣,几天之内便会作出指示。''

马英豪一点头。他是时常会和小柳治分享秘密的,几乎从少年时代起,他们便结下了深厚的友谊。可是此刻他的舌头在嘴里打了几个转,有些话,可说可不说的,就还是强忍着没有说。

小柳治对着无心一扬下巴,又问马英豪:``他\ldots{}\ldots{}怎么回事?''

马英豪思索着答道:``他不老实,我使用了一点手段。''

无心听不懂日本话,所以索性收了心,一味的只是连吃带喝。双手端起人头大的白瓷盆,他把盆里的残粥全倒进了嘴里。马英豪一不留神,见他竟然狼吞虎咽的吃光整桌饮食。不由自主的回头看了一眼,他就见无心那白亮亮的肚皮已经鼓起来了。

疑惑的心思又生出来了,他盯着无心的肚皮,联想起了蛙和蜥蜴。是蛙和蜥蜴成了精?他抬眼又端详了无心的面孔,看来看去,没有找到一丝动物的痕迹,除了黑眼珠太大。忍不住侧身向他伸出一只手,马英豪用手背蹭了蹭他紧绷的肚皮,又用手指捅了捅他的肚脐眼。

捅完之后,他忽然回过了神,发现无心正在仰头看他,小柳治也是对着他目瞪口呆。若无其事的冷着脸,他知道自己方才是失态了,好在没有脸红的习惯,可以厚着脸皮混过去。

收回手清了清喉咙,他对着小柳治正色说道:``无心的话,我信不过。现在我们带他去见白琉璃。他的话有没有准,白琉璃应该会有判断。''

小柳治不置可否的先出了餐厅,而他对着无心一使眼色:``走。''

无心扶着桌子站起了身:``我还光着?''

马英豪没理他,只向着门口一挥手。

马英豪像赶羊似的,用手杖戳着无心往前走。小柳治跟在一旁,先是默然无语,后来将要到密室门口之时,才突然说道:``马君,我认为佩华女士是很好的,你应该把她接到天津来和你一起生活。否则一个人孤独久了,难免会生出一些古怪的念头。''

马英豪莫名其妙的看他:``什么意思?''

小柳治不言语了,低着头继续往前走。马英豪心里有事,也无意追问。把目光又射向了前方的无心,马英豪从他的后脖颈开始,沿着脊梁骨往下看,越看越糊涂,因为对方实实在在是个人样。而小柳治瞥了他一眼,看他盯着无心一眼不眨,就暗暗叹息一声,感觉老友有些变态了。

三人进入密室之后,小柳治对着一缸血水死蛇,又是很不赞成的一皱眉头;同时看见马英豪把扔在屋角的一件军大衣递给了无心。军大衣是小柳治偶然落在马公馆的,落下之后就被马英豪据为己有,他来要也不给他了。

地下室十分阴寒,马英豪怕无心这个活宝贝受凉,所以特地把军大衣奉献给他。弯腰打开地面第一道铁门,一股子成分复杂的潮湿空气登时冲了上来。马英豪还算平静,无心不呼吸,也能忍耐,唯有小柳治当年是充分接触过白琉璃的,如今就抬手紧紧捂住口鼻,苦不堪言的想要逃。

三个人络绎下去,把上下所有电灯全部打开。及至脚踏实地了,马英豪用手杖敲了敲第二道铁门。仿佛应和似的,地下传出了一阵低微的铃铛声音。

马英豪蹲下来继续开锁。小柳治翻着白眼,快要被熏得背过气去。无心拢着军大衣的前襟,饶有兴味的旁观。忽然浅浅的呼吸了一次,他怀疑自己是掉到粪坑或者尸堆里了。

第二道铁门也被掀开了,三个人神态各异的踩着铁梯向下走去。越往下走,灯光越弱,迈下最后一级铁梯,他们几乎是陷入了黑暗之中。

角落中响起了微颤的铃声,一大堆黑黢黢的物事动了动,正是白琉璃。默然无语的注视着前方三人,他忽然轻轻的``呵''了一声。

马英豪和小柳治看不清白琉璃的面目,正想花一点时间来适应眼前的黑暗,不料旁边的无心却是毫无预兆的开了口:``人生何处不相逢,是你吗?''

角落中的乱七八糟的一大堆有了动静,是白琉璃连滚带爬的开始移动。铃铛声音越来越近,以至于小柳治忍不住后退了一步。一个蓬乱污秽的脑袋由下向上探到了无心面前,白琉璃偏着脸,露出了尚且完好的蔚蓝眼睛。死死盯住了无心,他硬着舌头哑着嗓子,咬牙切齿的说道:``骗子!''

气流自作主张的钻入了无心的鼻孔,混合着白琉璃身上的恶臭。无心一张嘴,``哇''的一声,吐了他一头一脸的大米粥。而白琉璃满不在乎的抬袖子一抹脸,低低的又说一声:``骗子!''

马英豪在一旁开了口:``白琉璃,你认识他?''

白琉璃仿佛已经不能站久。脱力似的委顿下去,他趴在了上方射下的一束光中:``五年前,在西康,他骗我。''

马英豪对着地上的白琉璃眨巴眨巴眼睛,真没看出他有什么可骗的,于是转向无心问道:``你骗了他?骗了什么?''

无心睁着两只大黑眼睛,像是落了网的动物。而不等他回答,白琉璃抢先答道:``他骗了我全部的身家性命\ldots{}\ldots{}''

无心立刻摇头:``你也不要太过分,我承认我是偷了你三百英镑。''

马英豪略一心算,暗想三百英镑不是小数目,可也不至于要了白琉璃的命。哪知白琉璃喘息着继续说道:``是三百二十四英镑,还有六十八块法币。若不是你说要和我结交,我怎么会把钱给你看?若不是你带着我所有的钱逃之夭夭,我又怎么会去对麦基土司的儿子下蛊?麦基土司又怎么会去拉萨请大喇嘛来对付我?我如果不受伤,又怎么会被自己的蛊虫反噬?如果我没有被反噬,又何至于牺牲掉我儿子的性命?''

无心一屁股坐在了肮脏地面上,盘着腿对白琉璃苦笑道:``全算在我的头上了?''

然后他抬手挠了挠头,感觉颇为羞愧。五年前他流浪到了西康,偶遇白琉璃之后,的确是瞄上了人家的钱。他没钱,穷得快要吸风饮露,不由得就动了劫富济贫的心思。当时的白琉璃已经臭名昭著,是当地一尊人见人怕的邪神。无心不怕,每天笑眯眯的跟着他,跟着跟着跟熟了,就带着他的钱逃跑了。白琉璃的三百多英镑,让他很舒服的过了两年好日子。

他没想到白琉璃会倒霉在三百英镑上——白琉璃手中的每一张钞票,都是来历不明。他像一朵乌云似的飘飘荡荡,随心所欲的勒索土司。没有土司敢拒绝他的索求,因为他真能让人神不知鬼不觉的中蛊。无心偷了他的钱,自认为是盗亦有道。但是再怎么有道,也还是盗。盗总是个不光彩的行为。而白琉璃素来精明恶毒,没想到自己会糊里糊涂的栽在一个陌生小子的手里,并且还引发了连锁反应,从丢钱到死了儿子,时间都没有超过一年。

无心见白琉璃伏在地上,一个披头散发的脑袋一直哆嗦,就试探着伸手去拍了拍他的头:``我想办法去弄钱,还给你六百英镑,好不好?''

然后他缩回了手,从食指肚上拔下一根锐利的黑刺。白琉璃是个不能碰的人,从头到脚都是杀人的机关。

白琉璃听到了他的话,但是无法回答,因为真动了气,一颗心就在腔子里怦怦的跳,乱了他的呼吸。而马英豪旁听至此,心想无心偷钱当然不对,但是白琉璃也有讹人之嫌。从小柳治手中接过一只白手套堵住鼻孔,他在恶臭的空气中说道:``你们的私人恩怨先放在一边,反正将来总有机会解决。现在谈一谈眼下的正事。''

他把无心方才对他说过的一套话,一字不差的重复了一遍。话音落下,他和小柳治对视一眼,随即问白琉璃道:``怎么样?他的办法可行吗?''

白琉璃缓缓的抬起了头,铃铛随着他的动作轻轻的响:``我不知道。咒术,我不大通。但是我奉劝你们,不要轻易听信他的话。他是个骗子!''

无心专心致志的转动着大衣纽扣,因为不能否认又不愿承认,所以只好装聋作哑。

白琉璃开始慢慢的向后退,一边退,一边喃喃的又骂:``骗子。''

无心把纽扣扯脱了,抻出了长长的线头。

马英豪万没想到会是如此的结果,和小柳治面面相觑,不知接下来该如何是好。

在马英豪和小柳治无所适从之时,百里之外的北京马宅,也是一片愁云惨淡。

马宅的生活照常继续着,但是马老爷的自由受了限制,换言之,他被软禁在家了。

马老爷在认清现实之后,开始坐在书房里痛骂自己的爹——老不死的积点什么不好,非要千里迢迢的运些古董回来;古董也罢了,他妈的还来历不明,带着杀气。

如果马宅花园里埋着一大坑金银财宝,事情绝不会发展到如今的地步,因为如果单只是有钱,还不至于碍了日本人的眼。可花园地下的古董,已经有了国宝的嫌疑——马老爷的爹,把题目开得太大了!

马老爷气疯了,发疯之余又很悲哀,因为他的日本朋友们全噤了声,连电话都不肯给他多打一个。于是他为了发泄怒火,开始打姨太太,打得马宅哀鸿遍野。

赛维和胜伊虽然没有挨揍的危险,但是一想到无心生死未卜,两人的心口就被堵瓷实了,连口茶水都咽不下,脸上也生出了好几个红疙瘩。到了夜里,两人也不睡觉,坐在厢房的罗汉床上大眼瞪小眼。

互瞪了良久,因为全没主意,所以他们打着哈欠,想要各就各位的去休息。可是还未等他们下床,玻璃窗子忽然被人``咚''的敲了一下。他们一起扭头望去,隔着一层窗帘,就听窗外响起了马俊杰的声音:``二哥三姐,开门哪!''

赛维和胜伊一愣,心想哪里来的二哥三姐?不是二姐三哥吗?老五年纪小小的,也糊涂了?

\chapter{未遂}

赛维对马俊杰一点好感情也没有,可他既然来了,屋内又亮着电灯,二姐三哥也没有硬着头皮装聋作哑的道理。胜伊见赛维没有动的意思,只好伸腿下床,懒洋洋的走去打开了房门插销,向外伸出脑袋问道:``大半夜的不睡觉,你来干什么?''

马俊杰没回答,直接像条大鱼似的从他腋下钻进了房。胜伊一怔,从来没见五弟如此灵动过。而马俊杰进门之后站在了赛维面前,未语先笑,笑得两道眉毛扬起来,是个兴高采烈的狡黠模样。

赛维面无表情的看着他,因为是看着他长大的,所以怀疑他此刻是得了失心疯。胜伊关了房门转过身,也不言语,倒要看看自己的混账小弟能闹出什么幺蛾子。而马俊杰笑了片刻,见没人搭理他,就悻悻的收了笑容。鬼头鬼脑的回头溜了胜伊一眼,他又开口唤道:``二哥三姐,你们也没睡呀?''

胜伊张了张嘴,正要纠正他的错误,可是忽然接收到了赛维递出的眼色,便清了清喉咙,自顾自的走回罗汉床前,和赛维并肩坐下了。

赛维知道马俊杰虽然性情孤介,但是并不糊涂,不该在辈分大小上犯错误。不动声色的盯着他的眼睛,她心中凛凛然的,只感觉此刻马俊杰十分不像马俊杰。

``我们不睡,是因为我们有事情要谈。''她不冷不热的开了口:``你怎么也跟着当夜猫子?你现在夜里不睡觉,白天不上学,个头刚比桌子高,就想丢开书本鬼混了?''

马俊杰背过了手,幼童似的站在原地扭了扭,随即向前一探头,压低声音问道:``你们是在担心大哥哥吗?''

赛维缓和了语气,拿出了一点大姐的温柔问道:``你是说无心吗?我们当然担心他。''

马俊杰上前一步,弯腰用手扶住了罗汉床的床沿,歪着脑袋去看赛维的眼睛:``那我们想办法去救他好不好?''

这时别说赛维,就连胜伊都看出他的不对劲了。胜伊强忍着不发抖,只下意识的掏出一条紫色的大手帕,轻轻一拭额角的冷汗。赛维的心也打了哆嗦,可因知道无心不在身边,胜伊又比自己更柔弱,所以没有指望,反倒坚强。

``你说得对。''她正色答道:``我们也在考虑这件事情。既然你愿意加入,我们正好多了个帮手。地上凉,你脱鞋上床,我们好好的商量商量。''

马俊杰答应一声,一转身坐在床沿,弯腰去解皮鞋的鞋带。赛维虎视眈眈的盯着他,等他解开鞋带刚一直腰,便猛扑上去,把他压在床上反剪了双手:``你不是俊杰!说,你到底是谁?''

马俊杰在她身下挣了挣,丝毫没有转圜的余地,同时两只腕子被她攥得生疼,仿佛骨头都要断裂。带着哭腔哼唧一声,他立刻投降:``我不是坏蛋,我是大哥哥的好朋友!''

赛维把一颗心都提到了喉咙口,双手像铁钳似的又紧又硬:``你说你是他的好朋友,我怎么先前没见过你?你又为什么会和我家老五一模一样?你方才冒充我家老五,到底是何居心?''

马俊杰显然是真疼了,两条腿在床上一蹬一蹬:``呜\ldots{}\ldots{}我叫小健,我的身体被大汽车撞坏了,所以才借了马俊杰的身体用\ldots{}\ldots{}''

此言一出,赛维和胜伊全都竖起了一层寒毛——今晚真见鬼了!

十分钟后,赛维松了手,小健得了自由。抱着膝盖躲出老远,他自己揉搓着腕上痛处,真是怕了赛维。

赛维和胜伊统一的跪坐在他对面,中间隔着一张小炕桌。赛维问道:``也就是说\ldots{}\ldots{}你是一只小鬼,上了俊杰的身?''

小健委委屈屈的答道:``天亮我就会把身体还给他的。''

赛维和胜伊对视一眼,然后继续问道:``既然你只能在夜里上他的身,又怎能和我们一起去救无心?白天你是俊杰,不会听我们的话;夜里你倒是和我们一条心了,可是一夜的工夫,不够用啊!''

说到这里,她顿了一下:``除非\ldots{}\ldots{}''

除非之后的内容,有点缺德,不是一个做姐姐的人应该想的。但赛维自从受过俊杰的欺骗之后,满心都是痛揍小弟的念头,马俊杰是死是活,都不能让再她动心。所以在短暂的沉吟之后,她压低声音说道:``除非我们赶夜里的火车出发,天亮之前在天津找家饭店落脚,把你绑起来堵住嘴。等到天黑你上了他的身,再放你和我们一起去救人。''

小健立刻点头:``我愿意。什么时候出发?''

赛维转向了胜伊:``我敢去,你去不去?你不想去也没关系,正好留下来看家。''

胜伊看看赛维,又看看小健,开口答道:``我也去。冒险就冒险,反正我不要落单。可是在出发之前,我们也得先筹划好了才行。首先出大门就不容易,你忘了我们家现在是实行宵禁的吗?''

胜伊所言非虚,马宅如今的确是处在一个非常的时期,前后宅门全被便衣特务把守了,闲杂人等白天可以随便出入,但是一到天黑就要关门上锁。赛维和胜伊尽可以大白天的公然走出马宅,可人人都知道他们是马家的小姐少爷,无论他们走去何处,身后都有眼睛紧盯着。

赛维思索片刻,没有想出高明主意,倒是小健怯生生的开了口:``你家还有一道没人站岗的小门,你们不知道吗?''

赛维和胜伊立刻一起望向了他:``在哪里?''

小健轻声答道:``花园里呀!''

胜伊还没明白,赛维不由自主的一拍大腿:``可不是,花园里还有一道门。''

胜伊恍然大悟——后花园的确是开着一道铁栅栏门,但是早在他的童年时代,就被马老爷下令封锁住了,原因是当年有个姨太太上演夜奔,想要从后花园的小门和汽车夫私逃,结果被鬼魅似的马老爷捉了个正着。姨太太和汽车夫是怎么死的,现在只有马宅的老妈妈们才记得了,仅存的遗迹,便是一道被铁链子胡乱缠绕起来的小栅栏门。

用胳膊肘一杵赛维的肋下,他犹犹豫豫的问道:``我们\ldots{}\ldots{}夜里走花园吗?''

赛维向他一瞪眼睛:``你不敢啊?''

赛维的气焰越高,胜伊的火苗越低。茫茫然的看了姐姐一眼,他摇了摇头:``我敢。大家有福同享、有难同当。再说他也算是我的准姐夫了,我去救他,也是应当。''

赛维不再理他,伸手拉开了炕桌下面的小抽屉,从里面摸出一本列车时刻表。对照时间查了几趟车次,她心里有了数,低声说道:``要走就快走,留在家里只怕夜长梦多。明天怎么样?就坐夜里十点钟的特快列车。''

小健四脚着地的爬到了桌边,连连点头:``好,好,你们一定要带上我呀,我很机灵的,什么都能做!''

赛维听了他的话,不禁若有所思的叹了一声,感觉小鬼的一言一行,都比五弟可爱得多。

小健得了答复,心满意足的告辞离去。而赛维和胜伊各自安歇。到了翌日,他们若无其事的混过一天。到了入夜时分,两人勉力加餐,各自突破极限,居然分别吃了一整碗米饭。待到老妈子丫头都散去睡了,胜伊挑了一件带有厚绒里子的外套穿上,自觉很温暖了,便穿过院子去东厢房见赛维。

赛维坐在罗汉床上,正在抬腿往脚上套长筒靴子。胜伊见了,悄声问道:``姐,怎么着?你要骑马去火车站?''

赛维没理他,穿好皮靴之后站起身,她拎起一件短短的皮夹克,预备着像个摩登女英雄似的,到天津飞檐走壁去救无心。

把贴身的钱包又摁了摁,姐弟二人蹑手蹑脚的出了门。在院外的阴影处,他们看到了同样全副武装的小健。小健仿佛是很珍惜马俊杰的身体,生怕冻坏了他,不但头戴猎帽,颈系围巾,还加了一副兔子毛的耳朵帽,是个要过冬的打扮。不知他在外面等了多久,见赛维和胜伊出来了,他笑出了一口小白牙:``姐姐,哥哥,走哇!''

然后他一马当先的做了领头人,因为先前已经在马宅游荡了许久,熟知所有道路。

三人鬼鬼祟祟的向宅子后方走,马宅近来一直是个愁云惨淡的气氛,时节又进入了深秋,寒气逼人,所以一旦入夜,宅子里的人便各归各位,不肯出屋。三人一路走得顺顺利利,眼看前方就是花园,可领路的小健忽然刹住脚步,把脸转向了左侧的花木丛。

在恐慌之前,赛维下意识的也跟着他扭了头。身后的胜伊则是抬起了手,强行捂住了口中一声惊叫。

花木之后,月影朦胧。一个花红柳绿的身影静静伫立在夜风中,花白长发随风飘动,长发之下,正是五姨太的面孔。

五姨太自从发疯之后,就被马老爷锁在了她平日所居的院落里。她倒还是个文疯子,在接下来的时日中不吵不闹,所以马宅人心惶惶,众人竟是一起淡忘了她。

小健认得五姨太,所以一时不知如何是好;胜伊看五姨太人不人鬼不鬼的,则是吓得两条腿一起没了骨头;唯有赛维定定的凝视着她,两只薄薄手掌垂在身体两边,细瘦手指缓缓握成了拳头。

``谁敢挡我们的路\ldots{}\ldots{}''她毫无顾忌的开了口,说给在场所有的活物听:``我就掐死谁!''

然后她向前一拍小健的肩膀:``走!''

小健毕竟是个小孩子,看出了赛维的权威,便心甘情愿的把她当成了主心骨。她让走,他就大踏步的继续前进。三人像一队临时拼凑出的大号童子军,齐步走着开进花园,没有人再回头。

花园里面,和先前相比,又换了风光。小河对岸的山顶凉亭,已经被日本兵用一座大帐篷彻底扣住,昼夜都有士兵看守。于是小健不敢靠近河边,只在花木丛中小心穿行。沿着河流的方向一直走,走到尽头便是花园的小门。

然而走了不久,小健忽然又停了脚步。三人抬头望向前方,再次看到了一丛玫瑰树后的五姨太。

没人知道她是怎么追上来的,甚至没人能确定她此刻是人是鬼。直挺挺的面对着三人,五姨太开了口,声音嘶哑而冷:``血。''

赛维心算着时间,不肯和个疯子多费口舌。把小健拉到自己身后,她迈开大步,对五姨太视而不见。

而五姨太轻声又道:``血,好多血。''

然后她抬手抱住肩膀,身体骤然开始剧烈战栗。双手渐渐下滑,她低头望着自己身体,开口发出怪异的哀鸣,看她的举动,竟仿佛是她的身体将要一分为二,而她正在用手臂极力箍住自己。

赛维不怕她疯,怕的是她发出动静,引来小河对岸的日本兵。暗暗的把牙一咬,她预备使用武力打晕五姨太。可是未等她出手,五姨太忽然猛一挺身,好像痛苦到了不堪的地步,张开双臂就往她身上扑。而赛维冷不防的见了她张牙舞爪的模样,吓得当胸踢出一脚。她虽然瘦,但是很有一股子爆发力气,满拟着一脚能把对方踢飞。不料五姨太顺势抱住了她的小腿,低头就咬,正咬在了她的靴尖上。隔着一层软牛皮,她很清楚的感觉到了五姨太的好牙口。拼命把腿往回一收,她随即暗叫不好——靴子被五姨太叼住留下了!

她光了一只脚,显然没了长途跋涉的资本。而五姨太把靴子向后一扔,十指芊芊扒住胸前袍襟,就像有人要挖她的心肺一样,龇牙咧嘴的仰起了头,身体一阵一阵的剧烈颤抖。忽然听得一声古怪轻响,胜伊大叫一声,发现五姨太竟然把手指插进了胸膛!

双手用力扒向两边,夜色之中,五姨太的胸襟是一片暗黑淋漓。神情狰狞的向前踉跄一步,她哑着嗓子说道:``血\ldots{}\ldots{}好多血\ldots{}\ldots{}''

无须号令,赛维一手扯起胜伊,一手扯起小健,沿着来路转身就逃。一鼓作气冲出花园地界,他们不敢停留,生怕五姨太和日本兵追随而来。正是气喘吁吁一路狂奔之时,他们迎面被管家堵住了。

管家看了他们的模样,十分惊奇,可是来不及多问,只急急的说道:``二小姐三少爷五少爷,稻叶大将刚刚来了,如今正在前头楼里和老爷说话。老爷偷着让我来向您几位报信,说是情况吉凶未卜,让大家都清醒着别睡!''

\chapter{两处闲愁}

赛维、胜伊以及小健,刚刚回房缓过了一口气,就接到家中的内线电话,被马老爷叫去了前头的小洋楼。

赛维换了一身家常衣服,做女英雄的豪情壮志全没有了;胜伊跟在一旁,一颗心就在腔子里怦怦直跳;马俊杰依旧是不受待见,不得召唤,于是小健正好如愿,独自留在房内等待消息。

赛维和胜伊出现在马老爷面前时,稻叶大将已然离去了。大将如风,倏忽来倏忽去,但已足以刮得马老爷面无人色。裹着一件红底白花的丝绸睡袍,马老爷因为也是出乎意料,所以一时忘形,脑袋上还顶着压发的小帽垫——他老人家天生一头卷发,须得时时镇压,否则一个脑袋能热闹成一颗大爆米花。

对着一对酷似自己的龙凤胎,马老爷顶着帽垫点了点头,咬牙切齿的从鼻孔中往外呼气:``你们的朋友在天津都说了些什么?稻叶把事情搞大了!''

赛维狐疑的正视了父亲:``爸爸,怎么了?稻叶来找你干什么?''

马老爷苗苗条条的站在楼梯上,微微的有一点摇晃,看起来绚丽而又婀娜,然而一张保养良好的干巴脸上,神情却是惶恐凶恶:``他\ldots{}\ldots{}他要派遣秘密小队,前往满洲寻找干尸!''

随即他目光如电的扫视了赛维和胜伊:``老大是站在他们一边的,一定是吹了什么妖风,让稻叶指名要我随行!我一把年纪了,一身的老骨头,跟着他们去满洲?''

话到此处,他恶狠狠的一咬下嘴唇:``除了我之外,还有你们!''

不等儿女回答,他失落的长叹一声:``我很后悔,当初不应该从政,我若是做学问,一定成绩也很好。如果我是个学者,大概早在战争爆发时就逃去重庆了,也不会为了名利,坏了名誉。至于后花园里的古董,我从未享受到它的任何好处,反倒要为它押上一条老命,思及至此,真是让我恨到肝胆俱裂。如果时间允许的话,我都想刨了你们爷爷的坟鞭尸!妈的!''

赛维和胜伊看了他的狰狞面貌,全吓得不敢言语。

马老爷又看了他们一眼,一双眼睛里燃烧着愤怒的火焰:``事到如今,我们已经走投无路,只好见机行事。从此刻开始,你们都给我老老实实的呆在家里待命。我可禁不住再出什么乱子了!想我为了政务呕心沥血,本以为明年可以高升一步,怎料到会有如今的一幕闹剧?高升一步可以不必想了,我现在只求能够从满洲平安返回。只要逃过此劫,我\ldots{}\ldots{}我宁可\ldots{}\ldots{}''

马老爷欲言又止,不肯再说,一双眼睛发着电,目光特别的有劲,似乎快要迸出火花。赛维和胜伊塌着肩膀垂着脑袋,全成了落网的鸟。其中赛维还算存有一点勇气,能够嗫嚅着说道:``爸爸,刚才我们在\ldots{}\ldots{}在外面见到了五姨娘。五姨娘胡言乱语的,还用手抓胸膛。天黑,看不清楚,好像都抓出血了\ldots{}\ldots{}''

马老爷不耐烦的一挥袖子:``让她去死!''

赛维立刻就闭了嘴。

翌日上午,一个日本兵在花园里发现了五姨太的尸首。管家去看了,回来硬说花园里有大野兽,因为五姨太是个开了膛的死状,开的不大,从心口撕扒往下,肠子还揣在肚子里,但是肺叶子可全晾在外头了。

马老爷根本不理会——他现在很闹心,天下人死绝了,也和他没有任何关系。

又过了一天,一辆全副武装的小汽车停在马宅门前,把马老爷和塞维姐弟全接走了。

马家三人踏上前往天津的旅途,一路心中惴惴,惶惶不可名状。与此同时,马英豪倒是把日子过得心旷神怡,心情类似幼童得到了一件新玩具,不但爱不释手,并且恨不能把玩具拆开,从里到外看个透彻。

伸手捏着无心的下巴,他像个牙科医生似的,握着手电筒往对方嗓子眼里瞧。嘴的确是人的嘴,嗓子眼柔嫩粉红的吞咽着口水。放下手电筒,他亲自上了手。手指肚试了试无心的牙齿,无心并没有生出獠牙,但是牙齿也够厉害,带着一种新生的锋利。

无心坐在椅子上,仰着头张大嘴巴,同时垂下眼帘看他。马英豪把嘴唇抿成了一条线,是个很紧张很专注的神情。拇指食指捏住他的门牙摇撼了几下,马英豪问道:``你是杂食动物吧?''

无心一听,简直气死了。奋力的一晃脑袋甩开了马英豪的双手,他开口答道:``我和你是一样的!''

马英豪没生气,手指轻轻抚过他的耳后和脖子:``你说实话,你的鳃在哪里?''

无心把脸扭开:``我不是鱼,我没有鳃。''

马英豪忽然捏住了他的鼻子,同时直勾勾的盯着他看。无心懒得再正视他,索性闭了眼睛。

良久之后,马英豪松了手,喃喃自语道:``不对啊\ldots{}\ldots{}不合乎道理\ldots{}\ldots{}''

然后他忽然问道:``赛维和胜伊知道你的本来面目吗?''

无心答道:``我们之间的事情,和你没有关系。''

马英豪后退了一步,把双臂环抱在胸前,换了个角度宏观的审视他:``真是奇怪\ldots{}\ldots{}你活了多少年了?''

无心发现马英豪简直堪称人间奇葩,自己连沧海桑田都见识过了,唯独看他稀奇:``大少爷,格物致知也该有个限度。我不知道我活了多少年了,我不识数,也请你不要再问了,现在是个文明的年头,个人都该保留一点隐私,对不对?''

马英豪站不稳,所以还是重新拄起了手杖:``有意思,你还会说`格物致知',还知道`文明'与`隐私'。看来你是很有智慧的,不可思议。''

然后他歪着脑袋,又去端详无心:``你交配过吗?''

无心愣了一下,随即起身向后转,背对着马英豪骑在了椅子上。双臂横撂在椅子背上,他俯身低头,把脸埋到了手臂之中。不能再理睬马英豪了,他已经和马英豪连续交谈了十几个小时,马英豪没有一句话是让他舒服的。

脚步声音由远及近,最后绕到了他的面前。一只手搭上了他的后脑勺,缓缓抚摸他细密的短头发:``为什么要接近赛维?我看你也是喜欢女人的吧?''

无心直起了腰,可是依旧低着头。抬手摸上头顶,他把马英豪的手拽到了面前。手很干净,手指修长,骨节微微凸出,正是一只规规矩矩的男人手。无心最后翻了马英豪一眼,发现马英豪居高临下,正在望着自己微笑。

因为实在是厌烦到了无以复加的地步,所以毫无预兆的,无心探头一口咬住了马英豪的手,咬出``咯吱''一声,仿佛筋肉骨骼都错了位。马英豪发出惨叫,正要抡起手杖去打无心,然而无心已经松了口。

虎口上出现了一排牙印,鲜血顺着牙印往外渗,很快就聚成了大血滴子。无心伸出舌头一舔血滴,然后抬头告诉马英豪:``不要问了,再问我就吃了你。''

马英豪握着手杖中段,用手柄轻轻一敲自己的太阳穴:``是我失误。我又把你当成人了,忘记了你比海蛇更厉害。''

然后他笑着把伤手送到无心嘴边:``还有血,要不要喝?''

无心打开了他的手,然后抬头望着他苦笑:``大少爷,你比白琉璃还要人命。''

十几个小时前,马英豪再次带他去见了白琉璃。白琉璃看起来是一副病入膏肓的模样,伏在地上只是喘气。从头至尾,他只和马英豪讲了几句话,完全不理睬无心。及至马英豪要带着无心离开了,他才像一条泥涂中的病蛇一样,将一只蓝眼睛转向了无心。

无心在他面前是个好性子,察觉到他的目光了,便情真意切的告诉他:``你多保重,有朝一日我发了财,一定还给你六百英镑外加两百法币。''

白琉璃缩在一大堆肮脏污秽的兽皮之中,气息奄奄的答道:``在我离开西康的时候,法币已经开始贬值了。''

无心略一思索,随即答道:``那我就不给你法币了,直接还你六百英镑。''

白琉璃的蓝眼睛在角落中黯淡了,往兽皮里又缩了缩,他忽然换了四川话,哑着嗓子含混骂道:``狗日的贼娃子。''

无心身在天津马公馆,除了没有自由之外,所见所闻也没有一样能令他快乐。他虽然喜欢和人亲近,但马英豪与白琉璃显然算是例外。

所以当他忽然见到赛维和胜伊之时,心情几乎就是狂喜了。

赛维和胜伊是在下午到达马公馆的,进门时身后还跟着几名便衣青年。马英豪当时刚刚打完一个长长的电话。放下电话带着无心走进客厅,他风度很好的对着二妹三弟点头:``路上辛苦了。''

赛维都存了杀他的心,可是因为杀不得,所以有说有笑,反倒比平时更友好:``大哥,我们下车之后已经休息了一阵子,并不辛苦,就是惦念着无心,想看他一眼。''

马英豪微微侧身,给身后的无心让了路。无心正越过他的肩头,向胜伊使眼色。胜伊接收到了他的无线电,也是挤眉弄眼的想要作出回答。忽然正式面对了赛维,无心收回目光,没好意思和她行拥抱礼,所以就只是望着她笑。

赛维经了大半天的奔波,脸上的胭脂粉全脱落了,显出了一点病容,可是一双眼睛相当的亮,是个人精的模样。无心笑,她上下打量了他,看他伸伸展展的安然无恙,不由得也笑了。

``反正大家都是合作的关系了。''她笑微微的对马英豪说:``大哥倒也大方一点呀!早知道他没有像样的衣服穿,我就从北京给他带一两套了。''

无心的确是穿的不对劲,身上是一套马英豪的旧睡衣,没有鞋袜,光着脚满楼跑。马英豪打了个哈哈,英俊的面孔皮笑肉不笑:``你们的朋友,和我不是一条心,我还不是怕他逃了?''

赛维听他公然的把无心当成囚徒看待,脸上肌肉抽搐,简直快要笑不下去:``以后我们替你看守他,看他往哪里逃。可是我们尽管愿意做狱卒了,监狱到底在哪里,大哥能否提前告诉我们呢?''

马英豪摇了摇头:``不急,等到出发的时候,你们自然就知道了。''

胜伊忽然说道:``我们只知道是去满洲,满洲可就大了,知道等于不知道。大哥,我们又不可能出去扩散消息,你私下告诉我们一点内幕,又有什么关系?''

无心不动声色的拉起了赛维的手,又回头问道:``我也去吗?''

马英豪一点头:``没错,你也去。''

无心问道:``去哪里?''

马英豪忽然笑了,看他和人一模一样。短暂的迟疑过后,他开口答道:``齐齐哈尔。''

无心感觉到赛维正在用力攥着自己的手,于是也回握了过去。一点隐秘的小喜悦在胸中缓缓生出,几日的分离之后,他们之间渐渐酿出了爱情的味道。赛维没有看他,他也没看赛维,两人只通过一点你来我往的小力气打着招呼。

赛维和胜伊尽管一团和气,恪守了作为妹妹弟弟的本分,但在半个小时之后,还是被更为和气的马英豪送走了。

赛维和胜伊都很识相,让走就走,因为马公馆门外站着荷枪实弹的卫兵,不是个寻常地方。

马公馆恢复了宁静。马英豪打开了一部留声机,放了一张日本唱片进去。演歌的调子颤巍巍的出来了,他问无心:``好不好听?''

无心赤脚蹲在一把椅子上,摇头答道:``不好听。''

马英豪饶有耐性的换了一张片子。唱针搭上唱片,大喇叭里响起了一段洪荒辽远的吟唱,他扭头去看无心:``蒙古调子,喜不喜欢?''

无心继续摇头:``不喜欢。''

马英豪伸手拍了拍他的脑袋:``你只喜欢吃。''

无心知道他始终是不把自己当人看,所以无话可说。

\chapter{半路折翼}

在一个雾蒙蒙的清晨,马英豪推开一扇木格子玻璃门,探头进去问道:``你在干什么?''

无心坐在抽水马桶上,``唰啦''一抖手中报纸,气急败坏的抬头答道:``明知故问,我在大便!''

马英豪用手杖轻轻一敲玻璃门:``抓紧时间。''

无心翻了个淋漓尽致的白眼。

马英豪又道:``衣服在浴室里,希望尺寸合适。''

无心歪着脑袋皱眉看他,同时轻声吐出一句话:``滚出去!''

马英豪一挑眉毛,后退一步,为他带上了玻璃门。

今天既然是启程出发的大日子,无心猜想自己一定有机会和赛维姐弟见面了。

他很高兴,虽然前途未卜,不能预料自己是踏上了一条什么道路。仔仔细细的洗了个澡,他穿上一身崭新的长袍马褂。挽起袖子坐到餐桌前,他对马英豪视而不见,眼里只有一大盘子热烧饼。

马英豪亲自给他盛了一碗米粥,口中说道:``打扮好了也不像少东家。''

无心强迫自己心平气和,不和他一般见识。忽然斜斜的瞟了他一眼,无心低下头开始吹着热气喝粥。而马英豪察觉到了他的一眼,心中不由得别扭了一下,因为有一丝悲悯的光闪过了无心的瞳孔。为什么是悲悯呢?他在对谁悲悯?又是为何悲悯?

马英豪没有多问。安安静静的吃过一顿早饭,他带着无心向外走去。无心好一阵子没出过门了,终于见了天日,却又是白雾弥漫,无天无日。一辆军用卡车停在马公馆的大门外,车上放着一只大木箱。无心若有所感,向马英豪问道:``还要带上白琉璃吗?''

马英豪点了点头,又说:``他不会和你结成同盟的,你还是乖乖的跟着我走吧!''

话音落下,一辆小汽车开到了门口。一名日本军官下了汽车,用日本话对马英豪打了一声招呼。马英豪一边回应,一边拉着无心的手往外走。碰触无心的感觉很刺激,因为他得时刻提防着无心咬人。他的左手直到现在还包着一层薄薄的纱布,纱布下面,是个结了血痂的牙印。

汽车发动,领着军用卡车驶上大街,直奔东局子机场。良久之后,汽车抵达机场,停在了一片开阔空地上。马英豪带着无心下了汽车,就见前方站了一大群便装人士,为首一人乃是西装革履的小柳治,旁边三位等高的老少瘦子,正是马老爷以及赛维胜伊;而胜伊身边站着个半大孩子,却是马俊杰。

双方会了面,无心见赛维和胜伊还是往昔的小姐少爷模样,马老爷也一如既往的很体面;而马英豪对着马俊杰笑了笑,开口问道:``俊杰也要去吗?''

小柳治用日本话低声说道:``很奇怪,他竟然藏在了汽车后备箱里,偷偷的跟来了天津。你的家人全没有发现,我们的人,也没有发现。''

马英豪又问了马俊杰一遍:``你想去?''

马俊杰的表情有些痴傻,茫茫然的张了张嘴,他小声答道:``我不知道\ldots{}\ldots{}''

他的确是不知道,他已经连着许多天都像是处在梦游之中,他甚至都不知道自己是怎么进入汽车后备箱的——那么远的路,那么冷的天,他居然抗下来了。

和小柳治对视一眼,马英豪不再理会他,只问:``现在登机?''

小柳治一点头,然后侧身向远方一挥手。一架灰头土脸的军用飞机静静的停在雾中,舱门大开,正在等候他们进入。

一行人等迈开步子,心事重重的登上飞机。机舱里已经有了几名乘客,也都是便装打扮,其中有一名富态的光头,一位精壮的青年,还有一个低眉顺眼的小女人。无心垂着双手,自作主张的就要去和赛维同座。赛维心中暗喜,不假思索的撵开胜伊,让无心快坐。胜伊十分不满,又见马英豪也是落单,吓得连忙一屁股坐到了马俊杰身边。未等他坐稳,同样落单的马老爷拉警铃似的清了清喉咙,胜伊略一寻思,强忍嫌恶,起身又挪到了父亲身边。几名士兵抬着一只大木箱也上了飞机,把木箱很妥当的安置到了机舱后部。

马英豪望着无心,见他坐得十分踏实,并且已经系好了安全带,就自找空座坐了,又对小柳治说道:``今天不是个好天气。''

小柳治神情不定的对他一笑,随即忽然双掌合十,闭目垂头拜了拜。

正当此时,飞机在跑道上开始缓缓滑行,他们的旅途,拉开了序幕。

无心生平第一次坐飞机,好奇的把脑袋一直探到舷窗前向外张望。赛维靠着窗子坐着,鼻尖可以蹭到他的鬓角。无心显然也有所知觉,忽然偏过脸对着赛维一笑,他摸索着又握住了对方的手。

赛维也抿嘴笑了,看无心的侧影很好看。她承认以貌取人是肤浅的行为,她自己也不是美人,然而野心勃勃,敢于为自己找一名美男子夫君。鼻尖在无心的短头发上蹭了蹭,她嗅到了一股子淡淡的香皂气味。眼珠在眼眶里四面八方的转了一周,她趁人不备,忽然一撅嘴,在无心的太阳穴上亲了一下。

无心把脑袋缓缓的向她歪了过去,最后竟是快要靠在了她的胸前。赛维低下头,正好可以看到他乌浓的眉毛与笔直的鼻梁。他的肩膀挤在她的胸前,没有肉感,只有肋骨。赛维也知道自己的缺憾,但是不大往心里去,只暗暗的对自己说:``他是我的。''

无心的身体越来越柔软沉重,像是被人抽去了骨头,懒洋洋的往她怀里依偎,眼皮也半垂了,是个很慵懒的舒服样子。忽然一攥赛维的手,他一歪头,把脑袋直送到了赛维的眼前,仿佛是想让赛维再亲一下。赛维腾出一只手,在他头上弹了一指头,又在马达轰鸣声中低低说道:``别闹。''

无心缓缓转过了脸,去看赛维的眼睛。赛维的相貌不大稳定,本质是带着病容的,可``十八无丑女'',搽点脂粉便是一朵桃花的颜色,当然,是朵贫瘠土地中生长出的瘦桃花,一不小心就是青黄不接。

无心和赛维含情脉脉的大眼瞪小眼,正是将要情不自禁之时,身下忽然起了震动。后方的马老爷和胜伊一起惊叫了一声,一直默然无语的胖子和青年却是面不改色。而小女人则是解开安全带起了身,迈着内八字步一路颠向前方驾驶舱,也是个八风不动的镇定模样。

马英豪先前一直在和小柳治讨论天气问题,此刻回头向后看了一眼,随即对着距离自己最近的无心和赛维说道:``不要怕,即便遇到最坏的情况,飞机也可以就地降落。''

小柳治听他说话很不吉利,故而转身摆了摆手,用中国话说道:``哪里,总不至于迫降。最近的天气不大好,飞机大概只是遇到了强气流。''

话音落下,飞机毫无预兆的在高空中翻了个身。无心本来正在赛维身边瘫软,此刻猛然挺身,一把将她搂到了怀里。马英豪勃然变色,极力的起身去看舱后大木箱。而小柳治一把将他拽着坐下,同时用日本话向前方高声吼道:``怎么回事?''

小女人从驾驶舱中踉踉跄跄的跑了出来,忙而不乱的坐回原位。未等她系好安全带,飞机接连着又打了几个滚。赛维死死的抱住了无心的腰,紧闭双眼咽下惊叫。马老爷咬紧牙关,还算镇定的抓住了胜伊的手。胜伊哀鸣一声,不是怕空难,而是因为被父亲结结实实的触碰了。马俊杰独自缩在最后方,双臂环抱着肩膀,面无表情,还是感觉自己在做噩梦。

一名飞行员从驾驶舱中冲了出来,对着全机舱人用日本话长篇大论。待他话音落下,坐在小女人身边的光头开了口,声若洪钟的做出反问,气息丝毫不乱。三言两语的交谈过后,光头用对小柳治一挥手。小柳治当即高声说道:``飞机遭遇到了强气流,即将紧急降落,请诸位打起精神,保重自己!''

马老爷登时大声问道:``我们现在到了哪里?''

小柳治无暇多想,望着白茫茫的窗外,他支支吾吾的答道:``也许是黑龙江?''

舱后忽然起了巨大的响动,众人回头一望,发现巨大木箱虽然被一层帆布网固定在了机舱地面上,但是经过几次大颠簸之后,帆布网有所松动,大木箱已经有了移位的趋势。木箱十分结实,四角包了铁皮,真能砸死活人。与此同时,飞机机头骤然翘起,在空中做了个鲤鱼打挺,随即倾斜着一头向下扎去。在众人的惊呼声中,大木箱子终于挣破帆布网的束缚,随着惯性横撞向了舱壁。一声巨响过后,机舱之内天翻地覆。胜伊又嚎叫了一声,因为马老爷拉起他的手,把他的手背贴上了自己的额头:``噢!我的上帝啊!''

飞机像是发了疟疾,打着摆子向下降落,仿佛随时可能失控。千辛万苦在崎岖山路上着了陆,飞机东倒西歪的向前疾冲,一路扫断无数草木,末了撞上一截断崖,算是强行止住了滑行。舱内的乘客们被吓得头晕目眩,所幸全未受伤。一个个连滚带爬的下了飞机,马英豪一手拄着手杖,一手扶着小柳治,在冷风中打了个寒战,无话可说。

马老爷背负双手,也不吭声,赛维和无心手拉着手,一起站在远处。倒是满面放光的光头最有主意,对着小柳治嘀嘀咕咕低语一番。小柳治随即做了翻译,原来光头认为当下的要务,乃是寻找援兵救助。寻找援兵,也不是为难的事情,到最近的村子里应该就能找到日军小队。此刻他们的队伍中有老有小,大部分人可以留下看守飞机,派出小部分人出去联络便可以了。

随即光头又插了嘴,建议无心和小柳治同去,又把自己身后的青年也推上前方:``还有金子纯。''

金子纯看起来是位结结实实的日本青年,无甚特别之处。而赛维一见无心要走,立刻表示自己也想随行。光头见她是个很利落的姑娘,并没有娇滴滴的态度,就点头表示了同意。

一行四人组成小队,仰头看了看白蒙蒙的天光,然后认定方向向林外走去。深秋时节,华北还有一点暖意,东北却是已经冷得有了冬天气息。四个人一路跑跑跳跳,不出片刻便走出老远。沿着山路一拐弯,小柳治和金子纯还在兴致勃勃的齐步走,无心却是停了脚步,感觉周遭气氛有些不大对劲。

果然,路边的荒草丛中窸窸窣窣有了响动,几只黑洞洞的枪口无声伸出,几个粗喉咙也一起开了腔:``站住!''

随着吆喝,几名虎背熊腰的大汉端着长短枪,弯腰从草丛中站起身走到了路上,将四个人团团围住。小柳治咽了口唾沫,极力说出最标准的中国话:``你们是什么人?''

远方来了一只小毛驴,驴背上坐着个穿花袄的小媳妇。待到小毛驴走近了,小媳妇拔出腰间的盒子炮,娇声嫩气的笑道:``此山是我开,此树是我栽,你说我们是什么人?''

小柳治暗叫不好,知道自己是遇上了土匪;而无心却是盯着女匪看直了眼——小媳妇生得明眉大眼苹果脸,太漂亮了!

\chapter{耳光响亮}

除了赛维之外,其余三人都知道自己是遇上土匪了。

小柳治走上前去,坦然而又恭敬的开始讨价还价,金子纯站在一旁,则是不动声色的做好了拔枪准备。无心站在后方,因为看女匪看的太痴迷,竟然不由自主的张了嘴,是个要流口水的架势——女匪真美,粉扑扑的脸蛋,黑鸦鸦的头发,一身水灵灵的兴旺新鲜劲儿,看年纪,正介于大姑娘和小媳妇之间。一手拎着盒子炮,一手攥着根细鞭子,女匪是一把柔韧的小细腰,把小花袄上的碎花都要穿活了。腰细,胸脯可是鼓鼓囊囊的很饱满,仿佛里面揣了两只不安分的白兔子。

赛维是在几分钟后才反应过来的。她第一次看见土匪,还是个女的,就上一眼下一眼的细瞧不止。及至瞧够了,她斜过眼珠,忽然发现无心一脸痴相,看女匪都看直了眼睛。依着她的审美观,她也觉得女匪长得挺好,可远远没到惊艳的地步。换句话说,她再怎么好,不也就是个村姑么?

她静静的盯着无心,倒要看他能够色迷心窍到什么地步;而驴背上的女匪也留意到了无心的目光,黑白分明的大眼睛滴溜溜一转,她隔着小柳治抬头问道:``哎,那小子,你可瞅我半天了,是不是等我给你一鞭子呢?''

无心连忙低了头,低头之后还忍不住抿嘴一笑,因为心目中的大美人搭理他了。

赛维双手插兜,歪着脑袋看他,倒要看他能不要脸到什么地步。

在满洲国的地界上,日本人是很常见的,所以小柳治在确定女匪不是游击队之后,便半真半假的自报了家门——他说自己是个商人,因为有几位当官的朋友,所以搭乘军用飞机要往哈尔滨去。结果飞机半路出了故障,降落在了山上,他就带了几个年轻的伙伴,想要下山找人帮忙。如果女英雄肯高抬贵手放一条生路的话,他们必会重谢。

女匪虽然厉害,但毕竟只是个匪,并且还不是大匪。她方才也瞧见一架飞机低低的扎进了山后,但是不该管的她不敢管,只想劫几个钱过年。女匪既然识相,小柳治又一团和气的不讨人嫌,所以双方立刻达成了合作的关系。小柳治把身上仅有的钞票大洋全给了女匪,而女匪调转方向,要带着他们往山下走。

一路上,小柳治和女匪就没停过嘴。女匪有个颇不好听的名字,叫做赵半瓢,因为当初是山下老赵家用半瓢大米换回来的童养媳。贱名好养活,所以她就成了半瓢。二十岁那年,半瓢的男人被山上的土匪杀了,赵家老两口又急又痛,也跟着去了。赵半瓢成了孤身一人,竟然很有作为,不但给丈夫报了仇,还占住一座山头,也成了当地的一霸。

赵半瓢骑着毛驴,不紧不慢的往前走,该说就说该笑就笑,气概和男人也差不多。忽然向后回了头,她问无心:``咋的?你看上我啦?''

无心的确是看上她了,但是动眼睛,不动心思,只是``看''而已。

赵半瓢见他是个挺好看的小白脸子,就又逗了他一句:``看上姑奶奶了就直说,姑奶奶一高兴,招你当个小女婿!''

此言一出,众人都笑,无心低了头,也是笑,只有赛维不笑。赛维沉着一张脸,一边走一边紧盯着他。

走过几条山路之后,赵半瓢就勒住驴子不肯走了。居高临下的一指前方,她指着远处洼地中的一片房屋说道:``那边儿住的全是你们日本人。地方我给你带到了,说吧,你咋谢我?''

小柳治向她一鞠躬,身上一丝军人的犷悍气都没有,笑嘻嘻的只是温和。他把余下三人留在原地,自己一个人往山下跑。而赵半瓢处在等待的期间,无所事事,就回头对着无心一挥鞭子:``你过来。''

无心乖乖的走过去了。

赵半瓢稳稳当当的坐在驴背上,笑模笑样的问他:``你多大了?''

无心有点结巴:``二、二十多了。''

赵半瓢又问:``有媳妇了吗?''

无心这回在近处看清了她,发现她说笑之时,眼角已经有了隐隐的细纹,不过瑕不掩瑜,她将来便是真老了,大概也会风韵犹存:``没有。''

赵半瓢轻轻抽了他一鞭子,分明只是在拿他开心:``没媳妇就盯着我看啊?不怕我挖了你的狗眼?小白脸子,没好心眼子,你给我滚一边去!''

无心挨了骂,但是丝毫不生气。美滋滋的转身向后走,他偶然一抬头,忽然正对了赛维箭簇一般的目光。脸上的笑容登时僵住了,他竟然忘记了身边还跟着个赛维!

赛维面无表情的看着他,同时点了点头,是心如死灰而又恍然大悟的模样。

无心一步一步的向她靠近,仿佛是被吓着了,眼睛一眨不眨的望着她。

正当此时,小柳治回来了。

小柳治肩负重任,不想去惹一条没名没姓的小地头蛇。他把沉甸甸的一口袋现大洋献给赵半瓢,算是和女匪结下情谊。赵半瓢得了钱,别无所求,便要抄小路回山里去。小柳治也带着自己这支小队踏上了归程。

四人一路无话,回到飞机迫降之处。众人全站在飞机下面,而小柳治报告道:``我们所在的地方,是吉林省境内。山下有我们的村庄,村长已经派人去了最近的县城,不会等待很久,就能有人过来接应我们。''

众人松了口气,开始嘤嘤嗡嗡的互相交谈。而无心见赛维直挺挺的站在寒风中,就凑到她的面前,微微弯腰唤了一声:``赛维?''

话音落下,他就觉眼前一花,同时耳边响起一声炸雷。顺着力道一歪,他猝不及防的跌坐在地。屁股都结结实实的硌疼了,他才意识到自己刚被赛维抽了个大嘴巴!

他捂着脸,半边面颊火辣辣的麻木着,一时觉不出疼。周遭立时寂静,全被赛维的一巴掌震了住。胜伊快步走去搀起了无心,又对赛维嚷道:``姐,你干什么呀?''

赛维上前一步,一把推开了胜伊,然后质问无心:``知不知道我为什么生气?''

无心放下了手,半张脸通红的,显出五指痕迹:``你放心,我不是见异思迁的人。''

赛维本想一挥手,潇洒的将他臭骂一顿,并且让他滚蛋。可是话到嘴边,她忽然又不大敢,怕无心会真的滚——她才不允许无心滚去找女土匪,无心是她的!她不放手,谁敢来抢?

长长的叹了一口气,她收敛了杀气,决定以柔克刚:``我不强求你,你随便。反正我们之间也还没有什么约定,法律上面更是完全没有关系。你是自由的。''

无心拉着她的手,走到僻静处停住。颇为惭愧的笑了笑,他低声说道:``你相信我。我对你说过的话,我都记得,也都算数。方才我看赵半瓢,只是因为她好看,我没有别的心思。''

赛维仰脸凝视着他:``看也不行。''

无心微笑着答道:``那我以后再也不看了。''

他的话全是至真至诚。以后的确是不打算再看了,要看,也等赛维老死之后再看,如果赛维愿意和他共度一生的话。美人代代都有,而赛维只能活几十年,他不想让赛维在有限的生命里愤怒伤心。

赛维鼓舞着斗志,本打算和无心大战一场,不料他不战而降,直接竖了白旗。无心的承诺来的太容易了,让她不能彻底相信。但一味的闹也不是办法,赛维拧着两道眉毛看他,忽然感觉无所适从。

赛维和无心一前一后的进了机舱,找了座位并肩坐下。无心又去握赛维的手,赛维躲了一下,没躲开,也就不躲了。

无心攥着她的手,皮肤软,骨头硬,瘦得像个爪子。她不是无心心目中的美人,怎么看都不是,哪怕她搽了满脸的脂粉。但是无心决定好好的爱她,就像自己别无选择一样,去爱她。

赛维忽然开了口:``疼不疼?''

无心老老实实的答道:``疼。''

赛维不看他,望着窗外低声说道:``气疯我了。''

无心抬手去揽她的肩膀,没敢再说话。

傍晚时分,一队日本兵开进山里,用翻斗摩托运走了飞机里的所有人和物。临行之前,小柳治对带头的队长说道:``山里面有土匪。''

无心听了,心中一动,知道赵半瓢要遭殃了。但知道归知道,他没法子去给她通风报信。

长长一队翻斗摩托把他们从山中送进了县城。一夜的休整过后,他们把飞机和飞行员留到当地,然后改乘火车继续前行。不出一天的工夫,他们便当真到达了哈尔滨。而从哈尔滨再去齐齐哈尔,之间不过几百里地,自然十分容易。

抵达齐齐哈尔之后,队伍中的众人才正式做了自我介绍。富态的光头名叫香川武夫,一直无声无息的小女人名叫小桥惠。除了姓名之外,香川武夫再不肯多说自己的来历,所以众人各怀心事,很明显的分成了中日两派。

马老爷一路上都是不多言不多语,直到此刻才开了口,向小柳治问道:``接下来,我们往哪里去?''

小柳治没有回答,香川武夫说道:``我们在这里住上几天,等一等消息。''

马老爷立刻又问:``等什么消息?''

香川武夫沉吟了一下:``事关机密,现在还不是发表的时候。''

马老爷一晃卷毛脑袋,似笑非笑的答道:``香川先生,你和我讲机密,很可笑。显然你们认为在我和我的儿女的头脑里,还隐藏着不为人知的信息,所以才把我们强行带了来。''

香川武夫仿佛是很感兴趣,点头笑道:``那么马先生,我们的想法是否正确呢?''

马老爷满不在乎的答道:``抱歉,既然你们不肯坦诚,我也只好弄一点玄虚了。还好我家里有一位好姑爷——想必你已经听小柳先生提过了,我的姑爷,并不害怕宝藏的诅咒。''

然后他扭头对着身边的无心微微一笑,随即对着香川武夫继续说道:``到了非常之地,当然就要用非常之人。你说我的姑爷是听你的,还是听我的?''

香川武夫摸了摸自己的光头,紧接着一挑眉毛,压低声音答道:``自从对古鼎做过了初步的鉴定之后,军部就派人进入了兴安岭地区。经过了这些天的考察,我们已经对当地有了一定的了解,甚至也听说了曾经有一批汉人军队闯入密林,从地下挖出了受诅咒的宝藏。但是传说中的密林究竟在什么地方,我们就无法确定了。''

马老爷想了想,又问:``大概的范围呢?''

香川武夫答道:``从呼伦贝尔草原额尔古纳河流域到大兴安岭。''

马老爷颓然坐在一把硬木椅子上,怀疑自己是有来无回了。忽然抬头瞄向香川武夫,他又问道:``古鼎\ldots{}\ldots{}是真货?''

香川武夫点头答道:``商代的铜鼎。''

马老爷略一思索,却是紧跟着又问:``你们到底是对古董有兴趣,还是对诅咒有兴趣?''

香川武夫很意外的一扬眉毛,不回答了。

马老爷满嘴日本话,赛维等人听不大懂,事后再去询问,马老爷却闭紧了嘴,不肯多说,只在背地里对赛维嘱咐道:``你看紧了无心,他是我们的救命星。''

赛维糊涂着,还想宽慰父亲:``爸爸,真要是出了事情,我们找机会逃就是了。反正你不是很老,我们也不是很小,凭着两条腿,哪里走不到?''

马老爷揉搓着衣角,向窗子外面张望:``你看外面的卫兵,我们连这道房门,都走不出去啊!''

马老爷这话说出不过一天,这一支东拼西凑的小队伍就又启了程。

\chapter{地堡}

在一个寒风呼号的傍晚,小小的队伍逆风而上,一头冲进了极北的冬天。

他们依旧是打扮成闲人模样,身后又增添了一支日军小队作为保镖。从齐齐哈尔到了海拉尔,又从海拉尔进入了茫茫的草原山林,一直不显山不露水的金子纯骤然成了全队的向导,带着队伍穿林海过雪原,最后竟是进入了一处秘密的要塞之中。马家几人看在眼里,这才知道原来队伍里面卧虎藏龙,大概连一直不声不响的小桥惠,都是不能小觑的。

要塞所在之处,并没有一个明确的地名。金子纯依靠指南针行进在林子里,最后在山腰一丛荒草中找到一扇铁门。香川武夫手里拿着一份潦草地图,紧紧跟在后方。小柳治一手搀着马英豪,一手按在腰间枪上。马家的一群瘦子们倒是伶俐了,裹着大皮袄走得汗涔涔。

金子纯弯腰打开锁头掀开铁门,门下是一眼宽敞的竖井。回头望了众人一眼,他用中国话说道:``这个要塞是空的,进去之后跟紧了我,否则会迷路。''

然后他率先跳下竖井,井壁上开着一人多高的大洞,直通地下。他下去的痛快,旁人见状,自然也就不再犹豫,接二连三的全进了洞,无心照例是跟在赛维和胜伊身边。香川武夫和金子纯打开了随身携带的手电筒,光柱在洞内晃了一瞬,无心看得清楚,就见这洞高过两米,宽也过两米,十分的开阔。洞壁全由大石砌成,上方还嵌着电线电灯,只是此刻没有通电,灯是黑的。石壁上面用大箭头做了种种记号,又用油漆大大小小的刷出数字,不知是何用意。

马老爷,因为此刻人单势孤,所以生平第一次的爱起了儿女。一手领着马俊杰,他环顾四周,越是看得详细,脸色越是惨白。马俊杰半睁着眼睛跟他走,像是病了,然而又没有病,只是精神不振。十几岁的半大孩子,心里也都是有数的,他在马家其实本来只想自保——保住自己,再保住娘。可是娘如今停在医院里冷冻着,自己也莫名其妙的进了深山老林。

仿佛是为了让赛维姐弟也能听懂似的,马老爷难得的说了中国话:``这洞子里的设施也很齐备了,为什么空置着不用?''

小柳治自从下了飞机之后,似乎就失去了发言权。香川武夫答道:``据我们了解,这一片地区,对于本地原住民来讲,属于禁地。''

马老爷是懂得一点军事学的,所以在前方一处方方正正的炮座前停了脚步:``对于原住民来讲,这里是禁地;对于日本军队来讲,这里也是禁地吗?''

话音落下,他认为自己问住了香川武夫,所以回过了头,倒要看他如何作答。哪知香川武夫坦然的点头答道:``诚然,对于军队来讲,这里也是禁地。''

马老爷又转向了炮座,炮座前方是个方方正正的洞口,四周用水泥抹平加固,因为角度巧妙,所以从炮座望出去,视野极其开阔,能看到山下辽远的荒原。

赛维和胜伊也挤上去看,都很惊叹,没想到一个小小的四方口,竟然囊括了大大的风景。马英豪的右腿不得力,一边扶着小柳治靠墙休息,一边抬眼去看无心。无心和所有人一样,都裹着一件过分厚重的大皮袄。臃肿的站在黑暗处,他像个无声的影子,正在专注的往地道深处凝望。

马英豪甩开了小柳治的手,拄着手杖慢慢的走向了黑暗:``无心,看什么呢?''

无心看了他一眼,然后转向前方,轻声答道:``看鬼。''

马英豪盯着他的脸,认为他是在胡说八道:``好看吗?''

无心摇了摇头,随即对着虚空一招手:``小健,过来,你不知道鬼能吃鬼?''

赛维和胜伊听在耳中,不为所动,因为和小健也算是相识;马老爷没听懂,但是强忍着不问也不动,只有马俊杰打了个冷战,似乎是嗅到了一丝熟悉的阴寒气息。

小健笑眯眯的飘到了无心的后脖颈,大白天的,他有点感觉力不从心。

无心继续向前看,看见一个模模糊糊的影子站在遥不可及之处。

香川武夫显然很重视无心的话,特地转向他问道:``你有驱鬼的办法吗?''

无心摇了摇头,只答:``去找白琉璃,他有办法。''

可是白琉璃此刻还在后方——他始终是不能见光,所以一直呆在大木箱里,需要用马车把他拉进山里。

香川武夫扫视了众人的面孔,开诚布公的说道:``是的,偶尔会有人在这里看到鬼魂,为了稳定军心,军部让士兵撤离了这座要塞。但是对于我们来讲,这里是最完美的大本营。''

金子纯随即说道:``我们今晚将在指挥所休息,指挥所紧靠粮库,粮库里面的食物很充裕,我们即使留下过冬,都没有问题。''

此言一出,仿佛一句不祥的谶语一般,让在场所有人都变了脸色。没有人想留在这里,和幽灵一起过冬。

即便和他们相比,幽灵只是少数派。

沿着通道继续向前,一拐弯就上了主干道长廊。主干道更为高大宽阔了,两边是平坦的水泥墙壁,上方修成半圆形的拱顶。可是由于没有直通向外的枪眼,光线不足,反而比方才走过的岔道更为幽暗。金子纯在墙上摸到开关摁了一下,一声轻响过后,洞中漆黑依旧,可见电线全被掐断了。

一行人紧跟着金子纯,在几只手电筒的照耀下向前走。最后金子纯率先停住脚步,转身面对了一扇大铁门。掏出钥匙打开铁门,他一马当先的走了进去。只听``嗤''的一声,他划燃火柴,点亮了室内一盏煤油灯。

灯光一亮,众人立时就感觉出了轻松。指挥所是一间空空荡荡的大屋子,靠着角落摆了两张行军床,除此之外,再无其它。

众人经过了长途的跋涉,如今到了落脚处,就不由自主的全部席地而坐。无心又躲进了角落里,赛维和胜伊分别偎在他的两侧。小柳治则是和马英豪坐在了小床上。

香川武夫没有坐。对着手中的地图又看了看,他用中国话低声道:``山中的通古斯人说,自古以来所有邪恶的巫师,都会选择死在这座山上。他们认为这片山林蕴藏着一种不为人知的力量,可以让巫师的灵魂永生。''

然后他一挑眉毛:``听起来像是讲给小孩子听的故事,是不是?希望它是真的,否则军部在此之前的所有调查,就都成了无用功。''

马老爷抬手捂嘴咳嗽了一声,反问道:``难道是凭着我们几个人的力量,把整座山挖一遍?直到挖出另一半干尸为止?''

香川武夫的光头在高悬着的煤油灯下闪闪发光:``当然不是,明天我们还会有后续队伍赶来帮忙。现在我们要做的,就是设法过夜,等待天亮。''

指挥所隔壁就是粮库,粮库里面不但有大米,还有各种罐头以及干菜。小桥惠一言不发的点起一只煤油炉,用罐头和大米煮了一锅肉粥。崭新的铝制饭盒成了他们的饭碗,呼呼噜噜的喝了一气,晚饭也就算是对付过去了。

赛维放下饭盒,轻轻一扯无心的袖子,低声说道:``你和我出去一趟,我\ldots{}\ldots{}我内急。''

胜伊听见了,也凑近了说道:``我也是,都憋了半天了。别人不出去,我也不敢出去,外面多黑啊!''

无心一挺身站起来,要护送二人出去方便。地堡之内的水电都被切断了,所以想要方便倒也容易,无须特地去找卫生间,随便寻觅个僻静地方就可以。

三人出了指挥所,在一处角落里停下了。无心背对了他们,就听姐弟二人互相隔了两三米远,各自都在窸窸窣窣的宽衣解带。温暖的尿骚味隐隐的弥漫开了,胜伊忽然``哎哟''一声:``真糟糕,尿到鞋上了,好恶心呀!''

赛维没言语,只感觉屁股冻得冰凉。尿净了之后站起身,她一边飞快的系腰带,一边横挪了一步,想要避开自己的尿。末了把皮袄下摆往下一放,她正要迈步向前,不料一条腿抬起来,却是脚踝一紧,拖拖拽拽的有了分量。

她一哆嗦,连忙低头去看。借着远方指挥所门口散发出的灯光,她清晰的看到了一只手——枯瘦的手,手指蜷曲,松松的合在了她的小腿上!

她气息一颤,没有尖叫,只带着哭腔低声唤道:``无心,无心,有手抓我!''

无心连忙转身弯腰去看,随即上前一脚踩住枯手的腕子,同时急道:``你走,快走!''

赛维奋力拔腿,因为脚上是一双长筒皮靴,所以倒还没有掉鞋的危险。强行挣脱了枯手的束缚,她扶着胜伊回身一瞧,登时吓白了脸——原来她的屁股后头,居然躺着一具日本兵的尸首!

尸首不知是因为干燥脱水,还是生前就很消瘦,此刻看起来宛如枝枝杈杈的一捆干柴。赛维方才一脚踩进了他的手中,倒不是他蓄意的吓人。尸首完整,身上的衣服也不算坏,甚至能有七八成新。

``无心\ldots{}\ldots{}''赛维用耳语般的轻声说道:``要不然\ldots{}\ldots{}我们到洞外去露营吧。''

无心退到了他们身边:``外面太冷,而且夜里也许会有大野兽。和野兽相比,还是鬼比较容易对付。''

正当此时,洞中远处响起了一串脚步声音,是整整齐齐的开步走。三人都没想到荒废的地堡中竟然会有军队走来,不禁一起觅声瞪大了眼睛张望。结果指挥所门前闪现出了臃肿人影,还真是小小的一队日本兵——傍晚护送他们进山的,自从他们入洞之后,日本兵就留在洞外,一直没有动静。

领头的一名士兵进了指挥所,片刻之后又出来了,带着一队日本兵返回岔道,并没有再出洞的意思,显然是打算在距离地面最近的地方过夜。而无心对着赛维和胜伊做了个噤声的手势,然后带着他们回了指挥所。

他们进门之时,香川武夫手中又多了一张新地图。抬头看了无心一眼,他接着方才的话头继续说道:``本地的人,死后全是采取风葬,而死在此地的巫师,因为不愿升天,所以会在风葬之处,把自己埋进土里。风葬,需要四棵大树作为支柱,上面用树枝架出平面,放置尸体。巫师死于地下,可是地上的工作,他不会省略的。很好,我们的小队刚才在附近搜索过了,类似风葬的痕迹,找到了三处。等到天亮,我们就逐一的去看一看。''

马老爷不阴不阳的说了一句:``我家里那具尸首,可是几十年前死的,就算有人为他余下的半具尸首举行了风葬,难道如今还看得出痕迹吗?''

香川武夫针锋相对的答道:``看不出,所以需要寻找!''

\chapter{征兆}

无心告诉香川武夫,说是外面不远处的拐角里躺着一具士兵干尸,看他一身单薄军装,应该死于温暖季节。

然后他就回到角落坐下,左拥右抱的搂住赛维和胜伊,半闭了眼睛想要睡觉。马老爷因为年纪大,所以占据了一张小床,听说外面有尸体,他纹丝不动的向下一躺,是个心如死灰的模样。

小桥惠蹲在墙边,点起了一只小小的洋炉子,铁皮烟囱贴着墙角向上走,一直通入换气孔。马英豪和小柳治也自找地方蜷缩着坐了,香川武夫则是占据了另一张床。

金子纯握着手电筒出去走了一圈,片刻过后回来了,用日本话咕哝了一句。不等香川武夫回答,躺在床上的马老爷忽然开了口:``什么?尸体的血液被抽干了?''

指挥所内的大部分人都通日本话,马老爷的反问,显然是问给赛维等人听的。无心刚刚解开了皮袄中间的几个纽扣,让赛维和胜伊把手伸到自己怀里取暖,听了马老爷的话,他没有回应,只往大皮袄里又缩了缩。

香川武夫被马老爷的尖锐嗓门吓了一跳,无言的回头看了他一眼,香川武夫点了点头,没再多说。而金子纯很仔细的锁好铁门,然后便也在洋炉子旁躺下了。

室内一片安静,连飘在屋角的小健都是一动不动。赛维和胜伊的手好像两片薄薄的叶子,隔着一层衬衫贴在无心的胸腹之间。赛维心安理得的闭上眼睛,想要摸摸他,可是又不好意思;胜伊窝在他的腋下,也感觉他很温暖洁净。

胜伊和赛维是在娘胎里挤着抱着长成人形的,他们分享一切,是天生的联盟,活到十八九了,两人之间还连着一条无形的脐带,互通有无。胜伊知道自己是弱一点,所以格外依赖强一点的赛维。不是他看得上通得过的人,他不会允许赛维去爱的。即便赛维用瘦削坚硬的拳头敲他捶他,他也不妥协。

他讨厌男人,喜爱女人,可女人们又都不喜爱他,所以他的伴侣只有赛维。无心是个男人中的例外,他和无心在一张床上睡觉,偶尔手臂碰了手臂,赤脚碰了赤脚,居然并不感到恶心。除此之外,他认为无心的确是长得挺俊,眼睛黑得像夜,眼中的光亮得像星。他的好相貌和好脾气,都让胜伊像爱赛维一样的爱他。

胜伊抬眼看了看无心,又在无心的皮袄中去捉赛维的手。姐弟二人的手一模一样,连尺寸都完全相同。赛维也仰脸看了看无心,然后仿佛很开心似的,像个顽童一样在胜伊指尖弹了一下。

无心依靠在墙壁上,已经闭了眼睛。煤油灯的光芒有限,并且偶尔跳动。他的一双眼睛陷在阴影之中,阴影很黑,他乍一看好像没了眼珠,只剩轮廓分明的两只眼窝。

一夜过后,小桥惠像只活闹钟,把室内众人全部叫醒,并且提前用大米和罐头煮了一锅饭。米饭比昨晚要干,结结实实的盛进大饭盒里。赛维和胜伊都很想刷刷牙齿,可是条件不大允许,所以他们只漱了漱口,又把牙刷伸到嘴里乱掏了掏。

香川武夫和马老爷谈起了当年旧事。马老爷翘着小手指捏着大勺子,慢条斯理的把自己的爹臭骂了一顿,最后做了总结陈词:``香川先生,我是知无不言、言无不尽。老挨刀的当初只说花园山下埋着宝贝,应该是价值连城,然而动不得,是有毒的肥肉烫手的山芋。扔了,可惜;不扔,又是瞪眼干看。''

他尖着嘴巴,吃了一口热气腾腾的米饭:``宝贝到底是从哪里挖出来的,老挨刀的自己都说不清楚。反正就是好一顿打仗,几乎杀光了一个部落,才把宝贝抢到手的。''

胜伊不敢往小床的方向去看,因为感觉马老爷吃相猥琐,马俊杰神情痴呆,马英豪更是不堪入目,并且有个阴险的鹰钩鼻子。至于几个日本男人,统一的全是马马虎虎,完全不值一提。蹲在地上对着赛维,姐弟二人闷头大嚼。粗糙的食物和浓烈的香气很富有刺激性,他们生平第一次狼吞虎咽,不假思索的吃了大半饭盒的肉和饭。

吃饱喝足之后,门外起了响动。金子纯打开房门向外张望,就见一群士兵拖拽着一只大木箱走出了岔道。回头对小柳治做了个手势,小柳治连忙带着马英豪走出去,指挥士兵把木箱往远处送。无心侧耳倾听,能够听到锁头拍打木箱的声音。钥匙插进锁眼中转动了,转动之后又转动了,箱盖开启了,最后是一阵微不可闻的铃铛声。

无心很不理解为什么马英豪如此信任白琉璃。白琉璃是不通人情世故的,很容易受骗,也很容易骗人,像一个赤诚无邪的魔鬼。

白琉璃并没有出现在人前,马英豪像放生一样打开了木箱,随他自由行动。反正地堡永远都是黑暗,正适合他濒临失明的蓝眼睛。

指挥所内的众人又喝了一些热水,感觉精神都很振奋了,便络绎返回最近的岔道。攀着铁梯向上爬出竖井,他们见了天日。虽然目前还算秋季,但是山林中的空气已经完全是冬天式的干冷。一大群人分散开来又拉又尿,提起裤子之后都是龇牙咧嘴,因为屁股全被冻成冰凉。金子纯经验丰富的谈笑风生,讲述一名日本士兵去年冬天在山里撒过尿后忘系裤扣,结果冻得鸡·巴坏死。香川武夫立刻摆了摆手,一派温和的笑道:``当着马小姐的面,不要胡说。''

赛维冷着脸,装没听见;不过队伍的气氛的确是升了温度,香川武夫拄着一根手杖向前走,口中说道:``我们还是来得太匆忙了,应该再带一两条好猎犬才对!''

小柳治毫不掩饰的说道:``可以去最近的据点借几条狼狗嘛!''

金子纯连连摇头:``不行,地下暗堡的道路已经被封锁了,想要到下一个据点,就得翻山路,太辛苦。''

香川武夫用牙齿咬住手套一晃脑袋,拽下手套光了右手。摸出地图又看了看,他向前一指,兴高采烈的说道:``哈!很近嘛,已经到了。''

众人望向前方,就见疏疏落落的树木之中,有四棵笔直的白桦特别醒目。如果把它们看成是四个点,那么画出线条就是个规规矩矩的正方形。四棵白桦树间横竖搭了几根枯枝败叶,正是一处风葬的遗迹。

香川武夫带上手套一挥手,身后的士兵立刻握着铁铲上前,先把上方横七竖八的枝叶拨开了,然后便弯了腰开始挖地。天虽然冷,但是土壤还没有真正上冻;士兵们训练有素的挖了一阵,挖出一坑新鲜潮湿的黑土。

因为坑中除了土再无其它,所以士兵不停,继续深挖。金子纯忽然叫了一声,向前跳进坑里,弯腰向坑底细瞧,随即直起身说道:``看,怎么会有个洞?''

他不说,旁人没有留意,包括士兵;他说了,所有人仔细一瞧,发现土中果然有个细小的洞眼。金子纯随手捡了一根树枝,往洞内插,插进两寸就插不进了,不知是到了底,还是拐了弯。

金子纯从士兵手中夺过铁铲,亲自去挖。几铲子下去,他停了动作,抬头去看香川武夫——洞眼是拐了弯!

香川武夫沉吟着答道:``也许是蛇钻洞冬眠。''

马家众人都是四体不勤、五谷不分,所以认为香川武夫的话有道理,只有金子纯做出了反驳:``可现在还没到冬眠的季节。''

香川武夫话一出口,也感觉不合科学。不过此地偏北,时令早于其它地方,即便有蛇秋眠,也不稀奇。

金子纯见香川武夫不能回答,便跳上地面,命令士兵继续挖。如此又向下挖了半米多深,一名士兵发出惊呼,是铲子从土中掘出了一只蜡黄的人脚。

顺着人脚清理泥土,士兵们从土中刨出了一具不着寸缕的干尸。泥土湿润,先前又不寒冷,尸体不腐烂已经是罕见,无论如何不该脱水。几把铲子把干尸抬上地面,士兵正要往上爬,香川武夫却是大喝一声,吓得所有人都一抖。

原来在尸体身下的地面上,赫然又点缀了几只小小洞眼。洞眼还没有铲子的木柄粗,清清楚楚的不知在干尸身下藏了多久。

香川武夫望着洞眼愣了一阵,随后转向无心问道:``你\ldots{}\ldots{}知道它的由来吗?''

无心实话实说:``我不知道。''

然后他后退了一步,向一名士兵伸手要了铲子。铲子是好钢铲,锋利如刀。他走到干尸之前,双手攥了铲子向下狠狠一斩。第一铲子铲掉了干尸的下巴,第二铲子,他直接铲断了干尸的脖子。残缺不全的头颅在地上滚了一圈,旁人看得清楚,头颅里面是空的!

空,但又不是完全的空,因为还存留着丝丝缕缕的筋脉,干尸失去的纯粹只是脑浆和鲜血。无心几铲子又斩开了他的身体,五脏六腑也都在,只是已经干结坚硬。

香川武夫摇了头:``不对\ldots{}\ldots{}''

的确是不对,本地的原住民,没有把尸体处理成干尸的习俗,即便死者是个罪大恶极的坏巫师,也没有。

对着士兵一挥手,香川武夫下了令:``继续挖!''

继续挖掘的结果,就是没有结果。

细小的洞眼弯弯曲曲,挖着挖着就失了踪迹,但是人人都看出细洞深不可测。深不可测有多深?再往下可就是地堡了!

望着地上分成几段的干尸,马老爷开了腔:``昨夜不是说地堡里也出现了一具干尸?彼干尸与此干尸,可有相似之处?''

马英豪听了父亲的言谈,厌恶到了头皮发麻的地步,同时又有些痛快,因为自己正在报仇。

香川武夫知道山中地堡从动工到完成,一直很不太平,及至军队进驻了,又隔三差五发生离奇事件,并且时常有人失踪,所以最后队伍才做了撤退。但要问彼干尸与此干尸有何关系,可是真没人知道,而且最好没关系,有关系才叫糟糕。

不置可否的沉默片刻,他把地图又展开看了一遍,然后一挥手:``走,我们去下一处!''

下一处,是个错误,因为地下要什么没什么,是士兵看走了眼。

赶在中午之前,他们抵达了第三处,然后又挖出了一具空壳子干尸。

悻悻的转向地堡方向,他们一无所获的想要返回。马老爷趁人不备落了后,一把将赛维拽到了身边,压低声音说道:``找机会就逃!''

赛维向马老爷歪了脑袋:``爸爸,你骗了他们?''

马老爷轻声耳语:``地堡的位置属于军事机密,不是我们应该知道的。他们之所以不防备我们,是因为\ldots{}\ldots{}我们是必死的人了。''

赛维的脑筋一转,恍然大悟,于是微微的一点头。

\chapter{吮吸}

赛维听了马老爷的话之后,心里什么都明白了。日本人,包括马英豪,并不相信马老爷对诅咒一无所知,所以要把他、以及和他最亲近的儿女一并带来塞北,事到临头了,不信马老爷不吐真相。

但是赛维自己考量着,感觉父亲好像真的是再无保留。马家祖辈既没出过神棍,也没出过圣人。指望着爷爷全知全能,实在不大现实。

一行人回到地堡入口处,金子纯下洞运了炊具和食品上来。小桥惠一言不发,又开始娴熟的生火煮饭。众人各自喝了一些烧开的雪水,在等待饭熟的空当里,赛维忽然说道:``无心,你陪我和胜伊去一下。''

去哪里,去干什么,她都没有明说。胜伊一怔,随即放下饭盒站起了身。无心则是完全的默然。三个人走向附近的一处小山坳,正是个要找地方解手的样子,于是其余几人不再关注,自顾自的继续喝热水。

在一棵老树后面,赛维悄声转述了马老爷方才说过的话。说过之后又命令无心背过身去,当真和胜伊在老树两边分别撒了一泡尿。无心望着山腰处的众人,开口说道:``白天想逃,大概是不容易。夜里地堡太黑,一旦有光又会惊动人,也不好走。今天你们先不要急,天黑之后我出去探一探路。地堡绝对不会只有一处入口,一旦找到新路了,我们就找机会逃。''

赛维蹲在老树的斜后方,仰着头去看他的后脑勺。山上的风又干又冷,触目之处都是衰草枯杨,对比之下,他雪白的皮肤和漆黑的头发就显得异常鲜嫩,然而又不是阳光雨露滋养出的鲜嫩,而是长久不见天日,在暗处沤出来的鲜嫩。

她飞快的提了裤子站起来,一边笨拙的搂起皮袄系腰带,一边心想:``他的头发还是不见长。''

随即另一个疑问也生了出来:``怎么没见他剪过指甲?''

赛维走上前去,拉起他的手看了看,怀疑他暗藏了很不卫生的生活习惯。然而他的指甲看起来整洁规矩,并没有被牙齿啃过的痕迹。

下午,香川武夫亲自带兵出发,其余人等则是回到地堡,烤着火炉养精蓄锐。马老爷能吃能喝,吃饱喝足之后就挺尸似的往床上一躺,不言不动。马俊杰席地而坐靠着床腿,迷迷糊糊的也是睡。马英豪和小柳治坐在火炉旁边,用日本话低低的交谈,谈着谈着,忽然哈哈的笑了,一边笑一边又看了无心一眼。小柳治留意到了他的目光,当即一拍他的右腿:``为什么总是看他?''

马英豪收回目光,垂下眼帘笑道:``他多有趣。''

小柳治一皱眉头,出于对好朋友的关心,决定回到天津之后,立刻逼着他和佩华同居。不甚自在的清了清喉咙,他换了话题说道:``白琉璃不见了。''

话音落下,房内忽然静了一瞬,遥远处依稀响起了似有似无的铃铛声音。马英豪向半开的门口张望一眼,门外人影一闪,他怀疑自己看到了一个血迹斑斑的小影子。

然后,他发现在一转眼的工夫里,无心竟然也消失了。而赛维和胜伊很安然的互相依偎,并不惊讶。

``胜伊!''他开口唤道:``无心呢?''

胜伊懒洋洋的答道:``撒尿去了。''

无心走在主干道走廊中。走廊一片黑暗,是真正的伸手不见五指。小健飘在他的肩膀上,轻声说话:``大哥哥,我有点怕。''

无心闭着眼睛,走得很快:``要不要我把你封住?''

小健想了想:``封住我也可以,不过你要把纸符贴到胸口。你说过我是凉的,我凉着你,你将来就不会忘记放我出来了。''

无心从怀里摸出一张裁好的小纸条,以及一根短短的铅笔头。扭头看了小健一眼,他郑重其事的说道:``放心,我忘不了你。''

然后他跪在地上,撅着屁股开始画符,同时听到小健嘱咐自己:``别让马俊杰死,他死了,我就找不到新身体了。''

无心猛一挥手,让纸符像刀一样平平的掠过了小健的咽喉。小健的幻象瞬间消失了,无心站起身,一边把纸符往怀里揣,一边视而不见的经过了两名日本兵。士兵也只是幻象,他们早已死在了地堡之中,因为不是好死,所以灵魂不散,总不甘心。虎视眈眈的盯着无心,他们却是没有动。

无心继续往前走,知道日本鬼畏缩的原因。地堡之中鬼比人多,而人能吃人,鬼也能吃鬼。小健都怕了,何况凡鬼?

他继续往前走,耳朵毫无预兆的一动,他听到了极其细微的摩擦声音,类似一条小蛇游过坚硬地面。

缓缓的俯下身去,他认为小蛇并没有远离。走兽一样四脚着地了,他正要静静寻觅小蛇的行踪,不料空中忽然响起了沉闷的鼓声——``砰''的一下,类似心跳。

一声鼓响之后,小蛇的行踪凭空消失。无心抬头怒道:``白琉璃,别捣乱!''

主干道上并没有白琉璃的影子,可不知他在何处长出了一口气,四面八方都是他的叹息:``我是救你。''

无心依然趴伏在地上,语气稍微和缓了些:``多此一举。我只想知道刚才经过的是什么东西。''

再没有回应了,白琉璃比鬼魂更像鬼魂。在无心的眼中,鬼魂还有行迹;但白琉璃神出鬼没,黑暗洞窟成了他的乐园,无心是真的找不到他。

无心走到了主干道的尽头,摸到了两扇紧锁着的高大铁门。铁门之后必定还有通道,也许是通往其它据点。此地的山底已经被日本军队挖空了,所有要塞的枪炮都在提防着苏联军队的进攻。

无心无可奈何,转身踏上返程,顺便又走了几条岔道。走着走着他不敢走了,因为地堡道路十分复杂,如果没有地图的话,必定迷路。

一无所获的返回了指挥所,他发现香川武夫还未回来。而金子纯脱了大皮袄,挽着袖子要去隔壁粮库找些零食打发时光。手里端着一只大饭盒,他对着室内众人笑道:``库里至少会有松子和榛子,如果牙齿够结实的话,就有的吃了!''

然后他推门向外走去,一步迈进走廊,他忽然低头``咦?''了一声,然后弯腰去看:``什么?蛇?''

小柳治听说外面有蛇,便起身要找件趁手的兵器去打蛇。然而还未等他抄起马老爷的手杖,外面``咣啷''一声饭盒落地,同时响起了金子纯的惨叫。通过大开着的房门,众人看得清清楚楚,只见一条一尺多长的黑色小蛇猛窜向上,一口咬住了金子纯的手腕。蛇身随即卷住猎物的小臂,一环一环的勒紧收缩。而金子纯的手臂僵直在了半空,原本是筋肉虬结的,此刻却迅速枯萎,仿佛皮肉鲜血化为一体,全被黑蛇吮吸了去,空余一层皮肤贴上骨骼。

小柳治愣在当地,握着手杖忘记上前。金子纯侧脸紧盯着自己左臂上的黑蛇,也像被魇住了似的,瞪着眼睛一动不动。眼看他粗壮的手臂从腕子开始一直枯萎向上,门口忽然闪过一道寒光,却是小桥惠拔出一把长刀,狠狠劈下了金子纯整条胳膊!

黑蛇吸了足够的血肉,身体饱满的肿胀了。``啪嗒''一声随着手臂落地,它在第二刀落下之前,倏忽间消失在了黑暗中。刀锋的寒气掠过金子纯的鼻端,让他如梦初醒似的回过了神。难以置信的张大了嘴,他从喉咙中发出颤抖凄惨的尖叫。断掉的手臂还在地上一抽一抽,一刹那间,他的半边身体已被汹涌的鲜血浸透。

所有人都傻了眼,只有小桥惠不慌不乱的打开随身携带的行军背包,往金子纯的创口上泼撒止痛药粉。金子纯左肩被劈下了小半,黄白色的药粉落在鲜红淋漓的血肉上,瞬间融化消失。伤势严重到了不可收拾的地步,他静静的侧躺在地上,不再叫了,因为已经疼得失去了知觉。

无心看着金子纯的惨状,心中悚然,忽然又联想起了干尸身下的细洞,他也明白了干尸的由来。

问题是,山上到底有多少黑蛇?如果只是零星几条,或许不足为惧;如果是成千上万——不,不会成千上万,如果真的很多,不会从来没有人提及它。

无心并不清楚黑蛇的习性,所以在小桥惠和小柳治把金子纯拽进室内之后,便出去清理了门前的粘稠血泊,免得血腥气会引来更多活物。

指挥所内,小柳治注视着奄奄一息的金子纯。片刻过后,他开口说道:``金子是我们的向导,如果没有了他,我们也许真的会在山里过冬——除非赶在第一场大雪之前,立刻出山!''

然后他转向了马英豪:``白琉璃在哪里?养兵千日,用兵一时。我们可不是带他来玩的!''

马英豪没说话,因为不知道白琉璃到底在哪里。对于白琉璃,他只能确定对方不会伤害自己,仅此而已。

无心蹲在门口,心无旁骛的用草纸擦血。擦着擦着,他抬起了头。

他看到在前方的岔路口拐角处,一条大蛇缓缓游过,蛇身足有水缸粗细,滑腻腻的反射了微弱灯光。

一把丢下手中草纸,他先用力关闭了指挥所房门,然后大踏步的走进了黑暗。

他也不是黑蛇的对手,他得去找白琉璃。

\chapter{狗咬狗}

无心发现自己想要在迷宫一般地堡中找到白琉璃,简直是不可能的。

白琉璃仿佛天生就是混在沼泽之中的幽灵,而此地幽暗深邃,正是堪称他的天堂。在距离指挥所最近的岔路口停住脚步,他脱下外层沉重的皮袄皮靴,露出里面一身柔软的短打扮。袜子也脱下来团成一团塞进了靴筒里,他赤脚踏上冰冷的水泥地面。轻轻巧巧的走了几步,他很满意,因为真正做到了无声无息。

然后像个猎人似的,他开始去寻找白琉璃。

无心跪伏在地上,贴着墙根前行,每过一段路途就抽抽鼻子,通过空气的成分来追踪白琉璃的行迹。白琉璃身上的臭,是有来历有渊源的,臭得不怀好意,和一堆大粪有着本质区别。无心记得自己当初和他相识之时,他还不是如此的恶劣。当时的白琉璃颇有人样,一头沉重的长发结成无数细辫。细辫子上涂了油脂,用嵌着宝石的带子扎成一束。油脂的气味很复杂,让无心联想起要腐不腐的生肉,以至于无心在饥饿的时候恨不能捧着他的脑袋啃一口;可在吃饱喝足之后,又往往会被他的气味熏到反胃。

无心闭了眼睛,十根手指和十根脚趾都要无限度的延伸了,在一切可借力之处借力,而发出的声音并不会响过一条小小的黑蛇。一个年轻的日本鬼站在甬道另一侧,笑眯眯的向前方做了个手势,随即瞬间飘远。前方有一团绚烂的光影闪烁,然而阴气森森,也许是某位死于此处的老巫师显灵了。

无心半走半爬,依靠着直觉选择方向,最后在一处岔道口前忽然放缓了速度。姿态柔软的拐了个直角,他睁开眼睛,感觉到了白琉璃的存在。

白琉璃依然是累赘臃肿的一大堆,也许是跪在地上,也许是坐在地上,后背对着水泥墙壁。上半身伏下去,他额头触地,双手交握着缩在怀里。忽然察觉到了无心的到来,他姿势不变,只叹息了一声。

无心直起了身,走兽一样蹲到了他的身边。双手垂下去抓住脚趾头,他的身体已经冻透了。受冻的滋味很不好受,他颤抖着发出声音:``白琉璃?''

白琉璃一动不动,只是唉声叹气:``呼\ldots{}\ldots{}''

无心深深的弯下了腰,歪着脑袋想要去看他的侧影:``你到了地堡之后,有没有见过黑色的小蛇?''

白琉璃本来就已经是半瞎,所以地堡内的黑暗很趁他的心意:``呼\ldots{}\ldots{}''

无心打了一个冷战,随口又道:``我不知道它到底是不是蛇,看着像蛇,可是它吸血。如果你遇到了,千万别让它靠近你,它不是一般的毒蛇,记住了吗?''

白琉璃微微偏过了脸,如梦方醒似的呻吟了一声:``嗯?''

无心抓着自己冰凉的脚趾头,自顾自的继续说道:``你和马英豪不是一伙的吗?他们现在被黑蛇困在指挥所里了,并且有个日本人已经受了重伤。你过去瞧瞧吧,看看有没有办法驱蛇?''

白琉璃缓缓的半直了腰,冷不防的问道:``你冷吗?''

无心恨不能把他拖回指挥所,但是又不肯轻易的得罪了他:``当然冷,我怕你逃,所以光着脚找来的!''

白琉璃慢吞吞的抬起一只手,拉扯身上层层叠叠的兽皮:``给你一件\ldots{}\ldots{}''

未等他把话说完,无心已经把脑袋摇出了风声:``不不不,我不怕冷!''

白琉璃登时停了动作,沉声问道:``为什么不要?''

无心想了一想,决定还是实话实说:``白琉璃,你太脏了。''

白琉璃沉默片刻,然后又问:``你嫌我?''

无心在纯粹的黑暗中迟疑着点头:``是\ldots{}\ldots{}''

下一秒,他张着嘴一怔,口中忽然多了活物。活物粗糙柔软,活泼泼的在他舌头上摇摆扭曲,是一条腥臭的、连蛊虫都能杀死的毒虫!

气急败坏的对准了白琉璃,无心``呸''的一声,把毒虫直啐到了他的脸上。随即伸出舌头呕了一声,他不给白琉璃机会,接二连三的把对方啐了个满脸花。白琉璃在污秽长发的掩护下,发出了低沉沙哑的冷笑:``骗子,请继续说!''

无心此刻的痛苦,甚于吃了大粪。左手伸出去撩开白琉璃的一侧头发,他扬起右手,结结实实的扇了对方一个嘴巴。白琉璃被他打得身子一歪,随即连滚带爬的重新坐正了,一只手同时不着痕迹的拂过地面。而在白琉璃抬手的刹那间,无心一屁股坐下去,痛叫着抬起了一只脚。一条蜈蚣死死的附在了他的脚背上,两排尖锐的虫足竟然一起扎进了他的皮肉中。

一脚蹬上白琉璃的下巴,他随即就地滚出老远,伸手去拔脚背上的蜈蚣。鲜血星星点点的渗出了,蜈蚣仿佛是怕他的血,自动的想要爬开,可是被他捏起来揪住两端,当场扯成了两截。

无心素来怕疼,所以如今不得不效仿了白琉璃,捧着伤脚唉声叹气。白琉璃托着下巴``呼\ldots{}\ldots{}''的出气;他也跟着张了嘴:``呼\ldots{}\ldots{}''

此起彼伏的叹了良久,无心熬过了疼,便又爬回了白琉璃面前,问道:``还疼吗?''

白琉璃低低的咳嗽了两声,像只小风箱似的喘道:``不疼了\ldots{}\ldots{}''

无心被蜈蚣咬过之后,对待白琉璃恭敬了许多:``既然不疼了,我们就走吧!''

白琉璃伸手摸上了他的脚背,摸到自家蜈蚣留下的两排清晰足迹,心中痛快了不少。收回手垂下头,他轻声说道:``你先走。''

无心怕他再放虫子咬人,所以分外有礼:``也好。我知道你有办法认路,路上小心,别走丢了。''

然后他站起身,乖乖的又道:``我走了,回头见。''

无心踏上归途,沿着甬道中央大步快跑,同时决定一分钱也不给白琉璃。白琉璃是个坏人,欺负白琉璃不算作恶。他难得欺负谁,因为无论谁都只能活几十年,让他不忍心去欺负。偶尔破一次戒,他别有一种快感。

找到自己的皮袄皮靴穿了上,他归心似箭的回了指挥所。敲开房门进了去,他发现室内加了一盏煤油灯,光明可以抵得上一只大电灯泡。金子纯的身上缠满了绷带,又包了一层粗帆布。帆布表面透出斑斑血迹,看起来比伤口本身更加恐怖。直挺挺的仰卧在一张小床上,他奄奄一息,嘴唇和面颊是统一的灰白了。

黑蛇有没有毒,已经无须去考据;单是大量的失血,便足以要了他的性命。他和与他分离的伤臂一样,都呈现出了枯萎之态。

房内的两个日本人,小柳治和小桥惠,都冷着面孔站在床边。赛维和胜伊缩在角落里看不清脸;马俊杰独自靠墙站着,被前方的马老爷挡住身影。马英豪拄着手杖站在中央地上,见无心回来了,当即开口问道:``你跑去了哪里?''

无心答道:``我去找了白琉璃。''

马英豪向他逼近了一步:``找到了吗?''

无心点了点头:``他说他随后就到。''

马英豪微微皱起了两道浓眉:``随后就到?你明知道他几乎不能走路,为什么不把他背回来抱回来?''

无心冷淡的摇头:``要去你去,我不去。''

马英豪发现自己是招惹了两个冤家,白琉璃已经是不听话,无心更是会咬人。一言不发的咬了咬牙,他想自己连路都走不利索,怎有能力搬运白琉璃?

正当此时,小桥惠低低的说了一句日本话,无心虽然听不懂,但是能够猜出意思——金子纯怕是要不成了。

门外依稀响起了脚步声音,是一大队翻毛皮鞋在水泥地上齐步走,显然是香川武夫回来了,然而人数又不对,因为进山为他们做保镖的,只是一支十几人的士兵小队。

谁也不敢开门去看个究竟,因为不知道门外角落里会不会埋伏着黑蛇。无心想起自己一眼瞥见的巨蛇,不知道该不该说。说了,也许会把日本人吓出山去,但是事情未完,他们不会善罢甘休,自己也未必有机会脱身;不说,又怕继续留在地堡中,赛维和胜伊会有危险。

外面的脚步声音从门前经过,不知是要往何处去。小柳治犹豫了一下,还是上前打开房门伸出了头:``香川先生,地堡里有毒蛇,请一定小心。''

香川武夫的光头在走廊里亮了一下:``唔,毒蛇?''

与此同时,小柳治看清了香川武夫身边的人员,的确是增添了至少十名士兵。其中几人抬着一只长长的木箱,不知里面装的是什么。

香川武夫说完``毒蛇''二字之后,便继续向前走去。整条队伍没入黑暗,很快不见了踪影。

小柳治没想到他会是个麻木不仁的态度,不禁愣了愣,随即缩回房内。如此又过了良久,白琉璃不见踪影,香川武夫则是返回来了。

他抽着鼻子进入指挥所,进门的时候还在自言自语:``是山外的人给我们送来了一些子弹。''

然后他抬眼看清了床上的金子纯。脸色骤然一变,他把目光转向了小桥惠。小桥惠小小的站在床边,不带感情的描述了不久前的一切——从金子纯想去粮库弄点干果当零食说起。

金子纯躺在床上,呼吸已经微弱到将近消失。香川武夫走到床前,居高临下的试探了他的鼻息。

俯下身摸了摸金子纯的头发,香川武夫凑到他的耳边,低低的、温柔的耳语了几句。金子纯紧闭双眼,一滴泪水流过了他的眼角。

然后香川武夫拦腰抱起了他,转身走出了指挥所。

几分钟后,上方遥遥的起了一声枪响。赛维和胜伊,包括马俊杰,一起打了个哆嗦,知道香川武夫已经枪毙了金子纯。不是因为金子纯犯了错误,而是因为金子纯是确定的不能活,所以同伴要用子弹结束他的痛苦,送他快走一程。

地上的香川武夫在收起手枪之后,抄起士兵丢在外面的铁铲,在地堡入口附近挖了一个深坑,埋葬了金子纯。天上飘着细碎的小雪花,凭着他的经验判断,小雪很快就会转为一场大雪。大雪封山,队伍中又失去了向导。将来的日子一定不好过,但是又不能空着双手打退堂鼓。刚刚进山才几天?现在下山往回返,于情于理都说不过去,没有办法向军部交待的。

香川武夫一手拄着铁铲,一手叉着腰,站在半山腰俯视山下,想要找个地方搭帐篷露营。队伍里没了金子纯,做什么都有点不大安心。雪花落在他的光头上,先是融化,后来就积成了薄薄的一层。轻易露营是不可以的,夜里会被野兽啃了脑袋,就算没有野兽光顾,也有可能在梦里冻死。他们和山民们比不得,山民们可以光着身子在大雪地里跑,他们不行。

香川武夫在大雪中浮想联翩,不肯下洞。而洞中的指挥所内,无心正在向众人描述自己所见的巨蛇。

待他话音落下之后,室内静了一瞬,随即马英豪摇了头:``不可能。''

小柳治也跟着摇头:``是不大可能。小蛇可以通过管道或者墙壁缝隙出入,大蛇\ldots{}\ldots{}如果真有水缸粗的大蛇出没,建造地堡的时候一定会有人发现。而且水泥后面都是大块的岩石,即便真有巨蛇在土里钻,也无法突破一层岩石。''

此言一出,无心傻了眼,因为发现他们说的也很有理。

\chapter{小鬼}

因为白琉璃是久候不至,所以无心怀疑他是迷路了。

香川武夫顶着一头薄雪回了来,见小桥惠已经将室内的血迹全部清除干净,就很满意的``嗯''了一声。蹲在火炉边伸出双手,他没有再提金子纯,只说:``外面下雪了。''

马老爷出声问道:``香川先生,你们到底是做了什么打算?如果始终是一无所获,难道我们还当真在山里过冬不成?况且地堡里面有毒蛇,是出乎大家意料的;我们好端端的,总没有住在毒蛇窟里的道理嘛!''

香川武夫抬头向他一笑,轻描淡写的答道:``荒野里有蛇,也是常见的事情。不要担心,地堡的储藏室里一定会有驱蛇的药物。等下我亲自过去找一找。''

马老爷张了张嘴,没说出什么来,末了一甩袖子坐在床上了:``总之我对你们是以诚相待,你们应该保证我一家人的生命安全。''

香川武夫依旧是笑,一张白脸被炉火映成红彤彤。

香川武夫在烤暖双手之后,当真去了一趟储藏室。储藏室紧挨着粮库,隔壁则是将校休息室。他找到一些雄黄,又从粮库里拎出一辫子大蒜。把蒜瓣捣碎之后混合雄黄,他用纱布兜住气味刺鼻的混合物,一团一团的包好了放在门口,说是可以驱蛇。小桥惠照旧煮饭,罐头的肉香混合了金子纯留下的血腥,众人端着空饭盒坐在地上,都感觉自己像是要吃人。

赛维和胜伊,自从目睹了金子纯的惨死之后,就再也没有说过话。双手捧着一饭盒粘稠的肉粥,他们闷头大嚼,一顿能吃过去一天的量。吃饱之后缩回角落,赛维向后靠着墙壁,胜伊闭着眼睛偎在她的身边。

无心在两人面前蹲下了,轻声问道:``你们怎么了?''

赛维半睁了眼睛,低声答道:``我们怕了。''

胜伊也哑着嗓子开了口,声音低得像耳语:``无心,我们不想死,我们想回家。''

无心勉强笑了一下:``我保证,一定送你们回家。而且是活着——你们和我,都活着。''

赛维和胜伊一起向他微笑了,赛维没有了脂粉的修饰,彻底露出原形,和胜伊的面貌是一模一样。对比之下,她不大像个女人,胜伊也不大像个男人。

二十来名日本士兵蹲在半山腰的岔道之中,是整座地堡的总看守。天黑了,指挥所内的两盏灯里都添了煤油。赛维靠在墙角昏昏欲睡,胜伊却是想要出去撒尿。马俊杰跟上了他,两个人一起开门进了走廊。

胜伊心中绝望恐慌,导致情绪低落,越发的讨厌男人,即便马俊杰还不算男人。他不理睬马俊杰,自己找了僻静地方解裤子。马俊杰板着一张小白脸,也不和他亲近。

掏出家伙哗哗哗的长尿了一场,胜伊长吁了一口气,一边系腰带,一边转身要往回走。然而回头之后他愣了一下,发现马俊杰不见了。

``俊杰?''他出声呼唤,然而没有回应。

他竖着两只耳朵又听了听,发现周围还是没有动静,便在狐疑之余,放心大胆的放了一个屁。此屁他忍了许久,一直无处释放,如今终于痛快了。

他怀疑马俊杰是尿过之后先往回返了,于是一边走向指挥所,一边又喊:``俊杰?''

视野之中并没有俊杰,只有远方一扇半开半掩的房门,关着满室明亮的灯光。马俊杰无端的消失让胜伊有些恐慌。他不是一位有责任感的好三哥,口中胡乱大喊着俊杰,他的步子越走越快,心惊胆战的直奔指挥所。

可就在距离指挥所十几米远处,他忽然听到身后传来了微弱的响动,像是哽咽,也像是叹息。寻觅着声音回头望去,他在微弱的光明之中,骤然爆发出了一声尖叫!

他看到了金子纯!

金子纯站在暗处,身上还裹着一层帆布。一只手紧紧扼住了马俊杰的细脖子,他满头满身都是土。而马俊杰面红耳赤的背靠了金子纯,双手还在拼命的拉扯对方的手。忽然意识到胜伊发现自己了,他拼命向前伸出双手,舌头长长的吐出来,同时痛苦的做出口型:``三哥\ldots{}\ldots{}''

胜伊下意识的上前几步,不加思索的握住了马俊杰的手。一握之下他怔住了——马俊杰的小手寒冷如冰,竟然硬的如同铁钳一般。双方的手指刚刚相触,他便被马俊杰一把抓了个紧。胜伊感觉不对劲,哭叫着想要往后退,然而为时已晚,他退不成了!

金子纯垂着头,仅余的一只手依旧掐着马俊杰的脖子。而马俊杰的脑袋渐渐歪成了不可思议的角度,同时双手越来越有劲,是把胜伊一点一点的往自己怀里拉。胜伊半蹲下去,靴底在水泥地上磨出声音。一个脑袋转向指挥所,他吓得哇哇大哭:``姐!无心!救命啊\ldots{}\ldots{}''

就在他一寸一寸蹭向马俊杰之时,指挥所内跑出了人。无心手里拿着马老爷的硬木手杖直冲而来。举起手杖比划了一下,他飞快的又看了马俊杰一眼,随即把牙一咬,一杖就抽上了马俊杰的手臂。走廊内响起``喀吧''一声,马俊杰一声不吭,两条小臂已然一起骨折。

胜伊嚎啕着拼命后退,退着退着回头看到赛维。赛维一把攥住了他的腕子,手指头陷进他的肉里,冷津津的直哆嗦。而无心咬破指尖,把血珠子迎面甩上了前方二人的面孔。金子纯和马俊杰像是被淋了镪水,登时抽搐着要躲。而无心趁热打铁,扑上去一手一个掐住了二人的脖子,同时大声吼道:``白琉璃!没死就给我滚出来!妈的闹鬼诈尸了!''

话音落下,他只觉手中的身体忽然一软。金子纯和马俊杰都像被人抽去了骨头一样,无心一松手,他们就沉重的瘫倒在地了。

香川武夫提着一盏煤油灯走近了,仔细的去照地上两具人身。马老爷跟在一旁,因为看清楚了马俊杰弯折的脖颈,所以当场惊叫了一声。

无心弯腰试了马俊杰的鼻息,随即起身答道:``五少爷已经死了,被金子纯掐死了。''

香川武夫哑着嗓子说道:``金子纯\ldots{}\ldots{}是我亲手埋进土中的\ldots{}\ldots{}''

无心转向了他:``香川先生,我们真的该离开了。这座山很邪门;埋在这里的巫师们阴魂不散,杀气比活人还重。金子纯的确是早死了,但他死后被恶鬼附体,又回了来。''

香川武夫一耸肩膀,因为气息紊乱,所以声音又轻又高,很有马老爷的风格:``难道死在这座山里的人,都会被恶鬼附体吗?''

无心想了一想,随即摇头:``不是。''

香川武夫做了个深呼吸,风笛似的从鼻孔中哼出响亮疑问:``嗯?''

无心答道:``我们所见到的几具干尸,不是都死得很老实吗?''

香川武夫把眼睛缓缓的瞪圆又眯细,一张保养良好的白脸慢慢转向了马老爷,马老爷蓬着一头无法无天的卷毛,目光凌厉的瞪了他一眼:``不要看我,我不知道!''

香川武夫的大白脸被马老爷瞪回了前方。对着无心出了一会儿神,他有很多话想要问,可是一时却又不知从何问起。无心噙着受了伤的手指头,一边翻着眼睛看他,一边用牙齿轻轻去咬创口。忽然抽出手指转过身,他在赛维和胜伊的眉心分别划了一指。淡红色的稀薄血液涂在了他们的皮肤上,而他们当着众人,心有灵犀的一言不发。

金子纯和马俊杰静静的躺在地上,无心瞥到了两团微光在他们身上浮动,仿佛受到了某种力量的牵引,微光向着一个方向闪烁不止。金子纯身上的光芒更盛一点,忽然明亮忽然又微弱,他的光芒凭空消失;而马俊杰的魂魄一点一点离了身体,斜斜的飘向了前方的岔路口。

无心随着魂魄迈开步子,走过长长的走廊,进入岔道之后又接连拐了几个弯。最后在一扇小铁门前,他看到了白琉璃。

白琉璃像个初学念经的小喇嘛,前仰后合的低诵不止,咒语的字字句句都是连绵着的,任谁也听不清他说的到底是什么。一只浅浅的小碗摆在地上,先前本是个孩子的头盖骨。碗中盛着一点腥红液体,液体里面又浸泡了一只指头长的小木人。

无心小心翼翼的俯身撩开了他的长发。看到了他半闭着的蓝眼睛,和一线肮脏苍白的额头。他的发际已经渗出了汗珠,黑色的睫毛随着声音不住震颤。

``你所收的魂魄。''无心轻声问道:``是个十几岁大的男孩子吗?''

白琉璃充耳不闻,继续摇头晃脑,汗水成股的流过了他的眉毛。无心环顾四周,发现马俊杰的魂魄消失了。

咒语戛然而止,白琉璃毫无预兆的睁开了眼睛:``他逃走了。''

无心有一点惊讶:``你竟然——''

无心是没想到凭着白琉璃的巫术,竟然连只小鬼都拘不住。而白琉璃垂下了头,低声说道:``他的怨气很重,你们小心着吧!''

然后他猛的一哆嗦,对着无心抬起了头:``什么来了?''

无心抽抽鼻子,没有嗅到异样的气味。可是发自本能的也感觉到了危险。忽然向一旁扭过头去,他瞬间睁大眼睛,看到了一条巨蛇!

巨蛇是黑色的,与黑暗融为一体。它明明是在游动,然而静得像个影子,蜿蜒的经过了路口。

无心蹲在白琉璃面前,压低声音说道:``是一条蛇,水缸粗的大蛇,我先前见过一次,可是他们都说地堡里不可能有巨蛇,不相信我。刚才,我又看到了。''

白琉璃自认并不符合蛇的胃口,所以不甚慌张:``哦,是蛇。''

然而无心随即又道:``可是\ldots{}\ldots{}它没有头。''

白琉璃答非所问:``我没有嗅到蛇的臭气,只嗅到了鬼魂的阴气。''

无心很想和白琉璃谈一谈巨蛇,然而白琉璃显然对鬼更有兴趣。无心无可奈何的伸手一指他:``我和你永远说不到一起去!现在我要去追大蛇,你要么就乖乖呆着别动,要么就去指挥所!''

随即他起身要走,不料刚一抬头,却是在暗处看到了影影绰绰的马俊杰。

马俊杰满怀仇恨的瞪视着无心——他只是想平安的长大,只是想分一点钱给娘做私房。然而娘死了,他也死了。为什么连鬼都要欺软怕硬,为什么只杀他,不杀胜伊?他想回北京,他不要再呆在暗无天日的地堡中了!

可是,他回不去了。

无心没理他,回头又问白琉璃:``附在金子纯身上的,还是金子纯的魂魄吗?''

白琉璃摇了摇头:``不是。''

无心点了点头,心想洞内鬼魂无数,而且全都颇有力量,忽然得到一具尸首,难免它们不生利用之心。看来在地堡之中,活是活不舒服,死后也不得轮回。

在临走之前,他对白琉璃说道:``小鬼在你身后。''然后拔腿便跑。

白琉璃没有回头,半闭了眼睛继续念咒。而马俊杰只觉身心涣散,慌忙乱飘一气,远远的避开了白琉璃。

无心去追大蛇,连着通过了几条甬道,终于看到了大蛇的尾巴。

他心中一喜,加快了脚步。不料大蛇忽然停了动作,长长的瘫在了地上。蛇尾渐渐膨胀,猛的一昂,竟是成了个头的样子,无声无息的迎向了无心。无心刹住脚步,只见前方由蛇尾变化成的蛇头无鼻无眼,只有一张不住蠕动收缩的巨口,口中黑洞洞的,仿佛直通巨蛇的腔子。

\chapter{蛇地面目}

无心站在水泥地上,仰头看着巨蛇越昂越高,由尾巴转化成的头颅巨口居高临下,是要把自己一口吞噬的姿势。可是在片刻之前,他看得真真切切,巨蛇分明是背离自己前行,绝对没有转身——从来没有听说过倒退爬行的虫蛇,可是尾巴又怎会变成了头颅?

巨蛇开始缓缓的向他移动了,无声无息,也无气味,像一团巨大的黑云,几乎就是在地面上柔软的飘。无心手无寸铁,想逃也来不及,索性站在原地不动,同时又发现了一个奇异现象——巨蛇没有牙齿!

没有牙齿,没有信子,黑洞洞的就只有一张口。无心忽然感觉它不像蛇,更像虫,一条前后贯通、不分首尾的虫。

巨口向下缓缓对准了他,仿佛是要做到一击即中。无心仰起了头,想要到它的腹内一探究竟。可是对着上方的巨口睁大了眼睛,他发现巨蛇身上显出了细密纹路,不是花纹,而是凹凸蠕动的纹路。

恍然大悟的退了一步,他明白了。日本人没有说错,地堡之内的确不存在硕大无朋的巨蛇,巨蛇是由无数条小黑蛇组成的!

小黑蛇们互相拧绞纠缠,构成一条庞然大物。任何被巨蛇吞入的活物,都会立刻遭受万蛇噬身的痛苦,然后瞬间成为一具干尸。

所以它不分首尾,无须转圜,带着凶恶的灵性。一旦大难当真临头,它自然会解体为无数细小黑蛇,墙壁缝和下水道都是它们的避难所。即便地堡崩塌,也有整整一座大山供它们钻洞穿梭。在地下,它们没有克星。

无心不想钻进巨蛇腹中了,如果当真被万千小蛇吮成了人干,他相信赛维和胜伊都会哭泣,但也只是哭泣而已。赛维和胜伊爱他,爱的有条件,凉丝丝。真心有几分?他不知道。可是对于他们来讲,已经尽力了。他们本来是相依为命、谁也不爱的。

他不敢以着人干的面目出现在他们面前,他们胆量有限,热血也有限,他不可以去吓唬他们。

于是无心骤然横着跳出老远,随即方向不变,贴着巨蛇的身体穿过走廊,逃进黑暗。巨蛇果然没有调头。高昂的头颅低垂下去,收缩成了细长尾巴。它并未像真正的蛇一样贴地游动,而是身体旋转向前,宛如一只硕大柔软的黑色钻头,闪电一样冲向了无心的背影。

虽然巨蛇根本不能算蛇,但是无心无计可施,只好还是把它当蛇来对付。在走廊中左右腾挪跑成了``之''字形,他眼看前方就到了头,连忙提前调整方向,拐进了尽头的一条岔路。察觉到巨蛇尾随而至了,他向前直奔,却是在岔路尽头一脚踏空,踉跄着扑了下去。身体跌在冷硬的台阶上,他发现自己竟然是跌上了一条向下的水泥楼梯。

他没想到地堡下面居然还有一层,当初金子纯也不曾提过。一挺身爬起来站稳了,他慌不择路,沿着楼梯向下疾行。落脚之处由平整渐渐变得崎岖,水泥台阶越往下越是粗糙,最后索性断崖似的没了路。无心停在最后一级台阶上,横了心纵身一跃。只听``咕咚''一声,他结结实实的摔在了黑土地上。

黑土地距离最后一级台阶,能有个半人多高,爬上去很容易,掉下来也摔不坏,是个没有杀伤性的大土坑。土坑是挖出来的,还是天然有的,一时瞧不出;无心抬头向上望去,就见巨蛇停在岔道地面上,身体一端慢慢沿着台阶探下,如同水蛭一般,越是伸得远,越是拉得细长。无心知道自己不带活人气,素来不招野兽,所以心中疑惑,不知道巨蛇为何对自己产生了兴趣。人在坑中无路可退,他眼看蛇头摇摇摆摆越来越近,只好贴着坑壁站直了,一动不敢动。

蛇头越过最后一级台阶,依然像只钻头似的,翻翻滚滚的四处游动。一条细小黑蛇脱离大蛇身体,自作主张的伸出脑袋,要往黑土里钻。蛇头漫无目的的蹭过了无心的小腿,因为细长,所以嘴巴的尺寸小了许多,然而不住的收缩扩张,是个贪婪饥渴的模样,仿佛随时预备着吞噬什么。

无心盯着蛇头,看它在坑中搜索一圈,无功而返,懒洋洋的缩回了上方走廊。恍然大悟的低下了头,他盯着已经没入土中大半的小黑蛇,心想原来它是依靠气流来捕捉猎物的。自己方才连走带跑,跑到哪里,它便追到哪里;自己不动了,它反倒失了目标。

自己可以做到完全的静,但是平常人至少还要呼吸,无论如何逃不脱它的追逐。低头抓住蛇尾向外一拽,他动作极快的掐住了蛇头。黑暗之中,他的黑眼珠是特别的大,没有光,他一样的能看。小黑蛇的身体柔软滑腻,太像一条肉虫,蛇头上鼓起一只嫩嫩的肉泡,像婴儿未睁的眼睛。无心用手指轻轻去摩挲肉泡,结果拨起了一层半透明的薄膜,薄膜下面,竟然真是一只圆圆的眼珠。

无心又扒开了它的嘴,嘴是一圈软肉,类似吸盘。牙齿尖锐细短,上下各有两枚,正可以一口咬破猎物的皮肉,而又不至于咬过之后抽拔不出,堵住伤口鲜血。无心见它在自己手中扭动得还很有劲,就试探着将一根手指插入了它的口中。指尖瞬间一痛,箍住一节手指的蛇嘴清清楚楚的吮吸了一下,随即却松了口。显然,无心的鲜血不合它的口味。

无心收回了手指,顺势又去拨弄了蛇头上方的一只眼。随着他的施力,小黑蛇在他手中开始挣扎;无心忽然向下一摁,只听一声轻微的响,他戳破了小黑蛇的眼珠。而小黑蛇当即松软成了他手中的一条绳子,正是死了。

无心握着蛇尾巴抡了一圈,心想自己但凡有一点头脑,都该马上逃出地堡,哪怕大雪封山,哪怕在外头冻硬了,也比窝在地堡里强。小蛇来了,可以戳它的一只眼;大蛇来了,怎么办?

无心在土坑中转了一圈,认为水泥楼梯大概是件半成品,大坑也应该是下一层地堡的入口。可惜台阶未完成,下一层地堡更是连影都没有。连滚带爬的上了台阶,他一路鬼鬼祟祟的走回了指挥所。

把小蛇的尸体摆在煤油灯下,他对着众人讲述了大蛇的底细。话音落下,香川武夫和马老爷当即开始斗嘴。马老爷蓬着一头卷发,委屈死了,无论如何想要回家;香川武夫肩负着任务,当然不能无功而返,而且并不相信马老爷是真的坦诚。小柳治和马英豪并肩站着,煤油灯下,他们脸色变幻不定,统一的灰头土脸。金子纯和马俊杰的尸体,被人用粗尼龙绳紧紧捆绑住了,直挺挺的叠在门旁墙边,小桥惠蹲在尸体头旁,似乎是个守卫。

香川武夫和马老爷全是巧舌如簧,吵了个天翻地覆。马老爷一直没有去看马俊杰,此刻忽然想起他是自己的儿子了,指着马俊杰的尸首对着香川武夫咆哮:``我的小儿子,已经死了!''

香川武夫脸色铁青,一只手按在腰间的手枪皮套上。

马老爷的气焰随着嗓门一起增长,抬手对着香川武夫的光头指指点点:``你们的人都是废物!废物!到目前为止,只有无心做了一点实事,而你们除了挖几个坑,死几个人之外,还有什么成绩?我告诉你们,不要妄想让我也死于你们的愚蠢!''

香川武夫拔出手枪,一枪指向了窝在角落里的赛维:``马先生,你还要继续说吗?''

马老爷登时闭了嘴,别的孩子爱死不死,赛维和胜伊是要好好活着的。他们多么的像他,他们三个才是纯粹的一家人。

香川武夫放下了枪,眼角余光瞄着房内所有人,同时咬牙切齿的低声说道:``大蛇,小蛇,都没有关系。军火库里有很多武器,我不相信它们比枪炮更厉害。靠近出口的岔道里比较安全,如果你们愿意,夜里可以去和士兵一起睡。现在我要去军火库取武器。''他面色阴鸷的点了点头:``谁想逃,就是死!''

香川武夫推开了房门,对着走廊吼了一句日本话。两名士兵立刻从附近的岔道中答应着跑上了主干道走廊。室内众人眼巴巴的望向门口,就见香川武夫从怀里摸出一张地图和一只手电筒,带着两名士兵拔腿就走,同时``咣''的一声摔了房门。

留在室内的人,谁也不看谁,唯有马老爷长叹一声:``作死啊!''

无心有点无所适从——和身边这一帮人在一起,他总感觉双方之间有隔膜。手里揉搓着软软的小黑蛇,他低头坐到了赛维身边。赛维也没理他,和胜伊一起抱着膝盖蜷成一团。

赛维没有历险的经验,此刻周身上下只有一颗心还活着。忽然懒得指望旁人了,她决定自己找出一条生路。

足足过了几十分钟,还是不见香川武夫返回。马英豪开了口,说道:``无心,出去瞧一瞧,找不到香川先生,找到白琉璃也是好的。自从送他进了地堡,我就没再见过他。''

无心靠墙站起了身,同时听得马老爷也出了声:``不许去,要去让他的同胞去。''

无心半蹲半站,贴着墙壁很犹豫,手里还攥着死蛇。赛维偶尔回了神,正好听到父亲和大哥的对话,当即也发表了意见:``坐下!''

无心死心塌地的慢慢溜下去了,又扭头看了马英豪一眼。马英豪皱起眉头望着他,不过没有怒色,仿佛是不肯和他一般见识。

然后,震天撼地的爆炸就开始了!

爆炸仿佛就发生在所有人的脑海心窝里,煤油灯的火苗全凌乱了,巨响几乎震得人呕出鲜血。铁门受了气流的鼓动,单薄的一道铜锁当场断裂。而灰头土脸的香川武夫一头扎进室内,用日本话嚷道:``撤退,撤退,到地面去!蛇来了!''

此言一出,无论听懂听不懂的统一起了立,然后也无须多问,众人一窝蜂的全涌了出去。一名浑身是血的日本士兵站在主干道上,正在用撬棍拼命去撬一只木箱。喀嚓一声木条断裂,士兵伸手进箱摸出手雷,哆嗦着打开保险,用力在墙上一磕,随即没头没脑的向前掷去。

赛维腿都软了,可是看清了士兵的动作——打开保险,再磕一下。看清之后她一把拽住胜伊,撒腿就跑,跑了几步之后回了头,又把无心也扯到了身边。她瘦而强硬的向前蛮顶,面目狰狞,气冲如牛。滚烫的气流一波又一波的冲击着岔道里的人,她寸步难行的在士兵之中移动,两只手不能兼顾了,她不假思索的松开了无心,专门拉扯胜伊。

一鼓作气的,她把烂泥一样的胜伊先推上了地面。冰冷的空气扑在她赤红的脸上,她像猴子一样随即攀援而上。下面有人托举了她,手很有劲,手指苍白,是无心。

赛维没有回头,上了地面之后见胜伊还瘫在地上,就俯下身拼了命的推他踢他,当他是个铺盖卷,一路让他滚出老远。胜伊屁用没有,说他是浪蹄子都是抬举了他,他都不如一般的好娘们儿坚强;但是赛维得先顾着他,他安全了,她才能腾出心思去看无心。

好在无心是不劳她费心的。她刚一出洞,无心就一屁股拱开小柳治,搭着铁梯窜上去了。

当最后一名活士兵逃出地堡之后,香川武夫在飘飘扬扬的大雪之中,干脆利落的锁上了入口铁门。

无心却是想起了一个人:``白琉璃还在里面!''

香川武夫满头满脸都是硝烟尘土:``他没有用,不要管他!''

无心也没有重新入洞的打算,但是想到白琉璃可能会死,他忽然感觉很难过。

``地堡还有其它入口吗?''他问香川武夫。

香川武夫不置可否的摇了摇头,答非所问的告诉他:``里面应该不会燃起大火,因为缺乏燃料。''

处处都是水泥墙和大铁门,并且铁门全部紧闭,的确是缺乏燃料。

香川武夫又道:``天亮之后再派人进去,今夜我们该做的,就是不要冻死!''

\chapter{夺路狂奔}

半山腰生起了几堆火,一群人本来就接连几天没有洗脸,如今加上烟熏火燎,越发有了鸠形鹄面的意思。两名日本兵死在了方才的混乱防御之中,加上金子纯和马俊杰,他们出师未捷,先丢了四条性命。

赛维从兜里摸出一条小手绢,用雪水浸湿了,自己托在手上擦了擦脸,又扭头给胜伊抹拭了眼睛。胜伊怏怏的半闭着眼,任她抹拭。无心则是守在一旁,希望赛维也给自己擦一擦。

他很有耐心的等待着,一直等到赛维把手帕扔到面前的雪上,等到手帕冻成薄薄的一片。患难见真情,赛维是有真情的,不过全倾注在胜伊身上。非得到了太平时节,才能匀出心思去爱无心。

天上飘着鹅毛大雪,幸好没有起风。无心讪讪的垂下头,被大雪覆盖成了一只雪球。

天亮之后,无人冻死,但是也没有粮食可吃。香川武夫打开了地堡入口,想要派人进去探探情形,顺便带些饮食炊具出来,可是扭头望了部下,他忽然感觉自己无人可派。士兵只剩下二十名不到,是不舍得让他们再枉死的,小柳治和马英豪是自己人,其中马英豪还是个瘸子,也不适宜让他们打冲锋冒险;马老爷老得干巴巴,而且一惯的狡猾别扭;马家的龙凤胎怎么看都是一对瘦弱的大孩子;小桥惠负责了一切后勤工作,也是绝对死不得。

于是,他对着无心一招手:``过来,你下去看一看,如果没有问题,就去一趟粮库,运些大米和罐头出来。''

他一边说,一边将一枚钥匙和一只手电筒给了无心。无心接过了这两样东西,正要下洞,不料赛维忽然叫道:``慢着!谁知道洞里还有多少蛇?你们不能让他赤手空拳的下去!''

香川武夫看了赛维一眼,随即拔出了自己的手枪递给无心。在无心接枪之前,赛维踩着厚雪跑到洞口附近,伸手去开一只木箱:``一把枪里能有多少子弹?给他炸弹用,炸弹比枪厉害。''

香川武夫一直没有留意过赛维,没想到她忽然机灵活泼起来,竟敢擅自去动手雷箱子。把手枪插回腰间皮套,他正要上前阻止,然而赛维已经像捧土豆似的,捧着三只手雷走过来了。

香川武夫犹豫了一下,没有多说。无心莫名其妙的揣了三只手雷,知道赛维此举必有所为,不过当众也不能问,便一言不发的跳入了竖井之中。

地堡内还残留着硝烟雾气,地面横着许多死蛇。无心打开手电筒,一边走一边大声的呼喊白琉璃。将要走到指挥所时,他忽然看到了马俊杰的鬼魂。

马俊杰还保留着死亡时的模样。一个脑袋歪折到了肩膀上,他冷冷的望着无心。

无心听到了他的问话:``你们要走了吗?''

怨气弥漫开来,带着杀意,无心看出他会是个麻烦的小鬼,有心让他魂飞魄散,可是在他动手之前,马俊杰的幻影渐渐淡化,飘飘忽忽的消失了。

无心关闭了手电筒,继续往前走。走廊里并没有活蛇,他停在指挥所门前,发现指挥所内一片狼藉,可是煤油灯居然还亮着一盏。门旁角落里,金子纯与马俊杰的尸首也都还在。

出了指挥所,无心一边去开隔壁粮库的铁门,一边高一声低一声的召唤白琉璃。铁门表面被手雷崩得坑坑洼洼,幸好锁眼依旧清楚。他从粮库里背出一袋大米,一些包装好的干菜与罐头。回到指挥所,他找出两根尼龙绳,把食物与尸首分别绑成两捆。死结刚刚系好,他忽然回头,看到了门外探头缩脑的白琉璃。

无心不喜欢他,可是此刻见了他,却是意外的高兴。随手捡起一只散落的罐头,他抠开铁皮走向白琉璃:``我想你也不会死。''

然后他蹲在白琉璃面前,把肉罐头放到了他的袍襟上:``你要不要水?''

白琉璃偏着脸,用一只蓝眼睛看他:``你们要走了?''

无心答道:``洞里有蛇,他们不敢再住下去,但是也不会走。''

他用手指从铁皮罐头里挖出一块肉,送进嘴里咀嚼:``现在不走,就走不成了。''

白琉璃轻声答道:``我很喜欢这里,我要留下来,哪里都不去了。''

无心舔了舔手指,又去罐头里拿肉吃:``一个人,还有蛇,不寂寞?不害怕?''

白琉璃看他吃得很香,就把铁皮罐头放到了他的面前:``不寂寞,这里有很多鬼魂,老的,小的,男的,女的。我看不到它们,但是我知道它们都在。它们比人更好,我爱他们胜过爱人。人的心思,我总是摸不透。我摸不透,我就很累,很愤怒。''

无心端起了罐头,向他微笑:``对不起。''

白琉璃佝偻着腰,喃喃的低语:``对不起我的人,都被我杀死了。他们变成了鬼,反倒不会再亏待我。''

话到这里,他阴恻恻的一笑。一只手托着怀里的婴尸,他四脚着地慢慢的向前爬。爬到了两具尸首跟前,他很怜爱的摸了摸马俊杰的头发,然后背对着无心说道:``把他们留给我吧,我的孩子都饿坏了。''

无心几口吃光了罐头,然后扛着食物站起身:``白琉璃,别威胁我。''

白琉璃低沉的笑出了声音,笑得身体颤抖:``你怕什么?如果我的诅咒能杀死你,你早死过千万次了。无心,为什么你会是个骗子?我想不通,我很想不通!''

无心迈步向外走去,听到白琉璃在后方又道:``你还会再回来的,骗人的人,也会被骗!''

小桥惠有了无心运出的粮食肉蔬,便开始埋锅造饭。众人吃饱喝足之后,香川武夫带了十几名士兵,开始了新一天的寻觅。小柳治和马英豪照例是并肩坐在火堆旁,头上共顶着一片雨布挡雪。

起初是天下太平的,可不知是谁先挑起了头,马老爷和马英豪一递一句的对了话,并且不是好话——大冷的天,夜里又是死里逃生,他们此刻自然没有心情再虚以委蛇。

双方的语气越来越激烈,马英豪反常的激动了,历数了马老爷一生的罪状——马英豪的娘得了重病,马老爷连个医生都不让人请,马夫人不是病死的,是躺在床上无人照顾,活活饿死的!而马英豪当时人在日本,回来之后就发现娘没了,娘的遗物也全没了。

马老爷立刻反唇相讥:``你娘不守妇道,死的好呀死得妙!''

马英豪气得头上直冒热气,当即又提起了自己的瘸腿——世上怎么会有父亲忍心把儿子打成残废?

马老爷哼哼冷笑:``谁知道你是哪里来的野种?''

马英豪气得脸也不要了,指着马老爷的鼻子问道:``你有证据吗?''

马老爷非常善于气人,抬手一点自己的太阳穴,他轻飘飘的答道:``我有直觉。''

马英豪像被大锤击中了心口,神情痛苦的一闭眼睛,随即低头去拔小柳治的手枪:``好,好,我今天就要弑父了!''

小柳治连忙掀了雨布向旁躲闪,而马老爷起身往赛维身后一避,扯着薄薄的小嗓子又道:``英豪,你问过我了,我还没有问你。你当我不知道你和佩华之间的丑事?她毕竟是你的庶母,你个畜生,竟然连人伦都不要了!''

马英豪怒不可遏,想要把马老爷的脑袋揪掉。马老爷转身就逃,赛维和胜伊对视一眼,起身也逃,临逃之前还扯了无心一把。

马英豪拄着手杖,站在大雪地上打哆嗦。而小柳治眼看马家四人越逃越远,不由得愣了一下,随即高声叫道:``不对,他们要跑!''

此言一出,远方的赛维从袖子里甩出一只沉重手雷。打开保险之后她顺手往旁边的老树上一磕,随即像扔一只铅球一样,把手雷狠狠的掷向了小柳治。小柳治盯着空中飞来的小小黑影,一秒的愣怔过后,他横着一跃,把马英豪扑倒在了雪地上。

与此同时,手雷在半空中爆炸了。

马家四个人跑疯了。

雪太厚了,一脚踩下去,不费劲就拔不出。马老爷老当益壮,一窜一窜的比谁都快。赛维学会了父亲的跑法,也加快了速度,唯有胜伊最弱,需要无心拉扯着才不掉队。但是弱有弱的好处,胜伊一个踉跄跌倒在地,顺着一道斜坡骨碌碌滚下去,彻底的领了先。

跑着跑着,马老爷尖叫一声:``妈的,我们留了脚印,怎么办?''

后方遥遥的起了枪响,显然他们并未甩掉追兵。赛维眼看前方又有一棵老树,就气喘吁吁的对着无心伸出了手:``你不是还有手雷吗?给我一个!''

无心立刻掏出一只给了她。而她一边跑一边除了保险。将手雷狠狠磕上树干,她漫无目的的转身向后又投了出去。一声巨响过后,她向下纵身一跃,口中大喝:``滚!''

马老爷和无心都很听话,当即卧倒,一起往下滚。胜伊先行一步,已经滚到坡底。眼看马老爷张牙舞爪的下来了,他大惊之下,一个鲤鱼打挺就要起身,可惜刚刚起到一半,他便被马老爷结结实实的压在了身下。喉间发出一声哀鸣,他恨不能立刻死了。

可正是要死不死之际,他忽然一扭头,发现自己眼前不知何时多了四只蹄子。

顺着蹄子往上瞧,他看到了一张马脸,一丛鹿角,以及一双黑毛毛的大眼睛。

他的眼睛缓缓睁圆了,随即扯着嗓子高声喊道:``姐,姐,你看,来了个四不像!这玩意儿不是万牲园才有吗?哎呀,姐,它舔我了!爸爸你别压着我\ldots{}\ldots{}哎呀,又舔我了,无心,救命呐!''

\chapter{孤独的猎人}

胜伊平时连猫狗都不碰的,如今陷在大雪地里动不得,被一只牛高马大的大鹿舔来舔去,就吓得通体酥软,同时又有些兴奋。马老爷都翻身爬起来了,他还在雪中摆着``大''字吱哇乱叫,不是让他姐来看四不像,就是让无心来救命。

赛维在大雪中站起来,没等站稳又跌坐下去。无心四脚着地的爬到了胜伊身边要扶起他,胜伊还在张嘴大叫,叫着叫着忽然不叫了,因为不小心和四不像亲了个嘴,舌头碰了四不像的舌头。

他不叫了,赛维却又出了声音,是斩截有力的一声``啊''。众人顺着她的目光望过去,就见雪坡尽头的松树林子里,有个人在向他们招手。招了几招之后,他举起一只长牛角似的号角,很悠扬的吹出了几声鹿鸣,随即扭头跑入了树林深处。四不像听了鹿鸣声音,撒开蹄子冲向了树林;而余下四人愣了愣,追着四不像的尾巴也迈开了步子。

树林很大,树木也密。松树在冬天被冻黑了,风景就显得冷峻阴森。林子外面还能听到断断续续的枪声,林子里面的积雪倒像是比外面地上薄了一点。胜伊抓住无心的手,一路跑得跌跌撞撞,到最后他实在是跑不动了,垂着一只手两只脚,被无心拖着前行。领路的四不像东一拐西一拐,在号角的指挥下越跑越快,末了和前方的人一起消失在了密林中。而马家几人渐渐停了脚步,发现自己暂时安全了。

胜伊气若游丝的趴在雪地上,还挣扎着要说话:``姐,四不像怎么没了呢?''

赛维不爱去万牲园,所以也不认识四不像:``我哪知道?是四不像吗?我看像马。爸爸,你看它是马吧?''

马老爷一屁股坐在了雪地上,气喘吁吁的答道:``是驯鹿\ldots{}\ldots{}和万牲园里的四不像不是一种\ldots{}\ldots{}不过驯鹿也叫四不像\ldots{}\ldots{}累死我了。''

无心留意观察着其余三人的动静,同时也跟着喘。对他来讲,喘和跑是一样的累,于是搭讪着四处走了一圈,发现树林的确是个安身的好地方,处处都是荒草,踏出了脚印也不明显,而且便于隐藏。慢慢转回了马家三人跟前,他听到胜伊缓过了一口气,在兴高采烈的问赛维:``姐,刚才出现的是什么人?驯鹿都有了,是不是圣诞老人来救我们了?哈哈哈!''

此言一出,马老爷和赛维一起叹了口气。马老爷认为自己只要赛维一个就够了,胜伊也是个累赘的货;赛维则是体会到了负担之沉重,因为不知道胜伊要蠢到哪天才算一站。

胜伊不知道自己糟糕到让父亲和姐姐都无法面对自己,还在沾沾自喜的回忆驯鹿。而赛维把无心叫到身边坐下,转移了话题说道:``我和爸爸昨夜里偷偷计划好的,趁着我们出了地堡,一定要抓住机会逃走。否则即便香川当真找到了干尸,我们也是难逃一死。''

无心感觉她许久都没有理睬过自己了,于是连忙点头:``对,没错,应该逃。''

马老爷又道:``趁着雪没下大,我们得尽快设法下山。否则就算香川不杀我们,我们在山里转久了,也得冻死饿死。''然后他向无心问道:``你会不会占卜?能否预测一下我们该往哪个方向走?''

无心摇了摇头,然后说道:``刚才引我们进树林的人,或许会熟悉山上的道路。''

马老爷环顾四周:``谁看清了他的模样?反正我是没看清。''

胜伊和无心一起摇头,只有赛维迟疑着说道:``我怎么感觉他是\ldots{}\ldots{}金发碧眼呢?''

无心从怀里摸出一张纸符,``嚓''的一声撕成两半。小健影影绰绰的出现了,居高临下的对着无心一挥手,送出一个无声的飞吻。

无心轻声说道:``去,看看附近有没有生人。''

小健笑嘻嘻的消失在了半空中,片刻之后回来了,正遇上马老爷问无心:``你也算是半仙之体了吧?''

无心有点窘迫,不知道他肯不肯让个半仙进驻家庭。忽见小健回来了,他匆忙摆了摆手:``哪里,不敢当。''

马老爷和颜悦色的对他一笑:``客气!''

无心继续摆手:``真不敢当。''

马老爷抿着薄嘴唇一转眼珠:``谦逊。''

无心实在是禁受不住马老爷笑成马老太太,于是茫茫然的持续摆手:``绝不敢当。''

赛维看了父亲的德行,有些羞愧,顺便摁下了无心的手。无心听到小健在自己耳边报告:``有个人,蹲在树上偷看你们!他带着一支牛角,还有一支枪呢!''

无心趁机起了身,又把脑袋歪向了小健:``他在哪里?''

小健蹲在他的肩膀上,轻快的答道:``你往左走\ldots{}\ldots{}大哥哥,我很想你,你有没有想我?马俊杰呢?他怎么不见了?''

无心怕小健知道了马俊杰的死讯,要发脾气,所以支吾着不肯回答。小健也是孩子心性,问过就算,并不寻根究底。可是未等他走出多远,遥遥的传来一声枪响,震得马家三人一起蹦了高。

无心无暇再去寻找陌生的窥视者了,他随着马老爷发足狂奔,一路往林子深处冲。而追到林子边缘的小柳治等人却是骤然刹住了脚步——方才的枪响,好像是是响出乱子了!

前方的一棵老树树洞之中,慢吞吞的探出了一只大黑脑袋,正是冬季刚刚开始蹲仓的黑熊受了惊动。受惊的黑熊,脾气自然不会好,直立着身体站在雪地上,它咆哮一声,一掌击折了一棵碗口粗的大松树。随即像个人似的,它昂首挺胸的走向了领头的小柳治。

小柳治身上只带了一只小手枪,拔出手枪退了一步,他知道黑熊和人不一样,想让它一枪毙命是根本不可能,而自己又并非猎人。

一秒钟后,小柳治打响了第一枪。

枪声接二连三的密集了,惊得林中人越逃越远。如此又过了半个小时,小柳治带着部下仓皇撤回山腰。黑熊被他们枪决了,他们也搭上了一名士兵的性命——士兵被黑熊抱在怀里舔了一口,整张脸都被舔没了;黑熊随即又对他动了武,把他的脑袋拍了个扁。

小柳治真是摸不清山林中的门道,尤其是没想到居然黑熊也会成为自己的劲敌。金子纯死了,他们缺少了山林百事通,香川武夫又不在,他越发的不敢再贸然行动。

与此同时,马家四人跑到精疲力竭,一起瘫在了雪地上。他们此起彼伏的喘着粗气,肚子里咕噜噜的鸣叫不止。早饭的能量早就消耗光了,午饭则是根本没有吃;支撑着一身沉重皮袄跑了许久,他们都感觉自己是要死。耳鸣目眩的大睁着眼,他们快要从口鼻中喷出火。无心翻了个身,从地上抓了雪往嘴里送;雪很洁净,甜丝丝的冰凉。忽然有了异样的感觉,他回头向后一望,就见一个人影从高大的白桦树上溜了下来,正是引他们入林、而又不肯露面的窥视者。

窥视者穿着一身兽皮制的厚重袍子,腰间别着牛角似的号角,背后挎着一杆猎枪。正如赛维所说的那样,他披着淡黄色的卷头发,一双眼睛蓝中透绿,乍一看像是个白俄了,然而又并非深目高鼻,相貌介于白俄和本地山民之间。

赛维一挺身坐起来了,眼睁睁的看着来人。而对方在和赛维对视一眼之后,就像吓了一跳似的,口中``呜''的叫了一声。

马老爷见晚辈们只会睁着眼睛发傻,于是亲自起身,拖着两条酸痛的老腿迎上前去,开口便道:``多谢英雄救命之恩。''

英雄总像是胆战心惊,打着结巴问道:``你、你们怕日本人?''

马老爷略一沉吟,决定实话实说:``日本人在追杀我们。''

英雄吐出了一口气,声音当即壮了许多:``日本人坏极了!''

和野人一般无二的英雄坐在一块凸起的老树根上,用磕磕绊绊的汉话做了自我介绍。原来他名叫伊凡,真是本地通古斯人和白俄流浪者的爱情结晶。他和部落里的所有人一样成长和生活,直到日本人来了。

日本人一来,白俄们吓得逃往了苏联。山里的人不懂世界大势,只知道日本人不喜欢五颜六色的眼睛。日本人隔三差五的上山巡视,伊凡因为会说汉话和俄语,山上山下到处跑,给他的部落惹来了许多麻烦;所以当他长到足够大了,脸皮也足够薄了,便很自觉的脱离部落,带着几头驯鹿独自进了深山老林。

伊凡对日本人是又恨又怕,所以在远远望见日本士兵端着枪追逐射击之时,他决定帮助弱者。但是由于摸不清虚实,导致他始终躲藏着,不敢贸然出现。

马老爷立刻就估量出了伊凡的价值,把一张干脸笑得沟壑纵横,同时向他要吃要喝。伊凡很高兴的把他们带回了自己的住处——他的家,就在树林里。

在路上他举起猎枪,不言不语的打下了七八只肥胖的大松鼠。松鼠们统一的很可爱,胜伊看在眼里,心疼极了,认为伊凡没人性。

伊凡自己住着一个小帐篷。通古斯人所谓的帐篷,也叫仙人柱,是把几十根木杆削尖了,一头向下插在地里,一头向天汇聚在一起,成个伞盖的样子,四周再围上兽皮遮风挡雪。仙人柱里有火塘,烟气袅袅的向上升起,仙人柱的尖顶是不封闭的,开着个圆圆的孔,让烟气丝丝缕缕的飘散到很远。

伊凡让他的客人进了仙人柱取暖,自己则是剥了松鼠的皮。在血淋淋的松鼠肉上抹了一层盐,他只在火上略烤了烤,然后就先把肉送给了赛维。赛维和他们部落里的女人都不一样,伊凡看惯了部落女人的扁平面孔,如今骤然见到赛维单薄的小下巴和清秀平淡的直鼻梁薄嘴唇,便感觉很奇异,忍不住的总要看她。

赛维接过了半生不熟的没皮松鼠,不吃会饿,吃了又怕,索性闭着眼睛,呲了牙齿去吃肉。马老爷倒是随遇而安,在松鼠肉的香气中开始展望未来,向伊凡打探下山的道路。

下山的道路自然是有的,伊凡很大方的取出了所有的酒,要给在场的男人们喝,顺便告诉他们:``山下全是日本人的军队。''

男人们接受了他的酒,马老爷抿了一口,辣得当场伸出舌头,胜伊见伊凡的指甲缝里还带着松鼠血,便没有真喝,只用嘴唇在不干不净的碗沿碰了碰。唯有无心大口喝了,是当水喝的。

马老爷把舌头收回口腔,自称是位学者,被日本人强迫着进山寻找半具干尸。随即他问伊凡:``你听说过巫师诅咒的故事吗?''

伊凡的脸色一变——听是听说过的,在几十年前,的确是有一位名声不大好的巫师,用自己的生命诅咒了一批财宝。但财宝还是被汉人军队抢走了。对于伊凡来讲,它实在只是故事,所以语焉不详,只说``死了很多的人''。

马老爷生怕激起伊凡的敌意,故而连忙表示同意,又指着赛维等人说道:``日本人为了威胁我,把我的孩子们都绑架来了。''

因为马老爷表现的十分和蔼,赛维也认认真真的吃了烤松鼠,无心又痛痛快快的喝了一大碗酒,所以伊凡对仙人柱内的汉人们十分满意,当即说道:``明天我偷偷下山去看一看,如果日本军队不多了,我就送你们逃走!''

\chapter{风声与火光}

因为一座仙人柱实在是容不下五个人睡,所以伊凡凭着一己之力,在自己的住所旁边,又搭建了一座新的仙人柱。

举起长牛角似的鹿哨,伊凡把他的驯鹿们召集回了营地。驯鹿的脖子上全拴着白铜铃铛,其中一只最为高大、额头上还带着一抹白色花纹的驯鹿径自走到胜伊面前,原来双方乃是老相识,胜伊和它亲过嘴。

驯鹿有一双温柔的大眼睛,自来熟的和胜伊挨挨蹭蹭。胜伊试着摸了它一下,见它不咬人也不踢人,就奓着胆子又摸了一下。

``姐。''他轻声唤道:``你看它呀,它又来舔我了。''

他姐吃松鼠肉塞了牙缝,正在愁眉苦脸的偷偷抠牙齿,没有听到他的呼唤。

伊凡干活干得很来劲,一边劳动一边偷眼去看赛维,干着干着他唱起了歌,声音悠扬,是唱给客人们听的。客人们听得十分惶恐,生怕他嗓门太大,会招来日本追兵。

然而结果出乎了所有人的预料,伊凡并没有招来日本追兵,反而是招来了一头小黑熊。

如今正是黑熊们刚开始蹲仓冬眠的季节,伊凡的歌声应该还不至于吵醒一头黑熊,所以小黑熊是怎么醒的,委实是桩无解之谜。总之小黑熊摇摇摆摆走向仙人柱时,伊凡手中只有一根细木棍,武器全在仙人柱里。忽见赛维怔怔的迎着黑熊站在人前,他灵机一动冲上去,三下五除二的解开了她的皮袄。皮袄里面是一件对襟毛衣,毛衣里面是丝绸衬衫。伊凡一边回头留意小黑熊的行动,一边扯开了赛维的里外上衣。赛维先是被熊吓呆了,此刻又被伊凡吓呆了。等她回过神时,她发现自己已经是袒胸露乳的面对了小黑熊。

细嫩的皮肤绷在条条肋骨上,寒风掠过了她的胸膛。两只乳\emph{房乃是似有似无的写意画,唯有乳}头受了寒冷刺激,鲜红的翘成两只小钉头。小黑熊直立着停在原地,对着赛维愣了一愣,而伊凡趁机钻进仙人柱,找出了他的猎枪和弓箭。

等他冲出来时,小黑熊已经转身走了。而赛维满脸通红的掩了怀,转向伊凡扬起手,抽出一记雷似的大耳光:``臭流氓!''

伊凡忍痛挨了她的巴掌,开口解释道:``女人在黑熊面前露了奶*子,黑熊就不伤害她。''

赛维听到``奶*子''二字,越发羞臊之极,高声骂道:``放屁!除非黑熊和你一样,都是个色胚!''

在场众人方才全见识了赛维的乳*房,并且全认为赛维根本无须大动肝火。赛维轻描淡写的胸部让人感觉她并未失了任何贞洁,胜伊甚至认为自己脱掉上衣之后,是可以冒充半裸赛维的。

马老爷上前左右劝了几句,又对赛维连连使出眼色,不许她得罪伊凡这位救命星。赛维背对着人系了纽扣,气得骂骂咧咧,忽然抬眼和无心对了目光,她像要哭似的,眼里闪烁了水光。

无心手足无措,因为不知道说什么才好。伊凡的动作太快了,当时大家眼前一花,赛维已经敞开了前襟。伊凡显然不是坏人,向他报仇并不合适;可赛维气得眼泪汪汪,也是让他不能坐视的。

``如果再有熊来了\ldots{}\ldots{}''他嗫嚅着说:``你藏到我身后去,我会保护你\ldots{}\ldots{}''

胜伊也来安慰她:``姐,如果再有熊来了,让我代替你好啦!反正我们两个也差不多\ldots{}\ldots{}''

赛维横了他一眼:``滚你的蛋!如果再有熊来了,我就撕了你喂熊。''

胜伊碰了个钉子,讪讪的垂着脑袋走回了大驯鹿身边。

赛维心中憋气,独自坐在一块大石头上,迎着寒风长吁短叹。无心帮着伊凡干活,也无暇去哄她。马老爷倒是有闲,然而身为父亲,有话也不好说。

傍晚时候,新的仙人柱彻底完工了。伊凡猎回了三四只肥美的大松鼠,把松鼠肉烤得顶风香出十里地。把烤好的嫩肉送到赛维面前,他用碧蓝的眼睛直瞪瞪的看着她。

赛维接受了松鼠肉,并且恨不能把伊凡和松鼠一起嚼了。

伊凡看出了赛维的怒意,所以还把松鼠的眼珠放在一只碗里,要给她吃——在他的文化里,吃了松鼠的眼睛,会走运的。

赛维不肯去吃松鼠的眼珠,伊凡让了一圈,马老爷和胜伊也对其避之惟恐不及,最后松鼠眼睛全进了无心的肚里。

吃饱喝足之后,赛维和胜伊进了新仙人柱,守着新挖出的火塘想要睡觉。无心想了想,也跟着弯腰进去了。地上铺了厚厚的兽皮,他躺在正中央,伸出两条手臂给赛维和胜伊做枕头。可是未等他们真正入眠,一直和伊凡坐在旧仙人柱里的马老爷却是过了来。

马老爷抱着膝盖蹲在火塘前,嘁嘁喳喳的说了一番话,说得赛维面目失色——伊凡刚刚向马老爷提了亲!

伊凡看到了赛维的两粒小奶\emph{头,也看到了赛维的愤怒与悲伤。他想赛维的奶}子一定是没有被别人看见过的,而第一次看见了它们的男人,当然应该娶了她,从此一辈子都只看她一个人的奶\emph{子,哪怕她的奶}子只是两座隐隐约约的小山包。

他理直气壮的向马老爷表明了心意,并且表示自己还可以从部落里得到二十头驯鹿,是个有财产的好猎手。马老爷惊得合不拢嘴,刚想说自家女儿已经有了男朋友,可是话到嘴边,他又怕伊凡求亲未遂,恼羞成怒,会把自己一家驱逐出去。

马老爷很爱赛维,但是到了生死关头,他的理智还是占了上风。他希望赛维可以敷衍一下伊凡,必要的时候,做一点牺牲也无所谓。然后他郑重其事的望向无心,又道:``等到我们逃出生天了,赛维当然还有选择对象的自由,我是不会干涉的。''

赛维心如明镜,很明白父亲的意思。不置可否的向后一仰,她像只肚皮朝天的青蛙,蜷着两条细长腿说道:``知道了。''

马老爷叹了一声,回到旧仙人柱,很诚恳的对伊凡打太极:``她的脾气很坏,还在赌气,不肯理我。睡吧睡吧,明天再说。''

伊凡没心眼。听了马老爷的话,他就真闭上眼睛等明天了。

新仙人柱里,火塘里的火炭呈金红色,缠绵的温暖了整座仙人柱。赛维大睁着眼睛向上望,通过仙人柱顶端的圆孔,她可以看到很高很远的银河。

她不想去和一个素不相识的年轻野人谈恋爱,即便只是虚与委蛇,也万分的不想。但此刻不是她使性子的时候,她也不能指使无心去和野人决斗。四条性命系成一串,一起悬在刀尖上,就算不顾念父亲,也得顾念胜伊。

慢慢的侧过脸,她看到无心也没有睡。胜伊倒是滚远了,侧身蜷在兽皮的边缘,喘出微弱的鼾声。

伸手抚摸了无心的鼻梁和嘴唇,她把他的脑袋扳向自己。他的眼睛又黑又亮,简直不是人类的亮法,然而眼神又是无比的温顺,仿佛是在渴求和享受着她的抚摸。

她和无心对视了片刻,心中像是灌满了山林的风,又浩荡又酸楚。他是她的,她爱他的眼睛,爱他的鼻子,爱他的嘴唇,爱他的皮肤。他真驯良、真可爱。手肘支地半撑起身体,她侧身俯视了无心。看着看着,她低下头,亲吻了无心。

吻过一次之后,她在寒冷的空气中轻声说道:``我的第一次\ldots{}\ldots{}想和你。''

无心静静的望着她,低声问道:``我们还没有结婚。万一你将来又不爱我了,怎么办?''

赛维笑了一下:``我如果不爱你了,就去另找新的爱人。''

无心抬手为她撩起鬓边的半长碎发:``我不想耽误了你。''

赛维依然是微笑,然而说起话却是咬牙切齿:``没有人能耽误我,我也不用任何人来负责。不是你睡了我,是我睡了你。''

话音落下,角落处的胜伊动了动耳朵。意识渐渐浮出睡眠,他微微睁开了一线眼缝,看到赛维解开了她的皮袄、毛衣、衬衫。而无心先是平平的躺在地上,后来忽然一跃而起,翻身把赛维裹到了身下。仙人柱里起了风浪,风浪压抑着涌动在两人之间。皮袄甩开了,胜伊看到无心拱起了雪白的屁股,仿佛他的身下竖着刀剑,让他的屁股只能像波浪一样起伏,永远不能落地。下方的两条细腿颤抖着打开,让胜伊紧握双拳,跟着一起战栗了。

上方的屁股饱满的翘起,随即向下猛一冲锋。在赛维发出的一声痛哼之中,胜伊紧紧一闭眼睛,仿佛和姐姐一起失去了童贞。

铺天盖地的悲伤席卷过了他,他半昏迷似的瘫在原地。他还是原来的他,可姐姐不是原来的姐姐了;他遭到了遗弃和背叛。

本来女人不喜欢他,男人不喜欢赛维,但现在不是了。雪白的无心还在反复碾压着赛维,胜伊一动不动,仿佛和姐姐一起被他碾压了。

风浪是很久之后才平息的,胜伊看到他们都穿戴好了,才想起了仙人柱内还有个自己。无心轻轻的呼唤他,显然是希望他不作回应,于是他如了他们的愿,死了一样装睡。

随即他们又抱在了一起,互相饿慌了似的乱啃一通。

后半夜,胜伊恍惚中又看到了无心的白屁股,还有赛维陷在无心短发里的手指。仿佛对方是一团火,他们互相抱紧了又松开,松开了又抱紧,全是被烫伤了的模样。

胜伊闭了眼睛,忽然感觉他们很无聊。

天亮之后,赛维心平气和的钻出仙人柱,用融化了的雪水洗脸漱口。她已经遂了自己的心愿。第一次很重要,她给自己制造了一个美妙的第一次。无论以后是否嫁给无心,仙人柱内的火光与风声,都足以让她心满意足的铭记终生。

伊凡吃了很多薄薄的肉干,然后背上猎枪,骑着驯鹿下山去了。

无心坐在向阳的石头上,半闭着眼睛晒太阳。

胜伊依靠着和他亲过嘴的大驯鹿,看一只小灰雀在树上叽叽喳喳、东啄西啄。

马老爷在火塘上加热掺了水的烈酒,想要让自己的老胳膊老腿血脉流通,以便随时可以灵活的逃命。

伊凡下午回了来,给赛维带了一包酥糖,又告诉马老爷:``山下有很多日本人,他们捕捉了许多猎民,让猎民像军队一样,在雪地里排队走步。''

\chapter{意外发现}

伊凡踩上松木制的长滑雪板,山上山下到处跑,想要为马老爷找出一条安全的出山路,然而山上有日本人,山下也有日本人,上下都不安全。伊凡见了日本人,就像松鼠见了猎人,因为天生就带着金发碧眼的招牌,在日本人的眼中,是非常的该杀。如果不是充分意识到了自己的该杀,他也不会冒险躲到山里。这座山对于本地所有的部落来讲,都是一处邪恶的禁地。

与此同时,香川武夫一路向上,找干尸快要找上了山巅。队伍里没有了无心,他便不敢再轻易的往地堡里进,地堡里有的是粮食物资,然而他们露营在外,夜夜都是冻得死去活来。据说金子纯很有在严寒北地生存的经验,可惜他死了,而且死前没来得及把他的知识传授给伙伴。营地夜夜燃着一大堆篝火,火烤胸前暖,风吹背后寒,小柳治有些后悔,认为自己当初不该让马英豪随行。

马英豪倒是不以为意,他双手捧着一杯热茶,人是坐在帐篷门口,后背在里前胸在外:``我一定要亲眼看看他的下场。''

``他''自然指的就是马老爷。他对马老爷的恨,不是三言两语可以尽述的。想让他放下仇恨,马老爷至少得赔给他一条健康的右腿。

整整两天的奔波过后,傍晚时分,伊凡再一次徒劳无功的回了营地。

马老爷,因为有求于他,所以有点怕他,不由自主的很谄媚,除了向他道辛苦之外,还出于本能一般,源源不断的做出承诺,又从身上搜出几张大额的钞票,要送给他。伊凡被他说得满脸迷茫。接过钞票看了看,他没看懂,又还给了马老爷,同时说了一句:``好看。''

马老爷拿着钞票,也是懵懂,没想到伊凡把钞票当画看。捏着钞票抖了抖,他伸着脑袋对伊凡说道:``钱,你不要钱吗?有了钱,才能去卖好东西呀!''

伊凡对着马老爷说道:``我有了皮子熊胆和鹿茸,什么好东西都换得来。你想要什么?''

然后他从一只铁皮罐子里挖出雪白的熊油,涂在列巴饼上去送给赛维。马老爷愣了愣,后知后觉的低声咕哝道:``我不要什么,我只是不知道你要什么。''

赛维心事重重的吃了伊凡递过来的列巴饼。她不大喜欢熊油的气味,列巴饼也是酸溜溜。一口接一口的咬嚼着,她想自己一家要把小野人吃空了。

小野人能有多大?二十来岁,大概和无心相仿佛,披散着一头阳光似的头发。对马家舍得奉献,也许只是为了要她。下意识的瞥了无心一眼,无心正在仰头喝酒。他是喝不醉的,身体对于酒简直不大吸收。伊凡因此很喜欢他,大口喝酒的人,不怕把自己喝醉的人,一定是坦诚的。

伊凡在山中太寂寞了,所以忽然有了客人,就很快活。天黑之后他点起了一堆火,给赛维烤了一只肥兔子,又拉着男人们跳舞——在他的部落里,他一直是出了名的爱唱爱跳。

马老爷和胜伊都委婉拒绝了,只有无心愿意陪他。无心明知道伊凡爱赛维,但是很奇妙的没有醋意,他看着伊凡和赛维,像是灵魂忽然倒退了千百年,居高临下的看着两个后人。他想自己还是不够爱赛维——爱是爱的,然而爱得不够;否则人的感情他都不缺少,他也懂得嫉妒的。

两个人站在火堆旁,无心很快就学会了伊凡的舞蹈。他们像两只笨拙的熊一样弯着腿,晃晃荡荡的对着摇摆跳跃。伊凡用一根细细的皮绳把头发绑成一束,一双碧蓝的眼睛湿漉漉的,带着醉意和情意,不时的瞟向赛维。马老爷含糊其辞的,总是不肯给他一句准话;他等了又等,等得醺醺然,不知道汉人的规矩,也不知道是不是汉人都不爱说明白话。

到了半夜,伊凡钻回仙人柱里睡了,其余人也都各回其位。他们不怕狼来,因为有驯鹿。如果狼敢偷袭,驯鹿会一蹄子把狼踢死。

胜伊侧身靠边躺了,闭着眼睛倾听外边的风动声,雪落声。

不远处的赛维和无心在悄悄的说话——不能总耽搁在山林里了,哪怕山下有日本人,也得走;或者是抢在日本人前头找出干尸,作为筹码和香川武夫谈条件。横竖在山里,大家都是外来客,全不占便宜。她看得清楚,香川武夫一行并没有携带电台;地堡里可能有电台,但是谁敢进地堡?只要香川武夫别招援军,那么谁有胜算,就不一定。

先头的话,还是正正经经。谈着谈着他们忽然安静了。胜伊知道他们在倾听自己的呼吸。

然后是一阵窸窸窣窣的响动,赛维``嗨''的轻笑一声,低低的说道:``抓住你了!''

无心嗤嗤的笑,笑着笑着回了头,轻声唤道:``胜伊?''

胜伊紧闭双眼,一动不动,同时就听无心对着赛维笑道:``睡了。''

塞维答道:``他睡得快——你别压我,让我先看看你,我还没有仔细看过呢!''

胜伊偷偷睁开了一只眼睛,跟着赛维一起看,看过之后闭了眼睛,第一次意识到自己是个小鸡仔。

仙人柱里起了风浪,无心的屁股就是雪白的浪头,一波一波的冲击着赛维。胜伊听到他姐喘得颤颤巍巍,还听到两人之间咕唧咕唧啪啪啪,两个屁股鼓起掌了。

于是他故意翻了个身,吓他们一跳。

天亮之后,伊凡早早的出了门,上午就回了来,对马家众人说道:``日本人在炸山!''

马老爷先还没听懂,深入的又问了问,才弄明白——山腰起了巨响和硝烟。巨响他们也听到了,但是当时不明所以,没有放在心上。此刻略想了想,马老爷望向赛维:``莫非\ldots{}\ldots{}他们找到了?''

赛维立刻摇了头:``不可能。如果找到了,何必还要上炸药?他们就不怕把干尸炸毁了?''

马老爷抬手摩着蓬乱卷发,沉吟不语。而伊凡见状,就说道:``我再去看一看。''

赛维听了,连忙向他一欠身:``别去!''

伊凡惊讶的看着她,很温柔的问道:``为什么?''

赛维张了张嘴,坐回原位说道:``危险,别去。''

马老爷一皱眉头,心想二姑娘怎么了?野人要去就让他去嘛,他不去谁去?

伊凡拉过一头驯鹿,还是要去。赛维坐在地上,心想他要是死在日本人手里,留下的食物和武器正好可以归自己所有,而且还免了其它方面的麻烦;大家这些天好吃好喝,也恢复了元气,就算没了野人,也一样能活。

可是眼看伊凡真要骑上驯鹿了,她又起了身:``别走!日本人无非就是发现了野兽或者毒蛇,不值得一看,你回来!''

伊凡牵着驯鹿,望着她傻笑。

赛维低头避开了他的目光,继续说道:``既然我们没有办法逃生,索性也去找干尸吧!如果找到了,不怕日本人不和我们谈判。干尸的另一半还在家里,我们凭着干尸,回了家再说!''

马老爷的两道平淡眉毛快要打结:``道理是这个道理,但是香川带着一支小队都找不到,我们活动范围有限,更是无从寻起呀!''

话音落下,常和胜伊亲近的、额头上带着一块白毛的大驯鹿从远处一瘸一拐的走了过来。伊凡迎上去弯腰一看,发现驯鹿不知跑去了哪里,后腿被蹭掉了一块皮,鲜血星星点点的洒了一路。

驯鹿也是今年刚上山的,因为不熟悉地形,所以偶尔受伤也是难免。伊凡沿着血迹走上驯鹿的来路,想要去消除那一处伤害驯鹿的隐患。而大驯鹿站在胜伊身后,舔了舔他露出来的一圈后脖颈。胜伊转过身,抬手搂住了鹿脖子。

脑海中浮现出仙人柱内的午夜风光,胜伊忽然生出厌倦情绪,不假思索的来了一句:``姐,我不想再去谈恋爱了,我往后就和驯鹿过吧!''

赛维瞪了他一眼,没想到他现在还有心情胡说八道。

伊凡在三棵老树之间,停了脚步。

雪地上陷下去一个小窝,显然是驯鹿蹄子踩出来的。小窝不是一般的土坑,伊凡蹲了,把手伸进窝里一摸,发现小窝居然是用尖角锐利的石头砌成的。

他起了好奇心,伸手去挖冰雪冻土,想要看看小窝是自然生成,还是有人故意在地下布了阵。可是只挖了一阵,他便目瞪口呆的傻了眼——在他刨出的小土坑里,他看到了一座用石头砌成的仙人柱的顶端。

开口是很小的,但是并没有被土填实,以至于驯鹿蹄子突破冰雪和土层之后,就可以陷入。从开口开始,越往下越大,像一把半合拢的雨伞。石头虽然嶙峋,但是一层搭一层,居然砌得很结实。往下不知还有多深,单凭着两只手挖,可是太费工夫了。

伊凡决定回去找无心来帮忙。赛维是女人,胜伊不是女人胜似女人,马老爷又是老人家,他数了一遍,就只有无心能用。

无心随着伊凡跑了过来,用木棍和铲子掘土深挖。其余人等也跟来了,啧啧称奇的旁观。土坑已经挖到一人来深,无心站在坑里直起腰,同时看到了坑外飘着的小健。

小健神出鬼没,时常是连着许久不见鬼影。无心看了他一眼,不知为何,感觉他飘的方位很顺眼。四面八方的环视一圈,他发现了顺眼的秘密——三棵老树加上小健,正好组成一个正方形。

一把夺过伊凡手中的铲子,他开口说道:``余下的活我来干,你上去。''

伊凡莫名其妙的看着他:``你以为我没有力气吗?''

无心叹了口气:``伊凡,我们好像挖到了一位巫师的坟。''

伊凡不安的动了动,刚要发问,不料在他一动之下,仙人柱上忽然脱落了一块半大不小的石头,砸得伊凡金鸡独立的向旁一跳;未等站稳,他骤然爆发出了一声惊叫。

脱落之处显出孔洞,半只蜡黄的人头垂落而出,枯萎的眼珠颜色混沌,定定的凝视着前方。

不等伊凡看清,无心绕过仙人柱冲上去,弯腰抱起伊凡就往上举,一鼓作气把他掀上了地面。随即转身伸手用力扒开仙人柱,他从里面拖出了半具蜷缩着的干尸。

\chapter{伊凡的爱情}

伊凡从小到大,一直是把禁山地下的巫师遗骸当成传说来听的,因为从来没有人会轻易上山,上了山的也大多只是不得已的路过,连停留都不会多做,更不会破土挖掘。

所以当看到传说成真,而真相又是如此恐怖之后,他吓得像是撒了癔症,坐在地上直哆嗦。坑中的干尸蜷缩成了一只蜡黄的大虾仁,身体切面却是平整。无心让地上众人后退了,自己带着干尸爬上地面。这一半干尸,和马老爷家里的那一半相比,仿佛是分别处在了两个极端。马老爷家里的半具干尸直挺挺硬撅撅,而且是闭着眼睛;而刚刚挖出的干尸却是抱着膝盖做胎儿状,一只眼睛也是睁着的。石头仙人柱被破坏了,外表已经是粗糙,里面更是棱角尖利。根据无心的知识,怪石垒成的仙人柱也许是象征着痛苦与禁锢。巫师把自己分成两半,一半安然沉睡着保护他的宝藏;另一半则是受着炼狱般的折磨,永远不见天日、不得伸展。

更深一层的道理,无心想不出了,目前仅有的一点学问,还不知他是从哪里听来的,他自己也不知道——前尘旧事一贯被他滔滔的遗忘,然而大浪淘沙,总会有片言只语留存在头脑角落里。

``就是它了!''他不让旁人看到干尸,干尸的身体里还封着魂魄,他怕旁人会受干尸的害:``去找块布,我要把它包好。''

伊凡爬起了身,一边喃喃的祷告,一边踉跄着往回走。无心抬起了头,看到小健远远站着,对自己不住的摆手。

无心从来没有留意过他,此刻一看,忽然发现他是个挺好看的小男孩,摆手的时候,稚嫩的手指微微弯曲,正是个很可爱的模样。

于是他就真诚的笑了一下:``怎么了?''

小健血淋淋的飘在树林里,声音在无心的耳中响起:``我害怕。''

无心答道:``别怕,我一会儿就画一道符,把它镇住。''

小健一点儿也不信任他的法术,于是一个影子越来越淡,末了就自作主张的消失了。

伊凡常年穿兽皮袍子,又不是小姑娘爱做新衣裳,所以营地里根本没有布。他慌里慌张的乱转一圈,一时想要带着马家人逃命,一时又想要回部落请萨满来帮忙。没等他想出眉目,他的双脚先行一步,已经带着一只半大不小的桦皮桶赶回了无心身后。

桦皮桶轻便结实,外层还印着花纹,上面也有个盖子。无心把干尸放进桶里盖严了,又用绳子上下捆了几道。拎起桶站起身,他见伊凡还是惊魂不定,就安慰道:``别怕,我也是位法师,我不怕鬼。''

话音落下,马老爷见缝插针的向无心一挑大拇指,同时对着伊凡说道:``他——半仙之体啊!''

伊凡惶恐的问道:``你也会跳神吗?''

无心来不及多说,泛泛的一点头:``啊,会!''

伊凡当即又道:``我有一头母驯鹿,总是生畸形崽子,你能不能跳神救它?''

无心睁大了眼睛:``我\ldots{}\ldots{}只会救人。''

伊凡很虔诚的望着他:``我们部落的索菲亚,好些年都生不出孩子,你能不能去看看她?''

无心拎着一桶干尸,十分为难:``我\ldots{}\ldots{}也不懂妇科。''

伊凡失望了:``哦,那你不如我们氏族的萨满。''

无心拎着干尸独自走路,不许旁人靠近自己。而马老爷有了筹码在手,精气神立刻就不一样了。一双老眼囊括了伊凡和赛维,他想自己也该让小野人认清现实了——他的女儿,岂是伊凡可以觊觎的?不过这话什么时候说呢?如果说了,小野人会不会像打灰鼠一样一枪崩了自己呢?毕竟四个人这些天连吃带喝,已经吃空了小野人的仓库。小野人枪法如神,他想杀人的话,必定一枪一个准,自己身边别说有个半仙,就算有个全仙,怕也是凶多吉少啊!

马老爷得意的开动脑筋,开始在心中掂对言辞,想要顺着小野人的性子,把话说明白了。

马老爷对于自己的智慧,素来很有自信,自认文可谈经论道,武可打爹骂娘,堪称是位文武双全的奇才,如果不是为了名利入了政界,专攻学问也能有所成就。古有孔圣孟圣,孰知他就学不成当今的马圣呢?他是见了天皇都能侃侃而谈的,不信说不老实一个小野人!

在回到营地之后,他为自己挑选了一处绝佳的说话地点——紧挨着一棵大树,如果伊凡敢动枪,他可以瞬间躲到树后,先逃一劫。

先对着赛维和胜伊交待了一番,马老爷整理了身心,鼓舞了勇气,然后把伊凡叫到树前,满面慈悲的告诉他:``我家的二姑娘,她不愿意啊!''

他抬手拍了拍伊凡的肩膀:``在北京,有个小伙子从小就很喜欢她,他们已经认识了十几年。她心里有了人,不肯再嫁给你。伊凡,强扭的瓜不甜,有缘无分,算了吧。''

话音落下,他做出一脸怜爱神情,观察着伊凡的反应。伊凡睁着一双蓝眼睛,傻乎乎的望着马老爷,仿佛忽然听不懂了汉话。

马老爷深知压抑之后的爆发更可怕,所以随时预备着往树后跳。微微皱着眉毛,他迎着伊凡的目光,似乎快要眼含热泪。忽然伸手拥抱伊凡,他像个最博爱的老人家一样,抬手拍了拍伊凡的后背,又在伊凡的耳边长长叹息了一声:``好小伙子,可惜我没有第二个女儿了。赛维是我的独生女,我娇惯了她十几年,我不舍得逼迫她啊!''

伊凡垂下了头,喃喃说道:``你不要逼迫她。花儿在原野上,才能一年又一年的开放;如果把它们折断插到花瓶里,它们很快就会死了。赛维有了喜欢的人,就让她去喜欢吧。如果强迫她嫁给我,她也会像花一样枯萎的。''

马老爷一愣,心想这小子什么意思?是在说漂亮话吗?

几分钟后,马老爷亲亲热热的陪着伊凡坐下了。

马老爷一辈子不懂得什么叫做愧疚,可在看出伊凡的一言一行全部发于赤诚之后,他真有点愧疚了。不由自主的,他又许了大愿:``等我回到北京了,我会派人来接你。我在北京过得还不错,能让你也享享清福。''

伊凡摇了摇头:``你们都住在房子里,窗户和门还要关着。夜里黑黑的,没有星星月亮,没有风和雪,也没有火。我想一想都受不了,我会在里面活活闷死的。''

然后他站起身,从仙人柱里翻出一把草药送到嘴里嚼烂了,吐出来敷在了大驯鹿的伤腿上。大驯鹿并不把一点小伤当回事,低下头轻轻啃着雪下的草。伊凡的蓝眼睛里满是忧伤,目光注视在白皑皑的大地上。

马老爷带着一点小愧疚和满身的大轻松,拍着屁股上的雪起身走到赛维面前,一是汇报胜利消息,二是商议如何利用筹码。

赛维听了马老爷带来的喜讯,不知为何,完全喜不起来。看着伊凡在远处伺弄着大驯鹿,她心里很不好受——如果伊凡胡搅蛮缠的大闹一场,反倒更能让她心安理得。回头又看了无心一眼,她发现无心守着桦皮桶,也是坐得很远。

``得找个人去做联络员。''她对马老爷说:``而且我们人少,得格外小心。''

马老爷沉吟着点头:``没错,不见兔子不撒鹰。玩命的事,不能不慎重。可是让谁去做联络员呢?你我不行,胜伊更是屁用没有。你那个半仙还得照顾尸首\ldots{}\ldots{}''他暗暗的向后一指,压低声音问道:``要不然,让他去吧!''

赛维立刻摇了头:``爸爸,他上山就是为了躲日本人,我们现在怎么能把他往日本人眼前推?''

她低下头,望着自己的两只手,手指细长,指甲也长了,分外的像爪子:``我去吧,至少我能把话说清楚。再说他们要的是干尸,又不是我。干尸不到手,他们不能杀我。''

马老爷一瞪眼睛,鬼鬼祟祟的质问:``万一把你扣下了呢?''

赛维也瞪了眼睛:``我们又不是勒索他,他们扣什么呀!我就告诉他们,说是干尸找到了,让他们带着我们下山回家。他们是为干尸来的,你说他们放着光明大道不走,非要杀了我再找你们,绕着弯的抢干尸吗?就算他们放不下坏主意,他们可以路上下手嘛!可等真上了路,兴许我们半路就跑了呢!''

马老爷张着嘴想了想,末了抬手拍了赛维一巴掌:``好姑娘,脑子够用。你要是个男孩就更好了,你要是个男孩的话,会像爸爸一样了不起的。''

赛维毫不遮掩的打了个哈欠,不爱听父亲说话。

无心听说赛维要回地堡去找日本人,当即表示不同意。

他让伊凡留在营地不动,自己则是在距离营地一里远处找了块平地,把新搭建的仙人柱拆了搬迁过去,让马家三人和伊凡既有联系,又不至于被日本人一网打尽。把桦皮桶放到马家的仙人柱外,他让赛维和胜伊盯紧了它,然后自己单枪匹马的往林外走去。

伊凡要把大驯鹿借给他骑,他没要,把驯鹿放养在了马家的仙人柱附近,因为怕自己夜里回不来,马家众人会被狼叼走。大驯鹿代替了他,就算是暂时的保镖了。

\chapter{奉献}

无心当初从半山腰往树林里逃时,因为是个顺风下坡,而且后有追兵,所以还不觉怎的;如今顶风冒雪的往山上走了,他在没过脚踝的积雪中一步一顿,感觉自己的耳朵都快被逆风齐根刮掉了。

他累极了,自己弯腰抓了雪往嘴里塞,心想自己早在几个月前还抱怨日子了无生趣,没想到紧接着就被卷进了偌大的漩涡。事已至此,他显然是和马家有点缘分,既然有缘,就帮忙帮到底、送佛送到西吧!至于将来赛维会不会要他,他倒是淡了。最好是要,不要也行。不要他,他就走。反正也是累透了,他真想找个地方歇一冬。

半路上,他把小健逮住了。

小健藏在他的后衣领里,问他:``马俊杰呢?''

无心实话实说:``死啦!''

小健气死了,当场飘到半空,张牙舞爪的撒泼:``让你给我看着,让你给我看着,结果你给我把他看死了!你赔!''

无心四脚着地的向前爬坡:``小健,你应该投胎去,孤魂野鬼做久了,苦头在后面呢!''

小健把一张鲜血淋漓的小脸凑到了他面前:``你想撵我?''

无心对着他吹出一口热气:``你看我,我就是个孤魂野鬼。孤独一天两天没关系,一辈子两辈子也没关系,可是没头没尾的一直寂寞着,就难熬了。你要是愿意,我可以让你魂飞魄散,像其他人一样。''

小健没说什么,悄悄的消失了。

无心到达半山腰时,发现除了马英豪和小柳治之外,香川武夫也在。几天不见,他们全有点小变样,因为剃刀落在了地堡,他们没法刮脸了。

忽然见他回来了,众人全都目瞪口呆的起了身。小柳治下意识的拔出了手枪,香川武夫则是探究式的向他一歪脑袋:``无心?''

无心开门见山的说道:``你们找到干尸了吗?''

香川武夫的一个脑袋左摇右晃,一双眼睛却是紧盯着他:``还没有。''

无心看见地面火堆上吊着一只铁锅,锅里咕嘟咕嘟的煮着肉,便径自走过去坐下了,抄起一把长柄勺子搅动肉汤:``你们找不到了。''

香川武夫的目光始终是随着他走:``为什么?''

无心感觉自己的血液都冻出了冰碴,于是舀起一勺热汤喝了:``因为干尸在我手里。''

香川武夫一挑眉毛,走到他身边蹲下了:``你,还是你们?''

无心一边喝汤,一边答道:``没有区别,一个意思。''

香川武夫审视着他的面孔,没看出什么端倪来,于是继续问道:``怎样才能把干尸给我们?你可以提出条件。''

无心摇了摇头:``没条件。我帮你们保存干尸,到了北京自然会给你们。''然后他抬头对着香川武夫一笑:``我要它没有用嘛,它又不好吃,对不对?''

话音落下,他捞起一块肉塞进嘴里。

香川武夫一皱眉头,感觉无心的话真是令人作呕。不过两道眉毛随即舒展开来,他和颜悦色的对无心说道:``我们并不打算立即返回北京。''

无心心中一动,有了不好的预感:``为什么?''

香川武夫盯着他答道:``北京马宅的干尸,已经被我运进地堡里了。''

此言一出,马英豪和小柳治全变了脸色。无心攥着长柄勺子,恨不能在香川武夫的光头上狠敲一下。可是嘴里咀嚼着肉块,他强忍着不动声色:``我怎么不知道?''

香川武夫微微一笑:``因为我也害怕诅咒,我也认为干尸是不吉利的,所以为了保证我们队伍的安全,上面特地组建一支小队专门运送干尸。他们比我们晚了一步,也的确是一路坎坷。''

说到这里,他向无心探了头,见神见鬼的压低了声音:``它的确是不祥的。还记得小队把它送入地堡的那天夜里吗?那天夜里,金子纯死了。''

然后他一摊双手,咬着牙笑:``多么可怕的巧合。''

无心不置可否的慢慢喝汤,心想马老爷和赛维的算盘全打错了,原来日本人是打算就地解决所有问题。

把一小锅汤喝到见底了,他在温暖的汗意中放下勺子,抬头望向了香川武夫:``如果巫师的灵魂当真复活,我是没有办法的。''

香川武夫对着马英豪一点头:``还有白琉璃。''

无心环顾四周:``诸位,你们都相信鬼神之说吧?''

马英豪和小柳治一起点了头,他们的启蒙老师是白琉璃。香川武夫则是沉吟——他一直暗暗的把诅咒和微生物学联系到一起,也许因为诅咒死去的人,只是感染了某种不为人知的细菌或病毒。但无论真相到底是什么,他都很愿意做一番探索。如果诅咒是真的,那么能否将其利用在战争中呢?稻叶大将并不是没有见过古董的乡巴佬,即便古董非常之古。之所以大动干戈的派他前来大兴安岭,乃是醉翁之意不在酒。

无心又道:``你们有没有想过,如果白琉璃不是巫师的对手,我们会落到什么下场?''

香川武夫淡然的答道:``无非是死。帝国的军人,不怕死。''

无心转向了马英豪:``你也不怕?''

马英豪显然是别有心肠,答非所问的说道:``马浩然在哪里?''

无心莫名其妙:``马浩然是谁?''

马英豪不情不愿的答道:``老不死的。''

无心恍然大悟:``哦\ldots{}\ldots{}''

然后他就闭了嘴,并不打算再提马老爷。马老爷心眼奇多,虽然百分之九十九都是坏心眼,但是聊胜于无。如果想凭着一己之力返回北京,队伍里还真少不得狡猾的马老爷。

香川武夫拿起长柄勺子,轻轻磕打着铁锅锅沿:``无心,我知道你的顾虑。放心,我不会伤害马家的人。只要我们最后活着,就一定会安全的带他们回北京。''

无心望着空锅,锅沿晃动着香川武夫的大手,手背青筋毕露,活生生的带着热量。抬手搭上香川武夫的手背,他忽然满怀悲悯的长叹了一声。一条愚痴而又执着的生命,在向着夭折的方向狂奔。

无心很少恨谁,此刻抬头环顾了周遭一张张肮脏而又年轻的面孔,他虽然知道他们杀人不眨眼,是人中的恶兽,但也依然没有恨意。

无心收回目光,不再深想。再想下去,他的心就要苍老了。

香川武夫则是不大得劲的捏着勺子。无心无端的摸了他的手,让他躲也不是,不躲也不是,心中不禁暗想:``我都奔四十了,值得一摸吗?''

没等他的念头闪过,无心轻轻拍了拍他的手背,然后说道:``天堂有路你不走,地狱无门自来投。我拦不住你们,我也不拦了。明天早上我还来,带着半具干尸。今天晚上\ldots{}\ldots{}你们喝点酒,吃点肉,好好过吧!''

在火堆前站起身,无心拍了拍身上的雪:``我走了,不要跟踪我。我说话算话,明早一定回来。''

无心沿着山路往下走,身后果然没有日本兵尾随。

半路上他捡了一块大树皮。人坐在树皮上,他顺着斜坡向下滑。滑雪的速度快极了,风声在他耳边呼呼的响。赶在天黑之前,他进了树林。拐弯抹角的快走一气,他找到了马家三人的仙人柱。

马老爷正在仙人柱内火塘上烧开水,赛维和胜伊则是相对而坐,猫头鹰似的守着中间的桦皮桶。忽然见他全须全尾的回来了,三个人全是雀跃不已。然而听到了他的报告之后,三个人又一起蔫了。

赛维问无心:``你怎么知道人家一定是要把干尸送进地堡里呢?他们就不能把地堡里的干尸运出来吗?反正就是把两半拼在一起,在哪里拼不是拼?''

无心反问:``在哪里都能拼,可是谁去把地堡里的一半运出来呢?除了我之外,还有人敢下地堡吗?''

胜伊低声说道:``反正太悬,死瘸子扔在地堡里的什么琉璃,恐怕早被毒蛇吸成人干了,到时候他们是一伙的,就你单枪匹马,你有胜算?''

无心故意笑道:``大不了我也死在里面,你们自己想法子回家吧!''

胜伊当即狠狠的瞪了他一眼,赛维则是鼻孔出气,老气横秋的扭头骂道:``说他妈的屁话!''

马老爷其实并不关心无心的死活,但是看儿女都愤慨动情了,自己不好太过冷漠,只好也跟着摇头:``无心,不要乱说。''

无心笑了:``我说着玩呢!我又没疯,当然不会去送死。我有办法,我是半仙嘛。马英豪扔在地堡里的巫师,本领也不如我。放心,一队人里顶数我最厉害,我一定能活着出来!''

四个人胡乱争论了一夜,末了在仙人柱里挤着睡成一团。

天亮之后,无心吃饱喝足了,拎起桦皮桶就要上路。临走之前赛维赶上去,扳着他的脑袋亲了一口。

无心微笑着走了,走出老远他回了头,见赛维和胜伊并肩站在仙人柱前,还在直勾勾的望着自己。抬起手用力的挥了挥,他转向前方,踏上了去路。

\chapter{魂兮归来}

无心走到半山腰,在地堡入口前打开了他的桦皮桶。香川武夫无所顾忌的上了前,瞧过之后点了点头,心想人真是有命也有运的,自己踏破铁鞋无觅处,无心得来全不费工夫。

心情无端变得沉重了,他请无心再进地堡,取出干尸,无心什么都没说,只是摇头。

香川武夫不敢太勉强他,于是转向小桥惠,用日本话低声说道:``你留在外面吧。如果发生万一之事,你立刻返回天津,把我们的所作所为,原原本本的汇报给稻叶大将。''

无心并不懂得日语,但是猜出了香川武夫的意思,所以当即说道:``所有人都要下地堡。活人越多,阳气越重,越能克制阴魂作祟。''

香川武夫没有多想,对着无心解释道:``她是个女人,用处不大。''

无心扭头看了小桥惠一眼,看她是个缩手缩脚的小女人。山下林中也有个小女人,为了那个小女人的活,他得让这个小女人死。

``不行。''他斩钉截铁的说道:``她必须下。''

香川武夫有心拔枪恐吓无心,但是一转念,又觉得没有必要。总而言之,他们来得太仓促,全怪稻叶大将催命似的催他出发。许多该做的准备都被省略了,他环视了身边二十来名士兵,旁人倒也罢了,只是金子纯的死,真是大损失。

现在后悔是来不及了,想要和外界联络,电台又在地堡里面;派人用两条腿往外走,一来是时间不足;二来大雪封山,未必能走出去。香川武夫又望向小柳治,他和小柳治一点儿都不熟,也根本不认识马英豪。稻叶大将把队伍搞得东拼西凑,像一件首尾不能呼应的残次品。如果从头开始就让他来经手,绝不会落到今天这般境地。

思及至此,香川武夫几乎有些愤慨了。手指缓缓划过缠在腰间的子弹带,他的光头反射了朝阳的光芒。

无心好整以暇的观察着所有人的表情。但凡这些人存有一丝的理智,都该马上收拾行装往山下跑。可他们已经上了无形的轨道,前途是注定的了。耳边忽然响起了小健的声音:``大哥哥,我来了,我给你做侦察兵,好不好?''

无心点了点头,心想等到这次脱身自由了,无论如何都要让小健魂飞魄散。小健是个小孩子,不懂事,趁着他还没有很痛苦,自己做主,让他解脱了吧!

这时,香川武夫已经走去打开了地堡铁门。

一名全副武装的日本兵和无心率先下了洞,领着头往地堡里走,后面的人络绎跳下,是一条长长的大尾巴。无心向前走了一段,忽然回头向后望去,同时嘴唇翕动,一五一十的清点人数。点到最后他迈步走到队尾,从入口伸出头去,面无表情的望着站在地面上的小桥惠。

小桥惠没想到他会折返回来,不禁愣了一下。从她的角度往下瞧,只能看到无心半张面孔。半张面孔是冷森森的白,眼睛陷落在眼眶里,黑的几乎不见眼白。小桥惠冷冷的注视着他,看他像个魔鬼。

无心和她对视片刻,末了一招手:``下来!''

小桥惠面无表情,俯身跳进竖井,从无心身边挤进了地堡。

无心转身走向队伍前头,一边走一边低声说道:``事到如今,各安天命。你们还闹什么?''

香川武夫从昨天开始,就听他说话句句都不对劲,越琢磨越是让人心惊。不甚自在的清了清喉咙,他开口说道:``我们就直奔目的地吧!''

无心拎着桦皮桶,无精打采的答道:``好。''

小柳治问道:``蛇\ldots{}\ldots{}没了?''

甬道里的确是挺干净,完全没有黑蛇的踪影。蛇的有无,显然不是人可以回答的问题。所以队伍里无人反应。香川武夫挥舞着手电筒辨认了方向,紧接着带队伍拐上了主干道。刚刚走出不远,他骤然停住脚步一皱眉头——地上赫然摆着一副长大的骨架,骨骼赤红,还有血肉存留。

高抬腿轻落脚,他跨过骨架继续走,每一步都像是走在奈何桥上,因为不能预料会不会有黑蛇蹿出咬人。目不斜视的经过了指挥所,他继续前行,最后转进一条岔路,岔路尽头正是一扇铁门。

香川武夫把手电筒给了身边士兵,一边摸钥匙一边问道:``白琉璃在哪里?''

马英豪和小柳治面面相觑,统一的认为自己是养了条白眼狼。

无心拎着桦皮桶,忽然爆发似的大吼一声:``白琉璃,我要死了!''

远处传来了轻飘飘的回答:``骗子,你活得好好的。''

马英豪万没想到白琉璃居然就在附近,气得漫无目的的骂道:``白琉璃,你没良心!自从我把你送进地堡之后,你有没有再见过我?整整一年啊,我养你不如养条狗!''

小柳治连忙一扯马英豪:``哎,不要激怒了他。''

白琉璃没了声音,显然并未被马英豪激怒。

香川武夫把钥匙插入锁孔,开始旋转开门。无心又道:``白琉璃,你小心着。我可要把两半干尸拼成一体了。''

话音落下,铁门暗锁咯噔一响。香川武夫捏着钥匙往外拽门。铁门沉重,开得吱吱嘎嘎。后方的小柳治用手电筒向内一照,就见室内空空荡荡,只在中央摆了一口棺材似的木头箱子。

香川武夫没有贸然进去。抬手摁了摁贴胸口挂着的护身符,他双手合什举到眉心,喃喃的念了几句佛。后方众人有样学样,也跟着双掌合十拜了拜。

迈步进了房间,香川武夫停在门口,对着身边的无心说道:``木箱的盖子是活的,可以掀开,里面就是\ldots{}\ldots{}那个。''

无心没言语,缓缓举起了手里的桦皮桶,然后转动眼珠望向了香川武夫。香川武夫的面孔渐渐扭曲了,因为看到桦皮桶正在隐隐的抖动。桶中发出细不可闻的声音,是干尸在和桶壁互相碰撞。

``它、它要活了?''香川武夫难以置信的问无心:``它会活?''

无心摇了摇头,向前走去:``我不知道。''

所有的手电筒都打开了,光线重叠着射向房间中央。无心弯腰放下桦皮桶,然后单手掀开了箱盖。长方形的大木箱里,长条条的摆放着半具干尸。光照之下,干尸的质地似乎有些异于先前。无心俯身去摸,发现干尸的皮肤竟然变得潮湿柔软了,像是将要腐朽的皮革。

转身揭开桶盖,他想要把桶中的干尸捧出,可是触及之处一片黏滑,干尸坚硬的关节也松弛了,蜷缩着的一臂一腿像是刚刚解冻一般,随着无心的动作变化形状。

又向香川武夫等人望了一眼,无心垂死挣扎似的又问一句:``我开始了?''

香川武夫恐慌而又兴奋的注视着他:``请吧!''

无心面对木箱,把手中半具干尸小心放置进去。禁锢在干尸里面的魂魄激烈的流转闪烁,凑成的完整身体则是越来越湿软,是在眼看着腐烂。四面八方的空气全乱了,成形不成形的鬼魂全被冲击成了零碎魂魄。无心忽然想起了小健,连忙喊了一声:``小健快跑!''

此言一出,香川武夫等人一窝蜂的退了出去。无心再看木箱中的干尸,发现干尸已经腐烂到了不分骨肉的地步。远方传来一声轰隆巨响,不知道是出了什么事故。一个头角峥嵘的幻影缓缓浮现在了暗中,无心扭头去看香川武夫等人,发现他们正在惶惶然的东张西望——显然,他们看不到室内的鬼影。

忽然,半空中起了一声沉闷的鼓响,震得鬼影一颤。

鼓声接二连三的密集了,鬼影仿佛落在水面上,忽明忽暗的始终不能稳定。无心紧闭了嘴,知道是白琉璃在救人。别人可以不管,马英豪他是不会不管的。现在他没法去阻拦白琉璃,只希望巫师死后法力尚存,别被白琉璃压下一头。可白琉璃若是落了败,恐怕也难逃一死;而他倒是没打算把白琉璃也一勺烩了。

无心很想和白琉璃交流一番,于是不动声色的向外退去,想要觅声去找白琉璃。不料他刚刚出了房间,腕子忽然一紧,却是被马英豪抓了住。

无心挣了一下,没挣开。马英豪气喘吁吁的问他:``你要往哪儿跑?''

无心扭头和他打了个照面,当即想起前尘旧事,恨不能当场咬他一口:``我找白琉璃去!''

马英豪到目前为止,和其他人一样,连个鬼渣都没看见,但是莫名的很心慌,从头到脚全不舒服:``去是可以,但要带上我和小柳!''

无心没空和他废话,于是深深的弯下了腰。下一秒,在马英豪的惨叫声中,他抽出手腕,转身冲进了黑暗。

小柳治吓了一跳,把手电筒直接转向了身后的马英豪:``你怎么了?''

马英豪抬起一只血淋淋的手,气急败坏的怒道:``妈的又被咬了!''

小柳治知道他左手虎口带着伤,此刻旧伤未愈,又填新伤,便要伸着脑袋细看。哪知还未看清,前方的香川武夫凑趣似的,冷不丁也吼了一嗓子,吓得小柳治手一哆嗦。眼皮一抬,他发现马英豪望着前方,也直了眼睛。

顺着众人的目光望向室内,他在一阵阵的闷响之中,就见空旷房间的四壁慢慢裂出许多缝隙,而无数小黑蛇争先恐后的游曳而出,竟然自动的扭绞盘旋,组成了一个高大的人形。

未等人形彻底完成,香川武夫摸出一只手雷,没头没脑的往室内一掷,随即吆喝着往后跑。众人都是聪明的,不消命令,自动的一哄而退,顺着来路就往回逃。然而没有逃出多远,速度最快的香川武夫又吼上了。

一块不知从何而来的大石摆在甬道上,彻底堵住了他们的生路!

马英豪因为腿瘸落后,此刻反倒容易撤退,占了先机。可在回头寻找新的岔路之时,他眼前一花,忽然看到了马俊杰!

马俊杰还是往昔的模样,穿着一身齐齐整整的小西装,笔直的站在远处路上。四周那么黑,他却是清清楚楚的仿佛放了光。

马英豪抬手揉了揉眼睛,以为自己产生了幻觉。揉过眼睛再向前看,马俊杰凭空的消失了,只留给了他一眼冷笑。

马英豪有些腿软,拉着小柳治说:``我看到——''

小柳治没空理他,眼看香川武夫就近撞开了一扇房门。他扛起马英豪,随着众人横挤了进去。房门咣当一关,香川武夫三下五除二的上了暗锁。背靠墙壁喘了几口粗气,香川武夫问道:``谁带了冲锋枪?不要步枪,要冲锋枪!''

日本兵的武装,素来是以步枪为主,所以此刻一起摇头。还是小桥惠冷静的说道:``军火库里有冲锋枪,也有子弹。''

香川武夫狠狠的呼出了一口气——军火库太远了!

与此同时,无心在臭气的引领下,在一处阴暗角落里找到了白琉璃。

他蹲在白琉璃面前,急三火四的说道:``你跟我走,我们想办法逃出去!否则我没有死,你先死了!逃出去之后,你还是回你的西康吧,吃大户的日子不是挺好过的?等我有了钱,我再还给你点儿,不就得了?''

白琉璃正在前仰后合的念咒,听了无心的话,他不耐烦的一挥手:``别烦我,它要来了!''

无心是非常的不怕鬼,所以听了这话,他转身就跑,想要去看一看巫师鬼魂的真面目。哪知刚刚跑过一条甬道,他便看到了黑黢黢的人形。

在分辨出了组成人形的一条条蠕动黑蛇之后,无心咽了口冰凉的唾沫,随即像条落水的四脚蛇一样,一摇头一摆尾,倏忽间就不见了。

无心像离弦之箭一样冲回白琉璃面前,也不多说,绕到身后一把扯住他的后衣领,拖了他就往岔路上跑。白琉璃念咒念得正酣,冷不防被他拽了个东倒西歪,险些咬了舌头。

\chapter{道不同}

无心弯腰拖着白琉璃快跑,白琉璃很沉重,是累累赘赘的一大堆;无心跑着跑着就要回头瞧他一眼,想扔了他不管,可是手指犹犹豫豫的要松不松,一副心肠总不能坚硬到底——一旦松了手,世上就再也没有白琉璃了。

无心和白琉璃就没有过情投意合的时候,连朋友都做不来。但是想到白琉璃也许会彻底没了影,他的手指迅速张开又收紧,在对方的后衣领上结结实实的抓了满把。有毛茸茸的活物蠕动过了他的指尖,他来不及管,慌不择路的乱窜一通,最后竟是一头扎进了先前住过的指挥所里。

指挥所的房门被手雷碎片崩走了形,但是勉强也能关拢。无心小心翼翼的关严房门,房门本是有暗锁的,暗锁如今失灵了,只剩插销还可以用。

铁门下方翘了一角,露出的孔隙,容得一只大耗子进出。无心趴在地上,用额头堵上了孔隙。走廊黑成一潭死水,他想要凭着感觉去确定敌人的方位。

白琉璃东倒西歪的坐稳当了,因为方才受了剧烈的颠簸,所以气息在胸腔里乱成了旋风。深深的俯下身去,他上气不接下气的咳嗽不止,咳着咳着气不够用了,他没了声音,只剩一个臃肿的身影在不停的颤抖抽搐。

无心没有发现敌情,于是有了闲心回头去看白琉璃:``巫师的灵魂真复活了,它用吸血的黑蛇组成了一个蛇人,当做新躯壳。''

白琉璃哑着嗓子低着头:``蛇人?''

无心警告似的向他竖起一根手指:``你小心点,蛇人可是决不能触碰的。''

白琉璃忽然一抬头:``无心,我想要巫师!''

无心望着他,随即明白了。白琉璃在邪术一道上几乎可以算作全才,除了蛊术之外,他也非常善于炮制大鬼小鬼。如果巫师的灵魂被他收服,他自然有办法将巫师的能量化为己用。

迎着无心的目光,白琉璃歪着脑袋偏着脸,从乱发之中露出一只蔚蓝的眼睛正视他:``总是苟延残喘的活,我也腻了。要么我杀了巫师,要么巫师杀了我。我也得一个结果。''

无心移开目光转向房门,同时轻声说道:``别闹。''

白琉璃伸手在身边四周摸索,摸到了一只变形的手电筒。手指轻轻拨动开关,手电筒内的小灯泡发出一点微弱的橘红光。白琉璃不敢正视灯泡,所以只百无聊赖的用它照了照无心的背影。无心的背影,他也看不大清了。无心骗了他三百英镑,从来没有人敢骗他的,但是无心就骗了。骗了,他也没办法。无心怎么杀都杀不死,他并不是没有杀过。既然无论如何都杀不死,就算了,不杀了;虽然偶尔想起往事,还是很伤心,很生气。

怀里沉甸甸的,是他的小儿子。手电筒的小灯泡熄灭了,他把手伸到怀里,摸了摸儿子的小脑袋。儿子是一团剧毒的肉,为了不让婴尸腐烂,他每天都用毒虫涂抹儿子的身体。一生中真是没有什么高兴的事情,不是被朋友骗,就是死了儿子。白琉璃摇了摇头,不肯再想。想多了,他会闹自杀。

无心撅着屁股跪伏在门前,一边留意着走廊情形,一边轻声说道:``白琉璃,不该管的你别管。你以为他们逃出地堡之后,还会再留着你养着你吗?''

白琉璃隔着层层肮脏兽皮,揉搓着怀里的儿子:``他们的事情,我不管。你心地不好,看谁都是坏人。''

无心忽然举起一只手,示意白琉璃闭嘴。外面走廊里有动静了,是沉重的躯体在地面磨蹭。蹭着蹭着忽然转为轻盈,无心把面颊贴上地面,用一只眼睛向外窥视。黑黢黢的影子弯着腰飞快掠过了他的视野,是一名日本兵背着两支长枪,正在仓皇的往前跑。脚步声音忽然一僵,无心听到了一声低低的``呃''。

紧接着是铿铿锵锵的几声响,是长枪落在了水泥地上,日本兵显然是遭遇了不测。无心向后一撤身,转向白琉璃低声说道:``又来蛇了。你往后退,别碍事。我找点东西,把门缝堵死。''

白琉璃的呼吸声音近在咫尺,可见一定听得清他的言语,然而纹丝不动。无心急了,又不敢高声大叫,只能憋足了力气斥道:``白琉璃,挪一挪!要命的时候到了,别添乱!我告诉你,我最怕疼。如果一会儿遭了蛇咬,我先揍你一顿!''

白琉璃气若游丝的答道:``你喷了我一脸口水。''

无心当即闭了嘴,想要酝酿一大口唾沫,直接啐飞白琉璃。然而白琉璃不给他机会,径自窸窸窣窣的向前挪去。移动之时他依旧深深俯身,右手伸长了拍在水泥地上,缓缓拂出一个半圆,末了掌侧向外对准孔隙,他无声的翕动嘴唇,用气流诵出了咒语。忽然抬手狠狠一拍地面,他猛然起身,从怀中不知掏出了个什么活物,向前伸直手臂运力攥紧了。掌中响起低低的破碎声音,浓稠的黑血顺着他的指缝,点点滴滴落上了地面。

将黑血在孔隙前方滴成一道弧线,白琉璃仰起头,长长的吁出了一口气。

无心像个鬼似的,声音绕过白琉璃的后脖颈往耳朵里钻:``有用?''

白琉璃忽然有些生气:``不相信我就滚出去!''

无心沉默了,沉默的原因不是力不能敌,而是太有胜算,不想让白琉璃气急败坏。及至白琉璃转身爬回到他身边了,他才小声劝道:``你听我一句吧,我又没有骗你的瘾。别管马英豪了,我想办法带你逃出地堡。逃出地堡之后,我再送你去医院治治眼睛。你放心,我怎么着都能弄到六百英镑还给你。到时候你有了钱,还是回西康。在西康当个财主娶个太太,多好啊!''

白琉璃冷静的答道:``我并不留恋尘世的繁华。''

无心皱着两道眉毛看他:``怎么个意思?''

白琉璃不看他,看也看不清:``我活不久了,可是巫师的力量如果能够为我所用,也许我还能再撑个三年五载。''

无心起身爬向了房门,同时头也不回的说道:``白琉璃,我不和你说了。说老实话,你现在真是没什么可骗的了。你要是有,我还得再骗你一次。''

白琉璃明白无心的意思,可就是不想和他合作。他认为无心是个坏人,他真的怕无心再骗自己。

无心身在暗室,无天无日的没有了时间概念。与此同时,香川武夫等人用手电筒照着手表,却是心中有数。

他们已经在房间里困了小半天了。

三名士兵一同潜出去寻找武器,结果只活着回来了一个。活着回来的扛了两支冲锋枪,子弹却又有限。据说他们三个刚一出军火库就走散了,各走各的路,谁也见不着谁。余下二人始终不归,显然是途中遇难了。

香川武夫掂量着手中的冲锋枪,因为平时很少用,所以握在手中不大习惯。小柳治上前也拿起一支,随口说道:``早知如此,我们不如把干尸运回北京再做拼接。''

香川武夫听他出言涣散军心,当即反驳道:``胡说八道!实验总是要有人来做的,难道把干尸运回北京,把责任推给别人,就万事大吉了吗?''

小柳治闭了嘴,不声不响的又去检查子弹。香川武夫则是继续说道:``我们不能再躲了,躲得时间久了,我们即便什么都不做,也会又饥又渴,失去战斗力。''

因为干电池也有限,所以房间里只亮了一支手电筒。香川武夫瞟了马英豪一眼,随即对小柳治吩咐道:``把冲锋枪交给小桥,你去保护你的朋友。据我所知,地堡还有几处出口,只是没有完工。凭着我们的力量,是可以把最后的土层挖开的。距离我们最近的出口,是在五条岔道之外。趁着现在外面没有动静,我们立刻试着冲一次。''

小柳治仿佛是有点不服气,可是回头看了马英豪一眼,他无可奈何,只能把冲锋枪交给了小桥惠。小桥惠用一根带子拦腰扎紧了皮袄,两边袖子也是挽得整整齐齐。动作娴熟的接了枪,她是一如既往的一言不发,面无表情的低头清点子弹数目。

香川武夫和小桥惠,成了队伍中的主力军。其余士兵各司其职,有开路的有殿后的,还有专门举着手电筒照明的。小柳治一手握着手枪,一手攥着马英豪的手臂,必要时决定拖着他跑。

房门一开,众人络绎而出,一路走得无声无息。眼看已然经过两条岔路口了,一名士兵惊呼一声,紧接着香川武夫向前开了火。在密集的枪声中,前方路上缓缓现出一个漆黑高大的人形,正是由巫师灵魂支配着的蛇人!

子弹扫过蛇人,死蛇脱落,立刻又有活蛇补充。黑蛇们绞在一起,是不生不死的一个整体。人形的步伐并没有声音,可是所有人都一起心烦意乱了,不但恐慌,而且想哭。小桥惠忽然嚷出一句日本话,香川武夫随即一挥手,在蛇人逼近之前带队拐进岔路。未完工的地堡里,可供隐蔽的房间实在是太多了。小桥惠一枪崩开一扇铁门,正要往内进入,不料身后忽然起了一声低低的哀鸣。众人一起回了头,就见一名落后的日本兵扔了手中步枪,面颊、脖子和手背,分别附着了一条黑蛇。张大嘴巴望着战友,他把五官扭曲到了极致,做出一个惊惧已极的表情。不知是谁用手电筒照向了他的面孔,人们就见他年轻的头脸迅速枯萎,只剩两只圆圆的眼珠凸出眼眶,脖子也细成了一把骨头。小柳治见他摇摇欲坠的似乎还要往队伍里扑,立刻抬手一枪,把他打得向后仰翻。香川武夫则是摸出手雷掷向蛇人。在轰鸣如雷的爆炸声中,他们一窝蜂的又进了一间空屋。

在亲眼目睹了蛇人的形象之后,再也没有人能够保持镇定了。门锁坏了,香川武夫用后背顶住房门,直着眼睛就只是喘。马英豪靠在小柳治的身上,用中国话低声说道:``完了完了,真是祸害活千年!他没有死,我先要死了。''

小柳治忽然想起了一个问题:``无心和白琉璃在哪里?他们是阴阳师,他们一定会有办法!''

没有人能回答他,无心已经是没了影,白琉璃更像是传说中的人物,似乎就只有小柳治和马英豪能够确定他的存在。

\chapter{鬼吃鬼}

两名日本士兵双脚蹬地背倚房门,死死的向后把门顶严。香川武夫已经清楚察觉到了自己的崩溃,可他是不能崩溃的,他崩溃了,整支队伍都会随之一起崩溃。双掌合十举到眉心,他笔直的面对墙角站立了,嘴唇翕动着念佛,念得无声无息而又滔滔不绝。心跳渐渐合了佛经的节奏,他紧锁眉头,汗湿的双手从僵硬恢复了柔软。

面前忽然响起了轻微的破裂声,同时步枪枪管贴着他的颈侧伸出,小桥惠紧咬牙关扣动了扳机。震耳欲聋的枪响过后,一条黑蛇从刚刚绽裂的墙壁缝隙中脱落坠下。

在众人的惊呼声中,香川武夫圆睁双目,极度的恐惧,让他几乎要愤怒了。

从军装大衣中扯出棉絮浸染烈酒,再紧紧缠上军匕刀尖。香川武夫制作了一支小小的火把,沿着墙壁缝隙反复烧灼。他认为自然界里没有不怕火的动物,黑蛇再厉害,也是动物中的一类。

在他忙碌之时,所有人都汇聚到了房间中央。怎么想都是没有活路,可还是得往活的一方面打算。死顶房门的两名士兵突然惊呼了,脚上的大头皮鞋蹭在水泥地上,正在一寸一寸的向前挪动。外面有力量在推门了!

众人一拥而上,拼了性命的撞向房门,一分一毫也不敢退让。事实证明,步枪对于蛇人是毫无作用的,冲锋枪对它也只能是``扰'',做不到``伤''。唯一有效的武器就是手雷,但是空间狭小,手雷不能任意使用。

僵持了片刻之后,外界的推力消失了,但是人们屏住呼吸,都认为蛇人并未远走。香川武夫趁机把地堡的地形图展开了,用手电筒照着图上路线,慌乱的寻找着出口。

可是未等他看出眉目,室内又起了轻响,是一声似有似无的破裂声。一点水泥碎屑顺着墙壁落下,不等旁人反应,小柳治发狂似的冲上去,一刀钉住了缝隙之中刚刚露头的黑蛇。

刀尖穿透蛇头,刺耳的划过了水泥墙面。小柳治张开了嘴,并没有一击即中的喜悦,而是带着哭腔
``哈''了一声。黑蛇已经软垂不动了,他还紧握刺刀刀柄,扎着墙壁不肯松手。

马英豪早在少年时代就和小柳治是朋友了,知道小柳治其实资质平平,根本不适合做一名军人。拄着手杖走上前,他抬起还在渗血的左手,强行摁下了小柳治攥刀的手臂。小柳治要发狂似的,又``哈''的出了一口气,随即咬牙切齿,像是要吃人。

马英豪绝望的看着他,因为和他是一样的悲观,所以没有安慰。

香川武夫说了话:``我们还是要冲锋,冲过三条岔路就有一条未完工的通道。我们——我们可以挖!''

然后他开始清点手中的手雷数量。

在他们自救的同时,白琉璃也在对他们施救。当然,白琉璃是醉翁之意不在酒,比起他们的性命,他对于蛇人本身更感兴趣。

指挥所门外很清净,没有任何活物经过,门内却是热闹,因为两个人的嘴都不闲着。方才无心在指挥所里找到小半杯水,给白琉璃喝了。白琉璃得了滋润,很快恢复了元气。从怀里摸出一只拳头大的幼童头骨摆在面前地上,他盘腿坐稳了,持久的嘀嘀咕咕。无心先是无可奈何的倾听,听着听着不服气了,低声反驳道:``怎么?难道全是我的错吗?当初我们在西康的时候,我白天给你做饭,晚上给你唱歌,我还给你养了两只小羊羔呢!''

白琉璃又摸出一只头骨,摸索着摆到自己的正后方:``我不喜欢吃你的饭,我也不喜欢你的羊羔。你唱的不是歌,是超度死人的经。我来过汉地很多次,我什么都知道。''

无心恨不能捶他一拳:``反正我不能和你过。我养小羊羔是为了喝奶的,结果被你喂了虫子——无论我养了什么,最后都是被你喂虫子!''

白琉璃的怀里是百宝囊,又摸出两只头骨,分别摆在左右两侧。无心不想让他出手帮助香川武夫等人,于是看他全摆整齐了,就伸手对着最近的头骨弹了一指头,把头骨弹移了位。

白琉璃抬起蓝眼睛,哑着嗓子威胁道:``你不要惹怒我!''

无心脸上不红不白的,起身围着白琉璃绕了一圈,把余下三只头骨全踢了个东倒西歪。末了停在白琉璃面前,他示威似的弯下腰,很认真的和白琉璃对视一眼,随即后退几步,洋洋得意的缩到角落去了。

白琉璃气得头疼,一边转着圈收拾骨头,一边喃喃的骂:``你个短命娃儿,脑壳遭门挤了。老子日你先人——嗯?少了一个?''

骷髅脑袋的确是少了一个,他找到三个,第四个不知滚到了哪里去。白琉璃开始四处寻找,心里也有点急,因为还是不想让马英豪和小柳治死。

无心伸手在地上摸,摸到一把散碎的豆子,还是当初撤退时遗留的。他把豆子一粒一粒的往嘴里送,因为饿极了。

无心饥饿,距离他们不远的香川武夫等人,自然更饿。他们身上还背着几十斤重的枪支弹药,而且身上除了一点烈酒之外,只有少许的水。

他们全都身强力壮,饮食多消耗大,比普通人更容易饿。比饥饿更可怕的,是前方没有出路和盼头。香川武夫用酒在地面上浇出一道弧线,弧线对着门口,像把弯弓似的拱向室内。所有人都各守位置准备好了,而两名顶门的士兵听香川武夫下了命令,立刻打开房门向内一跃,与此同时,香川武夫点燃地上的烈酒。士兵纵身越过瞬间窜起的火光,香川武夫看得清楚,就见几条黑蛇果然蠕动进门,可是被火线拦住,不能伤人。趁着火焰还亮,香川武夫连着几枪毙了黑蛇,随即跨过火线,向门外左右各扔出了几只手雷。大爆炸还未结束,室内众人已然一涌而出,辨明了方向直冲向前。

在第三条岔路口,众人心有灵犀的一起拐了弯。有人用手电筒向前照了,就见尽头拦着两扇对开的铁栅栏门,门后果然就是嶙峋不平的土石。香川武夫一枪崩开门锁,心中却是毫无喜悦可言——谁知道此地距离地面还有多远?也许是一米半米,凭着两只手就能刨开;也许是一里半里,他们没等服完苦役,就全死在地堡里了。

趁着身后还算太平,众人一拥而上打开铁栅栏门。士兵们因为一直在跟着香川武夫四处挖山,所以身上都带着工兵铲子。在香川武夫的指挥下,他们把挖出的土石全运送到了岔路口,堆成工事架起了冲锋枪。出了岔道再走几步,就能拐上主干道走廊。香川武夫回忆着粮库和军火库的位置,顺便又清点了身边人数,发现短短的一段路程,竟然又死了三名士兵。

香川武夫把所有人的武器都做了汇总,架在工事后方随时预备开火;又派了几个人手握手电筒和刺刀,专为对付藏在土中的黑蛇。负责挖掘的士兵全副武装,带着双层手套,头脸也都包严实了,只露一双眼睛。气喘吁吁的工作了半个多小时,地堡上空忽然响起了一声叹息。

随着叹息而来的,是一串清越的铜铃声。铜铃一晃一晃,响得很有节奏。岔道内的众人停了动作,就感觉心跳合了铜铃的节奏,一下一下不疾不缓,很是得劲。

然而得劲了没多久,铜铃的节奏忽然变了。

人们像是受了定身法,什么都忘记了,全部精神都集中在了自己的心率上。他们极力想让心跳追上铜铃,然而铜铃声音变化莫测。心跳随着铜铃忽疾忽缓,所有人都抓心挠肝的难受了。

小桥惠忽然喷出了一口鲜血,随即纵声尖叫,一边叫一边摇晃着踢打周遭人,又用日本话喊道:``不要听!鬼的铃,不要听!''

她明白了,其余人也明白了,但是一颗心不听指挥,执着的要追着铜铃声走。有人捂住心口俯下了身,有人想要开口发出声音扰乱铜铃,然而张了张嘴,声音哽在喉咙里,竟然发不出。

香川武夫忍着满胸膛的气血翻涌,伸手去摸有限的几枚手雷。可在他动手投掷之前,一阵沉闷鼓声忽然传来,压下了铜铃声音。

马英豪挣扎着站直了身体,惊喜的喊道:``是白琉璃!''

紧接着他眼前一花,倏忽闪过的光影让他愣了一下,他感觉自己好像是看到了马俊杰。

白琉璃为了防止无心捣乱,所以放出了一名卫兵。卫兵是只长着硬毛的大黑蝎子,围着他爬行不止。

无心果然老实了,静观白琉璃作法。白琉璃费了不少的力气,才从床底下找到了他的骷髅脑袋。四只头骨摆在前后左右,头盖骨光滑透亮,是被人摩挲过无数次的模样。白琉璃咬破手指,在四只头骨上画了血咒,然后又从怀里抓出一把粉末,均匀的洒在了血咒上面。把小小的人皮鼓放在腿上,他俯下身一边念咒一边拍着小鼓。

起初,他的小鼓仿佛受了损坏,拍不出声音,空中却是响了铃铛。无心听出铃声不对劲,但是到底怎么不对,他说不出,就见白琉璃身体一颤,紧接着继续拍他的小鼓。鼓声渐渐清晰了,和铃声一唱一和,响了个乱七八糟。

无心等到铃声稍弱了,开口唤道:``白琉璃?''

白琉璃也停了鼓声,然而俯身低头,一味的嗡嗡念咒,根本不理睬他。

于是无心自顾自的说道:``白琉璃,你乖乖坐着不要动,我出去看看情况。''

然后他起身拨开了门上插销。临出门时他迟疑了一下,末了从怀里摸出一张小纸条和一把小刀子。

刀尖刺破手指,他用自己的血在纸条上画出一道浅浅淡淡的驱鬼符。出门转身关了房门,他把纸符贴在了门缝上。

无心靠着墙根往前走,想要觅声寻找香川武夫等人,路上没被蛇咬,反倒是踩扁了好几条黑蛇的蛇头。豆子是不足以充饥的,他弯腰拎起一条死蛇,想吃,又嫌脏。

在暗处停了脚步,他看到了前方岔路口中的土石防线。防线后面人声鼎沸,有日本兵在狂呼乱叫。忽然一人张牙舞爪的跳过工事跑进了走廊,像只没头苍蝇似的又哭又喊。

一粒子弹结束了他的疯狂。两名日本兵出来,把尸首抬了回去。

无心没有手表,不知道自己已经在地堡里耽搁了多久。他攥着蛇尾巴,想象出了香川武夫等人的绝望。

他开始慢慢往后退。知己知彼,知道就好。然而一个脑袋忽然伸出工事,晃着手电筒左右张望了一番。无心正落在了手电筒的光柱中,和小柳治打了个照面。

小柳治大叫一声:``啊!无心!''

马英豪的声音随之而起:``抓住他!''

无心暗叫不好,拎着死蛇转身就跑。没等他跑出多远,后面起了枪响。追兵不想要他的命,手枪瞄准的是他两条腿。一个踉跄摔了个大马趴,他在剧痛之中爬起身,一摇一晃的继续逃。逃到半路他看到路口,立刻拐了弯。但是单手扶住墙壁,他在道路尽头,却是看到了小健。

自从进了地堡,小健就没了踪影。无心知道他有点小本事和小聪明,所以不很担心。可是此刻小健悬在空中闪闪烁烁,脸上神情十分惶恐。

小健身后飘着一个模糊的鬼影,正是马俊杰。

无心怔了一怔,脑子里猛的打了个霹雳——鬼吃鬼,马俊杰要把小健吞噬掉了!

他急得捏开蛇嘴,将蛇牙刺入自己的脖子,沾了鲜血之后把蛇抡圆了,用力甩向前方鬼影。他宁可让小健魂飞魄散,也不让他被鬼吃掉!

可是死蛇在鬼影前方落了地,小健意识到了他的存在,微弱的叫了一声:``大哥哥\ldots{}\ldots{}''

一声过后,他的影子彻底消失在了马俊杰身前。马俊杰对着无心冷冷一笑,随即无影无踪。

\chapter{人吃人}

无心坐在水泥地上,大睁着眼睛怔了半天,末了垂下头,拔萝卜似的用力拔下了右脚的沉重皮靴。

靴筒被子弹穿了个洞,然而靴子里面很干净。自从上了山就吃不好喝不好歇不好,他的鲜血都被熬干了,几乎无血可流。挽起层层裤管,他咬紧牙关忍住了痛,把手指插进小腿伤口之中,贴着骨头挖出了一颗子弹头。

子弹头表面沾染着薄薄一层血肉,被他扔进嘴里唆了唆。扭头``呸''的一声吐出子弹头,他又往道路尽头望去。尽头什么都没有了,他不是鬼,不知道被鬼吞噬是什么滋味,但是一定不好,他笃定的想,一定很不好。

用力的扳起小腿俯下身,他伸长舌头又舔了舔伤口。理好裤管套上皮靴,他扶着墙壁站起了身。

无心一瘸一拐的慢慢走,走到粮库取了一口袋肉罐头,然后悻悻的回到了指挥所。

肉罐头在口袋里互相碰撞磕打,很不安静。取下门上的纸符揣回怀里,他进了门,然后弯腰把口袋放在了角落里。

白琉璃还摆着他的阵法,但是鼓不敲了,经也不念了。臃肿的上半身向前趴伏在地,他看起来正是乱七八糟的一大堆。他不理睬无心,无心也不说话。拿出一个罐头切开铁皮,他慢慢的吃,一边吃一边想小健,想到最后出了神,含着一口牛肉忘记了咀嚼。

良久之后,突如其来的一声大爆炸震醒了他。俯身凑到门下孔隙前,他抽动鼻子嗅了嗅,没有嗅到硝烟气味。此时能在地堡里制造爆炸的,只有香川武夫一行人。无心心中一凛,暗想难道蛇人又出现了?

随即他把目光转向了白琉璃。白琉璃伏在地上一直不动,头上却是隐隐出了热汽。方才的爆炸巨响并没有影响到他,他正在聚集他的念力。

无心一边吃罐头一边向外窥视,疲倦了就闭上眼睛打个盹。白琉璃长久的一动不动,让无心偶尔产生怀疑,怀疑他是悄悄死了。

时间的概念是彻底消失了,把无心从睡眠中唤醒的,往往就是隔三差五的大爆炸。将最后一只肉罐头打开了放到白琉璃旁边,他侧身卧倒横在门前,迷糊着继续睡。

转眼之间,三天三夜过去了。

日本兵们还在绝望的挖掘着出路,即便他们已经负担不起了工兵铲子和一身厚重衣裳。什么食物都没有,他们距离粮库还有一段距离,而到目前为止,已经有两名士兵死在了这一段距离上。巫师鬼魂无影无踪而又无处不在,不止一个人见到了它的鬼影——是个典型萨满巫师的打扮,穿着神裙带着神帽。神帽像是古时战士的头盔,头上伸出两只牛角;神裙则是模糊绚烂,外面罩着一副金属肋骨。

只能看清这些了,它永远只是一闪而逝,在空中留下苍凉怨毒的叹息,索命的铃声倒是没有再响起过。

小桥惠捡了几条死蛇,想要把它们放到火上试着烤一烤。然而火苗燎过蛇身,蛇肉立刻散发出了浓烈的血腥气。

半焦的死蛇立刻就被小桥惠远远扔开了。她的小手在哆嗦,同时沮丧得要哭。为什么蛇肉是臭的?而且臭到无论如何不能下咽?他们都饿极了,香川武夫的光头都没了亮。

士兵们试着用手雷去炸山中土石。炸过一次,效果不算好,而且还崩伤了一个人的手。香川武夫盯着伤者手上汩汩流出的鲜血,盯了良久,然后去把扔在角落的一具士兵尸体拖到了小桥惠面前。

小柳治当即大喝了一声:``不行!''

一贯冷静的小桥惠有些茫然,谁也不看,只盯着尸体瞧。地堡里面不算很冷,尸体死了三四天,微微的也有了腐烂的征兆。香川武夫拄着一支步枪站直身体,冷森森的望着小柳治:``我们需要力量干活。牛马猪羊可以吃,他现在不过是一堆死了的骨肉,当然也可以吃。当然,你可以不吃,我不会勉强任何人。''

随即他对小桥惠一挥手。小桥惠跪坐在火堆边,神情木然的仰脸看了香川武夫一眼,紧接着把牙一咬,一张平淡的小脸忽然狰狞了。从腰间抽出一柄雪亮的军刀,她四脚着地的挪到尸首身边,开始去解对方的衣扣。

四周陷入了寂静,连挥着铲子的士兵都停了动作。停了片刻,他们又无望的继续挖了起来。

香气不动声色的弥漫开了,像一只大手,揉捏着所有人的肠胃。小桥惠把军刀倒转着递向了香川武夫,刀尖上挑着一块滋滋作响的肉。香川武夫接过军刀,对着油汪汪的肉块狠狠看了一眼,随即张嘴就将其吞了下去。

小柳治神情痛苦的一闭眼睛,又抬手去捂马英豪的脸,不想让他看到如此恐怖的场景。然而马英豪轻轻拂下了他的手掌,在他耳边低声说道:``我们得活着啊。''

香川武夫把军刀递还给了小桥惠,同时说道:``一人两块肉,省着点吃。''

马英豪吃了人肉,小柳治没有吃。

吃了人肉的士兵继续换班干活。出口是倾斜向上的,已经挖出很深。沿着挖出的斜坡走进深处,可以摸到土壤越来越凉,可见他们的方向并没有错。

不知过了多久,小桥惠将一具剔得干干净净的骨头架子扔出了岔道。一名坚持不肯吃人肉、因此也被剥夺了水壶的士兵,已经虚弱到了睁不开眼睛的程度,所以被香川武夫一枪毙了。这回他们吃出了经验,新鲜的脑浆和鲜血都没有浪费。

到了这般关头,小柳治的军官身份已经一文不值。马英豪知道小柳治一死,接下来被大家吃掉的就必定是自己这个中国人,因此捡了一小块最瘦的肉,强行塞进了小柳治的嘴里。小柳治哭丧着脸,舌头一拱一拱的想吐,被马英豪紧紧的捂住了嘴。马英豪在他耳边低声说道:``你要是死了,我也活不成了。''

小柳治呻吟一声,眼泪都出来了。喉结上下艰难的一滑,他把肉囫囵着咽进了肚里。

香川武夫的肠胃充实了,可不安的空气却是一直萦绕着他。蛇人没有再次攻击他们,但他并未感到轻松。蛇人如果要杀他们,真是太容易了,几只手雷和两支冲锋枪是拦不住它的。可它显然并未使出全力——它吊着他们的神经,越吊越高越吊越细,把他们吊成了吃人的魔鬼。

无心又出了一趟门,发现地堡内的鬼魂越来越少了。

暗暗潜到香川武夫的工事附近,他看见了巫师的鬼影。

它正在吞噬一只怨气冲天的日本鬼,日本鬼的幻影,和工事后面的日本兵是同样的装扮。无心不知道他是怎么死的,只知道他一定死得惨而不甘,像马俊杰一样,是只厉鬼。

他看了一阵,随后悄无声息的溜回了指挥所,告诉白琉璃:``我们有个粮库,蛇人也有个粮库。你当它只会玩蛇吗?它在吃鬼呢!''

白琉璃伏在地上,不言不动。

无心又道:``香川武夫他们已经开始吃人了。不是吃死人,是杀活人吃。巫师的鬼魂就守在工事外面,吃他们制造出来的厉鬼。地堡本来就够邪的,人还不是好死,你说变成的鬼会有多凶?''

白琉璃终于出了声音,声音微弱而又清晰:``好,很好。''

无心莫名其妙:``好在哪里?''

白琉璃答道:``如果他的力量能够归我所有,也许我就能让我的儿子复活了。''

无心不以为然的叹了一声:``你可饶了孩子吧!你儿子让你弄得没个人样,真复活了,将来也讨不到老婆。''

还有一句话,无心没说,就是白琉璃仿佛太过自信了——孰知不是巫师的本领更胜一筹呢?

无心依然是和白琉璃谈不拢,所以静观变化,等着香川武夫等人全军覆没。香川武夫一死,他也就可以放心的逃生了。

如此又过了不知几时几日,无心发现香川武夫等人,似乎是要疯了。

一条小小的岔道之内,扔满了血肉模糊的人身零碎。他们所剩人员已经不多,而且吃红了眼睛。谁也不敢表现出丝毫的虚弱,一瞬间的示弱都可能招来一粒子弹。

时间失去了意义,挖掘的工作也停止了。人的食欲像烈火一样蓬勃高涨,越大嚼越不满足。无心躲在暗处,看到香川武夫把嘴凑到还未断气的士兵颈上,大口大口的吸血。又有人冲过来扯起士兵的一只手,塞到嘴里咯吱咯吱的狠咬。

无心不想再看下去了。可就在他将走未走之时,工事前方的地面忽然波涛汹涌的起伏了,竟是无数黑蛇不知何时汇聚成了一片。巫师的鬼魂在半空中闪闪烁烁,而黑蛇扭绞着纠缠垒叠,迅速的组成了高大的人形。鬼魂消失不见了,取而代之的是一具漆黑蛇人。蛇人一步跨过无人防守的土石防线。距离防线最近的一名士兵含着满嘴血肉转向了它,正要惊呼着抄起步枪。然而蛇人已经抬手搭上了他的头顶。

一命活蹦乱跳的青年在蛇人掌下僵硬了身体,迅速枯萎成了蜡黄干尸。蛇人的身体表面布满了一收一缩的蛇嘴,碰一下便是死。

无心不想被混战殃及,于是调头便逃,把震天撼地的大爆炸全部抛到了身后。一个白而圆的物事挟着风声掠过他的头顶,咕咚一声落到地上。他定睛一看,看到了香川武夫的面孔。

香川武夫的脑袋齐颈而断,落地之后骨碌碌滚到了无心脚前。无心一见他死了,心中立时轻松了好些。一脚踢开对方的光头,他一溜烟的跑没影了。

大约是一个多小时后,无心蹑手蹑脚的返了回来。

岔路一带成了一处寂静的战场,土石残肢散落满地,水泥墙壁都坍塌了一片。

无心看了此地的惨象,知道香川武夫一部已经全灭。转身踏上来路,他想要回指挥部,劝白琉璃和自己一起设法离开地堡。然而刚刚转了一个弯,他颈上忽然一紧,却是有条手臂从后勒住了他的脖子。

紧接着,带着血腥味道的热气扑到了他的耳根,马英豪气喘吁吁的说道:``别动!''

\chapter{马英豪之死}

马英豪的手臂像是铁铸的一般,坚硬的环在无心的脖子上。另一只手也从后方伸到了他的面前,手中攥着的手雷缓缓蹭过了他的鼻尖。

手雷已经去了保险,只要再受一次碰撞,便可引发内部机关爆炸。无心的眼珠随着手雷转动,看到马英豪的动作很慢很稳,镇定得像是劫后余生,也像是回光返照。

``坐下\ldots{}\ldots{}''马英豪的嘴唇几乎快要贴上了他薄薄的耳朵,声音嘶哑,带着哄诱的语气:``乖乖坐下\ldots{}\ldots{}''

无心知道手雷的威力,所以没敢反抗,怕马英豪和自己同归于尽。自己被炸碎一次,不定需要多久才能重新成人。地堡里面没好吃没好喝,而且又冷,他成长的速度自然不会快。等他花几月半年的工夫长成了再见天日,赛维和胜伊早走的连影子都没了。

他向后靠在马英豪的怀里,一点一点的往下蹲。屁股着地之后他一垂眼帘,才发现马英豪的右腿瘸上加伤,皮靴的靴筒被炸烂了,里面的皮肉正血淋淋的翻着。

马英豪倚着墙壁坐住了,很舒适似的长吁了一口气:``好,好,我也歇歇。再不歇一歇,我连死的力气都没有了。''

无心望着前方问道:``既然你有了要死的心,为什么还要拉着我?''

马英豪有气无声的发笑:``我可不想孤零零的等死,临死还要受一场寂寞的苦吗?嘿嘿,我不受。''

然后他渐渐放松手臂,改用手去掐住了无心的脖子。他很久没有修剪过指甲了,长而肮脏的指甲随着他的力道,陷进了无心的肉里。无心的脖子很安静,没有血脉跳动,也没有气流出入。

无心又问:``你是怎么逃出来的?''

马英豪低声说道:``蛇人出现的时候,我就坐在紧挨防线的角落里。我一直在谋算着逃,小柳吃了人肉会吐,已经虚弱得连枪都举不起。我知道下一个成为食物的人一定是他,所以我想带着他逃。''

无心说道:``但是你失败了。''

马英豪表示同意:``是的,我失败了。我拽不动他,于是就一个人越过防线跑了。''

无心静了片刻,然后说道:``你害死了很多人。''

马英豪又笑了:``不要傻,他们迟早是要死的。他们是军人,逃不过战场。''

无心轻轻的摇了摇头:``死在地堡里的人,灵魂被巫师吞噬,永世不得超生。''

马英豪沉默半晌,末了答道:``生前不管死后事。小妖怪,不要恐吓我了,我不怕。不得超生又怎么样?六道轮回六道苦,我不快活,人间也是地狱。''

无心垂下了头:``如果你不贪婪,人间也不会变成地狱。''

马英豪在他耳后呼出热气:``贪婪?不,他们是贪婪,我不是,我是仇恨。我恨马浩然,可是我又奈何不了他。你当我图谋他的宝藏,是为了钱吗?不是的,我是想让他痛苦。他一贯视财如命,而我无法报复他,只能抓住一切机会,让他损失,让他痛苦!''

随即他的声音忽然轻了:``我是败在了猜忌上\ldots{}\ldots{}老五向我告密时,句句都是实话,可我不相信他。我让白琉璃作法折磨拷问二姨娘,二姨娘吓得实话实说,可我还是不相信。我拿了二姨娘的话又去试探老五。我们互不信任,把简单的事情搞成混乱复杂。''

无心头也不回的说道:``你信又不信,所以在河水里放了蛊,以免在你找到真相之前,会有其他人先下手?''

马英豪自顾自的继续说:``我根本不需要宝藏,我有钱用。知道地下仓库里全是古董之后,我立刻找了小柳治。我不要钱,我想献出国宝,换个大官当。升了官,我就能和马浩然斗一斗\ldots{}\ldots{}当然,如果能直接置他于死地,就更好了。他是个疯子,他饿死了我娘,还把我打成了残废。我恨他。''

手指在无心的脖子上加了力气,马英豪微微探头,审视了无心的侧影:``没想到螳螂捕蝉、黄雀在后。你个小妖怪,你早就知道我们会被巫师杀死,对不对?你对老二老三真是赤胆忠心啊,你不怕自己也死在地堡里吗?''

无心抬起了头,对着前方答道:``我是妖怪,不会死的。''

马英豪上气不接下气的笑了一声:``真厉害。如果我早养了你,就不会再要海蛇了。知道我为什么不接佩华回家吗?佩华怕蛇,见了蛇就要哭。我对不起小柳,更对不起佩华。马浩然回家之后一定会杀了她,劳驾你帮我给她带句话,让她快逃,逃到天津我家里。后面该怎么做,她心里有数。''

无心心中一动:``你不想杀我?''

掐在他脖子上的手指一点一点松了,另一只攥着手雷的手,也平平的向后撤去。马英豪放开了无心的脖子,顺势在他的脑袋上摸了一把:``白琉璃还好吗?''

无心俯下身,四脚着地的向前爬:``还活着。''

马英豪伸长手臂,在他后脑上弹了一指头,气若游丝的又说了一声:``小妖怪。''

无心站起了身,回头看他,忽然发现他是坐在了血泊里。腿上的伤不足以流出成滩的鲜血,他睁着乌溜溜的黑眼睛去看马英豪,知道马英豪一定是受了重伤。

而马英豪一手举起手雷,另一只手则是向他挥了挥:``去吧,去吧,记得救佩华。''

嵌满了手雷碎片的后背靠上水泥墙壁,马英豪仰起头闭了眼睛,把手中的手雷用力向地面一磕。

无心见状,撒腿就逃。逃出了没有十步,后方响起了一声大爆炸。他被气流推了个大马趴。连滚带爬的起身又折了回去,他对着半空中还未成形的一团微光,咬破手指甩出了几点鲜血。

微光遇到鲜血,登时闪闪烁烁的消散了。马英豪的肉体和灵魂一起灰飞烟灭了,免于被蛇撕咬,被鬼吞噬。

无心往岔道走。跨过一地七零八碎的土石血肉,他钻进了日本兵斜斜挖出的土洞。土壤真的是越深越凉,可见日本兵的大方向没有错。斜洞里还扔着一把很结实的铲子,无心握在手里掂了掂,感觉长短轻重都很合适。可惜指挥所里还有一位白琉璃,否则凭着他的力量,满可以直接开工了。

他不能由着白琉璃异想天开,扛着铲子离开岔道,他一边往回走,一边盘算着如何把白琉璃运出指挥所。想着想着,他握着铲子向前一铲,幻想自己像铲大粪似的,把白琉璃铲起来就走。

然后他忍不住笑了,因为大功告成,心里有一点高兴。忽然抽了抽鼻子,他嗅到一股子寒冷的水味。

对于无心来讲,万物都有气味。无心停了脚步,发现自己不知不觉走上了一条新路。回头望了望来路,他心里有了数——并没有迷路,是自己一时走神,提前拐了个弯。

他觅着水味往前走,最后在道路尽头,他发现了一扇半开半掩的大铁门。推开铁门向内伸进脑袋,他看到了一处大水池。

门后什么都没有,就是一个方方正正的大水池,四周留了可供行走的平台。生了薄薄水锈的水管从水泥天花板上伸出,顺着墙角一路走进池中。要看尺寸,水池可以容人来回游泳了;但是地堡里面显然不会有游泳池。没有游泳池,也不该有水牢,思来想去的,无心认定它是一处未完工的蓄水池。

池子里面一米深处,就是平静的水面。水不新鲜了,但也不脏。无心记清了路线,心想自己将来没了水喝,还可以到此地痛饮一番。

水有了,粮库里的食物也充足。无心几乎没了后顾之忧,扛着铲子往回走。可是刚刚上了主干道走廊,他就停了脚步。

在他前方不远处,他看到了蛇人。

空中回荡起了低沉鼓声,蛇人的身体随着鼓点颤抖,仿佛条条黑蛇都脱了力,随时会首尾松脱。无心双手横握着铲子,正打算上前痛打落水狗。不料蛇人的身体骤然崩溃,落了满地的黑蛇蠕动着四散逃窜。没了黑蛇的遮掩,巫师灵魂的真面目显露在了无心眼中。

无心发现巫师是面对着自己的,就勉强一笑,开口说道:``是我把你拼成一体的\ldots{}\ldots{}你还认不认识我了?''

鼓声还在持续,巫师的鬼影一闪一闪。

无心打算采取怀柔政策,所以和颜悦色的打商量:``你的灵魂是复活了,当初抢夺宝藏的人,也都死绝了。得饶人处且饶人,你让我们走吧。''

无心满拟着要和巫师和谈,可白琉璃的鼓声却是越来越激烈,似乎是专门要和他作对。前方巫师的灵魂没有做出回应,而是迅速后退消失了。

好好的谈判机会被毁了,无心气得恨不能一铲子拍死白琉璃。大步流星的走回指挥所,他开门便道:``你——''

``你''字出口之后,他立刻闭了嘴。白琉璃是最怕光的,此刻面前却是摆了一只小碗。碗里不知盛了什么油脂,一根灯捻正在幽幽的燃出绿光。白琉璃四周的骷髅脑袋全被黑气笼罩了,画在头盖骨上的血符正在慢慢褪色。白琉璃一边用手指叩着人皮鼓面,一边用带着伤口的手指,依次描画头盖骨上的血符。

无心放下铲子关了房门,又从怀里摸出先前画好的纸符贴上门缝。他误会了,原来不是白琉璃要杀巫师,是巫师要杀白琉璃!

\chapter{白琉璃的归宿}

惨绿的火苗平静的散发幽光,油灯后面的骷髅头上,鲜红的血咒又开始褪色了。

白琉璃深深的垂下了头,一张面孔藏在了凌乱长发之中。一层微弱的水汽笼罩了他的头脸,他半闭着眼睛,口中一直喃喃念诵着咒语。带着伤口的左手忽然用力拍向地面,粘稠的黑血顺着指尖伤口汩汩流出。白琉璃再次抬手摸向骷髅头顶,飞快的描绘了血咒笔画。

无心知道白琉璃是在作法对抗巫师灵魂,自己想要帮忙,却又不知从何帮起。蹑手蹑脚的从床底下捡起一只铝制的饭盒盖子,他想把自己的鲜血贡献给白琉璃;然而自己的鲜血专克毒邪之物,只怕帮忙不成,反倒要伤了白琉璃。

他抽出刀子,先用刀刃割破了手腕,只流出几滴稀薄的凉血。他转而又用刀尖刺破了脖子,点点滴滴的又挤出了一点鲜血。鲜血盛在饭盒盖里,是不起眼的一小滩。无心端着饭盒盖爬到白琉璃身边,陪着小心轻声说道:``你自己保重,我出去看看情况,很快就回来。''

白琉璃没理他,身体缓缓向下俯到地面。一直敲打着人皮鼓的右手也向前伸长了,层层叠叠的兽皮起了涌动,仿佛他的身上藏了活物。脊背忽然凸出了拳头大的鼓包,鼓包迅速的向上移动越过肩膀,一只斑斓蛇头倏地窜出了白琉璃的袖口。

无心早就看出白琉璃身上没少藏东西,可是万没想到居然养着偌大的凶物。蛇头是个眼熟的模样,额上只有一只横生的人眼,眼下则是四方口器。闪电一样游向门下孔隙,它虽然有着一米多长的身躯,可是蜿蜒灵动,竟然瞬间便是无影无踪。

无心站起了身,眼看骷髅头上血咒赫然,还没有消失的征兆,可见白琉璃至少在目前一段时间里一定安全。骷髅头是带有魔性的,被白琉璃施了血咒之后,就会帮助白琉璃汇聚念力。念力越强,血咒越清晰。

无心不知道自己能有多少自由活动的时间。端着一饭盒盖的鲜血转身出门,他把自己的纸符照样贴上门缝,然后开始四处寻找巫师的灵魂。

地堡道路四通八达,无心连走带跑,可是连巫师的影子都没有见到。他有些急了,转身想要回指挥所,可在距离指挥所几米远处,他骤然收住了脚步——他看到了满地密密麻麻的黑蛇!

黑蛇一条挨着一条,已经遍布了指挥所门外的地面。而之所以它们没有通过孔隙钻进指挥所,是因为孔隙之前盘着独眼大蛇。独眼大蛇收缩着它的四方大口,把头缓缓昂到极致,紧接着居高临下猛的向下一扎,它一口吞下了一团黑蛇。黑蛇蠕蠕的互相纠缠,缓缓沉入大蛇的咽喉。大蛇像个直上直下的管子,吞过一团之后,它再次昂起了头。

无心知道白琉璃不会为黑蛇所伤,但不知道他和巫师斗法会有什么结果。斗法不是斗殴,一场拳脚过后便能见分晓;他记得在五年前,白琉璃曾经不吃不喝连着念了十天的咒,活活咒死了当地一位德高望重的大喇嘛。能咒死人,自然也能被人咒死。大喇嘛死时遍体乌黑,活像中了剧毒;而他明知道白琉璃不是个好东西,可是帮亲不帮理,不想看到白琉璃也变成黑琉璃。

忽然间,无心瞧见了巫师。

隔着一片蛇阵,他看到了远处的巫师鬼影。巫师的模样很清楚,然而神帽下面黑洞洞,并没有面孔。一动不动的正对着无心,他当然不可能有表情,但无心察觉出了他的怨气,冲天的怨气。

和厉鬼是讲不出道理的,唯一的办法把它打成魂飞魄散。无心意识到自己不能再怪白琉璃惹是生非了。白琉璃没有错,即便白琉璃不出击,巫师也饶不了他们,因为他们是入侵者,是活人。巫师生前为什么要忍受非人的痛苦、让人把自己分割成为两半?为的就是报复!对手是谁都不重要了,重要的,是报复本身。

况且白琉璃若是死了,便会分离出一个力量强大的灵魂。如果能吞噬了他的灵魂,对于巫师来讲,裨益不言而喻。

无心用手指蘸了鲜血,弯腰草草涂抹了双脚皮靴。然后抬脚踏上黑蛇,他一步一步的向前走,脚下咕唧作响,是黑蛇被他踩碎了骨头,踩出了汁液。

他越是前行,巫师的鬼影越淡。无心停在了指挥所的门前,怀疑巫师只是在向自己示威。可就在他思索的空当里,半空中响起了铃铛声音。声音一抖一抖,像是衰朽之人的心跳。无心不知道铃声是真的存在,还是只是自己的幻听。不过无论如何,他是不怕的。

不怕,但是装成怕的样子,一步一步踉跄着走向鬼影。无心不知道自己伪装的像不像,因为从没中过任何摄魂术。跌跌撞撞的越走越快,他眼看鬼影终于近在咫尺了,举起饭盒盖子就要打去;不料在他动手的同时,两边墙壁忽然爆出破裂声音,几道箭簇似的黑影激射而出,正是黑蛇!

黑蛇冲撞了他的手腕和头脸,本意是要吸他的鲜血,可是未等动口,便被饭盒盖中泼洒出的鲜血洒中了。``仓啷''一声响,饭盒盖子落上了水泥地,无心失去了仅有的一点鲜血。而墙壁爆开的裂缝中涌出越来越多的黑蛇,在鬼影脚下汇聚叠加,组成人形。

无心见势不妙,转身就跑。趁着蛇人还未成形,他冲回了指挥所。背靠房门面对了室内的白琉璃,他发现绿色灯焰后方的骷髅头上,本来鲜红的血咒像在不停渗透一样,颜色正在越来越淡。

白琉璃垂着头,将一根长针插入左手的中指指尖。捏住针尾缓缓向内推去,他一直把针扎到了底。针尾最后也没入皮肉之中,他握住左手腕子,像是发了疟疾一般开始哆嗦。长针的针尾像是受了某种力量的催逼,一点一点滑出指尖。及至长针彻底脱离,针孔之中激射出了一股子黑血。

藉着黑血反复描画了四方骷髅上的血咒,白琉璃一直没有中断念咒。平日看他总是气若游丝,此刻的气息却是战栗而又充沛。咒语像潮水一样一波一波连绵不绝,他忽然仰起了头,尖削的下巴抬在幽绿火光之中,苍白皮肤上凝结了一层晶莹的水光。双眼紧紧的闭了,他神情痛苦的拧起了两道长眉。

无心不言不动,盘腿坐在了白琉璃身边。白琉璃摆出了要和敌人同归于尽的架势。无形的战争已经进行到了生死关头,他此刻所能做的,只有静观。

室内的空气升了温度,白琉璃将血淋淋的左手搭在侧面的骷髅头上,右手抬起来梆梆梆连敲三声人皮鼓,随即向天发出一声狮子吼。在吼声中,他举起右手狠狠击下,一掌把人皮鼓击成粉碎!

无心勃然变色,没想到他竟然亲手毁了自己的法器。

半空中的铃声被白琉璃的一吼震断了,直到白琉璃用手掌拨开了人皮鼓的碎片,铃声才断断续续的继续响起。四面墙壁之中起了闷响,仿佛是要破裂而又未破裂。无心心中一惊,当即起身环视四周,提防着黑蛇从墙壁裂缝之中趁虚而入。

大概是因为房内坐着白琉璃的缘故,四面墙壁始终是没有绽开缝隙。无心刚刚松了一口气,不料房门轧轧作响,竟是自动开了。门外黑影阴森,正是蛇人!

蛇人动作笨拙,一步一顿,显然是巫师灵魂受了白琉璃的攻击,此刻也只是要反守为攻而已。守门的怪蛇在地上抻成长长的一条,已然毙命;无心眼看白琉璃前方再无防线,情急之下索性抄起铲子纵身一扑,一铲子带着风,结结实实的拍上了蛇人的脑袋。

蛇人的脑袋登时变了形,然而立刻又自动的恢复了原样。后方的白琉璃一挥大袖,同时厉声喝道:``无心回来!''

无心连忙侧身一避,就见地上散落了一片绿莹莹的光点,萤火虫似的还挺美丽。光点迅速移动向了蛇人。无心一低头看清楚了,原来光点全是一指来长的小毛毛虫。小毛毛虫色彩鲜艳,身上缀着点点光斑,另有一层七长八短的毛刺。速度最快的小毛毛虫已经触到了蛇人的一只脚,也不知道它有多么厉害的毒性,触到黑蛇之后,黑蛇立时就松软了身体,皮绳似的脱落了它的组织。

蛇人力不能支的后退了,刚刚退到走廊,便瓦解成了一团缠杂不清的蛇堆。无心等到小毛毛虫全爬出去了,连忙关闭房门。回头再看白琉璃,他耳听铃声又起,和先前相比,也带了一种回光返照似的激烈。

白琉璃依旧仰着头。双手扒住胸前衣襟向两边一扯,他从层层兽皮之中露出了穿着锦袍的上半身。锦袍的底子已经看不出颜色了,金银线绣出的花纹也尽数模糊,然而尺寸是太合适了,正好显出他端正的肩膀和修长的手臂。摸索着将四只骷髅头在面前摆成一排,他忽然扭头睁眼,对着无心得意一笑。半盲的蓝眼睛,竟然一刹那间目光如电。

无心猛然向他走近一步:``白琉璃,我带你回西康!''

白琉璃闭上眼睛转向前方,用左手中指最后一次描绘了骷髅头上的血咒。念力本来是分布四方保护他的,如今汇聚一处,把他彻底曝露在外。神情傲然的微微扬起脸,他对着正前方连拍三次手掌。掌声响亮,盖住铃声。

然后他前仰后合的开始念咒,一念,就是一天一夜。

无心坐在他的斜前方,目不转睛的盯着他膝前的四只骷髅头。头上的血咒一直鲜红,半空中的铃声却是从断续变成微弱,又微弱到了消失。

无心担心会有黑蛇来偷袭白琉璃,所以不敢起身出门。白琉璃不吃不喝,消耗着他有限的生命力。他的长发被汗水打湿了,披散着一直垂到肩膀胸膛。

一天一夜之后,他提高了一个调门,身体越发摇晃得疯狂。前方的绿色灯焰忽然窜起一尺多高,与此同时,四只骷髅头像受了火炙一般,一起腾出了一股子火光。

火光熄灭,骷髅成了烟熏火燎的黑色,灯焰却是转成了明亮的黄色。白琉璃昂起头深深的吸了一口气。屏住呼吸顿了一顿,他垂下头,把气又长长的呼了出去。

地堡之内寂静到了恐怖的地步。无心四脚着地爬上前去,歪着脑袋去看白琉璃的脸:``结束了?''

白琉璃低低的答道:``嗯,结束了。''

无心紧盯着他又问:``你\ldots{}\ldots{}会死吗?''

白琉璃的声音越来越低:``嗯,会。''

无心抬手拨开了他挡在眼前的乱发:``不死行吗?''

白琉璃摇了摇头:``不行。''

无心去看他的眼睛,未等看清,白琉璃向前一扑,额头正好抵上了无心的肩膀。

无心没敢动,试探着用手拍了拍他的后背:``白琉璃,我们打个商量,不死好不好?''

白琉璃的声音微弱成了气流:``不好。''

无心叹了口气:``我\ldots{}\ldots{}我还欠你六百英镑呢。''

白琉璃轻声答道:``不要了。''

然后他又对无心说道:``我要死了,知道我为什么要死吗?因为我不想离开地堡,地堡很好,比西康好。巫师没了,地堡就是我的了,整座山也是我的了。我可以夏天看看花,冬天看看雪,真好。''

无心点了点头,一切都理解。白琉璃想要留在地堡,就得彻底打败巫师;否则巫师不会容他平安生活。白琉璃虽然也是位大巫师,但是神通不能带到死后,成鬼之后必定弱小,不但不是巫师的对手,甚至还有被巫师吞噬的危险。所以他要用他的命去镇压巫师。巫师没了,他就是地堡内最强大的游魂。

白琉璃的身体在渐渐变冷:``无心,虽然你是个骗子\ldots{}\ldots{}不过毕竟在西康陪过我大半年\ldots{}\ldots{}''

他的言语开始变得断断续续:``所以\ldots{}\ldots{}我决定把我的遗产\ldots{}\ldots{}留给你\ldots{}\ldots{}''

无心侧过脸,忧伤的注视着白琉璃的脑袋:``你的遗产是什么?''

白琉璃沉默片刻,然后答道:``唉\ldots{}\ldots{}记不清了,反正我身上的所有东西\ldots{}\ldots{}全留给你。''

无心又问:``后事怎么办?土葬还是火葬?''

白琉璃的口鼻间逸出了浅浅的气流:``风葬吧。''

一团柔和的白光颤巍巍的离开了白琉璃的身体,无心仰起头,知道白琉璃死了。

白光像一轮太朦胧的月亮,闪闪烁烁的停在半空。

无心望着白光,轻声说道:``你别急,我知道地堡里有个大水池。我先去给你洗个澡,然后继续去挖地道。我不会再骗你了,一定好好的风葬了你。''

\chapter{重见天日}

无心拎着白琉璃的后衣领,在空寂的甬道上慢慢走。一轮明月似的白光若即若离的飘在他的头上,是白琉璃的鬼魂还未成形。

在蓄水池的铁门外,无心停了脚步。把一路从各个开门房间里搜罗出来的什物逐样摆在地上,他先点燃了其中一盏煤油灯。一灯如豆,黑暗无边;向前向后看,都没有生机。无心蹲下了,展开了从将校休息室里带出的一床棉被。刀子割断棉线,他把棉被拆成了两片布和一团棉胎。被里被面都很干净,粘着有限的一点棉絮。他撕了两小块棉花揉成团,仔细的塞进鼻孔里,然后转向了白琉璃。

原来白琉璃真是有一点遗产的。

无心从他腰间解下了一条沉甸甸的银腰带。白银都成了黑色,只在花纹起伏处还能看出洁白的本质。把银腰带放到一旁,他将双手插到白琉璃的腋下,把他从一大堆肮脏兽皮中拖了出去。

层层兽皮里开始向外蠕动毒虫。趁着毒虫们还没有集体大逃亡,无心在兽皮上浇了煤油。一点火星迸上去,火苗子立时窜起多高。火中起了噼噼啪啪的微响,火焰的颜色不稳定,始终是介于黄绿之间。藏在兽皮之中的婴尸猛然坐起,是一身的筋骨烧缩了。

无心背对了火堆,继续为白琉璃脱衣服。肮脏的锦袍也被扔进火里了,地上``叮''的一响,是个变了形的小铃铛从袍袖中落了下去。

无心从被里上撕了一大块白布,把一块肥皂打成包裹,系在自己的脖子上。又用细布条编成长绳,一端绑在铁门把手上,另一端绑住了白琉璃的腰。将自己里外的衣裳尽数脱了,他赤条条的抱起白琉璃,试探着跳下了水池。

水有半人多深,白琉璃的尸首被布绳吊在水面,无心也解开了胸前的白布包袱。肥皂滑溜溜的浸透了水,他开始往白琉璃的头发上涂抹。白琉璃太脏了,肥皂打了好几遍,泡沫总是不见丰富。无心一手把他揽在胸前,一手裹了白布在他脸上细细的蹭,蹭了半天才蹭出一块干净肌肤。

池子里响起了哗啦啦的水声,是无心终于收拾出了白琉璃的头脸,大开大合的狠擦起了他的前胸后背。一团白光在他的眼角余光中飘飘荡荡,他无暇去看对方,咬牙切齿的忙着干活:``白琉璃,瞧你脏的!''

当兽皮和婴尸一起化为灰烬时,无心从水池里爬上来了。

他累极了,手脚都在发抖。拉着布绳拽上白琉璃,他抖了抖拆下的被面,把上面的棉絮又摘了摘,然后用它裹住了白琉璃。白琉璃还柔软着,被他穿戴整齐后扛在了肩上。拎起银腰带和煤油灯,无心抬头望向了半空中白琉璃的灵魂:``不要伪装月亮了,跟我走,陪我挖地道去!''

地堡内果然干净了,连黑蛇都失了踪影。无心清理了香川武夫等人留下的工事和残尸。在地道入口外挑了一块平整地方,他就地捡了一件军大衣铺好了,把白琉璃放在了上面。工兵铲子也是随处可见的,他就近抄起一把,在入洞之前,又仔细审视了白琉璃。

煤油灯的光芒毕竟是微弱,黯淡光线掩盖了白琉璃脸上的死亡颜色。他的神情很平静,长眉舒展,双目紧闭,合下漆黑的睫毛。无心看了又看,最后就对着白光说道:``月亮,你看看你,多漂亮啊!''

白光没理他,于是他一头钻进洞里,土拨鼠似的开挖了。

无心刚一进洞,远方暗处忽然闪现了一个小小的人影。

马俊杰的鬼魂凝视着煤油灯前的一团白光,一动不动,单是凝视。

他已经趁乱吞噬了好几只游魂,可是对于白琉璃,他没胜算。白琉璃的鬼魂邪气很重,人和鬼都能感觉得出,只有无心习惯成自然。

良久过后,他在虚空中消失了。

无心吭哧吭哧的挖了一天多,直到力不能支了才退出地道。土猴似的靠墙坐了,他发现白琉璃已经隐隐幻化出了人形。

人形不是他往昔的形象,是洗过澡后,无心口中的``漂亮''模样。一头长发看起来甚至还是湿漉漉。影影绰绰的悬在空中,他居高临下的审视无心,看起来严肃而又胸怀大志,很有地堡主人的派头。无心扬手摸了他一把,当然是摸了个空。手指从鬼影中穿过,无心疲惫不堪的闭了眼睛,一歪头就睡着了。

打了个短短的盹后,无心揉着眼睛爬起来,从皮袄口袋里掏出肉罐头吃。吃着吃着抬起了头,他问上方的鬼影:``看什么?''

白琉璃的眉目越发清晰了:``我死了,你还没有给我念过经。''

无心鼓着一边面颊嚼肉罐头:``你不是不爱听吗?''

然后他扔开空罐头盒子,抄起铲子又道:``不念了,念不动了。我干活去,你守着你的尸首。要是有蛇来了,你进洞里找我。''

摇头摆尾的钻进地道,他用脚向外蹬出了两堆土。地道深处隐隐响起了一段地藏经,声音模糊而又沉闷,仿佛和洞外隔着千山万水的距离。白琉璃静静听着,直到无心的调门忽然拔了个高!

寒冷的空气缓缓倒灌进了地堡,经文中断了,换成无心惊喜的大叫:``通了!通了!''

片刻之后,地道入口慌乱的伸出两只脚。无心蜷缩着退出地道,回身抱起白琉璃的尸体,口中说道:``我要走了。你给我的银腰带,我也揣好了。你还有话吗?有话就说。''

越来越清晰的鬼影悬在空中,白琉璃注视着无心摇了摇头。

无心定定的又看了他一眼,随即忽然笑了,一边笑,一边挥了挥手。搂着尸首跪在入口前,他不再回头,径直的爬了进去。

地道倾斜向上。无心伸出头时,正好看到了天边第一缕朝霞。这是个晴朗的冬日清晨,几只喜鹊在附近的枯树枝上叽叽喳喳。

单手撑地出了地道,他在白皑皑的大雪地上站直了身体。白琉璃的尸首还压在他的肩膀上,他回头去看小小的出口。白色大地上,黑洞洞的出口深不可测,仿佛是大山的一处伤口。

无心放下白琉璃,搬开一块大石堵住了出口。大石微微陷下,将来会和地面齐平。等到春暖花开了,地面长出一片青草,出口就会彻底消失。

树上只有喜鹊和麻雀,连只鹰都瞧不见。无心抱起白琉璃往林子里走,一边走一边东张西望。末了停在四棵笔直秀丽的白桦树之间,他弯腰放下了白琉璃。

以四棵白桦树为支柱,他从附近老树上折下长枝,一层一层纵横架在白桦树的枝杈上。眼看树枝搭成的四方平台足够结实了,他把白琉璃放了上去。

整理好了白琉璃的长发,他后退几步跪下了,把方才未唱完的地藏经唱到结束。起身打扫打扫身上的土和雪,他辨认清了方向,然后踏上了下山的路。

无心不知道自己在地堡里到底耽搁了多久,所以也不确定山下林子里是否还会有人等待自己。有人等当然好,没人等也没关系。在活地狱里走了一圈之后,他现在心中无欲无求,十分坦然。

一步一个脚印的走在雪地里,他简直快要拖不动自己的两条腿,然而又不能睡,一旦真睡着了,兴许醒来时胳膊腿儿就冻硬了。千辛万苦的挪到林子里,他扶着一棵松树弯下腰,抓起一把雪塞进了嘴里。

他渴极了,雪进了嘴,竟然是冰凉的甜丝丝。伸手再抓一把雪,他低着头刚要张嘴,忽然听到前方响起了一声尖叫。

他当即抬起了头,就见赛维张开双臂直冲而来,直把他撞了个仰面朝天。未等他去拥抱压在身上的赛维,半空中又起一声呐喊。胜伊从天而降,结结实实的扑到了赛维的后背上。两张脏兮兮的面孔一起凑到无心眼前,四只冰凉的手一起拍打了他的头脸。赛维和胜伊欢天喜地的大叫大嚷,各说各的。胜伊的嗓门很高,居然盖过了赛维,于是赛维一胳膊肘把他杵开,随即捧着无心的脸亲了一口。胜伊爬了上来,闹着叫道:``我也亲一下!''

无心抬起头,让胜伊也亲了一下,同时听赛维说道:``我们天天往山上望,总算把你盼回来了!你知不知道你走了多久?''

不等无心出声,胜伊作了回答:``十多天啦!''

赛维拍拍心口:``后来我们两个都害怕了。''

无心笑问:``怕什么?''

赛维给了他一拳:``你说呢?''

无心仰卧在白雪中,对着赛维和胜伊说道:``幸不辱命,我是地堡里唯一的活口。''

赛维微笑着看他,看他是个大英雄。往后的道路就是大家齐步走了,她可不想再让无心独自历险。

一挺身爬起来,她伸手拉扯了无心:``走,我们去见爸爸。爸爸昨天还说呢,只要你能成功,他就有办法带我们下山回北京。''

\chapter{离开山林}

在树林深处的仙人柱里,无心见到了蓬头垢面的马老爷。

马老爷是个随遇而安的人,到了什么山头唱什么歌。手里端着伊凡给他的小茶缸,他舒舒服服的偎在火塘旁边,丝毫不肯委屈了自己的一把老骨头。冷不丁的见无心回来了,他欢乐至极,险些把一缸子热茶全泼到了火塘里。拿出笼络伊凡的手段,他把无心拽到身边嘘寒问暖。听闻自己的敌人全在地堡里上了西天,他快活得仰天长笑,对着仙人柱顶端的圆孔好一串哈哈哈,震得仙人柱外的小鸟都飞走了。

无心已经把马老爷的底细了解了个七七八八,此刻冷眼旁观,就感觉马老爷嘴脸丑恶,不堪入目。但还是那句老话——帮人帮到底,送佛送到西。横竖已经走到今天这步了,不差最后一段路途。

赛维打湿了一条大手帕,扳着无心的脑袋给他擦了把脸。擦着擦着忽然停了动作,歪着脑袋细看:``鼻子里面塞了什么?''

无心堵住一边鼻孔,用力向外出气,结果喷出了一只小棉花球。将另外一只鼻孔里的小棉花球也喷到火塘里了,他颇为尴尬的望着面前众人发笑。胜伊好奇的蹲在火塘对面:``你堵着鼻子干什么?不憋得慌?''

无心讪讪的没有回答——他是把堵在鼻孔里的棉球给忘了。

幸而大家都不在意。赛维问胜伊:``伊凡给你的驯鹿奶呢?别小气,拿出来给他喝点!''

伊凡钻出仙人柱,从外面端回一只小铁盆。铁盆里是他用驯鹿奶冻成的冰激凌,虽然看起来和冰激凌毫无关系。铁盆放在火塘上燎了燎,赛维抄起一把匕首,把盆中的奶冰扎了个稀碎。而马老爷见无心已经拿着勺子吃起冻鹿奶了,便用长长的小手指甲敲了敲茶缸,开口说道:``明天,我们就可以下山去了。''

转动脑袋环视了面前的晚辈们,马老爷含着笑容,被自己的智慧所折服:``香川他们一完蛋,导致了个什么局面呢?''

马老爷顿了顿,对于无人回答的情形也很满意。伸出巴掌展开枯瘦的五指,他继续说道:``四个字,死无对证!''

津津有味的喝了一口热茶,他悠悠的道:``宝藏,巫师,诅咒,灵魂\ldots{}\ldots{}日本人对此很感兴趣啊,稻叶大将最感兴趣啊!可是他们的人都死了,只有我们活着。你说,日本人敢轻易杀了我吗?''

所有人都摇了头。

马老爷点了点头:``你们听好了,做人哪,最要紧的就是要有价值。有价值,就有发言权,就能做文章!''

赛维迟疑着说道:``爸爸,可是到了北京之后,我们的文章迟早会有结尾的一天\ldots{}\ldots{}''

马老爷微笑着摆了摆手:``我们不能让它结尾。文章只是个幌子,让日本人给我们一点时间。我们有了时间,就有活路。天下之大,只要我们肯隐姓埋名,哪里不能去?爸爸这些天已经盘算出大概的眉目了。你们放心,等着瞧好吧!''

然后他转向无心,莞尔一笑:``辛苦你了,你是我们的恩人啊!''

无心嘴上一圈奶渍,舌头也冻麻木了,有心谦逊几句,又不是很想理睬马老爷。幸好赛维跪到他的后方,伸手一勒他的脖子。他趁势向后一仰,借着玩笑含糊过去了。

赛维一直勒着无心,不是勒脖子,就是勒手臂,总之是一刻都不肯放松。胜伊出了仙人柱,骑着大驯鹿去找伊凡。额上带着一片白毛的大驯鹿已经和胜伊很亲近,但是胜伊天生胆小,上了鹿背便是向前一趴,双手抱着驯鹿脖子不敢放。等到驯鹿跑到了伊凡的仙人柱外停了蹄子,他不会下鹿,自己试探着倾斜身体,最后``咕咚''一声滚落到松软的白雪中。

伊凡在手心里涂抹了盐,正在让他的驯鹿们舔。听说无心平安归来了,他真心实意的很喜悦,想要杀一只小驯鹿庆祝。胜伊拼命阻拦了,于是伊凡只好翻出了一大块冻硬了的熊肉。先把胜伊抱上驯鹿背,伊凡随后带着酒肉也骑上了驯鹿。两人一前一后的走了一里地远,到达仙人柱时,马老爷还在展示自己的厚黑之学,无心听也不是,不听也不是,便和赛维一递一句的搭着话,两人想要找机会一起溜走。偏巧伊凡及时赶到,无心和赛维听着仙人柱外的欢声笑语,当即对了个眼色,然后一窝蜂的全出去了。

虽然伊凡绝不能成为马老爷的知音,但马老爷看他善良得像头怪物似的,倒是真挺喜欢他。因为明天就要下山了,马老爷无以为报,只好搜罗全身上下,把一只金壳子怀表和一尊连着金链子的、指节大的翡翠菩萨给了他。其中翡翠菩萨是贴身挂着的,水汪汪绿盈盈,还带着体温。马老爷郑重其事的告诉他:``记住,可别把它轻易送人。放到齐齐哈尔,它值一所小房。''

伊凡把菩萨挂在脖子上了,挺高兴,也挺茫然:``可以用它换盐和布吗?''

马老爷望着天想了想,只觉一言难尽:``算了,你仔细留着它,将来传给你的孩子吧。''

伊凡玩了一会儿怀表,末了把它还给了马老爷,因为不知道要它何用。生起一堆熊熊的篝火,他开始切肉烤肉,又问无心:``巫师的灵魂,真复活了吗?''

无心喝着他的烈酒,因为怕吓着他,所以只答:``活是活了,但又死了。不过你可别往山腰走,还是\ldots{}\ldots{}不很安全。''

伊凡对于鬼神素来是敬而远之,所以十分听话,绝没有登山探险的意愿。

熊肉上面细细的抹了一层盐,烤到半生不熟的时候,就被伊凡送进了嘴里。在十几天的时间里,他已经和赛维相熟。赛维不爱他,不爱就不爱吧,有出息的小伙子,不该因为没被姑娘选中而愁眉苦脸。伊凡只是把最嫩的肉全给了她,她不主动对他说话,他也不搭讪。

从白天闹到黑夜,夜里无心陪着酒醉的伊凡跳舞。伊凡知道他们要走了,所以格外的撒欢,东倒西歪的跳进了火堆里,幸亏无心眼疾手快,一把将他又拽了出来。伊凡的皮袍没有燃烧。在雪地上跺了跺脚,他继续跳。

仙人柱前弥漫着浓烈的酒肉香气,直到凌晨才散。伊凡小睡片刻,清醒之后双手抓雪擦了擦脸,然后抖擞精神,把马家几人全送上了驯鹿背。领着道路下了山,他在山脚的营地里,见到了他部落里的亲人。

马家众人下了驯鹿,和伊凡道了别。继续给他们做向导的人,是伊凡的朋友达西。达西是个矮墩墩的邋遢壮汉,只会讲有限的几句汉话。伊凡的朋友就是他的朋友,他当仁不让的上了路,从山林一直向外走到了最近的屯子里。

屯子里驻扎了一大队日本兵,自成一统的圈地建了兵营。达西挨过日本人的欺负,所以不肯靠近营门,只远远的指明了方向。马老爷看清楚了,转身对着达西拱手抱拳道了谢,随即昂起头清了清喉咙,摆出一副如丧考妣的哭丧脸,一步一步慢慢走向了营门。

赛维等人受过他的吩咐,此刻也是垂着头。营门两边的日本兵看马老爷造型奇特,满脑袋都是卷毛,就瞠目结舌的盯着他瞧。他都走到营门口了,两名日本兵才反应过来,当即大喝一声。日本兵脚边的大狼狗本来是在晒太阳打瞌睡,此刻随着士兵的暴喝也起来了,对着马老爷狂吠不止。

马老爷背了双手,不抬眼皮的说了一句日本话,当即震住了兵与狗:``我是稻叶新之助大将派出的特使。我们的勘探小队在距离本屯几十里外的雪山里,遭遇了灭顶之灾。''

十分钟后,他们见到了营中最有权威的犬神少佐。对于犬神少佐,马老爷依旧是面如死灰,并且不甚客气,直接要求他向天津军部发电。犬神少佐有点迷糊,因为稻叶大将是华北方面军的大将,而他犬神少佐是关东军的少佐。马老爷看出了他的迷糊,于是进一步的自报家门,沉着一张老脸自吹自擂,恨不能把自己抬到汪精卫陈公博的高度。

一个小时后,犬神少佐亲自往海拉尔军部发去电报,而电报当天又转去了新京总司令部。不过一夜的工夫,犬神少佐便接到了最新军令。

在伸手不见五指的凌晨时分,少佐派出营中一辆小汽车,要把马家众人直接送去海拉尔,还有一队骑兵随行做保镖。马老爷怀着满腹主意,一宿没睡。此刻在灯火照耀下,他板着脸往车里钻。一屁股在后排坐下了,他抬起头吁了口气,忽然一愣,随即扭头望向身边。

身边没有人。胜伊坐上了前方的副驾驶座。赛维在车外,还没来得及往车里钻。

马老爷用力眨了眨眼睛,认定自己是产生了幻觉——方才在汽车后视镜里,他恍惚看到了马俊杰。

赛维带着一身寒气上了车,坐到后排中央。无心紧跟着也坐上了,坐上之后,他东张西望的抽了抽鼻子。

赛维现在特别的爱他,一听他有动静,连忙问道:``是不是冻着了?''

无心心不在焉的摇了摇头。在进入车内的一瞬间,他仿佛嗅到了一丝阴寒气息,可是车里很干净,并无异常。

关严车门坐定了,他从怀里抽出了白琉璃留给他的银腰带。腰带刻着莲花纹路,通体黑得像煤。无心闲来无事,就用一块粗帆布缓缓摩擦着银腰带,想要把它擦出本来面目。他一边擦一边看了赛维一眼,赛维近来由于吃了太多的肉和油,居然胖了。不但胖了,皮肤也糙了,然而透出一层血色,反倒看着比先前的模样更生动。无心对她的要求一贯不高,因为感觉她是个刺儿头。她要真出落成了美人,非得兴风作浪不可。

汽车拖着骑兵尾巴,从黑夜驶入黎明。马老爷依靠车门假寐,赛维也枕着无心的肩膀睡了。无心收起了银腰带和帆布,闭上眼睛不言不动。前方的胜伊忽然大叫一声,吓得司机一哆嗦,却是他做了个噩梦,惊着了。

一行人抵达海拉尔之后,即刻登上军用飞机。没等马老爷把下一步的计谋筹划清楚,飞机已在天津东局子机场着陆。出了舱门走下舷梯,马老爷略微调整了表情,从肃杀转为惶恐。像个精神病人要发病似的,他一惊一乍的蓬着头发,莫测高深的直接去见稻叶大将了。

\chapter{勾魂}

正如马老爷的预料,稻叶大将被他玄之又玄的描述给震住了。

他要发疯似的哆嗦在大将面前,神情和语气都是受过大惊吓的模样。一段地堡历险记被他说得前言不搭后语,然而态度是非常的认真,认真的让稻叶大将暗暗冒冷汗,几乎怀疑马老爷也被鬼魇住了,恨不能当场一把火烧了他。

因为的确是死无对证了,所以稻叶大将暂时安抚住了马老爷,转而又去亲自面见了赛维胜伊以及无心。赛维和胜伊谨遵父亲的教诲,像两只绝望的病鸡崽子一样,伸着脖子驼着后背塌着肩膀,在稻叶大将面前有一句没一句的胡说八道。稻叶大将问得急了,胜伊就闭上眼睛不言语了,赛维更有一点表演的天分,瞪着眼睛对着大将发呆。

大将怀疑马家的人全吓出了心病,于是把注意力转移到了无心身上。据他所知,无心是个阴阳师一流的人物,想必不该害怕鬼神。可是面对面的交谈了一阵之后,大将很不舒服的闭了嘴。无心满嘴鬼话,每一句都令人毛骨悚然;问他人事,他睁着一双黑眼睛,却是一问三不知。

在大将一头雾水之际,马老爷又发了话,说要回家;还说此行千头万绪,他要回家休养几日,顺便把探险经历写成报告,呈给大将。

大将,由于认为自己还可以从干巴巴的马家人身上榨出些许养分,所以没有翻脸。既然不想翻脸,他便走了另一个极端,春风一样向马家众人送了暖。马老爷要回家,他就派出一辆汽车,把他眼中的四个精神病运往了北京。

在从天津到北京的路上,无心坐在汽车后排的座位上,一边慢慢擦拭着手中的银腰带,一边狐疑的东张西望。

汽车内总是残留着几丝地堡特有的阴寒气息,可是在他目光所及之处,却又并无鬼魂的踪影。他犯了嘀咕,又不能对旁人说,因为无凭无据,随便吓唬人也不对。

赛维知道大家虽然能回北京了,但远远没到平安大吉的程度。歪着脑袋偎在无心肩膀上,她直着眼睛出了神。无心的手指很灵活,正在捏着一块粗布摩擦莲花纹路。赛维盯着他白里透红的指尖,心中茫茫然的想:``指甲修得真好。''

半天过后,他们抵达了北京马宅。

他们总共也只走了一个来月,可出发时是秋季,马宅还有秋菊红叶装饰着;如今顶风冒雪的回了来,进门之后便是满目苍凉。既然马老爷并没有死,那马宅的规矩就不能变;留守的上下人等一起迎接出来。管家又偷偷的告诉马老爷,说是四姨太和家里的汽车夫私奔了,除了她自己的体己钱,旁的倒是没卷走什么。

马老爷点了点头,对于四姨太兴趣不大。马宅前后依旧是不缺少日本兵,后花园子则是成了一处小兵营。四面八方都是眼线,马老爷坐在书房内的写字台后,让管家去把门关上。等到管家关门回来了,马老爷把一张写满小字的信纸推到了他的面前。

管家拿起信纸一瞧,脸上立时变颜变色。从马老爷手中接过铅笔,他拉把椅子坐下来,开始在纸上回应。

与此同时,赛维和胜伊洗了澡换了衣裳,揽镜自照,都认为自己很需要一番修饰。胜伊嫌天冷,想要打电话让理发匠登门服务。夹着电话簿子走到赛维屋里,他和赛维讨论了当下的摩登发型,又说:``我可不想剪得太短,头发一短就不听话。姐你呢?你还烫吗?别烫了,你看你头发梢都烫黄了。''

赛维摸着头发,正要回答,可是心思比语言变化更快:``无心呢?''

胜伊伸手向外一指:``在我屋里擦银子呢。''然后他向赛维探了头,压低声音问道:``姐,你说他怎么不变模样啊?''

赛维也疑惑,轻声答道:``我也发现了,他\ldots{}\ldots{}他好像总是一个样儿。''

胜伊又道:``他是不是练什么功夫练得走火入魔了?你看他的头发从来都不见长,脸上也没胡须。没胡须倒没什么的,我脸上也挺干净,可是无多有少,下巴和嘴唇上总该有几根吧?我观察过他了,他真的是一根毛都没有。''

赛维沉吟着答道:``也有一根胡子都不长的人\ldots{}\ldots{}比如五姑父。''

胜伊点了点头:``对,可能他像五姑父,年轻的时候脸很光溜,越老越糙。''

赛维一听就不乐意了:``去你的吧!''

赛维和胜伊不声不响的打电话叫了一名理发匠,想要美化自己的形象。与此同时,无心趁着他们不留意,悄悄溜出院门,想要去找大太太佩华。

马宅太大,他虽然知道佩华是被打入冷宫的人物,但是冷宫在哪里,他不知道。沿着道路走向僻静处,他想佩华完全就是马老爷手边的一件摆设,而且还是一件失了宠犯了罪的摆设,一定享受不到什么好待遇。

然后他一抬头,骤然和佩华打了个照面。

佩华像一块不带滋味的面点心,平平淡淡的端庄着。无心正想着她,不料想着想着想出了个活人,就是一惊。而她站在青石板路上,对着无心微微笑了一下:``无心师父。''

无心也一躬身:``大太太。我有话——''

在他出声的同时,佩华也开了口:``我有话——''

两人异口同声的抢了话,随即又一起收了话音。无心对着佩华一点头:``大太太先说吧。''

佩华低下了头,轻声问道:``无心师父,我想问问大少爷的事——大少爷回来了吗?''

无心没有办法把马英豪的死讯说得婉转动听,所以在短暂的思索过后,他索性斩截答道:``他死了,是被手雷炸死的。爆炸前他和我在一起,让我给你带几句话。''

佩华本来就站得稳当,此刻听了一个``死''字,越发纹丝不动,人都成了塑像。等到无心把马英豪的遗言尽数转述了,她低低的``哦''了一声,仿佛脖子都僵硬了。

像个小面人似的,她规规矩矩的站在寒风里,也没有眼泪,也没有哽咽,单是站着。良久过后,她才慢吞吞的又问:``是\ldots{}\ldots{}一下子就走了吗?''

无心很笃定的告诉他:``是,手雷厉害,一下子就走了。''

佩华忽然晃了一下,抬眼望向无心,像个小女孩要求大人的保证似的:``不疼吧?''

无心坚定的摇头:``不疼。一秒钟的事,觉不出疼。''

佩华的一双眼睛渐渐闪烁出了水光:``走之前\ldots{}\ldots{}遭罪了吗?''

无心继续摇头:``没有。''

佩华对着无心浅浅一躬,声音轻飘飘的:``无心师父,谢谢你。''

佩华一步一步慢慢的往回挪,一直挪进了她的冷屋子里。

她在床上坐定了,眼泪在眼眶里转了几个圈儿,最后风干了,干得眼珠都苦涩。

她不叫人,老妈子也不出现。她一直坐一直坐,心里就想她和马英豪是怎么认识的,怎么相好的。马英豪不是个好伺候的,脾气也有点怪,时常对她不冷不热。她心里没有底,真被他折磨透了。

现在好了,再没有人能折磨她了。

光线黯淡的屋子里,忽然缓缓现出了一个熟悉的小影子。佩华抬了头,恍惚中看到了马俊杰。

``五少爷\ldots{}\ldots{}''她喃喃的说:``你不是死在外头了吗?''

马俊杰若隐若现的站在暗中,对她发笑:``我死了,大哥也死了。妈,你要不要来?你来了,就能看见大哥了。''

佩华梦游似的扶着床柱站起身:``我能看见英豪?''

马俊杰站在可望不可即之处,笑得十分可爱:``大哥死了,你也去死,你们就能永远在一起了。''

佩华的脑筋像是锈住了,丝毫不能转动。迷茫中听了马俊杰的话,她想马俊杰说得有理,为什么有理?不知道。反正自己得死,死了,就能看见英豪了。

踩着凳子上了高,她亟不可待的将一条尼龙带子挂上了床梁。脑袋伸进绳套里,她把脚下的凳子一踢。两只脚本来还可以踩上床沿的,但是小鬼的话始终在她耳中回荡,让她心甘情愿的伸直了腿。

马俊杰虎视眈眈的等待着。佩华的魂魄刚一离体,就被他全吞噬了。

无心躺在胜伊的身边,摸着黑擦腰带。马家人多眼杂,他反倒要和赛维保持一点距离。

他总感觉马宅有鬼,而且不是善茬。可鬼在哪里,他不知道。鬼仿佛无处不在,然而只躲着他。

翌日清晨,马老爷在床上听闻了佩华的死讯。戴着他的绣花小帽垫坐起身,他先是下意识的骂了一句:``贱货,还要闹殉情吗?''

话音落下,他若有所思的发了一会儿呆,随即猛的一拍手,脸上现出喜色。把他最信任钟爱的大管家叫到卧室,他嘁嘁喳喳的好一番嘱咐命令。而大管家出了卧室之后,立刻宣布了老爷的旨意,要为太太大办丧事,顺带着把冻在医院里的八姨太也一并捎上,再给死无全尸的大少爷和五少爷造个衣冠冢。

马家的人受着监视,但合理出入还是没有问题。管家每天穿梭似的里一趟外一趟,趁乱往外运出了大批黄金。黄金的终点站是上海。马老爷有个老姐姐在上海。老姐姐对弟弟的感情,和妈妈对儿子也差不多,即便弟弟是个天怒人怨的货色。

\chapter{险境}

赛维见无心天天擦银腰带,就给他拿来了一盒牙粉,让他用湿抹布蘸着牙粉擦,保准马上擦成雪亮。无心随口说道:``不用,我慢慢擦,反正闲着也没事做,正好打发时间。''

赛维描眉画眼的站在他面前,手托着牙粉盒子想了想,感觉无心的回答有点不对劲。

片刻过后,她放下牙粉盒子,对着无心露出的后脖颈抽了一大巴掌:``我在你眼前哪,你竟然闲着没事做?''

无心猝不及防,被她打得浑身一哆嗦,险些把银腰带扔了。仰头望着赛维眨巴眨巴眼睛,他拍了拍自己的大腿:``请坐?''

赛维一屁股压上了他的大腿,背对着他怒道:``我坐了怎么着?我坐也是应当应分!你都是我的,何况你两条腿!''

无心把额头抵上赛维的后背,一边擦腰带一边附和:``随便坐,欢迎坐。''

赛维来了月事,身上冷,小肚子疼,导致性情异常暴躁,没事还要找事,如今事情到了眼前,正合了她要发疯的心意。无心算是落了网,被她狠狠揉搓了一顿。而赛维大耍威风,正是得意之时,管家忽然来了,说是老爷请二小姐过去说话。

赛维一走,无心得了大赦。坐在椅子上静静的发了一会儿呆,他末了摇了摇头,又叹了一声。

马老爷对赛维说了什么,无人知晓。反正赛维天黑才回,进院之时谈笑风生,是个兴致很好的样子。胜伊则是窝在自己的卧室里蒙头大睡,赛维让他出来吃新鲜的巧克力蛋糕,他隔着一层棉被``哼''了一声,闷声闷气的不肯动。

赛维脾气好的时候,是真好。隔着一张小炕桌,她问无心:``生不生我的气?''

无心切着蛋糕答道:``不生气。''

赛维轻轻拍了拍他的手背:``别生气,我给你赔个不是,往后我再也不欺负你了。''

无心抬眼向她一笑,低声说道:``孩子话。''

赛维怔怔的看着他,心中十分后悔,悔不该白天对他连打带骂。

无心在赛维房里吃过蛋糕,因见天都黑透了,便要回胜伊房里睡觉。穿过小院推开了西厢房的房门,他经过外面的小房间,进了里间卧室。

蛋糕太甜了,所以他摸黑站在窗边桌前,轻手轻脚的给自己又倒了一杯冷茶。端着茶杯转向大床,他忽然发现床上被褥凌乱,胜伊不见了!

放下茶杯走到床前,他伸手一摸床边位置,感觉还有余温。转身大踏步冲出房屋,他迅速返回了赛维所在的东厢房。赛维正坐在梳妆台前,用小块棉纸蘸了冷霜擦脸,忽见无心冒冒失失的闯进门来,她愣眉愣眼的起了身:``怎么了?''

无心停在门口:``胜伊晚上出去了?''

赛维连忙否认:``他不是在房里睡了一整天吗?刚才我让他起床吃蛋糕,他还不愿意呢!''

无心脸色一变:``卧室里没有他。''

随即他上前抓住赛维的手腕:``你不要落单,跟着我走。我们一起去找胜伊!''

无心知道胜伊一定没走远,而赛维一边往院外走,一边高声问丫头看没看见三少爷。冬夜严寒,丫头们早都各回各位的歇息了,当然是一问三不知。接连几日都是晴天,地上只有下午落的一层薄雪。赛维临出门时提了一只小花灯笼,灯笼里面放着干电池和小灯泡,是个玩具似的小玩意儿。借着灯光仔细观察了地面,她忽然``咦?''了一声。

无心顺着她的目光望下去,发现地面上印了一个清清楚楚的人脚印,从形状尺寸来看,正是胜伊所留。

赛维惊讶了:``怎么?他出门\ldots{}\ldots{}没穿鞋?''

无心辨认了脚趾方向。胜伊的双脚大概是带着相当的热度,以至于他脚下的冰雪先融化后结冻,起初的几个脚印是特别的清楚。

``我怀疑宅子里还是不干净。''他压低声音对赛维说道:``好像有东西跟着我们,从山林一起回来了!''

赛维没出声,只瞪着眼睛向他做了个口型:``鬼?''

无心点了点头:``可是我始终看不到它,它好像一直在躲着我!''

赛维为了胜伊,是可以拼命的。此刻她深深的吸了一口气,镇定情绪之后轻声说道:``如果是它要害胜伊,恐怕见了你还是要躲藏的。我在前边走,你偷偷跟着我,见机行事,好不好?''

无心别无他法,只好答应。于是赛维亟不可待的转了身,大致的辨清了方向之后,她心急火燎的迈开了大步。走出不远,她忽然发现自己的目的地已经注定——只要不向两边花木丛中乱钻的话,道路尽头不就是花园了吗?

不祥的预感几乎压得她要呕血。她提起一口气开始小跑。小肚子里像是兜了一块生铁,沉甸甸的胀痛;手脚也没力气,虚汗顺着鬓角往下流。她只庆幸自己食欲还好,刚刚吃了一大块巧克力蛋糕。

小花灯笼像流星一样掠过黑暗,赛维的速度越来越快,小跑在不自觉间转成了狂奔。一个箭步越过横在地面的一块凸起山石,她落地之时腿软了一下,感觉自己一腔的鲜血都被震下来了。

寒冷的风刮过她的面颊,她像匹矫健的小母马,一路跑得四蹄腾空。花园多么的大,谁知道胜伊在哪里?甚至谁又知道胜伊是否真的在花园?赛维连方向都不辨了,凭着直觉冲向河边。小河对岸的山上修建了简易房子,此刻房中漆黑,看守宝藏的日本兵也都睡了。一弯惨白的月亮斜在空中,在白月与黑山之间,她遥遥看到了胜伊的身影。

胜伊就站在小河中央。

赛维吓得尖叫出声——小河冬天是冻不实的,两岸浅滩倒也罢了,河流中心永远只是一层冰盖。而半薄不厚的冰盖,是承受不住一个成年人的!

``胜伊!''她在河边收住脚步,嘶声的叫:``你疯了?给我回来!''

胜伊姿势怪异的歪着脖子,歪到极致,仿佛颈骨将要折断。似笑非笑的望着赛维,他的表情并不稳定,一时像胜伊,一时又不像。

冰面起了咔咔的裂响。胜伊的身体忽然一倾,是一只脚下冰面破碎,赤脚缓缓陷入了喷涌而出的冰水之中。不等无心出现,赛维丢了灯笼向前就跑。脚下的冰面不住的成片塌陷,她伸长手臂抓向胜伊,带着哭腔狂喊:``手给我!手给我啊!''

胜伊不为所动的望着她,一张面孔渐渐扭曲,复杂表情在他脸上交替闪现。一条手臂要抬不抬的动了动,他忽然轻声唤道:``姐——''

一声过后,他倏忽间变了脸,却是诡异的笑了。一边笑,一边笨拙的拖动双腿,在塞维面前后退一步,避开了她的双手。

赛维没有意识到他是在引诱自己深入,甚至没有注意到脚下冰面已经彻底支离破碎。正在她进一步的要追逐胜伊之时,两人之间的冰面忽然自下而上的受了冲撞。一个人影顶着水花一跃向上,一把揪住了胜伊的衣领,正是无心。

手指点上胜伊的眉心,无心一边画咒一边吼道:``赛维回去!''

赛维六神无主的停住了,同时发现自己站在了一块浮冰上,已经无路可回。胜伊落到无心手里,瞬间软得没了骨头也没了意识。而赛维进退两难的低下头,就见漆黑水面上印着一弯残月,以及一张顶熟悉的面孔。

``老五!''她难以置信的抱了脑袋,两条细腿失控似的抖战:``老五?''

在她出声之后,马俊杰的影子便消失了。

无心把赛维和胜伊全救上了岸,周身湿透了,风一吹,一身衣裳立刻冻出了冰碴子。

赛维带着他急急的往回走,心想无心要冻死了,又想我如果再欺负他一次,我就不是人。

进了院后,赛维没有声张,把人全赶进了自己住的东厢房。赛维做主,扒了胜伊的湿裤子,让他躺在自己的床上昏睡。无心也洗了个热水澡,换了一身干净衣裳。手里托着一条大毛巾,无心对赛维问道:``你看到了马俊杰?''

赛维连连点头:``我在水面看到了他的影子。就像倒影一样,很清楚。''

无心若有所思的擦着脑袋:``我也看到他了,他上了胜伊的身。''

赛维勃然变色:``他——''

无心继续说道:``鬼上身不是大事,驱出去就是了。我只是不明白一点——他是怎么来的。''

他放下毛巾,抬头望向赛维:``平常的小鬼,没有力量作祟。马俊杰刚死了不到一个月,怎么可能——''

他欲言又止的换了说法:``照理来讲,他一出地堡就该魂飞魄散了。''

赛维说道:``他\ldots{}\ldots{}他可能和别人不一样吧?你看他活着的时候也像个小鬼。可我们并没有害过他,他为什么要杀胜伊?他今天害了胜伊,明天是不是该害我了?''

无心没敢说``鬼怕恶人''四个字,怕赛维发飙,只说:``你还好。你比胜伊厉害,鬼也是欺软怕硬的。''

赛维给宅子前头的马老爷打了电话,有一说一,说得马老爷面如土色。

马老爷失眠一夜,翌日起床定了主意,抄起电话联络上了稻叶大将。字斟句酌的交谈一番之后,当天上午,一大队日本兵开进了马宅后花园。

马老爷打算让日本兵的凶气镇一镇马俊杰的邪气。而日本兵并没有意识到自己的妙用,他们只是分批下入地洞,搬运起了洞中古董。

因为动作太小心了,导致他们的速度很慢。马老爷远远的过去瞧了一眼,看他们从地洞中运出的全是大大小小的陶疙瘩。陶疙瘩并不能让马老爷动心,他素来喜欢直观的刺激,比如钞票的颜色,或者是金银的光芒。

第一部分报告书已经写完,并且送到了稻叶大将面前,马老爷计算着时间,认为自己还有十天半月的准备期,时间太多了,根本不需要。

胜伊昏睡了一夜一天,最后在一个阴霾的傍晚醒了。

他患了重感冒,两只鼻孔全不通气,被鬼上身前后的事情,也记不得了。

赛维也伤风了,并且腰酸肚子痛。裹着毛毯坐在床尾,她小声说道:``胜伊,家里不太平,我们真得快点走了。''

胜伊打了个喷嚏,病怏怏的起身坐到了赛维面前:``时间定了?''

赛维点了点头,声音轻成了耳语:``差不多。''

胜伊又问:``带无心吧?''

赛维理直气壮的答道:``当然带。爸爸说等我们在昆明安顿好了,就举行婚礼。''

\chapter{复仇}

马老爷用手捂着心口,独自坐在大床上发呆。鎏金床柱反射了水晶吊灯的明烈光芒,马老爷的卧室,素来装饰得偏于辉煌。

他是怕黑的,而在有大动作之前,又是格外的谨慎,甚至不肯叫个姨太太来陪睡。两厢相加,导致他方才做了个噩梦。下意识的抬手摸向胸前,他摸了个空,想起自己护身的翡翠菩萨早送给伊凡了。

曳地的厚呢窗帘,因为沉重,所以纹丝不动,让马老爷联想起一面居心叵测的夹壁墙。掀起棉被下了床,他穿着绣花软拖鞋来回走了几圈,忽然想起了自己死去的小儿子。烦躁的一撇嘴,他转身绕到了床尾。床尾距离墙壁还有一大片空间,于是对着大床摆了一只西式立柜。立柜门上嵌了一小块装饰用的梅花形玻璃镜,他对着镜子仔细审视了自己的面容——新剪过的卷发挺服帖,而一张面孔,他自己认为,也并未见老。

用长长的小手指甲刮了刮鬓角,他披上白底蓝花的睡袍,给自己倒了一杯茶。一口一口慢慢喝了,他无端的叹了一口气,后背凉飕飕的,心情也低落。

``五个孩子,如今就剩了两个。''他端着茶杯站在窗帘前,漫无目的的想:``政治生命也将要彻底结束了。''

他突然想哭,一边想哭,一边暗暗的惊讶,不知道自己的伤感是从何而来。他的头脑素来是条理分明,一生不知冲动为何物。

慢慢的把茶杯放到桌上,他脑海中浮出了一个新念头:``活着没意思啊!''

苍凉的长叹一声,他对着虚空点了点头。想起自己将要背井离乡,还不知道能不能平安跑出日占区。跑不出去,必定是死路一条;跑出去了,也无非是养老。没意思,真是没意思。

马老爷把双手插进睡袍口袋里,含着一点眼泪缓缓的踱,想自己死了倒比活着更享福。末了靠着床尾栏杆站稳了,他一抬头,又从梅花镜中看到了自己。

眼中的泪光让他骤然震惊了,他心思一动,立刻做了反省:``我在胡思乱想什么?''

然后他打了个冷战,关灯上床去了。

灯光一灭,富丽堂皇的卧室立刻堕入黑暗。梅花镜中浮现出了马俊杰的面孔,他的脸上没有表情,一双眼睛斜出去,盯着镜子里的大床,以及床上的马老爷。

马老爷没睡好,凌晨就起了床。下地之时他忽然打了个冷战,就像被寒风吹了光身子一样,汗毛竖起一大片。

吃饱喝足之后,他裹着貂皮褂子去了后花园,遥望小河对岸的动静。小河对岸的日本兵换了一批,其中有好些便装人物,干干净净架着眼镜。士兵们也全戴了白手套,昼夜不停的入洞出洞。马老爷看了良久,末了发现他们在搬石片。

马老爷掐指一算时间,认为此刻稻叶大将对自己没起疑心,家里的日本兵们也正把精力全放在陶疙瘩和石头片子上,此时不逃,更待何时?

马老爷把赛维叫到面前,父女二人关了房门,做了一场秘密的长谈。出了马老爷的书房,赛维回到自己院里,开始悄悄的收拾体己——她和胜伊两人的私房钱,全由她一人代管了。

无心坐在一旁,先是静静的擦腰带,擦着擦着犯了嘀咕,偷偷去看赛维。赛维忙死了,他却闲死了,这可不是个好形势。万一赛维意识到了,很有可能大发淫威。

赛维说话不算数,昨天又欺负了他,完全不占理,还做狮子吼。无心也说不上自己是更爱她还是更怕她,反正目前看来,他不是很敢独自坐在赛维身边。

赛维留意到了他的窥视,忙里偷闲的向他一笑,然后手里托着个小算盘,念念有词的进行计算。算着算着,她转向了无心:``你总看我干什么?我不用你陪,你如果坐着无聊,可以找胜伊玩;胜伊不是刚收到了一沓子新杂志吗?你向他要几本去。''

无心听她和声细语,戒备心立刻就放下了:``不用管我,我坐得住。''

赛维凑过来,很亲昵的兜头摸了他一把。

赛维避着外人的耳目,做贼似的忙了两天,最后收拾出一只粽子似的小皮箱。到了这天傍晚,她抄起内线电话,打到了马老爷的书房。因为害怕电话已经受到监听,所以她打了暗语,只说胜伊的感冒彻底好了,晚上想吃烤鸭子呢。

马老爷的声音有些微弱,然而言语很清楚,说是厨子手艺不行,让管家出门去把烤鸭子买回来吃。

赛维听了马老爷的回答,登时安了心。挂断电话之后,她对围在一旁的胜伊和无心低声说道:``管家马上要出发了。我们还是按照原计划,夜里走暗道。''

胜伊又恐慌又兴奋的搓了搓手:``姐,好刺激哦。''

赛维没理他。一只手搭在电话听筒上,她不知怎的,很想再给马老爷打个电话。可是打通了也无话可说,还可能引起父亲的误解,以为她这里出了什么意外。

与此同时,马老爷手握听筒,正在满头满脸的冒冷汗。他刚刚把管家打发走了,照理说一切都在按照计划进行,简直堪称天衣无缝,可他身上一阵一阵的发冷,眼角余光总像是能瞥到人影——然而扭头再去细看,却又什么都没有。

他没有食欲,让仆人把晚饭端到卧室里去。坐在窗前的小桌子边,他端起饭碗,没滋没味的往嘴里扒了一口米饭。米饭含在嘴里,硬是咽不下去,因为一颗心怦怦乱跳,跳得连章法都没有了。

视野边缘的影子又出现了,他故意的不看,可是双手不受控制的抖个不停。筷子在碗沿磕出一串细碎的声响,他低头张嘴,把米饭吐回了碗里。屋子里一定有古怪,他想,家里放着个半仙呢,此时不用,更待何时?

放下碗筷站起了身,他想打电话把无心叫过来。可就在他走向门口之时,头顶忽然响起滋啦啦的电流声音,紧接着吊灯熄灭,屋中立时就黑透了。

马老爷不敢耽搁,想要去叫仆人检查电路。大踏步上前拉开房门,他猛的顿住了脚步!

走廊里也黑了,黑得伸手不见五指。而马俊杰歪着脑袋,就站在他的面前。

马老爷颤着声音开了口:``你\ldots{}\ldots{}''

马俊杰阴恻恻的一笑,一个脑袋慢慢的正了过来。

马老爷一手扶了门框,一手摁了胸膛,身体开始往下溜。极度的恐惧让他的声音变得又高又尖:``你\ldots{}\ldots{}''

此刻,走廊两边的无尽黑暗中,现出了一张又一张熟悉的面孔。枯瘦的妇人,是被他关起来活活饿死的前头大太太,大太太身边跟着的,是马英豪和佩华。后方一片鲜艳光彩,正是盛装的四小姐和五姨太。心宽体胖的二姨太伴着一具无头的身子也出现了,无头的身子是谁?马老爷瞪大眼睛辨认出了,是一贯奇装异服的八姨太!

在益发剧烈的心跳之中,马老爷听到自己的声音颤抖着响起:``我不怕你们。我\ldots{}\ldots{}不怕\ldots{}\ldots{}你们。''

几分钟后,宅子里的电工接起了烧断的电线。仆人们把刚翻出来的蜡烛又放了回去。主人一直没有召唤,他们乐得休息。有人惦记着马老爷卧室里的残羹剩饭,想去收拾,但是卧室紧关着门,他们不敢妄动,只好姑且算了。

到了夜里八九点钟了,赛维穿得整整齐齐往院外走。将睡未睡的老妈子见她捧着一大摞物事,仿佛是很沉,便要去帮忙。她一扭身躲开了,又道:``我给爸爸送书去,一会儿回来,你们可别忘了给我留着门。''

老妈子答应了,而赛维走出不远,转身又折返回来,大声喊道:``胜伊,来帮个忙呀,我抱不动了!''

胜伊推门跑了出来,没说什么,脚不沾地的随着她快走。及至走远了,胜伊低声说道:``姐,我把手表给无心了。他看着时间呢,至多比我们晚到五分钟。''

赛维点了点头。大夜里的,三个人一起拎着箱子往外走,看着会令人生疑,所以只好分批行动。他们先走,无心随后找个借口再追出来。

赛维有力气,捧着伪装过的皮箱行走如飞。片刻过后到了前头楼里,她见楼下只有一名仆人值更,便故作无意的开口问道:``爸爸睡了吗?''

仆人恭而敬之答道:``好像是睡了,一直没叫过人。''

赛维做出很活泼的样子,一蹦一跳的往上走:``我瞧瞧去!''

胜伊一言不发,随着赛维三步两步上了二楼。二楼走廊里只亮了几盏壁灯,赛维停在马老爷的卧室门前,对着胜伊一使眼色。胜伊知道她腾不出手,于是上前敲响了房门:``爸爸——''

房门一敲即开,原来并未上锁。宽敞卧室里一片漆黑,灯也没开。赛维大胆的把手中箱子拎住了,因为对于父亲的卧室也不熟悉,所以伸手摸了摸两边墙壁,并没有摸到电灯开关。不过借着走廊内的昏暗光线,她依稀看到了床上的人影——马老爷背对着他们,正在侧卧着睡觉。

赛维疑惑极了,心想父亲此时绝对没有睡觉的道理,即便是打盹儿也不应该。把手里的皮箱和用来遮掩皮箱的杂志一起交给了胜伊,她走到床前,见马老爷穿着长袍马褂,脚上皮鞋都没脱,不是个正经大睡的模样。

微微弯下了腰,她试探着唤道:``爸爸?''

马老爷一动不动。

胜伊把杂志随手放在桌上,拎着皮箱也凑上去了:``姐,爸爸睡着了?''

赛维伸手去拍马老爷的手臂:``爸爸,醒醒啊,时间到啦。''

马老爷躺得很稳当,并不肯随着她的拍打而起反应。赛维急了,正要把他强行扳个仰面朝天,不料身边的胜伊忽然轻声唤道:``姐!''

赛维扭头看他:``嗯?''

胜伊苍白着脸,一只手颤巍巍的抬起来,指向了床尾立柜上的梅花镜。赛维顺着方向一望镜子,登时也怔住了。

居高临下的梅花镜照出了大床的全貌。背对着他们的马老爷翻着白眼,正在狞笑!

\chapter{逃出生天}

赛维张大了嘴,却只在喉咙里发出了细细一声哀鸣。抬起手臂狠狠的把胜伊扫到自己身后,她慌乱的想要后退。然而为时已晚,床上的马老爷似乎专在等待他们肝胆俱裂的这一刻。猛然起身向外一扑,他直挺挺的伸出双手,紧紧掐住了赛维的细脖子。

胜伊怕到了极致,反倒一声不吭。咬紧牙关举起皮箱,他绕过赛维走到床边,瞪圆了眼睛去砸马老爷的脑袋。砸过一下,他运足力气再砸。皮箱里面衬着钢铁骨架,比板砖更坚硬更有分量。马老爷的脖子``咔嚓''一歪,仿佛是骨头受了损;然而双手仿若钳子一般,已经掐得赛维伸了舌头。

胜伊忘记了叫,甚至连呼吸都停住了。他想姐要被爸爸掐死了,他一下又一下的猛砸马老爷的脑袋,直到马老爷的脑袋都变了形。赛维虽然到了生死关头,却还保留着一丝清明神智,两只手乱挥乱舞的拨着胜伊,她翻着白眼做口型,要让胜伊去找无心。

正当此时,无心到了。

无心进门时,谁也没有听到声音,唯有赛维感觉合在自己颈上的双手似乎略松了一下。她趁机握住马老爷的双手手腕,拼了命的想要掰开。可是未等她开始用力,一只手擦着她的头发伸向前方,将一张纸符贴上了马老爷的眉心。马老爷一仰头,竟是张嘴露齿要咬人——不咬无心,他向前去咬赛维。

无心用手掌捂住了他的嘴,不让他向前靠近赛维。赛维咬牙切齿的扯开了他的双手,喘着粗气接连后退了好几步。胜伊扶住了她,同时听到无心开了口:``五少爷,没完了?''

纸符的效力显现出来了,马老爷跪在床上不住的挺动,仿佛是要向上突破什么。而无心继续问道:``告诉我,你是怎么逃出地堡的?只要你实话实说,而且保证以后不再害人,我就放你一条生路!''

马老爷的眼皮开始剧烈地抖,无心的手掌贴在他的嘴上,清楚的察觉出他已经没了气息。

``我\ldots{}\ldots{}保\ldots{}\ldots{}证\ldots{}\ldots{}''马老爷回答了,声音单薄,正是马俊杰的孩子嗓门。

赛维和胜伊听在耳中,吓得面无人色,同时看到无心背过了一只手,竟然正在倒握着一把锋利匕首。刀刃切进皮肤,他已然是攥了一手的鲜血。

无心不动声色,伸向前方的手缓缓离开了马老爷的嘴唇。两根手指夹住对方眉心上的纸符,他低声说道:``毕竟是父子一场。我放你走,你也给你父亲留具全尸吧!''

然后他缓缓揭下纸符。随着纸符的移动,马俊杰的鬼影渐渐脱离出了马老爷的身体。眼看纸符就要彻底离开马老爷了,无心忽然扔了匕首,抬起血手在纸符上刷刷点点又画一道,随即把血符对着马俊杰一挥。血符平展如刀,所过之处一片空寂,马俊杰瞬间消失了。

马老爷的尸首颓然倒在床上,依旧是死不瞑目的狞笑着。无心用血手攥住纸符,回身对着赛维和胜伊说道:``今天有灵感,画符画得好。马俊杰已经被我收服了,接下来该怎么办?''

赛维的头脑一片空白。马老爷一死,她简直没了主心骨。做过几次深呼吸后,她战栗着答道:``有暗道\ldots{}\ldots{}我们走暗道\ldots{}\ldots{}''

暗道的确是有的,就在马老爷床下。马老爷的卧室位于二楼,可是因为当初建造时花了大心思,用了各种障眼法,竟然能够向下修出一条不显山不露水的地道。

拖出床下一口最大的箱子,赛维还记得上次马老爷在向自己介绍出逃计划时,曾经说明了所有细节。箱子下面的地砖是活动的,掀开地砖会看到一口井,井壁伸出长长的铁梯。沿着铁梯一路向下,落了地之后就沿着甬道走。

地砖撬开了,果然是有铁梯。三个人络绎下去,脚踏实地之后,也果然是见了甬道。赛维打开了手电筒,弯着腰往前走。甬道四壁修得粗糙,只用石板砌出了两边的墙。据说修暗道还是马老爷的父亲的主意。赛维的爷爷一直活在马家的传说之中,活着的时候,人送外号老疯子。

甬道太长了,三个人像三只鬼,一声不吭的低头走。前方的赛维忽然问道:``爸爸没了,我们还要去投奔姑母吗?''

胜伊跟在后方:``爸爸都把财产藏到姑母家里了\ldots{}\ldots{}''

赛维回头看了他一眼:``如果没有财产的事情,我也不问你。爸爸在,一切都好说;爸爸不在了,姑母对我们又有几分感情?如果我们去见了她,她会不会把我们卖给日本人?''

然后她目视前方,再不需要任何意见。

三个人在地道里走了足有一里地远。地道尽头竖着梯子,他们一个接一个往上攀登,末了在一户小四合院内的枯井口见了天日。四合院内守着马宅的管家——小院算是马宅隔街的邻居,常年锁着。管家傍晚偷偷进了院,一直在等待主人出现。

管家和马老爷挺有感情,听闻马老爷归了西,他恨不能一头扎进枯井里;再问是怎么死的,赛维低声答道:``好像是\ldots{}\ldots{}吓死的。''

管家吓了一跳:``吓死的?''

赛维正视了管家:``不能再回家了,家里有鬼。''

管家颤巍巍的伸出一个巴掌:``是\ldots{}\ldots{}五少爷?''

赛维点了点头:``是。''

管家捂了嘴,不敢再言语了。

赛维和胜伊随着管家进屋休息,两人全都镇定得过了分。无心独自蹲在门前台阶上,心想人有了喜怒哀乐的情绪,还是发散出去的好。赛维和胜伊明明受了大惊吓,可是转眼之间就成了满不在乎的模样。他不希望他们落下心病,他们落下了心病,还不是饶不了他?

将近黎明的时候,天色黑得像墨一样,然而远近起了鸡啼,阳气上升,阴气下沉。无心擦了一根火柴,用火苗燎了手中血符的尖端。血符成了紫黑色,里面封着马俊杰的魂魄。当然,也有小健。可惜一团火烧过去,无论是谁,都要魂飞魄散了。

血符燃得很慢,火苗似有似无。无心仰着脸往漆黑的虚空中看,就见零碎的魂魄像一抹抹五颜六色的光芒,飘飘忽忽的四散开来。``死''可真是了不得,正邪好恶全被它一笔勾销。生者纵有千本账,对于死者来讲,却是根本不算数。怪不得都说死者为大,死者的确是大。

不知道马俊杰吞噬了多少人的魂魄,在无心的眼中,四面八方都是微光。身后房中忽然有了动静,是赛维和胜伊走了出来。

火苗烧到了指尖捏着的纸符最后一角,他松了手,回过头。

赛维和胜伊依然很镇定:``无心,我们走。''

虽然旅途少了马老爷,但是计划不受影响,余下的三个人加上管家,还是成功的溜出了北京城。

赛维和胜伊显然是没有威力去约束管家的,南下的路刚走到一半,管家就自行溜了。而受惊的后果显现出来,赛维发作了无人能治的疑心病,认定姑母会对他们谋财害命;胜伊则是拒绝触碰一切外人。乘船的时候水手拉了他一把,他厌恶得当场大叫一声。上船之后掏出手帕,他几乎把自己手上的皮肤搓下一层。

抗战六年,从沦陷区到大后方,地下的交通网已经是相当的完善。赛维在疑心病的驱使下东一头西一头乱走,本来说好要去昆明的,也不去了,转而要去重庆。谁也管不了她了,她自封为一家之主,胜伊自然是没有发言权,无心也必须听她的话。

无心耐着性子,受了气也忍着,心想自己至少得忍到姐弟二人安顿下来。还是那句老话,帮人帮到底,送佛送到西,哪怕姐弟二人目前宛如两位变态。目前赛维难伺候的程度,仅比白琉璃好一点点。无心暗地里拨着算盘,心想眼下的生活乐不抵苦。实在不行的话,自己还是孤身流浪去吧。

经过了小半年的颠沛流离,在翌年的暮春时节,他们终于到了重庆。

重庆作为战时陪都,半个国的人都涌来了,又经营建设了好几年,自然别有一番繁华气象;而且日军的轰炸也停了,在重庆过起日子,倒是堪称太平。

赛维的小皮箱已经空了一小半,但还是有钱。城市外围开辟了许多花红柳绿的新村,她就在村里租了一套很体面的房子。房子虽是一层的平房,但是造得漂亮,颇有西洋风格,里外五间,十分够住。门外用小栅栏围了个绿草如茵的小院子,院中还种着几株碧桃。

无心吭哧吭哧的干活,把房屋内外都打扫干净了,卧室里的被褥也都铺整齐了。赛维小半年来第一次真心实意的露出了笑模样。家里连锅碗瓢盆都没有,她带着胜伊出去一趟,买回了大包小裹的卤菜点心,以及两瓶酒和一摞瓷碗。当天晚上,三个人好汉似的围着圆桌子坐了,赛维倒了三碗酒:``从今开始,我们就算重生了!''

胜伊美滋滋的笑,无心则是环视四周,认为自己总算是很对得起他们了。该来的迟早要来,他端起碗抿了一口酒,心想自己有话还是得说。再不说就该上床睡觉了,他不能永远让赛维糊里糊涂的和自己躺在一个被窝里。

``赛维,胜伊。''他开了口:``我有话要说。''

赛维和胜伊叼着卤鸡翅膀转向了他,异口同声的问道:``嗯?''

无心放下瓷碗,低声说道:``我有个秘密,想要告诉你们。''

赛维很少看他如此郑重,不禁捏着翅膀提起了心:``秘密?''

无心抬眼看了看她,又看了看胜伊,然后说道:``其实\ldots{}\ldots{}我不是人。''

此言一出,四座寂静。良久过后,胜伊吐出嘴里的细骨头,迟疑着开了口:``无心,你为什么要骂自己?你是不是对我姐变心了?''

赛维把啃剩一半的鸡翅膀往桌上一扔,面红耳赤的瞪着无心,翕动鼻孔直喘粗气:``别跟我打马虎眼。你说你到底是怎么个意思?你又看上谁了?你说你不是人就算了?我告诉你,没完!''

抄起桌边的手帕摁下了眼角呼之欲出的眼泪,赛维带了哭腔:``你说咱们三个,多不容易啊。都他妈死绝了,就活了咱们三个。现在刚刚安定了,你可好,跟我耍花花肠子。怎么着,是不是看我倒搭不值钱?还是嫌我没了爹,不能养你做阔姑爷了?''

无心听得张口结舌,发现自己的意思被姐弟二人弄了个满拧:``不是,我没起外心,我也没看上谁。我\ldots{}\ldots{}我这几天一直在干活,我哪有时间看人啊?你们误会了。''

胜伊板着脸,定定的看着他:``那你是什么意思?''

无心很为难的吸了口气,感觉怎么说都不准确:``我的意思是说\ldots{}\ldots{}我是个\ldots{}\ldots{}妖怪。''

话音落下,四座又是一片寂静。

胜伊的脸上渐渐浮出笑容,笑到最后绷不住了,他``嗤''的出了声:``你的英文名字是德古拉吗?''

赛维也笑了:``今晚是月圆之夜,你必须变个狼人给我瞧瞧。否则我们可不承认你是妖怪!不变狼人,变个大尾巴狐狸也成!''

\chapter{赛维的思想}

赛维和胜伊哈哈大笑,笑得连卤鸡翅膀都捏不住了。笑着笑着发现不对劲,因为无心没有跟着他们一起笑。

赛维渐渐的收住了笑容,对无心说道:``别闹了,你怎么不吃啊?''

无心穿得单薄,此刻低头解开里外两层衣扣,他袒露出胸膛,然后拉过了赛维一只干净手,贴到了自己的心口上。

赛维脸红了:``干什么?''

无心抬头望着她:``赛维,对不起,我真的\ldots{}\ldots{}是个妖怪。''

赛维扭头吐出一根鸡骨头,同时发现自己掌下没有心跳。

她以为自己是摸的位置不对,所以扔了卤鸡翅膀擦了擦手,双手拍上去左右来回的摸。胜伊见状,莫名其妙:``姐,你找什么呢?''

赛维迟迟疑疑的看向无心:``你\ldots{}\ldots{}你的心呢?''

然后她抬手去按无心的脖子两侧,要找动脉。脖子两侧很安静,薄薄的皮肤下有骨有肉,就是没有一跳一跳的大血管。

她的手开始哆嗦了,坐直身体又拉过了无心的双手。两只腕子也分别诊过了,没有脉搏。

手背贴了贴无心的额头,温度是有的。可是手指向下移到鼻端,却是没了呼吸。她忽然想起无心总是很静,又想起自己在最初和他相识的时候,就看他像一只又野又驯良的兽。可纵算他不是人,也不对劲。兽也该是活生生的,可无心并非如此。骤然起身退了一步,她颤声问道:``怎么回事?你死了吗?''

未等无心回答,胜伊抢了话:``姐,你疯啦?''

赛维面对胜伊,抬手指向无心:``他、他、他没有心跳也没有呼吸\ldots{}\ldots{}他死了。''

胜伊知道赛维不是大惊小怪的人,不禁也跟着站起了身。试探着伸出一只手,他效仿赛维,也把无心从上到下摸了一遍。摸完之后他退了一步,又退一步,瞪着无心不说话。

无心自己低头系好扣子,随即也想起立。不料他刚一欠身,赛维和胜伊便一起跌跌撞撞的撤出老远。无心知道他们是要怕自己躲自己了,便很识相的缓缓站起,慢慢走到了房门口:``你们别怕,我不会伤害你们。''

赛维苍白着脸,喃喃说道:``我们早就看你不对劲\ldots{}\ldots{}知道你不会伤害我们,可你到底是个什么东西变的?''

无心摇了摇头:``我不知道\ldots{}\ldots{}我总也不老,总也不死,很多很多年了\ldots{}\ldots{}我想我应该是个妖怪。''

然后他小声说道:``让我在后面的屋子里再住一夜行吗?如果你们怕我,我明早就走。''

赛维和胜伊一起成了木雕泥塑,看着他不言语。而他没有等到回答,就转身去赛维卧室收拾了自己的旅行袋,钻进了后面清理出的小储藏室。

赛维关了门。自顾自的坐在椅子上,她叹了一口气,低头望着桌上零零落落的几根鸡骨头。几大包的卤菜,还没有打开,可是谁又有心思再往嘴里吃喝?

``一百年也没一遭的事儿。''她轻声开了口:``让我给遇上了。''

端起瓷碗喝了一口酒,她神情痛苦的哈出一口酒气:``我演了大半年的聊斋,说出去谁能信?''

胜伊靠墙站着,小声问道:``姐,怎么办啊?他不是人,你还爱他吗?''

赛维出了半天的神,末了答道:``我爱他。我看过了他,再看别人就都看不上了。''

胜伊嗫嚅着点头:``是,他性格好,心地也好。他一直保护我们\ldots{}\ldots{}你欺负他,他也不闹脾气\ldots{}\ldots{}''

赛维把双脚踩上凳子横梁,赌气似的抱了膝盖,垂着脑袋咕哝道:``他还好看呢。身边的人,我就没见谁长得比他更好。''

胜伊忽然``咭''的笑了一声:``姐,你听见了吗?他说他不会老,也不会死。''

赛维依然垂着头:``听见了,谁知道是真是假。千年王八万年龟,难道他是乌龟王八修炼成精了?''

胜伊的心思转移了方向:``他要真是永远不老,姐,你就占便宜了。''

赛维听弟弟说话东一句西一句的,忍不住也是一笑。笑了一下之后不笑了,她低声说道:``我什么都想到了,你当你姐我是个傻的?我不傻,我都想到了。将来的日子怎么过,他不老实了我怎么降服他,我都想齐全了。可我想天想地,无论如何也没想到他是个——''

她欲言又止的把嘴唇抿成一条直线,直勾勾的望着前方怔了一阵,接着又道:``人算不如天算。''

赛维不睡觉,对着一桌子卤菜长久的发呆。她自认为是被狐狸精魇住的书生,虽然对狐狸精也怕,但是只要狐狸精自己不逃,书生是不忍放手的。

胜伊也没了主意——他素来是见了男子就烦,难得能对哪位同性产生好感,尤其同性的身份还是自己的姐夫。赛维若是真把无心赶走了,他不能阻拦;可是赛维必须负责给他再找个同样成色的新姐夫,否则他就不同意赛维结婚。

与此同时,无心在储藏室里打了个地铺,倒是躺得很安然。他盘算好了,如果赛维胜伊不肯要他,他就去川边混混。反正是个漫无目的,走走逛逛也不错。在过去的大半年里,他算是过足了和人亲近的瘾,在接下来的三年五载内,他都能安安稳稳的孤独生活了。

心安理得的闭了眼睛,他枕着自己的旅行袋睡着了。一觉醒来,他把地上的铺盖卷好,想要送回原位。然而伸手一推房门,他抱着铺盖见到了赛维和胜伊。

赛维和胜伊都顶着两只黑眼圈。赛维看他抱孩子似的抱着一卷子被褥,便低声问道:``睡好了?''

无心摸不清她的虚实,于是只点了点头。

赛维又问:``你想走吗?''

无心向她微笑了:``听你的。''

赛维忍住一个哈欠:``别走了。''

无心没想到她会如此痛快,居然真敢留下自己。不置可否的望着赛维,他类似一名饱足的老饕面对了满桌盛宴。吃,已经饱了,毫无食欲;不吃,又舍不得,因为几十年也遇不上一顿。

赛维在凌晨时分做下决定,随即就困得东倒西歪。胜伊一直陪着她,此刻抬起千斤重的眼皮,也说:``别走了。反正你不伤人,留下也没什么的。别走了,大家一起过吧。''

赛维认为胜伊补充得很全面,自己无话可说。忍无可忍的掩口打了一个大哈欠,她半闭着眼睛对无心说道:``我们要睡了,早饭你自己吃吧。''

无心眼看他们要走,忽然想起自己有所遗漏:``赛维,还有一件事。''

赛维抬头看他:``啊?''

无心凑到她耳边低语了几句。赛维听了,倒是不甚在乎:``我本来就不喜欢小孩子,烦都烦死了。将来胜伊结了婚,从胜伊家里过继一个就行。''

无心听了她的回答,始终是感觉不对劲,所以想要老调重弹:``可是我不会老,将来\ldots{}\ldots{}''

赛维摆了摆手:``将来就算我是老牛吃嫩草,可我也不白吃啊。男女要平等就彻底的平等,男人可以讨年轻的太太,我也可以嫁年轻的丈夫。我并不比男人差什么。嫩草嘛,男人吃得,女人也吃得。再说我现在还小着呢,要老也是以后的事情。''

话音落下,她哈欠连天的走了。胜伊闭着一只眼,猫头鹰似的看了他一眼,也跟着走了。

无心看赛维是困糊涂了,所以没有追着她深谈。赛维的思想还是简单了,她可以不在乎,但将来她的亲人、她的朋友,也能跟着她一起不在乎吗?

无心怎么想,怎么感觉事情没完。洗漱过后出了门,他双手插在衣兜里,沿着石阶路向上慢慢的走。山城的道路起起伏伏,他渐渐走不动了,就转向了路边一家下江面馆。面馆很简陋,屋檐长长的伸出去,檐下还摆着桌椅。大清早的,食客已经很多,无心在馆子里面找了个靠窗的位置坐下了,一边等着吃面,一边百无聊赖的往窗外望。忽然间,他一挑眉毛,怀疑自己是看到了赵半瓢。

就在街道的对面,一个穿着旧花布袄裤的利落妇人坐在路边,正在低头打开木箱,从里面向外一盒一盒的掏出香烟。偶尔的一扬脸一转头,无心看得清楚,见她黑油油的头发粉扑扑的脸,可不就是赵半瓢?

和半年前相比,赵半瓢显岁数了,左耳根下面还有一道长长的疤,几乎从脖子延伸到面颊,差一点就破了她的相。摆好她的香烟摊子之后,一名饱餐了的食客横穿街道,到她面前要买香烟。她抬头对人一笑,手脚麻利的收钱找钱,眼角眉梢全是精神,手指尖儿都带着力气。

无心虽然不知道她还能不能认出自己,但是不敢再看了,因为有点怕。对赵半瓢的怕,和对赛维的怕,不是一种怕。闷头吃了一大碗面,他会账起身,不知怎的,很不好意思,低着头溜出面馆回家了。

\chapter{赛维的爱情}

赛维在安居之后,立刻就交了一大队女朋友。

她所住的新村,房屋全都整洁美丽,邻居们也都平头正脸。世界战局越来越明朗,邻居们既然认定胜利指日可待,便全都有了娱乐的心思,附近的几幢豪宅里面,几乎天天都有舞会。赛维服装奢华,出手阔绰,三下五除二的就折服了周遭的太太小姐们。隔三差五的,她也请朋友们到家里来喝下午茶。家里已经雇下一名二十多岁的伶俐女仆,干干净净,很能张罗。在慵懒的午后时分,仕女们坐在马家的碧桃花下薄纱窗前,喝喝茶聊聊天,无论如何都是一种雅致的享受。

赛维并没有去办理法律上的手续,直接宣称无心是自己的丈夫。旁人见了赛维那种颐指气使的派头,立刻认定了马女士之夫是位吃软饭的小白脸。

无心不理会,在微微阴霾的午后,他素来是坐在卧室窗前的沙发椅上,低着头擦他的银腰带。银腰带已经被他擦亮大半,如今看起来正是半黑半白。偶尔想起死在地堡里的白琉璃,他并不动心。白琉璃和赛维一样,都会时不时的让他闹头痛。白琉璃更恶劣一些,但他个男人,自己忍无可忍了,可以欺负他一下。

他是不能去欺负赛维的,他要是真使了坏,赛维一定抵挡不住。

赛维教他学跳舞,跟着留声机在家里前一步后一步的转圈走。走着走着就不走了,赛维一把搂住了他,闭着眼睛靠在他胸前,半晌一动不动。一只手慢慢的从他后背往上走,走到后脑勺再往下滑。赛维的指尖拂过他的鼻梁嘴唇下巴,最后拍了拍他的脸:``无心,你白天怎么不理我?''

无心想了想,在满鼻子的香水味中答道:``白天我没有见到你,你不是晚饭前刚回来吗?''

赛维笑了:``诈你一下,看你会不会拿话敷衍我。''

然后她抱着无心左右摇晃了几下,喃喃说道:``还是你好。胜伊在外面丢人现眼,真气死我了。等他晚上回来了,你看我不骂死他!''

无心低头吻了吻她的头发,心想自己以后不能再去面馆偷看赵半瓢了,对不起塞维。赛维像个男子汉似的撑着一个家,并且不容许旁人插手,她有她的志气和辛苦。刁蛮泼辣就刁蛮泼辣吧,再刁再泼,还不就是几十年的光阴?大不了自己耐下性子,哄她几十年。几十年,不算什么。

赛维用双手环住了他的脖子,闭着眼睛又说:``无心,我爱你。我死了,我不管;我活着,就不许你离开我。将来我成了老太太,老得没法儿看了,你也不能走。你不喜欢我了,我还喜欢你呢。你不愿意理我,也得天天让我瞧你一眼。记住没?''

无心点头答应:``记住了。''

赛维拍了拍他的后背:``好孩子。''

无心用力的拥抱了她一下,感觉她胖了。她在山林里养成了个大胃口,到了重庆,依旧是能吃能喝。不少人都当面恭维马女士生得美丽,他有时候仔细瞧瞧她,发现她面颊的确是丰润了许多,手臂大腿也有肉了,敢于白白嫩嫩的晾在外面。

两人正是搂作一团之时,胜伊醉醺醺的冶游而归,回来撞枪口了。

赛维推开无心,揪住胜伊,劈头便问:``你把罗太太她娘家妹子怎么了?''

胜伊吓了一跳:``陈小姐吗?我没怎么啊,我就请她去看了两场电影,她还一场都没去!''

赛维用手指头狠戳胜伊的额头:``你够贱的!她不去就不去,你为什么请个没完?不看电影,就请听戏,不去听戏,就请吃饭。我告诉你,人家罗太太说你骚扰他妹子呢!妈的我在外面顶天立地,没想到被你个浪蹄子抹了一脸黑。本来我还想和罗太太合伙做点期货生意,今天听了她的话,气得我也没说出好的来!我告诉你马胜伊,从今晚开始你不许出门。我让无心看着你,你再敢出去骚,我打断你的狗腿!''

胜伊被她搡的站不住,一屁股坐在了椅子上。及至她气吞山河的骂完了,他带着酒气,忽然一抽鼻子,哭了。

``她们为什么都不喜欢我啊?''他委委屈屈的抹眼泪:``我长得不丑,不脏,也不穷。还有密斯陈\ldots{}\ldots{}我只是对她好,又不让她搭我什么,她至于背后嚼我的舌头吗?''

赛维兜头抽了他一巴掌:``要不然说你贱呢!''

胜伊真伤心了,哭得满脸眼泪:``姐,我是不是、是不是像爸爸啊?我是不是看起来特别、特别招人烦啊?她们当着我的面,说我是娘、娘娘腔。''

赛维立起两道眉毛:``她们?她们是谁?''

胜伊双手捂着脸,摇头不语,一味的只是抽抽搭搭。

赛维双手叉腰,喃喃的骂了一句,也不知道骂的是谁;端起茶杯想要喝口水,茶杯又是空的。嘴里嘟囔了一句``气死我了'',她转身出门去叫女仆烧开水。而胜伊见无心走到自己面前了,就向前一扑,把整张面孔撞到无心肚子上,``嗷''的一声开始痛哭。

无心摸了摸他的后脑勺,发现他很激动,短头发热腾腾的,都汗湿了。弯下腰扶起胜伊,他望着对方一双泪眼,想要做出一番安慰:``胜伊,别难过。我经常一个人过几十年,不也是活得好好的?人生也不过是几十年,一辈子很快就会过去了。''

胜伊听了他的美言,精神彻底崩溃,嘴咧得能塞进拳头,直着喉咙哇哇哇,眼泪和口水一起喷到了无心的脸上。无心没想到自己的肺腑之言起了负作用,不禁对着胜伊的嗓子眼愣了愣。幸而赛维及时回来了。手托毛巾给胜伊擦了一把脸,赛维叹息一声:``不知道哪个王八蛋带他喝了酒。一个娘胎里出来的,你说他怎么是这样儿啊?''

无心低声说道:``你别骂他了。我刚才看他喉咙红肿,是不是有点上火?''

赛维放下毛巾,俯身搀扶胜伊站起来,同时对无心说道:``肯定是上火。明天再给他找点药吃,今天赶紧让他上床睡吧。他比我晚生了一分钟,我感觉我比他老了十年。你别傻看着,过来帮我一把。他也胖了,怎么这么沉啊?''

无心把胜伊拦腰抱起来送去卧室床上,赛维跟在后面。等到安顿胜伊睡下了,赛维和无心对视一眼,无心笑了,赛维也跟着苦笑。

无心和赛维回了卧室,两人上床放了蚊帐。无心伸长一条手臂,让赛维当枕头。而赛维枕了片刻,忽然问道:``明早在家吃吧。胡妈天天早上出去买小笼包子回来,不比你自己去吃面条强?''

隔着一层蚊帐,无心望着窗外的路灯光芒:``好。''

赛维打了个哈欠,把手放上他的胸膛:``不让你去面馆,你生不生气?''

无心没听明白:``生气?生什么气?''

赛维探头凑到他的耳边,压低声音说道:``我也去过那家面馆,馆子对面有个香烟摊子,卖烟的人,我可认识。''

无心立刻扭头望向了她:``你别误会。''

赛维在他脸上掐了一把:``当我什么都不知道?你在我手心里呢!我知道你清白,但是跑去过眼瘾也不行!再说她有什么好看的呀?更要命的是她和我们有仇,我们到了重庆,本来一切都是从新开始了,万一被她翻出旧账,再去告发我们,警察再把我们当成汉奸逮起来,才叫倒霉倒到了姥姥家。往后不许去了,知不知道?''

无心侧身抱住了她:``知道,不去了。''

赛维仰脸看他,忽然怀疑他不是很爱自己,可是一想起他曾经那么舍生忘死的救过自己和胜伊,就安了心,认为自己是想多了。

翌日上午,无心在家里吃了小笼包子,然后把擦亮了的银腰带拎出来,挂在了客厅墙上的两根钉子上。腰带是一串银牌连缀成的,沉甸甸的垂成一条弧线,正好衬托出了上方挂着的一小幅油画,看起来有种不伦不类的协调。无心挂好之后审视一番,末了把腰带取了下来,感觉有些犄角旮旯的地方,还没有摩擦透亮。

手指裹了粗布,他用了力气,专蹭腰带缝隙。蹭着蹭着他停了手,忽然发现银牌侧面好像有机关。

他没声张,自己找了根缝衣针。银牌侧面皆有一点小孔,简直要看不出。他用针尖戳进小孔,用力一摁。结果就听里面``嘣''的一声,银牌子竟然像书本似的翻成两页,露出中间夹着的一片薄纸。

无心小心翼翼的取出薄纸,然后把银牌子两页合拢。机关咬合,恢复原样。展开薄纸再一瞧,无心皱了眉头,就见上面用极细的线条画了许多扭曲图案,一时也分辨不出是什么意思。

诸如此类的薄纸,他共取出了五张。五张纸合在一起,他只看出上面记载了白琉璃一门邪术的所有奥义。把五张纸谨慎收好,他把腰带重新挂回了客厅。

银色腰带反射了阳光,银牌上的莲花熠熠生辉。无心满意的点了点头,同时想起了死在地堡里的白琉璃。不听老人言、吃亏在眼前,他想白琉璃要是肯听自己的话,现在可能已经成了西康的财主,何至于会在苦寒之地成为孤魂野鬼?

胜伊下午醒了过来,垂头丧气的坐在床上,低声说道:``我娶头驯鹿算了。''

赛维没出门,在外面屋子里听了他的话,不由得笑出了声:``也真是邪了门。凭着你的条件,不应该没人要哇!''

胜伊表示同意:``对嘛,我们两个是一样的。''

赛维立刻走进门来,进行反驳:``谁跟你是一样的?''

胜伊扭头一看,见他姐烫着乌云似的卷发,穿着绸衬衫和西式长裤,脚上的凉鞋统共只有几根细带子,十根涂着蔻丹的脚趾头全见了天日。

胜伊也承认她一白胖,是比先前美了许多,于是像个妒妇似的酸溜溜:``当然不一样喽,我又找不到活妖怪当太太。''

赛维大踏步进了房,扬手就打了他一下子,又咬牙切齿的低声说道:``我的人,轮得到你说?你个没人要的货,老实在家呆着!''

赛维说变脸就变脸,一拳差点敲断了胜伊的细骨头。于是等赛维花枝招展的出门会朋友去了,他便哭丧着脸,走到无心面前诉苦:``姐夫,我姐又打我。''

无心听闻此言,当即找出黄历一看,然后变脸失色的答道:``快到日子了,再过几天你姐能吃人。''

再过几天,赛维又要来月事了。

\chapter{婚姻生活}

赛维的月事该来不来,心烦意乱,不由得就把怒火喷向了无心——是无心说他肯定鼓捣不出孩子,她才放心大胆的和他快活的。如今月事的日期到了,月事的影子却是无影无踪,她不由得怀疑他是胡说八道的撒了谎。

想到自己的肚子里也许已经揣上了一个小生命,她面赛铁板的坐在卧室椅子上,气得将要嚎啕。刚刚美丽了没几个月,她才不想挺着大肚皮生儿育女。无数的舞会和牌局正等着她,她真正独立的繁华岁月才刚刚开始。

``你骗我!''她把无心堵在床上,把他的鞋拎起来扔出门外,不让他逃:``我问你,有了孩子怎么办?''

无心仓促应战,连袜子都没穿。光着两只脚坐在床里,他怕赛维动手打人,故而还用棉被在身前堆起一座掩体:``赛维,不可能啊!''

床太大,赛维穿着一双系了繁复带子的皮凉鞋,脱了穿穿了脱的很麻烦,想要站在床边进行远距离打击,距离又过于远了,超出了她的手臂长度。虚张声势的对着无心一挥拳头,她继续发飙:``不可能?事实都摆在眼前了,你还有脸跟我嘴硬?好,很好,我们等着瞧吧,十个月后见分晓。我看出来了,你就是看不得我过几天好日子,非得把我折腾成黄脸婆了,你才满意。''

无心双手合十向她拜了拜,可怜巴巴的请她息怒:``赛维,你听我说,我自己是怎么回事,我清楚得很。远的我记不清,就说近百十来年吧,我也正经结过两次婚,都没留过一儿半女。赛维,你相信我,我没骗过你啊!''

赛维心里一股子一股子的往上窜火苗子。无心越乖,她越想把他抓过来狠狠欺负一顿:``你敢说你没骗过我?你偷着瞧赵半瓢的时候,怎么没向我打过报告呀?我要是不戳穿了你,你还当我是傻子呢!说,你是不是故意想让我在家给你下崽子,你好趁机出去骚?是不是结三次婚给你结美了,你憋着再结第四次呢?''

无心已经被她连着逼问了三个多小时,此刻实在是腻歪透了,便把棉被抖起来罩住自己,蜷成一团往床里一滚。赛维见他还学会装死狗了,越发怒不可遏。单腿跪到床上去,她一把扯开棉被,准确无误的直接捣向无心腿间。五指合拢抓了他□那一套物件,赛维咬牙一拧:``掐掉了你,让你作怪!''

无心疼得一个鲤鱼打挺,叫的声音都变了。

待到赛维傍晚出门了,无心盘腿坐在床上,搜索枯肠寻找避难之法。将从银腰带中取出的五张薄纸翻出来,他一边研究上面的细密图案,一边想起了白琉璃。既然马俊杰可以离开地堡,那么等白琉璃的修为足够强大了,自然也能来去自如。如今赛维的烦人程度,已经可以和白琉璃比肩,所以他不由自主的摇了摇头,感觉自己先前是把人间家庭想象得太美好了。

一张纸上的图案,给了他一点启发。于是在把薄纸收好之后,他盘腿坐在床上,先把手伸到裤裆里揉了揉痛处,然后扬起双手,合身向前``咣''的拍在了床上。拼了命的集中了心思,他回忆起了白琉璃常念的一句咒语。用舌头舔了舔嘴唇,他喃喃的诵道:``嗡嘛吱莫耶萨来哆!''随即猛一挺身,开始前仰后合的摇晃:``马赛维,不要欺负我。马赛维,不要欺负我。马赛维\ldots{}\ldots{}''

他使出了画符时的认真与虔诚,想要用自己的念力去对抗赛维的暴脾气。及至念到了口干舌燥之时,他收了声音,忽然感觉空气不对。晕头转向的睁开眼睛,他吓了一跳,发现房门开了一道缝,赛维不知在门外站了多久,正在通过门缝窥视他。

直勾勾的和他对视了片刻,赛维一推门进来了,双手叉腰问他:``你是在咒我吗?''

无心看她气色不对,心中就是一惊,摇着头轻声答道:``我没有。''

话音落下,他耳边起了一声巨响,正是赛维扬手抽了他一个大嘴巴。他没觉出疼,因为半边脸都麻木了。抬手捂了火热的面颊,无心委屈之极,眼睛里快要喷出火星子:``我总算是你的丈夫,你怎么说打就打?''

赛维恶狠狠的搡了他一把:``你个坏心眼烂心肠的妖怪,你敢咒我!你把我咒死了,你有什么好处?你是不是还想着赵半瓢呢?我告诉你,别以为我说我爱你,你就找不着北了!你敢学我五姑父,我活撕了你!''

无心一手撑在床沿上,垂下脑袋满地找鞋:``不过了,马赛维,我不和你过了!''

赛维一脚把他的鞋踢到了床底深处:``爱过不过,当我离不得你?''

无心和赛维吵了一夜,胜伊想要来劝架,结果被赛维撵了出去。到了天明时分,无心穿戴整齐了,提了旅行袋大踏步往外走。胜伊追上来拽他胳膊:``姐夫,姐夫,你别走啊。你走了我怎么办哪?她没了你,不得改骂我啊?你是把我丢火坑里了。''

无心一晃肩膀,头也不回:``你姐的脾气,我没法忍。''

话音落下,后方大开的玻璃窗里飞出了赛维的尖叫:``胜伊你回来。我倒贴完了,你又贴上去了,我们姐弟两个怎么全贱他一个人身上了?''

胜伊没理她,脚下步伐不停:``姐夫姐夫,你要去哪里?''

无心也没什么地方可去,抽出一秒钟想了想,他低声答道:``我下乡去。''

胜伊松了手,看他出院门了,连忙扭头跑回窗前,小声向赛维报告:``姐,他说他要下乡去。''

赛维人在房内,立刻走到窗口望向了他:``下乡?下哪个乡?下乡的长途汽车都是几个小时的长路,他连早饭都没吃,挨到乡下不饿死了?''

胜伊摇头答道:``他没说啊。''

赛维恨得瞪他:``他不说,你也不问?这么大的重庆,万一他跑丢了,我上哪儿找他去?''

胜伊转身往房门口走,且走且道:``怕他丢了,你就别发疯啊。我要是他,我也走。''

赛维和无心耍威风耍惯了,没想到泥人也有个土性。六神无主的原地转了个圈,她就感觉小肚子胀痛着难受。伸手从衣帽架上摘下了自己的小遮阳帽和玻璃皮包,她决定马上去把无心追回来。

在出门前,她去了一趟卫生间,发现月事来了。不发现则以,一旦发现了,越发感觉肚子疼身上冷。换了双半高跟的凉皮鞋,她一路小跑出了院门,左右张望了一番,发现无心早走得连影子都没了。

赛维先坐轿子再坐人力车,嚣张了一夜的气焰随着路途的延长而渐渐低落。等到临近长途汽车站了,她还没有看到无心的身影,不禁吓得手脚冰凉,心想他是凭着两只脚走下乡了?或者根本是在随口敷衍胜伊?

最后,在人山人海的汽车站里,她隔着车窗玻璃,看到了坐在车内后排的无心。

在看到无心的一刹那,她松了口气,只觉自己□瞬间开了闸,温暖的鲜血汩汩流出。她所在的位置,距离公共汽车太远,中间隔着等车的乘客,想要挤过去也不容易。售票的窗口倒是很近,她急了,索性掏出零钱买了车票。凭着票通过检查,一路横冲直撞的上了汽车。车里早满员了,站都站得拥挤。她东一头西一头的乱钻,一直钻到汽车最后排。毫无预兆的出现在无心面前,她没说话,一转身坐到了他的大腿上,又把他两条手臂拉起来,环到了自己腰间。冰凉汗湿的双手紧紧抓住了他的腕子,她低下头看着他一双手,有种劫后余生的庆幸。

无心低了脑袋,把额头抵上了赛维的后背。方才一个人上车坐下之后,他心里也是怪不得劲。和赛维过了一年了,赛维有坏的时候,也有好的时候。两个人分久必合合久必分的打打吵吵,闹都闹习惯了。

长途汽车一路疾驰,顺顺利利的到了歌乐山。赛维拉着无心下了汽车,急急忙忙的想找厕所,然而没找到。最后两人寻寻觅觅的到了荒凉处,无心放哨,让赛维在一棵老树后蹲下了。

赛维手忙脚乱的把自己重新收拾了一番,然后走到了无心面前,低声说道:``别生气啦,往后我再也不欺负你了。''

无心抱着自己的旅行袋,垂头说道:``我没咒你,也没想着赵半瓢。''

赛维在他脑袋上摸了一把:``我知道。我们赶下一班车回城吧。到了城里先不回家,我们两个吃西餐去。吃完西餐,再看场电影,好不好?''

无心的心软化了:``不带胜伊吗?''

赛维又握了他的手:``不管他了,我们两个玩一晚上。''紧接着她拍了拍无心的手臂,哄小孩子似的又道:``气头上的话,哪能当真呢?跟我走吧,啊?''

无心想了想,没想出什么来。而赛维知道他对自己总不会绝情到底,就趁热打铁的转了身,牵着他回车站去了。

赛维把胜伊抛到了脑后,和无心在城里又吃又喝,吃喝足了两人去了电影院,排长队买票。排队的时候两人还是手拉着手,赛维偷眼看着无心的侧影,不知道自己昨天怎么鬼迷心窍,非要和他决一死战。往事越想越是后悔,她暗暗下了决心,以后再也不欺负他了。

如此的决心,在赛维的一生中,一共下了无数次。她爱透了无心,也欺负透了无心。无心时常被她逼得火冒三丈,也时常被她哄得团团乱转。

离婚的话,雷打不动的每年都会被他们提起一次。赛维沾沾自喜的、得意洋洋的、和无心闹了一辈子离婚。

\begin{quote}
作者有话要说:赛维和无心的故事,到此就告一段落了。他们的结合并不是标准的美满婚姻,但是赛维一直和无心过到了她生命的尽头。
\end{quote}

\begin{quote}
接下来进入番外时间,讲述六年前无心和白琉璃在西康的恩怨情仇。

\end{quote}

\chapter{番外——无心和白琉璃(一)}

一九三八年春,西康。

明烈的阳光照耀着无垠的荒凉野原,无心半闭着眼睛,拖着两条腿在干燥的土地上慢慢走。北边打仗了,是大仗,日本军队开进中国,北国土地大片的沦陷,难民们不想做亡国奴,只能纷纷的往西南大后方跑。

他也跟着跑,跑得漫无目的而又奇快无比,先人一步的进了四川。在四川他没找到什么像样的活路,于是又从四川一路逛到了西康。到了西康干什么?不知道。

无心处处以人的标准来要求自己,而且还是好人。可一旦真饿极了,他精神空虚身体难受,就不由得要抛弃信条。此刻他舔着嘴唇东张西望,不但没有寻到猎物,连鲜美的绿草都没找到几根。偶尔会有褴褛肮脏的本地百姓从他身边经过,但他又不想吃人。

一双眼睛彻底闭上了,无心在温暖的阳光中犯了困。停住脚步向下一跪,他百无聊赖的歪倒在了土路旁边。侧身枕着蜷起的手臂,他低头向着来路望。两个野孩子正在远方打打闹闹,都是细胳膊细腿,骨头上面绷着一层黑皮。

无心的眼皮一颤一颤,和土地一样干燥的黑眼睛又要闭上了。可就在将闭未闭之时,视野中的两个野孩子忽然像受了针刺一样,步调一致的狂奔跑了。

当野孩子像小黑蚂蚁一样瞬间消失之后,道路尽头出现了一匹花枝招展的大白马。说大白马花枝招展,是因为它的辔头鞍子缰绳全都花花绿绿,胜过最鲜艳的花草。大白马上坐着一名同样华丽的青年。青年有一张白皙的面孔和一头浓密的发辫。发辫沉重的披散开来,头上顶着一块银牌,银牌上面缀着的大宝石在阳光下熠熠生辉,简直就是地上的星星。

一手松松拽着缰绳,一手举着一把黑色阳伞,青年架在鼻梁上的墨晶眼镜微微下滑,露出了两道眉毛和上眼皮的睫毛。一人一马施施然的缓缓而来,无心的眼睛越睁越大,看清了青年腰间的弯刀、配枪、以及绣着花的荷包。

挣扎着坐起了身,无心下意识的又开始舔嘴唇,心想我是乞讨,还是打劫?

他饿得发昏,恨不能冲上去一口咬出大白马的肥油。两条腿打着晃的支起了身体,他迎着来者抬起了头,结果发现青年已经到了自己面前。

青年仰着头,面无表情的没有看他,只自言自语的低低嘀咕了一声:``热啊!''

无心登时来了精神——青年会讲汉话!

他张了嘴,打劫的心思是没了,只想向青年要点儿吃的。可是青年并没有把路边的活物放在眼里。未等无心出声,他已然经过无心、继续前行了。

无心不假思索的一转身,快步追上了马屁股:``先生?''

青年勒住了马,回头看他:``汉人?''

无心立刻笑了:``对,我是汉人。先生,我要饿死了,你能不能行行好,给我点吃的?''

青年用手指把墨晶眼镜向下勾到鼻尖,露出了一双蔚蓝的眼睛。将无心上下打量了一番,他把眼镜向上推回原位,随即一挥手:``滚。''

然后他转向前方,驱使着大白马继续走了。

无心立刻跟上了他:``先生,我不白吃。我吃饱了,给你牵马好不好?瞧你的大白马多漂亮,你得找个马夫伺候它不是?''

青年在墨晶眼镜后面斜了他一眼:``你知道我是谁吗?''

无心微笑摇头,同时自然而然的快走几步,从他手中接过了五颜六色的缰绳。青年猝不及防的松了手,反应过来时,大白马已经被无心牵在手里了。两人对视一眼,无心的头和脸因为落了太多尘土,所以全是灰蒙蒙脏兮兮。青年看他笑得很贱,一脸讨好卖乖的奴才相,便扬起鞭子,在他脖子上不轻不重的抽了一下:``我是白琉璃。''

无心依旧是笑:``好名字,真好听。''

无心把大白马一直牵到了旺波土司的官寨。旺波土司是本地的大土司,官寨足有四五层楼高。白琉璃和旺波土司之间似乎存在着某种秘密关系,以至于可以在官寨后方单独占据一片很像样的房屋。房屋的陈设堪称华丽,床榻上面铺着来自汉地的上等丝绸。

白琉璃并不需要马夫,土司家的奴隶崽子会伺候他的一切。进房之后,他收了他的阳伞,摘了他的眼镜,脱了他的皮袍。舒舒服服的坐在床上,他翻了面前的无心一眼。不动声色的又想了想,他亲自给无心倒了一碗酥油茶。拇指指尖浸在茶里,他把碗一直端到了无心面前。

无心接过碗,仰头一饮而尽。抬起袖子一抹嘴,他在鼻子和下巴之间,抹出了一道本来肤色。双手捧着空碗,他垂着头,小声问道:``再喝一碗,行不行?''

白琉璃似笑非笑的接了碗,转身又给他倒了一碗。拇指再次浸过酥油茶,他把碗递向了无心:``喝吧。''

无心捧了碗,几大口又是喝了个精光。捧着空碗望向白琉璃,他讪讪的说道:``我还能喝。''

白琉璃拧起了眉毛,动作利落的接碗倒茶。酥油茶还是烫的,把第三碗送给无心,他自己抬手噙着拇指,感觉手指都要被酥油茶烫伤了。

无心总算是斯文了些,一口一口的喝,一边喝一边抬眼望着白琉璃。白琉璃吮着大拇指,蓝眼睛里射出冷森森的光。

当无心喝光了整整一大壶酥油茶后,白琉璃勃然变色,把安然无恙的他撵出了房。无心坐在房外的一块石头上晒太阳,知道白琉璃翻脸的原因——酥油茶里,被他下了毒。

或许是毒,或许是蛊。无心隐隐的能尝出异常滋味。是毒也罢,是蛊也罢,反正最终都会随着酥油茶一起被他尿进土里。他的身体,成不了它们滋生壮大的土壤。

一墙之隔的房内,坐着几近愤怒的白琉璃。无心骚扰了他一路,而居然不死。想到自己的蛊对无心失去了杀伤力,白琉璃在想不通之余,简直快要怀疑人生。

无心看出了白琉璃的富庶,所以白琉璃不驱逐他,他就赖在白琉璃的门口不走。等到酥油茶消化大半,太阳也晒足了,他起身进了房,对白琉璃笑道:``先生,有水吗?我想洗一洗?''

白琉璃抬袖子遮挡了眼前的阳光,不耐烦的看着他:``洗一洗?''

无心拍了拍自己的脑袋:``我太脏了。''

白琉璃不耐烦的一挥手:``外面有。''

无心不得要领:``外面\ldots{}\ldots{}哪有?''

白琉璃言简意赅的答道:``河里!''

无心在附近的小河里洗了个澡,洗了澡后又蹲在河边洗他的衣裳。肚里有食的感觉实在是美好,他把湿漉漉的袍子裤子搭在河边的矮树枝上,让春风把它们尽数吹干。藏民们都不吃鱼,但是白琉璃显然不是藏人。无心看到河水清澈,小鱼很多,就光着屁股站在浅滩中,弯腰徒手抓了五六条。用结实的草叶编成绳子穿过鱼鳃,他在傍晚时候,拎着一串小鱼回到了白琉璃的面前。

他问白琉璃:``你吃不吃鱼?''

问过之后,他试试探探的抬起了一只手。小鱼被碧绿的草绳穿成一串,还在垂死挣扎的摇头摆尾。几点水珠被鱼尾巴甩到了白琉璃的脸上,白琉璃向后一躲,心想他怎么还不死呢?

``我吃鱼。''白琉璃虎视眈眈的盯着他:``我什么都吃。''

无心想要讨好白琉璃,所以生了一小堆火,很仔细的烤熟了小鱼。白琉璃慢吞吞的吃了三条鱼,顺便又在余下几条鱼上下了蛊毒。颇为紧张的坐在床边,他提起精神等待无心暴毙。然而无心吃饱喝足之后,把一盆水端到了他的面前,当真履行起了仆人的职责:``先生,要洗脚吗?''

白琉璃认真的审视了他的气色,看他脸上白里透红,绝没有要死的意思。六神无主的摇了摇头,他茫茫然的答道:``不了,上个月已经洗过一次了。你\ldots{}\ldots{}感觉怎么样?''

无心若无其事的答道:``我感觉很好。''

白琉璃点了点头:``哦\ldots{}\ldots{}不要叫我先生,叫我白琉璃。''

无心的靴子已经烂穿了底,下午洗过澡后就一直是打着赤脚。白琉璃不洗,一盆水正好省给了他。及至他把自己收拾干净了,他问白琉璃:``能给我找个住处吗?''

白琉璃的居所,总共有好几间屋子,可是只有正当中的一间是可以休息的卧室。白琉璃没看他,只若有所思的向后一挥袖子。无心有点受宠若惊:``我和你一起睡?''

白琉璃一点头:``嗯。''

白琉璃的床榻柔软光滑,铺着层层丝绸。无心满以为自己能睡个舒服觉,不料等白琉璃在外侧也躺下了,他抽抽鼻子,忽然感觉周遭气味不对。

不着痕迹的把脸扭向白琉璃,他控制着力道吸气,发现白琉璃的身上有一种复杂奇异的臭。不像人的体味,倒像是油脂香料混合变质了,其中又加了一些化学品。其味之怪,真还不如大粪臭得纯正。

他可以不呼吸,但是白琉璃偶尔一翻身,自会扇动空气钻入他的鼻孔。他很难熬的转身背对了对方,心想与其享受臭烘烘的丝绸被褥,还不如出去露宿。

他一动,白琉璃开了口:``无心,你身体很好。''

无心知道他的意思,但是装傻:``是,我从来不生病。''

在接下来的两天里,白琉璃忿忿然的又给无心下了十几种蛊毒。到了第三天,他咬牙切齿的望着活蹦乱跳的无心,亲自烤了一只大黑蝎子给他吃,不吃不行,不吃就滚。

无心把黑蝎子吃了,嚼得满嘴脆响。吃完之后他出门了,白琉璃没有拦,等着他死在外面。

不料到了晚霞满天的傍晚时分,无心拎着两只断了脖子的画眉鸟,笑嘻嘻的又回来了。

白琉璃感觉自己的强大巫术在无心面前全成了笑话。悲哀的吃了一只烤画眉鸟,他低头咳嗽了两声,人一下子瘦了许多,围在腰间的白银腰带也松松的挂在了胯骨上。

到了夜里,白琉璃睡不着觉,坐在床上发呆。无心现在仰仗着他的食物以及房屋,所以不好抛了他独自大睡。打着赤膊蹲在他的身边,无心轻声问道:``你怎么不睡啊?''

白琉璃扭头望着窗外的白月亮:``我忧郁。''

无心很温柔的问道:``我给你唱首歌?''

白琉璃点了点头:``好。''

无心其实不大会唱,但是愿意安慰安慰白琉璃。开动脑筋思索片刻,他开口唱道:``啦啦啦,啦啦啦,我是卖报的——''

白琉璃一摆手:``算了算了,很吵。睡觉吧。明天你吃饱了就给我滚,我不要你了。''

无心躺下了,歪着脑袋看他的背影,是非常的不想滚。

翌日清晨,无心用净水把自己洗得头发黑皮肤白,然后熬酥油茶,把面饼和蜂蜜一起放到大盘子里,非常殷勤的为白琉璃预备早饭。

白琉璃吃了早饭,等着他自动滚。一直等到中午,无心给他烤了一块外焦里嫩的鹿肉。

白琉璃和他一起吃了肉。吃完之后他就不见了。白琉璃以为他滚了,心情平静许多。哪知到了天色将黑之时,他像个鬼似的,笑眯眯的又出现了。

\chapter{番外——无心和白琉璃(二)}

无心实在是没有更好的安身之处,所以只要白琉璃不往外推他,他就不走。

土司的家奴定期会给白琉璃送来粮食,鲜肉更是每天必有。白琉璃早上还未睡醒,就听耳边有人询问:``炖肉好不好?''他迷迷糊糊的``唔''了一声。然后在彻底清醒过后,就会嗅到满鼻子的肉香。

土司不会介意他私自收留一个汉人,他默默的吃着炖肉,吃了一块又一块。末了在嚼着肉汤里的煮蚕豆时,他决定暂时不再驱逐无心了——杀又杀不死,撵又撵不走;与其在他身上劳神费力,不如收他做个仆人,顺便研究研究他到底是个什么怪胎,为什么不怕自己的蛊毒。

无心盯着白琉璃的嘴,白琉璃每天都会用细盐擦牙齿,所以牙齿很白,比脸还白。脸也很白,但是因为一个礼拜至多洗一次,所以时常白得不甚纯粹。白琉璃把勺子一放,无心就到了开饭的时间。

白琉璃的胃口很有限,而无心又是位大方的厨子。背对着白琉璃蹲在地上,他留给白琉璃的只有一面后背和一个被旧裤子包裹着的屁股。白琉璃时常看不到他的后脑勺,因为他把脑袋埋到锅里去了。几顿油水富足的好饭过后,白琉璃发现无心正在奇妙的充盈——不是胖,而是充盈,皮肤里面含了水分,显出了应有的柔软与光泽。

无心在吃饱喝足之后,把注意力转向了白琉璃。白琉璃从早到晚,总像是无所事事。他仿佛是有眼疾,畏惧阳光,终日躲在阴暗处。无心嗅着他身上的怪味,看着他沉重的发辫,不禁身上做痒,替他难受。

``河水不凉。''他凑到白琉璃身边,察言观色的问道:``我带你去洗个澡,好不好?''白琉璃不看他,直接摇了摇头。无心哄着他:``洗干净了,很舒服的。''白琉璃轻声答道:``我不洗澡,怕伤元气。''无心暗暗吃了一惊:``你从来没洗过澡吗?''白琉璃略一迟疑:``有时候,擦一擦。''

无心从他的领口中嗅到了毒物的腥气:``今天很暖和,我给你擦擦身吧?''白琉璃缩了缩脖子,仿佛是被他的提议吓着了。

无心很愿意把白琉璃改头换面的打扫一番,因为白琉璃睡觉不安稳,夜里翻来覆去,翻得满屋子里都是奇异的臭气。然而他说了万千的好话,最后却只哄得白琉璃扯开领口,露出了左侧的肩膀和手臂。无心手里托着湿毛巾,发现他倒也算不得脏,只是皮肤表面似乎涂过某种油脂。湿毛巾轻轻的在他小臂上碰了碰,他一哆嗦,手臂像鱼似的从他手中抽出。

半边身体缩回锦袍里,他拢着袍襟说道:``不要了,凉。''无心把毛巾贴上了自己的脸:``不凉啊!''白琉璃坚决的摇头,而拒绝的原因,是无心后来才知道的——白琉璃的身体的确涂了油脂。油脂的成分和气味,可以安抚被他玩弄于股掌间的各色毒物。

白琉璃并不在乎自己的异味,反正身边常年没有亲近人,谁也不会挑剔他;而且他闻惯了,感觉很是麻木。除了他本人之外,和卧室相邻着的几间屋子也和他有异曲同工之妙——都阴暗,都神秘,都有着鲜明的古怪气味。白琉璃从来不允许无心进去,反正卧室对外开着门,无心根本也没有进去的必要。

当意识到无心是死心塌地的跟上自己时,白琉璃对他更有兴趣了。大清早的,他站在房内的窗前向外望。无心像官寨里的所有奴隶一样,穿着破衣打着赤脚。欣欣然的跪在一口大锅前,他正在动作娴熟的搅动一锅酥油茶。衣裳陈旧,他的头发和皮肤却是干干净净黑白分明。两只脚整整齐齐的交叠在屁股下面,露出了一小半脚掌和脚趾头,是鲜艳的粉红。忽然察觉到了白琉璃的目光,他回过头对着窗内一笑,黑眼睛里流光溢彩。

白琉璃对着自己点了点头,心想他是有资格陪伴自己的。白琉璃把无心当成了``自己人''。而在自己人面前,他毫无保留的露出了本来面目,导致无心立刻就起了外心——无心发现他喜怒无常,实在是个难伺候的人。

无心每天都要为他预备数目不定的几顿饭。早饭通常是很简单的,是酥油茶和糌粑,或者是面饼蘸蜂蜜。午饭就不正式准备了,无心可以随便烤点小东西给他吃。到了下午,无心要提前许久开工,因为摆在他面前的食材,很有可能是一头气势汹汹的大活羊。

除去固定的三餐,无心偶尔还要为白琉璃预备夜宵。不停的忙碌在火与锅之间,无心并没有落到好话,因为白琉璃肆无忌惮的挑三拣四,仿佛先前为他预备饮食的人全是御厨。到了夜里,白琉璃在床上闹失眠,翻来覆去的卷起满室腥风。无心远远的避开他,朦朦胧胧的想要尽快入睡。然而肩头忽然被他推了一下,他开口唤道:
``无心?''无心装睡,不想理他。

身后起了窸窸窣窣的响动,随即后背一暖,是白琉璃欠身贴上了他。柔软的丝绸袖子拂过了他的面颊,白琉璃很执着的去扒他的眼皮:``无心?''无心装不下去了,只好做如梦初醒状:``啊?''白琉璃说道:``我睡不着,你给我唱首歌吧。''无心眯着眼睛不想睁开:``你不是说我唱得不好吗?''白琉璃向后躺回去了:``唱吧。''

无心打了个轻飘飘的哈欠:``不唱了,还是睡吧。''然后他的小腿一痛,是被白琉璃狠狠踢了一脚:``唱!''无心叹息一声,背对着他清了清喉咙,用很苍凉的声音唱起了地藏经。白琉璃侧身望着他的背影,又伸手摸了摸他的后脑勺。

无心白天要干活,夜里要唱歌。干活唱歌倒也没什么的,反正吃饱喝足有力气。不过除了干活唱歌之外,他发现自己和白琉璃真是无话可说。白琉璃带上墨镜撑起阳伞,能在门口一坐坐上小半天。在门口坐腻了,他转身进入他的密室,关上房门继续一声不出。

无心很寂寞,于是在白琉璃的口粮中克扣了一些,用食物向牧民换了两只雪白的小羊羔。小动物没有不可爱的,小羊羔像两团小小的白云,咩咩的落在房前的草地上。无心算是有了个伴儿,时常抱着羊羔坐在草地上望风景。

白琉璃听到羊叫,无声无息的走出了房门。停在无心身后,他蹲下身摸了摸小羊羔的瘦脊背,又摸了摸无心的脑袋。无心侧过脸,低声笑道:``两只羊是一公一母,以后我们会有羊奶喝的。''白琉璃不置可否的一眨蓝眼睛,没说话。

无心因为无所事事,所以对于母羊羔的奶很有兴趣。他每天都把两只小羊收拾得干干净净,及至门口的青草被它们啃秃了,他就用一根细棍驱赶着它们往水草丰美的河边走。眼看小羊一天一天的长大了,这天上午他去官寨背一袋荞麦面,回来之后就发现两只小羊全不见了。

他急坏了,远远近近的找了个遍,最后进屋问白琉璃:``附近有狼吗?''白琉璃慢条斯理的往脖子上涂抹着一种古怪的白膏,一言不发的摇了摇头。无心无可奈何,只好作罢。如此过了几日,他在房屋内外嗅到了一股子罕有的腐臭气味。趁着白琉璃出门去了,他抽动鼻子,觅着气味推开了房中一扇木门。脑袋伸进去一瞧,他立时就傻了眼。

房中空空荡荡,只在正中央摆了一只鼎似的大铁盆。盆中盛着两只血淋淋的死羊羔。羊羔身上不知怎的,会有无数的出血点,咕嘟咕嘟的鼓出气泡,仿佛羊羔的尸体内部开了锅。他走近了,低头细看。正有一条细长的虫子从冒泡的血孔中蠕出了头。

无心很生气,坐在门口等着白琉璃回来,从天明一直等到天黑。最后在太阳快要落山之时,白琉璃终于骑着大白马,远远的出现了。无心打算对白琉璃做一番质问,不料白琉璃今天表现异常。从远方一直笑到近前,不知道他美的是哪一出。

无心看了他那个喜滋滋的德行,话在口中就犹豫着没有说。而白琉璃飞身下马,开口便道:``无心,恭喜我吧,我要做父亲了。''无心大吃一惊:``谁的孩子?''白琉璃瞪了眼睛,从墨镜后面露出半圈眼珠:``当然是我的!''无心又问:``还有人给你生孩子?''

白琉璃感觉他的言语都很不中听,于是抬手在他脸上拍了一下。等到白琉璃的手掠开了,无心的脸上显出了一个血点子,是不知被什么东西戳破了皮肉。

事后等到白琉璃消气了,才对无心说了实话。孩子的确是他的,因为他需要一个继承人。孩子的母亲是从汉地来的一个流浪女人,之所以愿意给他生孩子,是因为他给了女人一盒子雪亮的银元。现在女人藏在一处很隐秘的山洞里,有吃有喝。一旦把孩子生下来了,她自然就会带着银元回汉地去。

无心听了他的描述,认为那女人来历不明,所以很关切的追问了一句:``孩子真是你的吗?你别受了人家的骗。''白琉璃生气了,把一条硬壳大蜈蚣塞进了无心的领口里。

\chapter{番外——无心和白琉璃(三)}

无心躺在一处向阳的斜坡上,嘴里咬着一节草秆。牙关前后错动,草秆上下闪晃。一只金黄色的蜜蜂围着草秆嗡嗡了一阵,末了落在了无心的鼻尖上。无心懒洋洋的有一点高兴,蜜蜂的青睐,让他感觉自己像一朵讨人喜欢的花。

脚步声音由远及近的响起来了,蜜蜂振翅而飞,一片阴影笼罩了他的面孔。白琉璃居高临下的站在他身边,伸脚踢了踢他的软肋。他没看白琉璃,慢吞吞的坐起了身,扭头``呸''的一声把草秆啐出老远。

账还是要算的,在向白琉璃道过喜后,他想要为自己的小羊羔向白琉璃讨个说法。白琉璃是个敏于行讷于言的人物,当即表示自己没说法,于是无心开始和他赌气。

无心一如既往的给他做饭,床榻乱了,也会收拾;但是白天无心不理睬他了,夜里无心也不给他唱歌了。白琉璃爬到床里,向外一脚把他踢到了床下。床下也不凉,他侧身躺了,满不在乎的席地而睡。

白琉璃没想到他刀枪不入,不禁没了主意。好像骤然忘记了语言,他趴在床边,伸手向下去扳无心的肩膀,同时哑巴似的``啊''了一声。

无心很强硬的不肯动。于是他转而又去拍无心的脑袋:``啊!''

无心依旧是纹丝不动。

起身跟上白琉璃,无心赤脚踏过青翠草地。草长得都不算高,正好没过了他雪白的脚踝。他的裤腿已经散碎了,露出半截笔直的小腿。白琉璃并不是没有力量为他置办衣裳,非不能也,是不为也。他对无心此刻的寒伧模样十分满意,因为看起来正是个健康伶俐的好家奴。

他带着无心绕远路到了官寨前方,上楼和旺波土司作了一番长谈。末了土司毕恭毕敬的送他下楼,又让管家指挥奴隶,将一只竹筐拎到了他的面前。放到往日,白琉璃就得让土司的奴隶把竹筐一直送到官寨后方,但是现在有了无心,就不必再使用土司的奴隶了。

无心像只恭顺而又冷漠的牲口,白琉璃往回返,他就捧着竹筐跟上。及至回到住所,他把竹筐往门口地上一顿,然后又要往草坡走。

白琉璃叫住了他,让他杀一只小点儿的羊。一边说话,白琉璃一边掀开了竹筐的盖子。无心向内瞟了一眼,瞟得十分后悔,因为里面没有什么好东西,不是死蛇,就是骨骸。

白琉璃把竹筐拖进了他的密室。无心在外面宰羊。新鲜的羊排肉切成一条一条放在冷锅里,无心估摸着白琉璃一时半会儿不能露面,便拈起一条肉塞进嘴里。三嚼两嚼的把鲜肉吞咽了,他感觉味道还不错,便几次三番的伸手,把羊排肉吃了大半。

正是饱足之时,他一扭头,发现白琉璃竟然早已站在门口了。不安的咽了口唾沫,他的偷吃行为被捉了个现形,以至于他有点儿不好意思;然而白琉璃只笑了一下,蔚蓝的眼睛在睫毛掩映中波光闪烁,让人想起清澈的海。

无心收回目光,自顾自的开始忙着生火。

羊肉熟了之后,无心高高挑挑的堵在门口,低声问道:``吃不吃土豆泥?''

白琉璃正坐在床上发呆,冷不防听他开了口,不禁先愣了一下,随即反应过来,知道无心这是向自己示好了。

勉强压住脸上的笑意,他连连点头:``吃,吃。土豆泥里多加点酥油。''

等到羊肉和土豆泥全在房内摆好了,白琉璃又主动多搬了一把椅子放到桌子对面。旺波土司是个摩登的人,土司太太也在英国住过好些年。土司的摩登泽被四方,导致白琉璃的房内也摆了一副带着西洋风的结实桌椅。然而桌椅时常闲置,因为椅子总没有床舒服。

``进来。''白琉璃隔着窗子,对外面的无心招手:``一起吃。''

无心犹豫了一下,当真进屋在白琉璃的面前坐下了。桌子正中央摆着小山一样的羊肉和土豆泥。两个人微微一低头,就看不见对方的面孔。愚公移山似的默默吃了良久,白琉璃只挖去了山的一角。没滋没味的一歪脑袋,他俯身枕在了桌面上,从土豆泥的一侧露出眼睛去看无心。

``我的孩子很快就要出世了。''他告诉无心。

无心很有保留的一点头,还是感觉白琉璃被山洞里的女人给骗了。

无心太寂寞了,实在是想给自己找个伴儿,所以几日之后,房屋门口多了一只小黑狗。

小黑狗有着圆圆的眼睛和圆圆的鼻头,长大后会是一只很机灵的好猎犬。仰头张嘴露出几颗尖利的小牙,它奶声奶气的对着无心唧唧叫。小爪子踩上无心的脚背,它的眼神是婴儿的眼神。

无心很喜欢它,所以夜里听它在门口幽怨的哀鸣不止,就偷偷下床出门,抱着它又回了来。小黑狗只是需要一点温暖和爱抚,趴在无心怀里舔了舔鼻头,它立刻就老实了。

小黑狗老实了,白琉璃却又不老实了。他用胳膊肘狠杵无心的后背,让他把狗扔出去。无心抗命不从,但是态度很好:``我给你唱歌吧?''

白琉璃不言语了。等到估摸着无心睡着了,他小心翼翼的翻身凑过去,把手伸到无心身前,想要偷偷掐死狗崽。然而无心睡了,狗却没睡。他的手指刚像幽灵一样探过去,狗崽就吱吱大叫上了。

白琉璃吓了一跳,手臂当即顺势搭上无心,同时闭了眼睛装睡。无心惊醒了,一边拍着怀里狗崽,一边回头去看。见到白琉璃居然搂着自己睡觉,无心很不情愿的叹了口气,但是又不敢随便搬动对方,怕白琉璃醒了要闹事。无可奈何的躺回原位,他动静不小的又叹了一声。

一夜过后,天光大亮。白琉璃吃着蘸了蜂蜜的面饼,见无心正在用碎肉去喂小黑狗。小黑狗摇尾卖乖的样子,让他联想起了奴隶崽子。眼睛忽然一亮,他又发现了无心的新用处。

在小黑狗的骨架长得有型有款之时,无心出了趟门,回家之后发现小黑狗又没了。

小黑狗已经通了人性,比小羊羔更惹人怜爱。无心在卧室隔壁的密室里找到了小黑狗的尸体。这回他没有容许毒虫在小黑狗的体内滋生。把死狗拎到光天化日之下,他给小黑狗实行了火葬。

接连三天没理睬白琉璃后,他又给自己弄回了一对画眉鸟。白琉璃自己沉默寡言,但是希望无心能来逗着自己说话。无心不逗他只逗鸟,气得他拧断了画眉鸟的脖子,把它们扔进火堆里烧着吃了。

无心不好反复的闹脾气。哭笑不得的望着白琉璃,他暗暗定了主意,将来在离开白琉璃之时,必要将其痛揍一顿。

白琉璃盘腿坐在床上,嘴角还带着一丝黑灰,是刚吃过烤鸟肉的痕迹:``我的孩子马上就要出世,你什么都不要养了,只给我养孩子吧!''

无心垂下头,看着自己粘着草屑的赤脚:``我不会养孩子,你得找个奶妈才行。''

白琉璃固执的摇了摇头,他只相信无心。

无心倾斜着身体抬起一只脚,漫不经心的用脚趾头在地上写字:``孩子要吃奶,我又没有奶。''

白琉璃伸长脖子垂下眼帘,想要看他在写什么:``没关系,我们有羊奶。''

无心又道:``孩子的娘有奶,你让她先给孩子喂几个月,小孩子还是吃娘的奶最好。''

白琉璃拧起两道眉毛:``你是想偷懒吗?我要怎么样,就怎么样!我说了算,你说了不算。你要听我的,我不听你的!''

说完这话,他伸腿下床,穿了靴子就往外走。无心回头看他,只见他出门骑上大白马,在草地上迅速的颠没影了。

当天傍晚,太阳要落不落的时候,白琉璃回来了。

他像只手足无措的大猴子,缩手缩脚的踉跄下马,两只脚还未站稳,就一叠声的喊起了无心。无心快步跑到他的面前,就见他从怀里捧出了一个红赤赤的小婴儿。小婴儿还闭着眼睛,小身体柔嫩的将要半透明。白琉璃满脸都是笑,笑着看看婴儿,又笑着看看无心:``给你,给你。是个男孩子!''

无心倒是伺候过小孩子,所以接过婴儿之后,立刻就把婴儿抱舒服了。婴儿咧开薄薄的小嘴唇,低低的``耶?''了一声。而白琉璃从口袋里又掏出一只铁壳水壶。水壶上面绑着带子,他把带子套到了无心的脖子上:``这是羊奶。''

无心晃着脑袋一躲,没躲开:``哎?怎么着?真把我当奶妈使唤了?''

白琉璃翻身上了大白马,一抖缰绳又跑了。

当一轮明月升上天空时,白琉璃一手牵着大白马,一手牵着一只脏兮兮的胖母羊,慢慢的从远方走回了家。母羊的肚腹下垂着鼓胀的大奶,是他给儿子预备的粮仓。

母羊走得很不专心,时不时的低头啃草,搞得白琉璃总得用力拽它。距离家门越来越近了,门窗之中射出明黄色的温暖光芒;房门开着,白琉璃放眼望去,快乐的看到了无心。

无心坐在门槛上,双手抱着小小的婴儿,水壶放在脚旁地上。在温暖光明的背景中,他弯腰低头,是个委委屈屈的黑影子。

\chapter{番外——无心和白琉璃(四)}

无心让白琉璃去弄个胶皮嘴的玻璃奶瓶回来,白琉璃外出四处找了一圈,然而一无所获。

白琉璃的儿子已经睁开了眼睛,眼珠子是深沉的蓝黑色,有点老谋深算的意思。无心从早到晚的用小勺子舀了羊奶喂他,喂得不胜其烦。单手把婴儿托到母羊肚子底下,无心捏了羊□往他的嘴里送。母羊的奶水太充足了,无心的手指轻轻一捏,雪白的羊奶便喷射了婴儿一头一脸。婴儿呱呱的嚎哭起来,摇头摆尾张牙舞爪。白琉璃在房内听见了,隔着大开的窗户向无心怒吼:``你在干什么?''

无心跪在地上,扭头对着他正要回答,不料白琉璃怒不可遏的又叫道:``不要欺负我的儿子!''

无心把婴儿从羊肚子下面抱了出来,没好气的反驳道:``我是想要找个喂奶的新办法!''

白琉璃气势汹汹的伸手一指他:``你喂!就要你喂!''

无心微微张着嘴看他,胸膛里像是藏了一座火山。岩浆憋在嗓子眼里,随时能喷白琉璃一脸。

``你妈的。''他喃喃的骂道,抱着婴儿往远走,想要避开白琉璃的监视。白琉璃终日袖着双手,什么也不干,专门盯着他。婴儿略有哭闹,白琉璃便要痛心疾首的对他大呼小叫。

婴儿一到傍晚就哭,喂饱了也哭,哭得抽抽搭搭委委屈屈。无心抱着婴儿坐在门外的大石头上,手足无措的把臂弯晃成了摇篮。白琉璃困惑而又心痛的凑过来了,用手指逗弄着儿子的嫩下巴。婴儿哭得很卖力气,面红耳赤大汗淋漓。白琉璃急了,指尖轻轻去碰儿子的小嘴:``无心,他为什么一直哭?''

无心也是摸不清头脑:``你去找个养过孩子的女人问一问。''

话音落下,婴儿忽然安静了,小嘴吮住白琉璃的指尖,他仿佛得了某种安慰似的,一吮一吮的闭了眼睛,偶尔抽一口气。

无心恍然大悟:``哦,他要娘呢!孩子天生就离不得娘嘛!''

白琉璃抽出了手指——他的手不干净,不敢让儿子肆意的又吸又舔。一双蓝眼睛望向了无心,他脑筋一转,忽然有了高招。一挺身站起来,他快步进房拧了一把湿毛巾,随即回到无心面前,不由分说的扒开了无心的袍襟。手掌裹着湿毛巾胡乱擦拭了无心的胸膛,他夺过儿子就往对方胸前送。无心目瞪口呆的愣在大石头上,就见白琉璃准确利落的把婴儿小嘴贴上了自己的一侧□。而婴儿仿佛出自天性一般,竟然一口就把他叼住了。

``哎,白琉璃!''无心怕伤了孩子,所以姑且没有躲闪:``你过分了啊!''

白琉璃很专注的盯着儿子:``虽然小了一点,不过小孩子也不懂,能够骗他不哭就好。''

无心后仰着躲了一下,没躲开:``你没有吗?你自己骗去!''

白琉璃摇了摇头:``你没有毒,就用你吧!''

无心气得七窍生烟:``白琉璃,我不和你过了!''

白琉璃这才抬头面对了他,满脸的莫名其妙:``为什么?''

无心张口结舌,因为原因太多,一时也不能尽数。而白琉璃腾出一只手拍了拍他的肩膀:``我们还是过下去吧。自从你来了,我每天都很快乐。''

无心简直要落泪了:``你是快乐了,可我呢?''

白琉璃垂下眼帘望着儿子,用轻快的声音回答:``啊,不知道。''

无心瞪了他半天,然而白琉璃无动于衷。最后无心把脸转向了远方深深的夜色,胸前热烘烘的,还拱着个小猪似的活物。

这天晚上,无心是分外的垂头丧气,甚至有种受辱的感觉。白琉璃和他说话,他也不理了,倒在床上闷头就睡。白琉璃不睡,摸着黑逗儿子玩。婴儿躺在床上叽叽嘎嘎,声音不高,有种心平气和的乖。

如此到了翌日天明,白琉璃在吃过了一大盘土豆泥后,亲自用小勺子喂儿子喝羊奶。无心本来想去河里洗澡,袍子都脱了,然而半路又被白琉璃喊了回来。死气活样的把孩子抱稳当了,他百无聊赖的斜着眼睛,看白琉璃一小勺一小勺的舀起羊奶,送到婴儿的小嘴边,一次也就喂出一滴的分量。

及至喂光了一碗底的羊奶,白琉璃用**的小勺子刮了刮无心的□,想在这代用品上增加一点奶水气息,以便以假乱真。放下勺子小碗,他起身绕到无心身后,又把手伸到前方,在对方胸膛上捏起了一把肉:``儿子,看,妈妈。''

无心忍无可忍的仰起了头,拖着长声表示抱怨:``哎——呀——''

长声结束,无心用肩头狠狠撞开了白琉璃:``你还没完了?''

白琉璃一个踉跄跌坐下去。直眉瞪眼的想了想,他一翻身爬起来,却是钻进了他的密室。

片刻过后,他拎着一只绣花大荷包出来了。让无心抱着孩子在房内的床上坐好,他郑重其事的关了门窗,然后在无心面前打开荷包,从里面掏出了一沓崭新的钞票。捏着钞票向无心抖了抖,他压低声音说道:``我的钱,以后都归你管。你听我的话,我们好好过日子吧!''

无心一手抱着婴儿,一手把钞票接过来看了看:``这是哪国的钱?''

白琉璃郑重其事的答道:``是英镑,三百英镑。''然后他低头抻开荷包口:``除了英镑,还有几十块钱的法币。''

无心若有所思的点了点头:``英镑\ldots{}\ldots{}很值钱吧?''

白琉璃一扬眉毛:``当然。''

无心的眼睛亮了一下。

白琉璃把钞票放回大荷包里,又抽紧了荷包口。把荷包放到无心的手里,他很友爱的又拍了拍无心的胳膊。

无心一闲下来,就攥着白琉璃的大荷包浮想联翩。傍晚时分望着窗外的晚霞,他坐在阴暗的房内,满脑子都是活络主意。白琉璃和他的儿子全都吃饱喝足了,正在嬉闹。白琉璃捏着一根草,先是扫了扫无心的胸膛,又扫了扫儿子的小脸。婴儿躺在无心的臂弯里,扬起小手追逐草叶,追得哈哈大笑。白琉璃把婴儿的目光引到了无心身上,又用清朗的声音催促道:``吃奶,去,吃他的奶!''

小婴儿兴奋的``噢''了一声,然后在父亲的托举下,欢天喜地的扑向了无心。

无心没有做无谓的反抗。垂下眼帘望着身前的父子二人,他看到白琉璃还在逗蛐蛐似的用一根草秆逗着婴儿。

``真够讨厌的!''无心暗想:``我又要干活,又要照顾婴儿,还要被他当成玩物。妈的,老子不伺候了!''

无心一旦生出了``不伺候''的心思,立刻感觉天宽地阔。如此熬了十几天,他终于等到白琉璃又出了门。用一根布条把婴儿绑在床上,他揣起荷包,从床下翻出一双鞋穿好。推开房门东张西望了一番,他见远近无人,便撒腿跑了。

他是有备而跑,一路直奔四川,姑且不提。只说白琉璃当晚回了家,远远看到家里黑洞洞的没有点灯,心中就是一惊。及至距离家门近了,他听房内婴儿啼哭不止,房外的铁锅也是冷冷清清。推门进房一瞧,他见儿子在床上又拉又尿,嚎的上气不接下气。门外的母羊也跟着咩咩上了,吵得人心烦意乱。

慌忙挤了羊奶堵住儿子的嘴,他抱着婴儿房前房后跑了一圈,一边跑一边就听见自己在呼呼的喘粗气:``无心!''他大声的呼喊:``无心!''

四野寂静,哪里有人回答?

白琉璃单手抱着儿子,飞身上马跑向远方,一边跑一边继续呐喊:``无心!无心你回来啊!''

后半夜,白琉璃抱着哭累了的儿子回家了。

他自己也哑了嗓子。扯下床单扔在地上,他带着儿子往床上一躺。突然双眼一睁,他一个鲤鱼打挺坐起来,从床上到床下摸了一通,发现自己的大荷包也没有了。

人没了,钱也没了。他从中午到现在,还没有吃过一口饭。无心明明都答应和他一起过日子了,却又不声不响的偷偷携款逃走。想到无心骗了自己,白琉璃气得浑身颤抖。双手抓住被褥扭绞了一阵,他不解恨,攥了拳头向下狠狠一捶床板,随即开始满床打滚,一边打滚一边呻吟。婴儿窝在床角,好奇的睁大眼睛看着父亲,连哭都忘了。

白琉璃把床板捶得山响,``咕咚''一声滚到床下,他坐起来,一边扯着自己的袍子和腰带,一边伸腿用力去蹬前方的墙壁。两只脚敲鼓似的在墙上乱蹬了一气,他颤抖着骂了一声``骗子'',随即咬着手指起身冲出去,跪在门前地上仰天长啸。两只手薅住被母羊啃短了的青草,他拔一把向上一扔,再拔一把向上一扔。忽然看到无心常用的一只饭碗摆在锅子旁边,他跑过去拿起碗,高高举起摔在草地上,然后一脚接一脚把碗往土里踩:``骗子,骗子!''

白琉璃在门外一直闹到天亮,还是没能完全泄愤。铁锅已经被他不知扔到了哪里去,石头堆成的炉灶也被他拆了。他抹了自己一脸黑灰,滚得满头满脸都是草屑。最后在房内儿子的哭声中坐起身,他俯身一头撞向地面,抬起头又抽了自己两个大嘴巴。末了抬起袖子一抹眼睛,他也哭了。

\begin{quote}
——番外完
\end{quote}

\begin{quote}
作者有话要说:番外到此结束。接下来开始写本文第三部,讲述文革时期的故事。依旧是三人行,分别为无心,白琉璃的鬼魂,以及一位漂亮小姑娘。
\end{quote}

\part{文革时期}

\chapter{苏桃}

一九六七年春,河北。

苏桃斜挎着一只帆布书包,战战兢兢的走上了二楼。楼是旧式的小洋楼,坐落在文县一隅,还是清末时期的建筑,近十年来一直是空置着的。上个月随着父亲逃来此处之后,她始终是没有心思打扫环境,所以楼内处处肮脏;角落结着长长的灰尘,本是静止不动的,然而如今树欲静而风不止,在楼外一声高过一声的口号震动中,灰尘也柔曼的开始飘拂了。

父亲坐在门旁靠墙的硬木椅子上,见她来了,就仰起了一张苍老的面孔。苏桃停住脚步转向了他,茫然而又恐慌的唤了一声:``爸爸。''

老苏是个军人,人生经历就是一首陕北的信天游。年轻的时候是``骑洋马,挎洋枪,三哥哥吃了八路军的粮,有心回家看姑娘,打日本就顾不上。''人到中年了,又是``三八枪,没盖盖,八路军当兵的没太太,待到那打下榆林城,一人一个女学生。''虽然他打的不是榆林城,但的确是娶了个女学生。

女学生是中等地主家的女儿,又在中等城市里念了书,集小农与小布尔乔亚两种气质于一身,最终升华出了一个娇滴滴的苏桃。女学生一辈子看不上丈夫,带着独生女儿和丈夫两地分居。老苏倒是很爱她的,单相思,相思着倒好,因为见了面也没话说。

文化大革命开始不久,老苏就被打成了反革命黑帮分子。眼看他的上级保护伞们都被分批打倒且被踩上了一万只脚,他决定不能坐以待毙。然而未等他真正行动,就听说远在外省的妻子被当地红卫兵们推上了万人批斗大会的台子,当众用皮带劈头盖脸的抽,抽完了又剃阴阳头。大会结束后她回了家,当天夜里就跳楼自杀了。

等到女儿苏桃单枪匹马的逃到身边之后,老苏趁着自己只受批斗未受监视,在一位军中老友的保护下,火速逃来了文县,不显山不露水的暂时藏进了一所鬼宅似的小楼里。未等他喘匀了气,老友也完蛋了,被造反派押去了北京交代问题。老苏从首长落成了孤家寡人,并且不知怎的走漏风声,引来了新一批人马的围攻。

老苏依然是个行动派,趁夜用铁丝和铜锁死死封住了外面院门,又用湿泥巴和碎玻璃在墙头布了一道荆棘防线。但是他能拦得住人,拦不住声,而且拦也是暂时的拦,拦不长久。于是他彻夜未眠,一夜的工夫,把什么都想明白了。

苏桃站在门口,不敢往窗前凑。透过窗子可以清清楚楚的看到楼外情景。楼外的人员很杂,有红卫兵,也有本地工厂里的造反派,平时看着可能也都是一团和气的好人,不知怎的被邪魔附体,非要让素不相识的父亲投降,父亲不投降,就让父亲灭亡。忽然意识到了父亲的注目,她有点不好意思,扶着门框垂下了头。

老苏凝视着她,看她像她妈妈,是个美人。用粗糙的大手攥了攥女儿的小手,他开口问道:``东西都收拾好了?''苏桃点了点头,小声答道:``收拾好了。''老苏笑了一下,笑得满脸沟壑纵横:``好,收拾好了就快走。他们要往里冲了,院门挡不了多久。''

苏桃撩了他一眼,几乎被他惊人的老态刺痛了眼睛。从小到大,她一年能见父亲一面,因为不亲近,每次见面的印象反倒特别深刻。在她的印象中,父亲还是一个满面红光、高声大嗓的中年人。

``爸爸,一起走吧。''她带了哭腔:``妈妈没了,你不能留下我一个人,我一个人活不了啊!''老苏的嗓子哑了,喉咙像是被壅塞住了:``我目标太大,不利于你安全转移。''大巴掌狠狠一握女儿的手,他深深吸了一口气:``桃桃,对于爸爸来讲,杀头,我不怕;侮辱,我不受!''

随即他松了手。一双眼睛定定的盯着女儿。女儿十五岁,美得像一朵正当季节的桃花。暗暗的把牙一咬,他逼回了自己的眼泪,起身对着门外一挥手:``快走。非常时期,不要优柔寡断错失良机!''苏桃双手一起扳住了门框,惶恐悲伤的哭出了声:``爸爸,一起走吧,我求你了,一起走吧。要不然我和你一起死,我没家了,我没地方去!''

老苏屏住自己的呼吸和眼泪。拦腰抱起哇哇大哭的女儿,他一路咚咚咚的走下楼梯。脚步沉重,震得满地生尘。楼下一间小佛堂里,搬开佛龛有个锁着小铁门的暗道。老友在把他藏匿到此处时曾经告诉过他,说是暗道能用,直通外界。门锁被他夜里撬开了,铁门半开半掩的露出里面黑洞洞的世界。

把痛哭流涕的女儿强行塞进小铁门里,他拼了命的挤出声音:``我锁门了,你赶紧走!你想回来也没有路!''然后他``咣当''一声关了铁门,当真用锁头把铁门锁住了。重新把佛龛搬回原位,他小心翼翼的除去了自己留下的指纹。外面响起了哗啷啷的声音,他们当真开始冲击院门了。

老苏摸了摸绑在腰间的一圈炸药,以及插在手枪皮套里的配枪。两条腿忽然恢复了活力,他往楼上跑去,想要寻找一处绝佳的射击点。在老苏躲在窗边清点子弹、苏桃在漆黑的地道里绝望撼动铁门之时,无心随着人潮,涌出了文县火车站。

全国学生大串联的余波未尽,火车上的乘客之多,唯有沙丁鱼罐头可以与之媲美。无心在天津上车时,根本就没有走车门的心思。人在月台上做好准备,未等火车停稳,他就直接扒上车窗,像条四脚蛇似的游了进去。眼看身边的三人座位下面是个空当,他一言不发的继续钻,占据了座位下面的幽暗空间。

舒舒服服的侧身躺好了,他和苏桃一样,也有个帆布书包。书包里空空的,被他卷成一团当枕头。枕了片刻之后他一抬头,忽然想起书包里还有一条小白蛇。连忙欠身打开书包,他低头向内望去,就见小白蛇歪着脑袋,正用一只眼睛瞪他。

小白蛇是他从大兴安岭带出来的,蛇身上附着白琉璃的鬼魂。自从赛维和胜伊去世后,他就跑去了大兴安岭。山林已经变了模样,大片的树木都被砍伐了,大卡车昼夜不停的向山外运送木材。但是白琉璃所在的禁地还是老样子。一是因为此地偏僻,二是伐木工人不敢来。山中树木遮天蔽日,大白天的都闹鬼。

他在地堡中找到了白琉璃。白琉璃看了二十多年的花和雪,看得百无聊赖,见他忽然出现了,真是又惊又喜:``你来了?''无心在地堡中来回的走:``外面不大好混,不如到山里做野人。''白琉璃又问:``你是一个人?''无心坐在一口破木箱上:``嗯,我太太去年饿死了。''

赛维和胜伊,都没能度过大饥荒。胜伊一生结了两次婚又离了两次婚。感情生活的不幸让他活成了一个幽怨的小孩子。在长久的粗茶淡饭之后,他固执的闭了嘴,拒绝吃糠。可是赛维当时只能找到糠。胜伊胖胖的死了,营养不良导致他身体浮肿到变了形。

全城里都没有粮。无心把自己的棒子面糊糊留给赛维,想要出去另寻食物。然而城中的飞禽走兽全进了人的肚子。他往城外走,道路两边的树皮都被剥光了。树木白花花的晾在空气中,像是夹道欢迎的两排白骨。

后来,赛维也不吃了。赛维把仅有的一点棒子面熬成稀粥,然后关了房门,不让无心再走。一小锅稀粥就是无心接下来的饮食,她气若游丝的躺在床上,要无心陪陪自己,要自己一睁眼睛,就能看到无心。

她没有浮肿,是瘦成了皮包骨头的人干。十几年来她一手把握着整个家庭,像个大家长似的挣钱花钱,在体面的时候设法隐藏财富,在拮据的时候设法保留体面。她始终是不敢堂堂正正的抛头露面,因为父亲是大汉奸马浩然。藏头露尾的经营至今,她也累了。

她不让无心走,无心就不走。无心躺在她的身边,两人分享着一个被窝。他是她的丈夫,也像她的孩子。赛维一过三十岁,在街上见到同龄的妇人领着小儿女,也知道眼馋了。

赛维枕着他的手臂,很安静的走了。无心用手指描画着她的眉眼,想起了两人十几年的争吵,想起了她年轻时候的清秀模样。想到最后,他的眼睛涌出一滴很大的眼泪。眼泪是粘稠透明的胶质,凝在脸上不肯流。

无心在安葬了赛维之后,就开始了他的流浪。和白琉璃在地堡里住了几年,他得知外面的大饥荒已经彻底过去了,便又起了活动的心思。听闻他要走,白琉璃当即附在一条白蛇身上:``把我也带上吧!我在地堡里住太久了,想去看看外面的世界。''无心大摇其头:``不带不带,我烦你。''

白琉璃没说什么。等到无心睡着了,他盘在无心的脖子上,张嘴露出倒钩尖牙,对着无心的鼻尖就是一口。无心差点没疼死,白琉璃沾染了无心的鲜血,也险些魂飞魄散。双方两败俱伤,只好和谈。和谈的结果是双方各退一步,无心带白琉璃出门见世面,但是白琉璃路上必须听话。

无心在山里住了四年,万没想到四年之后,天地剧变,竟然换了一个世界。他审时度势,立刻学会了不少崭新的革命词,并且凭着自己面嫩,冒充大中学生,拿着伪造的介绍信混到各地的红卫兵接待站中骗吃骗喝。混着混着混到了文县,他出了火车站,独自走在一条安静小街上,并不知道自己在一个小时之后,就会遇到漂亮的小姑娘苏桃了。

\chapter{相遇}

苏桃一边抽泣,一边晃着手电筒弯着腰往前跑。暗道长得无边无际,前后只有她粗重的喘息声音在回荡。此时距离她与无心相遇,还有四十分钟。

无心依然东张西望的走在无人的小街上。小街一侧是成排的树木,树木之外则是荒原;另一侧砌了高高的红墙,红墙之内寂静无声。无心根据自己近几个月走南闯北的经验,猜测红墙之内应是一处机关,可到底是什么机关,就说不准了。

低头系好空瘪瘪的书包,又把一身的蓝布工人装整理了一番,最后蹲下身,他紧了紧脚上回力球鞋的鞋带。球鞋是他在南开大学红卫兵接待站里偷的,当时几十个人睡一间大教室,他在凌晨清醒之后,下了课桌拼成的大通铺,低头看到地上摆着一双崭新的球鞋,便不声不响的穿了上,抱着书包悄悄溜出大学,直奔火车站去了。

书包空瘪瘪,他的肚子也是空瘪瘪。文县当然也有红卫兵接待站,可是此地的斗争显然是异常激烈,火车站和主要街道都被游行队伍充满了,他一时竟然没有找到接待人员。没有就没有,他总有办法填饱肚子。仰起头望了望一人多高的红墙,他见墙头平坦,便起了主意,想要翻墙过去,探一探里面的情况。

眼看左右无人,他后退两步一个助跑,``噌''的直窜上墙。双手攀住墙头,他摇头摆尾的扭了几扭,轻而易举的将小半个身子探入了墙内。居高临下的放眼一瞧,他就见距离高墙不远,便是一排整整齐齐的红砖瓦房。阳光明媚,天气和暖,瓦房的后窗户三三两两的敞开了,可见房中全都无人。至于房屋前方是什么形势,就不得而知了。

无心轻轻巧巧的越过墙头跳了下去,猫着腰贴到大开的一间窗子下,慢慢抬头向内张望。房中靠窗摆着一张大办公桌,桌上堆着一沓文件,一支拧开了的钢笔,一把瓜子,几只柿饼。文件上面放了一盘红色印泥,印泥上面立着个挺大的木头印章。正对着后窗户的房门也开着,两名穿着旧军装的半大孩子大概是担负了卫兵的职责,背对着房内站在门口,偶尔左右晃一晃身体。

无心一看卫兵的模样,就猜出此地应该是某处造反派的总部。缓缓直起了腰,他打开自己的书包,随即出手如电。不过是一眨眼的工夫,瓜子和柿饼就全砸在了小白蛇的身上。眼看办公桌下的抽屉没有锁,他一边瞄着门口卫兵的动静,一边慢慢拉开抽屉。一只手忽然变得无限大,他在抽屉里抓出了一大把全国粮票。

小小心心的关了抽屉,他想要撤。临撤之前一犹豫,他一时使坏,把桌上的大红公章也一并揣进了书包。转身一窜上了墙头,他飞檐走壁的回到了墙外小路上。站在树后清点了贼赃,他把粮票数清楚了,放在书包里面的夹层口袋中;又把一沓文件打开了,仔细一瞧,原来不是文件,是一沓没抬头没落款空白介绍信。

在当今的世道里,介绍信可是有用的好东西。无心把空白介绍信折叠整齐了,放在另一个夹层口袋里。公章他没仔细看,随手用纸包了扔在书包深处。抓起一把瓜子托在手里,他上了路,一边嗑瓜子一边往前走。许多许多年前,他记得自己是来过文县的,不过当年那个文县和如今这个文县,似乎完全没有联系。

现在的文县是个工业区,因为有人在附近的猪头山里勘探出了铁矿,铁矿引来了一座钢厂,而钢厂发展壮大之后,新的大机械厂也在文县安家落户了。在县城里,土生土长的文县人占了少数,更多的居民是从外地迁来的工人家庭。单从繁华的程度来看,文县并不次于一般的城市了。

瓜子磕了一路,无心越磕越饿,打算找个小饭馆吃上一顿。不料就在他咽下最后一粒瓜子瓤时,远方忽然起了一声巨响,是个大爆炸的动静。无心脚步一顿,同时就见一个灰头土脸的影子从树木后面爬上路基。手扶大树觅声远望,影子一哆嗦,随即就蹲下不动了。

无心莫名其妙,因看来人耳后耷拉着两条毛刺刺的长辫子,可见是个姑娘,而且还是个小姑娘,便好心好意的上前说道:``你害怕了?没事,爆炸离我们远着呢,崩不着你。''苏桃含着满眼的泪水抬起了头,一眼瞧见了无心手臂上套着的红卫兵袖章。

鲜红的袖章像是一泼血,刺得她双眼生疼。而她本来就蹲在倾斜向下的路基上,此刻一时受惊,失了平衡。抱着膝盖向后一仰,她未等说话,已是一个后空翻滚了下去。无心和蔼可亲的弯着腰,正被她脚上的解放鞋踢中下巴。啊呀一声仰起头,他舌尖一痛,已被牙齿咬出了血。而苏桃一溜烟的滚到了路基下方的野地上。

四脚着地的爬起身,她惊慌失措的向上又看了无心一眼,同时一张嘴越咧越大,露出了个没遮没掩的哭相。无心揉着下巴,低头看她:``你没事吧?''苏桃想逃,可实在是逃不动了。两条腿打着颤撑住了身体,她抬手指向爆炸的方向,干张嘴发不出声,只用气流和口型说道:``爸爸\ldots{}\ldots{}是我爸爸\ldots{}\ldots{}''

眼泪滔滔的涌出眼角滑过面颊,她豁出命了,在紊乱的气息中高一声低一声的告诉无心:``我爸爸死了\ldots{}\ldots{}我无处可逃,你们要杀就杀,我没什么可交代的,我不怕死\ldots{}\ldots{}''无心隐隐明白了:``你爸爸\ldots{}\ldots{}''他思索着用了个新词:``自绝于人民了?''

苏桃穿着一身半新不旧的军装,袖子偏长了,两只手攥成拳头缩在袖口里。身体紧张的向前佝偻成了一张弓,她在春日艳阳下哭得满脸都是眼泪:``我爸爸没罪\ldots{}\ldots{}我爸爸没反对过毛主席\ldots{}\ldots{}''无心彻底明白了,眼看苏桃哭得面红耳赤,他有点手足无措,仿佛是大人没正经,把好好的孩子逗哭了。``别怕别怕。''他拍拍自己的胸膛:``我不管你家里的事,我是外地来的。你妈妈呢?一个人哭也没用,我带你找你妈妈去吧。''

苏桃摇摇头,眼泪源源不断的流,哭声却是始终哽在喉咙里:``妈妈也没了,妈妈让人逼死了。''无心生了恻隐之心,扶着大树往下面走:``有话上来说,下面全是泥。你放心,我是过路的人,不会检举你,也不会揭发你。''

避开昨夜小雨留下的一个个泥洼,无心从裤兜里摸出了一条手帕。迟迟疑疑的抬起一只手,他想给苏桃擦擦眼泪,可苏桃的年龄正处在小丫头与大姑娘之间,让异性拿不准应该如何对待她。眼看苏桃哭得直抽,无心一横心,一手托住她的后脑勺,一手用手帕抹了她的眼泪和鼻涕。

满面尘灰随着涕泪一起被拭去了,苏桃在金色的阳光中微微扬头,显出了两道弯弯的眉毛,一双清澈的眼睛。眉毛的笔触是柔软的,眼睛的颜色是分明的,她张开嘴吸了口气,柔软的嘴角随之抽搐了一下。

无心用手帕垫了手,最后在她的小鼻尖上又拧了一把:``别哭了,想想接下来该怎么办?''苏桃摇了摇头,后脑勺的头发中分梳开编了辫子,清晰的发缝就摩擦了无心的手掌:``我不知道,我没有亲人了。''她抽了口气:``我爸爸是孤儿。''又抽气:``我姥爷是地主。''继续抽气:``去年他和姥姥一起,让造反的给——''最后抽气:``活埋了。''

无心看她抽搭得直出汗,自己既问不出主意,她哭狠了没过劲,回答得也是辛苦。她肯定是走投无路了,自己若是抛了她不管,很不忍心。多俊俏的小姑娘啊,真要是落到造反派的手里,怕是死都不得好死。可若是管她,怎么管?

``你要是信得过我,就跟我走。''他低声说道:``能往哪里走,我也不知道,走一步看一步。你要是不信我,我给你十斤全国粮票,然后各走各的路。怎么样?你说吧。''苏桃垂着头,不说话。无心看她不言语,就从书包里摸出了几张粮票,要往她手里塞。然而她把手往后一撤,却是不肯要。

无心捏着粮票顿了顿:``你想\ldots{}\ldots{}跟我走?''苏桃依旧是一声不吭。无心拉起了她的手,转身向路基走了一步。他走一步,苏桃跟一步;他停了步子回头看苏桃,苏桃深深的低着头,不理他。无心一笑,扯着她几大步跑上路基。在小路上站稳了,他给苏桃从上到下拍了拍灰,同时问道:``你叫什么名字?多大了?''

苏桃不敢出声,一出声就憋不住眼泪,只能蚊子哼:``苏桃,十五。''无心打开书包,想要拿柿饼给她吃。然而低头一瞧,他大吃一惊。原来书包里至少有五个柿饼,如今却是只剩了一个。剩下的一个,也被小白蛇咬上了。

无心气得在蛇脑袋上凿了个爆栗,然后在书包里偷偷捏开蛇嘴,把柿饼从它的倒钩牙上摘了下来。还好,柿饼基本保持了完整,只是留下了两个洞眼,乃是小白蛇的牙印。白琉璃躲在小白蛇的躯体内,颇为不满的瞪了无心一眼。

把从蛇嘴里夺下的柿饼塞到苏桃的手里,他像个大哥哥似的,拉起她另一只手向前走:``吃吧,你是个命大的,得好好活着。你活好了,你死去的亲人才能瞑目。''

白琉璃躲在书包里,有日子没听无心说过这么通情达理的话了,便好奇的把脑袋伸出书包缝隙,想要窥视一下无心献媚的对象。哪知无心的感官十分敏锐,他的脑袋刚见天日,就被无心一指头又戳回去了。

无心和苏桃无处安身,漫无目的的走过一条小街,迎面却是看到一座大校园。校门并没有锁,门口的木牌上写着一排黑字,正是``文县重型机械厂子弟第一中学''。

如今的大中学校都停课了,操场一边的自行车棚里一辆车都没有,收发室也关了门,玻璃窗灰蒙蒙的。无心见状,心中一动,回头说道:``苏桃,我们进去瞧瞧?要是真没人的话,你找个地方先呆着,我出去买点吃的回来。''

苏桃还捏着柿饼,不过能够抬头面对无心了:``嗯。''无心还拉着她的一只手,有时候感觉她是个小妹妹,很自然;有时候又感觉她是个漂亮姑娘,不好意思。

探险似的进了校园,他和苏桃先往操场正中的教学大楼里走,大楼是三层,一进门不用远走,第一感觉就是久无人烟。无心走到了一楼的走廊尽头,把苏桃带进了一间空教室。空教室的窗户对着楼侧,他向苏桃吩咐道:``你蹲在角落里,不要轻易露头。一旦有人来了,你就跳窗户出去,往树丛里跑。我买了吃的就回来,你想吃什么?''

苏桃低头打开书包,从里面掏出了两块钱递给无心:``我们搭伙\ldots{}\ldots{}你出粮票我出钱吧。''无心真没钱,于是很痛快的接了钞票:``你想吃什么?''苏桃摇了摇头:``你吃什么我就吃什么。''无心答应了,又对她嘱咐道:``蹲好了,别打瞌睡,留神着外面的动静,记住我说的话。''

苏桃立刻走到靠窗的墙角处,抱着膝盖蹲下了:``我知道。''无心看她好像缓过精神了,便放了心。打开一扇窗户半掩了,他对着苏桃又点了点头,然后转身向外走去。

\chapter{鬼神缠身}

无心走在文县繁华的大街上,街道两边的电线杆子上都架了高音喇叭,正在播放革命歌曲。游行队伍还没有来,无心在``大海航行靠舵手''的歌声中进了一家饭馆,然而饭馆只卖面条,没有别的。无心身上连个饭盒都没有,没法把一碗连汤带水的热面条带回学校,所以出了饭馆继续往前走,想要找个面食铺子。可是一条大街都走到头了,硬是没找到。

耳听远方人山人海的口号声越来越近了,他当机立断的进入百货商店,买了一只铝饭盒。随即就近进了一家饭馆,他问了问服务员,得知想要买主食,必须附带炒菜。于是他要了一个肉丝炒白菜。在饭馆内的公用水龙头前洗了洗新饭盒,他在等着菜熟之时,又要了十个烧饼。

肉丝炒白菜总也不好,无心把十个烧饼用纸包好了塞进书包里,在饭馆里坐立不安。服务员是个又胖又大的姑娘,倚着墙壁横了他一眼:``等就等呗,你乱晃什么呀?''
无心骑在一条长板凳上,望着窗外答道:``我饿。''

胖姑娘当即一撇嘴,同时墙壁上的窗口里响起了一声吆喝,正是肉丝炒白菜出锅了。出锅之后也没有服务员的事,无心作为食客,自己走去窗口端了菜,把一盘热菜倒进了饭盒。

菜有了,主食也有了。无心挎着热气腾腾的书包,推开店门往外走。然而走出没有几步,就走不成了。前头山呼海啸,是一支千人游行队伍;后头海啸山呼,依然是一支千人游行队伍。两支队伍各喊各的口号,杀气凛凛的走了个顶头碰。

无心根据近几个月的所见所闻,怀疑两支队伍是对立的两派,正憋着干上一仗。紧靠街边贴了墙,前后的道路都被带着红卫兵袖章的青年们堵死了。忽然身边``咣''的一声,他扭头一看,发现胖服务员从里面把饭馆大门给锁上了。里面等菜的几名食客惶惶然的把脸贴上玻璃窗,全是受了惊的模样。

两派人马终于是面对面了。好像一对老冤家似的,一派高喊``革命无罪、造反有理'',另一派立刻附和``江青万岁''。在双方达成共识的基础上,其中一派骤然发起冲击,随即大混战就开始了。无心抱着他的书包,蜷缩着躲在了饭店的屋檐下。

大混战的两派似乎是以学生为主,武器以拳脚和牙齿为主。一个半大男孩一手抡着一条车链子,一手揪着个小姑娘,正在往小姑娘的头上猛抽。而小姑娘挨了几下狠的之后,大喝一声猛踢一脚,脚背正磕在半大男孩的裆口。无心拧着眉毛,清清楚楚的听到了一声带有破碎嫌疑的闷响。半大男孩也没有叫,翻着白眼就倒下去了。

满街越打越是失控,正是人仰马翻之际,一辆披红挂彩的大卡车从街尾开来了,卡车后斗上整整齐齐的站着一队工人,手里全拄着一人来高的木棒。卡车停在街尾开不动了,戴着安全帽的工人们络绎跳下,一声呐喊冲向前方。脑袋被车链子抽成花瓜的小姑娘见状,锐声叫道:``联指的同志们,看哪!他们带武器了!''

一个穿着褪色旧军装的大个子男学生踩上路边的水泥花坛,握着拳头吼道:``我们革命小将一不怕苦、二不怕死!革命不怕死,怕死不革命!杀了我一个,还有后来人!他们有援兵,我们也有援兵!''

话音落下,援兵果然来了。无心贴着墙边正想慢慢溜,一边溜一边就见大街另一端开来三辆卡车,卡车上面也是满载着青年工人。不过工人手中的武器甚为可怕,是一头削尖了的钢筋。带着钢筋的工人们,穿着灰色工作服;带着木棒的工人们,穿着蓝色工作服。

无心低头看了看自己的颜色,不由得吓出了冷汗——自己正穿着一身蓝布工人装!他知道凭着自己的装束,很有可能被人扎个透明窟窿。抱着书包紧贴了墙,他学螃蟹横着走。走出没有一米远,一个人高马大的小子一把搂住了他:``我逮着一个活的!''

众人都忙着打,没人理他。无心向他当胸击出一拳,小子硬挺着扛住了,死活就是不松手,同时扯着嗓子大喊大叫:``田小蕊,李萌萌,来帮一把啊!我活捉了一个红总的!''无心急了,拼了命的想要挣扎。然而对方粗胳膊长腿,箍着他死活不放。

双方正在纠缠,一只雪亮的钢筋尖反射阳光,在无心的眼前晃了个圈。无心立刻就不动了。面前手持钢筋的工人,是个黝黑黝黑的青年。皮肤黑,神情如果有颜色的话,应该也是阴沉沉的黑。上下打量了无心的模样,黑脸青年点了点头。而无心抢着喊道:``我是过路的!放了我吧,没我的事!''

黑脸青年冷笑一声,口中说道:``顾基,把他看住了!等到战争结束,我们再来处理俘虏!''搂着无心的小子立刻答应一声,然后搂的更紧了。黑脸青年手持钢筋改造的长矛,投身到了轰轰烈烈的战斗中去。无心背对着顾基面对着战场,大声问道:``红总是谁啊?''顾基瓮声瓮气的答道:``红色造反总司令部。''答完之后他又一愣:``你明知故问,装什么装?''

无心无可奈何:``顾同志,我真不是红总的。不信我给你看介绍信,我是从东北来的!''
顾基对着他的后脑勺骂道:``滚一边去吧!老子不信你的鬼话!''无心再问:``你们又是哪个组织啊?''顾基答道:``我们是联指的!''

无心明白了,所谓``联指'',就是无产阶级革命派联合指挥部。看来联指和红总是一对仇家,而自己要是光着屁股上街,兴许还不会卷进两派的大混战里。和无心一起明白的,是红总一派。红总一派在十分钟之内撤退了,留下了两具血淋淋的小尸首。

死的没人管,活的可有人看。无心被人反剪双手,一直押到了联指在文县的总部。总部占据了一所小学校,无心因为老老实实,所以没有挨打。末了抱着书包蹲在小学校的院里,他抬头望着顾基、被人称为陈部长的黑脸青年、以及头如花瓜、脚能碎蛋的红卫兵小将李萌萌。

李萌萌用毛巾擦着满头满脸的伤,人已经看不出模样了,脸蛋被车链子抽破了好几处皮。陈部长一身的鲜血,当然都是敌人的血。顾基的块头最大,人也最怂,是条茫茫然的尾巴,不是跟着李萌萌,就是跟着陈部长。

陈部长换了一身干净衣裳,手里拎着一条军用皮带走到无心面前。皮带折成几折握在手里,他微微弯腰,用皮带抬起了无心的下巴:``我问你,你是想坦白从宽呢,还是想抗拒从严?''无心打开了书包:``我给你看我的介绍信,我真不是红总的。''

他从烧饼和饭盒下面掏出了一张折好的纸,向上递给陈部长。陈部长接过来展开,垂着眼皮看了一遍,没看出真假来:``哦,你是哈尔滨三中的啊?''无心连忙点头:``是,是。你再看看我——我的衣服和红总不一样。乍一看挺像,其实不是一回事。''

陈部长居高临下的又问:``有学生证吗?''无心摇了摇头:``学生证在火车上挤丢了,就剩一张介绍信。''陈部长审视着他:``只有你一个人?''无心略一思索,立刻答道:``不是,还有几个初二初三的,在家是我的邻居。我们一起上的火车,下车的时候挤散了,我正满街找他们呢!''

陈部长刚要继续说话,院外却是气喘吁吁的跑来了一个人,在陈部长耳边说道:``赵健死在医院里了。姓苏的枪法准,子弹打得太刁了,就贴着心脏,医生都没法给他做手术。''

陈部长黑着一张脸,忿忿然的叹道:``黑帮分子真是罪大恶极,不但躲在资产阶级的小洋楼里负隅顽抗,临死还要拉上革命群众做垫背的!你们也是愚蠢至极,一百个人,逮不住一个,还搭了三条性命!我告诉你,省联指的三号勤务员,马上就要从保定过来指挥工作。三号代表的是一号,一号代表的是江青同志。你们把事情搞成一团糟,看看以后怎么向三号作报告!''

陈部长扬着黑脸,在院子里指点江山。而顾基吸了吸空气中的面香,低头瞄向了无心的书包。无心留意到了,只做不知。

陈部长单手叉腰做出伟人姿态,当着众人办起了公。无心眼看天色渐渐暗淡,心里惦记着藏在中学校里的苏桃,自己又饿得难受。而陈部长说到了一定的程度,竟然忘记了无心的存在,带着李萌萌出了门,院子里只剩了一个顾基,还在认真的充当看守。

无心从书包里拿了一个温暖的烧饼,起身递向了顾基。顾基警惕的瞥了他一眼,看他一脸的坦诚,便接过烧饼塞进嘴里,一口把烧饼咬成了一个月牙。无心看他吃得挺香,趁机问道:``什么时候能放我走啊?''顾基摇了头:``得听陈部长的。''无心又问:``你说了不算?''顾基显然是有些羞愧:``我不行。我什么都不是。''

无心望了望四周:``天都黑了,我还想找你们的红卫兵接待站睡觉呢!''顾基吃了一个烧饼之后,立刻和气许多:``放心,有你睡觉的地方,在哪儿不能对付一宿?''无心提了提裤子:``我想去趟厕所。放心,我不跑。反正误会都解释开了,我离了你们,也没地方去不是?''顾基指了指校园角落的厕所:``去吧,小心点儿,别掉坑里。''

无心笑模笑样的走向厕所,越走越快。及至进了臭气熏天的厕所,他望着后墙,开始筹划越狱。

在无心避开满地屎尿想要爬墙之时,苏桃在空教室里坐了足足半天。因为胆子小,她唯一的运动就是伸了伸胳膊腿儿。她没什么主意了,无心让她等,她就死心塌地的等。等到日落西山了,她又渴又饿,迷迷糊糊的入了睡。

不知睡了多久,她清醒了。醒后揉了揉眼睛,她忽然吃了一惊,发现白天还是空空荡荡的教学楼,此刻居然有了灯光——除了她所在的空教室。

她又惊又怕,抱着书包慢慢站起,绕过七扭八歪的桌椅走向了门口。走廊黑洞洞的长到无限,走廊两边的教室里散发出了冷森森的光。停在一扇门前,她从门上的玻璃窗向内望,就见教室里面桌椅井然,坐满了十几岁的学生。一位男老师站在讲台上,正在黑板上书写数学式子。

苏桃懵了。现在全国都在停课闹革命,怎么还会有学生老师来上晚自习?老师在黑板上一直写,学生低着头,在下面也是一直写。她蹑手蹑脚的转了身,又凑到对面一扇门前向内望。教室里也是同样的情景,她斜着眼睛瞟了黑板一眼,黑板上也是数学式子,以sin开头,没头没尾写了半黑板。苏桃心想看来他们是一个年级,正在学同样的知识。

再把目光投向学生,她越看越不对劲。忽然扭头又回了第一间教室门前,她在重新的观察之后,脑子里``嗡''的响了一声——两间教室的老师学生,竟然是同一批人!

她屏住呼吸继续往前,在前方第三间教室门口停住,看到上面是同样面目的老师在黑板上写着同样的式子,下面的学生,第一排坐着两个小胖子,第二排靠墙是一对双胞胎女生,最后一排坐着个穿篮球衣的高个子\ldots{}\ldots{}三间教室像是复制品,呈现着完全相同的情景!

苏桃怕了,转身要往外跑,可是脚下一个踉跄,她``咕咚''一声,倾斜着身体撞开了房门。诡异的宁静瞬间被打破了,书写式子的老师停下粉笔,慢慢的转身面对了她。她抬头一看,立时战栗着发出了一声尖叫!

男老师有着一张苍黑的脸,黑眼珠翻上去,露出了带着血丝的眼白。鼻子颧骨全都扭曲的高耸着,他张着嘴,露出了一嘴黑黄的牙齿。除去脸上的伤痕之外,他的脖子上翻开一道深深的红伤,甚至露出了白色的骨茬——如此的人,已经不应该是活人!

仿佛对苏桃的打扰十分不满,男老师一步一步走向了她,学生们起立了,面无表情的也逼向了她。苏桃连滚带爬的起了身,抱着书包要往前跑。然而走廊两边的教室门都打开了,无数个一模一样的男老师开始向她围攻。

她跌坐在地,正是吓得肝胆俱裂。然而眉心忽然重重的一痛,她狠狠一闭眼睛,再睁开时,发现周遭恢复了黑暗,而无心蹲在自己面前,正在关注的望着自己:``别怕,我回来了。''苏桃气息一颤。张开双臂搂住了无心的脖子,同时带着哭腔说道:``你怎么才回来?我刚做了个噩梦,吓死我了!''

无心拍着她的后背,没有说话。而苏桃眨了眨一双泪眼,心中忽然一惊,发现自己竟然身在走廊。

\chapter{相依为命}

苏桃傻了眼,一手拉着无心,一手指向走廊尽头,干张嘴说不出话。忽然松手扑向走廊一旁的教室房门,她大睁着眼睛往里瞧。教室里面空空荡荡的,别说人了,连老鼠都没一只。

无心明知道她方才是被鬼魇住了,但是不肯说破,怕吓着她,只问:``是不是梦游了?''苏桃一听``梦游''二字,感觉方才的经历起码从科学上说得通了,才透过了一口气,惶惶然的答道:``我没有梦游症呀!''无心思索着说道:``白天受了一天的惊吓和辛苦,难保晚上不会有些异常的反应。没事了,我们还回空教室里去吧!''

他拉着苏桃的手往回走,苏桃紧紧靠着他的手臂,看他像一座保护神。两人进了教室,还是在角落处坐定了,也不敢开灯。无心掏出上层的饭盒,打开了盖子放到苏桃面前:``没勺没筷子,用手抓着吃吧!中午就买好了,哪知道刚一出饭馆就遇上了两派打仗。我让联指的人抓走了,关了一下午。''

然后他又拿出了烧饼。教室里黑,苏桃不留意,无心却是眼尖,发现包着烧饼的油纸破了一大串窟窿,每个烧饼都被咬去了一点。从中间挑了个软和的烧饼递给苏桃,他暗暗把手伸进书包摸到小白蛇,在蛇脑袋上连弹九指。

苏桃接了烧饼,小声问道:``他们打你了吗?''无心摇头笑道:``没打。他们以为我是什么红总的,解释开了,也就完了。''苏桃撕了一块烧饼往嘴里送:``你别和他们硬碰硬,他们打死人不偿命的。''无心把饭盒向她推了推:``吃菜。别讲究了,自己伸手。不干不净,吃了没病。''

苏桃捏了一片白菜吃了,随即心事重重的望向无心:``明天\ldots{}\ldots{}你去哪里啊?''无心想了想,然后笑了:``我有点拿不准。和你说实话吧,我是从联指总部翻墙逃回来的。文县打得有点儿太厉害了,要是能走,我想走。''苏桃垂下了头:``我跟你一起走,行吗?''无心伸手摸了摸她的毛糙辫子:``行。我也是一个人,你跟我走,我们还能搭个伴儿。''

苏桃吃了两个烧饼,吃饱了。无心带着她往外走。学校里面必定会有自来水,他们穿过长长的走廊,在大楼另一端找到了水房。水房是间大水泥屋子,屋子一角立着个烧热水的锅炉,三面墙上都伸着水龙头。无心一个接一个的拧,总算拧出了一个有水的。

任凭流水放了一会儿,他约莫着有水锈也流光了,才刷了刷饭盒,又用饭盒接了小半盒水给了苏桃。苏桃咕咚咕咚喝了一气,无心又问:``想上厕所吗?''苏桃把饭盒还给了无心,喃喃的说:``不去了,怪害怕的,我能憋住。''

无心环视了伸手不见五指的水房,灵机一动:``要不然,你就在水房把问题解决了吧!我给你守门,你速战速决。''苏桃在黑暗中夹着腿,千分的害羞,万分的着急:``我\ldots{}\ldots{}''无心走到了门口,走廊里还有一点微光,他给了苏桃一个背影:``快点儿吧!''

苏桃解了裤子,靠墙蹲了。天下事常是事与愿违,她极力的想要做到斯文无声,然而环境太安静了,她心惊胆战的支着耳朵,感觉自己哗哗哗的尿出了一条大河。一条大河波浪宽,她面红耳赤的挪了挪脚,不想弄脏了自己的鞋。

提起裤子又洗了洗手,她走到无心身后,犹犹豫豫的把手塞到了他的手心里。无心的手挺温暖,比她的巴掌大了一圈。她有时候觉得无心是自己的同龄人,有时候又觉得无心是自己的叔叔辈。湿漉漉的握住了无心的手,她有了一点安全感。

两人回了空教室,苏桃坐在地上,问无心:``你家是什么成分呀?''无心紧挨着她坐了,轻声答道:``无产阶级,祖上是要饭的。''苏桃听了``祖上''两个字,凭空生出了一种陌生而又熟悉的感觉,文绉绉的,不合时宜。很羡慕的低下了头,她小声说道:``你出身真好。''

无心听了她的回答,忍不住嗤嗤的笑。苏桃的话没毛病,就因为没毛病,才让他发笑——在此朝代之前,怕是从来没有人发过苏桃的感慨。苏桃惊异的看了他一眼:``你笑什么?''无心没有正面回答,转而问道:``你不是文县人吧?''苏桃摇了摇头,慢吞吞的讲起了自己的来历。

她是没有故乡的人,一直随着母亲南北辗转。母亲和父亲是个若即若离的状态,不在一起,但也不远离,因为离得太远,母亲就享受不到父亲的特权了。父亲在南方,她们也在南方;父亲北上了,她们也跟着北上。

无心忽然发现了一个关键点:``在文县,没有人见过你,对不对?''苏桃``嗯''了一声:``我们夜里来的,直接就躲进了小楼里。''无心又问:``你身上有什么证件吗?''

苏桃打开自己的书包,书包里装着一套换洗衣裳,一本红宝书,一点女孩子离不得的零碎东西,还有一本户口簿。户口簿子里面还夹着一沓钞票。把户口簿打开了,他们借着窗外的月光一起看。户口簿上写着苏桃的学名,是苏平平三个字。

``家里人都叫我桃桃。''她告诉无心:``后来上了小学,妈妈说苏桃听着不正式,就改了苏平平。''无心拍了拍她的小脑袋:``桃桃。''苏桃笑了:``嗯。''

无心紧接着又说:``我们得找个地方,把你的户口本藏起来。从今往后,你就是我的同学。你的学生证和介绍信在路上丢了,现在什么都没有。记住了吗?''然后他望着苏桃的眼睛,正色说道:``还有一个问题——小楼里有没有留下你的照片?''苏桃连忙摇头:``我们都没有照片了。照片早在家里就被爸爸烧光了。''

无心和苏桃嘁嘁喳喳的商量了小半夜,末了偎在一起睡到了天亮。太阳一出,光芒万丈,苏桃就不害怕了。两人到了水房洗脸漱口,无心先洗完了,站在水房门口说道:``桃桃,早上吃剩烧饼吧,吃完了烧饼我们出去看看风声。要是没事的话,我们就想法子走。''

苏桃用一把塑料梳子蘸了水,正在歪着脑袋用力梳头发。无心理直气壮的喊她``桃桃''。她听在耳中,心里暖融融的,好像又有家了。把两条辫子利利索索的编好了,她腼腆的出了声:``无心同志,你把饭盒给我,我接点水喝。''

无心把饭盒递给了她:``叫我无心就行。反正你我也差不几岁。我可能是看着老相,其实年轻着呢。我刚上高三——''话没说完,他忽然感觉动静不对。斜着眼睛向下一瞧,他发现白琉璃不知何时从书包缝隙里伸出了脑袋。一个雪白的圆头圆脑上,两个黑豆眼睛正在若有所思的望着他。

无心正在装嫩,冷不防的和白琉璃对视了,登时恼羞成怒。而苏桃端着一饭盒凉水转过了身,正好面对了无心:``呀,你书包里的东西是什么呀?''无心攥着白琉璃的脑袋向外一抽,抽出了一条半米多长小白蛇:``它是我的宠物,养着玩的。你怕不怕?''

苏桃双手托着饭盒,对着白蛇左看右看:``不咬人啊?''无心握着白蛇中段:``不咬人,也没毒,还通人性呢。''说着他向左一指:``白琉璃,转!''蛇脑袋立刻转向了右方。无心连忙改往右指,可未等他开口,白琉璃把脑袋又摆向了左方。

无心对着苏桃笑道:``看见没有。我让他往东,他不敢不往西。''苏桃也笑了:``哦\ldots{}\ldots{}我还以为是它不听话呢。我原来只在图画书上见过蛇。书上的蛇都可吓人了,不像你的蛇好看。''

白琉璃听苏桃夸奖自己貌美,不禁满意的一吐信子。苏桃生得两弯秀眉,一双明眸,白白净净苗苗条条。他认为苏桃也挺美,有心凑上前和她亲近亲近;然而因为附在了蛇身上,不大擅长指挥白蛇的细长身体。所以在无心的手里扭了扭,他没有前进的本领,也就作罢了。

无心把白琉璃缠成一团塞回书包,然后带着苏桃回教室吃剩烧饼。两人干干净净的晒着朝阳,倒是舒服了,与此同时,在县城的另一端,联指所在的小学校里,却是一派紧张气氛——昨天夜里他们忽然收到保定急电,说是三号提前动身,今日上午就能乘汽车抵达文县了!

陈部长一夜未眠,脸更黑了。他的得力干将、十四岁的初一学生赵萌萌正处在鼻青脸肿的高潮时期,看着也不甚像人。指挥部里最体面的人物是顾基,顾基个子最高,肩膀最宽,浓眉大眼的很周正,不过走不到人前去,因为父亲虽然是工人阶级,爷爷却做过小军阀,在天津过了几十年纸醉金迷的腐朽生活,解放后还逃去了香港。如果不是和陈部长做了十年的同桌,顾基不但没有资格出入指挥部,而且早就被一并打成狗崽子了。

顾基有一块老罗马表,是爷爷传给父亲的,上个礼拜被他送给了陈部长。陈部长撸起袖子看了看时间,又回头望了望,见指挥部的核心人员都到齐了,而且精神很饱满。赵萌萌捂着红肿开裂的嘴角,低声问道:``部长,不用多找些人夹道欢迎吗?光是咱们几个,人太少了吧?''陈部长轻声答道:``三号的意思,不让我们声张。''赵萌萌咂了咂嘴:``太静了,显不出我们的热情啊!''

陈部长刚要回答,远方路上忽然出现了大卡车的影子。小学校所在的一片地区,是县联指的地盘,绝对不会有红总的人马入侵。可陈部长认为三号没有坐卡车来的道理,而且卡车一辆接一辆,居然连着来了五辆。

五辆卡车全是满载,只是后斗上面苫了雨布,看不清楚满载的内容。一辆军用吉普车殿了后,在它距离指挥部大门还有几十米远时,陈部长率领手下蜂拥而上。及至吉普车停了,他们立刻热情洋溢的唤道:``小丁猫同志,我们盼星星盼月亮,终于把你盼来了!''

吉普车后排车门一开,一位细条条的白面书生弯腰下了车。众人见了,皆是一愣,万没想到省联指的第三号人物,居然是个娃娃脸的大男孩子。而外号小丁猫的前高三学生丁小猫站在车旁,一手扶了扶鼻梁上的银框眼镜,另一只手夹着半根香烟,搭在了大开的车门上。阳光照着他洁净的白衬衫,他风度很好的对着陈部长一点头:``我代表一号以及我个人,先向奋斗在文县第一线的革命战友们问好。''

他是孩子的脸,声音却成熟,两厢相加,反而有种意外的魅力。很随便的和陈部长握了握手,他继续说道:``文县是个大县,但是革命的温度并不算高。''陈部长很惶恐:``昨天我们也和红总打了一场硬仗\ldots{}\ldots{}他们死了好几个。''

小丁猫笑了一下:``革命是暴动,是一个阶级推翻另一个阶级的暴烈的行动。几条人命不算什么。对敌人的仁慈,就是对自己的残忍。敌人的性命不算什么,我们自己的性命,也不算什么。为有牺牲多壮志,敢教日月换新天。必要的时候,可以大杀!''

陈部长等人一起激动了,而小丁猫用手里的烟卷一指人后的顾基,微笑问道:``你傻看着我干什么?''顾基高人一头的站在后方,结结巴巴的红了脸:``我、我\ldots{}\ldots{}对你很、很崇拜。''小丁猫笑了,不再理他。抬手对着前方卡车一指,他轻描淡写的又道:``我给你们带了一点礼物,希望可以给你们的革命热情加一加温。''

前方卡车的司机跳下了驾驶室。踮脚蹦跳着掀起后斗雨布一角。没了雨布的遮掩,成捆的半自动步枪曝露在了光天化日下。

\chapter{小丁猫}

小丁猫一手插进裤兜里,一手夹着半根烟,慢悠悠的往指挥部大门走。陈部长虽然面黑似铁,且有一身不显山不露水的腱子肉,但是在白皙的三号勤务员面前,平白无故的就矮了一截,素日铁一般的刚硬气质也软化了。像个高级跟班似的垂下双手,他微微弯着点腰,在小丁猫的身边紧紧跟随,又主动介绍道:``指挥部里坐镇的同志倒是不多,大家最近主要是下到工厂机关里去,挖出隐藏在革命群众中的反革命坏分子。''

小丁猫点了点头:``很好,革命群众一声吼,能让地球抖三抖。''然后他用手中的香烟向前一指:``指挥部有点不像样。''陈部长陪笑答道:``原来是钢厂子弟小学,地方是不宽敞。''小丁猫深吸了一口烟,然后扭头呼了出去,言简意赅的说道:``应该换一换。唯物主义者,物质决定意识。小门小户的指挥所,产生不出高瞻远瞩的决策。''

陈部长连忙答应。此时从保定随行而来的两名女将下了吉普车,也大踏步的赶了上来。其中一位五短身材的女将处在花样年华,生得头如麦斗,眼似钢铃,地位却高,乃是省联指十常委之一,本来名叫杜文思,去年八月改名杜敢闯;另一位女将是细条条的身材,细条条的面庞,穿一身黄绿色旧军装,形象类似腌黄瓜,名叫马秀红,是小丁猫的机要秘书。

杜敢闯和马秀红对小丁猫是忠心耿耿,而小丁猫终日面对着如此两位战友,不由得活成一朵傲雪寒梅,革命意志极其坚定,生活作风极其清白,乱七八糟的心思从来没有。眼珠斜向身边两位异性战友,小丁猫暗暗的一咬口中烟卷,顺势瞟向了陈部长旁边的李萌萌,他又是一皱眉头。

穿过校园进了指挥部的大办公室内,小丁猫直奔正题,让陈部长拿出文县地图,在联指地盘上做出标记。陈部长手握红蓝铅笔,在地图上大刀阔斧的画了几个大红圈:``小丁猫同志,钢厂、重一中、以及机械厂的东半部分,都被我们占领了。''

小丁猫把烟头向后交给马秀红:``县委大院被红总占了?''陈部长做汗颜状,挠着头羞涩的苦笑。小丁猫摇了摇头:``斗争总是有反复性的,没有关系。捣乱,失败,再捣乱,再失败,直至灭亡——这是红总的逻辑。斗争,失败,再斗争,再失败,直至胜利——这是我们的逻辑。伟大领袖毛主席的语录,应该成为你们斗争的指导思想。''

然后他扭头对着顾基一点头:``怎么又看我?''顾基软绵绵的微笑:``你说话太、太有水平了。''小丁猫伸手一指他:``你是什么出身?''顾基登时心虚了:``工、工人。''陈部长横了他一眼,见他居然敢越过自己,公然的对三号大拍马屁,真是忘了他爷爷干过的好事!

小丁猫不再理他,对着地图审视良久,末了问道:``重一中的条件怎么样?''陈部长不假思索的答道:``一中是大楼,三层,挺好的。''小丁猫抬头看他:``为什么不把指挥部放到一中?''陈部长立刻迟疑了:``一中\ldots{}\ldots{}我听说啊,我听别人说的,说是一中闹鬼。''

小丁猫向他探过了头:``闹鬼?''陈部长下意识的又要挠头:``他们说\ldots{}\ldots{}一中夜里,有人上课。''小丁猫歪了脑袋:``上课?''陈部长感觉自己有散布封建迷信之嫌,十分出汗:``不是真上课。是有人晚上进了一中楼里,可能是有幻觉吧,看见死了的老师,给学生上课——去年一中有几个老师,死在批斗大会上了。''

小丁猫不以为然的摇了摇头:``彻底的唯物主义者是无所畏惧的。我马上过去看看情况,如果一中能用,指挥部就立刻搬家。谁有自行车?吉普车就不开了,兴师动众也不大好。''他拨开人群望向顾基:``你有吗?''顾基深感荣幸,脸都红了:``有!''

小丁猫对着身边的杜敢闯和马秀红说道:``你们留下来,让小陈帮助你们迅速掌握文县的斗争情况。我自己出去逛逛。''马秀红十分关爱他:``要不要带几个人跟着?''小丁猫摆了摆手:``不必。我不往红总的地盘走,红总也根本不知道我来了文县。''

顾基因为有个混蛋的爷爷,自从懂事起,精神上就一直很有压力,总像是低人一等。如今小丁猫几次三番的主动和他说话,他受宠若惊,几乎要感激涕零。把自行车推到指挥部外,他很周到的询问小丁猫:``你怎么坐?骑着坐还是侧着坐?''小丁猫一挥手:``你骑你的,我跳上去。''

顾基抬腿骑上了车,因为还是紧张,所以把车骑得摇摇晃晃。小丁猫跟在后边跑了几步,抓住时机侧身向前一跳,屁股压得自行车一歪。顾基光顾着保持平衡,忘了留意方向。只听``咯噔''一声,他正轧上了横在路上的一块扁条石。东倒西歪的一抖车把,他奋力的一踩脚蹬,在向前猛蹿的同时,发现自己终于把车骑上正轨,轻巧多了。

一路迎风疾驰,他在二十分钟之内抵达了荒凉的一中。一捏车闸脚踩了地,他回头正要说话,可是眼角余光一扫,他忽然愣住了。小丁猫没有了!

他立刻下了自行车,前后左右的乱看,就见来路上远远的出现一大队自行车,其中领头的人是陈部长,陈部长骑着一辆半新不旧的飞鸽自行车,车后面侧身坐着的,正是小丁猫。

及至大队人马到了一中门口,顾基推着自行车张口结舌,陈部长则是开口便骂:``顾基你傻×啊!你知不知道你刚上路就把小丁猫同志给颠下去了?我们一大帮人在后面撵你都撵不上,你撅个屁股骑得还挺快!''

顾基都快吓哭了。小丁猫也没理他,跳下车径自迈步往校园里走。陈部长等人把自行车锁好了,一路小跑跟上。大上午的,没人害怕;小丁猫进了校门没走几步,忽然停下问道:``前边的教室里,是不是有人?''

陈部长举目远眺,隔着玻璃窗,影影绰绰的看见了空教室内的无心和苏桃。``好像就俩人,一男一女。''陈部长沉吟着回答:``是不是搞对象的?''小丁猫抬手一指脚下:``一中距离红总的势力范围很近,我们对待一切可疑人物,都不能轻心。大家分散包抄,抓住他们问一问!''

陈部长常年和人斗殴,很有作战经验。此刻对着身后的同学死党们一下令,众人立刻就分散开了。小丁猫回头对着顾基一招手:``别哭,赦你无罪,跟我走吧。''

陈部长等人堵住教室门口时,无心和苏桃正坐在课桌上翻花绳。清晨他们鬼鬼祟祟的出去了一趟,发现街上不时有红卫兵小队经过,空气中的硝烟味道还很浓,容不得他们大摇大摆的走;于是只买了一点吃喝回来,想要再等时机。

两人吃饱喝足了,无所事事。苏桃从书包里翻出小拇指粗的一卷红毛线,抻开了正好可以用来翻花绳。翻花绳当然是小女孩的游戏,不过无心也很愿意陪着她玩。昨天看苏桃垂着两条毛刺刺的辫子,他把对方当成了黄毛丫头看待;今天苏桃打扮整齐了,原来是一头黑亮亮厚实实的好头发,衬托着粉白的脸儿,美得不像寻常人家的姑娘。

俗话说``人靠衣裳马靠鞍''。本来人是靠着衣裳添彩的,但是一套没型没款的半旧军装穿在苏桃身上,因为和她的面孔太不配套,对比之下,反倒让她有了点落难公主的意思。

无心很喜欢她,不是垂涎她的肉体,也不是赞叹她的内涵,只是单纯的喜欢。像小男孩喜欢小女孩,像大哥哥喜欢小妹妹。所以在陈部长等人骤然出现在门口时,他跳下课桌,一下子就把苏桃扯到身后去了。

陈部长和他打了个照面,也是一怔,随即抬手指着他叫道:``好啊,又是你!''小丁猫四两拨千斤的拨开了陈部长,慢悠悠的进了教室。望着无心一眨眼睛,一挑眉毛,他半晌没说话,最后开了口:``躲在后面的,站出来!''

苏桃吓得腿都硬了,很艰难的横着挪了一步,她垂着头哆嗦成了一团,白皙的手指上还缠着红毛线。陈部长一双眼睛盯着苏桃,一张嘴抢着汇报道:``昨天我们在街上就见过他!他身份不明,很有可能是红总的人!''

无心也横挪一步,把苏桃又挡了住:``你查过我的身份,知道我不是。''小丁猫独自走到了他的面前:``你有什么身份证明?拿出来给我看看。''

无心转身从课桌上拿过自己的书包,打开了伸手往里面掏。书包里东西不少,以蜡纸包着的圆面包为主,还有几根香肠。掏出介绍信递给小丁猫,他规规矩矩的解释道:``我们两个是一起出来的,在火车上遭了贼,我们两个的学生证都没了,她的介绍信也丢了。昨天你们的人硬说我们是什么红总的,吓得我们不敢上街。''

小丁猫把介绍信看了一遍,然后双手捏住,``嚓''的一声撕成两半,揉成一团。将纸团丢在地上,他对着无心一伸手:``身份证明,给我看看。''无心和他对视一眼,然后垂下眼帘答道:``没证明了。''

小丁猫回头对陈部长说道:``革命是真刀真枪的干,不是隔着几张课桌动口不动手。''然后他一把抓住了苏桃露出的一只手。望着手指上的红毛线,他笑了一下:``小资产阶级的小情小调。''

苏桃拼了命的把手向后一抽,另一只手暗暗揪住了无心的后衣襟,同时低着头,坚决不看小丁猫。小丁猫也不勉强。转身走向门口,他对着陈部长下了命令:``他们两个身份不明,还在光天化日之下搞流氓活动。带走!''

陈部长一方人多势众,把无心和苏桃一起押走。而小丁猫带着顾基在楼上楼下巡视一番,又到楼顶天台向下俯视了整座校园。最后他问顾基:``我听小陈说,你一直是他的同桌?''顾基立刻点头:``是,从小学开始就是了。''小丁猫又问:``你是文县人吗?''顾基继续点头:``是。''

小丁猫居高临下的转向前方:``我也算是文县人吧,我生在猪嘴镇。十来岁了才迁去保定。猪嘴镇你去过没有?''顾基诚惶诚恐的答道:``原来我们总去,猪嘴镇不是挨着猪头山矿区吗?我们经常上山里玩。''小丁猫摇了摇头:``猪头山,没什么好玩的。''

\chapter{突发事件}

无心和苏桃走了老远的路,低着头从一中慢慢的往指挥部蹭。陈部长一手推着自行车,一手握着一根半长的树枝,拧着眉瞪着眼跟在后方,口中吆五喝六:``你们倒是快走哇!怎么着,还打算赖在半路不动了?''

嘴里一边说,他一边用树枝去戳前方两名俘虏。对着无心,他是混戳;对着苏桃,他的下手点就比较有讲究,专往后腰和屁股上使劲。苏桃刚刚过了哆嗦的劲儿,此刻明知道对方不是好戳,但也不敢出声,只能是背过一只手,尽量挡着屁股。陈部长看她手掌白里透红,忍不住又用树枝一杵她的手心:``挡什么挡?''

话音落下,他忽觉手中一滑,随即就发现自己的树枝已经被无心抽了出去。``咔嚓''一声把树枝掰成两截扔在地上,无心头也不回的说道:``要文斗,不要武斗。''然后他回头看了陈部长一眼:``想武斗,我也奉陪。''

陈部长看他眼神很凶恶,斗争意志不禁动摇了一下。有心踹他一脚,可是双手推着自行车,行动不是很自如。目光从无心的后背移到苏桃的屁股,苏桃穿着面口袋似的军裤,看着也没什么屁股。沿着屁股再往上瞧,陈部长盯着苏桃的后脖颈出了神,两只脚一步不停的走,同时在心里把她和田小蕊李萌萌等人做了对比。

在春日温暖的阳光下,陈部长想要是苏桃能跟自己好,自己就不和李萌萌狗扯羊皮了。苏桃要是不和自己好,自己也许可以对田小蕊再卖把子力气,但田小蕊又有点儿喜欢顾基。田小蕊要是真喜欢顾基,自己就不好出手了,毕竟从小和顾基玩大的,兄弟情分不能不讲。可顾基是个徒有其表的怂货,拿顾基当兄弟,是不是拖了自己的后腿呢?

陈部长塞了满脑子乱哄哄的爱恨情仇,都不知道自己是怎么走回指挥部的。

和清晨相比,指挥部的人气旺多了。一排红砖房坐落在小校园里,靠左的两间是宣传队的办公室。两间办公室全开着门窗,里面以女性为主。十七岁的田小蕊甩着齐耳短发,正在其中的一间里和同伴排练样板戏;隔壁屋子里人更多,却也更安静,因为全都低头站在大办公桌前,刷刷点点的写大字报。

写好了的大字报被挂在窗上墙上晾干,铺天盖地到处都是,五颜六色宛如万国旗。一只小白蝶扇着翅膀,掠过了树木碧绿的新叶和陈部长黝黑的面孔。陈部长的心情忽然极度舒畅了。

弯腰锁了自行车,他让人把无心和苏桃暂且关进右边的空屋,自己则是投身到了妇女工作中去。一个箭步跳上窗台,他笑嘻嘻的问屋里的田小蕊:``排练着哪?''田小蕊冷淡的对他一点头,然后做出李铁梅的姿态,咬牙切齿的锐声唱道:``我家的表嗷嗷嗷叔,数呜呜呜不清\ldots{}\ldots{}''

无心和苏桃进了空屋子。房门一锁,他们算是入了狱。苏桃靠墙站了,一只手还牵着无心的后衣襟;无心看她满身都是不打自招的嫌疑相,就扯开她的手,面对着她低声安慰道:``别怕,只要你我把话咬准了,他们也没证据断我们的案。''

苏桃小声说道:``我害怕。''无心俯身凑到她的耳边,嘁嘁喳喳的说道:``反正我们今早把该藏的都藏好了,他们就算搜我们的身,也搜不出什么来。你坦然点,得让他们看不出我们的底细。''

此言一出,白琉璃先听明白了,立刻从书包中伸出了头,摇摇摆摆的要往外爬——他挺喜欢自己的白蛇身体,万一无心过会儿被人揍了,他不心疼无心,只怕自己受到连累,会被无心压扁,或者被人剥了皮清蒸红烧。为了保证自己能够长久的做一条貌美白蛇,他决定钻到墙缝里避避风头。

无心常年和他气急败坏的作斗争,已经和他亲近到了心有灵犀的地步。把他抻出来扔到地上慢慢爬,他转向苏桃,用轻快的语气问道:``你的红绳呢?我们接着玩。玩着玩着你就不怕了。''

苏桃从口袋里摸出一小团红毛线,红毛线结成了一个大疙瘩,解也解不开,于是换了游戏——两人双掌合十,互相指尖抵着指尖,看谁动作最快,能够率先拍到对方的手背。

无心手快,所以故意控制着速度,想让苏桃也赢几次。苏桃很认真的骤然出击,双手``啪''的夹住了无心的双手。微微笑着抬头面对了无心,她小声说道:``你有时候像大人,有时候像小孩。''无心问她:``像小孩好不好?''苏桃点了点头:``挺好的。''

无心又问:``怎么个好?''苏桃慢慢松开了他的手:``能跟我一起玩呗。''无心笑了:``我也愿意和你一起玩。等到度过了眼下的难关,我带你多走几个地方。''苏桃抬眼看他:``你家人不管你呀?''无心摇头:``不管。''

苏桃不大好意思的一抿嘴,声音越来越低了:``我也没人管。''无心向她一扬下巴:``我比你大,我管你吧!''苏桃垂下了头,看无心斜挎着的书包上支出了一截帆布带子,就伸手拽住了缓缓揉搓:``行。''

两人正是窃窃私语之时,外面起了喧哗,原来是顾基骑着自行车,把小丁猫带回来了。紧接着房门一开,有人搬进了一张长课桌,又对着无心和苏桃吆喝道:``站好了,等着接受审讯!''

等到三把椅子也摆好了,小丁猫、陈部长以及李萌萌一起进了来。小丁猫当仁不让的坐了中间,陈部长和李萌萌分坐两边。李萌萌打开本子拧开钢笔,一只眼睛肿的看不见人了,她歪着脑袋,用另一只眼睛斜盯无心。而小丁猫一团和气的对着陈部长一点头:``小陈,你来问吧。我先听一听。''

陈部长答应了,随即正色面对前方,厉声吼道:``姓名年龄籍贯出身自己报!''无心开了口:``姓名无心,年龄\ldots{}\ldots{}二十,籍贯黑龙江,出身\ldots{}\ldots{}佃农。''此言一出,旁人没言语,小丁猫盯着自己撂在桌面上的两只手,``扑哧''一声乐了。笑完之后他对着苏桃一点头:``你继续说。''

苏桃胆战心惊的喃喃说道:``姓名苏桃,年龄十五,籍贯黑龙江,出身\ldots{}\ldots{}工人。''小丁猫问道:``什么工人?产业工人还是手工业工人?''苏桃被他问愣了,不知道其中有什么区别。无心替她答道:``产业工人。''

无心一出声,小丁猫就无声的笑,并且不看他。陈部长斜着眼睛窥视本组织的第三号领袖,心里直发毛:``小丁猫同志,他们的回答,是不是有问题啊?''小丁猫摆了摆手:``没什么,你问你的。''话音落下,他看了无心一眼,``扑哧''一声又乐了。

陈部长莫名其妙的清了清喉咙,开始老调重弹。他们无凭无据,当然没有让人信任的理由。陈部长做出威胁,要派人去黑龙江了解情况。见无心和苏桃一脸的麻木不仁,他转而又究起了细节,问苏桃的父母在哪家工厂,做什么工作,一个月工资多少,住什么房子,有几个兄弟姐妹。

正是问得苏桃前言不搭后语之时,身后的房门忽然被人撞开了,顾基伸进了一个汗津津的脑袋,半兴奋半惊骇的说道:``报告,红总出现新动向了!''小丁猫回头看他:``怎么了?''顾基本来是看着陈部长的,小丁猫一出声,他就把陈部长抛弃了:``红总把县委的大印给丢了!''

小丁猫睁圆了眼睛:``公章丢了?''顾基乐呵呵的点头:``丢了,今天刚闹出来的!公章,还有一沓子空白介绍信,好像还有上百斤的全国粮票,都丢了。怎么丢的我们不知道,反正红总现在把矛头指向了我们,说是我们派人偷的。''

如果把文县比作一国,公章就相当于玉玺。县委的原领导们早都被批倒批臭了,代表县委权力的物件,就只有公章一样。听闻红总丢了公章,小丁猫把桌子一拍,对着陈部长笑道:``好啊!战斗的机会来啦!如今摆在面前的就是两件事,第一,对红总迎战;第二,发动全部人马找公章!''

未等陈部长回答,又一名青年气喘吁吁的挤到了门口:``报告!机械厂里干起来了!红总先动的手,他们全带了砍刀!''陈部长立刻显出了不屑一顾的神气,而小丁猫命令道:``钢厂不是你们的吗?集合厂里武装部的全体人员,火速过来领枪!''

陈部长站起了身:``我这就去——他俩怎么办?还审吗?''小丁猫也跟着起立了:``先关着吧,有空再来处理他们。''
一群人说走就走。门上大锁一扣,无心和苏桃就又失了自由。

单手伸进书包里,无心对苏桃悄声说道:``他们要的公章,好像在我手里。''苏桃睁大了眼睛:``你怎么会有的?我们要是把公章给了他们,算不算立功赎罪啊?''无心在唇边竖起一根手指,``嘘''了一声:``先不给,让他们急一急,知道公章的分量。''苏桃环顾四周:``你找个地方把公章藏好吧。我总怕他们搜身。''

无心满屋转了一圈,没找到好地方,灵机一动,他把苏桃的书包要了过来。苏桃的书包堪称包罗万象,他把白纸包好的公章塞进了一卷尼龙袜子里,袜子上面又缠了两条月经带。苏桃有点难为情的蹲在一边旁观,心中感觉无心无所不能。往后要是能被他``管'',自己倒是很愿意的。

因为机械厂爆发了战争,所以指挥部乱哄哄了一阵之后,大部分人马都冲去了前线,只剩下几个能力差的看家,其中就有顾基一个。小丁猫本来还想让顾基跟着,可陈部长太了解他,不肯让他随行,并且告诉小丁猫:``别带他,他可笨了,个头还大,靶子似的。和他在一起,特别招打。''

顾基没办法,眼巴巴的看着人都走了。忽然想起空屋子里还关着个挺漂亮的小姑娘,他来了精神,趴在玻璃窗上想要往里看。然而他往里看的时候,无心也正打算往外看。两人隔着一层玻璃脸对了脸,都是一怔。顾基随即歪了脑袋换位置,不料无心一巴掌拍上玻璃窗,截断了他的视线。

接下来,苏桃站在房内,就看无心双掌翻飞,噼里啪啦的在窗玻璃上乱拍一气,掌掌都不落空,把外面顾基乱动的脑袋遮了个严密。顾基气坏了,隔着玻璃窗向他一指,高声骂道:``你妈×!''无心当即作出回应:``你还吃了我一个烧饼呢!''

顾基又骂:``你个反革命流氓分子!''无心岿然不动:``反正你吃了我一个烧饼!''顾基咣咣咣敲玻璃窗:``你是不是欠揍?''无心立刻敲了回去:``吃了我的烧饼还想打我?''顾基也不是太馋的人,偶尔吃了他一个烧饼,被他嚷得天下皆知,不禁急红了脸:``没完啦?''

苏桃见无心占了上风,又怕顾基真冲进来打人,就上前扯了扯无心的后衣襟:``不说了,我们不和他吵。''无心本来也没生气,苏桃一扯他,他就当真转身撤了。而顾基因为在陈部长面前总受欺负,所以此刻颇想趁机也欺负欺负无心和苏桃。叽叽咕咕的又骂几句,他见房内总没回应,才意犹未尽的走了。

\chapter{立了一功}

大下午的,无心和苏桃被关在空屋子里,也没人管理。门是锁严实了,窗户合页可能是锈住了,也推不动。苏桃畏畏缩缩的在角落里席地而坐,悄悄的解开了脚上鞋带。发现无心站在一旁望向自己了,她难为情的小声说道:``一直站着,脚都肿了。''

无心一屁股也坐下了,又把两人的书包叠在一起放在地上:``把鞋脱了,两只脚架到书包上。反正也没人来,能歇一会儿是一会儿。''

苏桃似乎是感觉自己的两只脚拿不出手,很心虚的不肯伸腿。无心看她始终是抱着膝盖,就亲自出手抓了她的脚踝,不由分说的抻直了她的双腿,并且扒掉了她脚上的解放鞋。苏桃穿着一双白底碎花的尼龙袜子,热烘烘的散发了解放鞋的胶皮味。

无心看苏桃坐舒服了,自己也跟着脱了鞋伸了腿。苏桃顺着他的腿往下看,就觉得他腿长,笔直笔直的伸出老远。忽然``哟''了一声,她问无心:``你怎么没穿袜子啊?''

无心一摊手,对着她笑道:``没袜子可穿。''

苏桃跪起身拽过自己的书包,打开了在里面翻找。她带了好几双新袜子,都是有弹性的,脚大脚小都能穿。翻出一双尺码最大的,她抬头递给无心:``你试一试,看看能不能对付着穿?''

无心摆了摆手:``不用,我穿不穿都行。''

苏桃踩在解放鞋上,蹲到无心脚边抻长袜子,想要比量比量。袜子毕竟是女式的,大得有限。苏桃把袜子抻了个细长,还是比无心的赤脚短了半截脚趾头。悻悻的卷起袜子塞回书包,她坐回原位,本来以为自己能为无心做点什么,结果还是未遂。

指挥部内的人越来越少,都跑去机械厂看热闹了,只有顾基身负重任,原地不动充当看守。晚饭前他跑到空屋窗外向内望了一眼,发现无心和苏桃靠在墙上,两人歪着脑袋偎在一起,居然是睡着了。

指挥部里空空荡荡,连个和他斗嘴的人都没了。他百无聊赖的抱着肩膀,想一想自己的家庭,想一想自己的前途,越想越是茫然。能够在指挥部里占据一席之地,乃是他的荣耀;其实他是没资格加入县联指的,全是陈部长提拔保护了他。陈部长能把他吸收进来,也能把他驱逐出去。他顶天立地的晃着大个子,感觉自己像只孤独的小鸟,无枝可依。

正在他伤感之时,小丁猫等人回来了。

小丁猫不高不矮的直鼻梁上,端端正正的架着银丝眼镜,镜片一尘不染。白衬衫的第一个领扣没有系,翻出的衬衫领子也是雪白。嘴里叼着一根香烟,他从吉普车上弯腰跳下。忽见顾基孤零零的站在指挥部大门口,他淡淡的一笑,开口问道:``关着的那二位怎么样?''

顾基一看他的精神头,就知道必是大胜而归,连忙也跟着昂首挺胸:``他俩挺好,在屋里晾脚丫子呢!''

陈部长随后也下车了,一张黑脸像黑铁蛋子似的,黑里透着光。遥遥的对着顾基一招手,他高声大气的笑道:``顾基,我操!今天了不得,阵势太大了。
红总让咱们打得撒丫子跑,咱们就是钟山风雨起苍黄,他妈百万雄师过大江啊!''

顾基很了解陈部长的水平,如今听他效仿小丁猫引经据典,心中暗暗的不以为然,并且转移话题道:``田小蕊她们早走了,你们看见她们了吗?''

陈部长气吞山河的哈哈大笑:``看见啦,她们晚上到钢厂大礼堂演节目。''

顾基本来是有点崇拜陈部长的,此刻在小丁猫云淡风轻的衬托下,他忽然发现陈部长没个人样,简直有点不堪入目。脑筋灵活的转了一圈,他转向小丁猫,嘻嘻一笑。

杜敢闯和马秀红也下车了,和陈部长一起簇拥着小丁猫往院内走。及至进了办公室,小丁猫坐在办公桌后,慢条斯理的从裤兜里掏出烟盒,给自己续了一根香烟。而陈部长由于太兴奋,所以还忍不住对着顾基大说大笑:``武卫国平时不是一贯的自成一派吗?今天也被咱们小丁猫同志给震住了!他拳头再硬,也没有子弹厉害不是?''

武卫国是钢厂造反派的头领,名义上是归了县联指,然而因为看不上县联指的学生班子,所以实际上是自立山头,往日并不把陈部长等人放在眼里。陈部长今天终于看到了武卫国服软,不禁痛快淋漓,恨不能在办公室内作狮子吼。而小丁猫慢慢吸了半根烟,然后对着陈部长挥了挥手:``你和顾基去准备一下晚饭。我和她们再对今天的战斗做一次总结。''

陈部长满口答应,带着顾基告辞而走。杜敢闯走过去关了房门,马秀红则是拎起暖壶倒了一杯热水,静静的送到了小丁猫手边。小丁猫察觉出身边多了一根柔情似水的黄瓜,但是硬着头皮不抬头。

杜敢闯严厉的看了马秀红一眼,随即搬了椅子坐在小丁猫对面,压低声音说道:``我认为,我们今天开了一个很好的头。''

小丁猫向前撩了她一眼,看她身板横宽,是条威武的女好汉。牙疼似的吸了一口气,他往地上弹了弹烟灰,不知道自己是怎么搞的,革命的世界天大地大,闲花野草处处生,自己却是弄来了这么一对左膀右臂。他自认不是浮浅之徒,对于二位女将的内涵,他也是很欣赏的;不过话说回来,她们长得还是太困难了,属于不可改造的对象。天天对着杜敢闯和马秀红,他时常感觉自己特别美丽,当男人都有点浪费。

杜敢闯看惯了小丁猫若有所思的模样,于是自顾自的侃侃而谈。最后她在半空中一挥拳头,阴谋家似的小声说道:``你的心思,我全明白。如果你真看好文县了,我就立刻开始行动,把文县打造成我们的新根据地!''

小丁猫点了点头,轻声答道:``文县的全面斗争尚未展开,大有我们作为的空间。在保定,我永远都只是三号,与其如此,不如另开一片天地。''

他抽烟抽得口苦。把半截香烟摁熄在了写字台的玻璃板上,他端起杯子喝了一口温水:``文县原有的组织,经过了大半年的发展,都已经基本固定成形。想要打破它们的铜墙铁壁,就要吸收新的力量加入。武卫国的势力还是很强大的,我们先不要动他。饭要一口一口的吃,路要一步一步的走。我们联合可联合的,打击可打击的。至于红总那边\ldots{}\ldots{}''

隔着一张大办公桌,杜敢闯把一张面孔探向了他:``先对内,再对外。''

小丁猫正视了她,眼镜片上流光一闪,算是回答。杜敢闯的黑黄脸膛、脸上的油光、以及太阳穴和额头上暴出的红痘子,都让他很受刺激。于是他摘了眼镜,心中发出一声苍凉的叹息:``哎呀妈呀\ldots{}\ldots{}''

与此同时,无心和苏桃穿好了鞋,开始预备吃晚饭了。

无心从书包里掏出圆面包。撕开外面的包装纸,他把面包送到苏桃面前:``吃吧,看来他们是不能管我们的饭了。''

苏桃盘腿坐在地上,伸手拿了面包:``要是有个水壶就好了。没水喝,怪干的。''

无心一指窗外:``看见刚才他们的阵势了吗?他们肯定是占上风了。既然如此,我就干脆把公章交给他们。一枚公章,总能换口水喝吧?''

苏桃发现书包里还剩了一根香肠,就把香肠夹到面包里,递还给了无心:``我吃一个面包就够了。''

无心接了面包,要掰香肠:``一人一半。''

苏桃推他的手:``我不要,我吃不了。''

无心蹲在她的面前,看着她笑:``这么一点玩意儿,会吃不了?''

苏桃摇头:``我不饿。''

无心抬手摸了摸她的头发:``撒谎是吧?''

苏桃垂了头,咕哝着说:``我没撒谎,我饭量小。''

无心叹了一声,扭头往窗外望,忽然看到小丁猫溜达进了院子,他连忙把面包香肠往苏桃手里一塞,然后迅速从书包里翻出了被月经带和袜子包裹起来的公章。三下五除二的剥出公章,他抬手一敲窗户,吸引了小丁猫的目光。

小丁猫本来正在沉思,几乎忘了无心和苏桃的存在。如今冷不防的见了他,不由得一愣。天色暗淡,空屋子里又没开电灯,他影影绰绰的只见无心在向自己招手,就好奇的走了过去。无心见他越来越近,便把公章送到嘴边呵了一口热气,然后结结实实的印在了玻璃上。

正当此时,陈部长领着一群部下,带着晚饭回来了。
小丁猫来不及吃晚饭,先让人打开了空屋子的房门。电灯也亮了,无心向他伸出手,手掌托着一只红通通的木头印章。

``我是在路上捡的。''他坦然的告诉小丁猫:``捡的时候我也不知道它有没有用,反正当时周围没别人,我就把它揣起来了。我是过路的人,公章给你们也行,给红总也行,对我来讲没区别。现在我给了你们,就是要表示我们不是坏人,对你们更是没有恶意。''

小丁猫一把抓过公章,低下头仔细看清了章上的字样。要笑不笑的一翘嘴角,他随即抬头说道:``你这么做就对了,我给你记上一功!''

无心把苏桃拉到了自己的斜后方:``我不用你给我记功,只要你能放了我们,我们就心满意足了。''

小丁猫往无心身后望,能望到苏桃低着的半张脸;半张脸像半朵桃花,眉眼都是墨画的。

``你立了功,和我们就算是一条战壕里的战友了。''小丁猫微微一笑,两只眼睛分别盯着无心和苏桃:``现在红总的牛鬼蛇神大队还在蠢蠢欲动,伺机反扑。我不能让我的战友出去冒险,今晚你们就和我们一起去住招待所吧。''

\chapter{招待所百态}

县招待所距离指挥部所在的小学校,足有三条大街远,乘公共汽车的话,正好是六站地。小丁猫等人在指挥部里对付着吃了一顿晚饭,然后便张罗着要去县招待所休息。吉普车发动起来,载着小丁猫和他的左膀右臂先出发了,其余众人推着自行车纷纷出了校园,其中陈部长目光如炬的盯住了苏桃,气冲冲的吆喝道:``你站住!我告诉你,你的问题还没交代清楚,别想浑水摸鱼半路逃跑!''

苏桃吓了一跳,垂着头不敢言语。而陈部长不等旁人开口,搬起自行车向下一顿:``你上我车,我亲自带你走!''苏桃惊魂不定的看了无心一眼,见无心点了头,便蹭着小步走向了陈部长。

陈部长杀气腾腾的黑着脸,越是细看苏桃,越感觉她和自己不是一个世界的人——不但不是一个世界,甚至不是一个品种。人皆有爱美之心,他真想和她亲近亲近,可因为明知道自己亲近不上,所以他怒从心头起、恶向胆边生,恨不能直接强奸了她。

车子后座向下一沉,是苏桃侧身坐上去了。陈部长踏上脚蹬,正要用力往下踩,不料李萌萌从后面赶了上来,尖锥锥的问道:``你带她,谁带我呀?''陈部长不耐烦的答道:``让顾基带你!''李萌萌一跺脚:``他带那个男的先走啦!''

陈部长举目一望,就见顾基人高马大的蹬着自行车,果然已经驮着无心骑出老远,便是气得骂道:``这×真能愁死人!驮个男的也能跑那么快。''李萌萌扯了他的袖子,鼻青脸肿很不好惹:``那我呢?我怎么走?''

陈部长被她缠得没法,回头看看其余人等,每个人的自行车都不空闲;无可奈何之下,只好让她坐上了自行车的前大梁。老牛似的向前伸了头,他拼了命的一蹬一蹬,总算是把沉重的自行车给骑上了路。

文县是个富庶先进的工业县,县里的招待所是近些年新建的三层楼,堪称县内的豪华一景。招待所的所长早在去年秋天便被全县人民批臭批倒,罪名却是含糊,非招待所内部人员不能知晓。

小丁猫的吉普车开进招待所大院时,这位前所长正在院子里载歌载舞的进行劳动改造。眼看吉普车亮着大灯停在眼前了,前所长一手拄着大笤帚,一手翘着兰花指平伸向前;双脚脚尖一点地,他以一只芭蕾小天鹅的姿势亮了个相。

司机和坐在副驾驶座上的杜敢闯全愣了。小丁猫在后排出了声:``这是什么情况?''未等旁人作答,前所长纵身几个大跳,蹿着箭步没影了。马秀红后知后觉的做了猜测:``精神病吧?''

小丁猫下了吉普车,一手叉腰,一手向前一招。站在门口的女服务员,本来是谁也不理的,但是看他来势不凡,派头更不凡,两只脚就自动的移向了他:``你们是吃啊还是住啊?''小丁猫往大院深处一指:``刚才扫院子的那个人,是怎么回事?''

服务员上下打量了他,神情隐隐的带了热度:``精神病,别理他。''

正当此时,后方的自行车大军也到达了。小丁猫是有备而来,不但武器充足,资金也充足;眼看招待所灯火辉煌,是个体面的地方,他就起了豪兴,要请县联指的小兄弟们吃顿像样的晚饭。陈部长一听,立刻悔恨不迭,因为方才在指挥部吃白食,已经吃了个十分饱。

招待所一楼便是饭店,小丁猫包下大厅,让服务员摆了五桌宴席。陈部长几个电话打出去,武卫国带着演出完毕的田小蕊等人也赶来了。武卫国是条三十来岁的壮汉,经过了白天一场武斗之后,现在已经很高看小丁猫。

宴席刚一开始,他便端着一杯啤酒主动走到了小丁猫面前,敬酒过后又低声说道:``你就放心的住,招待所外面,我给你派了二十名保镖。''
小丁猫抬手一拍他的肩膀,没言语,只点了点头,是一切尽在不言中的模样。

武卫国一回原位,陈部长也立刻端着酒杯上去了,不但敬了小丁猫,也敬了杜敢闯和马秀红。顾基高高大大的跟在他的身后,意意思思的总想插话,可是又找不到机会。小丁猫和陈部长先碰了杯,然后目光越过陈部长的肩头,他对着顾基也一举杯。顾基吓了一跳,刚要回应,然而小丁猫和陈部长已经痛饮完毕,开始打嗝了。

两大杯啤酒进了肚,小丁猫站在桌前,开始有点摇晃。马秀红见状,立刻不着痕迹的起身搀扶了他。而小丁猫来了兴致,遥遥的对着无心一挥手:``立功的那个,带着你的小朋友一起过来,我也敬你们一杯。''

无心带着苏桃走到了小丁猫面前。手里拿着一杯金黄的啤酒,他对着小丁猫说道:``苏桃是个小孩,不会喝酒,我代她敬你了。其实我也谈不上立功,无非是帮了个顺手的忙。你我萍水相逢,谁也犯不上为难谁,是不是?''

小丁猫轻轻巧巧的和他一碰杯,眼睛盯着杯口流光笑道:``萍水相逢,即是缘分。有缘千里来相会,来日方长。我喝一杯,你喝两杯,没错吧?''

无心笑了笑,感觉小丁猫是话里有话,可惜没听懂。而小丁猫干杯之后,当众伸手揪住了苏桃的一边衣袖,把她一点一点的扯到了自己面前。苏桃顺着他的力道往前挪着碎步,同时偷偷握住了无心的手,手指冰凉的,几乎快要痉挛。

无心知道苏桃怕这些人,怕得要死。可是未等他出手,小丁猫已经松了手。笑眯眯的正视了苏桃,小丁猫带着一点醉意说道:``苏桃同志,我们都是来自五湖四海,为了一个共同的革命目标,走到一起来了。''

他的语气很温和,然而苏桃却是情不自禁的一哆嗦,耳边什么声音都来了——皮带抽过皮肤,木棒敲打骨头,母亲在批斗大会上发出的哀嚎,最后让父亲化为灰烬的爆炸\ldots{}\ldots{}
``苏桃避开了小丁猫的目光,慢慢避回到了无心的身后,同时听见无心对小丁猫说话:``小孩,不会说话,今天被你们关了一天,吓得一直没过劲。她真有个好歹的,我也负不了责,所以明天我就打算带她回哈尔滨了。''

小丁猫慢条斯理的摇了摇头:``走什么呀?我不发话,你能出文县吗?''无心笑了一下:``不走就不走,反正在哪里都是一样的干革命!''小丁猫一点头:``这么想就对了。''

晚宴结束之后,众人休息的休息,回家的回家。小丁猫是要住单间的,陈部长看出三号今天有点铺张的意思,就跃跃欲试的想占个便宜,主动要求和无心住双人间——三号总不会供不起他一张床位。

小丁猫立刻就答应了,又让陈部长再找个人对苏桃进行陪伴加看守。李萌萌一听可以免费住招待所,立刻就活了心思——她家里住着一小间黑洞洞的破房子,父亲是个酒鬼,母亲思想极其落后,见了她就让她干家务活,还把她的革命行为诬蔑为``出去骚'',气得她昨天当胸击了母亲一拳。她要不是无处栖身,早离家出走了。

陈部长认为李萌萌伤势未愈,没有必要留在招待所里又吃又睡的献丑。立场坚定的驱逐了李萌萌,他让田小蕊留下来。各人都有了着落,小丁猫便在杜敢闯和马秀红的陪伴下进了单间,掩人耳目的进行密谋。

苏桃也该和田小蕊回房间去了,眼巴巴的望着无心,她靠墙站着,一步都不想动:``我住三楼,你住哪儿啊?''无心在她肩膀上轻轻拍了一下:``我住二楼,离你不远。你今天累着也吓着了,我看你有点发烧,回房之后马上睡觉,别光顾着玩,知道了吗?''

说到``睡觉''二字之时,他在苏桃肩上捏了一把。苏桃立刻抬头看他,心里隐隐约约的明白了:``知道,我回去就睡。''然后她果然显出病怏怏的模样,随着田小蕊回房去了。

无心和陈部长也下了楼。顾基茫茫然的跟在他们身后,等到陈部长进房间了,他伸着脑袋向内一瞧,见里面窗明几净,床上床单雪白,还铺着弹簧垫子。颇为艳羡的挤进了门,他一扯陈部长:``明天要是还不走的话,换我来住一宿吧?''陈部长挥了挥手:``明天再说,你现在该走就走!''

顾基看陈部长气色不善,只好讪讪的转了身。及至他走出房门了,陈部长在后面又小声说道:``顾基,你明天早上早点儿来,招待所提供早饭,听说是随便吃,还挺好。''顾基站没站相,摇晃着``嗯''了一声,悻悻的走了。陈部长正要关门,不料眼角人影一闪,他定睛细瞧,却是发现无心从自己身边挤了出去。

``哎?''他立刻就要追:``你干什么去?''无心头也不回的答道:``撒尿!''站在公共卫生间的小便池上,无心痛快淋漓的尿了一场。很舒服的打了一个寒颤,他睁开眼,却是一惊。

在幽暗的电灯光中,他看到面前贴着白瓷片的墙壁上,缓缓浮现出了白琉璃的上半身。潮湿的长发中分披散,发梢似乎还带着隐隐的水意,白琉璃的形象停留在了人生最末一次的沐浴后,两道长眉下面,一双蓝眼睛透出肃杀的光。

``你怎么不管我了?''他恶狠狠的逼问无心。无心环顾四周,见卫生间里没别人,这才小声答道:``你又不怕我连累你了?''白琉璃一扬下巴:``我告诉你,我卡在墙缝里爬不出来了。你明天立刻回去救我。''无心压低声音说道:``我哪知道明天能不能回去?以后我给你抓条小白狗,你做狗吧!''

白琉璃一甩袖子,很狂躁的怒道:``不!总之你明天务必要去把我弄出来,否则我就去上苏桃的身!''无心连忙摆手:``别,我去就是。你脾气太大,全是我把你惯坏了。现在这里人多眼杂,我不和你一般计较。等到将来有机会了,我跟你算一笔总账。''

\chapter{新工作}

大清早的,苏桃早早的起了床。邻床上的田小蕊还在晾着大腿酣睡,十七岁的姑娘了,已经发育的有型有款。苏桃看了她一眼,看得心惊肉跳。田小蕊昨天晚上几次三番的要和她说话,句句都是敲打她的老底。她记住了无心的嘱咐,把嘴闭得死紧,硬着头皮扛住了田小蕊阴一句阳一句的审问。

轻手轻脚的洗漱完毕,她挎着书包出了门,迈着大步跑到了二楼。抱着书包站在阴暗拐角处,她静静的等待无心。等了半天,无心没来,小丁猫倒是带着马秀红施施然的下楼了。一眼瞧见苏桃,小丁猫停了脚步。夹着香烟的手指向她微微一抬:``干什么呢?走啊,下楼吃饭去!''

苏桃畏畏缩缩的退了一步,做蚊子哼:``我等无心呢。''小丁猫意态潇洒的笑道:``等他干什么?他也是一样要下楼吃饭的嘛!走走走,一起走!''苏桃莫名的很怕他,眼看他把手伸到面前了,连忙向后退了一步:``我不。我\ldots{}\ldots{}我还不饿呢。''

马秀红冷眼旁观,看小丁猫笑嘻嘻的像个流氓,有损三号的身份和风采,就面无表情的咳嗽一声。与此同时,二楼走廊中房门一开,无心和陈部长一前一后的走出来了。苏桃算是看见了救命星,先是横行躲开了面前的小丁猫,然后一路小跑,到了无心面前。

无心很自然的拉住了她的手,又对小丁猫打了个招呼。小丁猫用手中的香烟一指苏桃,一团和气的笑道:``无心啊,苏桃小同志还是缺乏革命小将的气魄。女同志要飒爽英姿五尺枪,要不爱红装爱武装;扭扭捏捏羞于见人是不行的。''

无心抬手一拍苏桃的后背:``小孩嘛,怕生。''小丁猫摇了摇头:``小?不小啦!''无心又摸了摸苏桃的后脑勺:``不小,也不大。黄毛丫头,什么也不懂,由她去吧!''

一群人说到此处,还算谈笑风生。众人继续下楼,到了一楼的大餐厅里。早餐除了各色主食粥汤之外,还有凉拌的小菜和茶叶蛋。顾基果然如约而至,不是跟着陈部长,就是尾随小丁猫。有了他打掩护,无心和苏桃不声不响的单独占据了一套桌椅。

苏桃和他紧挨着坐了,脸上终于露出了一点轻松颜色。一边磕开一只茶叶蛋,她一边小声告诉无心:``三楼有浴室,我昨晚洗了个热水澡。今天晚上我们要是还在这儿住,你也上去洗一洗。牌子上面写了,男的是五点到七点,女的是七点到九点。你早点去,能洗好久呢。''

把茶叶蛋剥干净了放到无心的碗里,她嘀嘀咕咕的又说:``我还想起一件事儿,你可能都忘了,就是小白蛇——我们把小白蛇落在空屋子里了。要是他们不管我们,我们想着去把它找回来吧!''无心用筷子尖扎起茶叶蛋,咬了一口:``他们不能白养着我们,我有办法带你回去。''

无心和苏桃吃饱喝足之后,跑到小丁猫面前毛遂自荐,说自己会写毛笔字,抄大字报是把好手。小丁猫大清早的还喝啤酒。端着酒杯对陈部长一点头,他说道:``他们两个,可以一用。''

小丁猫一发话,陈部长立刻就心领神会了。而小丁猫接着方才的话题继续说道:``今天开始,两件大事。第一,把指挥部迁往一中;第二,严密防范红总反扑。杜敢闯同志已经在凌晨出发回保定了,等她的人员一到,我们立刻动工,把一中改建为联指的第一堡垒。另外,机械学院是怎么回事?怎么成了真空地带?''

机械学院是机械厂的产物,本质上是一所无名的大学。照理来讲,大学校园里面应该最是风起云涌,然而机械学院里不知怎么搞的,各系学生各立山头,关起门来打了个乱七八糟。又因为他们战斗力有限,所以联指和红总都不屑与联合他们。

``阶级斗争一根弦,只能紧不能松。''小丁猫喝着啤酒,慢条斯理的吩咐陈部长:``你去组织人马,预备召开万人批斗大会。把全县死剩下的牛鬼蛇神做个集合,我再负责给你从北京弄回个大家伙。我告诉你,联指的声音,必须盖过红总。''

陈部长一边点头,一边不安的窥视小丁猫。小丁猫和他年龄相仿佛,可他总感觉小丁猫的灵魂至少得有四五十岁了。

无心和苏桃先人一步的出了招待所大楼,站在院子里看风景。精神病前所长双手各攥一条大抹布,正甩着水袖擦拭一楼的窗玻璃。擦了一阵之后,两条抹布全乌黑了,他把头一扬,踮着脚尖横向移动,又举起双臂把两条抹布甩成两朵花。最后姿态轻盈的转了个圈,他弯腰端起一盆脏水,一路扭扭搭搭的进楼去了。

苏桃看他疯得出奇,忍不住笑。正好有名偷懒的服务员站在门口嗑瓜子,无心和她攀谈了三言两语,却是得知了前所长的详细罪行。前所长姓鲍名光,基本可以算是个好人,生平唯一的爱好就是加夜班,并且热情洋溢,时常邀请值更的年轻电工到自己房里睡觉。

文化大革命一发动,鲍光立刻就被曝光了,罪名是同性恋,并且被人从办公室的抽屉里搜出许多男子裸体画片。鲍光的妻子儿女当天就和他划清了界限,而鲍光本人成为反革命流氓分子,挂着牌子走遍全县的大小批斗会,被造反派们打得死去活来,不出一个月的工夫,他就疯了。

鲍光一疯,反倒占了便宜,因为造反派们不能再押他批斗了。批斗大会是个严肃的场合,牛鬼蛇神们全都如丧考妣,唯有他站在一旁,像个鹌鹑似的双手交握于下腹,对着革命群众们乱抛媚眼。

及至牛鬼蛇神们全都九十度向下弯腰接受批斗了,他也夹着两条腿一撅屁股,屁股翘得比头还高。小将们刚一抡皮带,他便捏着嗓子做鸡叫,咕咕哒哒的像要下蛋,逗得牛鬼蛇神和革命群众们一起大笑。小将们没了辙,又不好平白无故的杀了他,只好把他送回招待所,让他在所里劳动改造。

无心刚刚听完了鲍光的故事,小丁猫等人就出来了。无心带着苏桃在前面走,苏桃低声问无心:``什么是同性恋呀?''无心想了一下,随即答道:``就是说这个鲍光啊,不爱女人,爱男人。''苏桃听了,似懂非懂:``这不算病吧?我也不爱和男生玩,玩不到一起去。''无心一挑眉毛,发现苏桃一开口,就把自己堵得不知从何说起了。

众人一窝蜂的回了指挥部。指挥部里人来人往,已经很热闹。宣传队用来写大字报的房间里已经人满为患,无心抓住机会,立刻将一张桌子搬进昨天关押过自己的空屋子里。等到苏桃把墨水瓶和毛笔也运过来了,他铺开黄纸摆好架势,笔走龙蛇的先抄一篇。

抄完之后放了笔,他转身在墙角前蹲下,用一只眼睛往墙缝里望。房子太老了,墙缝裂开又粗又深的一条,里面正嵌着长长一条白蛇。白蛇大头冲下,已然一动不动。

无心把毛笔杆插进墙缝,先从上方挑出了白蛇的细尾巴。一手捏住尾巴尖,他控制着力气,慢慢的想把白蛇往外抽。苏桃歪着脑袋蹲在下方,能从墙缝深处看到白蛇的圆脑袋。圆脑袋上的两颗黑豆眼睛带着光点,光点浮动,就像它正望着她似的。

无心陪了无数的小心,费了许多的力气,终于把白蛇拽出了墙缝。白蛇脱了节似的瘫在地上,两颗黑豆眼睛眯成了细长形状,脊背上受了轻伤,一片鳞甲翘了起来。苏桃很心疼的掏出手帕蘸了水,轻轻的为它擦净伤口,又把翘起的鳞甲摁回原位。最后把手帕叠成一条,她给白蛇拦腰扎了个蝴蝶结,正好包住了它的伤口。

``它不能死吧?''苏桃问无心:``怎么都没反应了?''无心双手捧起了它:``死不了,你把我的书包打开。''苏桃看无心双手捧得高,便把书包也托到了胸前。无心把白蛇缓缓的往书包里送,不料白蛇忽然昂头一探,把个脑袋一直伸到了苏桃耳畔,随即仿佛力不能支似的,圆脑袋``啪嗒''一声,就落在了苏桃的领口里。

苏桃没害怕,用一根手指抚摸白蛇的脊梁:``无心,它一定是累坏了。''无心一手托着蛇身,一手把白蛇的脑袋抻了回来。把白蛇扔进书包里,他探头向内一瞧,就见白琉璃把两只眼睛眯得细细长长,雪白的蛇头上居然隐隐显出了人的表情,是个色迷迷的得意样子。

苏桃想了想,又红着脸低声笑道:``白蛇长得真好看,一点儿都不凶恶。我们好好的养它,兴许将来它成了精,就变成白娘子了。''无心起身把书包放到桌子上,低头继续往里瞧:``娘子,听见没有?桃桃等着你变成大美人呢!''

白蛇本来是细着眼睛翘着嘴角,像个人似的在笑;忽然听了无心的话,它立刻恢复了两只圆圆的黑豆眼睛,嘴角也当即下垂。一个脑袋往书包深处一钻,白琉璃不理他了。

无心和苏桃躲在屋子里,抄了整整一天的大字报。屋子里只要没人来,苏桃就很放松。高高挽起两只军装袖子,她把五颜六色的大字报晾得满屋都是。

无心在后面看她上蹿下跳的真卖力气,就放下毛笔,把她从窗户前面拽向后方:``你悠着干,横竖是没个完,我们索性磨洋工混日子,混一天算一天吧!''苏桃歪着脑袋对他笑:``要是我们天天都能在屋里抄大字报,没人管我们,就好了。''无心对她一笑,知道她是吓破了胆,有个遮风挡雨的窝供她藏身,她就心满意足。

苏桃用湿毛巾擦净了手上的浆糊,拎起无心的书包说道:``我和白娘子玩一会儿,你抄完了就叫我。''无心没回头,一边在水杯里洗毛笔,一边说道:``别让它往你身上爬。''苏桃把手伸到书包里了:``没事,它又不咬人。''无心背对着苏桃一咧嘴,好像都听到了白琉璃的奸笑声。

在指挥部混过一天之后,无心带上苏桃,随着大队人马又回了招待所。陈部长一整天都在一中校园里,挥汗如雨的要收拾出一个新指挥部。招待所里都开晚饭了,他还干劲十足的不露面。他不露面,小丁猫也没露面。无心吃饱之后,照例是挎着书包去二楼卫生间撒尿。苏桃像只惊弓之鸟似的,在外面靠墙站着等他。

卫生间开着窗户,傍晚时分,光线还不算很暗。无心登上小便池,闭着眼睛腆着肚子,正是要尿不尿之时,忽然感觉后脊梁不大舒服。莫名其妙的回过了头,他很意外的看见了小丁猫。

小丁猫手里拿着树干粗的一卷卫生纸,正在蹲坑。坑位之间砌着半人多高的矮墙,前边没门。小丁猫抱着卫生纸,像是蹲进了暗沉沉的洞里。对着无心一点头,他淡定的问道:``吃完了吗?''无心听了他的提问,真有心不搭理他:``嗯\ldots{}\ldots{}吃完了。你吃了吗?''小丁猫把下巴抵在卫生纸卷上,垂着眼皮答道:``还没有。''

无心转向前方,很勉强的挤出了几滴尿。系好裤子刚要走,小丁猫又发了话:``有火吗?''无心从裤兜里掏出了火柴:``有。''小丁猫从衬衫胸前的口袋里抽出一支香烟叼在嘴上,含糊说道:``好极了。''

无心没法子,划燃了一根火柴走上前去,给小丁猫点燃了香烟。小丁猫很销魂的深吸了一口,然后一边从鼻孔嘴角里往外喷烟,一边慢吞吞的扯下了长长一条卫生纸,向前塞到了无心怀里:``无以为报,给你点纸,拿去擦屁股吧!''

无心接了一大团卫生纸,哭笑不得,同时又很不自在。因怕小丁猫再说出什么惊人之语,他转身便往外走。结果刚一出门,就见陈部长带着个人,从远处走过来了。

陈部长素日生龙活虎、杀气腾腾,此刻却是单手扶了墙壁,一段路让他走得摇摇晃晃疲惫不堪。苏桃怕他,低着头装看不见;而无心一边叠着手中的卫生纸,一边迎着陈部长走上前去。手指点上卫生纸,他不动声色的低了头,发现陈部长身边跟着的,不是人。

不是人,是个鬼,而且是个面熟的鬼。鬼脸狰狞,曾在一中吓过苏桃。傍晚时分,阳气弱阴气盛,有些力量的阴魂,可以四处游荡了。

苏桃看不见鬼,正扭头等着无心走近。而无心收回手指,迟疑着没有画出符咒。回头盯住了陈部长的背影,他看见鬼影已然贴上陈部长的后背,而陈部长无精打采的半路拐弯,推门进了卫生间。

\chapter{长相守}

苏桃因为先前一直是活的干干净净,家变之后又一直活的不干不净,所以如今就把洗热水澡看成了一桩大事。她自己昨晚洗了,洗的舒服,今天就非要让无心也去洗。无心带着她往三楼走,一边走一边扭头往二楼走廊里看。走廊里很安静,陈部长进了卫生间后,再也没有发出动静。

田小蕊还没有上楼,于是苏桃趁机关门,换了一身单薄的蓝布衣裳。招待所的公用盥洗室里有水盆和肥皂,在无心洗澡的空当里,她埋头洗净了自己一身脏衣。雪白的泡沫从指间溢出,她拼了命的揉搓,不敢闲着。一个人闲着,她害怕。

走廊中由远及近的响起了脚步声,苏桃吓得停了手,大睁着眼睛往门口望。有人晃着大个子来了,脸上嘴上全都油光光的,正是顾基。顾基见了苏桃,也是一愣,随即就大踏步的走了进来:``洗衣服哪?''

苏桃一点头,点完了头才想起自己忘了微笑。未等她补个笑,顾基拧开旁边的水龙头,已经弯腰接水洗起了脸。三把两把洗干净了,他水淋淋的抬起头,自觉清爽了许多,自信心也增长了十分。侧身靠在水池边沿,他留恋着不走,笑模笑样的问苏桃:``好洗吗?''

苏桃听他满嘴废话,仿佛是不带目的,心中倒是轻松了一点,喃喃的答道:``好洗。''

顾基站没站相,人高马大的乱晃:``你是十五岁吧?''

苏桃点了点头:``嗯。''

顾基刚要继续说话,不料肩膀上温暖的一沉,扭头一瞧,他看到了一只雪白雪白的手。随即无心的声音响起来了,带着一点惊讶语气:``你看见陈部长了吗?''

顾基回头看他:``没看见''

无心热气腾腾的站在他面前,面孔被热水蒸成粉红色,看着过于鲜嫩了,几乎有些可怕:``陈部长一直在找你,刚才在二楼还向我问起过你呢!你快去看看他吧,他好像还挺着急。''

顾基不情不愿的叹了口气,转身往外走去。等他走远了,苏桃转向无心,小声问道:``你又去二楼了?''

无心摇头笑道:``骗他的。''

苏桃也笑了:``汗衫呢?''

无心把脱下的汗衫递给了苏桃,苏桃接过来摁进肥皂水里。无心一出现,她的心绪就安宁了。

等到苏桃把衣服全晾上了,两人便一起下楼进了院子,坐在水泥砌成的花坛边沿看夕阳。无心把两只手搭在了大腿上,苏桃很自然的拉过了他的一只手,自己伸巴掌和他比了比大小。无心的巴掌当然是比他大了一号,手指修长,掌心的皮肤也比她硬。她其实从很小的时候开始,就想和父亲比一比巴掌,但是父亲一年出现一次,太陌生了,她不好意思主动去拉父亲的手。后来长大了,十几岁的姑娘了,就更羞于和父亲亲近了。

家里只有母亲和她,空气中弥漫着的都是女人的气味。在纯粹女性化的世界里生活惯了,她对于男人有些本能的怕,唯有无心让她感觉温暖。从两人第一次相遇起,无心就表现得像个大哥哥和小爸爸——是她理想中的哥哥和爸爸,不和任何人分享,是她一个人的。

两人很平静的坐了许久,末了看夕阳彻底沉到地平线下了,精神病鲍所长也手舞足蹈的把院子扫完了,才一起回了楼内。

无心进了二楼的房间,迎面就见陈部长在床上半躺半坐,正捧着个搪瓷缸子吃饭菜。无心一边关门,一边说道:``陈部长,你脸色不好。''

陈部长呻吟一声:``你也看我脸色不好?方才在厕所里遇见小丁猫同志了,他也问我是不是生了病。我倒是没怎么的,就是累。你风不吹日不晒的抄一天大字报,哪知道我们是怎么干活的?妈的从早到晚,一分钟都没歇过。''

无心发现陈部长的身边左右很干净,孤魂野鬼全没有了。脱了鞋躺在床上,他把书包摆上了自己的肚皮。书包里的白琉璃带着一点分量,一动不动的压迫着他,让他想起苏桃的手,软绵绵热烘烘的,也带着一点分量。

陈部长用勺子刮着搪瓷缸子,一边刮一边嘟嘟囔囔:``你是不是耍顾基了?怎么告诉他我要找他?我什么时候说要找他了?你再敢耍他我饶不了你!你到底是怎么回事还没定论呢,就敢拿别人开涮了?''

无心不言语,心思又转移到了鬼上。鬼魂一路追随着陈部长,必是有所图谋,没有平白无故消失的道理。

无心竖着耳朵躺了一夜,捕捉房内房外的一切动静,然而一夜平安无事。翌日天亮,陈部长跑去餐厅大嚼一顿,又恢复了往昔的雄风。无心和苏桃则是前往指挥部,继续抄大字报。

如此过了五天,到了第六天中午,陈部长带着大队人马,终于把一中布置出了眉目。在小丁猫的指示下,他把一中的教学楼按照楼层分成三个区域。顶层的三楼是宿舍区,二楼是办公区,一楼是活动区。单独趴在校园一角的一趟平房,本来是学校里的食堂,如今也恢复了功能,继续开伙。教室内的桌椅被清了出去,在校园一角堆成一座枝枝杈杈的木头山。陈部长自己做了主,要把它充当柴禾,留给食堂烧火。

指挥部搬了家,全体人员一起离开了墙壁裂缝窗户透风的小学校。招待所的好日子也结束了,小丁猫为了安全起见,决定住到三楼的宿舍区里。

无心和苏桃在文县没落脚处,别无选择,也得住校。三楼的一整条长走廊,从楼梯开始分成了男女两区。田小蕊和李萌萌立刻宣布不回家了,按理来讲,苏桃就应该和她们一起住。然而想到自己以后每天早晚都要心惊胆战的装哑巴,夜里连句梦话都不敢说,苏桃当即痛苦的有些不能忍受。与此同时,无心也开了口:``我看走廊最里头有间小屋,顶多能放下一张床一张桌子。那屋给我吧,再给我弄张上下铺,我和苏桃一起住。''

此言一出,听众们一起愣了愣。小丁猫抬手抚额,低头叹道:``哎呀妈呀\ldots{}\ldots{}''

陈部长则是义愤填膺:``你俩还要不要一点脸了?我们这是革命的地方,不是腐化堕落的场所!''

李萌萌也一撇嘴:``流氓!''

苏桃红着脸,脑子里面嗡嗡响。一眨眼,她眨出了一滴很大的眼泪:``我们不分开。''

小丁猫一摊双手:``这叫什么事?你们又没有结婚,再说年龄也没到嘛!''

无心满不在乎的答道:``我们两个都不在乎,你们跟着害臊什么?反正我就和苏桃一起住了。你们要是不同意,我俩就去校园里打地铺。''

小丁猫摆了摆手:``不要冲动,打了地铺更丢人。''然后他回头问陈部长:``小陈,你有什么意见?''

陈部长被无心惊人的无耻要求震住了,一张嘴直打结巴:``这这这这算是搞、搞破鞋吧?''

小丁猫对于搞破鞋没有研究,所以也有点含糊:``他俩男未娶女未嫁,也算是搞破鞋吗?''

陈部长继续口吃:``反正就、就是不要脸呗!''

小丁猫把双臂环抱在胸前,对着无心和苏桃吸了一口气,又咂了一下嘴。马秀红在一旁见了,有点看不下去,低声提醒他道:``不要在小事上浪费精力。''

小丁猫深以为然的一点头:``那行,给他们弄张上下铺吧!看出来了,这两位是非好不可。红总要是能给他们一张床,他们能头也不回的跑到红总那边去!''

小丁猫一发话,陈部长也就无话可说,只得拧眉瞪眼的派出人手,将一张上下铺双层床搬进了走廊尽头的小屋。小屋里原本有一张课桌,此刻靠墙放了,倒也腾出了一小片空地。等到床摆好了,人也走了,无心关了房门,低声说道:``桃桃,我不是故意要坏你名誉,我是不放心你和她们在一起生活。''

苏桃刚才被``流氓''和``破鞋''两个词激出了眼泪,但是现在眼泪干了,情绪也就平定了:``我本来也不想和她们住,我更不怕坏了名誉。我是黑帮分子的女儿,爸爸妈妈都是自杀,我早就没名誉了。''然后她把书包摘下来放在课桌上,低声又道:``我的身份多瞒一天,我就多活一天,哪天暴露了,死活就由不得我了。造反派都把我爸爸妈妈逼死了,我还管他们怎么想怎么说?我才不管,我才不在乎。''

无心不知道怎么宽慰她,没法宽慰,她说的都是实话。转身撼了撼铁架子床,他问道:``我们怎么住?你选上铺还是下铺?''

苏桃扭头看了看,见床边焊着一道细细窄窄的小铁梯,无心是高挑身材,上下一定不便,于是答道:``我睡上铺。''

无心举手又摇了摇上铺的护栏:``夜里可别掉下来。''

苏桃推开窗户,向外望了望。校园里没了学生,但是花木还在,深深浅浅绿成一片。在无边无际的大恐怖中,她忽然小小的快乐了。

到了下午,无心找到了负责后勤的李萌萌,顶着对方的冷言冷语要来了两套被褥和一只半旧的铁盆。苏桃爬到上铺,铺好被褥,又把无心的书包拎到身边,从里面放出了白琉璃。白琉璃还系着大蝴蝶结,黑豆眼睛上的光点转了一圈,他认清现实,很自然的把脑袋搭在了苏桃的大腿上。无心一抬头看见了,不禁一皱眉头:``桃桃,别让他总缠着你。他挺通人性,你越对他好,他越讪脸。''

苏桃用手轻轻去握白琉璃的蛇身:``白娘子长得真可爱。''

夸完一句,她小心翼翼的把小白蛇推到一旁,下了床出门去上厕所。她刚一走,无心就扯着尾巴把白琉璃拽下来了。

白琉璃被他扔到下铺床里,当即笨拙的盘起了身体。无心伸手一指他的蛇头:``我说,你活着的时候挺正经的,怎么死了之后反倒变成色鬼了?当初咽气的时候和我装高雅,又要看花又要看雪,结果现在可好,改看小姑娘大腿根了。''

下铺暗处骤然显出了白琉璃的身影:``我看腻了,不行吗?你个来历不明的老妖怪,老骗子!''

无心俯身对着他一耸肩膀:``老妖怪又怎么样?我承认我是老妖怪,可我很英俊啊,我有胳膊有腿啊!你呢?你现在不过是条白蛇,盘成一堆像牛粪似的。哪天我不高兴了,剥了你的皮想清蒸就清蒸,想红烧就红烧。到时候我把你盛在碗里,看你还怎么骚!''

白琉璃彻底现形了,虽然还是幻影,然而看起来是特别的真切。因为骂不过无心,他气得在床上翻江倒海乱飘乱窜,又对无心变出狰狞的鬼相。无心针锋相对的做了个鬼脸,且把软塌塌的白蛇拎起来,在蛇头上嘣嘣嘣凿了好几个爆栗。

两人正在大战,房门忽然开了。苏桃慌里慌张的说道:``无心,你听到走廊里有人喊话了吗?说是今晚要到机械学院里开批斗会,楼里不留人,全都得去!''

白琉璃立刻附回白蛇的身,和无心一起扭头去看苏桃。无心很惊讶的望了望天色:``都快吃晚饭了,还开批斗大会?''

苏桃关了房门,小声说道:``有个女生,本来是和小丁猫在一起的,后来回了保定,刚才又回来了。好像是带了个什么反动学术权威,必须今天晚上就开会批判。还有啊,他们下午又派人出去打架了——不,不是打架,是抄家,就是为了晚上的大会做准备。''

无心看苏桃脸色煞白,便攥了她的手臂,把她拽到身前,抬手拍了拍她的后背:``别怕别怕,和我们没关系,我们去了也是做观众。''

话音落下,大开的窗口忽然吹进了一股子凉风。无心挡在苏桃面前,就感觉风凉得怪,不是春风该有的温度。走到窗边向外一望,他看到校园里刚刚停下两辆卡车,卡车后斗上满载着全副武装的青年工人。武卫国推开车门跳下来了,大踏步的上前和小丁猫握手。而消失了将近一个礼拜的杜敢闯站在小丁猫身边,派头很足的把双手背在了身后。

无心嗅到了空气中的杀气和鬼气,杀气是武卫国等人带来的,十分之重,正在缓缓压下弥漫在阴暗处的鬼气。

房门又被敲响了,无心转身走去开了门,来者却是顾基。顾基一边好奇的打量着房内情况,一边说道:``你俩赶紧下楼集合,我们要去学院了!''

无心装傻:``什么时候开晚饭啊?''

顾基很羡慕的缩回了脑袋:``先出发,到了学院有人给你们发面包。

\chapter{批判大会}

无心和苏桃一人得了一个印着``联指''字样的红袖章。苏桃挎着无心的书包,书包里面趴着白琉璃。无心本来不让她带,可她扭扭捏捏的不听话。白琉璃已经是她的宠物了,她舍不得把对方独自留在宿舍里。

两人一进校园,就成了众人眼中的怪物。人人都知道他俩公然的住一个屋了,堪称天下第一不要脸。苏桃在人前从来不抬头,永远跟在无心的斜后方,要么拉着无心的手,要么扯着无心的后衣襟。无心把双臂环抱到胸前,带了一点儿满不在乎的痞气,顶着四面八方的注目望天。

校园里乱过一阵之后,手握钢枪的工人们跳下卡车,把位置腾给了联指的人员。无心带着苏桃爬上其中一辆卡车,在角落里站稳当了。和他们挤在一起的是顾基——顾基红着脸很兴奋,同时又很自傲——和他同样出身的小子们,现在都成了过街的老鼠,唯有他攀着高枝左一蹿右一跳,还能坦坦然然的坐着卡车看热闹。

人们尽量的往卡车后斗上挤,挤满一辆走一辆。李萌萌带着一帮半大丫头,拎着浆糊桶站在地上等下一辆,一个十五六岁的小姑娘抱着满怀的彩色标语。标语卷成了卷子,有些褪色,染了她半脸花,然而小姑娘不在乎,斗志昂扬的又说又笑。

打头的卡车开出校园一上大街,无心和苏桃就都吃惊了。中午来时,街上还是一副常态,不料只过半天的工夫,大街就变成了红海洋。不知道是谁张罗出的大场面,满街都是半大不小的青年少年,有的举着红宝书,有的举着小红旗,已经熙熙攘攘的排好了长队,粗略一看人数,没有一千也有八百。队伍两边有联指的纠察队,吆五喝六的维护秩序,另有一支乐队排在一旁,正在拉着手风琴吹着小铜号,演奏一曲《大海航行靠舵手》,整条队伍随着音乐齐声合唱。大卡车靠着街边向前缓缓开动,无心居高临下的望着游行队伍,发现队伍中的人们仰头望着卡车,仿佛是十分羡慕。

乐队且行且奏,演奏一阵之后偃旗息鼓,路边电线杆子上悬挂的大喇叭出了声,取代乐队继续歌唱。游行队伍的行进速度略微缓慢了,因为前方打头的先锋小队停了脚步,随着音乐跳起了忠字舞。前头跳,前头跳完了后头跟着跳,队伍越汇越长,最后竟是一眼望不尽头尾,一路载歌载舞的往机械学院移动。

机械学院坐落在文县的一端,当初建造校园的时候,位置算得上是偏僻;然而随着文县人口越来越多,机械学院建成之后,反倒是落在了人窝子里。卡车开不动,随着队伍慢慢的前行,十里路走了将近两个小时。此时天光已经偏于黯淡,机械学院内提前开了路灯;被布置为批斗大会现场的体育场上,更是拉了电线加了探照灯,把前方的大主席台照了个雪亮。主席台下的空地被分成了几片区域,学院学生人数不多,已经整整齐齐的在一侧站成了方阵。无心等人排了队伍刚进会场,方阵就爆发出了呼声:``热烈欢迎联指的同志!''

小丁猫走在队伍外面,没有吭声,只对着杜敢闯一点头。杜敢闯当即挥着红宝书高声答道:``向学院的革命小将们致敬!''

她话音一落,后方众人立刻在李萌萌的指挥下齐声高呼:``致敬!致敬!''

学院里造反派的第一号领袖,颠颠的跑上前去和小丁猫握手,又和杜敢闯握手,再和陈部长武卫国等人握手。其余人等则是被安排着站好了,一抬头就能看清主席台。

会场之内歌声此起彼伏,等到游行队伍络绎进场了。全场在纠察队的指挥下渐渐肃静,有人上了主席台,将一排桌子上的麦克风挨个试了试声音。见麦克风全都出声,会场喇叭里立刻又响了音乐。在激昂澎湃的乐曲声中,小丁猫穿着一身整洁利落的衬衫长裤,在杜敢闯武卫国等人的簇拥下,一边鼓掌一边上了主席台。

陈部长带着一帮兄弟站在台下,像条黝黑的大狼狗,握着短棒巡视全场。乐曲声音骤然一停,小丁猫等人分主次落了座。照向主席台的电灯仿佛又提了亮度,主席台后贴着白纸黑字顶天立地的大标语,笔画分明的如同刀剑。兵分两路的大标语拥着前方一排造反领袖,领袖们全仿佛是从鬼门关里齐步并肩杀出来的。

小丁猫占据中央位子,电灯自下而上的射出光芒,烘托出了他一张阴森森的娃娃脸。而陈部长端着手臂小步跑到主席台下,面对着会场举起电池喇叭,高声喊道:``全体起立!''

``哗''的一声,场中成千上万的人,毫不犹豫的全打了立正。先前蹲着坐着的,当即向上一个鲤鱼打挺;先前站着的,则是把腰挺得更直、头抬得更高。主席台上的小丁猫等人也起了身,转向了主席台背景板贴着的毛主席像。所有人都把红宝书举到了胸前,杜敢闯高声喊道:``首先,让我们怀着对毛主席无限热爱、无限信仰、无限崇拜、无限忠诚的心情,敬祝我们心中最红最红的红太阳、伟大领袖毛主席万寿无疆!''

下面无数只手举起红宝书,挥成无边无际的红色波浪:``万寿无疆!万寿无疆!''

呼声结束,杜敢闯继续喊道:``敬祝毛主席的亲密战友林副主席身体永远健康!''

红色波浪在呐喊声中汹涌了:``永远健康!永远健康!''

杜敢闯转向场下:``下面,我们同唱革命歌曲《东方红》。预备——唱!''

革命群众们虎啸似的唱完一曲《东方红》,杜敢闯又主持学习了一段毛主席语录。一切结束之后,台上众人各归各位。小丁猫单手扶着麦克风,轻描淡写的讲了一段路线政策。然后把麦克风向旁一推,他率先起立。

他一起身,杜敢闯等人随即也跟着起了身。几名纠察队员上台把桌椅搬走。而小丁猫又一挥手,蹲在阴暗角落里的牛鬼蛇神们就被革命小将押上了台,其中打头阵的是个秃脑袋的老头子,一脸的松皮和老人斑,是杜敢闯特地从北京抓回来的资产阶级反动学术权威。此权威罪恶滔天,居然敢在旧社会和鲁迅打笔仗;不但打笔仗,还老而不死,活得比鲁迅长;真是不思悔改、反动到家。权威在北京各大学游走了小半年,已经被批的只剩了悠悠一口热气,但是杜敢闯需要他为革命发挥余热,所以带着亲信直入北京,抓野狗似的把权威塞进麻袋里,用吉普车一路运来了文县。

紧随权威上场的,是个六七十岁的老头子,名叫陈盖世。陈家本是文县第一大族,富贵的无法言喻,陈盖世年轻的时候,还在邻县买过一任县长当。日本人一来,陈县长宁死不屈,被打成半疯,疯了好几年才认识了人。刚清醒了没几年,他又倒了霉,差点没让政府当成土豪给镇压了。颠颠倒倒的活到如今,陈盖世的儿女家人被打死了十之八九,他没死,又疯了。

从陈盖世往后,是长长的一大串牛鬼蛇神,各有罪名,全挂着二三十斤重的大铁牌子。铁牌子是用细铁丝挂在脖子上的,细铁丝受了铁牌子的坠,刀刃似的往肉里勒。百十来人全上了台,权威却又出了状况,一个脑袋抬不起来,扣在头上的纸帽子不住的滑落到地。纸帽子是马粪纸糊的,是个一米多高的圆锥,正经戴都戴不稳,何况权威的一口热气已经撑不住了秃脑袋。小丁猫见纠察队员一直在给权威戴帽子,没完没了,破坏了大会的气氛,就对着杜敢闯一抬手,低声说道:``找几个钉子去!''

杜敢闯恍然大悟,立刻要来一盒摁钉。大踏步的走到权威面前,她用摁钉把纸帽子钉在了权威的头上。钉子刺破马粪纸,深深的扎进头皮。权威一动不动,仿佛是胸中的热气快要散尽了。

她好容易钉牢了权威的纸帽子,权威身边的陈盖世又疯叫上了,一嘴的牙没剩几个,透气漏风的胡喊:``小鬼子,我不怕你们。要打要杀——''

没等他胡言乱语完毕,杜敢闯从身边的纠察队员手中接过皮带。一皮带抽向了陈盖世的瘪嘴。皮带的铜头足有半斤来重,结结实实的凿上了陈盖世的牙床。老疯子立刻就不叫了,他被自己满嘴的鲜血给呛着了。

等到全体牛鬼蛇神都弯腰撅成九十度了,批判大会正式开始。小丁猫一直站在主席台一侧,他偶尔的一点头一微笑,一举手一投足,都表明他才是幕后的主持人,但是他始终没有亲自动手。杜敢闯活跃在了批斗大会第一线,一条武装带捆住了她的虎背熊腰,她一边疾呼批判,一边留意着小丁猫的反应。论长相,她自认不如马秀红,只能外表缺乏内里补,凭着自己的智慧和力量在小丁猫身边占据一席之地。虎虎生风的抡起皮带抽向牛鬼蛇神老家伙们,容貌和身材忽然都不算什么了,她是飒爽英姿五尺枪,她是天翻地覆慨而慷。

权威和陈盖世,不知是什么时候咽的气;仿佛在革命群众涌上主席台前,他们两个就被杜敢闯抽得不再动了。台上最后演变成了单方面的大混战,上百名牛鬼蛇神被小将们打得满台乱滚,鲜血顺着主席台往下滴滴答答的流。

苏桃站在队伍的边缘,从头到脚都冰凉的僵硬了。忽然意识到了左手的温暖,她艰难的低下头,发现自己的小拳头,被无心的大拳头包住了。

无心的热度融化了她,让她失控似的打了冷战。她把声音压到最低:``无心,我受不了,我们走吧。''

无心环顾四周,向她微微的歪过了头耳语道:``走不了,纠察队看着呢。别怕,没你的事。''

苏桃没敢说自己吓得憋了尿。低头闭眼咬紧牙关,她什么都不想了,只是希望时间快点过。

午夜时分,无心等人被大卡车运回了一中指挥部。食堂已经开了伙,预备了不要钱的晚饭。无心取出自己前一阵子买的大饭盒,带着苏桃去食堂打了满满一饭盒饭菜,又拿了两双筷子两只勺子。两人上楼回了小屋,无心对苏桃说:``吃吧,吃完就睡。再不睡天都要亮了。''

苏桃吃不下,眼前总晃着一片血红颜色。闷头喝了几口热水,她出门到公用的水房里洗漱了,然后回房爬到了上铺。屋里亮着电灯,上铺比下铺还亮。无心捧着饭盒背对着床,一边吃一边说道:``我不看你,你快脱了睡吧。''

苏桃知道他是好人,所以放心大胆的脱了外面衣裤。展开棉被盖住双腿,她缩进被窝里,又想方设法的脱下了汗衫里面紧贴身的半截小背心。小背心掖在枕头下,她重新套好汗衫,胸膛登时就松快多了。侧身躺在枕头上,她开口说道:``我脱完了。''

无心把饭盒放到桌子上,转身一拍她搭在护栏上的手背:``睡吧,别多想。世界不会永远都是一个模样,你还小,只要活着,就一定能等到转机。''

苏桃点了点头:``我知道,我能忍。''

无心叹了口气,端着饭盒出去倒剩饭。而白琉璃费了天大的力气,攀着床栏爬去了上铺。一头钻进被窝里,他百般曲折的一直向上,最后在苏桃眼前探出了头。

苏桃看着他的黑豆眼睛,又探头嗅了嗅他的脑袋,没有嗅到臭味。白琉璃一抬圆脑袋,在苏桃的嘴唇上蹭了一下,又慢慢的向前游动,一直游到了苏桃的颈窝下。苏桃不嫌他,拉了棉被盖到下巴,闭上眼睛睡了。

无心洗漱归来,早把白琉璃忘到了脑后。锁上房门关了电灯,他把衣裤一脱,滚上床也睡了。

无心和苏桃是真累,说睡就睡。到了万籁俱寂的黎明前夕,房内的空气忽然一颤,一个人形的黑影破墙而入,出现在了床前。

黑影脚下无根,缓缓飘向了上铺的苏桃。正在此时,白琉璃不声不响的出现在了黑影后方。黑影忽然混乱的闪烁了,仿佛是要向上升腾,然而影子越来越淡,最后生生的消散在了半空中。

白琉璃吞噬了一只怨气冲天的恶鬼,感觉十分满足。飘到上铺趴在苏桃身上,他瞬间消失。棉被边沿略微一动,他重新变回了小白蛇。

\chapter{革命生活}

凌晨时分,无心半睡半醒的把眼睛睁开一线,就听上铺起了窸窸窣窣的响动。苏桃总是比他早起一刻,因为要脱了汗衫穿小背心。在被窝里脱,束手束脚的太不容易,只好是趁着无心没醒,她做贼似的坐起来先脱后穿。

她一醒,白琉璃也跟着活动了,盘在枕头上昂起脑袋,两只黑豆眼睛一起往前使劲,直盯着前方一对毛桃似的小乳房。看着看着,他东倒西歪的游了过去,把脑袋搭在了苏桃的大腿上。苏桃浓厚的长发中分披下,乌云似的堆了满肩满背。黑发之间露出粉白的脸儿,白琉璃仰头看她,看她生得秀眉明眸,小嘴唇红通通的。

无心在宿舍里已经睡了一个礼拜,始终没有留意过白琉璃的行踪。上铺的动静越发大了,是苏桃起身穿了长裤。眼看一只赤脚伸下来踩住了床角的铁梯,无心闭上眼睛继续装睡,想让苏桃自自在在的把上衣穿好。而在他目不能视的空当里,白琉璃偷偷溜下床去,爬到床底藏起来了。

苏桃的邻居们都是男生,所以她须得赶在所有人的前头洗漱完毕。男生们都知道走廊尽头的小屋里住着个苏桃,浮想联翩之余,男生们的形象不由得走上两个极端,要么羞涩的严装密裹,要么奔放的赤身露体。

陈部长天天杀气腾腾的光着膀子,在走廊里来回的溜达,已经冻出了感冒,并且还被无心起了个外号,叫做黑背。又因为他的确是通体黝黑,所以外号立刻传开,被外界公认为是名副其实。
陈部长听说自己成了狼狗,怒不可遏,立刻和无心打了一架。

两人是在三楼水房里打的,陈部长提前把门锁上了,不许旁人进来劝架,想要一拳把无心打死;不料无心动作极快,总是在他出手之前出手。听众们聚在门外,就听水房里面噼里啪啦声震云霄,也不知道是谁在打谁。

末了房门一开,陈部长气冲冲的出现在了门口,满身都是巴掌红印。虎目圆睁怒视了面前的喽啰,他冲开人群怒道:``不打了!''顾基穿着大裤衩,端着水盆追上了他:``怎么不打了?''陈部长头也不回的骂道:``他像个老娘们儿似的,老他妈扇我。''顾基紧赶慢赶:``你揍他啊!''陈部长降低了一个调门:``他乱窜,我打不着!''

水房一役结束之后,陈部长把衣服又穿上了,同时越发的想要强奸苏桃。苏桃也从空气中嗅到了危险味道,所以一出房门就是东躲西藏,基本不会单独活动。东张西望的刷了牙洗了脸,她一分钟都不耽搁,该走就走。回房之后把门一关,眼里再有了无心,她披头散发的松了口气,一颗心算是跳平稳了。

无心已经穿戴整齐了,接了她的水盆往外走。屋里腾出了空地,她先开了窗户透气,然后坐上无心的床上,对着前方课桌上的一面圆镜梳头发编辫子。乌黑的头发在她指间一股一股的扭绞着,带着光泽和弹性。及至辫子梳利落了,她把鬓角碎发往耳后一掖,起身弯腰给无心叠了棉被,顺手抄起笤帚,把有限的一小块地面也扫了。

早饭照例是在去楼下的食堂吃。春日清晨的风,带着微寒的清新气。无心带着苏桃走在校园里,看到花木丛中已经有了鹅黄粉红的花影。扭头对着身边的苏桃一笑,他看苏桃也是一朵花;苏桃亦步亦趋的跟着他,不说话,花开在心里。

食堂的伙食很不错,起码比平常人家的饭菜要好。无心和苏桃坐在角落里,一个馒头还没吃完,顾基却是蓬着一头乱发来了。无心和他搭了话:``没洗脸吧?''顾基睡眼惺忪的告诉他:``我是来给小丁猫同志打饭的。''

无心抬头看了看墙壁上挂着的大钟:``他自己怎么不来?''顾基打了个哈欠:``他蹲厕所呢!''无心又问:``最近有活动吗?''顾基从大师傅手里接过装着馒头和咸菜丝的饭盒,嗤之以鼻:``你天天给他抄大字报,还用问我?''无心笑着咬了一口馒头,是真不知道。小丁猫的一切言行都是莫测高深,他看在眼里,看不明白。

顾基把饭盒送到了小丁猫的宿舍里。小丁猫住单间,能摆四张双层床睡八个人的寝室里,空空荡荡的只放了一张单人床和一套桌椅。顾基进门时,马秀红正在扫地。小丁猫面无表情的对他挥挥手,于是他很识相的放下饭盒就退下了。

双手捧着一杯热气腾腾的苦丁茶,小丁猫一口接一口的啜饮着。房门一开,杜敢闯虎虎生风的走进来了。对着马秀红严肃的一点头,她停到桌前开了口,声音却是出乎意料的柔软:``吃不吃早饭去?''小丁猫一指桌上敞开的饭盒,同时又摇了摇头:``吃不下。''

清晨是杜敢闯形象最佳的时刻,因为刚刚洗去脸上油光,能显出几分清洁相:``吃不下?''小丁猫点了点头:``光吃不拉,不是长久之计。''杜敢闯想了想,问道:``给你弄点番泻叶泡水喝?''小丁猫张嘴叹了口气:``再说吧,马秀红给我沏了一杯苦丁茶。如果苦丁茶没有效果,再试你的办法。''

此言一出,杜敢闯脸上一暗,额头和太阳穴上的粉刺则是鲜艳了许多。凭着她犷悍无匹的内秀,终究还是敌不过腌黄瓜似的马秀红。马秀红慢吞吞的扫着地,神情和心情都很淡定,并且没有要走的打算。

小丁猫放下茶杯,先让杜敢闯在自己身边坐下,然后问道:``红总最近有什么新动向吗?''杜敢闯略微来了一点精神:``他们的头目,前天去了长安县。''小丁猫放下茶杯,拉开抽屉找出烟盒:``长安县?''杜敢闯压低声音又道:``据可靠消息说,他们是找军火去了。''小丁猫抬眼看她:``他们有办法?''

马秀红像一缕香魂一样飘到小丁猫身边,划燃火柴给他点了烟。杜敢闯自动的将她从自己的视野中删除,开口答道:``长安县,有个军械库。''小丁猫当即一拍大腿:``他妈的!李作诚怎么还不到?''

杜敢闯瞟到马秀红又去扫地了,心里略微舒服了些许,感觉到了自己的价值:``李作诚昨天发来了电报,说他已经抢到了两架重机枪。如无意外的话,他在三天内必到。''小丁猫用手中的香烟在空中画了个圈:``他们一到文县,立刻封锁火车站,不许红总利用铁路运送武器。保定那边有新情况吗?''杜敢闯答道:``一号要组织队伍,冲击军区。''

小丁猫在烟灰缸里摁熄了香烟,抬手揉了揉肚子:``李作诚一到,我们立刻往长安县去,赶在红总之前占领军械库。在此期间宣传工作不能停,不要让我在文县听到红总的声音。''话音落下,他从抽屉里掏出一大卷卫生纸,转身就往外走。杜敢闯意犹未尽的站在原地,留恋着不肯走,直到马秀红把笤帚扫到了她的脚下。

到了上午时分,指挥部里人气旺盛了。无心挽着袖子蹲在校园地上,露天抄写大字报。一张大纸一个字,一行标题能贴满半面墙。田小蕊带着一帮十七八岁的姑娘围站一圈,都说他是一笔好字,不像李萌萌抄的大字报,乱七八糟,像狗爪子蘸了墨水挠出来的。

田小蕊看够了毛笔字,又居高临下的笑道:``苏桃,你别给他拌浆糊了。让我听听你的嗓子怎么样,要是好,我就吸收你进我们宣传队。''苏桃蹲在一旁守着个浆糊桶,抬头对着田小蕊笑了笑:``我不会唱,就会干活。''

田小蕊扭头对着女伴使了个眼色,女孩子们心照不宣、哄堂大笑。苏桃知道她们是在嘲笑自己离不得无心,火烧火燎的红了脸,她垂下头,在写好的大字报背后刷浆糊。

无心把毛笔伸进墨水瓶里搅了搅,一边审视着大字报,一边说道:``散了吧散了吧,让你们看个热闹,你们还看起没完了。从现在起,愿意给我刷浆糊的可以留下,不愿意刷的马上滚蛋。好好的大姑娘顶着太阳傻站着,不怕晒黑了你们的脸?''

田小蕊正要反驳,可是未等开口,身边女伴忽然一扯她的手臂。她转脸望去,就见小丁猫带着马秀红和顾基,一路慢悠悠的走了过来。未等宣传队员作鸟兽散,小丁猫已经停在了无心面前。背着双手弯下了腰,他仔细看了看无心的字,随即起身说道:``明天我要下乡去,你俩跟着我一起走,记得晚上去二楼领笔墨彩纸,明天都给我带齐全了。''

顾基低着头,依稀感觉到了田小蕊射向自己的目光。理智上讲,他知道田小蕊挺好看,配得过自己;可是田小蕊牙尖嘴利,自己又实在是怕她。而田小蕊瞪了他半天,见他佝偻着宽肩阔背装死狗,就气得把头一扭,恨他是个徒有其表的窝囊废。

小丁猫吩咐完了,迈步要走。然而杜敢闯带着两名青年从楼里匆匆跑出,凑到他耳边耳语了几句。小丁猫听后,扭头望向了顾基:``顾明堂是你父亲?''顾基吓了一跳:``他\ldots{}\ldots{}是。''小丁猫对他一笑:``顾明堂今天凌晨逃出钢厂保卫处,投奔红总了。''

顾基当即退了一步,一张脸褪了血色,变成煞白:``我不知道\ldots{}\ldots{}他可能是被人打急了\ldots{}\ldots{}不关我的事,我不知道。''小丁猫看着他,不说话。顾基在大恐慌中带了哭腔:``我真不知道\ldots{}\ldots{}我早就和他划清界限了,他是他我是我,我都连着一个多礼拜没回家了\ldots{}\ldots{}''

小丁猫轻描淡写的说道:``老子英雄儿好汉,老子反动儿混蛋,基本如此。''然后他一抬手:``把顾基关起来,等我闲了,再处理他。''

杜敢闯身后的两名青年一拥而上,反剪了顾基的双臂。顾基比在场的所有人都高大,都魁梧,可是在小丁猫面前弯了腰低了头,他只会呜呜的哭,两条长腿乱晃,吓得没了骨头。小丁猫又道:``派人去趟顾家,看看顾明堂的老婆还在不在。如果在的话,一并逮捕。''

顾基被押走了,宣传队也识相的散了。无心蹲在大太阳下,抄好一张大字报放到一旁晾着。苏桃怕大字报被风吹走,捡了两块石头压在纸上。不料杜敢闯忽然质问道:``你为什么用石头压迫革命的大字报?''苏桃吓得立刻就把石头挪开了,改用双手压住大字报的两边。

杜敢闯大踏步的走过苏桃身边,一脚踩上了她的手背。小丁猫不以为然的摇了摇头,跟着也走了。无心放下毛笔,抓过苏桃的手揉了揉,出声问道:``疼不疼?''苏桃意意思思的又想往他身后藏:``不疼。''无心小声说道:``你等着,晚上我给你报仇去。''

\chapter{下乡去}

傍晚在食堂里吃过了晚饭,无心带着苏桃上街溜达了一圈。在一家正要关门下班的副食品店里,无心买了半斤花生糖。把花生糖和苏桃一起送回宿舍,他从外面锁了房门,让苏桃吃着花生糖和白琉璃玩。

独自一个人在楼后坐到天黑,他开始络绎的见鬼。挑了一只最丑陋的鬼捉住了,他有心和对方谈谈条件,然而丑鬼层次极低,没有头脑,只会怨气冲天的乱窜。无心无可奈何,只好用一张纸符把它封住。捏着纸符的一角偷偷回到楼内,他溜到女生宿舍所在的一侧走廊。杜敢闯也是享受单间的待遇,单间正是靠着楼梯。趁着夜色浓重,他把纸符伸到杜敢闯的房门前,``嚓''的一声一撕两半。

然后在丑鬼成形之前,他蹑手蹑脚的溜了。

无心回到房内,见苏桃正在用湿毛巾给白琉璃擦身,擦得白琉璃雪白雪白。白琉璃长条条的瘫在无心的床上,细着眼睛仿佛在笑。苏桃看他和小女孩看布娃娃是一个心情,一边擦一边嘀嘀咕咕,白娘子长白娘子短的自言自语。忽见无心回来了,她立刻转移了对象:``你干什么去了?''

无心笑着摇头:``没事。''

苏桃狐疑的看着他,认为他肯定有事。无心靠墙站到桌边,拿起一块花生糖扔进嘴里:``白娘子也连着好几天没吃东西了,今晚把它扔出去,让它自己找些耗子蛤蟆充饥吧!''

苏桃用毛巾一角擦了擦白琉璃的蛇嘴,然后很为难的仰头望向无心:``好容易才把它擦干净的\ldots{}\ldots{}''

无心笑了:``逗你玩呢!随便找点什么喂它都行,它是条蛇嘛,吃一顿管十天,很好养的。''

苏桃又道:``是不是也该给它喝点水?''

无心一屁股坐在床上,捏着白琉璃的蛇尾巴一抖:``娘子,一会儿给你喝点我的洗脚水。''

白琉璃被蛇身所束缚,不能大发淫威的报仇。尾巴一甩卷了上去,他把眼睛恢复成了黑豆形状,扭开脑袋不理无心。

无心心情很好,又问苏桃:``何必擦它呢!现在擦完,夜里它指不定又溜到哪里去了,天亮还是一身灰。''

苏桃知道无心总怕自己被蛇咬,所以没敢实说白琉璃夜里是在自己的被窝里睡觉。白琉璃心中有鬼,登时紧张的昂起了圆脑袋。对着墙壁静等片刻,他没等到无心的追问,才放了心,缓缓的向下盘成了一大堆。刚刚在无心的枕边盘稳当了,远方忽然响起一声尖叫,随即枪响带出玻璃破碎的声音,外面立刻就乱套了。

苏桃和白琉璃一起猛的向上一窜,无心伸出双手,把苏桃摁了下去:``没你的事。''

苏桃把眼睛睁得奇大,黑眼珠像两枚黑围棋子:``怎么了?''

无心笑道:``给你报仇嘛!放心,我只是吓了她们一下。她们连人都敢杀,不会被我吓死的。''

走廊里乱过一阵之后,重新恢复了安静。无心出门一问,得知是杜敢闯在房里开了一枪,把窗玻璃打碎了。为什么开枪?没人知道。

夜里关了门闭了灯,无心躺在床上刚要睡,不料上方忽然垂下一个披头散发的脑袋。苏桃像只惊弓之鸟一样,小声问道:``她们会不会查到我们啊?''

无心侧身躺在被窝里,很笃定的告诉她:``不可能。''

披头散发的脑袋缩上去了,苏桃躺回原位,颈窝里微微的有些凉,是白琉璃依偎着她。

到了翌日清晨,无心拎着一桶浆糊和一口袋笔墨,苏桃抱着一大卷裁好的彩纸,低眉顺眼的出现在了小丁猫面前。小丁猫的身边依旧是跟着两名女将。马秀红一如既往的沉着长脸,对于外界没什么反应;杜敢闯则是五官扭曲,仿佛昨夜大吃一惊之后,表情一直没恢复正常。

武卫国带着一干人马也来了。众人上吉普车的上吉普车,上卡车的上卡车,一路开出文县,直奔距离文县最近的李各庄。

李各庄是个大庄,距离公社也近。李各庄生产队的大队长也是通过造反夺的权,由于村里还有反对派,所以大队长的地位很不稳固。大队长本来也算联指一派,但是没得过省联指的任何好处,只得了个中看不中用的名义。忽闻省里来了人,大队长心里盘算了一番,没盘算出眉目,于是作出一张喜眉笑眼的面孔,带着队员们热情迎接了小丁猫等人。而小丁猫果然是不辜负队长的期望,刚一进村,就让武卫国手下的工人们架起步枪,把全村包围了。

对于大队长,小丁猫说的很清楚。省联指一直没有分出心思来关注农村地区的革命情况,如今有心思了,所以他是专程来支援大队长革命的。李各庄因为没有先进思想的领导,战争一直停留在冷兵器时代,如今骤然见了真枪,大队长兴奋的一阵阵眩晕,立刻集合了民兵连。民兵们有枪没子弹,手中的武器只有大刀长矛;好在反对派和他们彼此彼此,未见得谁更高明。往日李各庄内也是大打三六九,小打天天有,但是打得乱七八糟,不成体统,也不分胜负。如今省联指来了人,而且明确支持大队长的革命立场,李各庄便猝不及防的被卷进了一场腥风血雨中。

武卫国等人杀出经验了,随着大队长走遍全村,脚上一双翻毛大皮鞋不知踹破了多少户院门。无心和苏桃躲在大队部里,就听外面一阵一阵的起枪声。无心一张接一张的写大标语,苏桃哆哆嗦嗦的刷浆糊。两个十六七的半大男孩守在一旁,似乎还是钢厂车间里的学徒。每当无心和苏桃合作制出一批标语了,男孩们便捧着标语跑出去,顶着流弹四处张贴。

也就是一个小时的工夫,战斗结束。无心推开窗户吸了口气,吸了一鼻子的血腥味道。负责贴标语的一名男孩跑了回来,被无心抓住问道:``外边怎么样了?''

男孩兴奋得双眼放亮:``马上就要在打麦场开大会了,你们不去瞧瞧?''

无心看了苏桃一眼,随即摇头道:``我们还有笔墨纸张要收拾呢,没时间去。''

男孩用力挣开了他的手:``你们不去,我可去了!''

男孩刚跑,外面响起一阵欢声笑语,正是小丁猫等人和大队长走进了院子。无心和苏桃蹲在角落里,一时来不及撤退;而小丁猫进房之后坐在了大队长的椅子上,一边从马秀红手中接过一只拧开了的水壶,一边对大队长说道:``反动派都是纸老虎,不堪一击。''

大队长对着他一挑大拇指:``还是你们厉害,你们水平高,战斗力强。你们一开火,他们全完!''

小丁猫一笑:``扫帚不到,灰尘照例不会自己跑掉。我们不打,他们还不知道我们的本事。''

大队长又问:``丁同志啊,那批俘虏咱们怎么处置?''

小丁猫平淡的答道:``该劳改的劳改,该杀头的杀头。''

大队长咬牙切齿:``妈了个×的,杀头都不解我的恨,我恨不能油炸了他们!''

小丁猫竖起一根手指:``好主意,去找口锅,油炸几个。''

大队长张嘴露出了傻相:``丁同志,不行啊。''

小丁猫看他:``怎么不行?''

大队长坦诚的告诉他:``太费油了。''

小丁猫挑起两道眉毛:``没有油,总有水吧?我查过了,李各庄从去年到现在,只死了三个人。三个人啊,说明什么?说明你李各庄阶级斗争的盖子没揭开!我亲自带兵来支持你,你不行动,我就换别人来行动!''

小丁猫说完了话,起身便走。大队长愣了愣,连忙颠颠跟上。无心和苏桃缩在角落里,知道他们是开大会去了。

傍晚时分,大会结束,李各庄彻底成了联指的地盘,李各庄的民兵也将随时听候联指的调遣。晚饭是在大队部里吃的,为无心张贴标语的两个男孩,一个苍白着脸吃不下饭,另一个端着饭碗,一直在一块破砖上蹭鞋底。打麦场的土地都被鲜血浸透了,男孩穿了一双新球鞋,踩得满脚泥泞。

小丁猫没有即刻返回文县,他提前摸透了各生产队的状况,凡是能拉拢的,全部拉拢;不能拉拢的,就扶植一方消灭一方。实在是针插不进水泼不进的,他没办法,只好绕道。

在外面跑了一个礼拜,到了第八天中午,小丁猫决定打道回府。临行之前,他在武卫国等人的簇拥下站在一条土路上,对无心问道:``你看,我们这一趟的成绩如何?''

无心拎着一只空桶,桶里装着一大把毛笔:``我水平太低,说不明白。''

小丁猫微微探头:``苏桃说说。''

苏桃空了手,因为彩纸都被用光了。低头望着地面,她嗫嚅着答道:``我也说不好。''

从小丁猫的角度望过去,只能看到苏桃的额头和眉毛。苏桃的眉毛弯弯的,一根根眉毛紧贴着肉,匀匀称称的由里向外生长。小丁猫若有所思的出了神,认为从眉毛看,苏桃还是个处女。

``在村里住了一个礼拜。''小丁猫对着苏桃开了口:``不习惯吧?''

苏桃像只蚊子似的,低着头哼哼道:``还行。''

小丁猫又道:``你和无心分工协作,效率还挺高。''

苏桃从鼻子里``嗡''了一声。

小丁猫微微弯着腰,一双眼睛从苏桃移向了无心。他从无心的鼻梁开始往下看,看到最后,伸手摸了摸无心的脸:``你是挺嫩,年轻嘛,哈哈。''

无心歪着脑袋,想和小丁猫对视。然而小丁猫避开了他的视线,直起腰开始张罗着上车了。

当晚,他们回了文县指挥部。指挥部里来了新人,名叫李作诚,本是个退伍兵,如今投在小丁猫麾下,和杜敢闯齐头并进,堪称雌雄双煞,只是头脑不如杜敢闯,导致在联指内部地位不高。小丁猫跑到文县另起炉灶,他立刻就在保定拉了一批人马,充当小丁猫的嫡系队伍,同时又把能弄到的枪支弹药装上车皮,在杜敢闯的指挥下一并运来了文县。

小丁猫和李作诚吃了一顿晚饭,然后李作诚立刻就和田小蕊相好上了。田小蕊对顾基是哀其不幸怒其不争,哀怒久了,索性放弃。李作诚人高马大的,看背影有点顾基的意思,论本质比顾基威武了万倍。

得知李作诚和田小蕊上大街晒月亮去了,小丁猫回了指挥部宿舍,让马秀红把无心和苏桃叫了来。摊开一张雪白的宣纸,他让无心给自己抄一首毛主席诗词。

端着一杯苦丁茶在地上走来走去,他唉声叹气,因为肠胃造反,已经连续三天只进不出。心思从便秘问题转移到女人身上,他青春年少的身体忽然有些亢奋。

为了打消自己的亢奋,他让马秀红给自己念了一段报纸上的评论文章。评论文章很不合他的心意,他抬手问道:``什么报纸?''

马秀红答道:``燕山日报。''

他又问:``文县的?''

马秀红一点头:``是。''

他一点头:``明天派人封了它!''

然后他留意的看了马秀红一眼,又想了想杜敢闯。心头欲火渐渐平息了,他又成了傲雪寒梅。转身望向了无心和苏桃,他想自己面对马杜二人太久了,精神兴许太受压抑,居然见了漂亮面孔就发痴。其实宣传队里也有几个美的,只要他一勾手指,她们自会送上门来,不过\ldots{}\ldots{}

无心的声音打断了他的思绪:``丁同志,我写完了。''

小丁猫不置可否的一点头:``苏桃天天跟着你刷浆糊,可惜了。可以让她多学习多锻炼,以后给我当个秘书。''

无心向他笑了一下:``一个小丫头,能把浆糊刷好就不错了。''

小丁猫笑问:``你倒是把她看得很紧,可她自己愿意吗?''

无心把毛笔插进一杯清水里涮了涮:``我怎么把她带出来的,就得怎么把她带回去。她的意见,不算数。''

小丁猫伸手一指他:``专制!''

无心把毛笔涮干净了,没回答。小丁猫对他和苏桃一直不算坏,然而好的阴气森森,还不如光明正大的坏。

小丁猫喝了一口苦丁茶,正要开口,不料窗外忽然火光闪烁,随即发出一声惊天动地的巨响。天花板上的电灯晃了几晃,楼上楼下一起爆发了尖锐的惊叫。小丁猫被震得耳鸣片刻,眼前光影闪烁,就见校园对面的一片荒弃厂房中腾起火光,竟是中了炸弹的光景。强定心神扶了扶眼镜,他走到桌前刚要抄起电话,外面又是一声大爆炸,依旧是炸在了厂房之中。房门一开,杜敢闯冲进来喊道:``红总开炮了!''

小丁猫猛然回头:``红总有炮?''

杜敢闯没理他,转身跑出去大声呼喊,命令全楼人员迅速下楼隐蔽。未等小丁猫等人往楼下跑,武卫国气喘吁吁的冲了上来:``大家别慌!最新消息,红总只有两发炮弹,刚才全打偏了!''

无心趁乱,和苏桃一起溜回了宿舍。

苏桃怕人胜过怕死。进了小屋关了门,她从床底下捧出了白琉璃。无心不让苏桃带白琉璃出远门,所以他独自在小屋里混了一个礼拜。骤然见到无心和苏桃回来了,他十分快乐,一尾巴缠上无心的脖子,脑袋则是搭上了苏桃的肩膀。无心和苏桃受了他的束缚,不得不面对面紧贴着站立。苏桃正对着无心,想到他敢对杜敢闯使坏,他敢对小丁猫反驳,苏桃把额头向前一抵,抵上了无心的胸膛。

无心松松的拥抱了她一下,又抬手轻轻一拍她的后脑勺:``桃桃,他们打起来才好。他们忙着打仗,就不会注意我们了。''

\chapter{恐惧}

无心和苏桃蹲在指挥部后的阴暗处,拢了一堆火烤红薯。红薯是粮食作物,一斤粮票能换三斤红薯。无心手里有的是粮票,于是上午带着苏桃跑了一趟粮站,冒着流弹的危险抱回了一堆奇形怪状、并且已经在地窖里过了一冬的的丑红薯。

大饭盒架在火上,红薯放在饭盒里。两人烟熏火燎的相对蹲着,抱着膝盖偷偷的快乐。苏桃正在长身体,一天给几顿吃几顿,而且还带着孩子心性,烤红薯三个字对她来讲,正是又吃又玩。

无心不怕烫,挑了一个小红薯掰开了,里面热气腾腾的露出红瓤。撅嘴吹开了一层热气,他把大的一块递给苏桃:``尝尝,甜不甜?''苏桃双手捧着红薯,因为太烫,所以一口咬下去,嘴里咝咝哈哈的又吸气又吹气:``甜,像糖似的。''

无心也咬了一口,红薯软软的粘上他的舌头,烫得他紧紧一闭眼睛。苏桃见了,连忙放下手里的红薯,拿了水壶要给他喝。而无心未等喝水,就听不远处起了``砰砰砰''的响声。觅声望去,他看到了楼后的一排平房。平房是一中先前的体育器材室,为了防盗,窗户外面都焊了铁栅栏。隔着栅栏和玻璃窗,无心看到了顾基的脸。

顾基已经被关了半个月了,一天只给一顿饭,毒打倒是管够,一天至少两三顿,偶尔还加夜宵。他本来是人高马大的架子,如今就剩了架子,像副大号骷髅似的,佝偻在暗沉沉的房间里敲窗户。

无心隐隐明白了他的意思。从大饭盒里挑出最大的一只红薯,他起身走向了平房。顾基所拍的玻璃窗破了一角,无心抬手把红薯从窗洞里塞了进去。顾基一把接住红薯,双手捧着低下头,``吭''的张嘴就是一大口。

三嚼两嚼之后,他带着哭相抬起头,哀哀的说道:``我想见小丁猫同志\ldots{}\ldots{}我早就和顾明堂划清界限了,我都半年多没和他说话了,我是冤枉的\ldots{}\ldots{}无心,我知道你是好人,你从来没欺负过谁。行行好帮帮忙,你替我向小丁猫同志传个话吧,我实在是熬不住了,他们天天打我\ldots{}\ldots{}老陈也不露面了\ldots{}\ldots{}''

话说到此,他含着一点红薯,呜呜的哭出了声。细脖子挑着个大脑袋,他瘦出了鸡蛋大的喉结。无心拍了拍手上的黑灰,有点不知如何是好,只能是不置可否的点了点头。

无心回到火堆前蹲下,苏桃小声问道:``一个够他吃吗?''无心勉强笑了一下:``再给就没你的份了。''苏桃托着一块烤红薯,低声说道:``要是被关的是黑背,我就不管了。''

顾基是个狐假虎威的软蛋,苏桃没亲眼见他干过什么大坏事,所以觉得他和自己是同命相怜;陈部长就不一样了,苏桃在陈部长面前永远是低眉顺眼的垂着头,目光射在地上,带着极度的恐惧和嫌恶。

三斤红薯全烤熟了,无心又给了顾基一只,但是始终没有多说什么——顾基的母亲前天被联指处决了,尸体吊在街边的大树上,专为震慑和报复顾明堂。因为顾明堂的驾驶技术是极其高明,能开着卡车夜行十八弯的山路,秘密的把一门迫击炮运到红总指挥部。

他是小军阀的私生子,或许小军阀根本就对他的儿子身份有所怀疑。小时候,他倒也过了几天少爷日子,不过少爷日子太久远了,他已经记不太清。及至小军阀在四九年时带着一大票家人逃去了香港,他和母亲孤零零的留在文县,终于意识到了小军阀有多害人。

他是年初时被武卫国抓进钢厂保卫处的,起初还想好好做人,两个月后意识到好好做人是天方夜谭。趁着自己胳膊腿儿还听使唤,他一狠心,跳楼逃了。顾明堂为保卫处里的其他犯人做了个坏榜样,于是单杀了他的老婆还不够。他的独生儿子已是落网之鱼,武卫国灵机一动,把顾明堂的老娘也拖出了家门。在钢厂内部的大批斗会上,老太太被人用烙铁烙死了。

无心认为顾基不是个坚强人,所以不肯再刺激他。眼看他狼吞虎咽的只顾着吃红薯,他带着苏桃悄悄撤退了。

红总不知道从哪里弄来了枪支弹药,双方的战斗立时先进了许多。街上的热闹劲儿明显是下降了不少,两方的革命小将光顾着厮杀,已经没有心思四处游行。无心没什么地方可去,只好带着苏桃回楼。

一楼的大教室里,一队女声正在练习合唱。无心从门口向内溜了一眼,见小丁猫带着李作诚和武卫国,正坐在合唱队前观看。李作诚和武卫国都是三十来岁的高大汉子,把小丁猫衬托成了白脸小男孩。但高大汉子左右簇拥着小男孩,小男孩气定神闲的用手指在椅子扶手上打着拍子,一双眼睛躲在眼镜片后,眼神堪称苍老,老的几乎无欲无求了。

无心刚要走,小丁猫目光如电,一眼叨住了他:``无心?''无心把脑袋伸回了教室,对着小丁猫一点头。小丁猫又问:``有事?''无心一团和气的告诉他:``我找李萌萌,问她有没有新任务给我。''小丁猫歪着脑袋向前问:``小蕊啊,看见李萌萌了吗?''

领唱的田小蕊向前迈了一步:``李萌萌和陈部长刚出去了。''小丁猫一点头,然后对着无心一招手:``看来是没什么新任务,进来听听歌吧,我们的宣传队,水平倒是真不一般。苏桃呢?让她也来。总拎着个浆糊桶到处跑,有什么前途?''

无心见自己是逃不过了,只好领着苏桃进了门。而小丁猫仿佛是兴致不错,笑模笑样的又道:``会工作,也要会娱乐。劳逸结合,才能提高效率嘛!再有一点,就是要沉稳、镇定。不管风吹浪打,胜似闲庭信步。红总有军区支持,我们也有省委支持。好戏在后头,大家慢慢看。''

无心本来不想答茬,但是犹豫了一下,他在小丁猫身边弯了腰:``丁同志,我刚才在楼后,见到了顾基。顾基说他想见你一面,还说他是冤枉的。''

小丁猫望着前方一大排十七八岁的合唱队员,开口笑道:``哟,你还学会替人求情了?''无心直起了腰:``我没面子替他求情,就是传句话而已。''小丁猫嘿嘿嘿的笑了一气,然而拍拍巴掌解散了合唱队,当真命人去把顾基带了过来。

顾基是被人拖进房中的,从头到脚几乎没了好地方,鞋也丢了,脚踝脚趾全都红肿透亮。身上的衬衫本来是白色的,如今被皮鞭抽出一道一道的口子,口子里面鲜红紫黑,是深深浅浅的血痂。抬头一见了小丁猫,他登时就哭了。及至两边人松了手,他跪在地上,开始捶胸顿足的嚎啕。

小丁猫点了一根烟,对着他吐了个烟圈,顺便向他通知了顾家人的死讯。顾基听后,愣了一下,随即继续大哭,嘴里乱叫着妈妈奶奶,是个瘦骨嶙峋的大号孤儿。

小丁猫不为所动,一边抽烟一边说道:``我可以给你一个戴罪立功的机会。罪你是戴定了,你父亲帮着红总运送迫击炮,要炮轰联指指挥部,用心何其险恶,手段何其狠毒。至于立不立功,则是要看你个人的表现。''

顾基睁着一双泪眼望向小丁猫,人仿佛都傻了,抽抽搭搭的只说:``我立功,我一定立功。我划清界限,我不是他儿子\ldots{}\ldots{}我和他坚决斗争,斗争到底\ldots{}\ldots{}''小丁猫居高临下的望着他:``让你杀了顾明堂,你敢不敢?''顾基茫然的流着眼泪:``敢\ldots{}\ldots{}我敢\ldots{}\ldots{}你饶了我,我什么都敢\ldots{}\ldots{}''

小丁猫笑了一声,命人把顾基架走了。而顾基在起身之前,还匍匐着给小丁猫磕了一个响头。小丁猫是他的救世主,小丁猫一手攥着他的生死。他心里没有恨,丝毫没有。在救世主面前是不能讲道理的,只有忏悔,只有感恩。被人像拖死狗一样拖出大教室,他知道自己又活了,小丁猫一句话,抵得上自己一条命。

顾基一走,小丁猫也不听合唱了。带着众人上二楼回了办公室,他让苏桃和无心帮着马秀红抄文件。一张大办公桌横在屋子里,马秀红坐在一端,无心和苏桃坐在另一端,三人低着头,闷声不响的写字。小丁猫则是把杜敢闯也叫了来。几个人在屋子一角围成一圈低声交谈,无心竖起耳朵,依稀只听到``长安县''``军械库''``民兵连''等词。而交谈到了最后,杜敢闯和李作诚就一起走了。

当天晚上,杜敢闯和李作诚带领上千的队伍偷偷出了文县,一路和各村庄的民兵会合,直奔长安县的解放军驻地,抢军火去了。

苏桃在办公室里抄了一下午文件,被小丁猫拍了无数次肩膀和后脑勺,一拍一哆嗦。后来小丁猫顶着马秀红的冷眼,弯腰趴在苏桃旁边的桌面上,近距离的关怀问道:``累不累?''苏桃在他满嘴的苦丁茶气中寒毛直竖:``不累。''小丁猫笑了:``不累的话,晚上再来继续抄?''苏桃愣了愣:``累。''

无心抬了头:``丁同志,离我爱人远点儿。''此言一出,马秀红从鼻孔中呼出两道快意的冷气。小丁猫则是讶然:``爱人?''无心义正词严的点头答道:``没错,迟早的事。等她岁数一到,我们两个就去登记。''小丁猫笑了:``信不信我让你爱人变成寡妇?''

无心不出声了,低头继续写字,显然是被小丁猫镇了住。而小丁猫伸长手臂,劈头盖脸的摸了他一把,嘴里哈哈哈的笑了一大串。笑声未歇,窗外忽然光芒一闪,随即起了一声大爆炸。屋里众人吓了一跳,小丁猫随即直起腰怒道:``他妈的怎么又开了炮?不是说红总没有炮弹了吗?''

无心一把扯起苏桃,大喊一声转身就往外冲。苏桃吓傻了,握着一支圆珠笔没头没脑的跟着他逃。他两个一有动作,小丁猫和马秀红也清醒了。一拉抽屉拿出一把手枪,小丁猫刚要招呼马秀红,不料马秀红动作更快,连推带抱的把他拥了出去。

有了上次的炮击经验,此时楼内的情形尚算有序,正好没到洗漱休息的时间,所以满楼的男男女女衣冠整齐,说跑就跑。小丁猫正在盘算如何避难,冷不防又一枚炮弹从天而降,分毫不差的炸中了楼后的体育器材室。火光冲天而起,楼内的气氛立刻紧张到了十分。

联指的精兵悍将全去了长安县,如今坐镇的就只有武卫国和陈部长。陈部长近来和李萌萌勾勾搭搭,又时常是不知所踪。第三枚炮弹落到了一条街外,爆出了漫天的火光硝烟。所有人都跑进校园里了,无心和苏桃落了后——他们忙着上了一躺三楼,回房用书包装出了他们的粮票、钞票以及正在打瞌睡的白琉璃。

小丁猫下了往钢厂撤退的命令,然后自己坐上吉普车飞快的逃了。无心和苏桃随着人流往前跑。跑着跑着,身边的一个小姑娘猛一挺身,紧接着像截木头似的倒了地。无心没想到此时街上会有流弹,连忙带着苏桃靠了边。路边一面凹进一块的砖墙成了他们的掩体。

无心搂着苏桃极力的缩成一团。街是小街,没有路灯,无心把苏桃团成了一团,把她在怀里抱成了小女孩小女婴。苏桃的呼吸紊乱的扑在他的脖子上,他听见苏桃问自己:``无心,红总会打过来杀人吗?''无心一下一下拍着她的手臂:``不好说,要看武卫国他们怎么反击了。''

苏桃是很容易想到死的,怕到受不了的时候,她的思维往往就直接跳到一个``死''字上去。抬手搂住无心的脖子,她很认命的闭了眼睛。

与此同时,白琉璃轻飘飘的出现在了无心眼前。悬在夜空中环顾四周,他仿佛是懒得搭理无心,只向前做了一个手势。无心领会了,拉起苏桃起身就跑。跑着跑着,他听到白琉璃告诉自己:``十字路口向左拐。''

他果然左拐了,左拐了十分钟后,红总的五辆卡车在炮火的掩护下,一路长驱直入,经过了十字路口。

\chapter{青云山}

苏桃知道无心和自己一样,都是初来乍到的外地人,除了坐落着副食品商店的主要大街走过几遍,其余路线一无所知。一手死死的抱着书包,她只见无心如有神助一般,跑着跑着就拐了弯,拐得毫无预兆而又次次正确,仿佛有人给他引路。

最后他忽然刹住了脚步,领着苏桃冲进了一条漆黑的小胡同。胡同两边的人家都是大门紧闭,院子里一丝一毫的光亮也没有,生怕自家与众不同,会招来造反派的枪弹。无心搂着苏桃,在伸手不见五指的旮旯里又蹲下了。

白琉璃悬在夜空中,周身隐隐笼罩着一层浅色光晕,像轮大月亮似的看热闹。街上有人开枪了,有人还击了。红总的人跑来跑去,联指的人不甘示弱,你来我往的也露了头。再远一点的路口处堆起了沙袋,一个十三四岁的小男孩趴在沙袋上,仿佛是头顶心中了弹,脑袋整个的开了花。

有人猫腰抓住小男孩的脚,把尸首拖到了路边;重机枪架上了沙袋,还是半大孩子的新战士们仿佛是第一次摸枪,笨手笨脚的摆弄着弹夹。沙袋前方扔着一把步枪,还是当年日本鬼子留下的三八大盖。

一辆架着机关枪的大卡车缓缓驶向路口。沙袋后方的一个愣头青不声不响的推动了重机枪的扳机。重机枪失控似的喷出一串火舌,副射手猝不及防,吓得``嗷''一嗓子。

白琉璃在大兴安岭中看了几十年的花和雪,精神生活淡出了鸟。后来好容易等来了一个无心做伴,两人又是话不投机半句多。如今望着满街流星似的枪火,他高兴的手舞足蹈。盘腿飘在夜空中,他翘起嘴角扬起眉毛,两只手不停的在膝盖上拍。

无心张着嘴仰头看他,认为白琉璃趣味极低,不可救药——前一阵子在南开大学遇到两名女红卫兵对骂对打的时候,白琉璃也是乐得前仰后合。苏桃见无心呆呆的望天,便也跟着一起仰了脸,可是只看到了几个星星。

街上的枪声响了一夜,将近到了凌晨时分,白琉璃缓缓降到了无心面前。作为一只强大的游魂,他的鬼影在无心眼中,已经清晰到了纤毫毕现的程度。

苏桃迷迷糊糊的闭着眼睛,莫名的感觉到了一丝寒意。而白琉璃对无心说道:``外面已经停火了,要走快走。''不等无心回答,他钻回了小白蛇的身体。做鬼很自在,做蛇就不自在了;很费力的从书包缝隙里伸出圆脑袋,他总是调动不清从头到尾的一长串蛇骨头。

苏桃睁了眼,把白琉璃的脑袋掖回了书包里。混混沌沌的随着无心站起身,她揉着眼睛环顾四周,发现天边已经隐隐透了光明。``天快亮了。''她小声问无心:``我们接下来往哪里去啊?''

无心摸着脑袋,知道联指的人是撤到钢厂里去了,可是他和苏桃都不知道钢厂的具体位置,想紧随大部队都不能够。想要趁机脱离联指,也不可能,因为文县火车站早被联指的人马封锁了,文县目前已经成了个半瘫痪半封闭的状态。

一手把苏桃拉到身后,他沿着墙根慢慢的往外走。蹑手蹑脚的出了胡同上了大街,正是心惊胆战之时,远方乱七八糟的跑来一队人,领头一位头破血流,正是背着步枪的武卫国。武卫国猛的见了他们,也是一愣,随即脚步不停的一挥手:``走走走!''

无心带着苏桃一路奔跑追上了他:``现在打的怎么样了?''武卫国显然是累极了,喘息着拖起两条腿,根本无暇理睬无心。穿过两条大街之后,他们面前出现了一座大铁门,正是钢厂到了。

钢厂院内一片混乱,小丁猫一手叉腰,一手夹着烟卷,正在侧耳倾听陈部长讲话。细细长长的马秀红拄着一杆细细长长的步枪,横眉冷目的守在一旁。武卫国气喘吁吁的冲到小丁猫面前,极力控制着气息说道:``建设大街失守了,他们火力太猛,我们顶不住!''

小丁猫吸了一口烟,然后平平淡淡的说道:``打得赢就打,打不赢就走。马上集合全部人员和车,我们撤出文县,上青云山。''武卫国心里服他,而且知道他有后手,杜敢闯和李作诚带着队伍在外头,说不定什么时候就能卷土而归。一言不发的扭头便走,他调集了全部人员,开始着手进行大撤退。

无心和苏桃挤上了一辆破卡车。卡车刚要哼哼哧哧的开动,一名青年发了疯似的跑进大院嚷道:``田小蕊她们让红总活捉了!''小丁猫不为所动的上了吉普车,留下陈部长站在原地吼叫:``谁让她们出去打仗的?她们会打个屁呀!''

想到美丽的田小蕊被俘虏了,陈部长对她死了的心,痛得当场复活:``你们傻啊,让她们往前边乱跑?男人打仗,一帮骚×跟着凑什么热闹?''青年被他吼傻了,怔怔的答道:``田小蕊说她会开枪,能顶一阵子。''

陈部长心里明镜似的,所以哑着嗓子吼得十分痛楚:``她会开她妈了个×!''李萌萌站在一旁,知道陈部长见了漂亮的就害单相思,故而伸手狠狠一拽他的袖子:``别吵吵了,赶紧上车!''陈部长使劲一挥手:``你给我滚一边儿去!''

青云山位于文县与长安县之间,既不算雄,也不算奇,但是山清水秀的挺美。几十年前,山后开过一座金矿,据说矿主中有一位就是顾基的爷爷。

金矿很小,挖了几年就山穷水尽了,矿场遗迹早被草木遮盖。山顶上还有一座道观,名叫青云观,旧社会时乃是一处豪华风雅的场所,按照资产判断,住持道长们全可算作是大地主。如今道士们早被革命小将撵下山还俗了,青云观就成了一处空壳子。

联指的队伍仓皇离开文县,一路直奔青云山避难。汽车停在山下,众人排着队伍往山上走,武卫国一边走一边留意身边地形,设下关卡。山上的道观非得用人和钱供着,才能体面;一旦没人管理了,就显出一副衰败的荒野相,幸而房屋还算结实,足能遮风挡雨。

苏桃跑了一夜一天,没吃没喝,实在是支持不住了。无心背着她往山上走,起初一段路还走得很稳,及至经过了第一道山门,苏桃发现他的身体在微微的颤,便挣扎着要下地:``无心,你是不是也累坏了?''无心一晃肩膀,两只手托住了她的腿:``我不累,你趴着吧。''苏桃小声说道:``你都打颤了。''无心侧过脸:``真不累,累了我就不背你了。''

进入道观的青石板路已经残破不堪,路边的野草生长得蓬蓬勃勃,披头散发的盖住了路面。道观之内也是了无生机,大殿内的神像全被打碎了,也分不清神仙们谁是谁。马秀红擦出一张桌子让小丁猫坐了,武卫国走到小丁猫身边说道:``你说得对!青云山的确是易守难攻。只要粮食充足,红总他们一辈子也别想打上来!''

小丁猫的娃娃脸看起来苍白松弛。抬手扶了扶眼镜,他疲惫的答道:``我们也不会在山上守一辈子。马上派个通信员下山去长安县,联系杜敢闯和李作诚,让他们相机而动,自行制定作战计划。''

武卫国答应一声,自去安排。陈部长为了田小蕊心痛一路,此刻刚刚有点过了劲,便张罗着埋锅造饭。正张罗得头头是道,他心中一紧,忽然想起了自己的寡妇妈——自己是跑了,妈还在县里呢!自己在红总的黑名单上肯定是有一号的,文县落到红总手里,妈会不会受连累?

陈部长的黑脸颜色不定。背着双手来回的走了两步,他有点慌,又想起自己的妈平时只顾着攒钱,不给自己好吃不给自己好喝,自己出来干革命,回家还要背着``瞎胡闹''的罪名,被她拿着笤帚疙瘩追着打。抽着鼻子嗅了嗅饭香,他咽了口唾沫,硬着心肠想:``革命免不了要有牺牲,我还是先吃饭吧!''

联指的人马东倒西歪,一个个全累得直不起腰。小丁猫坐在供桌上望着部下们,感觉此情此景着实狼狈。而无心从书包里取出大饭盒,满满的盛了一饭盒米饭,又要了一些咸菜丝,随即带着苏桃往后方僻静处走去。

苏桃和他手拉着手,有点缩头缩脑:``后面还有房子哪?''无心笑了一下:``走着看吧,前头太乱。''苏桃跟着他走,一路偶尔看到五彩斑斓的残破神像,就感觉怪瘆得慌。

末了他们进了一处小院,院子里有个大花坛,里面野花野草生得密密匝匝,小院四周还带着一道精致的游廊。房门洞开着,玻璃全碎了,可见房内空空荡荡,只有一张大罗汉床。可能是没人意识到红木罗汉床的价值,也因为大罗汉床太沉重太结实了,除了床围子被刀砍斧剁出了累累伤痕之外,罗汉床本身居然还算完整。

无心和苏桃坐在游廊低矮的栏杆上,分食一饭盒的米饭和咸菜丝。苏桃吃了几口,抬头说道:``幸亏把饭盒也带上了。山里没食堂,它就是我们两个的饭碗和水杯了。''然后她探头细看无心的面孔:``你怎么了?不高兴了?''

无心摇头笑了一下:``我是看到道观的样子太凄惨,又想起它当年应该也是兴盛过的,就有些\ldots{}\ldots{}''他欲言又止的不说了。苏桃明白他的意思,心中也是一阵戚戚然。

到了夜里,众人各找地方安身,无心和苏桃就悄悄睡在了房内的大罗汉床上。床上什么都没有,无心伸了胳膊给苏桃当枕头。苏桃轻轻的枕了他的手臂,脖子紧张着,总怕压了他。无心侧身转向她,伸手一摁她的脑袋:``桃桃,睡吧。''

苏桃闭了眼睛,渐渐的枕踏实了。面前忽然有了风声,她睁眼一瞧,是白琉璃游出书包,长条条的伸在了两人之间。一个圆脑袋搭上无心的手臂,他顺便又贴上了苏桃的鼻尖。

苏桃摸了摸他的后背,无心也弹了弹他的脑袋。然后两个人一起安心的闭了眼睛,只有白琉璃依旧圆睁二目——他是条蛇,没有眼皮。

\chapter{避世之心}

无心和苏桃在红木罗汉床上对付了一夜,床板坚硬,又没有被褥,导致他们虽然疲惫至极,一夜过后却是全没有睡懒觉的心思。清晨两人到道观前头,从井里摇上一桶水洗漱了,因见早饭还没影子,就又回了后方小院。苏桃坐在游廊栏杆上,用手指梳头发编辫子;无心则是回到房内,从床下捡到了一本破烂经书。经书被撕过也被烧过,没头没尾四边焦黑,想必是破四旧活动中的幸存者。无心百无聊赖,一边把白琉璃抓过来横撂在大腿上,一边心不在焉的浏览经书,一看之下,发现它还是本佛经,纸质泛黄,竖版印刷的大黑字疏疏落落,看着倒是不累眼睛。

苏桃编好辫子,自得其乐的走到院内花坛前摘花弄草。而无心在房内忽然一拍腿上的白琉璃,小声惊道:``哎呀!我怎么一直就没想到?''

白琉璃吓了一跳,登时昂起了脑袋看他。

无心转身把薄薄的残经塞进了书包里,胳膊肘上有了轻微的触感,是白琉璃在用脑袋撞他。他大惊失色的说了半截话,吊起了白琉璃的好奇心。而无心回头将一根手指竖到唇边,``嘘''了一声:``我好像知道我是什么了,有空告诉你,你也帮我参谋参谋,看我想的对不对。''

白琉璃知道现在不是他长篇大论的时候,故而通情达理的一吐信子,表示同意。无心的底细他已经全知道了,几年前无心跑到地堡之时,还曾万念俱灰的闹过一阵子自杀,当然是过程残酷,结局未遂。

无心把白琉璃扯开扔回了罗汉床,然后拿起饭盒站起身:``我去看看饭熟了没有,你留下来保护桃桃。''

白琉璃趴在苏桃躺了一夜的位置上,卷起尾巴一拍床板,意思是知道了。

无心端着大饭盒跑到了前院,正好赶上小丁猫在调兵遣将。山下隔三差五的会有枪声响起,据说是红总的前锋队已经到了。

武卫国拎着枪,带着一队人下山了,陈部长黑着脸,负责道观四周的布防。小丁猫坐在一块大石头上,忽见无心来了,就向他招了招手。

无心走到了他面前,虽然对他的战争毫无兴趣,但是一言不发也不合人情。低头望着自己的空饭盒,他忍饥挨饿的开了口:``仗\ldots{}\ldots{}不好打吧?''

小丁猫倒是一团和气:``是不大好打,队伍里有内奸,透了我们的动向给红总,让红总搞了一次大偷袭。幸好青云山的地势很不错,是一座天然的堡垒。''

无心点了点头:``嗯,是。''

小丁猫歪着脑袋抬眼看他:``如果换了你是我,你也打不赢。''

无心迎着他的目光,同时发现他垂下眼帘,并不肯和自己对视。

``丁同志。''无心也是一团和气:``有时候听你说话,感觉你好像在很早之前就认识我。''

小丁猫似笑非笑的叹了一声:``相逢何必曾相识,忙你的去吧!''

无心狐疑的答应一声,转身走开。小丁猫的确是个人,从头到脚都没有一丝鬼气,没有借尸还魂的可能。可无心把前因后果翻来覆去的想了一遍,末了一阵心虚,暗想小丁猫对自己堪称宽容,而宽容的目的,仿佛只是为了留下自己。留下自己有什么用?自己除了会抄大字报之外,也没什么用,真打仗了,也只能做一名自身难保的看客。

无心没想明白,端着一饭盒大米饭走了。

苏桃坐在游廊出口的台阶上,面前摆着一大束马蹄莲的绿叶子,正在埋头编花篮。白琉璃盘在一旁,脖子上套着一只小小的花环。忽见无心回来了,她仰脸一笑,又高举了手里的小花篮,意思是要给无心看。无心十分捧场,当即托着热饭盒对苏桃的手艺进行了夸奖。他的热烈赞美超出了苏桃的预期,导致她十分脸红。几乎忸怩了。

两人挤着坐在台阶上,一边吃饭一边说闲话。闲话没说两句,山下忽然起了轰隆隆的炮响。前方传来了尖锐的哨声,正是紧急集合的号令。苏桃匆忙盖好饭盒,又用两条长长的马蹄莲叶子把饭盒捆好。无心则是进房拎出书包。一边弯腰把白琉璃捞起来塞进书包里,他一边回头又向房内望了一眼。望过之后,他麻木的扯起苏桃,向院外跑去了。

他们到达集合地点之时,陈部长正在拿着电池喇叭喊话。原来杜敢闯李作诚已经从长安县凯旋而归,如今正在炮轰山下的红总前锋队。而山上众人也可以在武卫国等人的掩护下,开始下山了。

话音落下,小丁猫带了头,匆匆的踏上了下山的石板路,给他开路的人,却是顾基。

走在最前方的人,很有遭到流弹的危险,尤其顾基又是个门板似的大个子,越发类似盾牌。但是小丁猫让他开路,他就开路。鼻青脸肿一瘸一拐的走在山路上,他的两条手臂垂着不敢动,因为被人悬在房梁上长久的吊过,关节筋骨都受了伤害。没想到小丁猫还记得他,还肯用他,他幸福得将要落泪了。

一行人在山路上排开一字长蛇阵,因为次序也没有一定之规,所以无心和苏桃走在了最后。走着走着,苏桃忽然低声说道:``要是能留在山上就好了。''

无心转头看她:``山上要什么没什么,好在哪里?''

苏桃答道:``好在没人管我们。''

无心笑道:``也没饭吃啊。''

苏桃一想也是,就拉着无心的手不吭声了。晚春的太阳晒热了她的头皮,她微微出了点汗。很留恋的向后回头,她忍不住又道:``无心,你记住路了吗?以后要是有人抓我们,我们就逃到这座山里来。山上有房子,我们就不会冻死;没有大米,我们可以挖野菜吃。''

她素来像个猫似的不多言不多语,如今忽然有板有眼说了一大串,惹得无心忍不住看了她一眼,结果发现她是一本正经,并非玩笑。

用力攥了攥苏桃的手,他知道恐惧的阴影始终笼罩着她。苏桃不挑吃不挑穿,人生中唯一的要求就是不被人抓。

而苏桃认认真真的又道:``我们两个再加上白娘子,住到山里也不会闷的。''

无心听她越说越真了,一时不知应该怎么回答。顺着她说,怕她走火入魔的真会小隐隐于山;逆着她说,又不忍心。低下头走了一段路,他总算找到了新话题:``回去之后想着买双新鞋。''

苏桃脚上的解放鞋偏大,穿上后非得把鞋带勒紧了才成。苏桃听了,自己提着裤腿向下一看,就见一双鞋又大又扁,衬得脚踝十分之细,就抬头对着无心笑道:``好像一双鸭子脚。''

然后她拖着一双大鞋,啪嗒啪嗒的和无心继续赶路了。

无心下山之时,红总的前锋队已经被李作诚的队伍轰出了几十里。杜敢闯和李作诚在长安县干得特别顺利,一边派出精兵和留守在长安县的红总人员对战,一边号召了无数民兵冲击军械库。没人敢向革命群众开枪,换了军队首长亲自出场,也是一样。本地的首长曾经抵挡过一次红总的冲击,基本算是成功,所以面对联指故技重施,派了一群膀大腰圆的士兵组成人墙。不料联指使用人海战术,三下五除二的就把人墙冲垮了。一拥而入进了军械库,联指的人抢,跟着联指一起来的民兵也抢。所有人都抢红了眼,甚至还窝里反的干了一仗。

联指的人抢完了,红总的人卷土重来。眼看联指的人撤走了,他们进去接着抢。但是他们的运气不如联指,因为长安县附近的村民闻讯而来,打着造反派的大旗也跟着抢。抢完之后村民们没往远走,一出县城就打起来了。红总队伍慢了一步,被炮火困在了长安县内;联指队伍则是先人一步,一路杀回了文县。

文县没有城墙,城里城外可以摆开阵势随便开炮。红总还未在文县立稳脚跟,就被联指猛烈的炮火轰了个东倒西歪。战斗持续了整整一夜,到了天明时分,红总撤出文县,联指又回了来。

小丁猫运筹帷幄之中,根本不上前线,所以联指几员大将全都烟熏火燎的没人样了,只有他依然干干净净。安安然然的回到了一中指挥部,他发现一中大楼竟是安然无恙,显然红总还没来得及火烧联指的总部。

回到二楼办公室里,他从马秀红手中接过一杯苦丁茶。刚刚啜饮了一小口,陈部长敲门进来了。灰头土脸的站在办公桌前,他低声说道:``我们刚刚捉到了几个红总的活口,得知田小蕊等五名同志,在被俘的第二天,就\ldots{}\ldots{}壮烈牺牲了。''

小丁猫似乎是很慨叹,拧着眉毛呼出了一口气:``按照烈士的规格,好好安葬了她们。''

陈部长继续说道:``活口里面,有顾明堂一个。''

小丁猫一挑眉毛:``把顾明堂先关起来。''

陈部长怀着哀恸的心情,在确定自己的寡妇妈躲在地窖里逃过了一劫之后,便带着一队兄弟,扛着步枪和铁锹,押着三名战俘往城边走。战俘之一是个十八岁的毛头小子,陈部长在县中读高三的时候,毛头小子正好读高二,两人还在一起打过篮球。毛头小子把陈部长等人带到了城边几座新坟前,喃喃的说道:``就埋在这儿了。''

陈部长一愣:``你们这么好心,还给她们立了墓碑?''

毛头小子连连摇头:``不是给她们立的,她们是——''

陈部长此刻也看清了墓碑上的字样。转身用一把刺刀抵上毛头小子的眼珠,他面目狰狞的问道:``说!她们到底是怎么死的?''

毛头小子吓得一动都不敢动:``是、是我们陈司令下的命令,我可没碰过她们,是陈司令身边的人——''

陈部长手上用了劲:``别他妈啰嗦,我就想知道你们是怎么把她们给祸害死的!''

毛头小子打了结巴:``是先、先轮奸,后来就开、开枪扫射。我们也死了好几个人,陈司令说要把她们压在棺材底下,给牺牲了的同志们垫、垫棺材。''

陈部长一刺刀就捅出去了,直戳进了毛头小子的脑子里。然后对着身后的弟兄们一挥手,众人放下步枪抄起铁锹,开始挖坟掘墓。

在陈部长忙着处决俘虏安葬烈士之时,无心给苏桃弄到了一双搭袢的小布鞋——他和苏桃都没有布票。没有布票,就买不到布制品;幸好他脑子活络,用粮票和人换了布票,又用布票去买了鞋。两人回了与世隔绝的小宿舍里,苏桃换了布鞋来回走了两趟,又跺了跺脚,高兴的告诉无心:``不大不小,正合适。''

坐在床边抬起双脚互相磕了磕,她继续对着无心笑:``真凉快。''

无心靠墙站着,很怜爱的看她:``晚上我们打壶热水回来,让你洗个澡。''

苏桃欢喜的点头,又对无心说道:``我给白娘子也洗一洗。''

无心当即摇头:``他就算了。''

白琉璃把脑袋搭在苏桃的大腿根上,恨恨的瞪了无心一眼。

\chapter{所谓天人}

无心因为和苏桃睡一间屋,遭到了全走廊所有男性的敌视。无可奈何,他只好去找小丁猫,借了一只暖壶和一只水桶。

吃过晚饭之后,无心打了两暖壶热水以及一大桶冷水送进宿舍,又从外面锁了房门,让苏桃自己留在房里洗澡。挎着书包装好白琉璃,他走到了楼后的僻静处。从体育器材室的遗址上搬来一块水泥墩子,他稳稳当当的坐好了,打开书包先抻出白琉璃,再取出薄薄的半册残经。白琉璃精神焕发的在他面前盘成一堆,一个脑袋昂了老高。

面对着对方一双炯炯有神的黑豆眼睛,无心压低声音说道:``白琉璃,原来我一直以为我是个妖怪,但是现在,我怀疑我是搞错了。''

话音落下,他一抖手上的残经:``它的名字,叫做《本事经》。你知道我做过许多年和尚,基本没有我没读过的佛经。《本事经》我肯定也是念过,虽然我后来全忘了。不过忘了也没关系,因为原来念了也白念。''

白琉璃有点走神,感觉无心像个老糊涂,啰啰嗦嗦的不进正题。

无心伸手一托白琉璃的圆脑袋,郑重其事的说道:``白琉璃,我发现我可能是个天人。天人你知道吧?六道轮回里面最高级的一道,就是天道。活在天道中的生命,就是天人。''

白琉璃刚刚百无聊赖的一吐信子,骤然听到``天人''二字,因为啼笑皆非,以至于信子吐出之后忘了收回。

无心兴致勃勃的翻开书页:``你看,经书上说第一,天人寿命长,具体的我就不念了,反正里面普普通通的天人,都能活个几百万岁;第二,天人长相好,这一点我就更符合了,从古至今,还从来没有人说过我丑;第三,天人很快乐,当然啦,我一直是不怎么快乐,因为我不在天界在人间嘛!''

说到这里,他把手里的残经放下了,一双眼睛乌溜溜的射出光芒:``白琉璃,天人是天生的洁净,我也很洁净,只要别把我扔到粪坑里,我一百年不洗澡都不会臭。白琉璃,你是什么眼神?我臭不臭你还不知道吗?我在认真的和你说话,你不要斜着眼睛看我。还看?还看?好,我证明给你看!''

无心低头解开腰带扯开裤子,抓起白琉璃塞到了自己的裤裆里。捂着裤腰等了十秒钟,他攥着白琉璃的脑袋,把对方又向上抻了出来:``我臭吗?''

小白蛇一缩信子,同时白琉璃气急败坏的在无心面前现了身:``下流的骗子!你是天人?不要往自己脸上贴金了!竟敢冒犯我,我要杀了你!''

话一出口,白琉璃伸开双臂猛地一挥。体育器材室的废墟上瞬间飞起一块板砖,``砰''的一声拍在了无心的脑袋上。

在天色蒙蒙黑的时候,无心挎着书包扶着墙,一步一步的上了三楼。打开走廊尽头的小宿舍门,他探头进房,嗅到了一鼻子热腾腾水淋淋的香味。苏桃穿着短衫长裤,正在用抹布擦拭双层床的栏杆。披着湿头发对无心一笑,她开口问道:``我有半个小时就够了,你怎么才回来?''

无心支吾着没说出什么,拎着水桶出去倒水,又把暖壶还给了小丁猫。把书包挂在床栏上,他早早的上了床,侧身在被窝里蜷成了一团。

苏桃平时看他总是一个劲儿,仿佛永远乐观,如今见他状态有异,在熄灯之后就惦念得睡不着。后来忍无可忍的从上方探□,她低声问道:``无心,你怎么了?''

无心在黑暗的下铺上呻吟了一声:``我没事,就是有点头疼。''

苏桃的脑袋缩上去了,取而代之的是一只赤脚踩上了床尾的铁梯。苏桃在夜色的掩护下,穿着花布裤衩下了床,伸手去摸无心的额头:``不是病了吧?''

无心悻悻的摇头:``你睡你的,我可能是晚上被风吹了头,睡一觉就好了。''

苏桃没主意,手足无措的站在床前,不知如何是好。后来在无心的催促下爬回上铺,她颇为担忧的钻回了被窝。

等到苏桃睡熟之后,白琉璃得意洋洋的现出了影子,正好悬在了无心的腰腹上方。无心把脸藏在棉被下面,声音小小的说道:``别打了,我承认我是老妖怪。''

白琉璃怀疑他是在装可怜,不过装得太逼真了,让人不得不饶恕他:``我不打你了,可是你以后也不许再对我吹嘘你是什么天人。''

无心躲在棉被下面,半晌没有说话。白琉璃看他彻底老实了,正是满意的要走,不料他忽然又出了声:``我依然感觉我是从天界不小心掉到人间的\ldots{}\ldots{}''

白琉璃怒视了他:``还说?''

无心在棉被下面摇了摇头:``不说了。''

白琉璃虎视眈眈的盯了他良久,在确定他是真闭嘴了之后,终于心满意足的消失在黑暗中。回到小白蛇体内,他舒舒服服的在苏桃身边趴好了,正要休憩一阵,哪知下方一阵嘤嘤嗡嗡,正是无心藏在被窝里自言自语:``我怎样才能回去呢?''

无心向白琉璃袒露心迹以及身体,结果换得一顿板砖。一觉醒来,他认定白琉璃不是自己的知音,便一言不发的独自思索了片刻,片刻之后肚子里叽里咕噜乱叫,他没想出主意,只想出了食欲。

上午,他和苏桃在一楼写了几副挽联,准备挂到田小蕊等人的追悼会上。田小蕊等人生的伟大、死的光荣,截去被红总轮奸的一段不提,英勇就义的事迹还是值得宣扬一下的。

挽联写完了,无心上楼去了小丁猫的办公室,想要询问下一步的工作。马秀红给他开了门,而他见房内赫然正跪着一个顾基,就迟疑着没有往里进。倒是小丁猫出了声:``无心吗?进来吧!''

然后他转向顾基,接着方才的话头继续说道:``你不要跪,我也不需要你跪。你要革命就动手,你不革命就滚蛋。''

顾基有些恍惚,只是感觉跪着更对劲,跪着更有安全感:``他毕竟是我爸爸\ldots{}\ldots{}''他带着哭腔哀求道:``我不是决心不强意志不坚,我是真的——真的下不去手啊。求求你别让我干了,换别人吧!我不给他求情,我也不给他收尸,我让他罪有应得遗臭万年\ldots{}\ldots{}我求你了\ldots{}\ldots{}''

他嘴里说着,咚咚又磕了几个头。小丁猫翘着二郎腿坐在椅子上,叼着香烟喷云吐雾:``顾基,你让我很失望。''

顾基闭上眼睛,眼泪扑簌簌的往下落。他从小处处都不如人,因为家庭出身饱受压迫。没想到像小丁猫这样的大人物居然会对他寄予了希望,而他十恶不赦,竟然让小丁猫同志感到了失望。他哭得抽抽搭搭,肝肠寸断,不是为了即将赴死的父亲,也不是为了已然惨死的母亲和奶奶。他是自责而又恐慌,因为不想孤魂野鬼的一个人混日子。他要和小丁猫闹革命,一个人生活,他害怕。

小丁猫静静的等着他哭,等他把杂念都哭干净了,才轻而坚定的说道:``真正的革命者,是六亲不认的。你的战友才是你的亲人,革命路线才是你人生的方向。''

无心靠墙站着,心想小丁猫可能在娘肚子里就是一块老谋深算的胎了。

小丁猫不再理睬顾基,端着椅子原地转了个方向,对着无心一招手:``你过来。''

然后他拉开抽屉,从里面拿出一本册子扔在桌上:``有人揭发你搞封建迷信。自己看吧,是不是你的东西?''

无心拿起桌上的残经翻了翻——昨晚让白琉璃打慌了,他抱着书包就跑,而佛经又不是什么稀罕物,所以他随手一扔,根本也没想带上。

``不是。''无心很笃定的答道:``这书我根本看不懂。''

小丁猫讥讽的咂了咂嘴:``年纪小,不懂也是正常的。''

无心望着他眨了眨眼睛,终于是忍不住问道:``你\ldots{}\ldots{}你到底是谁?''

小丁猫把残经收回了抽屉:``远的不谈了,只说眼前,你来干什么?''

无心盯着小丁猫,怎么看怎么陌生,而且自己也不会有一个不到二十岁的小故人:``那个\ldots{}\ldots{}挽联写完了。''

小丁猫有点不耐烦:``写完就写完了,这也值得上楼一说?下午去帮宣传队忙一忙吧,我这里没有抄写任务了。''

无心一头雾水的离了办公室,然后也并没有去宣传队帮忙,而是带着苏桃出去逛了一下午。到了傍晚,两人回到宿舍。苏桃手里拿着一根雪糕,进门之后先去看白琉璃。咬下一小块雪糕送到蛇嘴边,她逗了半天,小白蛇却是趴在床上,一动不动。

``无心。''苏桃惊讶了:``你看啊,白娘子怎么不理人了?''

无心凑过去,用手指拨了拨白蛇的脑袋:``白琉璃?''

小白蛇依旧是没反应。

无心抬起了蛇脑袋,发现小白蛇的黑豆眼睛里没了光点。白琉璃此刻没有附在蛇身上——白琉璃去哪里了?

``没事。''他一边安慰苏桃,一边把小白蛇装进书包:``蛇有时候是会懒一点,也许是吃得太多,也许是感觉太冷。别管它,它安静几天就恢复了。''

弯腰拎起屋角的暖壶,他对苏桃又道:``你吃你的,我去打水。''

无心花了很长时间才拎回了一壶开水。两人洗漱过后,关灯就寝。苏桃身边没了小白蛇,总像是少了点什么,让她睡得不自在。午夜时分,她迷迷糊糊的自动醒了。掀开棉被爬向床尾,她想把小白蛇放到自己的被窝里暖一暖。然而一手扶着护栏向下一望,她登时一愣,就见下铺空空荡荡的堆着棉被,无心却是不见了。

\chapter{白琉璃归来}

无心穿着棉裤裤衩和汗衫,趿拉着球鞋走在三楼走廊里。走廊两边都是教室,虽然如今被当成宿舍居住,可是透过门上的玻璃窗,还是可以窥见室内情形。走廊的天花板上亮着一盏昏暗小灯,微弱的光明冲不淡室内室外的黑暗,反倒把走廊照得越发深不可测。

无心很担心白琉璃,同时又认为白琉璃实在是无须让自己担心。做人的,根本意识不到白琉璃的存在,当然也不会去伤害他;做鬼的,不被白琉璃伤害就不错了。但是白琉璃毫无预兆的不知所踪,让他不能不出门找一找。

大半夜的,两边房中全是漆黑一片,此起彼伏的拉扯着粗重鼾声,唯有楼梯口处亮起一线绿光,是小丁猫的宿舍门没关严。宿舍门口有浅淡的影子时隐时现,分明正是一只夜游至此的鬼魂。一中所在的位置,不能算佳,因为前前后后都开阔得一览无余,太阳从早到晚的当空照,四周无水无木。先前空旷无人的时候倒也罢了,如今人一多,阳气立刻压倒了阴气。活动在楼内的鬼魂越来越少了,它们无处吸取能量,所以纷纷的消散;阳盛阴衰,气无所聚,也不是好事。

无心停了脚步,不明白游魂怎么会向着灯光走,除非是因为小丁猫住单间,勉强算是人单势薄。如果游魂想要去害小丁猫,他是绝不会出手阻拦的。虽然他一贯的挺爱人,并且懒得和任何活人计较,但是一个人若是做出了如魔似鬼的事情,无心没办法,只好把对方归到魔鬼一类。恶鬼杀魔鬼,和他没有一分钱的关系,他只想尽快找到讨厌鬼白琉璃。

球鞋的软底踏在走廊地上,一丝一毫的声音都不发出。无心站在暗中,静观前方的鬼魂动作。大部分的鬼魂除了能够现形吓人之外,力量还不如一阵风。无心不怕它和自己作对,但是怕它忽然吹起小风,会惊动了房内的小丁猫。

正是观望之时,门口的鬼魂忽然开始闪闪烁烁。随即白琉璃在它的后方出现了,他的影子越清晰,鬼魂的影子越暗淡。眼看在门口探头缩脑的鬼魂将要被他彻底吞噬。门缝中忽然飞出一线白光,正好掠过了白琉璃的鬼影。仿佛只是一瞬间的工夫,白琉璃消失了!

无心见势不妙,连忙大踏步的向前跑。三步两步到了近前,他扭头往墙壁上一看,只见一张小纸条斜斜的切进了白墙,此刻正在嗡嗡的颤动。若不是纸条上一片空白,无心真要以为它是一张镇鬼符。

脚步声音震动了房内,小丁猫的声音传出来了:``大半夜的,谁在外面?''

无心转向了门缝:``是我,无心。''

从门缝中向内望,可见房中桌上亮着一盏绿色的小台灯。小丁猫坐在桌后,一手翻书一手执笔,正在低头写着什么。手写着,头低着,他忙里偷闲的继续问:``不睡觉,跑出来干什么?''

无心不假思索的答道:``我有点失眠,睡不着。''

小丁猫把手中的圆珠笔往稿纸本子上一拍,抬起头打了个哈欠,又端起茶杯喝了一口堪比黄连的浓苦丁茶:``失眠?你倒是娇气得很。革命群众们白天挥汗如雨的战斗一天,夜里上床沾了枕头就睡。你白天无所事事,夜里四处溜达,还美其名曰失眠!''

无心慢慢的往后退,一边伸手去碰切入墙内的白纸条,一边唯唯诺诺的答道:``我马上回屋,以后再也不失眠了。''

指尖一碰纸条,他心中一惊。纸条像刀片一样寒冷坚硬,而且正在高速的抖。符咒他也画过无数了,收鬼的手艺也早练纯熟了,但还没遇过如此作怪的纸符。

他不明就里,不敢妄动,怕伤了被封在符里的白琉璃。身边的房门开了,小丁猫甩着手伸出头,显然是张了嘴要说话,可是未等他发出声音,半空中忽然起了一声轻微的爆响。细碎纸屑随着混乱气流,在两人面前打了个小小的旋儿。无心扭头再望,发现竟是纸符自行炸了。

小丁猫莫名其妙的环顾四周:``怎么回事?你刚听见声音没有?''

无心摇了摇头:``没听见。''

小丁猫伸手一指他的鼻尖:``什么东西?碎纸?''

无心抬手揉了揉鼻子:``啊?''

小丁猫看他一问三不知,不禁不耐烦的一挥手:``啊什么啊,你回去吧。''

然后他把房门一关,把一脸傻相的无心和自己隔绝。甩着满是热汗的手,他若有所思的回到了座位坐好。端起苦丁茶又抿了一口,他正打算给自己点一根香烟,不料外面忽然有人敲响了房门,吓得他一哆嗦:``谁?''

门外人坦坦然然的作了回答:``我,杜敢闯。''

小丁猫一咧嘴,手指夹着刚刚取出的香烟起了身。绕过桌子走去打开房门,他就见杜敢闯穿得整整齐齐干干净净,一边腋下夹着一只牛皮纸袋。对着小丁猫一点头,她不请自入的进了房中:``我看你的房内还亮着灯,想你应该是没有睡,所以过来送份文件。''

小丁猫用膝盖把房门顶上,然后转身笑道:``我以为我已经是夜猫子了,没想到你也一样在熬夜。怎么?上面又有新消息了吗?''

杜敢闯把牛皮纸袋放到了桌上,望着桌面的内容答道:``前天**同志、张春桥同志、姚文元同志在人民大会堂接见了上百名革命小将。这是大会的会议记录,你可以读一读。''

小丁猫走到座位一旁,弯腰拉开抽屉去找火柴。而杜敢闯见他桌上一片狼藉,除了纸笔书籍之外,还有一沓沓裁好的凌乱纸条。其中一些纸条上面已经写了字,另一些空白的,则是被胡乱堆在一旁。拿起一张纸条看了看,她发现上面写的是前一阵子中央军委下达的《十条命令》。十条命令当真被他用十条白纸写成了十条,可见在她到来之前,小丁猫一定是在逐条的进行深研究。

杜敢闯放下手中的纸条,郑重其事的抬头望向了小丁猫:``小丁猫同志,我要批评你。''

小丁猫刚刚点燃了香烟。深吸一口抬了头,泛绿的台灯灯光自上而下的照耀着两人,正是显出了杜敢闯一脸的横肉,满额的痘子。而杜敢闯自知形象不美,所以神情格外肃杀,表明自己一颗红心目中无人,对小丁猫绝无癞蛤蟆想吃天鹅肉的妄念。

两人打了一秒钟的照面,小丁猫的腿肚子有点要转筋:``我怎、怎么了?''

杜敢闯高傲而又深情的凝视着他:``你对你自己的身体,太不负责任了。''

小丁猫扭头打了个喷嚏,然后探头望着杜敢闯:``啊?''

杜敢闯勉强露出了爽朗笑容:``你天天熬夜,在饮食上也是有一顿没一顿,时间久了,身体可是要撑不住的。你累垮了,谁来带领大家和阶级敌人作斗争?谁来带领大家去消灭牛鬼蛇神反革命?丁同志,你要记住,你的身体,不是你一个人的身体。你的身体,属于联指的全体战士。''

小丁猫笑着点头,虽然感觉杜敢闯说话不伦不类,好像要带人把自己分而食之,不过意是好意,自己不能不识好歹:``好,我知道了,我马上就休息。''

杜敢闯看了他斯文的面貌,听了他温柔的声音,两只脚不由得钉在水泥地上,无论如何拔不动:``《十条》不必再看了,大方向我们已经抓准,其余的细枝末节,可以不必深究。''随即她仿照苏联电影里的女主角,一甩头发自信的笑:``我的政治水平,你可以信得过!''

小丁猫端起茶杯喝了一大口:``那是,那是。我们从初中起就是同学,这个\ldots{}\ldots{}我当然很了解你。''

杜敢闯斜靠在桌边,四周万籁俱寂,房内亮着一盏幽幽的小灯。气氛太美好了,她是真不愿意走:``还有明天的追悼会——''她飞快的转动脑筋,找出话题来谈:``时间上,和上个月定下的忆苦思甜报告会起了冲突。''

她不走,小丁猫意意思思的也不敢坐:``没有关系,追悼会是追悼会,报告会是报告会。追悼会放到机械学院去开,报告会是在钢厂大礼堂。年纪大的去追悼会,年纪小的去报告会,双管齐下,互不耽误。''

杜敢闯深以为然的点了头,脑子里忽然想起了初中时读过的《红楼梦》。单凭智慧来论,如果小丁猫是贾宝玉,那么她就有自信兼任林黛玉与薛宝钗,至于马秀红,则是彻头彻尾的属于袭人一流。若是真往长远了想,她认为自己也许会容下马秀红——大家都是同学,看都看惯了,虽然也有醋意,但总像是肥水不流外人田。

杜敢闯想出了神,直到小丁猫把手挥到了她的面前:``杜敢闯,我要睡了,你也去睡吧。为了革命,你我都要保重身体。''

说完这话,小丁猫又拿起牛皮纸袋笑了笑:``记录我会认真的看,有时间我们就此讨论一下。''

杜敢闯意犹未尽的答应一声,知道自己不走不行了。为了显示自己的大方,她几近豪爽的露齿一笑,然而转身走向门口。小丁猫一直把她送进走廊,又目送她经过楼梯口进入女生宿舍区了,才轻轻的关了房门,转身叹道:``哎呀妈呀。''

小丁猫关了台灯,上床睡觉。与此同时,无心正在男厕所和白琉璃说话。白琉璃气得有点走形,鼻子不是鼻子脸不是脸的。无心低声问道:``你跑哪儿去了?''

白琉璃答道:``我去了楼后,想抓几只鬼吃。''

无心把一只手伸进裤衩里抓痒:``然后呢?你抓鬼吃我没意见,可是怎么该回来不回来?你就非得折腾我一趟,让我大半夜的出门找你?''

白琉璃怒道:``难道是我不想回来吗?是有人在楼后布阵困住了我!''

无心抓下了几根毛,抽出手吹出一口气,把毛吹飞:``什么?''

白琉璃虽然做了几十年的鬼,但是看了无心的举动,还是下意识的侧身一躲:``不要扯你的毛了,我说有人在楼后布了阵!是什么阵,我不知道;我只知道魂魄一旦进去,就很难出来。''

无心挠了挠屁股,又挠了挠头:``那你是怎么出来的?''

白琉璃的嘴脸又不好看了:``无心,我是一般的鬼魂吗?''

无心知道他是相当的不一般,连镇鬼的纸符都能被他打破。和白琉璃讲道理是讲不通的,他只能是把对方当成驴来摩挲:``是,我知道你厉害。你在大兴安岭也吃了几十年的鬼了,只要你安安生生的别出事,再过几十年你都能修炼成煞。但是为了我和桃桃的安稳觉,你现在能不能老实做蛇,不要惹事?我告诉你今时不同往日,现在的世道很不好混。你要是再胡闹,我可把你送回大兴安岭不管了。''

白琉璃一瞪蓝眼睛:``你——''

无心不等他发飙,立刻双手合什拜了拜:``乖,大巫师,跟我回屋吧。小半天没见你,我和桃桃都想死你了。明早我还有活要干呢,求你让我好好睡几个小时吧!''

白琉璃的思想素来不成体系,方才他本来预备大闹一场,不过听无心说了几句软话之后,他心思活动,不知不觉的失了锐气,糊里糊涂的就和无心回了宿舍。而无心推门一进,迎面看到上铺床上坐着苏桃,便立刻关严房门,小声问道:``怎么醒了?''

苏桃抱着棉被一直在等他,他不在,她就躺不住。如今总算把他盼了回来,她松了一口气:``我刚才醒了,看你不在,就等着你呢。''

无心站在地上,仰头看她:``我是去了厕所,没事,睡吧。''

苏桃慢慢躺下了,侧身对着床外又道:``无心,我们明天也要去追悼会吗?''

无心抬手抓住护栏:``不想去?''

苏桃``嗯''了一声,嗫嚅着又道:``听说他们要在追悼会上杀人\ldots{}\ldots{}''

无心伸长手臂,摸了摸她的头发:``明天钢厂大礼堂的报告会也需要人手,我们到时候想办法去钢厂。报告会总不怕吧?''

隔着一层微凉的长发,苏桃感受着他手掌的温度和重量:``不杀人就不怕。''

无心向她笑了一下:``睡吧,明天一醒,白娘子也醒了,我们带他去忆苦思甜。''

安抚着苏桃睡下之后,无心没闲着。他无声无息的画了一道专镇邪祟的纸符,摸索着贴在了下铺床板的背面。他的纸符是制不住白琉璃的,但是可以对付一般的小鬼。既然有人收鬼,自然就有人用鬼。如今这间小小的宿舍就算是他的家,家里有个禁不住吓的小姑娘,他不能不有所防备。

作者有话要说:祝大家圣诞快乐O(∩\_∩)O\textasciitilde{}

\chapter{忆苦思甜}

大清早的,无心和苏桃装了一肚子杂合面馒头和咸菜丝,拎起一只浆糊桶往钢厂走。临走时他怕管事的阻拦,所以特地做出忙忙碌碌理直气壮的模样,只和宣传队里一个十三岁的小丫头打了一声招呼。没等小丫头反应过来,他已经和苏桃跑没影了。

苏桃挎着书包,书包里装着白琉璃和水壶。因为害怕半路会被人捉回去参加追悼会,所以一路跑得张皇失措。及至进了钢厂内部的大礼堂,她要来热水把浆糊和上了,心里才稍稍安定下来。

负责主持忆苦思甜报告会的人物,乃是武卫国手下的一位女将。该女将声名显赫,本是厂医院里的一名小护士,因为去年号称用毛泽东思想治好了精神病而名声大噪,还上了报纸。在上了报纸之后的一个月内,该护士医术精进,在毛泽东思想的指导下又无师自通的使盲人重见光明,哑巴开口歌唱。

当然,受惠的盲人和哑巴始终是只闻其名未见其人,但是也无人深究,因为敢管小护士的医院领导已经全被批倒批臭。小护士本人则是扶摇直上,成了厂里的风云人物之一。

大礼堂十分宽敞,听众们全是停课闹革命的红小兵红卫兵,从七岁到十七岁应有尽有。作报告的老贫农们则是小护士亲自下乡请进城的,个个都是能言善辩之士,此刻正穿着破夹袄在台下坐成一排,吧嗒吧嗒的抽烟袋。台上的桌椅还未摆好,无心踩着板凳登高上远,一张一张的贴标语,苏桃一手拎着浆糊桶,一手虚虚的拢着他的小腿,生怕他会一脚踩空。

台上热闹,台下更热闹,歌声此起彼伏,还有小队在众人面前跳忠字舞。忆苦思甜报告会的气氛总还算是和平的,及至台上布置完毕了,无心和苏桃退到队伍后方,在角落里找了空座坐好。老贫农们上了台,礼堂内的喇叭里也放了音乐:``天上布满星,月牙亮晶晶,生产队里开大会,诉苦把冤伸,万恶的旧社会,穷人的血泪账,千头万绪千头万绪涌上了我的心\ldots{}\ldots{}''

一曲《不忘阶级苦》终了,台下的大孩子小孩子们再合唱一遍。及至小护士把开场白说完了,对伟大领袖毛主席和毛主席的亲密战友林副统帅也敬祝完了,报告会进入正题,开始请老贫农忆苦。第一位老贫农,生活和嘴皮子都非常之贫,眉飞色舞的讲述他年轻时候如何在地主家里干一天活偷两天懒,又是如何气得地主婆站在田垄上骂他。

提到自己饱受压迫的岁月,老贫农得意的大笑:``他老地主敢不给我们扛活的吃好喝好?他不给我们喂足了,我们就不给他干活,我们就给他磨洋工,他能怎么的?他打我杀我?我光脚不怕穿鞋的,我怕啥?大不了他撵我,他撵我我上别的庄子去!嘁!逼急了我,我烧他的房!''

此老贫农越说越横,一身大无畏的流氓无产者气概。后来主持人听他把自己的生平越讲越细,刁蛮有余,凄惨不足,便当机立断,请他先歇一歇。第二位老贫农慢条斯理的,说起话来就中听多了,而且是真苦——年纪小小没了爹娘,十几岁去闯关东,一个孤人混日子,混到最后又回了关内老家。

提起往昔岁月,老贫农微微一笑:``我那时候年纪小啊,重活干不了,就在一户人家里帮工,帮人家跑跑颠颠干杂活。那时候我一个月能挣八块绵羊票,八块钱不少哇,能买两百来斤白面了。我那时候最喜欢吃什么?我就喜欢吃大麻花。嗬,刚炸好的大麻花,这么粗,这么长,那个脆啊,你们没吃过,你们不知道。好吃啊,真好吃。''

老贫农说到这里,悠然神往的咂了咂嘴,接着回顾起了高丽馆子里的冷面:``人家那伙计,真是个本事,你一个电话打过去,那边马上把冷面给你送家来。你们是没看见,那小伙计骑个自行车,一手扶车把,一手托个大盘子,一盘子里高高摞上五六碗面,一路过来,绝不给你洒一滴汤,有点儿意思吧?''

红卫兵红小兵们咽着唾沫,感觉是挺有意思,因为其中有相当数量的革命小将在早上过来忆苦之前,就只啃着窝头喝了一碗棒子面粥。
老贫农斜眼望着大礼堂高高的天花板,继续讲述他记忆中的美食,讲得听众们垂涎三尺,连小护士都有点熬不住了,悄声让老贫农讲些阶级苦血泪仇。

老贫农十分识相,话锋一转到了解放后,饭食也立刻降级到了野菜汤榆钱饭,然而依旧绘声绘色,听得小将们恨不能出了礼堂就去刨地上树。小护士看穿了第二位老贫农的本质,认定他是个吃货,便当即中止了他的报告,换第三位老贫农登场。

第三位老贫农开腔不到十分钟,场下开始有孩子嘤嘤哭泣了,台上的小护士也红了眼圈——太惨了,一家五个孩子饿死了仨,出去要饭还不让出村,偷着出去了因为没证明,又让民兵用枪托给杵了回来。

哭声渐渐连成了片,苏桃也跟着抹眼泪。小护士扯了一块卫生纸一擤鼻涕,忽然感觉不对劲。侧耳细听片刻,她伸手把老贫农面前的麦克风拿走了——老贫农讲的是五六十年代大饥荒的事情,和旧社会没个屁关系。

趁着大小孩子们没反应过来,最后一位老贫农粉墨登场。这位老贫农规规矩矩一本正经,不说吃不说穿,开口便道:``我家祖宗八代全是要饭的,我爷爷死在了要饭的路上,我爸爸也死在了要饭的路上,只有我赶上了好时候,生在旧社会,长在新中国。''

小护士抓住机会,立刻起身呼喊口号:``牢记阶级苦,不忘血泪仇!''老贫农淡然的继续说道:``我们解放前受尽了地主老财的压迫和剥削,解放后我分了地,成了家,过上了幸福的生活。''小护士再次呼喊:``翻身不忘共·产·党,永远忠于毛主席!''

台下响起一片激烈的掌声,而老贫农超然物外的说道:``原来地主老财们站着房躺着地,黄的是金白的是银,我们劳动人民,得伸着手向他们要吃要喝。现在他们跟我们一样穷了,他们一穷,我就啥也要不来了,也得跟着种地了。''小护士端起茶杯:``老大爷,你喝口水。''

因为小护士识人不明,弄来四位糊里糊涂的老贫农,导致忆苦思甜报告会在一种说不清道不明的古怪气氛中宣告落幕。听众们一人得了一只成分复杂的糠窝头和一块糖,糠窝头是苦,糖是甜,精神上忆苦思甜完毕了,肉体上还要再演练一遍。

虽然孩子们都是没有好吃好穿,但用来忆苦的糠窝头还是突破了革命小将们的忍受极限。无心和其他的半大孩子一样,一出大礼堂就偷偷找地方把糠窝头扔了,苏桃则是仰起头小声问他:``真有那么粗那么长的大麻花吗?要是有的话,我一顿吃半根就够了。''

无心拎着叮当乱响的空浆糊桶,把手里剩下的一块糖塞进了苏桃的衣兜里:``有,但是麻花太大了不好炸,所以那么大的麻花很少见。''苏桃立刻又问:``你吃过吗?''无心摇了摇头:``没吃过,吃过小的。''
苏桃望着他又问:``旧社会的饭店,还能派服务员把饭菜送到家里去呀?''无心拍了拍她的后脑勺:``能。''

苏桃想了想,因为感觉不可思议,所以莫名的有一点兴奋:``现在还有榆树钱吗?''无心笑道:``榆树钱没有了,已经过季节了,要吃得等明年。''苏桃有点失望,对着无心说道:``那\ldots{}\ldots{}给我买个小圆面包吧!''
无心问道:``我现在花的都是你的钱,你还用向我提申请?''苏桃反问:``你不是说你要管我吗?''无心被她问住了,左思右想,无话可答。

在返回指挥部的路上,无心花了二两粮票和一毛钱,买了一个小面包给苏桃。苏桃手上有两百块钱,是老苏留给她的活命钱。二百块钱得花到哪天,无心心里也没有数,所以计划得很仔细。苏桃站在僻静处,打开包装的蜡纸之后,撕下绵软的半块面包给无心。无心摇头表示不要,但是她很执着的伸着手,不肯收回。

无心把面包接了,鸟啄似的咬了小小一口。等到苏桃把自己的一份吃光了,他拉过苏桃的手,把余下半块放到了她的手中。``我是大人了,已经长成了,吃什么都一样。''他告诉苏桃:``你多吃一点,以后长得结实。''苏桃低声嘀咕:``我也是大人了。''无心轻轻一扯她的辫子:``等到文化大革命结束了,你再长大吧!''

苏桃把半块面包捏了捏,面包禁不住捏,看着挺大,一捏就没。一口咬下一半,她知道无心说得有理。她也想做个没人搭理的小丫头,可她分明是时时刻刻都在成长。她的肩膀还是薄薄的,然而胸脯已经把紧贴身的小背心顶出了明显的波澜;她的腰还是细细的,然而两条大腿已经饱满的有了分量。

她隐隐约约的能意识到自己的好看,越好看,越害怕,像是逃难路上露了财,反倒比一贫如洗更危险。但她同时也清楚,知道自己什么都没有,就剩一个天生的好看了。

拍了拍手上的面包渣滓,她跟着无心往回走。他们回到一中指挥部时,指挥部里已经很热闹。追悼会早结束了,顾明堂也死了。无心和苏桃正要直接进食堂,不料半路却是被陈部长拦了住。

陈部长用手巾包了个小包袱,里面装着一小包退烧药和两个白面馒头。把无心扯到食堂后方,他很诚恳的说道:``无心,求你件事。''无心警惕的看着他:``说。''陈部长把手巾包送到他面前:``你帮我把这个递给顾基,顾基回来之后又被关起来了。''

无心很惊讶:``顾明堂不是死了吗?怎么还关他?''陈部长垂着黑黝黝的脑袋:``他\ldots{}\ldots{}他在台子上给他那个混蛋爹嚎丧了。''无心压低了声音:``不是说要让他动手吗?''

陈部长叹了一口气:``是,他是下手了,他打的第一枪,打完之后顾明堂还没死呢,他就嚎上了。反正弄得小丁猫同志挺不高兴的,他要是真不行,可以早说,也不是非他不可,是吧?''无心又问:``你怎么不自己去送?''陈部长当即摇头:``我\ldots{}\ldots{}我不敢。你胆子大,连我都敢揍,你帮个忙。''无心犹豫了一下,把手巾包接过来了。

顾基就被关在一楼走廊尽头的空储藏室里。储藏室里干燥通风,本是用来堆放教材的,如今教材没了,里面只有一个顾基。储藏室的窗户正对着楼侧的方向,窗扇大开,外面焊着铁栅栏。无心让苏桃先去食堂吃饭,自己则是蹑手蹑脚的靠近窗口,对着房内轻声唤道:``顾基,我给你送吃的来了!''

顾基抱着膝盖坐在角落里,怔怔的抬头去望无心。无心摇了摇手巾包,因为看他可怜,所以极力的做出和颜悦色:``黑背让我给你带了馒头。''

话音落下,顾基忽然一跃而起直扑窗口。伸出一只大手死死攥住无心的腕子,他深吸一口气,扯着大嗓门吼道:``来人哪!有人给黑五类狗崽子送饭!来人哪!我抓住一个反革命坏分子,我戴罪立功了!''无心吓了一跳,想要再跑,就跑不成了。顾基手如铁钳,一直攥到他的骨头上去了。

\chapter{因祸得福}

隔着铁栅栏,顾基抓住无心好一番咆哮,把楼前正要往食堂走的男男女女全惊动了。陈部长正站在食堂门口等消息,冷不防的听到了顾基的嘶喊,登时抬手一拍额头:``操他妈的,这傻×是真疯了。''

然后他悄悄的往暗处退,恨不能凌空消失。不出片刻的工夫,无心和顾基全被人押到楼前的大太阳下了,小丁猫从楼内出了来,走到无心面前问道:``怎么回事?''无心急得分辩:``是黑——陈部长让我给顾基送点儿吃的,和我本人没有任何关系!''

此言一出,陈部长像个鬼似的,忽然从人群中冒出来了。抽出皮带握紧了,他一皮带就抽上了无心的脑袋:``放你妈的屁!你敢污蔑老子!''苏桃冲了上来,伸手去护无心的头脸:``他说的是实话!我作证,就是陈部长让他去送的!''

陈部长听闻此言,吓得肝胆俱裂,一皮带又抡上了苏桃的肩头。``啪''的一声过后,他打出了兴头,反手一皮带又抽向了苏桃的脸蛋:``看你这个又酸又臭的资产阶级小姐德行!你个搞破鞋的也敢往老子头上泼脏水!''

苏桃没挨过打,不知道躲。``嗷''一嗓子哭出声,她只觉半边面颊像是没了皮,火烧火燎火辣辣。而无心被人反剪了双手不能动,眼看陈部长双眼放光,还要追着去打苏桃,便骤然发力向前一冲,一头把陈部长顶了个四脚朝天。操场一片大乱,小丁猫挥了挥手,命人把无心和顾基押走。

苏桃独自回了房,关上房门放出白琉璃,她身边也没个伴儿,只好捂着脸对白琉璃哭诉:``他怎么那么坏啊?光天化日就撒谎,撒了谎还要打人。无心是好心帮他,一片好心换了一副狼心狗肺。我们真傻,明知道他不是好东西,还帮他的忙\ldots{}\ldots{}''

白琉璃一吐信子,同时越昂越高,最后用冰凉的圆脑袋和苏桃贴了贴脸。苏桃真想抱住什么痛哭一场,可白琉璃也就比麻绳粗一点,实在不够一抱。眼泪扑簌簌的落下来,她起身扯了毛巾擦了擦脸,然后把白琉璃装回书包挂上床栏,决定出去再探一探风声——不往远走,直接去二楼找小丁猫。

虽然在她眼中,小丁猫有种阴阳怪气的危险性,不过对方有一个好处,就是不会轻易的动手打人;而她的要求也不高,只要不挨打就好。

苏桃出了门,没有走到二楼,在三楼的宿舍区里就遇到了小丁猫。小丁猫独自一个人打开了宿舍门,仿佛是正要进房休息。苏桃连忙鼓足勇气,猫叫似的唤道:``丁同志!''小丁猫闻声转头,仿佛是愣了一下,随即恍然大悟的对她一笑:``有事?''

苏桃停在半路不肯走了,身体似乎快要撑不住一身偏大的旧军装:``丁同志,无心是冤枉的。''小丁猫对她一招手:``有话进来说。''苏桃硬着头皮往前蹭,一步一步的慢慢挪。小丁猫很有耐心的站在门口等待,及至她终于迈过门槛了,他立刻把门一关,顺手又划上了插销。

苏桃不敢再往里走了,小丁猫让她坐,她也不坐,紧靠门板垂头站着,喃喃的说:``丁同志,你相信我吧,我可以给无心作证。我们本来是要去食堂吃饭的,陈部长半路拦住我们,他说——''

没等她说完,小丁猫已经不动声色的贴到她身前了。一手抬起她的下巴,另一只手抚上她的面颊,小丁猫比她高了大半个头,居高临下的问道:``小陈人黑手也黑,我一时没拦住,他就招呼到你脸上去了。现在感觉怎么样?还疼不疼了?''苏桃怔了一下,随即低头扭脸横着移动,两只手畏畏缩缩的抬起来,对小丁猫是要推又不敢推:``不疼。''

小丁猫深深的吸了一口气,发现美人就是美人,美人的气息都是甜暖的。纵观指挥部上下,唯有苏桃能配得上自己这个一表人才的处男。他的自我感觉一向良好,虽然夜里偶尔会骚动得翻来覆去,但是对于一般的大姑娘小媳妇,他还不往眼里放。

``大方一点嘛!''他抬起双手撑住门板,把苏桃困在了自己怀中:``扭扭捏捏,不是个革命小将应有的样子。''

苏桃从来没细看过他,如今近距离的相对了,她迅速的撩了他一眼,就见他一双眼睛躲在玻璃镜片后面,不大,但是眼珠子精光四射,尖锥锥的能扎人。瞬间的一眼就让她看够了,她大着胆子伸手推他,推了肩膀推胸膛,然而推不动:``我要走了\ldots{}\ldots{}''

小丁猫又吸了一口气,似乎是很吃惊,以至于要``倒吸一口冷气'':``苏桃啊,你这是干什么?你天天和无心一个屋住着,不该怕男人呀!''苏桃吓得不敢动了,直愣愣的抬眼看他。小丁猫的神情和语气都让她感觉陌生——即便小丁猫不是一位年少的革命领袖,这话也不该从他嘴里出来。

小丁猫向她一笑:``还是你挑着人怕,不怕无心只怕我?''苏桃彻底不说话了。小丁猫看着白白净净,其实是杆老烟枪,话没说两句,先喷了她一脸的烟油子味。她是在她妈妈营造出的女儿国中长到十四岁的,小丁猫的锐利目光和呛人烟味让她又反感又恐慌的想起了三个字:``臭男人''。

她背过了一只手,摸索着要去拉开插销。而小丁猫见她要逃,却是放下双手后退了一步:``不要怕,你可以走。对于无心的反革命行为,你放心,我们也会秉公处理。你回去想想吧,如果有了新的思想变化,可以随时过来向我汇报。''苏桃没言语,转身拉开房门就跑了。

苏桃回到宿舍,坐在下铺床上抹眼泪。她再傻,也明白小丁猫的意思了——小丁猫等着自己过去``汇报''呢。但无心是不能不救的,没了无心,她简直不知道该怎么继续活。抓过无心的枕巾擦了擦眼泪,她转念一想,一颗心渐渐沉下去降了温。

凭着她的出身,活着就算是占了便宜,想要活得体面清白,则是根本没有可能。现在还没有人真正知道她的底细,万一哪天暴露了,她就是千人踩万人踏的命运。真的,赶上这个世道了,还装什么金枝玉叶。先把无心救出来再说吧,他们要杀无心,还不就是一转念的事情。

苏桃哽咽着起身走到桌前,端起水杯喝了口水,然后就打算再去找小丁猫。可是偶然的向窗外一瞧,她只见小丁猫带着马秀红和杜敢闯跳上吉普车,匆匆忙忙的向外出发。吉普车开出不久,陈部长带着一大队青年涌进校园内的车棚里,骑着自行车摇摇晃晃的也追了出去。

宣传队的小姑娘们站在树下阴凉处,三三两两交头接耳,校门外的大街上有一辆卡车逆流而来。卡车急急的停在大门口,后斗上跳下一大队带着联指红袖章的青年工人,全副武装的把守了校园大门。

苏桃看得莫名其妙,猜测他们又是打仗去了。既然陈部长和小丁猫都走了,她的胆量立刻有所增长。鬼鬼祟祟的楼上楼下走了一圈,她没有发现无心的踪影。楼后的体育器材室已经被炸成了坑,还能充当监狱的地方,就是食堂旁边的小粮库了。

苏桃不敢在光天化日之下公然去找无心。傍晚她拿着大饭盒打了一份菜和两个大窝头,自己回房吃了半个窝头,然后盖好饭盒盖子,静静坐着。眼看天黑了,校园里也空旷了,她把饭盒捆好了放进书包,正要出门,不料一直盘在枕头上的白琉璃先她一步游下了地,直奔门口而去。

门是锁着的,他在门前回过了头,望着苏桃似有所语。苏桃知道他是无心养久了的,已经很通人性,这时便轻声说道:``白娘子,我出去找无心,你好好在家里等着我吧。''白琉璃退到一旁,不再动作。而苏桃轻手轻脚的开了门,未等往外迈步,脚下猛的闪过一道白影,正是白琉璃自作主张的进了走廊。向前爬出一米多远,他回了头,又去看苏桃。

苏桃没时间逗他玩。小心翼翼的锁了房门,她弯着腰伸着手,想要把他捉到书包里装好。白琉璃先是不动,待她真走近了,才又向前一窜,引路似的把她引向了楼梯口。他动,苏桃跟着动;他不动了,苏桃正要追逐,然而一阵脚步声由远及近的响起来,正是有人在前方经过了。

苏桃隐约明白了白琉璃的意思,老老实实的随着他走。一路平平安安的出了大楼,白琉璃继续把她领到了食堂粮库的后面。眼看苏桃趴在后窗户外焊着的铁栅栏上往里瞧了,白琉璃自动的攀着她的腿往上爬,一路爬进了书包里。

屋子里外都是黑,校园虽然亮着路灯,却是照耀不到粮库后方。苏桃什么也瞧不见,只好抬手敲了敲窗玻璃。里面立刻有了回应。苏桃高兴极了,压低声音唤道:``无心,开窗户啊!''

无心在里面撼了撼窗子,发现窗户合页都锈死了,奋力的拉扯了三五下,才将一扇窗户微微的拽开了缝隙。苏桃通过铁栅栏,从缝隙中伸进了手指:``无心,你挨打了吗?''指尖有了触感,里面传出无心的声音:``没有,他们中午把我锁在这里之后,就再没人管过我。''

苏桃又拼命的往里看:``顾基不在吧?''无心勾了勾她的手指头:``不知道他去哪里了,反正是没和我在一起。''苏桃只不过是碰了无心几下,心中便生出了一股子快乐情绪:``你把窗户再打开点儿,我给你带了饭。''

无心一手拉住窗把手,一脚蹬住窗台,拼了命的又推又拉。末了只听``喀拉''一声,一扇窗户被他生生拽掉了。两人痛痛快快的相对了,不由得一起发笑。无心从铁栅栏后接过了苏桃递过的饭盒:``又添了一条破坏公物的罪过。''苏桃小声说道:``他们可能是打仗去了,现在指挥部里除了站岗的,再也没有管事的人,你放心吃吧。''

无心打开饭盒,掏出一只窝头往嘴里塞,同时眼珠一转,就见白琉璃在苏桃身后现了影子。裹着一团白光悬浮在半空中,他端端正正的盘腿坐好,同时低头倾身,双臂下垂。无心一边咀嚼着窝头,一边就听天上响起了咒语。咒语以一声``嗡''开头,``嗡''过之后停顿了十秒钟,白琉璃又开了口:``嗡嘛吱莫耶萨来哆!''

无心不假思索的从铁栅栏里伸出了手,用力的把苏桃往一旁拨:``桃桃,你让开,站到一边去。''苏桃不明就里,糊里糊涂的退了一步,又退一步。无心见她退得足够远了,自己也跟着移向后方。

白琉璃生前最擅长的就是咒术,咒术强大与否,要看精神力量;如今他虽然肉体消亡,但是精神尚存,且在地堡里安心修炼了几十年。所以对于白琉璃的念力,无心的心中十分有数。

咒语声音连绵不绝,同时铁栅栏凭空开始吱吱嘎嘎的作响。忽然``嘣''的一声大响,栅栏中的一根铁条竟然无端断裂。白琉璃猛一仰头,抬手用力在膝盖上拍了一下,随即低头对着无心一笑。

无心没言语,只在暗中把双手合到眉心,对着白琉璃一点头。然后上前两步握住断裂铁条,他咬牙切齿的用了力。苏桃也上来帮忙,帮得糊里糊涂,因为不知道铁条为何会自行断裂。

``怎么就断了呢?''她使了吃奶的力气掰铁条,一边掰一边问。无心信口胡说八道:``可能是锈得太厉害了吧!''苏桃忙着运力,也没多想。两人正在合作之时,天上忽然起了一阵巨响。苏桃仰头望天,随口说道:``过飞机了?''紧接着她睁大了眼睛,抬手指天:``无心,来飞机了!直升飞机!''

巨响越来越近,最后直升飞机竟是要在校园上空缓缓降落。能够调动直升飞机的人物,如今除了军方,就是中央。苏桃虽然不知对方是什么来头,但是平白无故从天而降,总不像是带着善意。手上疯狂的加了力气,她和无心总算是掰出了一个能够容纳脑袋出入的洞口。

无心动作灵活,踩上窗台俯下了身,脑袋一得自由,后面的肩膀腰腹像蛇一样的游动而出,他一手拿饭盒,一手抓苏桃,迈开大步就往校园围墙跑。跑到半路,后方起了一声枪响。一个尖利的姑娘声音划破了夜色:``解放军来啦!''

无心吓了一跳。把饭盒塞到苏桃怀里,他弯腰扛起苏桃就往墙上送。等到苏桃扒着墙头翻过去了,他也跟着越过了围墙。解放军是来干什么的,他们不知道。但如果军方站在联指一派,不会落地就开枪。不管情况如何,无心决定带着苏桃先避一避风头。小丁猫都没了影,他犯不上留下来吃枪子。

\chapter{墟上阳光}

无心没有跑远,因为想知道指挥部里到底是要出什么事情。直升飞机只有一架,不可能再有士兵从天而降,于是他拉着苏桃躲在暗中,审时度势的走走停停,一条街一条街的撤退。最后他们绕了个远,很巧妙的溜进了一中对面的破厂房里。厂房受过一次炮轰,如今断壁残垣高高矮矮的矗立在月色下,无边无际的占据了很大一片地盘。

无心和苏桃埋伏在半截墙后,看到一中的校门大敞四开,守在指挥部里的人,无论男女,都被刀枪逼着站成了一队。武器也被尽数收缴了,因为指挥部里没有主心骨,所以上上下下都很痛快的投了降。有人高声质问解放军的来历,但是马上就被枪托封住了嘴。

无心和苏桃,因为两人的来历全都不禁推敲,所以对于自由都很看重。眼看解放军把指挥所的一大队人押解走了,他们溜进了一处有棚有顶的空平房里,靠着墙坐下喘气。喘了没有两三口,无心灵机一动,把自己和苏桃臂上的红袖章全摘掉了,团成一团塞进书包里。袖章上带着联指字样,如今联指莫名其妙的被军队一锅端了,他们不能再顶着联指的名义露面。

最后一队解放军也撤走了,楼门和校门全被贴了封条。无心对着苏桃一笑:``明天的日子,又不知道该怎么过了。''然后他一手托了苏桃的后脑勺,借着月光仔细看她脸上的伤:``疼不疼?''苏桃不假思索的答道:``不疼。''顿了顿,她小声的改了口:``有一点点疼。''无心放下了手,对着她苦笑:``打成小花脸了,好在没伤皮肉,慢慢等着淤青退吧。''

苏桃望着无心,看到无心的半边面孔被月光镀了一层温柔的光芒,还看到无心的眼睛是缀着星星的无垠夜空。其实她并不很在乎自己被打成小花脸,因为她如今的身份,和一张丑脸子正相衬。横竖都是不得见光,文化大革命的巨浪,早把她卷到了人间最边缘。

一只野猫在门口向内探头缩脑,见有人在,便竖着尾巴飞檐走壁的逃了。夜里起了风,在房里能听到微微的风声。无心本是靠着墙壁席地而坐,此时便扭头去问苏桃:``冷不冷?''

苏桃缩在旧军装里,``嗯''了一声。无心得了回答,便侧身握住她的手臂往怀里带。双方都是心有灵犀,苏桃顺着他的力道,不言不语的坐上了他的大腿,趴上了他的胸膛。闭上眼睛静静呼吸,她想无心用腿和手臂给自己围了一个家。

无心重新靠上墙壁,歪着脑袋去看窗外的一轮白月亮。苏桃的头发乱了,后脑勺毛刺刺的抵着他的下巴,浓厚长发中分梳开,露出一线热烘烘的青白头皮。一只手搭在他的肩膀上,软软的带着分量,透露出十分的软弱,十分的依赖。

无心轻轻拍着苏桃的手臂,想让苏桃睡一会儿。在他的眼中心中,苏桃是小猫一样小狗一样,小婴儿一样小天使一样;无知无邪,无产无辜。

苏桃的呼吸渐渐平和深长了,显然是已经朦胧入睡。白琉璃无声的爬出书包,盘在苏桃的手臂上昂起头。无心抬手握住他的颈子,然后低头吻了吻苏桃的头发,又抬头吻了吻他的嘴;一颗心忽然无比的苍老了,仿佛苏桃和白琉璃都是他的孩子。

手一松,雪白的蛇头立刻向后一避,白琉璃在黑暗中现了形。大睁着蓝眼睛怒视无心,他似乎是又感觉自己受了冒犯。然而无心抱着苏桃闭上眼睛,很安静的垂下了头。白琉璃凝视了无心片刻,转身去找板砖,没找到,于是附回白蛇身体,决定算了。

天明之后,无心和苏桃从厂房的一侧废弃偏门中出了来。饭盒里的窝头和菜已经被他们分而食之,吃得不饱不饿,反倒逗出了馋虫。天气暖和,夜里露宿也冻不死,于是苏桃感觉活在破厂房里也不错。一手拉着无心的手,她在砖头瓦砾之中很灵活的跳跃行走。

废墟之中,偶尔会有波斯菊在阳光与风中摇摇曳曳。夏天还没到,可是波斯菊已经鼓了花苞。苏桃摇了摇无心的手,指着波斯菊给他看:``我家院子里到处都是它。它可好养了,不用管,自己就能开满一夏天。''

无心深一脚浅一脚的站在废墟里,转身扶她越过矮矮的一堵残墙:``野花嘛,当然好养。''苏桃紧赶慢赶的追逐着他:``不是野花,它有名字的,叫波斯菊。''无心很惊讶:``怎么着?它还是波斯来的?''

苏桃成了个自鸣得意的小女孩,因为有人宠,所以不耐烦:``哎呀,不是的。''说完之后,她偏过脸去看无心。无心也看她,看她右边脸蛋上赫然一道宽宽的瘀伤,正是青中透紫,紫里渗红。迈开步伐继续前进,无心咕哝了一句:``我应该宰了黑背。''

说话的工夫,两人上了大街。街上倒是没有解放军,然而四处可见带着红总袖章的纠察队。无心略略一动脑子,大概猜出了其中前因后果——早就听小丁猫提起过,红总背后是有军方支持的。军队的番号,他记不住,总之任务是从外地过来``支左''。

天下还没有哪家造反派肯承认自己是``右''的,你左我也左,看你军队支持哪一方。显然,在这支军队的眼中,红总为左,联指为右。而在另一方面,省委似乎是另有看法,否则联指在保定的总部不会源源不断的弄来枪支弹药;文县的分部也不会有胆量跑去长安县冲击军械库。

文县肃静而又热闹了,无心在街上走了一圈,听了满耳朵的片言只语,经过一番拼拼凑凑,他得知了这样一个事实:小丁猫已被军方活捉、押回保定;联指总部也受到了极大威胁,很有可能会被定性为反革命组织。

红总卷土重来,单看街上的气氛,也知道今天必定会有一场热烈的庆祝游行;热烈之余,又别有一层恐怖——红总正在满城抓人,凡是和联指有关系的人,如今全成了纠察队的逮捕对象。联指会杀人,红总同样会杀。

无心在空气中嗅到了浓烈的血腥气,心里后悔自己当初不该往文县来。当初抗战的时候,就数冀中平原的游击队打得热闹;打出了成绩也打成了传统;如今农民们放下锄头抄起枪,依然不怯。千里大平原,烽火漫长天。村里打得比城里还热闹。但他一转念,又想自己若是不来文县,现在世上可能已经没有桃桃了。

无心和苏桃进了一家小饭馆,买了二十个烧饼和一盘炒菜,以及一大块咸菜疙瘩,又在水龙头上灌满了水壶。狼吞虎咽的填饱了肚皮,他们将余下的烧饼和咸菜疙瘩揣进书包,挎上水壶要回破厂房去。不料刚一出饭馆,便遇上了纠察队封锁道路。

整条街上的人都老实站好了,一一接受盘问。及至轮到了无心和苏桃,两人乖乖的背了一段毛主席语录,言谈举止都没有破绽。可就在纠察队员转身要走之际,白琉璃忽然从书包缝隙里向外一顶,正是顶出了一团红布。原来他在书包里和咸菜疙瘩作伴,实在是被熏得不能忍受,所以吐着信子想要出来透一口气。不料一时慌张,他竟然一头顶出了书包里的私货。

纠察队员弯腰捡起红布,展开一开,正是印着联指字样的两只袖章。双目放出凶光,他像见了宝贝似的盯住无心和苏桃,同时大喝一声:``来人啊,又逮着两条漏网之鱼!''无心和苏桃全傻了眼,没想到白琉璃会如此添乱。

立刻有人端着步枪冲上来了,吆喝着让他们自己往前走。路口停着一辆大卡车,卡车后斗站满了灰头土脸的乘客,全是红总抓到的联指分子。众目睽睽之下,没有逃脱的可能。无心和苏桃垂头丧气的爬上卡车,知道自己这两条漏网之鱼,这回是要进油锅了。

苏桃苍白了脸,心里想起了田小蕊。很留恋的又看了无心一眼,她冷静的下了决心。她不走田小蕊的路,一旦察觉到了危险,她会像爸爸一样,自己给自己一个痛快。

街上的盘查还未结束,但是大卡车装满之后就发动了。无心用心记着沿途风景,直到大卡车把他们运入了机械厂。机械厂和钢厂遥遥相对,分别位于文县两端。和钢厂一样,机械厂也停工了。红总一夜抓了上千的人,一边抓,一边自行寻找监狱。好在文县最不缺少的就是工厂,工厂里面,空置的厂房也有的是。

一卡车人被纠察队员用刺刀撵进了一间厂房。厂房先前不知是放什么大机器的,上下足有两三层楼高,从天花板向下半米处,开着方方正正的窗口,窗口倒是没焊铁条,因为高得犹如天窗,一般的贼根本连窗户边都摸不着。

顶天立地的大铁门喀喇喇的关严了,几十名男女像蝼蚁一样,沉默的或站或坐。唯有无心仰头望着窗口,认为自己并非真是死路一条。把苏桃拽到自己身边,他弯腰对着她嘁嘁喳喳的耳语了一阵。苏桃听到最后,半信半疑的睁大了眼睛,末了抬头一望窗户,她缓缓的摇了头,压低声音说道:``无心,不行啊,万一半路掉下来,非摔死不可。''无心一拍她的后背:``夜里你等着瞧吧,我说能爬,就真能爬。''

无心和苏桃在厂房里混了一天,其间大门完全不开,吃喝拉撒全是自己想办法。无心和苏桃就着咸菜疙瘩吃了烧饼喝了凉水。白琉璃知道自己闯了大祸,悻悻的趴在书包里不肯动。倒是无心没有闲心和他计较,捧着书包摸着白琉璃,他趁着无人注意,和白琉璃秘密交谈了一阵,给了白琉璃一个将功补过的机会。

及至到了入夜时分,内外还是一片寂静。眼看周遭众人都是一副心如死灰的德行,大门也的确是关得铁桶一般严密,无心紧了紧鞋带腰带,又把书包挎好了。双手拍上墙壁,他纵身向上一跃,壁虎一样贴上了墙。

苏桃虽然事前和他商量妥了,可是如今真要行动,还是感觉没有成功的可能。效仿无心扑上水泥墙,她本是预备着直接碰壁,不料仿佛有股子力量在下方托着她护着她似的,她居然成功的真贴上了墙。

与此同时,无心已经手足并用的爬出一段高度。低头向下望了一眼,他见白琉璃把苏桃举得很稳,便放了心,摇头摆尾的继续向上。厂房里有人没睡,张着嘴瞪着眼去看无心和苏桃,以为自己是在梦里见了妖怪。

无心早就发现自己爬比走快,水泥墙壁粗糙不平,更是适合他攀登。一鼓作气靠近了窗户,他停下来歇了口气,随即向上一窜,把脑袋直接伸出了窗子。只听``咚''的一声,他额头一痛,竟然是合人迎面撞了个顶头碰。窗外随即响起一声惊叫,脑袋的主人在他一撞之下,一扬双臂倒栽下去。

无心大吃一惊,手按窗台向外张望,就见一副钢梯搭在厂房外墙上,梯下地面站着一群手持电筒的军装青年。而一名彪形大汉在梯子中段使了个手舞足蹈的鲤鱼打挺,竟然不但阻住下滑之势,而且双脚用力一蹬梯子,凌空一个跟头翻回了站立之姿。

无心一声没吭的缩回脑袋,知道自己是撞在了枪口上。然而钢梯上的大汉不依不饶,仰天长吼:``上边的小白脸,你给老子滚出来!''

\chapter{武林高手}

彪形大汉动如脱兔,三下五除二的攀援而上,把个脑袋重新插回窗口,正好看到无心肚皮贴墙在往下溜。大汉生得虎背熊腰大脑袋,不能轻易通过窗口,于是探头进去,居高临下的伸手一指无心,虎啸似的吼道:``好小子!我看清你了!''

无心仰着头,恨不能哭一场。早知如此,不如不逃,被人堵了个正着,罪过更大了。

大汉缩回脑袋下了钢梯,带领人马绕过厂房。一时间厂房内外的电灯全通了电,照耀得方圆几里地内灯火通明。两扇大门缓缓而开,守门的红总战士像真正军人一样打了个立正,昂首挺胸的做出了夹道欢迎的姿态。而大汉在一队绿军装的簇拥下进了厂房,一只手叉着腰,另一只手指向前方:``是你吧?''

无心和苏桃刚刚落地不久。苏桃躲在无心身后,无心无处可躲,只好在骤然亮起的灯光中一点头:``是我。''

大汉收回了手,摸着下巴翻着白眼往窗户上望:``我说,你是怎么爬上去的?''

无心被他问住了:``我就是\ldots{}\ldots{}慢慢爬的。''

大汉仔细的观察了对面墙壁,见墙上既无绳索也无坑凹,连根能借力的排水管子都没有。光秃秃的一大面水泥墙,实在不是人能爬的。不以为然的一扬眉毛,他挥了挥手:``你再爬一遍给我瞧瞧。''

无心回头向苏桃递了个眼神,然后不情不愿的转身走向墙壁。苏桃低着头要往一旁躲,然而并未逃过大汉的火眼金睛。大汉看了她一眼,登时一惊:``我的娘,好这半脸胎记,青面兽啊?''

与此同时,无心开始爬墙。仿佛手脚胸腹都带着吸盘似的,他周身肌肉一起运力,四脚蛇似的往上蹭,速度还挺快。爬到一半他回了头:``还爬吗?''

大汉双手叉腰仰起脑袋:``嘿嘿,有点儿意思!''随即他伸出大巴掌一招:``下来吧!再爬就到了顶,你还不又得跑了?''

无心一个转身,从半空中直接跳了下来。落地之后他搓了搓手,对着大汉犹犹豫豫的问道:``请问您怎么称呼?''

大汉对于无心的斯文嗤之以鼻。垂下眼帘看了看自己的手,他仿佛预备着要扇谁一个大嘴巴:``我就是陈大光。你们这帮联指的狗崽子,不应该不认识我吧?''

此言一出,厂房内的联指人员一起冷了面孔,表示自己与陈大光这个首席敌人势不两立,唯有无心既无信仰也无骨气,立刻陪笑一弯腰:``陈司令,久仰久仰。''

陈大光一瞪眼睛:``你这王八蛋可是够怪的,怎么一张嘴就像个国民党反动派?你说,你他妈在联指是干什么的?''

无心走投无路,只好一味的柔顺:``报告陈司令,我没干什么,我就帮着宣传队抄了几天大字报。''

陈大光点了点头:``哦,怪不得呢,原来是个臭知识分子!''

无心生平第一次被人赞为知识分子,虽然知道这四个字现在不是好话,但是想了一想,还是感觉有些惭愧:``不敢当,抬举了。''

陈大光没理他,扭头对身边的人发表评论:``真他妈像国民党反动派。我要不是看他有几手真功夫,现在就把他的脑袋拧下来。''

在得到随从的附和之后,红总的陈大光司令环视了周遭情景,感觉联指的狗崽子们坐牢坐得太舒服,于是下了命令,让人把一整天水米未沾牙的狗崽子们押出厂房,跪在一片瓦砾堆上等天亮。无心和苏桃被留在了厂房里,因为陈大光来了兴致,要和他练练拳脚。而无心趁机说情,把苏桃也留在了身边。

陈大光脱了外面的旧军装,露出里面一身半袖汗衫,汗衫背面还印着数字,乃是去年春季机械学院运动会的福利品。原来陈大光本是机械学院内的四年级学生,虽然名义上是大学生,其实学问很有限,是因为中学时篮球打得有点成绩,作为特长生被机械学院录取的。陈大光的老家在沧州,沧州是个尚武的地方,老老少少都会两下子。陈大光练了十几年螳螂拳,平日深藏不露,直到去年夏天风云突变,他感觉自己有了用武之地了,才开始公然的大展身手。本来红总成立之时只有三个人,他,他上铺的兄弟,以及上铺兄弟十三岁的小弟。陈大光立下壮志,在各种公共场合做螳螂状,对各路牛鬼蛇神以及不臣服他的革命小将进行无差别攻击。所以红总的队伍是他凭着一双手打出来捏出来的,只要他在,红总即便是被联指赶进村里了,也依旧众志成城,绝无分裂的危险。

陈大光打到如今,自认为一身功夫在河北地界应该是天下无敌了,又由于革命重担压在肩,他无暇往远了走,故而在无人可打无肉可吃之时,常有寂寞如雪之感。如今逮到一个会飞檐走壁的反革命分子,于他来讲,简直就是个绝佳的玩具。下令把厂房大门一关,他摇头晃肩甩手甩脚,非要和无心切磋一番。无心见了他筐大的脑袋斗大的拳头,深知单打独斗的话,自己很可能被他捶成馅饼;于是提起精神,随时预备着上墙。

空旷的厂房里面,响起了虎虎的拳风。苏桃抱着书包坐在墙角,看得傻了眼。如此足过了一个多小时,陈大光终于意识到单用拳头是不行的了,于是立刻推门出去,就地抄起一根钢筋当做齐眉棍。除了螳螂拳之外,他是刀枪棍棒全会用。手握钢筋大踏步的回了厂房,他一个跟头翻到水泥墙前,举着钢筋开始往上戳:``你妈×,到底下不下来?''

无心贴在墙上:``我下去还不让你打死了?''

陈大光在水泥墙上敲出一串火星:``我告诉你,从开始到现在,你就没落过地。你再不下来,我一棍砸死你那个青面兽!''

无心低头看他:``陈司令,那还是个小孩儿呢,你别吓唬她啊!''

陈大光不动声色的后退几步,随即一个助跑猛然跳起,一钢筋就把无心敲下来了。无心就地一滚,顺势抱着脑袋缩成了一团。陈大光绕着他走了一圈,末了拄着钢筋抱怨道:``你说你是个什么东西嘛!我还以为我找到了对手,没想到你是个刺猬。你说吧,你是怎么个意思?是要和我顽抗到底啊,还是打算向我求饶?''

无心侧过脸,向他露出了一只眼睛:``我想求饶。陈司令,你放了我们两个吧。我们在联指就是打杂的,联指散了,我们另找活路去。''

陈司令,因为知道他轻功非凡,所以愿意和他多谈几句:``你准备找什么活路?''

无心把两只眼睛全露出来了:``我们是出来串联的学生,路上走散了,就剩了我和她还在一起。前两个月刚到文县,我们就被联指的人抓起来了。因为我会写毛笔字,又没什么问题,所以才被他们留在了宣传队抄大字报。我们身上的证明全被联指的人收走了,现在要什么没什么,回家都没钱买车票。要说以后怎么办,我也不知道。我想带着她慢慢往北走,反正家里也没人管我们,我们不着急,走多久算多久吧!''

陈大光用钢筋杵了他一下:``你家是哪儿的?''

无心抬起了头:``黑龙江。''

陈大光又问:``那地方挺冷吧?''

无心立刻点头:``是,冷。''

陈大光继续问:``有师父吗?''

无心摇了头:``没有。''

陈大光拄着钢筋傲然而立,还想说话,然而未等他开口,忽然有人推开大门,把他叫走了。

他从出去到回来,其间只用了不到半个小时。可就在这半个小时的工夫里,无心带着苏桃又逃一次,逃成功了。

无心没地方去,身上没有证明和介绍信,想住旅馆也不能够。于是趁着夜色,他们又回了一中对面的厂房废墟里。这里邻着联指的指挥部,最危险也最安全。把苏桃安顿在破房子里,无心爬墙进了校园,从食堂里偷运出了许多食物;又攀着排水管上了三楼,推开窗户进了宿舍区,随便抱出了一床棉被。

大包小裹的回到苏桃面前,两人围着棉被偎在了一起,面前盘着白琉璃。苏桃很快乐,无心便陪着她快乐。两人各自对着白琉璃伸出一根食指,无心问道:``娘子,你要我们哪一个?''

白琉璃的力量虽然强大,但也不是无穷无尽。此刻颇为疲惫的撩了二人一眼,他张嘴衔住了苏桃的指尖。

无心笑了:``娘子,你选错啦!我是男的,桃桃是女的。''

苏桃把手指从蛇嘴里抽出来,同时小声对无心说道:``你是许仙,我是小青。白娘子本来就是先和小青在一起的。''

然后她扭头去看无心:``我们要是永远都能在这里过日子就好了。''

无心听了,摇头一笑:``孩子话。这里好像垃圾堆一样,哪能长住?''

苏桃摸着白琉璃的脑袋,不说话了。

到了第二天,无心和苏桃不敢露面,在一堵墙后晒着太阳吃水果罐头。白琉璃长长的躺在阴凉处,头上倒扣着一朵半开的粉色波斯菊,是苏桃给他找来的遮阳帽。

苏桃从来没有这样肮脏狼狈过,与此同时,她又有种劫后余生的幸福。她没敢说,因为一旦说了,就会被无心归类为``孩子话''。

忽然转向了无心,她开口问道:``那个爱装螳螂的人,还会再抓我们吗?''

\chapter{新的阵营}

春雨下起来了,沙沙的落,润物细无声。波斯菊和荒草一起碧绿了,微绽的花苞被细茎子向上托举着,越托越高,一直越过残留着碎玻璃的窗台,颤巍巍的活动在窗内苏桃的身边。

苏桃已经三天没洗脸了,水太有限,只够喝的。她灰头土脸上的青紫瘀伤已经不再作痛,但是颜色越发浓重,青紫下面透出红色的血点子,瘀伤边缘则是隐隐的泛黄。仰头望着无心,她看无心的面孔和手指。无心也是三天没洗脸,然而并不算脏。

一段毛线绷在他的修长手指上,东拉西扯是个复杂的图形。``看看,我翻了个`板凳'。''无心对着苏桃笑道:``轮到你了。''苏桃收回目光,用双手小指勾上了毛线。小雨天,一段毛线也够他们翻小半天的花绳。手指主动一挑,反被毛线缠住。

苏桃忽然不想玩了,抬起一只手搭上无心指间纵横的毛线,她举起另一只手,摸了摸无心的眉毛。指尖从眉头画到眉尾,她活了十五岁,无心是她见过的最漂亮的男人。

无心以为是自己的眉目脏了,所以俯身歪了脑袋,闭着眼睛任苏桃为自己清理。苏桃用手指肚轻轻掠过他的睫毛,他缓缓的睁了眼睛,睫毛扫过她的心。心里满满的,有风有雨有晴天,鼓荡着怦怦跳。她扭头望向窗外,窗外的闲花野草断壁残垣,都被小雨洗刷得好干净,像无心一样干净。

废墟里也不安静,下午小雨刚停,远方的大街上就起了锣鼓喧天的热闹。天天都有游行,天天都有庆祝,因为文县刚刚成立了革命委员会。年初王洪文在上海成立了全国第一个革命委员会,开了个轰轰烈烈的头,从此革委会如同雨后春笋,开始在全国各地萌芽。

各级政府全被打倒了,革委会就是革命化的新政府。陈大光卷土重来回到文县之后,第一是``宜将剩勇追穷寇'',满城扫荡联指分子;第二便是占据了先前的县政府大院,匆匆忙忙的建立起了革命委员会,自封主任,等于过去的县太爷。

其中的道理,不要说是在学院里混过四年的陈大光,就算换了村里的大队长小组长,也是一样的能明白——有些甜头就是先到先得,谁先在文县站稳脚跟了,上头就承认谁;如果谁都站不稳,始终是混战,那上头兴许直接派下军队,把一县的冤家们通通镇压。

无心不敢上街,天天靠着一中食堂过日子。食堂里存留的剩馒头干饼子很快就被他们吃光了,余下的罐头倒是还有不少。罐头本来是稀罕物,可是天天吃也受不了。大中午的,无心袖着双手晒太阳,很想吃口新鲜的热饭热菜。

废墟上偶尔会有大老鼠经过,他舔着嘴唇,心想抓只老鼠烤烤吃了也不错,不过苏桃还在身边呢,当着个小姑娘吃老鼠,未免有点不好意思。苏桃坐在他的身边,双手捧着个大玻璃瓶,仰头去喝瓶中剩下的水果汁。白琉璃趴在一旁,刚刚吞了一块很大的罐头牛肉,撑得肚皮有些变形,并且完全爬不动了。

正是万籁俱寂之时,两人忽然听到有汽车由远及近的驶向了一中。苏桃吓得立刻放下了玻璃瓶子,又把白琉璃拎起来塞进书包。无心则是转身从矮墙头上露出一双眼睛,远远的望向一中门口。

一中门前的小街,已经是寂静很久了,平日除了猫狗之外,再无生机。两辆大卡车一前一后的停在校门外,有穿着绿军装的青年跳下卡车后斗,背着步枪大踏步上前去撕封条。

无心缩了下去,对苏桃小声说道:``应该是红总的人,可能是来搬东西的。''苏桃伸手指了指不远处的破房子:``我们进去躲躲吧。''无心正要回答,忽然感觉身旁有异。扭头一看,他大吃一惊,只见一只肥硕的大狼狗站在瓦砾堆上,正支愣着一对耳朵看人。未等无心做出反应,大狼狗狗嘴一张,很响亮的吠出了声。

无心一个箭步就扑向了它,想要掐住它的脖子。然而狼狗也是相当的机灵,并不肯坐以待毙。一瞬间的工夫,它又狂吠了一大串,早惊动了街上的人员。有人吆喝着跑向了废墟,一边跑一边端起步枪,也不警告,直接扣动扳机扫射了一排子弹。

一排子弹是贴着无心的头皮飞过去的,无心抱着狼狗,当即无条件投降。又因为知道自己和苏桃形迹可疑,对方满可以实行无产阶级专政,把自己和苏桃就地正法;所以放了狼狗举起手,他对着来人说道:``我要见陈大光。''

青年绿军装吼道:``要见陈主任?陈主任是你想见就能见的吗?''无心立刻答道:``我陪陈大光练过拳,他知道我。拳没练完我就走了,他可能还在找我呢!''绿军装半信半疑的看着他,手指还扣在扳机上。

半个小时后,被反绑了双手的无心和苏桃,以及从一中楼内运出的几套好桌椅,一起上了卡车。卡车把人和物全运进了革委会大院,陈大光站在院内,毫无准备的和无心相见了。

``哟!''他像一根擎天柱似的矗在院子里,上下打量无心:``你?''无心含羞带愧的对他一笑:``陈\ldots{}\ldots{}主任,是我。''陈大光又撩了苏桃一眼,感觉这丫头蓬头垢面,已经彻底没法看了:``你跑哪儿去了?''无心斜着眼睛盯着地面,意意思思的答道:``我们也没地方可去,就在一中对面的废墟里住了几天。刚才我们正靠墙晒太阳呢,没想到让狗逮住了\ldots{}\ldots{}''

话音未落,一路押解他们的绿军装当场怒不可遏:``你骂谁呢?''无心转身向他一点头:``我没说你,我说的是真狗。''绿军装性如烈火,不堪受辱:``什么意思?谁是假狗?''无心连着几天没吃好喝好,精神有点恍惚:``没有假狗,全是真的。''

陈大光自从做了革委会主任之后,已经迅速培养出了一点官威。此刻一眼皮把绿军装弹开,他背着双手去问无心:``你来找我干什么?''无心无精打采的答道:``我怕被他们当成联指分子,所以\ldots{}\ldots{}''陈大光一瞪眼睛:``所以什么?''

无心想了想,随即继续说道:``所以陈主任,我想和你打个商量。你给我们一天三顿饭,我随时陪你练功夫。除了练功夫之外,我还可以负责给你打杂跑腿干零活,行不行?''陈大光抬手挠了挠头,发现无心只要一开腔,自己就要梦回旧社会。换了个双手叉腰的姿势傲然而立,他找到了一点地主老财的感觉,因为面前正站着一个新出炉的狗腿子。

陈大光是习武之人,对于无心的轻功,他是相当的高看。把个高手推出去毙了,未免太可惜。但是不毙,又实在是太便宜了他。摸着下巴眨巴眨巴眼睛,陈主任遇到了一道无解的难题,有心一拳把无心击飞,可是凭着无心的速度,他又很有可能是一拳打空,当众出丑。

等到卡车上的木器家伙全被人搬进革委会房里了,司机也把卡车开出大院了,陈大光才终于又出了声:``你打算下次什么时候跑?''无心对着他一弯腰:``不跑了,我们连饭都吃不上,还能往哪儿跑啊。''

陈大光点了点头,随即竖起两根小棒槌似的手指:``我对你有两句话。第一,收起你这副国民党反动派的臭德行!老子最看不惯小白脸,你再敢和老子装神弄鬼,老子弄死你!第二,老子不用你舞文弄墨耍笔杆子,你向后转,看见门口的小房没?你滚进去,给老子看大门吧!还有你带的这头青面兽,自己想法子安排。我们这是革委会大院,管不了你们这帮牛鬼蛇神,知不知道?''

无心不动声色的松了一口气——自己在联指混了好几十天,身份来历又都不明,能在革委会大院得个看守大门的差事,已经算是走运了。

县政府是一排整整齐齐的平房,无心初到文县之时,曾经翻越后墙,从被红总征用的政府办公室里偷了公章粮票以及瓜子柿饼。陈大光不讲排场,只看历史。走在县政府的大院子里,他身心愉悦,很有一种光宗耀祖的自得。

先前给县政府大院守门的老头子,因为儿子在联指中是个头目,所以如今全家都是生死不明。无心占据了收发室小屋,忙了一个下午之后,便尽数掌握了工作内容。革委会大院门口有站岗的民兵,重要事务也轮不到他经手,他只要负责收清报纸信件、早上再扫扫院子就可以了。

收发室里只有一张单人床,到了晚上,无心没了主意。苏桃毕竟是个姑娘,两人睡一个屋倒也罢了,真挤一张床,还是一张窄窄的单人床,可就太不合适。无心找了几张旧报纸铺在地上:``桃桃,你睡你的,我打地铺。''

苏桃下午洗了头发,耳朵脖子也擦干净了:``无心,地上太凉。我们头脚颠倒着睡吧,头脚颠倒了就不占地方。''无心往报纸上一躺,又把苏桃脱下的外衣卷成一卷塞到头下:``我先对付一宿,要是真冷,明天再说。睡吧睡吧,今天算我们运气好。远的也不想了,我们先吃它几天再说。''

无心在地上熬了一夜,翌日凌晨就醒了。革委会里也有食堂,凌晨还未开伙,但是热水彻夜都有。无心出去灌了一水壶开水,回房之后慢慢的喝。扭头看了苏桃一眼,房内阴暗,苏桃躺在床上,睡得正酣。无心淡然的把脸扭开了,扭到一半,他猛的又转向了苏桃,发现苏桃的被窝里伸出了白琉璃的圆脑袋。

无心蹑手蹑脚的走上前去,把白琉璃从被窝中缓缓的抽出,然后将其打了个蝴蝶结,一弯腰扔到床底下去了。

\chapter{革委会生活}

五月的午后,空气中已经隐隐有了夏日味道。无心蹲在收发室窗外的小黑板前,蓝布工人装的上衣已经脱掉了,露出里面一件白里透黄的短袖汗衫。一手拿着一沓子信,一手捏着半根白粉笔,他把收信人的名字整整齐齐的抄上小黑板,以便往来的工作人员可以自行取信。

最后一笔未落,他猛的一跃而起窜上了窗台。而陈大光一击未中,当即收手,带着身边几名随从施施然的走出大门去了。无心跳下窗台,描完最后一笔,然后把小黑板挂在了窗旁一根突出的钉头上。开门回房把信送进桌上的纸盒子里,他对着苏桃一笑。

苏桃坐在床上,正在翻看没人要的旧报纸。无心顶着投机倒把的罪名,想方设法的换了一丈多的布票。拿着布票和钞票去了百货商店,他给苏桃买了一身的确良衣裤。蓝衬衫黑裤子,除了衬衫是个圆领子,其余没有一处带着女性气息,真是没什么好看的,不过的确要比旧军装凉快。

苏桃脸上的青紫瘀伤也日益淡化了,偶尔随着无心出出入入,已经会有人格外留意的看她。陈大光昨天才真正意识到了苏桃的存在,他背着手问苏桃:``你那脸上,不是胎记啊?''苏桃被他衬托得十分渺小,低下头蚂蚁似的嘤嘤嗡嗡:``不是。''

陈大光一皱眉头:``你多大了?差不多就和无心扯个证吧!不明不白的总在一间屋里住着,也好说不好听不是?''苏桃红着脸,从嗓子眼里``嗡''了一声。

等到陈大光走了,无心拿着一根红豆冰棍回来了,苏桃关上门,伸手一扯无心的袖子:``刚才陈主任来了。''无心自从有了苏桃,天天防贼似的防备各路男人,听闻此言,便是一惊:``他说什么了?''

苏桃松了手,面红耳赤的答道:``他说\ldots{}\ldots{}他说让咱俩扯个证。''无心一愣:``证?什么证?''苏桃满头满脸的发烧:``好像是\ldots{}\ldots{}结婚证。''无心松了口气:``扯他的蛋!你没说你岁数不够吗?''苏桃摇了摇头,嗫嚅着说道:``没有。''

无心把红豆冰棍送到苏桃手里:``吃吧,下次再有人问你这事,你就不吭声。我发现这世道装疯卖傻也是条活路。你猜我刚才遇见谁了?我在胡同里撞见了招待所里的那个精神病所长。那家伙买了面包香肠汽水,正偷着吃呢!他这精神病可是挺俏皮,不但不用上批斗会,而且有工作有饭吃,没事还能溜出去改善伙食。''

苏桃把红豆冰棍举到无心面前,让他先咬了一口,然后心事重重的坐回床上,一边翻报纸一边舔冰棍。白琉璃懒洋洋的趴在床角,一双黑豆眼睛雾蒙蒙的覆了白膜。无心走到床边,把他捧起来送到一盆温水中——白琉璃要蜕皮了。

白琉璃生怕他又要把自己往床底下送,当即在盆里翻江倒海表示抗议。无心无可奈何的蹲在盆前,用手一点一点的往他身上撩水:``眼睛都蒙瞎了,还和我闹。''

苏桃扭头问道:``过两天,是不是一定能复明?''无心微笑点头:``一定能。等他眼睛亮堂了,就要开始蜕皮了。老皮一蜕,他又能漂亮不少。''苏桃跟着笑了:``白娘子现在也挺漂亮的。''白琉璃觅声抬头,去找苏桃。无心在他的头顶上连弹几指,弹得白琉璃一阵乱点头:``趁着水没凉,你乖乖给我趴下多泡一泡。''

白琉璃目不能视,泡完温水澡后就急急的爬回了床上,吐着信子往苏桃怀里钻。蜕皮之前的感觉实在是不舒服,所以他很需要一点温柔的呵护。无心对他一贯不温柔,要说呵护,也是重手重脚,哪像苏桃不是夸他就是摸他?

无心端起水盆,斜着眼睛骂道:``不要脸的,往哪儿钻呢?''白琉璃从苏桃的衬衫下摆中探出了脑袋。苏桃以为他是要给自己做腰带,故而满不在乎:``白娘子和我亲呢!''

无心有话不好说,又不能和一条蛇纠缠不休,无奈之下,只得姑且出门去泼了水。拎着盆正要往回走,前方的平房门口出来了人,乃是革委会的副主任朱建红。

朱建红是二十七八岁的年纪,本是机械厂里的播音员,生得颇为俊俏,尚未成婚,每天无微不至的关怀着陈大光。一周总有个一两晚要向陈大光单独汇报工作,非到鸡叫汇报不完。无心心如明镜,每逢主任和副主任要秉烛夜谈了,自会关好大门,熄灯睡觉。

朱建红把无心叫到面前,让他去给自己打一暖壶开水。无心跑了一趟水房,把开水给她拎进了办公室。朱建红颇为热情,从抽屉里抓了一把红枣给他。他没推辞,双手接了。转身出门回了收发室,他对苏桃说道:``桃桃,给你吃枣。''

苏桃正在屋里扫地,忽然见了红枣,就很高兴:``呀!哪儿来的呀?''无心接过了她的扫帚:``别人给的,吃吧。''苏桃像只耗子似的,一枚枣啃半天,舍不得快吃。及至到了傍晚,革委会都下班了,大院也空旷了,她嘴里还含着一枚枣核不肯吐。忽见陈大光带着一群委员从外面回了来,她连忙一闪身,躲进了房内。

朱建红出门迎接了陈大光,众人在院内谈笑风生,直到无心拿着两个馒头出现在了大院门口。陈大光一回头看见他了,当即对他一招手:``你干什么去了?''无心一举手里的馒头:``晚上食堂不开伙,我去买了馒头当晚饭。''陈大光继续招手:``过来过来,陪我练两招。今天我欺负欺负你个没吃饭的,看看我到底能不能逮住你。''无心把馒头送回收发室,然后独自走到了陈大光面前:``行,练吧。''

周围观众登时散开,陈大光脱了上衣往朱建红手中一甩,露出一身起伏分明的腱子肉,胸前赫然一枚毛主席像章,正是别进了皮肉里。对着无心做了个螳螂捕蝉式,他在众人的叫好声中猛然出击,一瞬间就把无心给吓跑了。

接下来,无心逃啊逃,主任追啊追。革委会的院子太大了,两个人一前一后转着圈跑。陈大光猫腰伸着两只手,抓鸡似的对无心进行围追堵截。最后无心走投无路要跳墙,被陈大光眼疾手快的攥住脚踝,把他从墙头一把拽了下来。千辛万苦逮着人了,陈大光兴奋至极,当即在无心身上大展拳脚。及至他打痛快了,无心蜷在地上,已是一动不动。

陈大光从朱建红手中接了上衣穿好,弯腰拍了拍无心的后脑勺:``哎?死啦?''无心低低的哼了一声,慢慢的垂头坐起了身。陈大光仰天大笑:``你可没跑出我如来佛的五指山吧?''无心抱着膝盖,平白无故的挨了一顿胖揍,从头到脚无一处不痛。

而陈大光兴高采烈,用脚尖又踢了踢他:``你也算是不错了,放心,虽然你原来跟联指干过,但是我不和你翻旧账。只要你是真革命,我就敢收你。联指的小丁——丁什么来着?猫还是狗?反正他们的头儿骂过我们是牛鬼蛇神总司令部,就是因为我们不挑拣嘛!今天呢,我也不让你白陪我练。一会儿我们去吃饭,带你一个。''

话音落下,他兴致高昂的又对身边人说道:``这几天大家也辛苦了。晚上的批斗会加个项目,斗斗破鞋轻松一下。''众人听到``斗破鞋''三个字,立刻快活的哄堂大笑了。

陈大光让无心随行,无心不敢不去。回房向苏桃嘱咐了几句,他跟着陈大光等人出了门。在招待所的餐厅里吃了一顿鱼肉之后,他们果然前往机械学院,参加了当晚的露天批斗会和小丁猫相比,陈大光显然属于粗豪一派。血雨腥风的批斗会一结束,为广大群众喜闻乐见的斗破鞋就开始了。本县有名的破鞋们排队上了台子,逐个讲述自己风流经历,而且十分具体,听得陈大光哈哈大笑,又拍巴掌又拍大腿。他上铺的兄弟、红总元老之一忽然站起身,高声嚷道:``不对,重说!你俩到底是谁先脱的裤子?''

一个白白净净不到三十岁的青年破鞋站在台子上,因为被斗过太多次了,所以十分麻木:``他非得要和我亲嘴,一边亲嘴一边脱裤子,我说不行,他说没人看见\ldots{}\ldots{}''斗破鞋的时候,台上台下没有孩子,全都是结了婚的大男大女和老男老女,一个个听得嘻嘻哈哈,比看戏还来劲。

无心看了一场斗破鞋,听得心猿意马。午夜时分他回了革委会大院,苏桃已经在靠墙的小床上睡了,身体紧贴着墙壁一侧,是给无心留出的位置。无心虽然不大上床,但是有时夜凉,他也会在苏桃身边挤一挤。

轻手轻脚的在地上铺了报纸躺好了,无心弓着腰睁着眼,裤裆长久的支着帐篷。白琉璃忽然浮现在了半空中,影子微微的有点模糊,因为控制一条要蜕皮的懒蛇很费精力。居高临下的审视了无心,他开口问道:``你想女人了?''

无心侧卧在报纸上,没出声,只望着白琉璃点了点头。白琉璃看了苏桃一眼:``你不会想\ldots{}\ldots{}''无心摇了摇头。对于苏桃,他是长兄如父。白琉璃又问:``我去找个女人给你?''无心继续摇头,然后闭上眼睛,扭头把脸埋进了臂弯里。

翌日凌晨,无心早早起床,出门扶着大笤帚扫院子。扫过院子之后,他开了大门。开始有人络绎来了,一天的报纸和信件也到了。苏桃端着饭盒去食堂打饭,无心照例蹲在小黑板前,抄写收信人的名字。抄着抄着他忽然一怔,因为发现最后一封信的收信人竟是自己。

他没声张,挂好小黑板之后回了收发室,偷偷的撕开信封展开信纸。信是马秀红写的,不知怎的知道了无心的下落,很诚恳的请求无心帮忙联系县内同志。信的末尾附了一个通信地址,原来马秀红人在保定,并没有陪着小丁猫去蹲大狱。

无心拿着信思索片刻,末了划根火柴,把信烧了。他能确定陈大光对自己存着一点爱才之心,可是始终猜不透小丁猫对自己到底是什么意思。小丁猫对他的庇护一直笼罩着一层不知吉凶的神秘色彩,所以他宁愿留在革委会看大门。

\chapter{夜色惊心}

午夜时分,无心睡不着觉,坐在收发室门外看星星看月亮。在大院的另一端,一间办公室刚刚熄了灯,想必是陈大光与朱建红谈工作谈到了新阶段,要开始真抓实干了。

收发室里很安静,苏桃还在长身体,只要天下太平,她就不由自主的要贪吃贪睡。一只来历不明的小蛤蟆跳出草丛,蹦上了无心的脚面。无心当即一抖腿,嘴里轻轻的斥了一声``去'',小蛤蟆翻滚落地,呱呱叫了两声,当真离去了。

小蛤蟆刚走,白琉璃又回来了。最近他做蛇做得很辛苦,蛇皮蜕过嘴巴之后便再没动静,以至于他每天缠在无心给他预备好的一捆粗糙树枝上,烦躁不堪的蹭来蹭去。白天既是十分难熬,夜里他便必定溜出蛇身,轻轻松松的四处游荡一番。披头散发的悬在空中,他兴致很好的告诉无心:``有两个人正在那边的屋子里生小孩。''

他当初找女人是为了生小孩,所以以己度人,把一切男欢女爱的行为全都统称为生小孩。无心坐在门前的一级水泥台阶上,垂着头闷闷的答道:``这和我有什么关系?''白琉璃缓缓下降,与他高度齐平:``那个女人,好像是很喜欢男人。等到那个男人走了,我可以把她带出来给你。''

无心压低声音告诉他:``你不懂。男的是革委会主任,我是个看大门的。那个女人再喜欢男人,也不可能看上我。就算你把她带到我面前了,她也至多是给我一个大嘴巴。''白琉璃认认真真的想了一想:``那我把她杀了,她就不会打你了。''无心立刻摇头:``和死人相好,我疯了?''

白琉璃发现无心还挺挑剔。眼看无心天天夜里不睡觉,挺着下身一根棒槌在外面当猫头鹰,他于心不忍,实在是想伸出援手:``有办法了。''他郑重其事而又自鸣得意的告诉无心:``我可以上她的身。我上了她的身,你想让她怎么样,我就让她怎么样。''

无心终于抬头正视了白琉璃。直勾勾的看了半晌,他清了清喉咙,侧身扶墙站起了身,低声答道:``不了,你的好意,我心领了。''

白琉璃看他神情有异,不禁莫名其妙:``真不要吗?''无心慢吞吞的转身背对了白琉璃,颇为尴尬的答道:``你如果上了她的身,那我睡她和睡你不是一样的了?你我几十年的交情,我实在是\ldots{}\ldots{}下不去手。''低头用鞋尖轻轻踢着地下一块小小石头,他很羞涩的又笑了一下:``再说\ldots{}\ldots{}你可能是不知道,其实我有点怕你。

话音落下,他只听耳后一阵劲风。一声巨响震动脑髓,他被白琉璃用小黑板拍在了墙上。白琉璃一片赤诚,想要为他排忧解难,不料他一肚子花花肠子,居然踢着石头往邪里想。三下五除二的把他拍倒在地,白琉璃气冲冲的回了房,钻回蛇身睡觉去了。

无心趴了半天才缓过这口气。慢吞吞的坐起来,他一腔骚动的春情被拍得一丝不剩,十分冷静的喃喃骂道:``他妈的,我说什么了?怎么还动了手?我活得真够冤,人打我,鬼也打我。''

无心在一只不肯远离的小蛤蟆的陪伴下,抱着脑袋忍痛,直到前方陈大光的办公室又亮了灯。陈大光发泄过革命热情之后,通常要到院子里的公共厕所撒一泡尿。无心不想和他打照面,于是起身开门,悄悄的回房去了。

再说陈大光在厕所里放水完毕,回到办公室和朱建红又噼噼啪啪亲了几个嘴。潦草的披上一身绿军装,他坐在椅子上弯腰系鞋带。朱建红站在一旁,一边把手伸进衣服里整理胸罩,一边说道:``半夜三更的还回去干什么?怎么着?下半夜还有人等你?''

陈大光在革委会附近有套住房,步行的话也不过几分钟的时间:``办公室怎么睡?你那屋还有张值夜班的床,我这屋屁也没有,打地铺啊?''朱建红知道他有主意,所以不是很敢惹他,只能以柔克刚:``你终于知道你屋里该有张床了?总让我躺桌子,你倒是不心疼我硌得慌。''

陈大光一摆手:``行啦,我逼着你躺了?我请你来的?我告诉你,我最烦娘们儿跟我唧唧歪歪耍嘴皮子,老子没空伺候,知道吗?你回去歇着吧,咱们明天见,好吧?''朱建红知道陈大光就是没好话,但是心里有数,不耽误他干好事。而陈大光知道大门是早锁了,又懒得再叫无心开门,于是直接跳墙出去,大摇大摆的回家了。

陈大光一走,革委会的办公区里就再没了旁人。朱建红坐在陈大光的皮面椅子上,拉开写字台的抽屉进行检查,想要找出其他狐狸精的蛛丝马迹。正是翻得来劲之时,她偶然一抬眼,忽然吓了一跳——通过半开的房门,她看到门外的水泥台阶上坐着个人!

人是背影,借着房中的灯光,可以看到他穿着一身脏兮兮的旧军装,手臂上还套着个红袖章。朱建红第一反应是无心来了,可是转念一想,无心不是无故乱窜的人,而且平时也没见他对自己有多亲近。关了抽屉出了声,她很严厉的问了一声:``是谁坐在外面?''

对方一动不动,而朱建红视力极佳,略一歪头看清了对方臂上的红袖章,竟是赫然印着``联指''二字。浑身寒毛骤然竖起,她没有找到趁手的武器,索性伸手拎起写字台旁的暖壶,一挺身站了起来:``到底是谁?说话!''

居高临下的放出目光,她发现门外木雕泥塑似的不速之客在水泥地上投下了一片阴影。唯物主义者的盔甲土崩瓦解了,她想起了她姥姥曾经宣扬过的封建迷信:鬼没影子,人有影子。是人就好,朱建红只杀人,不怕人。拎着暖壶向前又迈一步,她粗着喉咙喝道:``小兔崽子,少给老娘装神弄鬼!县革委会大院是你胡闹的地方?你赶紧给我站起来!''

终于,门外的人影缓缓的动了。一个脑袋慢慢的向后扭转,朱建红瞪着他的侧影,就见他脸上糊着一张黄纸,黄纸渗出斑斑血迹。人偶似的将脖子扭转了一百八十度,他在门口射出的一道光中,直直的面对了朱建红。

朱建红怔了两三秒钟,随即发出一声惊叫。一双腿打着颤的要向后转,可她随即想到窗户是紧关着的,想要打开也需要时间。要通过房门往外跑,可是谁敢迎着那么一个东西前进?一瞬间的工夫,朱建红把什么都看清了——外面的东西满身都是湿土,根本就是从地下爬出来的!

想起被红总押到城外成批枪决的联指分子,朱建红目眦欲裂,``嗷''一嗓子举起暖壶,像投掷炸药包一样,狠狠的砸向了门外的怪物。在跑与不跑之间犹豫了一刹那,她上前几步,``砰''的一声推上了房门。手忙脚乱的划了插销,她带着哭腔先喊陈大光,及至意识到陈大光已经走了,才绝望的又喊无心。

收发室与办公区之间隔着偌大一处空院子,此时又是午夜时分,她根本不知道自己的嘶叫能否惊动熟睡的无心。猛的瞧见写字台上的电话,她得了救星,三步两步的跑上去抄起话筒,然而话筒里一点动静都没有,电话线断了!

她拼命的拍打了拨号盘,又用力的插拔了电话线,但无论怎么折腾,电话都成了死物。房内的电灯忽然灭了,她在黑暗中又出了一身黏腻的冷汗——电话线能断,电线自然也可以断。手里死死的握着话筒,她僵硬在了写字台前。一双眼睛望向前方,她看到那个东西又在窗外出现了!

一张被黄纸遮去五官的面孔从下方缓缓升起贴上玻璃,革委会不必防贼,直接就是一层窗户,没有任何保护措施。那个东西抬起了手,一拳凿碎了一块玻璃。皮破肉烂的巴掌伸进房了,指甲缝里嵌着血和泥。

朱建红深吸了一口气,扭头就往门口跑。拔开插销推了门,她在身后又一阵玻璃破碎和窗框断裂的刺耳声中,疯狂的冲了出去:``大光!无心!来人哪!闹鬼啦!''
她没跑出几步,窗外的东西就通过窗户进了房,直通通的追上了她。

她虽然喊得热闹,但是内心并不把陈大光或者无心当成救命星来指望。一拐弯换了方向,她开始向自己的办公室疾奔——她的办公室里有手枪!然而未等她到达终点,一双冰凉黏腻的手已经合上了她的脖子。腐臭的恶气萦绕了她,她在极度的惊惧中,又从喉咙里挤出了一声锐叫。

大门口有了动静,是手电筒的光芒伴随着无心的疑问:``怎么了?有事吗?''朱建红强撑着不肯倒,在夜色中张牙舞爪,要对无心做出回应。眼角余光瞥到无心开始跑向自己了,她瞪圆了眼睛忍受窒息的痛苦,脖子上的筋肉全绷紧了,她使出余力对抗那个东西铁钳一般的双手。

无心晃着手电筒跑向办公区,起初还以为是朱建红在和人打架,跑出一半的路程了,他才意识到朱建红的对手不是个人。一阵风似的冲到近前,他飞快的看清了形势,然后没有去拉扯双方,而是猛然拍上不速之客的面孔,一把抓住了对方脸上的黄纸。与此同时,朱建红只觉脖子一松。连忙掰开那两只手,她喘息着回了头,对着眼前面孔当即又嚎了一声!

黄纸仿佛是粘在了这人的脸皮上,无心刚才的一抓,只抓下了中央的一大片纸。没了黄纸的遮挡,这人腐烂的眼眶和雪白的鼻梁骨一起曝露在了月光下。牙关格格的响了几声,他踉跄着似乎还要动,然而无心手如闪电,接二连三的掠过他的面庞,将黄纸撕了个干干净净。当最后一片黄纸脱落之时,他委顿在地,彻底不动了。

朱建红到底是经过大阵仗的,一颗心方才都要吓炸了,现在却又很快恢复了镇定。无心摆弄着手里的黄纸,黄纸又厚又韧,背面笔走龙蛇,还有图案。蹲在地上拼好碎纸,他发现黄纸上画着的是一道符。

朱建红喘匀了气,低头也看:``这是什么东西?''无心抬头答道:``不知道。不像画也不像字。朱副主任,发生了什么事?地上这位怎么——怎么——''他打了结巴,是个不知如何是好的模样。朱建红没开口,开了口也一样要打结巴。神情凝重的出了半天的神,她感觉自己随时可能失控发疯。

``不能等天亮了。''她思索着答道:``可能是有阶级敌人搞破坏,我们必须马上通知陈主任,让他来决定下一步的反击策略。''无心站起来了:``行,我知道陈主任的住址,我这就去找。''朱建红一把拽住了他:``不行!你不能把我一个人留下!''

无心把苏桃托付给了白琉璃,然后带着苏建红去找陈大光。陈大光还没有睡,正在家里和县评剧团的女演员谈心。朱建红无暇和他算账,把他叫出来后,她说了实话:``大光,革委会闹鬼了!''陈大光知道朱建红不是傻老娘们儿,所以十分诧异:``你扯什么蛋呢?''

朱建红带着哭腔哀求道:``大光,我没心思对你胡说八道。你看我这脖子,我告诉你要不是无心救了我一命,明天你就见不着我了。我不是吃醋捉奸来了,你快跟我走一趟吧!''

陈大光把女演员锁在屋里,然后披着上衣出了门,一路且行且问,听了个一头雾水。及至到了革委会大院里,他看着瘫在地上的尸首,也傻了眼。拼好的黄纸摆在地上,微微的被风吹乱了,但还没大走样。陈大光先看人再看纸,末了说道:``这小子的确是联指的人,可是\ldots{}\ldots{}''他转向了朱建红:``好几天前就被我们给毙了啊!''

无心插了嘴:``主任,副主任,那张黄纸看着够邪的,要是没用的话,是不是烧了它更合适?朱副主任刚才也看见了,黄纸一碎,这人——这鬼就不动了。''不等陈大光回答,朱建红心有余悸的点头:``对,对,快烧了吧。''

无心见陈大光不反对,就划根火柴点燃了黄纸。一把火烧过去,无心仰起脸,看到几点光芒零落四散。陈朱二人并未瞧出异状。

陈大光背着手,沉着脸对无心说道:``我告诉你,这就是敌人在故弄玄虚,想要扰乱我们的军心。所以今晚的事情,你一定要保密,高度的保密。你敢出去嚼舌头,我就撕了你喂狗!''无心连连点头:``我知道,你放心。''

\chapter{局势逼人}

陈大光从食堂后方的煤堆里捡了一只破筐,然后支使无心去把地上的尸首抱进筐里。无心往后一躲:``陈主任,我不敢。''陈大光现在没时间大发淫威,无心既然不听话,他就挽起袖子亲自动手,连拖带拽的把尸首弄进了筐里。尸首是软的,露出的皮肤已经偏于湿黏。朱建红渐渐缓过了神,理智一占上风,她在恐惧之余开始作呕。

陈大光双手叉腰对着破筐,显现出了革命领袖的超人智勇。革委会刚刚成立不久,城内的联指分子也还没有尽数落网,他像一尊威武凶神似的瞪着尸首,怀疑尸首的背后隐藏了大阴谋。革委会如今是红总掌权,是红总权力的象征。他作为红总的领袖,必须维护革委会的尊严。革委会大院就是文县的圣地,谁家的圣地夜里会闹活鬼?

``鬼鬼神神的事情,我是不信的。''他低声开了口:``但是\ldots{}\ldots{}''朱建红直挺挺的站着,幼时从她姥姥嘴里听得的奇谈怪论正在她脑子里兴风作浪。三个人中数她年纪最大,她以老大姐的身份,犹犹豫豫的开了口:``我姥姥说她年轻的时候,家里有人撞了邪祟,她亲眼\ldots{}\ldots{}''

陈大光不耐烦的一挥手:``别扯你的陈谷子烂芝麻了,没人听你姥姥的鬼故事。咱们就说眼下——他妈的一个都入了土的人,死得透透的了,你看他前胸口上还有弹孔呢,怎么就神不知鬼不觉的跑到了革委会?''

朱建红受她姥姥的影响很深,此刻不由自主的又开了腔:``我姥姥说有些孤魂野鬼本事大,能够附着死人作怪。''陈大光一咬牙:``姐姐,别提你姥姥了!妈的敌人就是敌人,枪毙都拦不住他继续反革命。无心你过来,帮我把筐抬到房后去!老子不怕鬼,老子现在就把他烧成灰!''

无心抓着筐边,和陈大光一起把尸首抬去了房后。陈大光拎了汽油浇进筐中,然后扔出一根火柴。火苗``腾''的就窜上了天,陈大光在身后墙壁上投下一个极其巨大的黑影,影子随着火光动,他不动,是真正的坚如磐石。

尸首烧到一半,无心得了敕令,独自回了收发室。拧把毛巾擦了擦手脸,他关了房门,对床上的苏桃说道:``睡吧,没事。''

苏桃一直蹲在床上,不敢下地也不敢开门:``外面是有人打架了吗?''无心答道:``是,朱建红和一个女人打起来了。两人下手都狠,叫得惊天动地。''苏桃这才放心的躺下了:``哦,怪不得我看你和朱建红出大门了呢,原来是找陈主任来劝架。''

无心怕自己身上烟熏火燎的有气味,又懒得再打地铺,便在床尾蜷缩着侧卧成了一团:``不管他们的事,我可真得睡了。''苏桃看他闭了眼睛,自己也跟着靠边躺了,先是抱着膝盖睡得老实。及至睡深沉了,她不知不觉的伸长了腿,两只赤脚全蹬进了无心的怀里。无心迷迷糊糊的抱了她的小腿,很惬意的一直睡到了大天亮。

天明之后,一切如常。革委会的工作人员络绎出现,几名工人站在房后,为陈大光的办公室安装新窗户。无心抱着新到的报纸,挨间办公室发放一遍。末了兑了一盆温水回到收发室,他把白琉璃泡进水中,决定亲自帮他蜕皮。苏桃则是拿了粉笔蹲到门外,替他往小黑板上抄写今日的收信人姓名。

无心一边往白琉璃的身上撩水,一边压低声音说道:``昨夜我真是开了眼界,居然有人能用纸符封住魂魄,再通过纸符把魂魄过到死人身上。你见过吗?''白琉璃死气活样的盘在水里,不理睬他。

无心自顾自的继续说道:``甭管是死了多久的尸首,只要刨出来贴上纸符,自动就能借尸还魂,够厉害吧?纸符一揭,魂魄随着纸符走,尸首还是尸首,什么破绽都没有。''

表层粗糙的蛇皮遇了温热的水,慢慢变得柔软膨胀。眼看老皮要和身体分离开了,无心捏住蛇头下方的一点硬皮,开始小心翼翼的揭。苏桃挂好小黑板进了房,蹲在一边旁观:``无心,他疼不疼呀?''无心抬头对她一笑:``不疼,蛇都是要蜕皮的,蜕一次皮,就长大一点。可惜他是条笨蛇,自己不会蜕,非得让人帮忙。''

无心轻轻的把皮退到白琉璃的尾巴尖,呈现给苏桃的正是一条半透明的细长蛇蜕。白琉璃晶莹剔透的盘在水中,一个脑袋搭上盆沿,很舒服的细了眼睛。苏桃高兴极了,小声笑道:``哎呀,你看他白得像玉。''

无心也了却了一桩心事,故意把蛇蜕提到白琉璃面前摇晃:``娘子,看看你的长筒丝袜。''白琉璃气得把脑袋转向苏桃一边,依然不肯理他。无心来了劲,挤到苏桃身边,俯身歪头要和他对视:``你也辛苦了,我去给你弄点好吃的补一补,你乖乖等着我吧!''

无心说到做到,当天下午就去煤堆附近掏了一窝老鼠。从中挑了几只粉粉嫩嫩没长毛的老鼠崽子,他回到收发室,一只一只的喂给白琉璃吃。白琉璃吃多了,胀得如同一根大擀面杖,快要不能弯曲。千辛万苦的爬到了苏桃的枕头下,他开始雷打不动的休息。

苏桃无所事事的坐在一旁看书,书是无心从废纸堆里捡出的一本鲁迅文集,如今读书也是带有危险性的行为,无心在废纸堆前选来选去,末了感觉还是读鲁迅最保险。

平安无事的到了晚上,眼看天黑了,苏桃也躺上床了,无心便打算关门睡觉。不料陈大光飘然而至,鬼鬼祟祟的把无心叫出了门。无心脖子上搭着一条毛巾,一边关门一边问道:``陈主任,有事吗?''

陈大光自然是有事,不过在开口之前,他先望着无心愣了一下——之前从来没在夜里正经观察过对方,他此刻正眼一瞧,差点被无心吓了一跳。收发室里关了灯,只剩外面门上还亮着一盏照明的小灯泡。灯光斜斜的照在无心脸上,照出一张明暗错落的面孔,微凹的黑眼窝里,两只乌溜溜的大眼珠子仿佛在自行放光。陈大光万没料到他竟有如此之大的眼睛,而且灵动得过分,让他联想到了精怪鬼魅。

``你\ldots{}\ldots{}''陈大光拉着长声迟疑了:``睡了吗?''无心拽下脖子上的毛巾:``陈主任,我显然是没睡呀!''陈大光知道自己是问了废话,当即恢复理智改了口:``我知道你没睡。进去穿衣服,出来跟我走。''无心托着湿毛巾擦着后脖颈,上下审视单枪匹马的陈大光:``去哪里?''陈大光避而不答,只是一扬下巴:``快点,别让我等你!''

无心让苏桃从里插了房门插销,自行睡觉;然后跟着陈大光走出了革委会大门。自从经过了前些时日的武斗,文县百姓自动执行了宵禁,夜里根本没人上街。陈大光步伐矫健,一边走一边说出了自己的用意——他打算亲自去趟城边的行刑场,倒要看看是谁刨了联指的乱坟。

无心一听,当即要打退堂鼓:``陈主任,这么重要的任务,派给我不大合适吧?''陈大光对他一瞪眼睛:``谁让你已经知道了?难道放着知情人不用,反倒把消息扩散给旁人?我告诉你,这件事不简单,绝对有阴毛!还有,兵贵精不贵多,凭着你我二人的身手,够用了!''无心紧赶慢赶的跟着他,心想陈大光``谋''``毛''不分,大学真是念到狗肚子里了。

陈大光走了两条街,却是到了他自己的住处。他如今一步登天,占据了一套独门独户的好房屋。从院子里推出一辆漆黑锃亮的自行车,他将一把工兵铲交到无心手里,然后飞身上车,回头说道:``走!''等到无心在后座坐稳当了,陈大光踏下脚蹬,破空之箭一样冲进黑暗。

他是太有劲了,自行车被他骑出了汽车的速度。无心坐在后头,就听耳边风声呼呼直响。不过片刻的工夫,他们便到了一片漆黑的城边。

文县的城内城外很难界定,因为建设得太快,今天是城外,明天楼房一起,就是城内了。不过此刻的城边真是名符其实,四面八方一片空旷,半分人气都没有。又由于红总近来常在此处杀人,所以连野孩子们都不敢来玩了。陈大光艺高人胆大,把自行车往一个坟头上一推,他拿着手电筒开始一边照一边走。

地上坑坑洼洼的不平坦,高高低低的荒草在夜风中摇曳。无心忽然踉跄了一下,低头看时,地面伸出了一只肮脏的小手,刚才绊住了他的脚。陈大光漠然的用手电筒一扫,嘴里骂道:``谁干的混蛋活?埋人都埋不明白。''然后他停了脚步,晃着手电筒大范围扫视。无心轻声说道:``范围太大,又没个坟头,不好找啊。''

陈大光沉吟不语,忽然向前举起了手电筒,他大声喝道:``谁?站住!''光圈一颤,无心也看清楚了——草丛中有个人,一猫腰不见了踪影!陈大光拔出腰间手枪,对着前方连开三枪,随即迈开大步就往前追。无心正要追随,可是手握着工兵铲顿了一顿,他原地一个转身,一铲子拍中了身后的突袭者。

突袭者一身血衣,脸上蒙着黄纸,动作僵硬而又凶狠,直通通的扑向无心。无心无暇去撕对方的纸符,情急之下退无可退,索性举起铲子猛劈向下。工兵铲是苏联货,钢口极好,宛如大刀。一声闷响过后,行尸的头颅被斜砍成了两半。

纸符顺着伤口裂开了,行尸居然不倒,而且转身有了要逃的意思。而陈大光一无所获的折返回来,夺过无心手里的兵工铲高高举起,只听一声大喝,他竟然用工兵铲把行尸深深钉在了土地上。呼哧呼哧的喘着粗气,他居高临下的瞪了眼睛:``我倒要看看你是个什么东西!''

无心帮他握住了锹把:``好,看吧!''陈大光弯了腰,发现少了半个脑袋的尸首居然还在微微的挣扎颤抖。伸手剥下一片黄纸,他直起腰望向了无心,难以置信的开口问道:``难道\ldots{}\ldots{}真是闹鬼?''无心虽然是不想卷入任何一方的势力,不过在陈大光的注视下,他必须作出回答:``可能\ldots{}\ldots{}是吧!''

陈大光的面孔有些扭曲。忽然双手拔起兵工铲,他把脚下的行尸铲了个稀烂。末了把兵工铲向旁一丢,他咬牙切齿的说道:``有鬼老子也不怕!不是老子下命令,他们也做不成鬼!老子让他们做人,他们是人;老子让他们做鬼,他们就得乖乖当鬼!''然后他伸手一指无心的鼻子尖:``保密!听见没有?''无心一点头,转而问道:``你追到什么了吗?''

陈大光双手叉腰,吐出了一口气:``没追到,跑得太快。看背影好像是个女人。''无心思索着说道:``陈主任,我不懂什么。我随便说一说,你随便听一听。地上的东西,我感觉很危险,因为我劈了它一铲子之后,它知道跑。''

陈大光皱着眉头:``你是说\ldots{}\ldots{}它有脑子?''无心字斟句酌的答道:``可能是有,当然,远远比不上人。但它既不怕死也不怕疼,又有一点智慧,如果进城捣起乱,恐怕是不大好办。''

陈大光走去扶起了自行车:``先回城,回去再说。妈的我是哑巴吃黄连、有苦说不出。万一上头知道我和鬼干上了,还不得怀疑我有精神病?就怪朱建红一张破嘴天天说她姥姥,现在可好,她姥姥说的全成真了!无心,你行,你胆子不小,真敢和它对着干。算我没看走眼,你是个人才。''

陈大光带着无心回了城中。把无心放回革委会收发室,他自己回了家。关上房门倒了杯酒,他一边咂摸着滋味,一边活动心思。坟地里的怪东西是他的敌人,联指也是他的敌人,而且坟地里的确是藏着活人,让他不能不顺势想起了大牢里的小丁猫;小丁猫是个祸患,自己应该想办法尽早除了他。

陈大光略略想出了眉目,放下酒杯上床睡觉。他的人生至爱一是螳螂拳,二是女人。他可没有耐性借酒消愁。

\chapter{百思不得其解}

革委会的大门外有一棵老树,树上住着一窝不知品种的大灰雀。因为白琉璃始终是不肯理睬无心,所以无心爬到树上,掏走了大灰雀小部分的鸟蛋。把鸟蛋拿进收发室里一五一十的数清楚了,他将其分成两半,一半给了苏桃,另一半喂了白琉璃。白琉璃千辛万苦的蜕了一次皮,十分需要进补。盘在床上仰起脑袋,他把大嘴张得像瓢似的,等着无心磕破鸟蛋,把蛋清蛋黄倒进他的嘴巴里。

吃了三枚鸟蛋之后,他心满意足的闭了嘴。而无心擦了擦手,抬头去看苏桃,心想给桃桃弄点什么好东西补一补呢?平日的饮食以窝头为主,连白面馒头都少有,要说饱是能吃饱的,但也只是吃饱而已,想要根红豆冰棒,都得算着日子买,买了一根不舍得咬,全是一口一口舔干净的。

无心尽管知道大家都穷,苏桃不能算是受了委屈,但心里还是不大舒服。和苏桃认识几个月了,她一直没见长。无心怀疑她是亏欠了营养,因为毕竟年纪还小,不该到此为止就定型了。

到了夜里要睡未睡的时候,无心问苏桃:``桃桃,你说是原来的联指好,还是现在的红总好?''苏桃侧身躺在床上,辫子散开了,满肩满背都是头发:``我看\ldots{}\ldots{}是红总好。''无心和她头脚颠倒着躺,鼻子尖正对着白琉璃的尾巴尖:``红总好在哪里?''苏桃怕自己踢了无心的脑袋,所以两条腿伸得直直的:``红总的人,好像更正经似的。''

苏桃此言非虚,因为革委会作为一县的新政府,里面除了造反派是主力之外,还有先前留下的老干部以及军方人员。整体氛围是机关式的,和联指指挥部的气氛自然大不相同。无心点了点头:``要是联指像红总一样,哪天又打了回来\ldots{}\ldots{}''

苏桃亲眼见过联指杀人,但是没亲眼见过红总杀人,所以思想带了一点偏向性:``联指太坏了,里面没有好人,还是别回来了。''
无心想起了小丁猫,想起了杜敢闯,想起了陈部长,想过一大串人物之后,他承认苏桃说的不谬。其实陈大光也不是什么好东西,不过凶恶得一目了然,让他心里比较踏实。

此刻陈大光正在办公室里值夜班,革委会夜里不留人,陈大光怕再闹起鬼,惹出坏影响。近几天他显然是十分郁闷,连女人和螳螂拳都不能使他开心颜,因为他无产阶级的铁拳,找不到实施专政的具体对象。

陈大光夜里醒醒睡睡,时刻提防着有鬼来袭。无心也是醒醒睡睡,心里盘算着自己的立场。从城边到革委会,并不是一段短途,行尸哪儿都不去,专门走长路夜袭革委会,必定是背后有人操纵。目的是什么?目的可以有很多,其中之一无心能够确定,就是扰乱人心,让革委会不能正常运作。

革委会基本就是红总的革委会,而联指是百足之虫死而不僵,能够和红总抗衡的,在文县地界,也就只有它的残余力量了。无心决定帮陈大光一把,毕竟陈大光在文县已经杀过劲了,如果联指卷土重来,必定又是一场腥风血雨。想起阴恻恻的小丁猫,他在夜色中一皱眉头。

无心一觉醒来,照例是洗漱过后出门扫院子。陈大光正在院子里练螳螂拳,面容堪称憔悴。无心扶着大笤帚问答:``陈主任,夜里没事吧?''陈大光黑着脸:``还行。''无心又道:``附近有没有还俗的老道?有的老道会画符,兴许能有点儿用。''

陈大光保持着螳螂捕蝉的动作,扭头看他:``胡说八道!要是让人知道我找老道去了,我还有脸再混吗?''说完之后他意犹未尽,又捏着指头做了个螳螂爪,在无心肩头勾了一下。无心扛不住他的力量,当即一躲。上下又看了陈大光一眼,他慢悠悠的开了口:``朱副主任她姥姥呢?她姥姥好像也是位见多识广的老人家,也许能给你出出主意。''

陈大光嗤之以鼻:``她姥姥十年前就入土了。''无心抄起大笤帚,一边走一边又道:``其实\ldots{}\ldots{}''陈大光听他还有话说,登时提起了精神,可是无心到此为止,不肯说了。陈大光一把扯住了他:``你等会儿!其实什么?''

无心沉吟着答道:``其实\ldots{}\ldots{}算了,我不说了。宣扬封建迷信也是有罪过的事情,我刚吃了几天安稳饭,犯不上自找麻烦。陈主任,松手吧,你看人都来上班了,你我拉拉扯扯的也不像话。''陈大光松了手:``别跟我装模作样,咱们有话晚上说!''

陈大光把无心的话放在了心上。熬到傍晚众人下班,他把正锁大门的无心又揪了住:``走,到我办公室去!''无心乖乖的跟他去了。房门一关,办公室里没了别人。陈大光坐在写字台上,大模大样的问无心:``我说,你是不是有什么法子?无心,你我萍水相逢,我对你不能算坏,你见死不救可不行!''

无心拽过一把椅子坐下了,平平淡淡的低声答道:``要说大获全胜,我不敢打包票,我只能说我有一点方法可以挡一挡或者治一治。''陈大光知道他不是胡言乱语的人,所以立刻来了精神:``你真能?''

无心平时对他挺恭顺,如今一反常态,神情反倒冷了:``不知道能不能,试试看吧。但是我有条件。''陈大光一扬下巴:``说!''无心抬眼看他:``按月给我工资。我的户口本不在身边,你还得负责我每个月的粮票。''

陈大光笑了:``我还以为是多高的条件,原来就是钱和粮票。无心,我告诉你,革委会里我是说一不二,我想提拔谁就提拔谁。只要你真是个好样的,我肯定不能总让你看大门。你说你要试试看,好,马上给我试。不过你打算怎么试?''

无心一本正经的说道:``我打算夜里自己去趟坟地,超度超度亡魂。''陈大光看妖怪似的看着他,听他说话都新鲜:``你从哪儿学来的本事?还超度亡魂?''无心站起了身:``解放前我舅舅是和尚我叔叔是道士,我奶奶跳大神我爸爸当半仙。''

陈大光立刻挥了挥手:``真是书香门第,赶紧去吧!我不走,就在办公室里等着——你是今天夜里上坟去吧?我借你个新手电筒?''

无心没要他的手电筒,而是借用了他的自行车。天一黑,无心就出了门。一路顺顺利利的骑到城边,他在距离坟地一里地外就下了自行车。把自行车倚着路边大树放好,他步行前进,悄无声息的抵达了坟地。

坟地下面,至少埋了上百条青春年少的人命。到了明年此时,土地必将肥得草都不长。萤火虫和鬼火混成一队,在起伏的地面上闪闪烁烁。无心成了一只走兽,隐身似的钻进了草丛里。一条斑斓大蛇游过他的脚面,他低头看了一眼,发现和此蛇相比,白琉璃真是美如天仙了。

他不动,蛇也不当他是个活物,自顾自的爬向远方。而无心双掌合十低头跪了,开始无声的翕动嘴唇念经。坟地上的怨气太重了,底下的尸骸没有一具是好死的。让他把怨气尽数化解,他做不到,只能是尽力而为。鬼魂时常像个委屈愤怒的孩子,不讲理也不听理,而好的法师要会哄会劝,让它们心甘情愿的不计较。不计较了,不爱不恨了,就入轮回了。

无数成了形的鬼魂仿佛听到了无心的佛经,觅声而来围住了他,做狰狞相,做恶鬼相。然而做鬼也不是容易的事情,魂魄不是好聚的,有些小鬼刚把鬼脸做到一半,就不由自主的魂飞魄散,化成了几线黯淡的光芒。无心不抬头,不回应。直到远方起了窸窸窣窣的声响。不动声色的伏下了身,他睁开眼睛望向前方黑暗。

暗中活动着一个黑影,看动作不是死鬼,是活人,背对着无心不知在干什么。无心四脚着地的出了荒草丛,随即起身猛的冲去,纵身一跃扑到了对方。双方抱着打了几个滚,无心借着月光向下一望,只见对方仰着一张青黄不接的长脸,正是马秀红!

无心对马秀红一直没什么印象,因为她不多言不多语,虽有如无。可是此刻马秀红长脸扭曲,对着身上的无心怒骂:``呸!叛徒!''无心看她如同疯魔一般,满嘴牙缝碧绿碧绿的,不知道是吃了多少天老野菜。双臂用力箍住了她,他开口问道:``是小丁猫让你来的?''

马秀红双目赤红:``别用你的臭嘴叫他的名字!你尽管押着我去见陈大光吧!革命不怕死,怕死不革命。你跟着红总走,迟早是自取灭亡!''无心心中一动:``纸符是小丁猫给你的?''

马秀红恨透了小丁猫身边的一切叛徒,若不是口干舌燥,非迎面啐他个满脸花不可:``怎么?你们怕了?还是想彻底的治死他?我告诉你,你们的苦头在后头呢!将来有对你们清算的一天!''

无心知道老实人发起疯,比疯子更厉害。他决定先把马秀红带走,可是未等他行动,他的胸膛忽然狠狠的一痛。低头看时,他惊讶的发现马秀红不知何时腾出一只手,竟然将一把铁锥子扎进了自己的心口。

他愣了,伸手想要去拔。马秀红存了必死之心,咬紧牙关对他拼命一推,生生的把他从身上掀了下去。连滚带爬的起身跑出几步,她回头狞笑了一下,暗想自己这一锥子扎得真是地方,不但杀人灭口,顺便还除了组织中的一个叛徒。

无心眼看着马秀红逃了,没有追,因为伤处实在是疼得厉害。自己低头握住锥子向外一拔,锥子尖带出了几点血。坐在地上忍了片刻,他垂头丧气的爬起来,同时发现马秀红方才背对自己忙碌不已,原来是在挖尸首。

如今城里都是火化,想要找到囫囵尸首,除了去乡下刨坟掘墓,就是来城边的乱坟岗子。死了马秀红,还有后来人,所以把事情弄清楚就是了,不必非得抓她。无心骑上自行车往城里走,心里想着小丁猫。小丁猫的手段,让他想起了一位故人——岳绮罗。

虽然他和岳绮罗之间已经隔了四五十年的距离,不过偶一回想,还是感觉她十分万恶。小丁猫的手段真像岳绮罗,但是性格又真不像岳绮罗。岳绮罗残忍孤介,小丁猫和她着实不是一个路子。兴许是岳绮罗逃出鬼洞投了胎又转了性?

无心想了一路,末了自己对自己摇头,感觉就算转性,也不该转得这么彻底。岳绮罗素来对人间没兴趣,而小丁猫对人间可是太有兴趣了。人都进了监狱,还有闲心遥控部下,潜入文县兴风作浪。

无心回到革委会之后,先去见了陈大光,如实的作了汇报。陈大光看他无精打采的,还挺关心:``你怎么了?''无心和陈大光一样,也是哑巴吃黄连、有苦说不出。支吾着回了收发室,他悄悄的上床躺好。自己把手伸进汗衫里摸了摸,摸到了心口处一个清清楚楚的锥子眼。

从棉被的缝隙里揪出一点棉花揉成团,他把锥子眼塞住,然后在渐渐淡化的疼痛中入睡了。

\chapter{马秀红之死}

陈大光说是天气热了,怕闹瘟疫,故而派人去了一趟城边乱坟岗子,把尸首一层层的胡乱刨出,放把大火烧成了灰。

他这行为合情合理,任谁也挑不出毛病。等到大火灭了,他心里轻松了许多,同时撒开天罗地网,开始全力搜捕马秀红。马秀红活得像只老山羊似的,每天风餐露宿吃野菜,应该没有力量远遁。陈大光打算把她当成人证交给上头,让上头加快速度,赶紧把小丁猫处理掉。

无心只管鬼事,不管人事。陈大光给了他几张收据,让他自己去财务组报销。无心高高兴兴的领了钱,上街给苏桃买了饼干回来吃。苏桃坐在床上,正在穿针引线的给他补汗衫。外衣和汗衫上面无缘无故的多了个洞眼,四周还缀了几个深褐色的点子。她问无心这是怎么弄的,无心一脸茫然,表示自己也不知道。

她不会做针线活,费了牛劲才用针线把洞眼平平整整的缝合。眼看革委会到了下班时间,她拎着水盆站在阴影里,等到人都走得差不多了,才去水龙头接了自来水,搬着小板凳坐在收发室外洗衣服。天气热,衣服换得勤,非得天天洗。盆里架起搓衣板,她很来劲的搓着领圈袖口,白色的泡沫从指间噗嗤噗嗤的往外冒。无心不脏,领子袖子都没有油泥,搓上几把就足够干净了。

陈大光晚了一步,大院都空旷了,他才带着个评剧团里的女演员走出办公室。他不要名声,在男女问题上是公开的胡搞,朱建红根本制不住他。出门之前他留意的看了苏桃一眼,看过就算,没把她往心里放。在他眼中,非得像朱建红之流才算女人,苏桃脸上还带着一层细细的茸毛,身体缩在灰扑扑的衣裤里,怎么看都是个畏手畏脚的小丫头。他甚至怀疑无心和苏桃之间真是清清白白,否则的话,苏桃不该总是一副生瓜蛋子似的青涩模样。

苏桃知道陈大光看自己了,但是低着头没出声。身后哗啷啷有了金属声音,是无心锁了大门。

今天是端午节,革委会里没人值夜班,都回家过节去了。无心把大门钥匙放回收发室,然后拿出了一瓶桃子罐头。走到苏桃身边蹲下了,他用一把白铜钥匙去撬罐头瓶盖:``大过节的,我们也没粽子吃,爸爸给你开个罐头吧!''

此言一出,苏桃当即笑了:``不要脸,你才多大啊!''

无心也跟着笑:``反正比你大。''

苏桃正要反驳,无心已经把打开了的桃子罐头递向了她:``擦擦手,别洗了。''

苏桃手上加快了速度:``马上就得,你先吃。''

三下五除二的洗净衣服晾好了,苏桃和无心坐在院内的水泥花坛上吃罐头。留在房内的白琉璃也没闲着,正在试图吞下一只生鸭蛋。如此到了天黑,外面的两个人回了房,迎面就见鸭蛋被白琉璃用身体勒了个稀碎,蛋黄蛋清涂了满床,白琉璃自己也粘了一嘴的鸭蛋皮。

``哎呀''一声过后,连苏桃都不维护白琉璃了。无心用一只大勺子在白琉璃头上连敲十下,然后把他拎到院里的水龙头下冲洗;又让苏桃撤下床单送过来,反正他已经湿了手,索性连床单一并洗了算了。

苏桃趁着他洗洗涮涮,抓紧时间回了房,想要偷偷脱下紧贴身的小背心。仔仔细细的关了门窗,她一边解纽扣一边转向床边。身体刚刚转到一半,她忽然回头望向窗口,因为方才眼角余光中仿佛有黑影掠过。

窗外一片肃静,院门也锁得牢固,只有一只乌鸦站在铁栅栏大门上,扯着粗喉咙叫了几声。苏桃松了口气,没想到自己被乌鸦吓了一跳。快手快脚的脱了外面衣裤和小背心,她换上一件旧到柔软的汗衫,展开被子先上床了。耳边隐隐响着水声,一定是无心还在大洗床单。她靠墙闭了眼睛,不管无心上不上床睡,反正她给他留出了位置。

她贪睡,躺下不久就犯了迷糊。正是似睡非睡之际,她朦朦胧胧的感觉房门开了。脚步声音越来越近,她向床里又挪了挪。突然抽了抽鼻子,她嗅到了空气中的土腥味道。莫名其妙的睁开眼睛,她以为无心又把什么东西弄脏了,可就在睁眼的一刹那间,她忽见一道寒光从天而降。下意识的抓起被子向上一挡,只听``噗''的一声,锐利的刀尖刺透棉被,一直逼向了她的眉心。

惊惶失措的惊叫一声,苏桃发现刀尖正在作势向上拔出。下意识的一个鲤鱼打挺,她随着刀尖的方向走,把棉被兜头蒙在了来人的头上。赤脚跳下床去,她披头散发的要往外跑。然而后方的人身体一晃甩掉棉被,一手持刀扎向了她的后背。门槛不平,苏桃在出门时脚下踏了个空,不由自主的身体一歪靠上门框。刀锋贴着她的半截衣袖刺出去,半路一转方向又去抹她的脖子。苏桃再也无处可逃了,情急之下伸手去抓对方的腕子。目光同时一斜,她看清了来人的面目:``马——''

马秀红一言不发,眼看她双手一起攥了自己的腕子,她挥起另一只手,将一张黄色纸符拍向了苏桃的脸。苏桃扭头一躲,只听``啪''的一声,纸符斜斜的贴上了她半边面颊。半边面颊瞬间起了一层鸡皮疙瘩,阴冷的寒气直入骨髓,心中随之气血翻涌。正是危急之时,门外又是一阵冷风,脑海深处仿佛响起了一声怒吼,震得她身体一颤,紧附皮肤的纸符居然一松,自行向下滑落了些许。看到纸符将要脱落,马秀红伸手想要去抓,可是眼前白光一闪,无心动作更快,已经一把扯下了纸符。飞起一脚把马秀红直踹到了房内,无心从门口拎起一条锁大门的铁链子,上前一链子抽飞了马秀红手中的尖刀。

马秀红躺在地上,绝望而又愤慨的瞪着他。无心知道她是丧心病狂的了,所以也不多问。直接用铁链子反绑了她的双手。

制服了马秀红之后,无心再看苏桃,就见苏桃吓得脸色煞白,汗衫袖子也被刀刃割出一条口子,里面伤了皮肉,幸而不深,只渗出了一点鲜血。

无心用一条手帕给她包了伤口,又不住的摩挲了她的头发。让她重新穿了衣裤,无心对着悬在半空的白琉璃使了个眼色,然后把马秀红锁在房内,领着苏桃去找了陈大光——不敢再把苏桃一个人留在房里了,方才苏桃是运气好,如果运气不好,被马秀红一刀捅死也不稀奇。

陈大光正在家里和女演员过节,忽见无心来了,不禁大皱眉头。可是听了无心的一番报告之后,他脸色一沉,披了衣服就往外走。

大步流星的回了革委会,他看到了伏在地上喘息不止的马秀红。双手叉腰犯了疑惑,他问无心:``你说她是怎么进来的?要是爬后墙的话,从后院到收发室,她得经过大院,你不能没看见;要是走大门的话,你这大门又是提前锁了的。莫非我们这个革委会里还有暗道?''

无心双手抓住院门栏杆撼了撼,又仔细审视了马秀红,末了得出了答案:``没有暗道,她就是钻大门进来的!''

陈大光恍然大悟——院门栏杆之间存有距离,一般人当然是通不过,但马秀红天赋异禀,十分细长,却是能钻。从无心手中接过纸符又看了看,他点头自语:``好,老子不怕你来,就怕你不来!''

然后他亲自动手,抓鸡似的把马秀红拎走了。马秀红死狗似的随他拖拽,一声不吭,一丝不动。

无心抱着苏桃坐了一夜。苏桃真是吓着了,无论如何睡不着觉。无心轻轻拍着她的后背,白琉璃也盘在她的腿上。苏桃仰头问他:``马秀红为什么不去找红总的人?我们又没有害过她。''

无心轻声反问:``你看她是讲道理的人吗?''

然后他把苏桃向上抱了抱,尽量不让她往自己安静的胸膛上靠。苏桃枕着他的肩膀,又问:``她为什么要往我脸上贴纸?''

无心歪着脑袋,用面颊去贴她微热的额头:``小丁猫下了大狱,总没消息,她可能是急疯了。''

苏桃小声说道:``小丁猫怪吓人的,还有人喜欢他。''

一夜过后,陈大光号称自己单枪匹马捕捉到了小丁猫的机要秘书,这个细长的混账秘书隐藏在城中,扇阴风点鬼火,或密谋于暗室,或行动于黑夜,上蹿下跳,企图变天,真是罪大恶极。

针对马秀红的专案组立刻成立了。陈大光摩拳擦掌,必要在她身上做些文章,置小丁猫于死地。不料未等审讯开始,保定忽然发来急电,说是小丁猫越狱了。

由于上头迟迟的不肯给联指定性,所以监狱里的小丁猫始终是不见天日也不得结果。据说他在狱中表现十分之好,既不造反也不绝食。等到狱卒对他都放松警惕了,他一天夜里平白无故的就没了。

陈大光气得直拍大腿,一腔怒火全发泄在了马秀红身上。然而马秀红不吃不喝不招供,死不承认小丁猫有罪。熬了三天的酷刑,第四天早上,她在牢房墙上写下``红色江山万岁''六个血字,然后趁人不备,一头撞死了。

当初和小丁猫一起加入联指的时候,小丁猫曾经微笑着告诉她,说自己要打出一片红色江山。她总记着,至死不忘,至死不渝。

短命的专案组随着马秀红之死而解散,幕后黑手也没能被揪出。听说联指的一号已经逃去了北京,二号则是潜入乡村,三号又刚刚越了狱,陈大光心里暗暗敲鼓,发现自己宝座不稳,战争根本就没有结束。

所以在按照惯例下乡视察之时,除了应有的随行人员之外,他额外又带上了无心。他能打人,无心能打鬼,两人合力,正是天下无敌。无心出门,自然也得领着苏桃和白琉璃。于是在一个阴雨靡靡的夏日清晨,众人欣欣然的乘车出发,直奔最近的公社去了。

\chapter{夜会}

平日在革委会大院里,无心虽然时常见识陈大光的官威,可由于陈大光毕竟还是个年轻人物,私底下忍不住一派随便,尤其对无心并不讲究礼数,故而他还意识不到陈大光的权势。及至出了县城下了乡,无心开了眼界,才发现原来陈大光真是有着土皇帝一般的高身份。陈大光乘坐一辆苏联嘎斯69吉普车,又轻又快的行驶在柏油路上,后方跟着几辆大卡车,满载着他的部下。出城之后没过多久,他们便抵达了最近的猪嘴公社。猪嘴公社本名猪嘴镇,紧挨着猪头山。因为猪头山早成了矿区,所以猪嘴公社受了恩惠,也很繁华。陈大光一下吉普车,就被公社干部和先进社员们包围了。一边缓步前行,一边享受着四面八方的热情恭维,陈大光飘飘然的,认为文化大革命真是好,如果没有文化大革命,他去年夏天毕了业,现在至多是在一中当体育老师。体育老师和县革委会主任相比,地下天上,没有可比性。

公社里杀猪宰羊,款待县里来人。陈大光自知学问不济,说不出漂亮话,所以谨言慎行,保持自己莫测高深的伟岸形象。旁人没他的顾虑,一个个兴致勃勃的东走西逛,欣赏镇上不甚地道的田园风光。无心带着苏桃满镇里转了一圈,随口说道:``变化真大,原来镇上就只有一条正经大街。''

苏桃好奇的看他:``你怎么知道?''

无心把两只手插进衣兜里:``听别人说的。''

白琉璃从书包缝隙中伸出圆脑袋,并没有看到什么好风光。飞快的一吐信子,他因为近来吃得太多,动弹不得,于是懒洋洋的缩回了书包。

无心带着苏桃踏上了归途,心里想起了月牙。月牙要是活到现在,也是个老太太了。岁月是能把一个人活活风干的,仿佛有一只干枯苍老的手抚摸了他的头和脸,让他在大太阳下恍惚了一下。

苏桃抽了抽鼻子,扯着他的袖子问道:``你闻到香味了吗?''

无心吸了一口气:``闻到了,真香。''

公社的伙食太好了,陈大光和朱建红等人,在公社干部的陪同下吃小灶。小灶精美,大灶也不赖,成盆的炖肉往桌上端。在动筷子之前,众人统一起立,手持红宝书齐声叫道:``敬祝伟大领袖的**万寿无疆!万寿无疆!万寿无疆!''

随着``万寿无疆''四个字的重复,众人手里的红宝书向斜上方挥舞三次。然后继续喊道:``敬祝林副统帅身体健康!身体健康!身体健康!''

敬祝结束之后,满食堂的人又高歌了一曲《东方红》。唱完最后一句,食堂桌椅声音响成一片,筷子也都箭簇一般直射猪肉。无心和苏桃连主食都不要了,专挑五花肉大嚼,吃了个满嘴流油。

饭后的节目,是革命群众大联欢。热闹了大半天之后,又是一顿丰盛晚饭。陈大光明天还要下到生产大队里视察,所以夜宿猪嘴公社。照理来讲,县里的干部们应该被分派到老百姓家居住,不过无心带着个不离手的半大丫头,住到哪家都不合适,于是陈大光善解人意,让他和自己一起在公社大院里居住。

无心和苏桃得了一间宽敞屋子安身,屋子里砌着半截火炕,两人总算能够宽宽展展的睡一夜。但是先前两人凑合着挤,总像是不得已的对付,还算自然;如今舒舒服服的并肩躺了,小两口似的,反倒要让人往深了多想。

两人洗漱过后,无心和苏桃头脚颠倒着躺了,各自盖着一床新被。新被不大,苏桃盖着正合适,无心则是顾了上就顾不得下,不是露肩就是露脚。苏桃一时睡不着,睁着眼睛往窗外看,视野边缘翘着无心的脚趾头。白琉璃在被子上爬来爬去,末了把脑袋往她颈窝里一拱,乖乖的不动了。

无心无声无息的躺在炕上,苏桃都睡了,他还清醒着,心里走马灯似的闪现旧人旧事。正是出神之际,他下意识的猛一歪头望向房门,就见紧闭着的房门前方,探头探脑的飘进了一只鬼。

此鬼形容凄惨,生前不知被谁把半边脑袋敲了个稀烂,一只眼珠被挤出眼眶,险伶伶的吊在脸上;一身工人装更是遍布鲜血,看不出本来颜色。无心立刻半闭了眼睛,想要看看对方意欲何为。而惨鬼试试探探的飘到炕边,伸手想要推他,可惜力量微弱,一只手纯粹只是幻影,连阵风都扇不动。

惨鬼仿佛是急了,开始呼唤:``哎,醒醒,醒醒啊!我知道你是能看到我们的,你睁眼呀!''

无心装聋作哑,一动不动。

惨鬼原地转了个圈,飘飘荡荡的穿墙而出。不过片刻的工夫,他带着四名同伙回来了。四名同伙全和他是相似的打扮,有的死相还算干净,有的则是没个人样。无心眯着眼睛,就听他们在房内嘁嘁喳喳,正在商量如何把自己叫醒。一番谈论过后,四鬼站成一排,惨鬼站在人前,抬起双手打起拍子:``天大地大——预备——唱!''

四鬼一起发声,开始小合唱:``天大地大不如党的恩情大,爹亲娘亲不如**亲,千好万好不如社会主义好,河深海深不如阶级友爱深\ldots{}\ldots{}''

一曲终了,惨鬼回头往炕上看:``他怎么还没醒?''

五只鬼实在是能力有限,连根针都拈不起,站在炕前干着急。无可奈何之下,他们在房内又跳了一阵忠字舞,唱了五遍国际歌。无心被他们吵得心乱如麻,不得不睁开眼睛望向了他们。而他们见无心总算醒了,立刻一起向房门指,仿佛是要让他走。

无心不出声,做了口型问道:``干什么?''

惨鬼答道:``有人找你。''

无心又问:``谁?''

五鬼一起摇头:``不知道。''

无心想了一想,伸手捏住白琉璃的尾巴尖晃了晃。白琉璃缓缓的蜷缩身体回了头,无心没言语,只对他使了个眼色,又把一边眉毛向地下的五鬼一扬。白琉璃会意,慢吞吞的又趴下了。

无心穿了衣裤,系好鞋带,随着五鬼悄悄出门。大门口有民兵站岗,他怕受人盘问,故而翻墙而出。五鬼直接穿墙,鬼鬼祟祟的领着他往镇外走。都走出老远了,领头的惨鬼才发现了问题:``怎么少了两个?''

众鬼面面相觑,又一起去看无心。无心饶有兴味的问道:``看什么?不过是少了两只鬼而已,兴许他们刚投胎去了呢!说实话,到底是谁让你们来找我的?''

余下三鬼现出了一点可怜相:``同志,我们真不知道。他住在洞里,我们没有见过他的脸。''

无心嗤之以鼻:``胡说八道!难道你们想要见谁,还得走大门不成?''

三鬼当即保证:``我们可没胡说。他呆的地方,我们进不去!''

无心看出它们三个无论做人做鬼,大概都是糊涂蛋一流,所以不再废话,继续前行。与此同时,留在房中的白琉璃吞了两只慢走一步的可怜鬼。脱出蛇身站在房内,他心旷神怡的看看炕上的小姑娘,再看看窗外的大月亮。

无心在夜色中疾行了一个小时,进入了紧挨镇子的猪头山矿区。矿里上下全忙着闹革命,生产早停止了。三只鬼恪尽职守的领着他穿过一片荒凉厂区,末了停在一处小山包前,他们不动了。

小山包是座石头山,下方黑洞洞的掩着两扇大铁门,门缝中隐隐透出微光,可见山体中应该是开辟出了一座仓库,或者是一处防空洞。无心抛下三鬼,径自向前走。及至走到铁门前,铁门却是自动开了。

一名全副武装的青年探出了头,目光锐利的审视了无心。而无心向内一望,就见半空中吊着个昏黄的小灯泡。灯泡之下有限的一圈光明中,摆着一炕桌简单酒菜。小丁猫在桌后席地而坐,一手夹着香烟,一手端着酒杯,笑吟吟的对着他一点头。

无心不等人让,自动的绕过青年走到了桌前。小丁猫放下酒杯,歪着脑袋吸了一口烟,然后喷云吐雾的抬手做了个下压动作:``坐。''然后他端起酒杯,津津有味的又咂了一口。

无心看了他这个连抽带喝的劲儿,忽然有点不知从何说起。弯腰在水泥地上盘腿坐了,无心思索着问道:``你\ldots{}\ldots{}还好?''

小丁猫深深的吸了一口气,然后长长的呼了出来:``我是还好。暮色苍茫看劲松,乱云飞渡仍从容。你看我从不从容?''

无心扫了桌面一眼:``从不从容我不知道,不过我看你饭量倒是见长。''

小丁猫笑了,从桌角烟盒里抽出一根香烟续到嘴上:``听说马秀红死了?''

无心点了点头,随即抬眼望向了他:``小丁猫,你到底是谁?''

小丁猫夹了一筷子凉菜送到嘴里,边嚼边答:``我是小丁猫。''

无心手摁桌沿:``我问的是过去,不是现在。''

小丁猫吱喽一口酒,又抽了口烟:``过去我是小小丁猫。''

无心摁着桌子站起身,扭头走向门口:``我回去睡觉了。''

小丁猫嘿嘿发笑:``回来回来,我们有话好说,你急什么?莫非你和陈大光混出感情了,现在看我很闹心?''

无心转身又回来了:``你一句实话都没有,我们怎么谈?''

小丁猫从身后拎出一只白酒瓶子,对着无心晃了晃:``说来话长,给我坐下。茅台,要不要来几杯?''

无心皱起眉毛望着小丁猫,看他眉飞色舞满嘴闲话,忽然很想揍他一顿。

作者有话要说:祝大家元旦快乐,在新的一年里平安健康O(∩\_∩)O\textasciitilde{}

\chapter{原来是他}

小丁猫从桌子下面摸出一只搪瓷杯子,倒了半杯酒送到无心面前。然后放下酒瓶抄起筷子,他端起了手边的一只小碟子。碟子里摞着几只油汪汪的荷包蛋,他叼着香烟垂下眼帘,夹起一只软颤颤的荷包蛋送到了无心面前的菜盘子里:``吃吧,是溏心的吗?''

无心看了一眼:``好像是。''

小丁猫一听,伸筷子把荷包蛋又夹回去了:``是的话给我,我爱吃溏心的。''

无心看他一张嘴同时抽烟说话吃菜,分明是很不够用,就不耐烦的摆了摆手:``我不饿,你也别管我。我的本意不是陪你吃喝,你找我有话说,我对你也有话说。你我都别弄玄虚,有一说一吧!''

小丁猫用手指夹了烟,端着碟子先把溏心荷包蛋吃了,然后伸出舌头一舔嘴唇上的蛋黄,对着门口青年说道:``小李,你回避一下。''

青年答应一声,开门出去。空洞洞的黑屋子关了门,只剩了灯下的无心和小丁猫。无心望着小丁猫,轻声问道:``你在很久之前见过我,对不对?''

小丁猫微笑点头,抿了口酒:``对!''

无心听他回答得痛快,心中反倒越发生疑:``你到底是谁?''

小丁猫抬手一扶眼镜,对着无心喷出了一口烟雾:``猜!''

无心盯住了小丁猫的眼睛:``你不会是\ldots{}\ldots{}岳绮罗吧?''

小丁猫一摆手:``错!我要是老岳,早把你活吃了。''

无心彻底糊涂了:``你认识岳绮罗?''

小丁猫歪着脑袋,一本正经的反问:``不是你把她送给我的吗?你应该看得出,我很领你的情啊!''

无心眨巴眨巴眼睛:``你到底是谁?''

小丁猫往地上弹了弹烟灰:``你先告诉我,你是个什么东西。我算着也有几十年了,你怎么一点都没变样?你说实话,别和我打马虎眼。我的记性很好,绝不会认错了人。''

无心一拍桌子,恍然大悟:``我知道你是谁了!你是鬼洞里的——你出来了?''

小丁猫深深的一点头:``聪明!''

无心难以置信的睁大了眼睛:``你\ldots{}\ldots{}我记得你是个女的呀!''

小丁猫嘿嘿一笑:``何以见得?你检查过?''

无心把一双眼睛睁到了极致:``不可能!你绝对是个女人!别看我只见过你一面,我记得很清楚!''

小丁猫看了他硕大无匹的黑眼仁,当即抬手一挡眼睛:``你闭眼吧,太吓人了。我是男是女,我自己还不知道?你凭什么非说我是个女人?你看过我的×了?我人在坛子里,难道你是透视眼?''

无心端起搪瓷杯子喝了口酒:``好,好,就算你是男人。''

小丁猫向他一举酒杯:``美男子。''

无心一点头:``好,好,就算你是美男子,然后呢?''

小丁猫叹了口气:``木秀于林,风必摧之;堆高于岸,流必湍之;行高于人,众必非之。具体的往事我记不清了,总而言之,我是不得好死,死后还不得安葬,被人斩断手脚装进了坛子。你不知道,土洞下面是有阵法的,若不是猪头山被炸开了,我现在可能还在洞里。''

无心微微向他探了头:``然后呢?''

小丁猫深吸了一口烟:``然后?我生前是个热心肠的好人,死后虽然身陷地狱,不得超生,但是我心一如生前,并不祸害山中生灵。''

无心都听了他的妙语,几乎惊呆了:``好好的地洞让你搞成了一个有进无出的鬼窟,你还说你是好人?''

小丁猫夹着烟卷一瞪眼睛:``客人来都来了,我身为主人,还能往外撵吗?谁家的主人不留客?我和客人又没有仇!''

无心听得哭笑不得:``好,好,然后呢?''

小丁猫望着灯泡,悠然神往的喷云吐雾:``然后,老岳就来了。老岳本事不小,脾气也不小,在洞里闹了好几年。不过我宅心仁厚,最后还是感化了她。她什么都好,就是一根筋,念念不忘的想要杀了你。我觉得打打杀杀不大好,你认为呢?''

无心深以为然:``是不大好。尤其是打我杀我,就更不好了。然后呢?''

小丁猫一扬眉毛:``然后?然后猪头山被人挖得四分五裂。我赶在炸山之前,把陪伴我多年的朋友们——包括老岳——全部吃掉了。老岳不甘心,总在我心里折腾,搞得小时候家里人以为我有精神分裂症,直到十年前,她才安静了。''

无心听了小丁猫的话,感觉自己也要精神分裂:``哦,她还没有魂飞魄散?''

小丁猫歪着脑袋凝视无心,半晌没言语,末了才答道:``说不清楚,我们好像已经合二为一了,否则不能解释为什么我有时候见了你会百感交集。''

无心的眼睛恢复了正常大小,同时向后略躲了躲:``你\ldots{}\ldots{}是个男的吧?''

小丁猫用香烟向下一指自己的裤裆:``脱了给你看看?''

无心连忙摇头:``不不不,我相信你。''

小丁猫慨叹一声:``老岳是个学富五车的人,在洞里几十年,教会了我很多知识。否则我现在至多做个孤魂野鬼,哪能转世成人?话说回来,无心,你到底是怎么回事?我记得当年你和老岳在我的洞里一阵好打,你走的时候可都没人样了。''

无心被他戳中心事,想要自称天人,又怕遭他嘲笑。端起酒杯又喝一口,他撩了对方一眼:``不知道。我一直不死,我自己也没办法。''

小丁猫抄起筷子翻翻捡捡,又挑了个溏心荷包蛋吃了。舔着嘴唇抬起头,他回归了现实:``我越狱了,你还跟不跟我干?''

房中寂静片刻,无心忽然说道:``我记得你真是个女人。我见的人多了,不会分不清男女。''

小丁猫把筷子往桌上一拍:``本来我心里藏着个老岳就够难受了,你他妈的还总说我是女人。是不是非得等到我把你妈日了,你才承认老子是带把儿的?``无心听了,毫不动气:``你要是能找到我妈,我甘愿叫你一声爸。''

此言一出,小丁猫被他堵得打了个饱嗝。

小丁猫对着无心抽了半盒香烟,并且不再正视他的眼睛。岳绮罗的灵魂埋伏在他的血液骨骼肌肉之中,无影无形、无处不在;而他没有力量把岳绮罗彻底消化掉。和无心对视的时候,他会不由自主的替岳绮罗痛苦,虽然他本人对无心并无意见。他其实早已完全清楚无心的底细,所以格外希望他能成为自己的左膀右臂,就算成不了膀臂,留他在身边也是好的。为什么好?他不知道。他转生时几乎是和岳绮罗分享了一具婴儿身体,虽然后来他很快占了上风,但是岳绮罗的影响,他永远摆不脱。

然而无心告诉他:``我不想跟着你。你们的事业,我没兴趣。我不会去向陈大光告发你的行踪,你是谁不是谁,和我也没有关系。我走了,别找我。''

小丁猫笑着问他:``万一我将来把红总给灭了,你怎么办?''

无心站起了身:``到时再说。''

小丁猫一挑眉毛:``好,我们到时候见。''

.

无心转身向外走去,守门的青年听到了脚步声音,自动从外拉开铁门。时到深夜,山里空气微凉,带着一点新鲜的草木香。无心大踏步的向前走,同时感觉自己方才只是做了个荒谬的梦。

一路疾行回了镇子,他翻墙进入公社大院之时,正见自己房内白光闪烁。蹑手蹑脚的推门进去一瞧,他先是大吃一惊,不知道是谁惹恼了白琉璃,气得他手舞足蹈的发疯;静观片刻之后,他转而啼笑皆非,发现原来是白琉璃在学人跳忠字舞。

听着苏桃气息均匀,睡得很熟,他轻轻的进房关门,一边脱衣服一边问道:``大半夜的,闹什么呢?''

白琉璃停了动作,悬在半空中问道:``怎么样?是谁找你?''

无心上了炕:``是位故人——应该算一位还是算两位,我说不清楚。''

白琉璃又问:``到底是谁?你告诉我。''

无心抬头答道:``他们活着的时候还没有你呢,说了你也不认识。算了,睡觉吧!''

话音落下,无心往下要躺。可在将躺未躺之际,他忽然又坐起了身:``白琉璃,问你句话,你还想不想转世投胎了?''

白琉璃摇了摇头:``不想。''

无心得了答案,彻底躺下。既然白琉璃愿意做鬼,他就更没有必要去和小丁猫合作了

\chapter{夜宿黑水洼}

无心和苏桃坐在台下,仰着脸看台上正在表演的群口相声《绞索套住美国佬》。陈大光在猪嘴公社住了几天,视察了公社大大小小的生产队,如同新皇帝视察自己的领土,越看越美,处处都要亲自走到。如今在他离开公社之前,公社特地又开了一场联欢会,专为了让县里干部高兴。

快板书一结束,报幕员昂首挺胸的上了台,高声说道:``下面请听快板《多米尼加人民想念**》!''

台下响起了热烈的掌声,无心和苏桃趁机低头,一起往嘴里塞了一块硬糖。

联欢会以欢乐为主,一场快板结束之后,活报剧《美蒋特务无处逃》上演,其中女主角生得明眸皓齿,导致陈大光直了眼睛垂涎三尺。及至联欢会落了幕,陈大光春情勃发的上了台,骚头骚脑的发表讲话:``看了同志们的表演,我很受感染,不由得兽性大发,要为猪嘴公社作一首诗!''

台下众人听他诗兴变兽性,略有知识的都含笑低头。而陈大光清了清喉咙,高声诵道:``猪头山下大草原,猪嘴社员意志坚。主席思想照方向啊,敢叫荒山变良田!''

四面八方立时掌声雷动,虽然猪头山下并没有大草原。

陈大光发散了诗兴,又和活报剧女主角进行了亲切的谈话。末了受到时间的限制,他恋恋不舍的上了吉普车,前往妃子岭公社。妃子岭公社和猪嘴公社一样,是个大社,辖着五个生产大队。五个生产大队全卧在山窝子里,东一处西一处,相距甚远。陈大光不出县城,还不知道自己的领土面积。如今当真一步一步的走了,才发现自己是真了不起。

在一个阳光明媚的下午,陈大光一行抵达了喇嘛山生产队。无心和苏桃坐久了马车,颠得浑身骨头疼。进村之后得了自由,两人在井台旁的大树荫下坐了,无心从书包里掏出一根早熟的水萝卜递给苏桃。水萝卜不过是巴掌长,红皮白心又甜又辣,苏桃咬了一口,嚼的嘴里喀嚓喀嚓。无心低着头,把另一根水萝卜从白琉璃的利齿上往下摘——白琉璃自作主张的趴在书包里仿效神农尝百草,无论见了什么食物,都要张嘴咬上一口。结果今天倒钩牙扎进水萝卜里,吞不下拔不出,他的大嘴张了小半天。

无心知道白琉璃嘴里干净,所以并不嫌弃。摘下水萝卜之后咬了一口,他在满嘴新鲜汁水中倾斜身体,用肩膀轻轻一撞苏桃。苏桃一边嚼水萝卜,一边摇晃着撞了回去。

``要是总能在外面逛\ldots{}\ldots{}''苏桃说道:``也挺好。''

无心三口两口吃光了水萝卜,侧了身去解苏桃的辫子。头发乱了,辫子毛刺刺的不像话。苏桃小口小口的啃着水萝卜,任凭无心用手指为自己梳通头发。一条辫子利利落落的编好了,苏桃转了个身,把另一侧乱发送到无心面前。无心距离她很近,她的眼角余光可以瞥到他的眉目。指甲划过头皮,指头穿过黑发,嘴里的水萝卜忽然失了滋味,她怔怔的望着前方,听见自己的心在跳。

重新束好的辫梢垂到胸前,她慢慢的扭脸去看无心。其实她才真是``自绝于人民''。除了无心,她谁都不认。在人间,她与一切绝缘。

无心迎着她的目光微笑了:``看什么?我可没头发给你梳。''

苏桃也跟着笑了,抬手轻轻去摸无心的脑袋:``你的头发怎么总也不见长啊?''

无心答道:``不长还不好?省了去理发店的钱。''

苏桃收回了手,小声笑道:``一年能省好几根冰棒。''

无心正要回答,不料忽有一名青年从远方呼喊道:``无心,你俩也来吧!喇嘛山住不下,陈主任要带咱们几个先去黑水洼。''

无心拉着苏桃站起了身:``去黑水洼?去黑水洼不是还得翻一座山吗?''

青年且行且答,越走越远:``现在出发,翻山也来得及!''

无心无可奈何,只得和苏桃强打精神往大队部走。大队部里已经预备好了一架大马车,因为从喇嘛山生产队到黑水洼生产队,其间翻山越岭,虽然也有一条柏油道路,但是入夏之后经了几场大暴雨,路上几段山体滑坡,早已不能通行。而不走公路走山路的话,再好的吉普车禁不住颠簸,所以无论是为了人还是为了吉普车,都是乘坐马车更合适。

县里干部下了乡,都是住在村民家里。喇嘛山太穷了,村中以东倒西歪的土坯房为主,像样的房屋没几间。县里干部都是天仙一般的人物,怎能把他们往半塌不塌的破房子里安顿?陈大光一贯爱在小事上面发扬风格,横竖早晚都得往黑水洼走,早走晚走都一样,于是带上几名伶俐心腹,他先行出发了。

无心和苏桃吃了两根水萝卜,本以为可以舒舒展展的逛一下午,没想到又被陈大光抓上了马车。抱着膝盖坐在车板子上,他们颠出了一路的萝卜屁。幸而革命群众们一贯豪迈,不以放屁为耻。苏桃深深的垂着头,恨不能把脑袋缩进衣领子里,无心俯身用双手捧着脑袋,造型也是十分忧郁。白琉璃盘在书包里不明就里,还以为是路况不好,马车作响。

大马车走了两个多小时,暮色苍茫之际,终于抵达了黑水洼。黑水洼生产队的大队长知道县革委会主任要来,但是记忆中的时间是明天,如今骤然听说陈大光下凡了,吓得趿拉着鞋往外跑。及至听说陈大光是来投宿的,大队长立刻派人把自家房屋收拾出了两间,自己则是带着妻儿老小住到了大队部里。照理来讲,两间房屋也就够一马车的人居住了,可是一马车的人中有个苏桃,无心和苏桃又是绝不拆伴。苏桃大小是个女的,虽然已经是公认的不检点,但是只对无心一个人不检点,还不能算是骚狐狸精。陈大光一时发□心,又见邻居也是砖瓦房子,就让大队长去了一趟隔壁,额外要了一间干净屋子给无心和苏桃居住。

无心很是感激陈大光的好意,及至吃过有酒有肉的晚饭过后,他让苏桃带着白琉璃回房休息,自己陪着陈大光在村里溜达。村民们得知县里来人了,因为怯官,吓得不敢出屋,村巷之中一片寂静。大队长带着几个大队干部尾随了陈大光,察言观色的说说笑笑。如此走了不久,前方一户人家门户大开,却是传出隐隐的哭声。陈大光停了脚步,伸手向前一指,回头问大队长赵广和:``老赵,怎么回事?''

赵广和勃然变色,变色之后忽又笑了:``陈主任,他们家我知道,前天死了闺女,还没出殡呢。''

陈大光听了,心不在焉的又问:``他们家什么成分?''

赵广和立刻答道:``地主。过去全村数他家是第一富,把咱们贫下中农都压迫惨了。''

陈大光一扬眉毛:``一个地主后代,死就死了,还嚎什么?现在大好形势一片大好,他们至于为个丫头往死里嚎吗?''

赵广和摩拳擦掌:``陈主任说得对,他们一家子牛鬼蛇神,不知道是为谁嚎呢!''

陈大光点了点头:``再说你听他们嚎的驴叫一样,影响也不好嘛!''

话音落下,他忽然感觉袖子一紧,转脸一瞧,发现是无心扯了自己。莫名其妙的一挑眉毛,他当众一挥手:``你们都往后去,我和他说两句话。''

等到大队长等人当真后退了,陈大光就听无心说道:``院子里的人,不是好死。''

陈大光一瞪眼睛:``莫非里面有阴毛?''

无心听了他的言辞,当即想笑,但是强忍着没敢笑:``没有阴谋,我只是说死者不安,阴魂不散,你没事最好不要再往前走了,万一冲撞了什么,对你不大好。''

陈大光想了想,低声又问:``你的意思是\ldots{}\ldots{}他们家的死人能复活?''

无心缓缓的摇了头:``不是\ldots{}\ldots{}总之我感觉他们家里阴气太重,所以劝你一句。''

陈大光伸手一指他的鼻子尖:``你是越来越吓人了。''

陈大光听人劝,吃饱饭,果然背着手往回溜达。将要走回住处了,他偶然回头一瞧,忽然发现无心不见了。

不动声色的抬手摸了摸藏在腰间的手枪,陈大光犯了嘀咕,心想难道自己又要见鬼了?

与此同时,无心已经悄悄的按照原路返回。觅着哭声走到院门前,他迈步跨过门槛,停在了院中一对老夫妇的面前。老夫妇都是衣衫褴褛的模样,身下也没个板凳,东倒西歪的坐在地上。身后房门大开,可见屋内黑洞洞的家徒四壁,正中央摆着一扇用砖垫起的门板,门板上面直挺挺的躺着一具尸首。

老夫妇骤然见了生人,连忙互相扶持着站起了身。无心不等他们相问,直接开口说道:``死了几天了?''

老太太蓬着一脑袋白头发,仿佛是被人欺负狠了,颤颤巍巍的有问必答:``两、两天了。''

无心歪着脑袋又看了看房中尸首,发现尸首竟然穿了一件肮脏的红袄,头脸上面则是盖了一块四四方方的白布。

``没给孩子换身衣裳?''无心问老太太:``没有新衣裳,旧的也行。''

老太太狠狠的一闭眼睛,挤出了一串大眼泪珠子。无须回答,无心明白了——旧的也没有。

无心叹了口气,又道:``赶在太阳落山之前,把她火化了吧。自己的孩子自己知道,我不多说。''

话音落下,他转身要走。老头子一把抓住了他的手臂:``你等等!''

无心停了脚步:``有些话你不用说,我也不用听。火化尸首,应该不算反革命行为。''

老头子长长的吸了一口气,再开口就又带了哭腔:``我家姑娘走得不甘心哪!没有比她更冤枉的了。''

无心转回了身,对着老头子说道:``我不是县里的干部,我也不能给你伸冤。''

老头子大概是很久都没有受过外人的好意了,听了无心的话,他一脸眼泪顺着沟壑纵横的皱纹往下淌。气喘吁吁的抬起手,他往东指又往西指,口中用气流送出颤声:``他们都知道\ldots{}\ldots{}他们都知道\ldots{}\ldots{}没人说一句公道话\ldots{}\ldots{}''

语无伦次的,老头子诉说了自家姑娘的死因。原来姑娘名叫小翠,今年刚满十七岁,生得有模有样,正经是个漂亮姑娘——她要是不漂亮倒好了,就因为漂亮,才落进了大队长赵广和的眼里。赵广和作为黑水洼一霸,爱好与陈大光十分类似,专爱赏鉴妇女。小翠被他祸害了一年,村民们因为不敢评论赵广和,不说话又憋得慌,于是柿子挑软的捏,统一的认为小翠是只骚狐狸。年初小翠怀了身孕,由于没结婚,开不出介绍信去医院做流产手术,所以赵广和把她堵在屋里,直接用拳脚给她堕了胎。

然后,小翠就疯了。

穿上家里压箱底的小红袄,她满村里哭哭笑笑的乱跑。爹娘忙着干农活,没时间看管她,结果她自己爬上高大山石,跌下来摔死了。

``我知道小翠不对劲\ldots{}\ldots{}''老头子见神见鬼的告诉无心:``她一直在七窍流血,流了两天一夜。我去找了村里的半仙,她用蜡封了小翠的七窍,封了七次都封不住。不对劲就不对劲吧,我和她娘都不怕她。不做亏心事,不怕鬼叫门。我们不怕,有人怕。''

无心本意是要劝老两口毁灭尸首,然而听了老头子一席话之后,他决定不管闲事,回去睡觉。可就在他预备告辞之时,院内忽然掠过一阵凉风,屋内小翠脸上的白布帕子被风掀起一角,露出了半边扭曲面孔。

无心回头望向院门,想要看看风的来历。耳边骤然响起两声惊叫,他连忙望向面前两位老人,就见老夫妇两个一起伸手指向房内。而方才还停在门板上的尸首,居然在一瞬间不见了。

又一阵凉风穿屋而过,吹得两扇破窗呱嗒呱嗒直响。无心心中一寒,只觉周遭阴气陡然上升。正要转身往院外走,他两条手臂忽然一痛。抬眼望去,就见老两口子分别拽住了自己的胳膊,两双浑浊老眼陷在松垮眼皮里,方才黯淡的目光已经转为锐利。眼看手臂被死死的禁锢住了,他猛的向下弯腰侧身,把衣服前襟送到手边。扯住一边衣襟狠狠一拽,纽扣粒粒崩开,而他身体下蹲顺势一溜,双臂从衣袖之中飞快的抽出。随即一脚踹倒了最近的老头子,他转身几步冲出院门,在昏暗的暮色中大声喊道:``陈大光!出事了!''

\chapter{暗影重重}

无心且奔且喊,喊过两三声之后忽然闭了嘴,发现自己喊的内容不大对劲。回头向后望了一眼,后方并没有追兵,村巷依旧是空空荡荡,只有最近的一扇院门微微开了一道缝隙,一只眼睛惶惑的凑在门缝后方,是个战战兢兢的偷窥者。

无心停了脚步移动目光,要和门缝中的眼睛对视。那只眼睛立刻战栗着闪开了,摇摇欲坠的柴门也立刻关了个严丝合缝。与此同时,在几条巷子外的民兵小队闻声而来,因为认出无心是从县里来的干部,所以格外的紧张:``同志,怎么了?''

无心慢慢的抬手指向了巷子深处的小翠家:``那边有敌人在搞破坏。''

民兵一听,立刻来了精神:``他们干什么了?''

无心思索着答道:``我从她家门前经过,她家的人\ldots{}\ldots{}抢了我的上衣!''

民兵有点儿傻眼:``啊?他家还敢明抢?两个老不死的真是嚎丧嚎迷了心。同志你不要怕,我们这就过去一趟!''

话音落下,几名民兵雄纠纠气昂昂的走向前方。及至到了小翠家门口,两扇院门大开,为首的民兵大踏步的走进院子一瞧,登时发了傻:``不对啊,他家的小翠,不是明天埋吗?''

其余众人紧随而入,因为房屋只有两间,所以一瞬间就搜查完毕了。小翠没有了,老两口子也不见了。暮色黯淡苍茫,天边却是一片胭脂红。民兵们面面相觑的站在院内,有人说道:``那两个老×不会是埋人去了吧?''

听众之一打了个哈欠,把脑袋伸出院门向巷子口望:``县干部已经走了,咱们也回去歇着吧。要不然怎么办?到坟地里找人去?''

此言一出,立刻得到旁人的附和:``对,明天再说吧。明天让队长拿主意。''

在民兵们意意思思的往外撤时,无心已经见到了陈大光。把今夜的见闻原原本本讲述了一遍,他最后告诉陈大光:``夜里睡觉惊醒一点,不怕一万,只怕万一。''

陈大光差一点就想邀请无心与自己同眠了,不过转念一想,又怕自己露怯丢人:``无心哪,如果有事的话——我是说如果,能有什么事?像在县里似的,死人复活了找活人报仇?''

无心被他问住了:``我又不是鬼,不知道她的心思。大概\ldots{}\ldots{}是吧!''

陈大光沉吟着点了点头,不再多说,只在上床之前偷藏了一把柴刀。只要敌人是有形的,无论如何凶悍狠厉,他都有信心把对方剁成肉馅。

无心对陈大光尽过了心,忙忙的出门进门,回了自己所住的小院。推门向内一瞧,他发现苏桃刚刚洗了头发,此刻正坐在炕沿上满头满脸的擦拭水珠。一手挽着沉甸甸湿漉漉的长发,一手托着条半干的白毛巾,她含着胸脯,仿佛带不动头发一样,偏着脸儿去看无心。外面的的确良衬衫和里面贴身的小背心都脱掉了,她身上就只剩了一层薄薄的汗衫,领口袖口都洗得失了形状,松松垮垮的勾勒出了她的身体线条,前胸鼓着影影绰绰两只毛桃。

房内亮着一盏油灯,无心一边关门,一边吸了一口空气中的水汽:``洗头发了?''

苏桃仿佛时刻防备着外人窃听似的,小声答道:``嗯,可算洗成了。昨天我一解辫子,闻着头发都馊了。''

然后她放下毛巾一甩头发,粉白的面孔半隐在潮湿乌黑的长发之中。抬手把乱发掖到耳后,她抬脚往炕上缩:``我给你留了一盆水,在地上呢。''

无心走去拿了她的毛巾,而她就自动的转身背对了炕下,自己垂头用一绺头发去逗白琉璃。无心很潦草的洗漱一遍,又拧了毛巾浑身擦了擦汗。末了一口吹灭油灯,他关门上炕,拍了拍枕头说道:``桃桃,今晚我们一头睡。''

苏桃愣了一下,但是也没有多问。四脚着地的爬到无心身边躺下了,她不假思索的枕上了无心的手臂。抬眼望向对面的无心,她忽然开口问道:``无心,多大年龄才能结婚呀?''

无心抬起一只手,张开五指和她合掌:``多大年龄?我不知道,不是十八就是二十,不是二十就是二十二,总之非得是大姑娘才行。''

合拢手指握住了苏桃的手,他微微低头去看她的眼睛:``怎么?陈大光又催你和我扯证了?''

苏桃晚上根本没见陈大光的面,然而也没有辩解,只在心中暗算。取个中间值吧,就算是二十。她离二十岁还有五年的光阴,对于十几岁的孩子来讲,五年真是漫长的几乎吓人。

试探着把额头抵上无心的一边锁骨,她低声又问:``无心,破房子里的波斯菊,现在是不是已经开成片了?''

无心推着她的肩膀,把她翻成了背对自己的姿态。全神贯注留意着房屋内外的动静,他心不在焉的随口答道:``当然。''

苏桃是个悲观的人,甚至不知道自己能不能活到五年后去扯那一张结婚证。回忆着暮春时节他们住过的废墟和废墟上要开未开的波斯菊,她满心苍凉的闭了眼睛。小腿上面有一点分量在动,是白琉璃摇头摆尾的要凑上来了。一个温凉的圆脑袋触了触她的手心,她轻轻动了手指,在白琉璃的脊背上摸了一下。

屋中越发黑暗寂静了,可以听到隔壁的房东夫妇在打呼噜。炕是三面靠墙砌在了窗下,无心睁眼望着窗外,先前进村时不留意,倒也罢了;如今心里起了提防,才发现此地的风水阴气很重。黑水洼整个儿的坐落在群山之中,大山遮天蔽日的围成一圈,让黑水洼阳气不通阴气不动。当然,偏阳偏阴都不是大事,小问题而已,既不伤人也不害命;可是村里新添了厉鬼,阴上加阴,就有点不好办了。无心用一条手臂松松的环住了苏桃的腰,同时看到外面漆黑一片,天幕之上无星无月。忽然一股子异常的气息惊动了他,他狐疑的坐起了身,感觉门外似乎是来了妖精。

妖精属于阴邪一路,和人相比,它们倒是和鬼更亲近。无心对着白琉璃使了个眼色,然后下炕穿鞋,悄无声息的往外走。越是靠近门口,妖气越重,但是此妖气与众不同,十分清新,不带血气。推开房门向外一瞧,他看到院墙头上果然有活物,乃是一只灰扑扑的大猫头鹰。

猫头鹰很常见,是种昼伏夜出的动物,美也不是很美,坏也不是很坏,等闲无人去招惹它。猛的发现有人出来了,它蹲在墙头一动不动,只发出了一串凄厉喑哑的叫声。

一般来讲,村民对它都是视而不见,因为嫌它不是个吉利东西。它一出声,更是预示着要出人命,然而无心并不理会它的警告。蹑手蹑脚的一直走到院墙前,他昂首挺胸的和猫头鹰对视了。猫头鹰是大眼睛,他也是大眼睛。双方大眼贼似的对视良久,末了猫头鹰眼中的光芒忽然一收,又侧了身抬起一只翅膀,掩住自己凶恶的尖嘴。乌溜溜的大眼睛漾起一层亮晶晶的泪光,它换了一副楚楚可怜的嘴脸。

无心弯腰把鞋脱了一只,抡起手臂对着猫头鹰就是一鞋底子:``少对我装可爱,你给我往远走!''

猫头鹰被他拍得一晃,立刻拍着翅膀飞了。原来此猫头鹰活了上百年,当真是带有几分妖气。为妖作怪的东西,都爱往阴气重的地方走,因为利于修行。如今它有所知觉,趁着夜色飞来黑水洼,想要吸取几分鬼魅的精华。不料刚在一家墙头上停稳了,便和无心对了眼。它虽然也有尖嘴利爪,但是胆子奇小,以和为贵。无端的挨了一鞋底子,它不敢恋战,扇着大翅膀飞到别人家去了。

无心回了房,守着苏桃熬了一夜,莫说是鬼,屁也没有等来一个。翌日天明,朝霞如火。赵广和听说小翠家无端的没了人,县干部还被小翠的爹娘抢走了一件上衣,便气势汹汹的带着人杀了过去,把小翠家抄了个底朝天。

吃过早饭之后,陈大光打着哈欠,开始和赵广和谈工作。谈了没有几句,小雨下起来了。

下小雨的时候,谁也没当回事。不料小雨越下越来劲,居然很快转为中雨,又转为大雨。大雨一下,黑水洼向外的交通就算是彻底断绝。陈大光出不去,原定中午从喇嘛山出发的其余干部也进不来。

无心和苏桃百无聊赖的混到傍晚,倒是足足的休息了一整天。夜里雨水停了,大队部里亮起了电灯,赵广和召集了村里的宣传队,要让陈主任看看自己的宣传水平。村民一天没出工,吃过晚饭后听说有节目看,三三两两的都凑来了大队部。而宣传队里的大姑娘小媳妇训练有素,直接就把大队部的一间空屋当成了后台。

赵广和先是陪着陈大光看样板戏,看着看着他起了身,偷偷进了空屋。屋中留着个小媳妇,正在对着镜子安装假辫子。赵广和和她亲嘴摸乳的嬉闹了一番,眼看就要成就好事了,小媳妇却是推了他一把,说是憋着尿呢,得先去趟茅房。

赵广和放她去了,自己掩了房门等待。屋子里的气味不算好闻,妇女和妇女也是不一样的,未必人人都是香香肉,尤其到了夏天,更是有的一身汗香有的一身汗臭。不甚自在的抽了抽鼻子,他眼角忽有红影闪过。犹犹豫豫的扭过头,他睁了眼睛向后瞧。

在赵广和等待之时,小媳妇匆匆忙忙的撒了尿。系好裤带跑回空屋,她一推门,就见赵广和正在扭头向后瞧。

回身关了房门,小媳妇笑问:``看什么呢?你再不动,台上的人可要唱完回来了。''

话音落下,赵广和木雕泥塑一般,依旧一动不动。小媳妇看他固执的出奇,索性上前拽了他一把:``有什么好的让你看直了眼?''

赵广和应声而倒,向前仆上了小媳妇的胸腹。而小媳妇居高临下的看清了,登时发出一声惨叫——赵广和满脸是血,眼睛鼻口都被撕扯成了血窟窿,哪里还有活气?

惨叫之声穿透墙壁,直达前台。民兵队长一个挺身先起来了,扛着一杆步枪就往大队部里猛冲。余下观众面面相觑,未等有所反应,大队部内响起了民兵队长的吼声。

旁人不知所谓,陈大光却是心里隐隐的有一点数。扭头和无心对视了一眼,他稳如磐石的坐着不动。不是什么人都可以享受到他的保护,他的螳螂拳只为自己而出。

台上的歌声停了,半空中起了几声猫头鹰叫。观众们一起打了哆嗦,知道这叫声有多么不祥。民兵队长拖着步枪跑出来了,变脸失色的叫道:``赵队长死了!有人杀了赵队长!''

大队部院里的电灯忽然熄灭了,不止一个人联想起了无故失踪的小翠一家。陈大光不能不发话了,命令民兵点起火把,他大包大揽的要亲自去后台查看现场。

无心被他点了名,必须跟随,苏桃则是和其余几名同行的县里干部站在前院。及至见过了赵广和的尸首,陈大光随口说道:``阶级敌人真是丧心病狂——''

话音未落,无心用力一扯他的后衣襟。他当即闭了嘴,怀疑自己是说错了话。转身正要往外走,他忽听无心发出疑问:``谁把房门反锁了?''

陈大光心中一惊,同时抽了抽鼻子:``无心,你闻没闻到臭味?''

无心记得自己随着陈大光进屋时,民兵队长就站在门口,并且还为自己开了房门。深深的吸了一口气,他开口答道:``陈主任,我闻到了。''

空气中的确是夹杂了一股子腥臭。无心越想越是不对,一脚踹向门板,他高声呼喊民兵队长:``小李,开门!''

陈大光扯开无心,正想飞出一脚。不料就在他运力之时,脖子上忽然森森的一凉,抬手摸时,他怪叫一声,因为摸到了几根黏腻纤细的手指。无心回过了头,就见一个身穿红袄的女人站在陈大光身后,双手紧紧锁住了他的脖子。女人的披头散发之中显露出了面孔,面孔竟是一片模糊,整张脸都覆上了凹凸不平的白色蜡油!

``小翠!''无心大声喊道:``我们是外来的人,没有害过你,你快松手,入土为安吧!''

小翠纹丝不动,两只手缓缓合紧。而陈大光虽有一身的武艺,但如今被人扼住了喉咙,自然也是施展不出。无心情急之下,不得不把手指送到牙关狠狠咬下。然而未等他咬出自己的血,陈大光挣扎着拔出了腰间手枪,对着身后就扣动了扳机。无心见势不妙,当即向后一窜。而在枪声响起的同时,小翠的头颅彻底爆炸,红的白的黄的从天而降,溅了陈大光一头一脸。颈上的双手立时松开了,陈大光一摸脸回了头,只见无头的尸首晃了一晃,随即竟然一路后退着疾行,伶伶俐俐的越过了后窗户。

陈大光不敢细想对自己头脸上的液体,作呕之下怒发冲冠,拎着枪就跳窗户追上去了,一路且追且骂:``操你个贼娘的!老子又没日过你的骚×,你和老子做什么对?''

无心留在房内,反正手指上已经见了血,索性蹲下来先在赵广和的额头上画了一道血符镇住魂魄。然后他起了身,打算跳出窗户去追陈大光。可是未等他动作,身后忽然起了轻轻的一声``嗤啦''。

他向后转过了身。空屋子有岁数了,门旁还有一扇老式的木格子窗,没镶玻璃,只糊了一层报纸。报纸刚刚被人捅了个窟窿,窟窿后面是民兵队长的一只眼睛。

眼睛和无心对视了一阵,随即向后移开了。取代眼睛的,是一根漆黑的枪管。

未等枪口射出子弹,无心像个鬼影似的,一瞬间就窜出了后窗户。

在黑水洼一片大乱之时,黑水洼附近的一座高山上,小丁猫席地而坐,正在摆弄一张白纸。顾基挎着手枪,顶天立地的站在一旁。他的亲人,算起来都是死在了小丁猫手里,而他自己无依无靠,只有小丁猫还肯要他。他一个人是活不下去的,他离不开小丁猫。

对于小丁猫,他既然没法去往死里恨,只好走上另一个极端,往死里爱。忠心耿耿的站在小丁猫身边,他看小丁猫用手指在纸上画了个阴阳鱼。手指没颜色,画了等于没画。盘腿坐稳当了,他把白纸放在面前的草地上。双手捧着脑袋弯下了腰,他闭上眼睛静默许久。四野无风,白纸却是自动的转了个圈。

一名青年轻轻走到了他的身后,弯腰说道:``丁同志,马婆子来了。''

小丁猫直起腰睁开眼睛:``带她过来。''

一个衣衫褴褛的婆子,拄着一根木棍走到了小丁猫面前。小丁猫抬头问她:``交待给你的事情,你做了吗?''

马婆子挤着一脸的皱纹,仿佛是很惶恐:``做了,做了。我这几天一直在大队部食堂帮工,你给我的纸符,我烧成了透透的灰,全混到菜里给他吃了。''

小丁猫又问:``那丫头的爹娘呢?''

马婆子答道:``他们两个人都信我,解放前他们家老爷子中过邪,就是请我给他禳治的。昨晚他们就都跑了,他们自己也是愿意,说姑娘没了,他们活着也没盼头。要是能给姑娘报了仇,他们死后下地狱也心甘。''

小丁猫点了点头:``好。如果我成功了,会让你彻底的翻身。你回去吧,没事不要露面。''

马婆子千恩万谢的走了。而小丁猫仰头做了个深呼吸,看到一只大猫头鹰蹲在树上,正在鬼头鬼脑的四处张望。

\chapter{一场乱战}

无心没想到陈大光这么能跑,野马似的顺风狂飙。好在他也是个腿脚利索的,一边跑一边还有气息高喊:``陈主任!陈大光,别跑了,你给我回来!''

陈大光气疯了,一言不发的追着前头尸首。小翠没了脑袋,然而依旧正面对着陈大光,两条腿倒退着飞快行走。陈大光步伐不停,回手甩出一枪,子弹贴着无心的头皮飞出去,正中了民兵队长的肩膀。步枪登时就脱手落地了,民兵队长手捂枪伤怔了一下,随即弯腰就要捡枪。然而未等他抬起头,无心的手指已经摁上了他的眉心。指尖用力试了一试,无心发现民兵队长目前还是个活人,但是仿佛魂魄受了损,已经彻底失了神志。对于这样一个不死不活的凶恶之徒,无心一时无计可施。民兵队长抬起了头,一条手臂都被鲜血浸透了,可是面无表情,单手还要举枪射击。无心趁着他力量有限,双手握住枪管用力一拽,随即转身继续去追陈大光。民兵队长呆呆的站在原地,似乎灵魂和枪一起被无心夺走了。

无心继续狂奔,再次追上陈大光时,他们已经到了村外的坟地。坟地阴气最重,是个鬼魂作怪的好场所。小翠停了脚步,揸着两只手忽然一挺身,粘稠恶臭的黑血开始顺着脖腔子往外涌。她的脑袋是被陈大光一枪崩碎了,参差的皮肉骨茬先是被黑血糊住,黑血越涌越多,并不流淌,而是颤巍巍的积成了一个人头大的黑血球,乍一看竟也像个脑袋似的。陈大光抬手又要开枪,可在扣动扳机之前,却被无心狠狠踹了一脚:``退后!她的血有毒!''

陈大光经过了一番长跑过后,理智渐渐恢复。无心让他退,他就退。飞快的退出了十米开外,他眼看小翠又要扑向自己了,下意识的又举起了手枪,同时听到无心喊道:``快打!''

一粒子弹射出去,正中对方的血头。只听``噗''的一声轻响,黑血当即四处飞溅。无心眼尖,忽见小翠的脖腔子里似乎被黑血顶出了一团物事,夜色浓重,也看不出是什么。趁机扑上去抓住小翠的衣襟,他伸手要去掏那东西。不料他刚出手,腿上忽然一紧,低头看时,就见一双干枯老手抓住了自己的小腿。顺着老手一路看过去,他看到了小翠的娘。

小翠的娘面目扭曲,一双老眼鼓凸出来,是个怒不可遏的疯狂模样。一口咬上无心的小腿,她合紧牙关,晃着脑袋使劲。无心忍住疼痛,趁着小翠尚且无力反抗,伸手从稀烂的腔子里一把掏出了那团东西。握着东西一松手,小翠仰面摔倒,再无反应。无心手上淋漓的黑血滴落到小翠娘的脸上,老太太猛一哆嗦,可是死不松口。身后骤然响起了陈大光的怒吼,无心只觉身边疾风一掠。地上``咔嚓''一声响,低头看时,陈大光已经一掌劈上了老太太的后脖颈。

无心任凭陈大光弯腰处置老太太,自己展开了手中的东西一瞧,发现它却是一张揉成团的纸符。仰面朝天的想了一想,他心里有了数,低头对陈大光说道:``陈主任,明天天亮之后,不管有没有雨,我们都得立刻离开这里。''

陈大光把断了颈骨的老太太拖开一扔,直起腰来问道:``不就是闹鬼吗?''

无心先是让他后退一步,与自己保持了距离,然后才把纸符亮给了他看:``闹鬼不假,鬼后面有人,也不假。黑水洼的事情没有完,我们人少势孤,留在这里有危险。''

陈大光伸着脑袋一看:``莫非\ldots{}\ldots{}还是联指的人在暗中捣鬼?''

无心在一处小水洼前蹲下了,用泥水洗了洗双手:``不知道。总之人比鬼危险,鬼么,尤其是新鬼,除了脾气大爱记仇之外,一般都是一根筋。''

陈大光双手叉腰一点头:``你这点儿家传的知识,倒是挺有用。''随后他扭头再一看地上的老太太,却见老太太死不瞑目,枯树皮似的老脸上星星点点,全是黑斑。黑斑黑的还不纯粹,像是墨水滴在了软纸上,一圈一圈的越渗越大越渗越淡,蔓延了个不可收拾。

``这怎么办?''他问无心:``老不死的变模样了!''

无心挽起裤管,去看小腿上的咬伤:``点火烧了她。''

陈大光摸出身上的火柴:``要是有汽油就好了。''

无心伸手向他要了火柴,然后把再无动静的小翠拖到了老太太身边。小翠自从被无心取出了堵在腔子里的纸符之后,黑血失控似的淌了满身。对着陈大光挥了挥手,他划燃火柴,扔到了小翠的身上。

火苗一遇黑血,登时腾起了一人多高的绿光。无心弯腰在身边的泥水坑里又洗了洗手,紧接着转身就要带陈大光走。陈大光莫名其妙:``怎么回事?那娘们儿死前喝煤油了?''

无心被他问了个啼笑皆非:``小翠死了三天,天气又这么热,她早腐烂了。你当她腔子里涌出的只是血吗?我告诉你,她的血和油\ldots{}\ldots{}''

陈大光一摆手:``别他妈说了,真够恶心的!''

话音未落,村里起了枪声。陈大光停了脚步,犹豫着不知道该不该继续走。无心却是着了急,撒腿就要往前跑。陈大光一把拽住了他:``且慢!先看看形势!''

无心如同泥鳅一般,摇头摆尾的滑出了他的掌握:``你等着吧,我去给你探消息。''

陈大光再要说话,已然晚了。无心一头冲进夜色,倏忽间就没了踪影。陈大光双手叉腰思索了片刻,末了往暗处一躲,一粒一粒的开始往弹匣里续子弹。不远处正烧着绿幽幽的一团烈火,忽然火中二人猛的坐起了身,吓得陈大光寒毛直竖。定睛再去细看,他松了一口气,发现不过是尸首被火烧缩了筋,并非又活了。

再说无心冲回了村中,在巷道里迎面正遇上了陈大光部下的几名精兵。几名精兵跑得张皇失措,见了无心,开口便问:``主任呢?妈的黑水洼要造反了,他们的民兵队长说我们是假冒的干部,要抓我们!''

无心从人群中看到了苏桃。一把将苏桃扯到自己身边,他急急忙忙的答道:``跟我走!陈主任现在很安全,正在村外等着我们!''

陈大光在精兵眼中,有着偶像的地位。一听偶像安然无恙,精兵们立刻昂扬了斗志,跟着无心撒丫子狂奔。好容易跑出了巷子口,众人心有灵犀一般,忽然一起停了脚步。仰起头望向夜空,他们就见一个鸟大的玩意儿劈空而来,黑黢黢的正要从天而降。骤然发出一声呐喊,众人像是马蜂见了火一样,无须号令,``嗡''的一声四散而逃。未等他们跑远,一枚炮弹斜斜的落到巷子口,轰然一声巨响,炸了个天翻地覆。

无心护着苏桃伏倒在地,约莫着爆炸已经发生过了,他拉起苏桃起身又逃。苏桃吓到极致,反而麻木不仁的挺镇定。单手把书包捂到胸前,她不看前不看后,迈开两条腿一味的跑。待到无心猛然收住脚步之时,她一头撞到无心身上,气喘吁吁的向前一瞧,她发现自己竟然冲进了一片坟地,不远处还烧着一堆火,火苗很平稳的泛着绿光。

``陈大光!''无心领着苏桃,四处寻找陈大光:``你还在吗?''

坟地内外并无回答,远方村中的喊杀声音却是越来越响越来越近了。无心无处可逃,只好带着苏桃穿过坟地,往山上的草木林中躲藏。白天刚下了一天的雨,平地空气畅通,泥土已经干爽了大半,山中道路沟沟坎坎,则是依旧泥泞。无心弯腰脱了脚上鞋袜交给苏桃,然后高高的挽起裤管,扛起苏桃就往黑暗处的山地里走。村中又腾起了一团火光,不知是炮弹爆炸,还是村中民兵胡作非为。

无心进了林子,把苏桃放在了一截老树桩上站好。自己走到一旁甩了甩脚上的泥巴,他扭头对着苏桃苦笑:``林子里太黑了,怕不怕?''

苏桃答道:``我不怕。他们爱打仗就打去,可是我们怎么办呢?''

话音落下,她不由自主的打了个寒颤。白琉璃从她身后缓缓浮现,越升越高,回村里看人打仗去了。

无心认为白琉璃趣味极低,已经不可救药;当着苏桃的面又不好和他对话,只好视而不见的由着他去。而他在脱离蛇身的一瞬间,阴邪之气尽露,免不了会让苏桃有所知觉。苏桃以为雨后风凉,冷过一阵倒也罢了;另有一位专爱鬼魂的精灵却是闻气而来。未等它在树枝上停稳,无心俯身捡起一块石头,一下子就把它打下来了。三步两步的赶上去,无心眼疾手快的把一只大猫头鹰摁在了泥水里:``好家伙,又是你!你总跟着我干什么?''

苏桃吓了一跳,把眼睛睁到极致:``无心,谁来了?''

无心头也不回的瞪着猫头鹰:``没事,是只动物,一会儿给你看。''

猫头鹰仰面朝天的缩了爪子,也把眼睛睁到了极致。无心的两只手都压在了它的肚皮上,它挣扎不起,情急之下只好使出绝技。两只翅膀向前一拥遮住尖嘴,它露出两只圆溜溜的乌黑眼睛,闪着泪光望向无心。

无心见的妖精多了,根本不受它的迷惑:``少装!你说你到底存了什么心?不说实话我吃了你!''

猫头鹰道行有限,不会说人话。把个脑袋微微转动向上一扭,它展开一只翅膀,对无心换了个造型。

无心目露凶光,捏住了它的尖嘴:``好你个妖精,不说实话就掰了你的嘴!''

猫头鹰吓坏了,修行了上百年,第一次遇到无心这么凶恶的对头。泪水在眼眶里滴溜溜的打了转,它见无心始终是横眉怒目,只好换了一招。张开翅膀眼睛一眯,对着无心做了个笑脸。

无心嗅着它身上清淡的妖气,瞧出它是个小小的妖精,不比普通的猫头鹰高明多少,想要兴风作浪,至少也还需要百十来年的光阴。低下头一鼻子拱上猫头鹰的羽毛,他始终怀疑对方是有意尾随自己,所以想要嗅上一嗅,看它身上是否带有鬼气人气。猫头鹰被他拱得很不好意思,当即抬起翅膀把脸蒙上了。

苏桃站在后方的矮树桩上,先以为是无心抓了只大兔子。不料无心自言自语一阵之后,忽然把脸埋到兔子肚皮上去了。她提了裤管正要下去看个分明,无心已然起身转向了她,一只手拎着猫头鹰的两只膀子。

``桃桃,别往下走,太脏。''他一边说一边把猫头鹰拎到苏桃面前:``其实也没什么稀奇的,你瞧瞧。''

苏桃低头一看,就见一只大猫头鹰正对自己,两只圆眼睛大大的,一只尖嘴巴小小的,不禁失笑:``好大的夜猫子啊!你还和它说话?它是只鸟儿,听得懂吗?''

无心听她没有追究``妖精''二字,倒是十分侥幸:``应该能听得懂。动物活久了都有灵性。你看它比一般的小猫还大,肯定也是个有岁数的。''

苏桃继续和猫头鹰对视,先是觉得它可怜可爱,一双大眼睛水汪汪的,而且不像一般的夜猫子一样,会在夜里目露贼光。看着看着,她直着眼睛开始一动不动。无心旁观良久,发现怪不得这猫头鹰百般造作,原来是通晓了迷魂术。一巴掌扇到猫头鹰的后脑勺上,他开口说道:``听说吃了猫头鹰的肉,一辈子不得头晕病。''

猫头鹰张了张嘴,好像要说人话,可是最终只发出了一声难听的叫。苏桃在它的叫声中眨了眨眼睛,显然也有些发懵,嘀嘀咕咕的自言自语:``看夜猫子都看走神了。''

无心扯了几根柔软的藤条,让苏桃将其编成三股辫子。用藤条辫子把猫头鹰捆好了吊在树枝上,他认定它是个危险分子,在自己脱险之前,不许它乱飞乱走。

与此同时,埋伏在山中的联指残军,开始往山下冲锋了。在这激动人心的反攻时刻,小丁猫百年一遇的坏了肚子,不但不能亲临前线,甚至在后方也站不起身,只能蹲在草丛里一泻千里。顾基背着一把半自动步枪,不住的接到前方线报,大声的读给小丁猫听。小丁猫奄奄一息的叼着一根烟卷,气若游丝的做出指示:``先把黑水洼的民兵小队控制住\ldots{}\ldots{}嗯\ldots{}\ldots{}不要让陈大光逃脱\ldots{}\ldots{}啊\ldots{}\ldots{}必要的话,可以先给村民一个下马威\ldots{}\ldots{}呜\ldots{}\ldots{}我要死了。''

顾基迟疑着回头问他:``丁同志,你\ldots{}\ldots{}你还要卫生纸吗?''

小丁猫带着哭腔答道:``我什么都不要,你快去传令吧。今晚陈大光不死,明天我们就得死了!''话音落下,他自己又叹息了一声:``哎呀妈啊\ldots{}\ldots{}''

\chapter{人各有计}

午夜时分,无心抱着肩膀蹲在伸手不见五指的林子里。苏桃盘腿坐在树桩上,困得不住点头。远方山下村中偶尔还会响起零星枪声,战况到底如何,无心想象不出。

到了后半夜时,苏桃抱着小腿埋头睡了。白琉璃像轮大月亮似的飘然而归,周身笼罩一层柔和白光。端端的悬在无心面前,他兴高采烈的说道:``我看到了一个熟人!''

无心没言语,单是抬眼看他。

于是他继续说道:``我看到了那个戴眼镜的小男孩。''

无心立刻明白了,他说的是小丁猫。

白琉璃快活的拍了拍膝盖,又大声说道:``小男孩下山的时候摔倒了,一路滚到了泥水坑里。进村之后,他跳到食堂的大锅里洗了个澡。''

悠然神往的微微仰起头,白琉璃回想起了食堂情景。大锅下面还生着火,小丁猫蹲在锅里,因为没戴眼镜,所以把两只眼睛眯得又细又长,像一只目光迷离的白条鸡。

无心捡起一块石头,在地上一笔一划的写字,让白琉璃去找陈大光。白琉璃兴致高昂,当即同意。可在临行之前,他忽然发现了吊在树枝上的大猫头鹰。围着猫头鹰转了一圈,他没看出好来;而猫头鹰睁着两只探照灯似的大眼睛,吸了一鼻子非常浓郁的阴气,觉察出周围有强大的鬼魂出现了,不过它毕竟还是肉眼凡胎,如果鬼魂不肯主动现身,它和人一样,也不能看出鬼魂的形象与影踪。

白琉璃生平没和妖精打过交道,猫头鹰既然不会做自我介绍,他看过就算,也没往心里去。一路飘向远方,他奉命去找陈大光。

不出三五分钟,白琉璃回了来,欣欣然的对着无心一招手。无心看他一脸得意,显然是方才看人打仗看高兴了,浑然不知当下的危险。叫醒苏桃背到背上,他双手向后托住她的大腿,一张嘴则是叼着藤条辫子,辫子下面自然还是五花大绑的猫头鹰。随着白琉璃穿过短短一片林地,他果然看到了陈大光等人。

陈大光身边只剩了三名手下,一个个泥水淋漓没个人样,全都各找高地盘踞了。忽见无心拖泥带水的走了来,陈大光登时来了精神:``你没死啊?''

无心腾不出嘴来回答。找块山石把苏桃放下了,他双手抱住大猫头鹰:``你跑哪儿去了?不是说好了在坟地等我吗?''

陈大光摇头叹息:``别提了,你前脚一走,后脚就来了个糟老头子,拿锥子往我眼睛里扎!推也推不开打也打不死,我和他撕扯了半天,等到把糟老头子处理完了,我再回原地一看,就见了他们三个——你怀里抱了个什么东西?把谁家孩子偷出来了?''

无心搂着大猫头鹰,感觉对方沉甸甸的还挺温暖:``我又不吃人,偷孩子有什么用?这是一只大夜猫子,我刚才抓的。''

陈大光很有闲心的一乐:``我倒忘了你有飞檐走壁的本事。好这大夜猫子,比正经猫都大——我说无心,夜猫子肉能不能吃?''

无心手背有了痛感,是猫头鹰扭了头,在可怜巴巴的轻轻啄他。略一犹豫,他告诉陈大光:``肯定不好吃。''

陈大光其实是饿了,他个子大,力气和饭量都远远超出凡人。抬手摸了摸脑袋,他叹了口气:``这他娘的,我们还被困在山上了。谁还记得回喇嘛山的路?老在林子里蹲着可不行,在文县地界敢对咱们红总下死手的,除了联指没别人!狗日的小丁猫,肯定是他,他越了狱,一直没消息,原来是在这儿等着我呢!妈的,他可别落在我手里。落到我手里了,我先让狗日了他,日完再把他剁碎了喂狗!''

此言一出,愁眉苦脸的部下们忍不住笑了。而陈大光随即仰头望天:``别他娘的傻笑了,谁会看星星辨方向?当初咱们是坐马车走山路来的,现在让我找山路,我肯定是找不着。把方向定准了,咱们直接翻山吧!''

陈大光以及他的三名小兵,全是县城里长大的孩子,仰着脑袋看了半天,连北斗七星都没找到。还是无心又把苏桃背了起来:``你们要是信得过我,就跟我走。喇嘛山在黑水洼的北边,我们朝北走!''

陈大光别无选择,只能信他。无心背着苏桃领头走,因为还是怀疑猫头鹰别有用心,所以不肯放它。把藤条辫子重新整理了一番,他把猫头鹰挂在了脖子上。大猫头鹰随着他的步伐晃晃荡荡,很认命的没有乱动。

无心成了陈大光的向导,白琉璃则是成了无心的向导。陈大光等人越走越冷,就感觉周围阴森森的,从心里往外冒凉气。苏桃趴在无心的后背上,也打了几个喷嚏。

兴许是抄了近道的缘故,无心一行人居然未到天亮便出了山。众人心中恐慌,一个个走得十分有劲。及至在微薄的晨曦中进入喇嘛山生产队时,他们容光焕发的红着脸,倒像是在黑水洼遇到了美事。气喘吁吁的进了黑水洼大队部,陈大光打了赤脚,因为脚上的胶鞋沾满泥巴,已经足有好几斤重。

喇嘛山的大队长慌里慌张迎接了他们,由于并不知道黑水洼发生了内乱,故而对于陈主任的形象很觉惊讶。随即朱建红也蓬着头发赶来了:``哟?你们怎么了?''

陈大光一夜没睡,全凭一股子战斗热情支撑了身心:``有敌人埋伏在黑水洼附近的山里,趁夜向村中开炮,我怀疑是联指串通了黑水洼里的反革命特务,要对县革委会和黑水洼人民反攻倒算。''

朱建红大吃一惊:``联指?''

陈大光迎着窗口阳光,缓缓一举斗大的拳头:``趁着联指在黑水洼还没站稳脚跟,我们必须马上行动,给予敌人最沉重的一击!''

朱建红看了他高瞻远瞩的造型,登时爱得意乱情迷,很酥软的答道:``是。''

在陈大光进行战略部署之时,无心站在大队部的后院,给猫头鹰松了绑。拍了拍猫头鹰的后脑勺,他低声说道:``现在不怕你去通风报信了,你走吧,我不吃你。''

然后他托着猫头鹰向上一举,猫头鹰立刻展开两只大翅膀,头也不回的逃了。

苏桃端着一只大饭盒,走到了他的身边:``吃饭了。''

无心接过饭盒,见里面满满盛了饭菜:``你吃了吗?''

苏桃答道:``我吃了。你坐下,我给你捶捶腿。''

无心已经用井水冲去了腿脚的泥巴。趿拉着球鞋蹲在青砖地上,他托着饭盒往嘴里扒饭:``不用,我不累。''

苏桃用毛巾给他擦了擦短头发上的水珠,想他背着自己跑了一夜。

无心饿极了,吃得狼吞虎咽。仰起头用勺子把最后一口饭菜刮进嘴里,他鼓着腮帮子正在大嚼,不料前院忽然起了喧哗。和苏桃对视了一眼,他把饭盒盖子一扣,拉起苏桃就跑向了大队部前门。

前门停着三辆已经发动了的大卡车,陈大光换了一身整洁军装,正在吆五喝六的进行指挥。忽然见了无心,他当即把手一挥:``上车,撤退!''

无心莫名其妙:``怎么了?''

陈大光高声答道:``联指的兵下山了,没有战斗力的都先撤去后方!''

无心当即扯着苏桃跳上卡车。一辆卡车装满了,立刻驶向村外的盘山土路。从喇嘛山生产队到妃子岭公社,路途虽然遥远,但因道路一直平坦通畅,所以反倒好走。战斗号角突然吹响,县里干部和公社干部都是猝不及防。大队部的广播员开始广播,召集村中的外来干部立刻到大队部集合。第一辆卡车都开出村了,第二辆卡车还没上满人。

陈大光没在山里打过仗,所以一边部署民兵防御,一边也存了随时撤退的心思,只是不对人说。与此同时,小丁猫坐在黑水洼的大队部里,却是美滋滋的别有一番心思。

总在山里混,真让他吃不消。白皙的手臂从半袖衬衫中露出来,因为半夜在锅里洗过了澡,所以他自己摸着自己,摸得满心怜惜,自认是个皮光肉滑的处男,将来不知会便宜了哪家的黄花大姑娘。

杜敢闯从北京发回的密信,摊开在面前的木桌子上。自从得知了马秀红的死讯,杜敢闯对他的控制欲明显增强了许多。新的秘书是她从保定的联指总部中挑选出来的,名叫丁小甜,名不副实,是个五大三粗的女杰,根本不甜。

杜敢闯在信里告诉他,联指翻身的日子已经近在眼前。红总身后的保护伞如今在中央已经说不上话,而联指到底是左是右,有几位首长已经明确表了态度。所以小丁猫现在可以着手准备反攻,至少先占住一块根据地,进可攻退可守。

小丁猫把信反复读了三遍,读得心中晴空万里。房门一开,顾基带着风走了进来。在小丁猫身边弯下腰,他虔诚而又谨慎的说道:``丁同志,最新消息,红总果然开始分批撤退了。''

小丁猫微微一笑,把手从衬衫下面伸进去,抚摸着自己的条条肋骨——风餐露宿,日理万机,都他娘的瘦了;肚皮也是瘪到了家,因为里面一点存货都没有了,凭着昨夜的泻法,能把肠子保住就算不错。

``我们的人半夜出发,现在应该也到达地点了吧?''他问顾基。

顾基的头脑一片空白,所以特地想了一想之后,才认真答道:``应该是早到了。''

小丁猫摘下眼镜,对着镜片呵了一口热气,然后扯起衬衫一角擦了擦:``没想到陈大光跑得这么快,一座大山根本拦不住他。他要跑,我就让他跑,看他到底能够跑出多远。''

顾基点了点头,又``嗯''了一声。

小丁猫又问:``民兵队长和马婆子,都解决了吗?''

顾基继续点头:``夜里都处决了。''

小丁猫若有所思的没言语。民兵队长和马婆子都死得冤枉,民兵队长无意中吃了马婆子下给他的符灰,宛如一道符贴进了五脏六腑。小翠的阴气把他一冲,符中的魂魄立时有所感应,突破纸符占据了他的躯壳。至于马婆子——马婆子身为村中的半仙,只不过是生活艰难,所以才受了他的收买,替他炮制了小翠的尸首,也替他蛊惑煽动了小翠的父母。

``战争是流血的政治,有奋斗就会有牺牲。''他轻描淡写的为死者作了总结:``把他们火化了吧!''

顾基答应一声,转身就走,临出门时一弯腰,因为个子太高,门框太低。小丁猫盯着他的背影叹了口气,心想自己身边一帮牛头马面,顾基居然就算是其中的美男子了。无心倒是有点邪运,要什么没什么,却能勾搭上苏桃。有日子没见苏桃了,不知道她有没有继续发育。如果自己将来有了大出息,苏桃倒也够格做一名首长夫人。

从苏桃又联想到了无心,小丁猫忽然抬手一摁心口,无声的说道:``老岳,你别这样。那小子不值得让你念念不忘,你乖乖睡吧,别让我痛苦。你无论怎么急,我也不能娶了无心,我是个男人嘛,对不对?''

胸中一阵莫名的苦楚愤怒渐渐淡化了,小丁猫松了一口气,总算是把岳绮罗又压了下去。

小丁猫怀着鬼胎,指挥部下队伍攻打喇嘛山。喇嘛山生产队的卡车全开走了,东倒西歪的走在盘山土路上。土路受了大雨冲刷,不但坑坑洼洼,而且带着斜坡,十分危险。三辆卡车起初开得还算顺利,可是刚刚走过一座大山,路况就急剧恶化了。

卡车之间距离极远,因为出发时间不一,后车又不敢放开速度追逐前车。无心所在的卡车开着开着,忽然就听身后一声巨响。车上众人扭头看时,只见先前走过的一段路上土石成堆,竟是路侧山体无端起了爆炸。

有人发了慌:``是炮弹吗?''

反驳立刻来了:``黑水洼的炮弹能飞到这里来?''

话音未落,前方又一声巨响,卡车一个急刹,差一点就受了前方山体爆炸的波及。

干部们吓坏了,心惊肉跳的下了卡车,又搬又刨的清理路上土石。好容易腾出道路了,卡车重新发动,走出没多远,前方山体又爆炸了。

这回谁都看清楚了,分明是有人在山壁中埋了炸药。可是看清楚了也没有用,后有追兵,分秒都听不得。司机赌了性命把卡车往前开,开着开着``轰隆''一声,山又炸了。

满车的人都傻了眼,硬着头皮下车开路,把脑袋都系在了裤腰带上。如此忙了整整一天,距离妃子岭公社还有一座山没有走。乘客们无吃少喝,骂着娘下了车。在苍茫的暮色中,他们决定按照原路向后走,去和后方两辆卡车中的同志会合。接下来是怎么办,大家总得商量个主意出来。

无心随着人流前行,走着走着,耳边忽然响起了白琉璃的声音:``不要去。''

无心当即神情痛苦的一停步,有人见了问道:``你怎么了?''

无心倒吸了一口气,扶着苏桃退到路边,慢慢的要往下坐:``扭了脚,疼!''

\chapter{镇魂}

无心对于自己的前途,是彻底的一无所知。人群经过之后,他的脚落了地。苏桃早就看出他是装的,但是不明就里,当众也不敢问。现在看人没了,她小小的出了声:``无心,我们为什么不跟着他们走?''

无心扭头望着苏桃,忽然叹了口气。一个十几岁的小丫头,根本不该到穷乡僻壤里出生入死。

前方的人顺着山路拐了一个弯,拐完一个弯,还有一个弯。无心带着苏桃回了卡车,卡车内外空无一人,他顺着大开的车窗爬进驾驶室,摸出了司机偷藏的一包饼干。

饼干是用油纸包裹着的,看着好像肥皂,是方方正正的一大块。无心和苏桃飞快的把饼干吃了个一干二净,然后回了原路继续等待。天真黑了,夜风凉飕飕的吹,始终不见人归。无心等不住了,打开书包说道:``娘子,你陪着桃桃,我去瞧一眼。''

白琉璃一吐信子,表示同意。

无心沿着土路走,拐了一个弯之后,他看到了半空中悬着一只鬼影。鬼影正在缓缓的淡化,魂魄宛如微弱的流星,从他身上逸散而出。他认出了鬼影的身份,正是打头卡车的司机。

一阵风掠地而来,夹杂着甜腥的鲜血气。无心继续慢慢的走,走着走着,他在一处弯路口停住了脚步。探出脑袋向旁望去,他看到了一条空空荡荡的崎岖路。之所以崎岖,是因为路面受了爆炸的影响。几只无精打采的鬼魂飘在半空中,一个个的死相都很不好看,大概也是受了爆炸的连累。

无心并不怕鬼魂,尤其是新鬼力量微弱,眼看着正在魂飞魄散。轻轻的迈步拐了弯,他继续往前走。末了停在土路中央的大坑前,道路一边的山壁已经崩溃了,另一边是向下的陡坡,陡坡足有十几丈深,坡上生着不成气候的枯瘦草木。一辆大卡车零零碎碎的滚在坡底,后斗的布蓬还存留着,依稀可见布蓬下面有人。

卡车里的人,遇难是正常的,可是前来寻找他们的人,不该一起失了影踪。无心蹲在路边伸下一条腿,蹬住陡坡试了试,感觉还不算滑,便连跑带溜的一路向下,直奔卡车而去。

越是往下,血腥气越重。无心停在卡车之前,刚刚直起了身,不料忽有一阵凉风斜斜的拂过了他的鼻尖。卡车的残破布蓬被风掀起了一角,一只凝满干涸血迹的手直挺挺的伸向了他。

人死久了,已经变硬。无心盯着面前的手,忽然发现这手有点古怪——手掌手腕都算干净,泛黑的浓血是从手指尖开始往下蔓延的。若说是手指尖受了伤,可指甲全都完好,完全没有伤口。

无心不动声色的转身走向卡车驾驶室,卡车侧躺在地上,驾驶室的窗口向上成了天窗。司机仿佛在临死前曾经试着往外爬,上半身都伸到车窗外了,两条腿却是骨断筋折的卡在了座位下方。伸长双手趴在车门上,他面孔向下,倒是还算干净。

无心知道司机都不是空手的人,身边必定藏着武器。爬上车门站稳了,他抓着后衣领把司机向上一拎,司机僵着双臂顺势直起了身,一个脑袋依然低着。利落的把司机拽出车门推向地面,他自己跳入驾驶室内,因为近些天来随着卡车东走西逛,见多识广,所以他立刻就从座位下面抽出了一把带着皮鞘的砍刀。

从破碎车窗中站起了身,他飞身一跃跳下了地。正要迈步走向卡车后斗,他脚步一顿,忽然感觉身后有了异样的动静。一把除下刀上的皮鞘,他将刀刃缓缓的划过手掌。忽然向后一转身,他看到了司机的脸。

司机的脸已经被碎玻璃扎成面目全非,咽喉也裂开了一条黑洞洞的伤口。踉跄着起身扑向无心,他微微张开了嘴,口中隐约可见一角白色,正是揉成了一团的纸符。无心先是不动,及至他扑得近了,无心横着挥出一刀,寒光过处,人头落地。身体与纸符断了联系,立刻僵直着向后仰倒,不再动弹。

无心转身走向卡车后斗。静静的站到了车尾,他提着砍刀向内望,就见车中人叠着人,仿佛还在争先恐后的向外冲,一个个全大张着双手,做着高声疾呼的表情,眼珠子似乎将要瞪出眼眶,拉长了的扭曲面孔上,一张嘴全是异常的大。一阵刺骨的阴风吹上了无心的脊背,半空中响起了刺耳的猫头鹰叫。

无心向天猛一抬头,看到了大猫头鹰的黑影。而大猫头鹰眼神不比他差,低头和他对视一眼,大猫头鹰把嘴一闭,当即沿着原路掉头飞了。

刀尖挑开后斗的布蓬,无心向车尾靠近了一步。车中忽然起了窸窸窣窣的声响,仿佛是有人在用指甲抓挠后斗的铁板。眼角余光扫过最近的一排尸首,他忽然狞笑了一下,因为发现它们无一例外,指尖全带着血。月色之下,它们的嘴唇也是暗红——干血的颜色!

单手举起砍刀,刀刃反射了月亮的光芒。银白的光一闪而逝,带着若有若无的一声``嚓''。一只人头滚落了,整齐的腔子口里,还塞着一团染了血的纸符。

无心伸手取了纸符,向后一扔。随即抓了另一只头颅的长发,他挥刀再砍。小丁猫的战术实在是让他反感至极。很好的生命,年纪轻轻,无端的就被他毁灭了;很好的肉体,年纪轻轻,无端的就被他利用了。无心没有时间与精力再给他们留全尸,因为一个小翠已经让人吃不消,一车的小翠一起上阵,更不是他单枪匹马可以对付的。

一具躯体缓缓的爬向了车尾,在无心力不能及的范围内四脚着地,走兽一般的瞄准了他。忽然纵身一跃扑向无心,他亮出了一口血淋淋的牙齿。而无心猛一侧身,避开了他第一次的攻击。等到他落了地,无心不等他起身,直接一刀剁向了他的脖子。脑袋骨碌碌的顺着斜坡滚出老远,身体趴在草丛中,安静了。

无心虽然知道借尸还魂的东西都伶俐不到哪里去,不过既然攻击已经开始,行尸们必定都会渐次苏生。单凭体力来论,自己也不是它们的对手。忽然灵机一动,他一扯布蓬盖住后斗,随即绕到卡车车顶一侧。划破手指挤出了鲜血,他忍痛在布蓬上画起了符咒。符咒是专用来镇压一切邪祟的,他平时很少使用,笔画生疏。布蓬下面起起伏伏,显然他的符咒有点灵验,可是法力有限,未必能够持久。一道符画完了,他抓紧时间跑去车头,想要从卡车油箱里弄些汽油。

费了偌大的力气,他用一根长长的胶皮管子,把汽油引去了后方的布蓬上。他没开过卡车,但是在几十年前,赛维的日子还好过时,曾经买过一辆小汽车让他开。如今的卡车和当时的汽车不甚相同,不过构造大同小异。

一根火柴扔上布蓬,火焰腾空而起。无心听到了真正的鬼哭,吱吱呀呀,宛如鼠类的惨叫。拎起砍刀继续向坡下走去,他得找到余下的尸首。小丁猫打得好算盘——干部们半路失踪,必定会引人前来寻找,来一个,死一个;来两个,死一双。陈大光如果在卡车上,自然死得利索;如果晚走一步不在卡车上,只要他夜里经过山路,就必定逃不过行尸们的拦截。而陈大光除非有飞机可坐,否则必定要走山路。山路被炸成了一团糟,陈大光怎么走,都要从白天走到夜里。一到夜里,人就不是鬼的对手了。

大猫头鹰又来了,显然是有所图谋。无心不再理它,而是跟着它走。沿着土坡又走了一段路,他看到了与自己同车的伙伴们。

伙伴们死得很惨,全被人抓烂了面孔和咽喉。大猫头鹰在他头顶犹犹豫豫的盘旋着,想要吃点人肉,又怕他不允许。猫头鹰爱好和平,觅食之时只抓小田鼠、小兔子、以及小鸡小蛇;和它身材相仿佛的动物,它是一概的不招惹。小动物不足以让它饱腹,于是它此刻留恋不走,想要饱啖一顿人肉。

无心弯腰检查了几人的口腔咽喉,没有发现纸符,可见他们的确是死得彻底。直起身继续向前走去,他记得还应该有一辆大卡车殿后。

在三里地外,无心又放了一把火。

凌晨时分,他疲惫不堪的回到了苏桃面前,苏桃要去看他,他却是连连摆手,说自己身上太脏。又提起其余的人,他告诉苏桃:``都死了。''

苏桃``哦''了一声。

无心四仰八叉的躺在土路上,侧过脸看她:``你怕不怕?''

苏桃检查了内心情绪,发现自己不怎么怕。几个月前她见了人都怕得要死,如今像是麻木了,什么都不怕了。

无心仰脸又去看了夜空中的星月,感觉自己其实也是个没用的货,有力气卖给陈大光,目的是希求对方庇护自己和苏桃。自己没本事,保护不了苏桃,可怜苏桃还当自己是天下唯一的亲人。

一身的血点子在慢慢的风干,他向旁伸出一只手,抓住了苏桃的脚踝。苏桃一动不动的任他抓着,心里空荡荡,什么也没想。

翌日上午,陈大光和朱建红双双出现了。

他们是骑马走的,前半夜就出发了,没经山路,穿了林子,往死里走也只走出了这般的速度。他们在喇嘛山生产队里就听说了山路上发生了大爆炸;及至走到林子中了,他们隔着远远的距离,又看到了山下隐隐的火光。

在无心身边一勒缰绳,陈大光居高临下的质问:``你怎么没死?''

无心依靠山壁坐着,脸上颜色并不好看:``我死了,你怎么活?''

陈大光一听,倒像他死了自己就要守寡一般,不禁鼻孔出气:``除了你们两个,再没别人了?''

无心点了点头:``嗯,没别人了。''

陈大光一瞪眼睛:``到底是怎么回事?''

无心扶着苏桃起了身:``说来话长。有水吗?''

陈大光没有水,而是把无心和苏桃分别拽到了马上。马蹄子呱嗒呱嗒的敲击路面,他们飞快的继续逃了。

埋在山中的炸弹也许是定时炸弹,昨天依次炸过了,今天再无存货。一路颠颠簸簸的到了妃子岭公社,陈大光恶狠狠的苦笑,心想自己这一趟堪称全军覆没——此仇不报非君子,他饶不了小丁猫。

一封电报发出去,全县的武装民兵全集合到了妃子岭。陈大光从无心口中得知了小丁猫的阴谋诡计,又想起了整整三卡车的人命,不禁怒发冲冠。亲自率兵上了阵,他拉着大炮直奔喇嘛山而去。

无心和苏桃却是不再往前线跑了,他们得了陈大光的许可,两人回文县去了。

陈大光翻山越岭,一进喇嘛山就发现情况不对。再接再厉的杀入黑水洼,形势越发的糟糕了——联指的人马居然已经撤出了黑水洼。

没等他调转人马撤出山区,后方情报十万火急的送到了他面前:联指被中央划为左派革命组织,如今已在保定和文县齐头并进,各自聚集了几千人马。保定比较远,姑且不提;只说文县外围,已经被联指的队伍占据了。

陈大光被人抄了大本营,带着一票人马陷在了山中。而文县内外僵持不下,无心和苏桃躲在革委会的收发室里,因为食堂不再正经开火,革委会也面临瘫痪,所以他们只好自力更生,用砖头搭了个炉灶,架着饭盒煮粥吃,菜只有一道,是咸盐拌黄瓜。两人无处可跑,并且听说联指已经占了上风,就愁得唉声叹气,终日盼着陈大光力挽狂澜、早日归来。

\chapter{正义秘书}


小丁猫是毫无预兆的进了文县。进入文县之后他直奔县中最高权力机关——县革委会。

革委会食堂的大师傅上街买菜时被流弹打死了,所以食堂已经连着三天完全没开伙。革委会的办公区域也是空空荡荡,红总自顾不暇,人员全都集合去了城外战场,余下的老干部与军代表也是各有心肠。老干部都是被打倒又被拎起的人物,非常的识时务,一看情况不妙,立刻大姑娘似的全缩在家里大门不出二门不迈。军代表的行踪则是无人知晓,大概是看时局失控,也自行遁了。毕竟革命群众天下无敌,打死个把军人不是问题。

人都走了,唯有无心和苏桃无处可去。火车站是被封锁了的,想要离开文县,须得凭着两只脚硬走,并且很可能误入战场吃子弹。于是在小丁猫的敞篷吉普车停在革委会大门前时,无心和苏桃正是并肩站在收发室外吃小黄瓜。猝不及防的和小丁猫打了个照面,两人都偏于木然,嘴里喀嚓喀嚓的始终没断咀嚼。

小丁猫坐在敞篷吉普车里,车是够野的,还披着零零落落的草木伪装,小丁猫却是斯文,照例穿着雪白的短袖衬衫。一边手臂搭在车门上,香烟在他指间生出袅袅青烟。扭头望着无心和苏桃,他发现两人都瘦了,也都白了,嫩得像两根刚剥了皮的水葱,一高一矮一大一小,并肩站得很整齐。不知怎么搞的,小丁猫忽然感觉苏桃长的有点儿像无心,无心有一双黑到惊人的大眼珠子,而苏桃的黑眼珠仿佛也有扩大的趋势。无心的黑眼睛里藏着一点儿动物的光,人味不纯;苏桃望着前方,眼珠子也是黑得很深远,看谁都像是立足于千里之外。

风度很好的一挥手,小丁猫跳下吉普车:``开门!''

无心和苏桃一起回了收发室,片刻之后无心一个人走出来了,手里拎着一串钥匙。挑挑拣拣的选出一枚打开了革委会大门,他没问什么,因为事情是明摆着的,陈大光这个不争气的东西,当真被小丁猫挤下台了。

小丁猫洋洋得意的进了大院,身后跟着全副武装的顾基。对着无心一招手,他笑眯眯的说道:``来,你给我做个向导。''

无心手里还攥着半根黄瓜:``向导?革委会就这么几排房子,没有什么可看的。''

小丁猫单手插进裤兜里,顺势又吸了一口烟:``陈大光的办公室是哪一间?''

无心无可奈何,只好迈步走去。把小丁猫领到陈大光的办公室中。陈大光的办公室不怕人瞧,因为他学问有限,万事全从脑子里过,房里不存机密文件。小丁猫绕过写字台,一屁股坐上了陈大光的皮面椅子。很舒适的把两只脚架上写字台沿,他摇头摆尾的扭了几扭,又长长的叹出一口气:``听说陈大光特别好色?''

无心站在门口,忙里偷闲的咬了一口黄瓜:``不知道。''

小丁猫闭了眼睛,心中自觉很亏得慌。陈大光的淫威是很出名的,不用特意打听,种种逸事会自动的往人耳朵里灌。在这一点,他真是比不上陈大光,委屈自己了。

他自怜自怨,一时出了神。无心趁机看了顾基一眼。顾基明明察觉到了他的目光,但是心虚,不敢去面对他。

与此同时,革委会院门外又来了一辆吉普车。车门一开,下了一名五短三粗的女性。此女名叫丁小甜,正是小丁猫的新秘书。说起来她和小丁猫都姓丁,应该算是亲近的本家。不过小丁猫羞于与她为伍,对她一直尊重得很,半句玩笑都不肯开。丁小甜对于小丁猫的做派十分高看,认定他是一位品行高洁的年轻领袖。

丁小甜知道小丁猫比自己快了一步,所以如今也不犹豫,直接就往大院里走。经过收发室,她隔着玻璃向内扫了一眼,影影绰绰的看里面好像有人,便顺手推开了门,心想这是哪个红总余孽?

然后,她就看见了苏桃。

苏桃已经吃掉了小黄瓜。此刻正是无所事事的坐在小床边上。觅声抬头望向门口,她的辫子乱了,乱发之中,越发显得一张脸是异常白净,几乎给了丁小甜一个惊鸿一瞥的印象。而丁小甜万没想到小小的收发室里藏了个这么好看的小姑娘,讶异之余立刻起了警惕心:``你是干什么的?''

苏桃站起了身,垂头答道:``看大门的。''

丁小甜又问:``你多大了?''

苏桃盯着地面,感觉自己和前方的陌生人之间隔了一层薄膜:``十五岁。''

丁小甜还要继续盘问,不料无心拎着一大串钥匙,叮叮当当的走了回来。丁小甜目光如电,立刻转向了他:``你又是干什么的?''

无心莫名其妙的看着她:``我是看大门的。''

丁小甜审视着他:``一扇大门要两个人看吗?''

无心隐约明白了她的心意:``哦,我们两个拿一个人的工资粮票。''

丁小甜看了看无心,又扭头看了看房内的苏桃:``你俩是什么关系?''

无心晃着手里的钥匙串:``我俩是\ldots{}\ldots{}对象关系。''

丁小甜没想到无心会做出如此不堪入耳的回答。强作镇定的点了点头,她不再多说,向前走去寻找小丁猫了。

无论是谁占据了革委会,无心和苏桃的日子总还得过。小丁猫和丁小甜在办公室里开了个小会,及至散会之后他们出了来,发现院内薄烟缭绕,却是无心和苏桃在院子角落里拢了一堆火,正在烤麻雀吃。小丁猫走到苏桃身边,脚步顿了一顿,然而最终没停,还是继续前进了。

丁小甜和小丁猫上了一辆吉普车。在小丁猫身边坐稳了,她有感而发的说道:``丁同志,革委会看大门的人,是一对\ldots{}\ldots{}''

她不知道该用哪一个词来形容,但语气是温和的,因为自知站在正义的一方,最清白最纯洁,所以可以坦然面对一切罪恶:``女的才十五岁,男的我看怎么也得二十多了。他们公然在革委会大院里搞流氓活动,我认为影响很不好。''

小丁猫微微颔首:``你认为应该怎么办呢?''

丁小甜坚定的答道:``我认为男的该负主要责任。他年龄大,很可能是他别有用心,迷惑了女孩子。''

小丁猫慢条斯理的继续问:``那我们应该怎么处置他们呢?''

丁小甜想了一想,随即答道:``先把他们隔离开来,再对那个女孩子进行教育,让她迷途知返,重新做人。''

小丁猫轻飘飘的一拍巴掌:``丁小甜,你的思路很对,可以按照你的主意来办。''

因为丁小甜不会伺候人,所以小丁猫看她是可有可无。红总还在县外虎视眈眈的意图反扑,小丁猫忙着布置战线,便把县内事情全交给丁小甜处理。丁小甜带了两名战士回到革委会大院,直接让人把无心绑了。

苏桃登时红了眼睛,先是张开双臂挡在无心身前。及至她被战士一下子搡开了,她转到无心身后,死死的搂住了他的腰:``你们干什么?我们不是红总的人,你们凭什么抓我们?''

丁小甜平心静气的说道:``这位小妹妹,你不知道你已经被他拐上了岔路吗?''

苏桃厌恶的望着丁小甜:``没人拐我,我是自愿!''

丁小甜看了她的顽固态度,不禁惋惜的摇了摇头。对着两名战士做了个手势,她开口说道:``我看你年纪还小,所以对你采取柔和的手段,你不要敬酒不吃吃罚酒。''

两名战士开始押着无心往外走。无心临走前还想对苏桃耳语几句,可是战士们力大无穷,一阵风似的就把他拥出去了。无心暗暗叫苦,又恨白琉璃不务正业,大白天的也跑出去看人打仗。

及至战士把无心押向革委会的办公区了,丁小甜才转向苏桃。无心一走,苏桃就垂了眼皮,木雕泥塑似的在地上一站。丁小甜看出她是铁了心的要往邪路上走,便从身上的军用挎包里掏出一本红宝书,恭恭敬敬的摆在窗前放置信件报纸的小桌子上:``从今天开始,你就给我抄红宝书。我要让**思想的光芒照亮你头脑中的阴暗角落。一会儿我让人给你送纸笔,你等着吧。''

丁小甜对苏桃很感兴趣,甚至生出几分怜惜。派人把纸笔送到收发室了,她转而去审问无心。无心只说要见小丁猫,除此之外一言不发。丁小甜看他软硬不吃,心中十分恼怒。既然自己不能触及他的灵魂,只好退而求其次,触及他的皮肉。拎起皮带走到无心面前,她把对方的小白脸子打成了满脸花。无心没骨气,疼了就叫,叫得荡气回肠,如同春夜闹猫。两名战士忍不住嘻嘻笑,唯有丁小甜怒发冲冠,笑不出来。

她认为无心实在是太罪恶了,罪恶的苗子,就该连根铲除,不留余情!

丁小甜忙着拆散流氓情侣,拆得全神贯注,以至于忘了去干正事。及至到了翌日上午,小丁猫四处找不到她,只好亲自又来了一趟革委会。刚一进院门,就见苏桃坐在窗前桌后,正在低头写字。

小丁猫心中一动,又看四方无人。一推门进了收发室,他轻松的问道:``写什么呢?''

苏桃停了笔,站起身答道:``抄红宝书呢。''

小丁猫笑了:``很要求进步嘛!''

苏桃沉着脸:``丁小甜说我如果不抄书,她就不给无心饭吃。''

小丁猫向苏桃逼近了一步:``她还说什么了?''

苏桃面无表情的答道:``她还说无心是流氓,说我被流氓骗了。''

小丁猫继续逼近:``那你到底有没有被他骗呢?''

苏桃不看他,盯着地面答道:``他不是骗子。''

小丁猫溜了房内一眼,见窗户上方横着一根铁丝,挂了一块白布充当窗帘,白布如今被拨到了窗边。心中忽然躁动了,他上下又把苏桃打量了一番。忽然转身拉拢了窗帘,他一手捂住了苏桃的嘴,另一只手开始去解自己的裤腰带。苏桃先是一愣,紧接着吓得手舞足蹈,对他又打又踢。而他此时却是下了决心,正所谓择日不如撞日,索性今天就把生米煮成熟饭得了!


\chapter{受伤的小丁猫}

小丁猫拧着眉瞪着眼咬着牙,感觉自己胸前这两扇薄薄的排骨,快要被苏桃的拳头击塌了。

他没想到一个靠稀粥黄瓜麻雀度日的小丫头,竟有如此的神力。他的裤腰带在搏斗中已经解开了,苏桃的衣裳却还是森严壁垒,只有衬衫领口被他扯脱了一枚纽扣。小丁猫把她压在身下,极力的想要将她双手反剪着捆绑住。然而苏桃趴在床上猛然一撅屁股,当场把他拱到了床下。落地之后一个鲤鱼打挺,他在刹那间又翻上了床。气喘吁吁的怒道:``叫吧,叫吧!我看你能叫来哪位救兵!''

苏桃没有余力喊叫了,也知道小丁猫所言非虚,世上除了无心之外,当真是再没有人肯救自己。一张小床被两人压迫得吱嘎作响。仰面朝天的看小丁猫压过来了,她亮出一口整整齐齐的白牙齿——好像横亮了一把大刀似的,她对着小丁猫狠狠一抬头,一排好牙当场磕上了小丁猫的下巴。

小丁猫哀鸣一声,抬手去捂痛处。苏桃趁机拼命推他,小丁猫如落浪中,颠颠簸簸的上下乱摆,无论如何不能控制苏桃;想要去撕苏桃的裤子,新的确良又太结实。苏桃感到一只手就在自己的□乱抓,当即伸手下去,用指甲狠抠小丁猫的手背。小丁猫把手一躲,苏桃摸到了一条热烘烘硬邦邦的东西,这东西不是她的,就必然是小丁猫的,她连想都不想,对着它便挠了一把。在小丁猫的惨叫声中,她的手指触到了一丛乱毛。顺势合拢五指抓住了毛,她大叫一声狠命一揪。小丁猫惨叫未停,痛嚎又起。而苏桃抬手一瞧,就见手上抓了满满一把阴毛,毛发黑亮亮的打着卷儿,发根上还染着星星点点的鲜血。

小丁猫捂着□翻滚下床,痛苦之余意识到自己犯了一个根本性的大错误——裤子脱得太早了!

苏桃喘着粗气坐在床上,眼睛和脸都是红的。向下看到了小丁猫的半**,她这才知道大男人和小男娃不是一回事。她只见过光着屁股的小男孩,所以面对着龇牙咧嘴的小丁猫,她感到了一种无法忍受的厌恶和刺激。小丁猫双手捂着的东西红通通的,让她想起了扒了皮的小麻雀。

小丁猫在地上躺了半天,末了抹着眼泪爬起来了。

``好,好。''他是个整洁利落的人,一边对苏桃含泪发狠,一边有条不紊的一层一层提裤子。先用白色裤衩兜住了他□的挂彩秃鸟,再把白衬衣的下摆抻平。最后提起裤子,他把白衬衣平平整整的扎进了裤腰里:``苏桃,你敢这么对我!''

苏桃站在床边,弯腰捡起了领口掉落的纽扣。一侧的麻花辫子散了,她像个疯子似的,从乱发之中看人。

小丁猫想到自己连苏桃都打不过,几乎悲从中来:``好,好。从今以后我有话不和你说,我找无心说!''

苏桃攥着自己的纽扣,胸前两个正在发育的毛桃子全被小丁猫狠狠的揉搓过了,现在正痛得厉害。气喘吁吁的望着小丁猫,她绝望的想:``没活路了。''

慢慢的收回目光,她的呼吸和心跳一起紊乱。沉睡已久的头脑忽然苏醒了,她茫然的发问:``这是个什么世道?还讲理吗?还有理吗?''

``如果无心死了\ldots{}\ldots{}''她哑着嗓子开了口:``我也死去。''

然后她抬眼正视了小丁猫:``什么破世界,我才不稀罕!''

小丁猫狞笑了一下:``你说什么?你敢说现在的世界破?''

苏桃也冷笑了,冷意很足:``我说了,什么破世界!呸!破世界!''

她一强硬,小丁猫反倒有些手足无措。要说打,他没有余力;要说不打,未免又太轻饶了她。眼睁睁的看着苏桃,他不认为自己是□未遂,倒是感觉苏桃给脸不要脸,导致自己失了恋。

小丁猫给苏桃下了禁足令,又让人看守了收发室。白琉璃偶然回了来,先是发现苏桃一个人站在地上,直着眼睛发呆;他不明就里,飘出房去,在革委会大院的一件办公室里找到了无心。

和无心一相见,他就傻了眼:``啊!你怎么了?''

无心被人吊在了房梁上。抬眼一看白琉璃,他奄奄一息的怒道:``你还知道回来?我当你在战场上又死了一次呢!''

大中午的,烈日高悬,阳气极足。在这个阳盛阴衰的时候,白琉璃想要用念力截断悬挂无心的粗麻绳,可是试了又试,却是力不从心。无心摇了摇头,低声说道:``白琉璃,现在我不用你,等到了夜里你再来。桃桃呢?我一晚上没回去,她怎么样了?''

白琉璃如实答道:``她好像是刚起床,头发都没有梳。''

无心一闭眼睛:``你到她身边去吧,如果有人欺负她,你能保护就保护她,不能保护了,就马上来告诉我。''

白琉璃躲在了房中暗处:``夜里我救你走。''

无心把眼睛睁开了一半,很不信任的斜瞟着白琉璃。白琉璃的确是有本领,不过他的本领显然不大适合救人越狱。就算白琉璃能把他从空屋子里放出去,可接下来的路,还是得让他和苏桃自己走。整座县城都是联指的地盘里,无产阶级专政无处不在,即便他们跑去穷乡僻壤了,凭着他们来历不明的身份,照样会被村民抓起来扭送去大队部。

``白琉璃\ldots{}\ldots{}''他忽然小声开了口:``你想不想回家?''

白琉璃一扬头,蓝色的眼睛斜睨天花板:``我不想。''

无心知道他一贯不通情理,所以也不理他,自顾自的嘀咕:``实在没办法的话,我们带桃桃回大兴安岭吧!其实我真不愿意走这一步,在那地方住久了,桃桃非变成野人不可。''

白琉璃一言不发,因为他在外面混得很开心,看人武斗看了个不亦乐乎。

白琉璃回了一趟收发室,发现苏桃坐在窗前,正在写字。附回到了白蛇身上,他爬上了苏桃的大腿。把一个圆脑袋昂到了苏桃面前,他忽然发现对方含了满眼的泪。

苏桃对着白琉璃的黑豆眼睛,满心都是叫天天不应、叫地地不灵的凄惶。撅起嘴唇亲了亲白琉璃的脑袋,她哽咽着小声说道:``你要真是白娘子该多好啊!你是白娘子,水漫金山淹了他们。''

一滴泪水滴在了白琉璃的头顶上,白琉璃忽然通了一点人味。冰凉的绕上苏桃的脖子,他一吐信子,有心施法现形安慰安慰苏桃,可又怕把苏桃当场吓死。无可奈何之下,他只好用嘴巴触了触苏桃的耳垂。

如此混到了傍晚时分,丁小甜来了。

丁小甜听小丁猫说苏桃发了疯,坐在收发室里造谣生事,他亲自去看望她,结果被她挠了一顿。丁小甜看苏桃是相当的可人疼,并且因她年纪小,所以也必定是受了小白脸的蛊惑。思及至此,她不打算去找苏桃的晦气,倒是认定无心是个臭流氓,恨不能像杀臭虫似的一指头将他碾死。未等白琉璃前去救人,她先让手下的小将把无心押了出来。反革命流氓犯的大铁牌子往脖子上一挂,无心糊里糊涂的就混在一大队牛鬼蛇神之中,排队游街去了。他被吊了小半天,胳膊几乎脱臼,下午又挨了一顿揍。此刻苦不堪言的走在街上,他深深的低着头,因为唉声叹气的太过明显,又被身边的红卫兵抽了一皮带。

在无心游街的同时,小丁猫坐在临时下榻的招待所里,也是愁眉苦脸。嘴角叼着一根香烟,他脱了裤子,一手捏着自己的命根子,一手捏着个浸了酒精的棉球,忍痛擦拭□上的创伤。苏桃的爪子真是厉害,把他的小肚子挠破了好几处,左一道右一道鲜红的,一碰就疼,还没法向别人诉苦。他真有心不要苏桃了,可无论是杀了她还是放了她,都让他感觉可惜。咝咝哈哈的吸着凉气,他疼得挤眉弄眼,心想自己还是太纯洁、太稚嫩了。好在当时只解了裤子,万一脱成精光,非被苏桃挠成烂桃不可。

``我是个秀才。''他又暗暗的想:``秀才遇见兵,有理说不清。我可能是长得不如无心好看,但也差不许多,不至于他是苏桃的宝,我就是苏桃的草。看来问题全在苏桃身上,年幼无知,不识好歹。我先关着她,等养好了伤再和她算账!''

思及至此,他没了心事。拉开抽屉找出一把小剪子,他比了比□两边阴毛的长度,发现自己算是被苏桃用手揪成了阴阳头。嚓嚓嚓的修剪一番,他放下剪子提起内裤,抚平衬衫系好外裤。畏寒似的抱住肩膀,虽然面前没有敌人,但他还是下意识的保护了自己的肋骨。

无心死去活来的游了小半夜的街,末了回到革委会的空屋子里,倒头就睡。丁小甜见收发室里还亮着灯,就想去和苏桃谈一谈心。然而苏桃像个老蔫萝卜似的,也不软也不硬,丁小甜说,她就听;丁小甜不说了,她面无表情,也不出声。

丁小甜看了她这样子,莫名的很痛心。出了收发室,她斥退身边随从,独自在革委会大院里散步沉思。正是入神之时,眼角忽然掠过一道黑影,她扭头一瞧,却是发现了一只大猫头鹰。

丁小甜只在画报上见过猫头鹰,如今看到了活的,就很好奇。猫头鹰蹲在墙头上,一动不动的也去望她。双方对视了片刻,猫头鹰振翅而飞,丁小甜依然保持着扭头瞪眼的姿势,却是已经中了猫头鹰的**术。

\chapter{所谓感化}

收发室已经熄了灯关了门,革委会大院里也是黑沉沉的不见一点光明。等在大门口的人被蚊子咬得狠了,忍不住走进院内去寻找丁小甜。结果到了一堵围墙附近,他们看到了一个雕塑似的黑影。

``丁秘书?''有人开了口:``你看什么呢?''

丁小甜扭头面对着墙头,一动不动。

一只手轻轻的拍了她一下:``丁秘书?''

因为她始终是没反应,所以轻拍渐渐转为了重拍:``丁秘书!''

丁小甜一哆嗦,如梦初醒的转向了来人:``怎么了?''

对方恭敬的对着她微笑:``没事,刚才看你一直对着墙头发呆,我们不知道是怎么回事。''

丁小甜这才感觉到了脖子的酸痛,落了枕似的,将要不敢动:``你们在外面等了多久了?''

那人撸起衣袖,借着月光看了看手表:``两个多小时吧!''

丁小甜莫名其妙的摇了摇脑袋,真不知道自己站了那么久。回想起发呆前的那一刻,她只记得自己看到了一只非常大的猫头鹰。

丁小甜等人披星戴月的走了,只留一个人持枪守门。收发室的房门从外面锁严实了,丁小甜给苏桃留了个搪瓷尿盆,杜绝了她以上厕所为名趁机野跑的机会。从玻璃窗里向外看,能够看到大门前的看守者,窗户下方的木头格子是能左右活动的,像个小小的拉门,平时用来从内向外递信,如今苏桃轻轻的打开了一线,把鼻尖凑到缝隙前吸了一口清凉的空气。

转身回到了小床边,她抚摸了盘在枕头上的白琉璃。白琉璃正在思索着要不要去把无心救出来。要说救,他是能救的,但是白天看无心的意思,似乎并不急于得到自由。无心的思想一贯比他复杂,于是他打算等苏桃睡了,自己再去和无心好好商量商量。

然而苏桃就是不睡。

`

苏桃坐在小床上,平时觉得床太小了,小得让两个人全伸不开腿;可是如今她伸手左拍拍右拍拍,发现床板竟然无边无际,左右全拍不到头。真想无心啊,她徒劳的抽着鼻子,想要捕捉无心留下的气味。

``白娘子。''她轻声开了口:``你要是只小鸽子或者小狼狗该多好啊,鸽子认路,狗通人性,也许还能替我去给无心送个信。我知道无心就在那边的一排空房子里,可我出不去,我没法子去见他。''

她叹了口气:``除了无心,我谁都不想见。我讨厌死那些人了,看了他们我就要吐。我以后要和无心结婚,结了婚就没人能拆开我们了。''

白琉璃游到了床下,沿着椅子一路上行,最后爬到了窗台上,回头对着苏桃嘶嘶的吐信子。苏桃正在东一句西一句的自言自语,忽然见了白琉璃的举动,她不禁一愣,穿了鞋往窗前走。而白琉璃先对着窗户缝隙一探头,随即催促似的转向苏桃,又吐信子又卷尾巴。

苏桃隐隐明白了他的意思:``白娘子,你\ldots{}\ldots{}你要帮我给他送信吗?''

白琉璃像个人似的,晃着脑袋点了点头。

苏桃睁大眼睛,虽然感觉不可思议,但是因为走投无路,所以决定相信白琉璃。从报纸上面撕下一条白边,她用铅笔小小的写了几行字,讲清了自己如今的情形。然后用一根毛线把纸条和铅笔头全绑在了白琉璃的身体上,她把木格子窗微微又推开了一点,然后趁着看守者背对自己,悄悄的把白琉璃放了出去。

白琉璃得偿所愿,既安慰了苏桃,又可以去见无心,一路摇头摆尾,急急忙忙的扭向院子深处。正是带劲儿之时,冷不防一个黑影从天而降,他只觉尾巴一痛,猛的回头看时,发觉自己的身体已经被一只大猫头鹰用利爪踩住。大猫头鹰身躯伟岸,目露贼光,一张大嘴堪比金雕,低头对着他的脑袋就要啄。白琉璃最是爱惜自己的蛇身,眼看猫头鹰想要吃了自己,当即怒不可遏,鬼魂还未脱离蛇身,已经对着猫头鹰恶狠狠的发出了一声狮子吼。大猫头鹰不见鬼魂,只见白蛇,一张尖嘴都张开了,忽然脑中起了巨响,一股子阴邪的鬼气直冲胸膛。力不能支的松了爪子向后一仰,它周身的羽毛都炸开了,体积登时比方才又大了一倍。瞪眼张嘴的喘着气,它既享受着周遭的森森鬼气,又被鬼气重重的激荡了身心,几乎当场昏厥。拍着翅膀勉强飞上墙头,它迅速缩成一团企图隐身,真是感觉又痛苦又畅快。放眼再看地面,它只见地上的白蛇凌空飘起,一溜烟的直奔房屋而去。

白琉璃托着白蛇飘到无心面前,发现无心正睡得深沉。一板砖唤醒了他,白琉璃让他看苏桃的纸条。

无心睡眼惺忪的看过字条,又捏着铅笔条在下面写了回信。忽然看到地上白蛇软瘫,尾巴尖鲜红的渗了血,他开口问道:``你受伤了?''

白琉璃怒道:``来的路上遭了偷袭,是只大猫头鹰,想要吃我。''

无心把白蛇扯到腿上:``大猫头鹰?不会是在黑水洼遇见的那只吧?''

白琉璃想了一想,不能确定,因为猫头鹰都是一个德行:``也许是?总之大得很。''他张开双臂比划了一个尺寸,拖着长声描述:``那——么大!''

无心捏起白蛇的尾巴尖,送到嘴里吮了一口,然后扭头吐出带血的唾沫:``一般的猫头鹰哪有那么大的?兴许就是黑水洼的那一只。那只猫头鹰的来路,我始终是不清楚,我只知道它和你一样,喜欢往战场上凑。战场上有人肉给它吃嘛!''

白琉璃坐在无心面前,拧着两道长眉告诉他:``你轻一点,我的鳞都翘起了一片。''

无心含着白蛇尾巴,用舌尖轻轻压下翘起的蛇鳞,又含糊的告诉他:``别怕。等你过几天再蜕一次皮,伤就彻底好了。一会儿你还回去陪桃桃,我先不走了,外面都是联指的人,我肯定出不了文县。不如留下来先和他们对付着,等到有了机会再说。''

在白琉璃和无心嘁嘁喳喳之时,苏桃一直守在窗前等待。外面有猫头鹰在鸣叫,声音难听到了极点,让人心惊肉跳。不知过了多久,一个圆圆的小脑袋探进了窗口,正是白琉璃回来了。

苏桃欢天喜地的接他进来,取下他身上的纸条展开了看。看过之后她长长的出了一口气,在白琉璃的脑袋上亲了好几下,然后脱了鞋上了床,心满意足的睡了。

翌日清晨,丁小甜上班似的,又来了。

掏出钥匙打开锁头,她放苏桃出去倒尿盆以及洗漱。等到苏桃端着尿盆回来了,她笔直的站立在朝阳光芒之中,横宽的粗壮身体被她从视觉上拔高了些许。默然无语的审视着苏桃,她看苏桃本来是朵含苞待放的白莲花,却因无人呵护,被罪恶的小白脸子浇了一泡热尿。白莲花不知道自己是受了亵渎,反倒喜滋滋的汲取了养分,死心塌地的爱上了小白脸子。

苏桃不知道她是如此的高看自己。对着挂在墙上的一面小圆镜,她不言不语的梳头发编辫子。头发太厚了,乌云似的堆了满肩垂了满背。手背在黑发中闪动穿行,显得手特别白,发特别黑。垂着眼帘目光散乱,她谁也不看,粉扑扑的嫩脸上毫无表情。

等她把自己收拾利落了,丁小甜开始检查她的功课。翻着满布黑字的稿纸本子,她见苏桃的确是抄够了数目,才满意的点了点头。

在早饭前,她带着苏桃站在房内,手握红宝书对准了墙上一幅毛主席像。先是敬祝毛主席万寿无疆,再敬祝林副统帅永远健康,一边敬祝一边挥动手中的红宝书。敬祝完毕之后,她带着苏桃高歌一曲《东方红》,末了又把红宝书翻开了,朗朗的诵读了一段毛主席语录:``伟大领袖毛主席教导我们,节约粮食问题,要十分抓紧,按人定量,忙时多吃,闲时少吃,忙事吃干,闲时吃稀,杂以番薯、青菜、萝卜、瓜豆、芋头之类。''

苏桃嗡嗡的跟着她念,肚子饿得叽里咕噜乱响。然而丁小甜坚决的要除去她身上好逸恶劳的腐朽习气,明知道她腹如鼓鸣,可硬是不让她吃早饭,宁愿自己也饿着肚皮陪她。把苏桃领出收发室,她迎着阳光说道:``军队向前进,生产长一寸。加强纪律性,革命无不胜!''

然后她摆开架势,带着苏桃跳了一支忠字舞。舞毕之后意犹未尽,她又让苏桃随着自己做了一套毛主席语录操。苏桃的肚子里本来就只有糙米黄瓜一类,且早在昨晚就消化殆尽,如今大清早的水米没沾牙,却要没完没了的载歌载舞,不由得有些支持不住。丁小甜走到她面前,严肃的看着她,见她出了一头一脸的汗,鬓角都湿了。

丁小甜很欣慰,认为自己既净化了苏桃的灵魂,又锻炼了苏桃的肉体。黑白之间是容不得灰色存在的,她感觉苏桃像一只迷途羔羊,自己既然见到了她,就理所当然的该拯救她。

把自己带来的饭盒打开,饭盒里面装了两个人的早饭,是杂合面的大馒头和腌黄瓜。两个人一起在桌边坐下了,苏桃拿起馒头嗅了嗅,鼻子里甜丝丝的全是白面味道。

``丁秘书\ldots{}\ldots{}''她小声问道:``无心有饭吃吗?''

丁小甜沉着脸,没有回答。

苏桃不问了,慢慢的撕着馒头皮往嘴里送。丁小甜看了她的吃相,又是个看不惯:``不要做出这副娇滴滴的样子,不想吃就不要吃了。''

苏桃不撕皮了,当即在馒头上咬了一口。她也知道自己边吃边玩,吃得不爽快,不过母亲似乎从来不把狼吞虎咽当成美德,无心也认为女孩子天然的应该慢条斯理一点。女人都狼吞虎咽了,男人是不是就得茹毛饮血生咬活剥了?

吃过一个馒头之后,丁小甜离去,苏桃开始抄写毛主席语录。慢吞吞的抄到傍晚,在开饭之前,丁小甜又来了。

丁小甜在敬祝完毕之后,带她进行晚汇报,检讨一天来的错误行为。苏桃早有准备,说自己白天抄语录的时候贪玩,在陈旧的木制窗框上抠了个坑。咕咕哝哝的忏悔了一阵之后,丁小甜教她打了一套当下最流行的毛主席诗词拳。苏桃一边手舞足蹈,一边得知陈大光的螳螂拳如今已经走上颂古非今、宣扬封建迷信、培养资产阶级个人主义的修正道路了。要是放到北京,陈大光刚一伪装螳螂,就足够被人捉去批斗了。

丁小甜终日忙碌,晚上还要专程教导苏桃打拳,也很疲惫。但是她以奉献和牺牲为荣,如果在教拳的过程中累死了,她也会含笑九泉。

吃过一顿热馒头之后,丁小甜正视着苏桃的眼睛,温和而又坚决的让她写一份思想汇报,汇报今天一整天的思想动态。苏桃被她弄得无可奈何,只能连连的点头答应。坦荡的正气笼罩在丁小甜的横圆脸上,让她看起来已经无所谓了美丑,纯粹成了一座象征或者图腾。

心中忽然受了一点感动,苏桃轻声说道:``我没骗人,小丁猫真的很坏!''

丁小甜定定的凝视着她,不发一言。

苏桃垂下了头:``不信算了,反正我知道我自己是诚实的。下次他敢再来欺负我,我还打他。''

丁小甜不是不信,是不想信,不敢信,也不能信。让她相信她的领袖強姦未遂?她接受不了。

丁小甜锁了收发室,带着自己的部下走出了革委会大院。小丁猫躲在招待所里一天没露面,他的吉普车就暂时拨给了她使用。吉普车停在路口,她须得走上将近一里地的路途。

沿着大街没走多远,她忽然在路边看到了一个古怪的小男孩。

小男孩大概也就是十岁上下的年纪,赤脚蹲在一棵老树下,脚趾头抓着地,趾甲都泛了白。两条手臂软软的垂在地上,他穿着一身大而无当的旧军装。丁小甜急着走路,匆忙中看了他一眼,结果险些被他奇大的黑眼睛吓了一跳。可怜巴巴的仰头望着丁小甜,小男孩一言不发,单只是望。

丁小甜被他看得心里很不好受,好在饭盒里还剩了半个杂合面馒头,被她拿出来扔给了小男孩。有心再问问他家在何处,可是时间有限,她还忙着回招待所向小丁猫汇报工作,实在是不能停留了。

及至坐上了吉普车,丁小甜一拍大腿,忽然明白了自己为何看那男孩刺眼——那男孩长得太像无心了!

无心那个长相堪称出奇,眼珠子太黑脸太白。小男孩与他如此相似,让丁小甜怀疑他是无心的弟弟。可是吉普车已然发动,她犯不上因为个小男孩再半路折回了。

与此同时,小男孩用脚趾头踩住馒头,一个脑袋骤然向下直贴地面。张嘴咬下一口馒头,他直着脖子吞了下去。抬起头把脑袋转了二百七十度,他眼珠子一斜,把背后的风景都看清楚了。

一个馒头没吃完,他力不能支的挪到了暗处。片刻之后,暗处扑啦啦飞出一只大猫头鹰。昨天他被白琉璃的鬼气冲撞了一下,仿佛习武之人打通了任督二脉似的,竟是骤然精进,凌晨时分变幻出了人形。可惜人形不能持久,而且四肢不听调动。悄悄的落到院墙头上,他心里打着如意算盘,希望昨夜的强大鬼魂能再出现一次。

\chapter{走为上策}

无心双手拿着一份认罪书,站在空屋子里结结巴巴的念。认罪书是三个小时前写完的,暴打是两个小时前挨的,丁小甜是一个小时前来的。总之他一直不得消停,舌头在牙齿上磕破了,说起话来满嘴吸气,像是刚刚喝了一大口热汤。丁小甜背着手站在他面前,一边上下审视他,一边想想苏桃,想想前几天在革委会院外遇见的大眼睛小男孩。真有心宰了无心这种白脸子臭流氓,可丁小甜素来按照规章制度办事,无心罪不至死,她没法杀他。

她起了私心,想要诱导无心罪上加罪。等到无心把一份认罪书念完了,她清了清喉咙,向无心问道:``再讲一讲你现在对红总和陈大光的新认识吧!''

无心抬眼看她,不假思索的开始骂街:``红总是彻头彻尾的反革命组织,陈大光更是组成了一个牛鬼蛇神总司令部,妄想翻账企图变天,让广大革命群众吃二茬苦遭二茬罪,手段何其毒辣,用心何其险恶,真是一个耳朵大一个耳朵小,猪狗养的;蝙蝠身上插鸡毛,他们算什么鸟?芝麻地里撒黄豆,一群杂种;吊死鬼搽粉,死不要脸\ldots{}\ldots{}''

丁小甜连忙抬手:``好了好了,你再专门谈一谈你对陈大光的新看法。''

无心双手下垂捏着认罪书,毫不犹豫的又开了口:``陈大光是野狗日的丫头养的穷凶极恶无耻下流占集体便宜睡剧团演员,我要坚决和他划清界限,再见了他我一言不发先给他一个大嘴巴,然后一记窝心脚,不把他揍成猪头肉我不姓吴。''

丁小甜皱着眉毛看他,没想到他居然一点骨气也没有。如果换了自己落入红总手里,自己可是死也不会诋毁组织一句。再听他满嘴的语言,多么牙碜的话都敢说,倒是够识时务的,完全不顽抗。

丁小甜没谈过恋爱,可是知道花言巧语的小白脸对于小姑娘多么具有迷惑性。苏桃坏吗?苏桃不坏,经过了她近几日的言传身教,如今每天都在乖乖的学习红宝书,思想汇报也是天天都写。丁小甜很欣慰,同时相信自己只要把她再关一阵子,就必能让她脱胎换骨,与无心一刀两断了。

丁小甜拿无心没有办法,无心怎么打都打不死,并且是个软脊梁,让她没法子再对他动刀枪。

``如果你能保证不再去骚扰苏桃。''她派头很足的在无心面前踱来踱去:``我可以考虑放了你。''

无心一瞬间就给了她回答:``我不找她了,你放了我吧!''

丁小甜居高临下的扫了他一眼,虽然实际上是他更高,不过丁小甜自觉灵魂已经立于雪山之巅,见了谁都是无愧无邪。

离开无心走去了收发室,她又见了苏桃。苏桃正坐在窗下桌前写字,见她开门进来了,便放了铅笔站起身。

收发室虽然可以开窗户,但是空气没有对流,白天还是热得要命。丁小甜嗅着空气中的汗意,忽然说道:``和我走,我带你去洗个热水澡。''

苏桃把铅笔收进了抽屉里,同时低声说道:``你怎么有时间天天来看我?你们不要干革命吗?''

丁小甜没言语。杜敢闯已经从北京来文县了,像个垂帘听政的太后似的,一手抓着小丁猫,一手抓着联指。如果不嫌麻烦细细算的话,丁小甜和杜敢闯还有一点亲戚关系,两人之间也有着许多年的友情。丁小甜无须像旁人一样去拍杜敢闯的马屁,所以一旦清闲了,便能随心所欲的四处走一走。

苏桃又问:``去哪里洗澡?我不去招待所。''

丁小甜认为她在唧唧歪歪的磨蹭,勉强压下满心的不耐烦,她沉静而又严肃的注视着苏桃:``去钢厂的职工浴池。''

苏桃跟着丁小甜出了门,乘着吉普车往钢厂的澡堂子走。她难得的洗了个热水澡,洗得简直快要脱一层皮。及至回到革委会大院了,她得了许可,披着湿头发坐在阴凉处洗衣裳。湿头发很快就被夏日的热风吹干了,黑亮亮蓬松松,闪烁着缎子的光泽。偶然鬓发随风扬起,露出她的侧影——她瘦了,骨骼清晰,皮肤紧绷,脸蛋上总透出一点粉红。

丁小甜默默的望着她,心里有一点沉默的欢喜。她真希望苏桃可以成为一名纯洁的好姑娘,和自己并肩踏上革命的征途。

正在出神之际,门口守卫的呵斥声音惊醒了她。她扭头一瞧,很惊讶的看到了黑眼睛小男孩。

小男孩还是穿着一身太过宽大的旧军装,裤管衣袖全都挽起了好几层,衣服扣子倒是都系严了,然而一圈领子歪斜着,竟能让他露出半个肩头。睁着一双炯炯有神的大眼睛,他探头缩脑的往院内张望。

苏桃随着丁小甜向外看,乍一见小男孩,她也惊异的``呀''了一声,心想他和无心有关系吗?好一双大眼睛,和无心简直是一个模子刻出来的!

守卫不许闲杂人等在革委会前乱张望,有心把小男孩撵走,不料丁小甜忽然开了口:``小朋友,你要找谁?''

小男孩抿了抿嘴,没有回答。十个脚趾头紧紧的抓了水泥地面,他横着迈了一步,随即双脚一起向前一蹦,身体不动,脑袋却是向前探出老远。一双眼睛扫视了院内风景,他收回脑袋转了身。试探着向前迈出一步,他随即又是一蹦。

没等走远,他被丁小甜薅着衣领拎进了院内:``说,你的家长在哪里?''

小男孩惶恐的仰头看她,同时从喉咙里发出了含糊的声音:``嗥!''

丁小甜听他有话不说,还敢学猫头鹰叫。有心吓唬吓唬他,可是和他对视了一刹那,她不由自主的心软了:``你说你家在哪里,我送你回家!''

小男孩又``嗥''了一声。

苏桃插了嘴:``他可能是\ldots{}\ldots{}不会说话吧。''

小男孩立刻点头。

丁小甜看了苏桃一眼:``你不要管,洗好了就回房去!''

苏桃乖乖的泼了水晾了衣裳,然后转身回了收发室。她可不敢管闲事了,她连一个无心还救不出来呢。

丁小甜眼里不揉沙子,站在大太阳下逼问小男孩的来历。小男孩仰着一张干干净净的小娃娃脸,一双大围棋子似的黑眼珠闪烁着可怜兮兮的水光,翘鼻子小嘴唇,可爱是可爱极了,但是可爱的过了火,几乎显出了几分突兀。对着丁小甜鸣叫了一声,他眼看对方不肯放了自己,情急之下扭头伸嘴一啄,两排牙齿正是啃上了丁小甜的手背。丁小甜猝不及防,吃痛松手。而小男孩转身一步蹿出老远,随即东倒西歪撒腿就跑,两条手臂紧紧的贴在身体两侧,虽然步伐无比的凌乱,上身却是纹丝不动。丁小甜揉了揉手背,追出去再瞧,就见小男孩的背影闪闪烁烁,时有时无的出没在沿街的大树之后。街角忽然腾空飞起一只大猫头鹰,小男孩随之不见了踪影。

丁小甜莫名其妙,还想追究,但是时间又不允许,自己已然在革委会里耽搁了太久,必须去找杜敢闯接受新工作了。

丁小甜是走了,但她留下了看守作为耳目,继续监视苏桃的一举一动。苏桃老老实实的抄语录写汇报,晚饭是看守敲窗户送给她的,她不消人吩咐,在吃喝之前高声敬祝,又念了一段语录,唱了一首《大海航行靠舵手》,该做的仪式都做齐了,她才坐在窗前,开始享用她的一份杂合面馒头和咸菜丝。及至天色一黑,她悄无声息的打开窗缝,把白琉璃又放出去了。

白琉璃最近因为又要蜕皮,所以有些懒洋洋。身上捆着小纸条和铅笔头,他慢吞吞的游出窗口,往无心的小监狱走。刚走到半路,便又遇见了大猫头鹰。

大猫头鹰虽然看不见鬼,但是很会追踪鬼魂。蹲在墙头徒劳的等了好几夜,今日白天他变成人形,就感觉革委会的收发室里藏着一股子淡极了的阴气,想要靠近了瞧一瞧,却是被个粗壮的女将一把抓住。仓皇逃走之后,他趁着夜色又回来了。炯炯双目忽然瞧见地上的白蛇,他高兴之极,拍着翅膀从天而降,心想自己只要一叨蛇尾,必定就能引来阴魂。不料白琉璃处在蜕皮的时期,虽说他本质上并不是蛇,可既然寄居在了蛇身体里,免不得也要沾上几分蛇气。蛇在蜕皮之时周身不适,没有脾气好的,白琉璃也不例外。一见猫头鹰卷土重来故技重施,他当即挣出蛇身发动念力。猫头鹰衔着蛇尾巴还没有合嘴,忽觉一阵凉气直渗入层层羽毛深处。身体立时冻僵了似的动不得了,他张着大嘴,伸着爪子直通通的跌倒在地。

白琉璃把猫头鹰和自己的蛇身一起运起,直奔无心的牢房而去。无心如今除了胖揍管够之外,其余再没有管够的。他打算把猫头鹰从窗户上的铁栅栏间塞进去,让无心吃了补补身体。

无心如今每天都忙得很,丁小甜恨他如仇,再忙也不忘收拾他。一有批斗大会,必定把他当成流氓推上台亮亮相,引得台下的看客们指指点点。上台的次数久了,他有了一点小名气,一听说街上要斗流氓了,比较清闲的妇女群众们必定蜂拥而来,喜气洋洋的专为了看无心。有时候他在台上被人单拎出来骂一顿打一顿,观众们睁着眼吸着气,都感觉美男子挨揍,是场富有刺激性的好戏。

白琉璃把猫头鹰从窗外往里塞。猫头鹰太大了,两条大腿挤在栅栏之间,而白琉璃又不是力工,让他凭着意念卖力气,实在是太难为了他。无心扶着墙站起身,东倒西歪的走到窗前:``白琉璃,你给我带了什么东西?''

白琉璃直接穿墙而入:``是只大猫头鹰,上次就是它啄伤了我的尾巴。你扒了它的皮吃肉吧。''

无心咽了口唾沫,抓着猫头鹰的两只爪子就往里拽:``好主意。白琉璃,没想到你这么关心我,我还以为你又去看打仗了。''

白琉璃把自己的蛇身送进了房内。而猫头鹰此时略略恢复了一点知觉,就觉自己周身快被铁栏挤压变形,一身的羽毛全被蹭了个乱七八糟。正想扇动翅膀做一点挣扎,不料无心抬脚踩住窗台,双臂猛一用力。一声轻响,羽毛纷飞,他已经被无心拽进房了。

入夜之前下了一阵小雨,房屋没关窗户,所以无心冻得双手冰凉。快乐的把大猫头鹰搂到怀里,他一屁股坐在地上,一手捏着猫头鹰的尖嘴,一手掖到猫头鹰的翅膀下:``嘿嘿,又是你?''

话音落下,他把舌头长长的伸出去,在嘴唇四周舔了一圈。松开对方的尖嘴,他开始用手指去拔猫头鹰脖子上的羽毛。猫头鹰看他要以杀鸡的手法对待自己了,吓得肝胆俱裂。而无心拔着拔着,忽然想起了更重要的事。用两条腿把猫头鹰夹住了,他解下白蛇身上的纸笔,展开了去看上面小字。一边看一边又问:``白琉璃,那个丁秘书真没欺负桃桃?''

白琉璃悬在了他的头顶上:``她还好,只是每天逼着桃桃抄书跳舞打拳唱歌。哦对了,她今天还带桃桃去洗了澡。无心,为什么桃桃不用香料,皮肤也是香的?少女都很香吗?''

无心把纸条摁在猫头鹰的脑袋上,捏着小铅笔头写回信:``你可以去闻一闻丁秘书。''

白琉璃一本正经的答道:``我闻不到,我没有和丁秘书睡过觉。''

无心写着写着停了笔,仰起头思索片刻,低头继续写:``白琉璃,我不能再留在这里了。你知道,我的伤好得太快,已经引起了他们的怀疑。我打算带桃桃走。刚才我忽然想起了一件事——钢厂里面铺着铁轨,有专用的车皮直通猪头山矿区。如果火车还通,我们就扒火车走;如果火车不通,我们也可以沿着铁轨走。你在县里见过火车道吗?没有吧?我猜火车道的沿线一定是很荒凉,应该没有人烟。''

白琉璃低头看他,发现他瘦了:``你打算怎么逃?''

无心摇了摇头:``你让我想一想。''

白琉璃不知道无心能走哪条路。革委会的大门前总不断人,后院的院墙前一阵子被炮弹轰出了一个豁子,是无心往日出入的后门,不过豁子外面也有卫兵。让白琉璃出手,白琉璃只能是花费时间与力量去咒死他们,可是卫兵轮换着来,他简直不知道自己应该诅咒哪一位才合适。如果放弃咒术使用板砖,卫兵又不会像无心一样由着他打。

白琉璃正在盘算如何闹鬼吓走卫兵,不想无心腿间忽然缭绕起了淡淡的黑烟。他随着无心一起望去,就见大猫头鹰在烟雾中变了形状,居然成了一个缩着肩膀的光屁股小男孩。两只小手抱了拳头,他蹙着两道眉毛向无心拜了又拜,想要求饶。而无心和白琉璃张着嘴望着他,统一的全呆了。

最后,是白琉璃先开了口:``无心,你是偷偷的和妖精生孩子了吗?''

无心抬起双手捧住了小男孩的脸蛋:``白琉璃,别胡说八道。我能不能生,你还不知道?''

小男孩嗅着空气中浓郁的阴气,身体惬意之极,只是担心被吃,精神上很受折磨。对着无心闪烁了一阵子泪光,他见无心无动于衷,便眯着眼睛又是一笑,小嘴巴咧开了,里面露出一条尖尖的鸟舌头。

无心问道:``你知不知道你很像我?''

小男孩六神无主闭了嘴。

无心又道:``看在你这么像我的份上,我就不吃你了。不过你要帮我个忙,否则我今夜不吃,明夜还是要吃的。哪怕你跑到天涯海角了,我也能让人抓到你!''

小男孩望着他,不住的眨巴大眼睛。

无心扯过他一只耳朵,秘密的耳语了良久。末了抬起头,他追问一句:``听懂了吗?''

小男孩``呼——''的叫了一声。

无心在他头顶拍了一下:``好了,现在马上变回猫头鹰。''

在淡淡的黑烟之中,小男孩恢复了真面目。无心把双手插在猫头鹰的大翅膀下取暖,又和白琉璃嘁嘁喳喳的商量了一番。末了白琉璃带着回信出了窗户,一路游回收发室去了。

\chapter{苏桃的愿望}

苏桃趁夜从窗缝中等回了白琉璃。解下他身上的纸条看了又看,末了她效仿电影里的地下工作者,把纸条塞进嘴里嚼碎吃掉了。和衣上床躺好了,她细细的思量许久,末了喜滋滋的一笑,闭眼睡了。

到了翌日,她照旧的抄抄写写,丁小甜有事出门,顺路过来看了她一眼,见她正在伏案学习红宝书,神情十分沉静,便是非常满意。

如此平平安安的混过了一天,到了傍晚,她拉了窗帘,偷偷把白天省下的一个半窝头用手绢包好,放进了书包里。又将水壶也灌满了,她弯腰从床底下捞出了正要蜕皮的白琉璃,让他与水壶同行,一起到书包里和窝头作伴去。

等到夜色浓重了,她关了电灯拉开窗帘,站在暗中静静的向外张望。门外的看守刚换班了,新来的一位坐在门外水泥地上,正在低头点烟。一只大猫头鹰无声的掠过窗前,苏桃把脸贴上玻璃极力的向外望,只见大猫头鹰收拢翅膀落在看守面前。看守仿佛是吓了一跳,可因见猫头鹰呆呆的站着,并不扑人,才立刻又松弛了身心。

苏桃从昨夜的纸条上得知今晚会有一只大猫头鹰出场。她以为凭着猫头鹰的身量,必把看守啄得抱头鼠窜,不料看守和猫头鹰对了眼,互相都是一动不动。正在她焦急之际,一个脑袋忽然从下而上升到了她的面前,隔着一层玻璃窗,她先是惊骇,随即惊喜——无心来了!

无心看起来颇为吓人,身体姑且不论,只说曝露在外的头脸,两边耳朵全是血淋淋的,面颊也是遍布擦伤,仿佛刚从荆棘丛中钻过。对着苏桃一举手中的半截细铁丝,他开始去撬门外的锁头。丁小甜对于苏桃的本事很有数,并不打算把她当贼防,门外只挂了一枚半旧的小锁头,略略心灵手巧的人都能把它捅开。三下五除二的撬了锁头,苏桃挎起书包拉开房门,一大步迈到了门外。

看守还在外面呆坐,对身后的动静不闻不问。大猫头鹰已经拍着翅膀飞走了,苏桃一把握住无心的手,抬眼看着他满头满脸的伤,嘴唇颤了一颤,却是说不出话。无心把锁头重新挂到门上,然后带着苏桃撒腿向后就跑。最后冲过后院墙上的一道豁口,苏桃忙中一瞥,发现豁口外面也站着一名荷枪实弹的守卫。守卫双眼发直,不知在盯着什么出神。

出了革委会大院又狂奔了两里地,两人渐渐放慢了速度。白琉璃脱离蛇身,成了他们的侦察兵。无心听到前方将要有巡逻队经过了,连忙带着苏桃往路边暗处一躲。苏桃趁机喘匀了气,又伸手轻轻去摸无心的耳朵,低声问道:``疼不疼?''

无心夜里使出吃奶的力气掰弯了窗上栅栏中的一根铁条,估摸着脑袋可以伸出去了,他先是脱了衣裤扔到窗外,然后光溜溜的往外挤,几乎把周身上下蹭去了一层皮。抬手握住了苏桃的手,他低声答道:``不疼,皮肉伤,好得快。''

苏桃想他都想疯了,如今终于又靠在了他的身边,真有一种重生的感觉,纵算逃脱不成,双双死了也心甘。歪着脑袋靠上无心的肩膀,她忽然一甩辫子,把近一阵子的禁闭生活和丁小甜严肃老相的面孔一起甩到九霄云外去了。

无心警惕的注视着前方,等到前方的白琉璃转身对他一点头了,他拉着苏桃站起了身:``桃桃,快走!''苏桃连忙跟上了他。两人摸着黑向前疾行,必要在午夜之前潜入钢厂。

钢厂彻底停产之后,厂区已被武卫国改造成了一处要塞。对于无心和苏桃来讲,要塞的坏处是森严壁垒,危险性极高;好处是联指人员有限,不可能像工人一样昼夜遍布厂区。深夜时候,定有无人的路可以通行。

两个人一路走走停停,末了竟是当真平安到达了钢厂的东大门。东大门不是正门,规模很小,大门是封锁着的,但是外面也站了两名全副武装的联指战士。无心让苏桃靠着工厂围墙站住了,自己低头四处察看。

与此同时,白琉璃已经飘到一名联指战士的头顶,两条始终盘着的腿放下了,他骑在了人家的脖子上。战士很明显的打了冷战,对面的战友出声问道:``哎,你哆嗦什么?''战士没有出声,因为白琉璃正在用手指轻轻叩着他的天灵盖。他从头顶心到喉咙口一起紧了又紧,竟是已经发不出了声音。

白琉璃之所以很少在苏桃面前肆意游荡,正是因为知道自己的阴气会有多重多伤人。弯腰捧住了战士的脑袋,他闭了眼睛,开始喃喃的念咒。在他的咒语声中,无心弯下腰,从墙角泥土中捡起了半截指头粗的钢条。无声无息的走向前方人影,他一边走一边举起钢条,在所有人都无知觉之时,他一钢条抽上了联指战士的后脑勺。只听低低的一声闷响,战士头也不回,直接栽倒。

对面的战士眼看战友遭了偷袭,可是脖子脑袋全都僵硬,手脚又冷又沉的不听调动。无心扬起钢条猛的敲下,钢条穿过白琉璃的身体,把战士打得白眼一翻,也仰面朝天的摔倒不动了。从两名战士身上搜出了钥匙和武器,无心打开大门,带着苏桃进了工厂。

工厂的围墙规格并不统一,东大门内可能是贮存了重要的生产资料,所以围墙高耸,上面还拦了一圈铁丝网。无心一手领着苏桃,一手拎着一把精钢打造的短刀。战士身上当然也有枪,但是无心认为步枪的动静太大,一旦开了枪,自己非彻底暴露位置不可,况且自己并非神枪手,有了枪也用不好。

苏桃看他忽然行忽然止,仿佛能够未卜先知一样,心中却是毫不起疑。她对无心是无条件的信服,无心的一切都合理,合理得让她根本不必再费思量。无心跑,她就跑;无心停,她就停,不看方向不看前路,单是追着一个无心。

厂区里有水泥路,有花园式的小树林。无心顶着无数的蚊虫开路,最后带着苏桃上了一座荒山。说是荒山,其实只是黄土堆成的一个大土包,上面遍生长草,是处无人管理的荒凉区域。带着苏桃站在草丛中,他向远方眺望,只见山下横着两道雪亮的铁轨,一节蒸汽火车头停在铁轨上,后面接着短短几节车厢,全是敞车。苏桃揉了揉眼睛,和无心一起看清楚了——车里装载的竟然是几门迫击炮!

无心不知道如今红总和联指到底打到了何种地步,可是见联指已经开始往外运炮,便知战况一定激烈到了不可收拾的程度。火车头附近也站了几个人,其中一人挺胸叠肚,正是杜敢闯。杜敢闯一身军装,又剪了个偏于男式的短头发,看着越发富有豪气。一手拿着一个纸卷,她对面前几名器宇轩昂的青年长篇大论了一番,然后在青年的簇拥下转身离去。余下几名工人模样的人各自上了火车,却是都聚集在了火车头,并没有人往后面车厢去。

无心来了精神,带着苏桃小心翼翼的往下走。大半夜的,火车拉起了汽笛,雪白蒸汽腾腾的往外喷。眼看火车即将开动了,无心和苏桃快跑几步纵身一跃,轻轻巧巧的扒上了车皮。摇头摆尾的翻入车厢,两人抱着肩膀向下一缩,守着一对铁轮子挤着坐了。

火车越开越快,夜风急急的掠过头皮。苏桃望着无心,忽然粲然一笑。无心也是微笑,同时却又问道:``笑什么?''苏桃双臂环抱了膝盖,小声答道:``我们远远的逃走,去大西北或者大西南吧!''

无心没想到她会有如此的远大志向,不禁继续追问:``去大西北大西南干什么?''苏桃认真的答道:``当盲流呀!''

无心哑然失笑,听苏桃真心实意的告诉自己:``我原来听爸爸说,有人在内地犯了罪,怕被人抓,就逃去新疆西藏。到新疆可以给人摘棉花,到西藏可以给人放牛马。地广人稀的地方,没人管的。''无心一揪她的辫子:``你才多大,准备去当一辈子盲流啊?''苏桃双手握住了他的手:``盲流就盲流呗。盲流也是一样的吃饭穿衣过日子。''

无心伤痕累累的右手被她握着,从手到心,起了一线柔软的暖意。等到逃出文县的武斗战场了,也许他可以带苏桃回大兴安岭避一避。

火车开得很快,苏桃偶尔抬头向外望,看到暗影重重的景色一幕幕急速后退。把脑袋又转向了无心,她低着头去摸自己的鞋尖:``脚长大了,把鞋面顶了个洞。''无心也用手指一摁她的脚趾头:``等到安稳了,给你换双新鞋。''

苏桃细声答道:``秋天再说吧,夏天又不冷。''无心拍了拍她的小腿:``不冷也不能露脚趾头,它又不是凉鞋。''苏桃缩了缩脚:``就当它是凉鞋穿嘛。''

两人唧唧咕咕的说起闲话,不知道闲事怎么会有那么多,说了一件又有一件。苏桃忽然想起了自己的存货,打开书包掏出一个窝头递给无心,让他快吃。在无心狼吞虎咽的空当里,她的嘴也不闲着:``白娘子又要蜕皮了,你不是说蜕皮之前应该让他泡泡澡吗?现在可是没水给他。我身上正出汗呢,把他揣到我怀里行不行?''

远在一节车厢之外的白琉璃本是骑在炮筒上,听了苏桃的言语,他匆匆的腾空而起,飞快的钻回了蛇身里去。等他附体完毕,却听书包外的无心满嘴窝头,含糊答道:``别理他,他自己也能蜕,顶多是慢一点。''白琉璃气得咬住了自己的尾巴尖,想要一砖拍死无心。

不出片刻的工夫,火车已经出了文县地界。原来联指和红总的阵地如同犬牙交错,乱七八糟的互相深入。火车道一线是被联指占住了的,所以火车可以公然的昼夜往返。出了文县不久,火车却是缓缓停了,由于是临时刹车,铁轨上火星乱迸。无心和苏桃吓得趴伏在车厢里,一动不敢动。车厢外面起了争执声音,仿佛是一队联指人马想要卸炮,可火车上的押运人员坚决不肯,说炮是运往猪头山阵地的,他们做不了主。

两方人员都是粗鲁的亡命徒,说着说着就动了武。有人开始明抢,攀着车皮往上爬;火车则是自顾自的鸣笛冒气,正在作势要继续开动。忽然起了一声枪响,远方有人通过电池喇叭高声喝问:``你们干什么哪?''

此言一出,枪声响得越发激烈了。而电池喇嘛静默了半分来钟,随即猛的起了高调:``来人啊,有奸细!红总冒充我们的队伍抢火车啦!''

此言一出,枪声立时响成一片,车皮抵挡不住子弹,被打出点点孔洞。无心见状,索性趁乱下车。自己冒着流弹起身先把一条腿迈出去了,他伸手去抓苏桃,想要抱着苏桃向下一滚,就算摔也是先摔自己。苏桃不消吩咐,心知肚明,弯腰迈步抓住了他的手。可是与此同时,她脸色一变,发现自己的左小腿竟然是卡在铁轮子里了。

怎么卡的,她不知道。她惊惶的拽了又拽,硌得骨头生疼,小腿却是丝毫没有活动的余地。眼看无心正迎着子弹等待自己,她带着哭腔喊道:``你先走,我、我\ldots{}\ldots{}''话未说完,她左臂骤然受了一击,力道狠狠的直透骨头。愣愣的低头一看,她大惊失色,发现自己的衣袖破了一道口子,鲜血正在滔滔的往外涌。

在疼痛来袭之前,她弓起灵活的右腿站稳了,对着无心狠狠一推:``快走啊!''无心身体一晃,侧身栽出车外。未等他爬起来,火车向后一退,随即居然又开动了。

起身追向火车,他拼了命的要去扒上车厢。车厢里的苏桃已然觉出了痛苦。盲流暂时是当不成了,忽然想起了书包里的窝头和水,她单手摘下书包,咬牙把书包向外一掷。随即仰面朝天的躺在车厢里,她在血腥气中望着天上的星星月亮,怀疑自己是要死了。

后方的无心捡起书包,一跃而起扑向车厢。然而一粒子弹贯通了他的身体,他的方向随之偏了,张牙舞爪的扑了个空。在剧痛之中抬起头,他只见火车穿过枪林弹雨,轰隆隆的朝猪头山方向开去了。

\chapter{天各一方}

无心趴在铁轨上,身体仿佛是被一根铁钉直直的钉在了土地上。远方依稀可见蒸汽的影子,最后一节车厢顺着铁轨转了弯,消失在了他的视野中。随着火车的远去,枪声渐渐疏落了,有穿着解放鞋的大脚丫子从他脊背上踏过,跑出没有几步,大脚丫子又折了回来:``哟,你不是无心吗?''

无心忍痛抬起了头,看到了一张面熟的脏脸子,不知道姓名,只知道他仿佛是陈大光身边众多跟班中的一员。上方的声音继续问他:``你跟联指干了?''无心连忙摇头,勉强出声答道:``我是扒火车\ldots{}\ldots{}逃出文县的,没想到你们半路劫了火车\ldots{}\ldots{}''

瞄准他的枪口放下了:``我想你也不能投降。怎么着,你受伤了?''无心单手死死抠住一侧铁轨,疼得周身一起颤抖。

一场混战之后,联指的火车线被红总掐断了,可惜红总没能追上火车,迫击炮还是被死里逃生的联指人员运去了猪头山。

在附近村庄中的一间砖瓦房里,无心见到了陈大光。陈大光还是老样子,无心被人背进房时,他正站在地上吃烙饼卷肉。烙饼和肉的分量都很足,卷好了比胳膊还粗,大炮似的直杵进陈大光的大嘴里。咯吱一声咬下满满一大口,他的舌头在嘴里转动不开了,只能直眉瞪眼的望着无心。还是旁边的人做了解释:``司令,我们半路捡了个他,好像是受伤了,没看出伤在哪儿,反正就是说疼。''

陈大光鸡蛋大的喉结上下一滑,把烙饼和肉一起吞咽入肚:``无心?你来了?''无心踉跄着向前走了两步,直接趴上了冰凉的土炕。子弹把他打了个透心凉,可是因为营养不良,无血可流,所以大半夜的,谁也不知道他到底是怎么了。

``让我躺躺\ldots{}\ldots{}''他五内如焚的轻声说道:``有话明天再说。''陈大光不明就里,看他派头还不小。有心逼问他几句,但看他表情又是真痛苦。张嘴咬了一口烙饼,他带着其余人等到隔壁屋去了。

无心独自趴在炕上,默默的忍痛。白琉璃从书包中伸出了一个蛇脑袋,吐着信子昂头看他。他气若游丝的低声说道:``不要碰我,我身上有血。''

白琉璃缩回脑袋,片刻之后衔着一块窝头又伸出来了。原来他认为无心一贯馋嘴,如今受了偌大的痛苦,自己无话可以安慰,只能喂他一口食吃,聊表寸心。然而无心把脸一扭,并不领情。

白琉璃再次缩回书包,倒钩牙扎在窝头里摘不下来,他一着急,自己把窝头吞了;同时听到无心在书包外面唉声叹气:``桃桃会不会死?不好说啊,她趴在车厢里,铁皮又不能防弹,谁知道她的命够不够结实呢?我记得她的胳膊还让子弹蹭了一下\ldots{}\ldots{}''

话未说完,他趴在炕上安静了。多说无益,他想桃桃命苦,一直是在苦挣苦扎的努力活,然而最后却是想当个盲流都不能够。

白琉璃夜里出发,沿着火车道要去猪头山找苏桃。起初一段路走得很顺利,因为夜里阴气重,正能让他随心所欲的活动;及至天光亮了,沿途的阳气和杀气十分之重,一般的鬼魅早蛰伏了,而他虽然不在乎,可也感到了隐隐的虚弱。

无心留在陈大光的院子里,经过了大半夜的休息,身体也有所恢复了。他穿着一件破旧汗衫,前后各被子弹穿了个洞,洞口边沿染着一圈血迹。这样的伤情是没法向人交待的,他灵机一动,把汗衫撕成零碎布条,捡了其中结实的缠到腰间遮住伤口,其余的则是揉成一团扔了。

陈大光的生活是首尾相连的,昨夜吃着烙饼卷肉离去,今晨吃着烙饼卷肉归来。踩着门槛站稳了,他上下打量着无心,发现他满身都是将要愈合的红伤,而且瘦了,皮肤呈现出了苍白的蜡质,让人感觉他是硬的。

``怎么回事?''他问无心:``真受伤了?''无心抬头看他,没有回答。陈大光先是和他对视,但很快发现他看的不是自己,是自己手中的烙饼卷肉。他在小事小物上素来大方。迈步进屋停在无心面前,他把手里咬了一口的烙饼卷肉递向无心:``饿啦?''

无心接过了他的食物,低头一口咬下半截,也没嚼,饼与肉抱着团的通过喉咙进了胃。再接着几口彻底吃干净了,他终于有力气开了口:``我把苏桃弄丢了。''陈大光居高临下的审视他:``听说你扒火车了?''无心低头舔了舔手指头上的油:``嗯,我们在文县熬不住了,想要逃。没想到半路出了事。我跳了火车,她没跳成。''

陈大光总认为苏桃发育未成,毫无风韵,并且永远穿戴得灰扑扑,老鼠似的低头乱窜。于是毫无同情心的问无心道:``她死啦?''无心摇了摇头:``不知道。''

陈大光懒得在苏桃身上多费心思,直接告诉无心:``枪杆子里出政权,要战斗就要有牺牲,难免的事儿!你别太往心里去,我跟你说啊,建红上个礼拜也牺牲了。我在红总烈士墓后边给她单独立了一座碑。她跟我好了一年整,她没了,我心里能不难受吗?可是难受也没办法,男子汉大丈夫嘛,革命还得继续干,是不是?''

然后他转身出去了,片刻之后带着一桌早饭回来,是分开的新鲜烙饼和炖肉。无心知道红总缺地盘但是不缺物资,因为一支红总队伍新近去了一趟长安县,把粮店商铺银行全打劫了。

全国人民都在执行的早请示晚汇报,被陈大光把门一关,自行忽略了。陈大光暗地里是个无信仰者,之所以热爱革命,无非是想夺权,至少是不去一中当体育老师。抄起烙饼刚刚吃了一口,村子里的大喇叭出声音了,先是播放了一阵《东方红》,随即转成了哀乐与讣告,悼念昨夜战争中的红总死难烈士。陈大光活动着他方正结实的下颚,一口一口吃得有滋有味,神情姿态都是绝对的冷酷。

无心忽然开了口:``我想去趟猪头山。''陈大光抬眼看他:``别拿命不当命了,你留着命跟我干吧!''说着他扭头向地上啐出一粒花椒:``我不要管事的,我只要干事的!''无心答道:``苏桃是死是活,我想要个准信。''

陈大光不屑的``嗤''了一声:``你真是闲出屁了!明对你说吧,现在我不敢去打猪头山。联指在猪头山布防了,对着山下摆了一排迫击炮。想上山得再等两天,石家庄马上来人对我们进行武装支援,等援兵一到,我就开始大反攻。''

无心一言不发的吃吃喝喝,心里并不打算和陈大光合作。到了下午时分,白琉璃喜气洋洋的回来了。``桃桃没有死!''他告诉无心:``有人用吉普车把她接下山了。''无心登时有了笑模样:``是谁接的她?''白琉璃想了一想,然后答道:``是丁秘书。''

无心知道丁小甜对待苏桃还不算坏。而且人在就好,哪怕被丁小甜打一顿骂一顿呢,和生死相比,也都不是大事了。无心立刻有了精神。弯腰扶墙出了门,他偷偷摸进院内厨房,自作主张的加餐一顿。等他转身回到房内了,白琉璃躲在阴暗角落里说道:``猫头鹰又出现了,一路总是跟着我。''

无心爬到炕上,对白琉璃悄声说道:``妖精鬼魅的习性,和人都是反着来的。他专跑死人堆坟圈子,要的就是那里的一点阴气。像你这么伟大的灵魂,不世出的死巫师,你一个人顶得上一坑尸首。他见了你,还不像苍蝇见了屎似的?''

白琉璃听了无心的妙喻,气得把脸一扭:``龟儿子!''无心自从得知了苏桃的情况,心中轻松之极,看白琉璃不高兴了,他连忙双手合什拜了拜:``别生气别生气,我换个说法,像蜜蜂见了花似的,行了吧?''

无心说到这里,就觉得伤口也不甚疼了。自己出去要了一盆水,他从书包里掏出白琉璃的蛇身,浸在水中帮他蜕皮。又对白琉璃说道:``劳你的驾,今晚你再回文县一趟,看看能不能找到桃桃。我虽然见不到她,可只要知道她平安,心里就舒服了。''白琉璃并不拿腔作势,一听请求便答应了。蹲在炕上低着头,他饶有兴味的看着无心为自己的蛇身揭去旧皮。

在这天的傍晚时分,苏桃回到了文县。丁小甜站在地上,凝视着苏桃。苏桃的的确良上衣已经脱了,露出里面一件没型没款的旧汗衫,右臂手臂被包扎好了,外层还能隐隐透出血迹。垂头坐在一把椅子上,她蓬头垢面,一只鞋没有了,裤管还被刮开了一道口子。

``苏桃。''她语重心长的开了口:``你真是让我失望。''苏桃嗫嚅着答道:``我们不是叛徒,我们只是想跑。你们看不惯我们,说我们是搞破鞋,我们就换个地方好了。''丁小甜瞪着她,语气渐渐严厉了:``你知不知道你的行为等同于叛变?''

苏桃拿出老蔫萝卜的派头,温柔疲沓的不合作:``我们又不是联指的人,我们也不是要去投奔红总。''丁小甜伸手一指她的鼻尖:``你怎么不是联指的人?你和无心没为联指工作过吗?''苏桃喃喃的问一答一:``我们也给红总看过大门\ldots{}\ldots{}只是为了挣饭吃,我们不懂革命的。''

丁小甜没想到在当今的时代里,居然还有人公然说出这样软绵绵的没骨头话:``你还是个少年人吗?你还有一点点信仰和热血吗?''苏桃嗡嗡的说:``我信毛主席。''

此言一出,丁小甜没法挑错,同时心中越发恼火。苏桃越是难办,她对苏桃越是上心。苏桃像个大蚊子似的,麻木不仁一味的嗡嗡嗡,真真气到她心里去了。

``既然你不是联指的人,为什么到达猪头山之后,指名点姓的要找我?''苏桃低眉顺眼的望着自己的大腿:``他们说我是奸细,要枪毙我,我想找你给我作证。''丁小甜冷笑一声:``在我眼中,你的行为与叛徒奸细无异!''

苏桃对丁小甜东一句西一句的敷衍了半天,听到此处,她忽然心中一动,起了一点小聪明。可怜巴巴的看了丁小甜一眼,她小声说道:``除了无心,我就只和你熟悉。我想找你救我。''

丁小甜粗声怒道:``哦!是么?原来我和那个小白脸可以比肩了?''苏桃嘤嘤的说:``我知道你是好人。''丁小甜像个好汉似的一晃双肩,嗓门越发粗了:``哦!我又是好人了?''苏桃为了活命,苦着脸对丁小甜勉强一笑:``嘻\ldots{}\ldots{}''丁小甜皱着眉头一摆手:``不要做出这种不庄重的样子!''

一番乱七八糟的长谈过后,苏桃发现丁小甜其实有一点刀子嘴豆腐心的意思,起码对待自己是真够豆腐。仿佛隐隐受到了某种启发似的,她发现只要自己肯动脑筋,倒也能够在丁小甜的羽翼下暂时自保。丁小甜虽然只是个秘书,不过和杜敢闯关系很好,导致她拥有了钦差大臣的身份,说话十分有分量。

因为苏桃受了伤,所以晚餐由杂合面馒头变成了两块蛋糕和一杯冲开的奶粉。苏桃舔嘴咂舌的吃了一块蛋糕,然后对着余下一块愣了好久。不知怎的,她忽然一点儿也不想吃了,因为总感觉那一块应该是留给无心的。

趁着丁小甜不注意,她用一张白纸偷偷的包好蛋糕藏到了床角。结果第二天起床一看,她发现蛋糕上面已然生了一层绿毛。对着绿毛蛋糕叹了口气,她想无心在哪里呢?

\chapter{丁小甜的内心世界}

大清早的,丁小甜起了床,自以为已经醒得够早,不料睁眼一瞧,发现对面床上的苏桃已经没了影子。一床被子叠得整整齐齐摆在床头,床单抹得一丝不皱。

为了保险起见,她把苏桃带进了县招待所。苏桃起初死活不同意,说是招待所里住着小丁猫。丁小甜先是向她诚恳的表了态度,表示自己绝对能够保证她的人身安全,然后揪着衣领连轰带撵,丁小甜像一名牧鹅少年似的,把苏桃一路赶上了吉普车。

小丁猫等人住在三楼,丁小甜则是带着苏桃住在二楼。杜敢闯对于她的所作所为完全掌握,并没有表示反对,因为要引蛇出洞似的看一看小丁猫到底对苏桃有多垂涎,是单纯的垂涎,还是真动了感情。杜敢闯不敢奢望自己能和小丁猫产生革命爱情,退而求其次,只想让小丁猫纯纯洁洁的姑且单身活着,权当是为她不见天日的小爱情守贞。

她为他太拼命了,前一阵子联指组织摇摇欲坠,她让小丁猫深居简出,自己顶着风头往北京跑。她甚至愿意为小丁猫付出生命,所以小丁猫也不能太悠游自在、太没良心。

丁小甜穿戴整齐之时,苏桃端着水盆推门回了房。丰盈蓬乱的乌发之间露出一张水淋淋的白脸。睁着大眼睛看了丁小甜一眼,她不甚情愿似的开口唤道:``早上好。''丁小甜没理她,心里完全不动气的骂道:``死德性。''

等到丁小甜也洗漱过了,苏桃已经坐在了两张小床之间的小木桌前。她的右臂虽然受的是皮肉伤,但是动作之际也一样的疼。丁小甜严肃的、一脸不赞成的给她编出两条麻花辫子,编得不松不紧还挺好。编完之后一斜眼睛,她忽然发现自己的被褥已经被苏桃叠整齐了,心中不禁似喜似怒的有了情绪。

在苏桃的后背上拍了一巴掌,她正气凛然的说道:``走了!''

苏桃起身出门,跟着她到了一楼餐厅。餐厅里已经站满了联指人员,整齐划一的做早请示。连说带唱又学习了一段毛主席语录,早饭终于露面了。人们纷纷落座,如同落潮一般显出了小丁猫。小丁猫正站在餐桌前和杜敢闯说话,苏桃低头大嚼,装看不见;丁小甜扫了他一眼,心中反感而又肃然。对于这个白白净净的小老烟枪,她说不准自己该给出个什么评价,反正她不爱小丁猫。

她二十岁了,知道自己长得不好看,所以不去碰壁,索性谁也不爱。对于异性是一贯的敬而远之,对于同性她也不亲近;太聪明的女生,比如杜敢闯,让她只把对方当成无性别的战友;太平庸的女生,比如无数人,又让她嗤之以鼻不往眼里放。

苏桃的相貌本来是会让她产生距离感的,可苏桃同时又有一点孩子气,有一点小聪明,有一点懦弱有一点柔韧,还有一点执迷不悟的小堕落。这么一个别别扭扭的小美人儿让她想起了自己的妹妹——其实她根本没有妹妹,她只是觉得如果自己有妹妹的话,像苏桃这样就挺好。有貌,让自己看着能够生出怜爱;无才,让自己可以挥洒满腔的思想与才华,再怎么丑也高她一头。自己如同一名牧人,扭送一头迷途羔羊返回正路。

丁小甜一边喝粥,一边浮想联翩。而小丁猫和杜敢闯交谈完毕,落座之时远远的瞟了苏桃一眼。瞟过之后,他怪委屈的哼了一声——满餐厅的男女老少加起来,都比不上苏桃。难道是他下三滥吗?不是的,他品位高,他有什么办法?

可惜马秀红死了,他身边的平衡被彻底打破。杜敢闯最近蹬鼻子上脸,跃跃欲试的想要控制他。小丁猫很是不满,时常想用烟头在对方的脸上摁一下。

吃饱喝足之后,丁小甜带着苏桃回了二楼房间。房门一关,丁小甜清了清喉咙,正要发表一篇义正词严的高论教育苏桃,不料苏桃坐在床上,翻开一本红宝书念起了毛主席语录。丁小甜对于政治一贯敏感,不能阻止苏桃学习语录。双手插在军装口袋里,她张了张嘴,末了哑口无言,转身推门离去。而苏桃降了一个调子,顺势往桌面一望,却是意外的看到了一只信封。

桌面只比棋盘大不多,上面有什么没什么,她心里最有数。伸手试试探探的拿起信封,她心想自己和丁小甜出去吃早饭时,房门一直锁着,怎么会有人往房里送信?下意识起身走到门前,她背靠门板站住了,然后慌里慌张的撕开封口。信封上面只字皆无,里面的信瓤却是内容丰富。展开来飞速阅读了上面的小字,她抬头望着窗外愣了愣,随即低头又读一遍。这回彻底读明白了,她转身去了卫生间,把信封信纸撕了个细碎,全扔进下水道里冲了个干净。

信是无心写给她的,报了平安,也有其它细细碎碎的嘱咐。她望着前方半开的窗户,仍然想不通信是谁送进来的。大白天的,招待所院里人来人往,邮差总不能公然的爬上二楼;而且无心怎么知道她搬进了招待所?苏桃心里七上八下的,心想难道自己身边藏着红总的眼线?可是谁最有眼线的嫌疑呢?苏桃忽然想起了疯所长鲍光——鲍光起码不会和联指是一条心,而无心又曾经说过他像是装疯。

苏桃走到窗前,隔着一张桌子向外张望。阳光已经格外明烈了,照得她心里也是一片亮堂。有真正的军人出出入入,小丁猫打扮得像个讲文明懂礼貌的高中生,正在带着武卫国往外走。一辆吉普车在大门外发动了,一名青年坐在副驾驶座上,手里横握着一把冲锋枪。

在将要上车之时,后方忽然追上了个杜敢闯。小丁猫转身面对了她,阳光劈头盖脸的洒了他满身,深深浅浅的阴影勾勒出了他柔软松弛的皮肤与单薄纤细的骨架,让他显出了一种带着稚气的老态。苏桃立刻缩回了头,仿佛是被小丁猫的奇异面貌吓到了。

到了晚上,丁小甜回房休息。苏桃穿着汗衫坐在床边,她则是弯腰为苏桃解开绷带换药。她的手背皮肤还算细嫩,然而颜色与规格都是粗糙的,黑红的手指关节分明,指甲也是扁扁的大而无当。其实乍一看,她和杜敢闯实在是相像,但又丑的不是一路。杜敢闯是纯女性的丑,像个颇有担当与谋略的悍妇;而丁小甜则带了一点男性化,看着有棱有角无趣味,让人忽略她的性别,直奔她的思想与立场。

伤口是长长的一道,已经结了鲜红的痂。丁小甜给她撒了一层药粉,然后没有包扎,让她晾一晾伤口。对着房内的毛主席像,丁小甜开始带她做晚汇报,忏悔一天中所犯下的罪过。苏桃站在她的身边,就听她自言自语:``今天有个老太太来找我求情,让我们给她儿子一个痛快,把活埋改成枪毙。我看她白发苍苍的样子,竟然产生了怜悯。''

然后她流利的背出了一串语录:``我们对敌人仁慈,便是对同志残忍。各同志要鉴往知来,惩前毖后,千万不要忘记`我们不给敌人以致命打击,敌人便给我们以致命打击'这句话。''晚汇报结束之后,苏桃忍不住问丁小甜:``不打不行吗?谁和谁都没有仇,谁也不是外国杀过来的侵略者,干嘛非要争个你死我活?''

丁小甜看着她,像是在看一只无知的动物,不耐烦而又无可奈何:``你不懂。这是主义之争,不是个人之争。主义之争,不是东风压倒西风,就是西风压倒东风,没有中间路线可走。你不要这么早睡,再学习一会儿。一万年太久,只争朝夕,只要你好好学习,天天向上,我就再给你冲一杯奶粉。''

苏桃乖乖的坐在桌前翻开了毛主席语录。眼睛盯着白纸黑字,心里想着无心,嘴巴等着奶粉。在苏桃浮想联翩的喝热牛奶时,无心也在陈大光的院子里加餐。陈大光背着手从外面走回来,一进院门就发现厨房里亮了灯。拐到门口向内一瞧,他发现无心正站在一口铁锅前吃肉。

陈大光不心疼肉,但是向下看到了他布条都绑不住的鼓肚子,不禁有些担心:``我说你是馋啊,还是想寻死?''无心鼓着两腮转向了他:``我饿了。''陈大光点了点头:``我不是舍不得给你吃,我是没见过你这个吃法。反正你自己小心点,别吃出人命就行。''

陈大光嘱咐完了,自行离去。而无心很努力的往嗓子里又噎了一块肉,然后才回了房。刚一进门,他就发现房里多了活物。大猫头鹰蹲在后窗台上,正在盯着炕上的白蛇出神。无心关了房门,上炕把猫头鹰捧到了腿上。双手插进对方暖茸茸的大翅膀下面,他低声问道:``找到她了吗?''猫头鹰低低的叫了一声。

无心高兴极了,抬头唤道:``白琉璃,过来过来,不能让人家白白辛苦一场。''白琉璃离了蛇身,张开双臂做了个拥抱的姿势,把猫头鹰和无心一起抱住。猫头鹰把眼睛一眯,舒服死了。无心弯腰把下巴抵上猫头鹰的头顶:``以后只要你帮我一次,我就让他抱你一个小时。他最听我的,我说话算话。''

白琉璃斜着蓝眼睛看他:``不要吹牛了。''无心不理他,自顾自的继续说道:``而且我很会抓鬼。只要你乖乖的,我就让你身边永远有鬼作伴。''

一股子淡淡的黑烟升起,无心的怀里少了猫头鹰,多了小男孩。小男孩凭着妖精的直觉,歪着脑袋去向白琉璃靠近。白琉璃看看猫头鹰的人模样,抬头问无心:``你小时候就是这样子吧?''无心近距离的看着白琉璃的蓝眼睛:``我哪有小时候?''

猫头鹰感觉身后这位鬼魂必定和炕上的白蛇有点关系。所以一个小时之后,他变回原形,拍着翅膀飞出后窗户,决定趁夜打猎,抓几只小田鼠小兔子回来喂蛇。

无心走到了隔壁陈大光的屋子里,因为刚才陈大光扯着喉咙千里传音,说是自己白天弄到了一把好刀,让无心过去看看。无心饶有兴味的去看宝刀,然而一进屋门就感觉不大对劲,而陈大光手持一把小菜刀,在一个小灯泡的照耀下,对他嘿嘿发笑。

无心后退一步:``你干什么?''陈大光把刀举到面前:``看看,这还是当年日本鬼子留下的菜刀,锈得像铁片子似的。我让人把它捡回来重新磨了一遍,没想到磨完一看,妈的钢口这么好!''

无心对菜刀没兴趣,只问:``你今天杀人了?''陈大光一摇头:``没呀!''无心抽了抽鼻子:``你屋子里有血腥气。''陈大光闻了闻自己的手,又扯起衣袖也闻了闻,最后把菜刀送到鼻尖:``是刀有点儿腥。''

无心伸手接过菜刀看了又看,没看出什么来,于是把刀还给了陈大光:``陈主任,不是我说。来历不明的凶器最好别要,你知道谁用它干过什么?''

陈大光满不在乎的笑道:``它能干什么?顶多就是杀人呗!''说完他举起菜刀当镜子照。刀面平整,正能影影绰绰映出他的面孔。忽然一呲牙,他对着菜刀抠去了牙缝的韭菜。无心看了他的行为,感觉着实是不怎么体面,便趁机溜回房去了。

陈大光本以为他会是自己的知音,不料他对自己的菜刀毫无兴趣。悻悻的独自站在地上,他挥刀在空中劈了一下,然后伸舌头一舔牙齿,打算再对着菜刀清理一下口腔。对着刀面亮出一口结结实实的大牙,他怔了一下,忽然发现自己的影子很扭曲。影子上面出现了两个红色的光点,眼睛似的对着他闪了一闪。随即菜刀脱了他的手,仿佛被人操纵了似的一刀砍向他的脑袋!

陈大光大叫一声,顺手举起炕上的小桌一挡,菜刀当即砍透了桌面,直逼陈大光的眉心。陈大光把小桌向旁一扔,迈开大步就往门口跑:``无心!来人啊!''

未等推开房门,他只觉后背一痛,正是被菜刀浅浅的划破了皮肉。他不敢回头,撞开房门直往外冲。无心闻声而来,正好看到菜刀在追着陈大光行凶。迎着菜刀一跃而起,他双掌一合,竟是把菜刀夹在了掌中。

陈大光一后背血,嗓子都吓成了破锣:``怎么回事?什么情况?它怎么活了?''无心也不知道是怎么回事,不过见多识广,并不惊讶。夹着菜刀一溜烟跑去院角的露天茅房,他抬手用力向下一掼,把菜刀扔进粪坑里去了。

\chapter{大光与刀}

门外的卫兵闻声冲入院内,以为有人要行刺陈大光,可是未等他们举枪呼喝,就脚不沾地的被陈大光又撵出去了。陈大光虽然挂了彩,但是很能忍痛,没事人似的还问无心:``扔粪坑里去了?''无心看他后背洇开了一大片殷红血迹,不由得一咧嘴,替他害疼:``大粪辟邪,扔进去应该就没事了。''

话音落下,茅房里面``轰''的一声巨响,铺天盖地的屎尿之中激射出一道寒光
,正是菜刀直钉在了院门的粗木门框上,力透三寸,钉入之后还嗡嗡的颤出声音,可惜无人欣赏它的锋利,因为院内院外的众人全被从天而降的大粪给震住了。

以茅房为中心,方圆十米之内的人全都吐瘫了。陈大光虽然一贯意志坚定,可是此刻也几乎呕出了苦胆。无心光着屁股坐在一大桶井水里,下巴搭在桶沿上,眼睛已经睁不开。

陈大光周身涂抹了半块肥皂,几乎搓掉了身上一层皮。末了让人给自己往后背伤口撒了一甁云南白药,他缓过气了,开始报仇。张开大巴掌抓住无心的天灵盖,他一把将对方摁进水里,另一只拳头由上至下击入水中,捶得桶中水花四溅。及至他松了手,无心向上抬起了头,无精打采的说道:``好疼啊。''

陈大光指着他的鼻子尖质问:``你不说扔进大粪坑里就没事了吗?''无心扒着桶沿,从水里捞出一块香皂浑身蹭了一通,然后答道:``唉\ldots{}\ldots{}''

午夜时分,无心水淋淋的回了房。白琉璃没有看懂茅房爆炸事件,如今就围着无心飘来飘去,想要让他讲讲来龙去脉,然而无心并不理他,悻悻的只是想睡。陈大光打着赤膊站在院内,却是还在研究钉在门框上的菜刀——才一会儿的工夫,菜刀居然又生锈了!

他不敢再妄动,心中惴惴的想:``它既然能杀我,自然也能杀别人。如果它听了我的话,自己飞去文县把小丁猫宰了,岂不是妙得很?''

他越想越美,夜不能寐。及至到了翌日清晨,他先放出风声,说联指的奸细昨夜潜入生产队,在陈主任的茅房里安置炸弹,意图谋杀陈主任。生产队的队员们如今也不干农活了,全跟着红总慌慌的闹革命。听闻了联指分子的恶毒行径,队员们纷纷咋舌,说也就是陈主任福大命大,换了旁人,早给炸成鸡飞蛋打了。

一边煽动着村民们的愤怒情绪,陈大光一边把无心叫到了自己房中共进早餐。昨夜他一时暴躁,把无心狠捶了一通,如今为了赔礼,他特地让人给无心炖了一只小母鸡。等到无心把两只鸡大腿全吃了,他开了口:``无心,菜刀可还在门框上呢。你说它是不是成精了?''

无心抬头看他:``陈主任,你到底是在哪儿捡的菜刀?''陈大光用筷子向窗外一指:``我在妇女主任家捡的,她家养了一群鸡,这把菜刀是她家用来给鸡剁食的。''无心思索了一阵,末了答道:``吃完饭我们过去一趟,问问这把菜刀的来历。''陈大光推开窗户吼了一声,直接让院外的卫兵去把妇女主任叫来。

妇女主任是个三十来岁的胖媳妇,因为误以为陈大光爱上了自己,所以正在谋算着把糟糠之夫踹了。面泛桃花的站在炕前,她问陈大光:``陈主任,你有什么指示?''陈大光放了筷子,盘腿转向了她:``我问你,你家那把破菜刀,是从哪儿来的?用了多长时间了?''

这问题让妇女主任十分失望:``菜刀呀?菜刀是日本鬼子投降那年,他们炊事班扔下的。我爹捡回家一直用到现在——陈主任,这不算犯错误吧?我们家可是八辈贫农啊!''陈大光安抚似的摇头一笑,随即又问:``你爹拿这把菜刀,杀过人吗?''

妇女主任几乎惊悚了:``没有没有,绝对没有。这把菜刀钝得不像样,连鸡都杀不死。也就是十年前,黄鼠狼子钻我家鸡棚偷鸡吃,让我爹拿它打了一下子。''无心忽然问道:``打死了吗?''妇女主任点了点头:``刀背敲脑壳,把黄鼠狼子给敲死了。''

无心对陈大光使了个眼色,等到陈大光把妇女主任打发走了,无心告诉陈大光:``陈主任,别问了。既然菜刀已经不再作怪,你就地挖个深坑,把它埋掉也就是了。''陈大光笑而不语,同时细细回想着自己昨夜的一举一动。想到最后,他嘿嘿嘿的坏笑了一串。抄起筷子在面前的大砂锅里捞了捞,他忽然收敛笑容骂道:``操!就给我留了个鸡屁股!''

无心吃饱喝足回了房,发现白琉璃也是吃饱喝足,不知刚吞了什么东西,蛇身中段胀得极粗。而大猫头鹰从后窗户飞到了炕上,正在很友爱的用尖嘴在白琉璃身上左蹭蹭右蹭蹭。忽然看到无心进门了,猫头鹰展开一只翅膀向下一扑,竟然试图把白琉璃藏住。无心脱鞋上了炕,在猫头鹰的头上挠了挠:``藏什么藏?他只认我。你有藏他的心思,不如拍拍我的马屁。''

话音落下,窗外忽然起了一阵尖锥锥的叫声,是个大姑娘穿过院子直进了陈大光的房间:``主任,省里来人啦!''陈大光刚把院门框上的菜刀拔出来了,正在屋里对着它出神。听了大姑娘的召唤,他忙忙的披上衬衫穿了胶鞋,临出门前还把锈迹斑斑的破菜刀藏在了枕头下。

他这一走,便是连着三天没有回来。到了第四天的清晨,他风尘仆仆的出现在了无心面前,开口便道:``你小子倒是有点儿运气,我们要和联指谈判了。''无心眼睛一亮:``要是你们停战了,我和苏桃是不是就能见面了?''陈大光答道:``没死就能。''

说完这话,他转身又走。不过半天的工夫,一个半大孩子在院门口扯起嗓子,让无心准备出发。无心穿着陈大光给他的一身军装,再用书包装起白琉璃。大猫头鹰是不消吩咐的,因为甩都甩不开。挤上陈大光的吉普车,他喜滋滋的向前望——天天守着白琉璃和猫头鹰过日子,生活里一点新鲜滋味都没有,他对苏桃真是想念极了。

联指和红总的队伍虽然还是对峙状态,不过炮火已经暂时停息,并且留出一条安全通道,专供红总高层出入县城。文县是个工业大县,一旦闹出了大动静,便能直接惊动北京。联指作为一个全省性的组织,在河北境内四处和人干仗,其中身在保定的一号二号因为太招人恨,所以行踪神鬼莫测,已经是任谁也找不到他们。倒是三号常驻文县,一抓一个准。

上层人物出了面,希望联指和红总可以停止武斗,组成革命大联合。小丁猫听说陈大光从石家庄找来了援兵,心中正是不安;而陈大光怀着鬼胎,态度也是柔顺;双方一拍即合,居然同意进行谈判。

陈大光到达文县之时,正是下午时分。谈判不是一件抢时间的事情,所以下午时间专门用来召开联欢大会。在机械学院的大礼堂里,陈大光与小丁猫第一次近距离的会面了。

大礼堂里兵分左右,全被双方的精兵占据。在前方台下的空地上,小丁猫和杜敢闯微笑而来,然后一起向上仰望了陈大光的尊容。陈大光万没想到小丁猫本人居然是个一脸稚气的书生。双方伸出了手,他的大巴掌如同一大面粗砂纸,轻轻握了握小丁猫的小手,又轻轻握了握杜敢闯的小手;心想若是单打独斗,自己咣咣两拳便能要了他们的狗命。

小丁猫不怕红总,但是有点害怕陈大光本人,因为他连苏桃都打不过,如果陈大光出手——无须出手,一屁股便能把他坐冒泡。要笑不笑的寒暄几句,他忽然看到了陈大光身后的无心。颇为讶异的一挑眉毛,他用手里的烟卷一指无心,玩笑似的说道:``墙头草。''陈大光一抬蒲扇似的大手,慈眉善目的笑道:``不,应该是向日葵。''

小丁猫听陈大光自赞为太阳,脸上越发笑得欢畅:``哈哈,是冬天的向日葵吧?''陈大光听懂了小丁猫的歇后语。听他暗讽无心欠日,陈大光脸上的神情登时不大好看了,心想你打狗还要看主人,你敢当着我的面骂无心?

与此同时,苏桃正在礼堂后台给丁小甜做跟班。丁小甜并无重担在肩,只是对于谈判一事很不赞成,导致情绪有些低落。联欢大会开始了,后台一直热闹着。一个小姑娘站在角落里,对着镜子往脸上涂抹黑油彩,伪装非洲人。一名戴着眼镜的青年蹦蹦跳跳的越过一地道具,站在丁小甜面前说道:``丁秘书,糟糕啊。诗朗诵《全世界人民热爱毛主席》里面的美国人,普通话怎么练也练不准。''

丁小甜心不在焉的问道:``为什么不换一个普通话好的?''青年答道:``普通话好的都没他鼻子大。''丁小甜摇了摇头:``算了,就是他吧!''等到青年走了,苏桃嘀嘀咕咕的说道:``你要是不爱在后台呆着,我们就去前头看节目吧!''丁小甜固执的告诉她:``我不想和红总的人坐在一起。''

一群花红柳绿肤色各异的演员聚在一起,开始预备上场表演大型诗朗诵。大热的天气,众人脸上深深浅浅的油彩都被汗水冲了个一塌糊涂。其中一个顶着黄色假发的小伙子率先跑出去了,对着话筒高声诵道:``额四一个美国人,额们美国人民苦大仇深。可恨那狗总统约翰逊,提起来不由得劳苦大众泪满襟\ldots{}\ldots{}''

苏桃不敢笑,偷偷的摸到舞台退场一侧,想要去看礼堂内的情形。礼堂里黑压压的全是人头,可是不知怎的,她一眼就看到了第二排的无心。

她一声不吭。回头又看了丁小甜一眼,她悄悄的推开后台小门进了外面走廊。礼堂两侧分列着几个安全出口,她走过走廊,从距离无心最近的安全出口探出了头。而无心本来正在看节目,下意识的一扭头,正和苏桃打了个照面。

他也是不动声色,只说要去厕所,起身经过无数条大腿,直奔安全出口而去。苏桃不再理他,自顾自的转身先走。两人一前一后的穿过走廊,末了进入了一间未上锁的小屋。屋里扔着成堆的背景布,而苏桃转身面对了无心,也没说话,直接一头扑到了他的怀里。

无心也是沉默,同时一下一下的轻拍了她的后背。苏桃的手臂真有劲,快要勒到他的肋骨。他低头一吻对方的头发,轻声问道:``桃桃,伤好得怎么样了?''苏桃把脸埋在他的胸前,闷声闷气的答道:``已经不疼了。''

然后她不知道应该再说什么了,只恨不能把自己和无心揉成一体,以后再也不分开。无心还要再问,可是忽觉后脑勺一痛,回头看时,却是看到了丁小甜。

丁小甜一脸嫌恶的看着他,同时用一把开了保险子弹上膛的手枪对准了他的脑袋。如果不是双方谈判在即,她会一眼不眨的马上扣动扳机。在她眼中无心就像魔鬼一样,阴魂不散的对一个好女孩子死缠烂打。

\chapter{二女对战}

丁小甜大声叫来了人,让他们把苏桃押出大礼堂。苏桃没反抗,临走时用手指在无心的手心里划了一下。联指人多势众,如果无心动武,结果必定是被人暴打一顿。她对无心虽然是千千万万的舍不得,不过识时务者为俊杰,她得审时度势的听话。

苏桃走后,丁小甜放下了枪。满怀仇恨的注视着无心,她有千言万语,一时却又不知从何说起。无心看着她那双暴出血丝的红眼睛,心中却是略略的明白了。

他想丁小甜是嫉妒自己的,而且是极度的嫉妒。有些感情常常来的不可思议不可理喻,越无缘由,越是强烈。丁小甜的下颚呈现出了突兀的棱角,让她的面孔看起来是无比的方正。无心知道她正在咬牙切齿,咬得牙根都酸了。

``你这样做,最后能有什么结果?''他问丁小甜,语气很温和,不是怕了她,是感激她对苏桃的一点真情实意。如果没有真情实意,她犯不上往死里恨他。

丁小甜的下颚渐渐松弛了,松弛得很勉强,因为脸上肌肉依旧紧绷:``我是为了她好。''

无心很奇异的生出了父亲心态,心平气和的告诉她:``桃桃是个最平常不过的孩子,她也只想过最平常不过的生活。你要干革命,可以,但是不应该逼着她走你的路。''

丁小甜的冷笑藏在了瞳孔深处,对于对方的言语嗤之以鼻:``不走我的路,走你的路?十几岁的女生,陪着你鬼混陪着你堕落?无心,收起你的花言巧语吧!不革命就是反革命,没有中间路线。不要怀揣着你的蛇蝎心肠对我装高姿态,我告诉你,如果下次再让我看到你招惹苏桃,我绝不会像今天这样手软!''

话音落下,她转身就走。无心的肤色与容貌都让她感到厌恶。在血与火的大时代里,一个男人长成那个样子,本身就是一种不务正业的表现。

无心独自站在小屋门口,背对着一地五颜六色的背景布,无可奈何的叹了口气。

在联欢大会结束之前,无心回到了礼堂。前排的陈大光无意去和小丁猫共进晚餐,所以估摸着时间差不多了,便从怀里摸出了一把小菜刀,正是那把砍出了他的伤又崩了他一头粪的奇刀。他在乘车出发之前,在生产队里找了个僻静地方,把它重新磨了个锃明雪亮。因为上次出事是在他对着刀片照过镜子之后,所以他这回十分谨慎,特地提前戴上了一副大口罩,生怕又被菜刀认出来。把刀磨好了,他又给它套上了提前特制的牛皮刀鞘,让它姑且不见天日。

及至大会终于落幕了,众人鼓着掌全体起立,让丁陈两位同志先走。陈大光出了礼堂,在上车之前亮出菜刀:``丁同志,别急着走,我们也算是第一次正式见面,我送你一样小礼物吧。''

小丁猫见他向自己双手奉上一把套着皮鞘的小菜刀,不禁愣了一下:``这是\ldots{}\ldots{}''

陈大光笑道:``一把好刀,我也是偶然弄到的。你拿去看看,要是嫌它的形状不好,也可以送到铁匠铺里改一改。''

小丁猫笑了一下,接过菜刀拎住了:``好,谢了啊!''

然后两人各自上车,小丁猫是回了县招待所,陈大光则是住进了机械学院附近的一家旅社。旅社还是民国年间的建筑,是座结结实实的小二层楼。陈大光回到房内,先是关了门哈哈哈大笑一通,然后开始调动人马,自行其事。无心并不知晓他的所作所为,悻悻的在他隔壁房间里躺了,他颇为忧郁的思念着苏桃。

在无心躺在床上装死狗之时,苏桃和丁小甜在县城另一端的招待所里,倒是统一的活蹦乱跳。苏桃坐在床边望着窗户,夕阳余晖把她的面孔镀成了灿烂的金红色,配上她的怒目与撅嘴,和画报上的革命女将形象有异曲同工之妙。丁小甜站在一旁,痛心疾首的将她斥责良久,真是快要说出了嘴里的血,没想到最后只换来了她这么一副``谁敢压迫''的造型。忍无可忍的上前一步,她对着苏桃后背打了一巴掌:``你装什么哑巴?听没听到我对你说的话?''

苏桃不看她,气哼哼的望着夕阳余晖说道:``敌军围困万千重,我自岿然不动。''

丁小甜记得她是个小猫脾气蚊子声音,不想今天看了无心一眼之后,她居然还会和自己一递一句的拌嘴了。对着她的肩头又击一拳,丁小甜提高了音量:``你是怎么回事?敢为了那个小白脸和我对着干了?''

苏桃还是不看她:``不管风吹浪打,胜似闲庭信步。''

丁小甜狠狠的搡了她一把:``在大是大非的问题上,你不要妄想逃避!''

苏桃猝不及防,顺着她的一搡向后仰在了床上。因为知道丁小甜和自己闹破天了也是``内部矛盾'',所以她也有了一点小脾气。一挺身坐起来,她倔头倔脑的转向了对方:``你再打我,我可还手啦!''

丁小甜马上就又给了她一下子:``你还,你还!''

苏桃愤然而起,当即对着丁小甜抡起双臂。丁小甜不堪忍受自己的权威受到挑战,立刻以彼之道还治彼身。不大的房间里瞬间乱了套,一大一小两个女生施展起了王八拳,劈头盖脸的对着胡捶。苏桃打着打着就落了眼泪,吭哧吭哧的一边抽泣一边战斗。而丁小甜越打越是心虚,感觉自己的觉悟和水平被苏桃拉到了一个新低——自己居然和一个小姑娘撕撕扯扯的动起了手,而且练的还是王八拳。

丁小甜意识到了自己此刻的行为有多愚蠢,所以决定速战速决。一掌把苏桃扇到床上,她双手叉腰高声怒喝:``还闹?!''

苏桃不闹了,因为右臂凝结的血痂刚刚被挣破了,顺着胳膊流下了一滴血珠子。她撕了一块卫生纸捂住伤口,蓬着两条乱辫子,哭得满脸通红。丁小甜严肃了身心,居高临下的质问她:``装什么呀?你少打我啦?''

苏桃带着哭腔反问:``你多大劲?我多大劲?你还拿脚踹我了呢,我可没踢过你!''

丁小甜正要反驳,不料楼上忽然起了一声尖叫,随即``砰''的一声巨响,仿佛是有人用力撞开了门板。连忙走去开门进了走廊,她高声问道:``楼上怎么了?''

片刻之后,顾基颤声做了回答:``没事\ldots{}\ldots{}丁、丁同志走路摔、摔了一跤。''

丁小甜信以为真,转身回房继续和苏桃纠缠不清的讲道理。吉普车从钢厂医院拉了一名医生一名护士过来,她也没有留意。

等到医生和护士默默的撤退了,三楼的小丁猫站在地上,叼着香烟吁了一口气。顾基坐在一旁的椅子上,左手已经被绷带缠成了熊掌。鲜血透过绷带,在手掌外侧渗出一片鲜红——在不久之前,他刚刚失去了一根小拇指。

小丁猫研究陈大光的礼物时,他正站在一旁发呆。不知道菜刀里面有什么玄虚,总而言之小丁猫忽然就尖叫了,他一个激灵,只见菜刀凌空飞起,正在迎头劈向小丁猫!

他下意识的伸手一挡,随即护着小丁猫破门而出。菜刀还在空中滴溜溜的打着转儿,像是被一道看不见的屏障笼罩住了。而小丁猫推开他迈步回房,居然伸出右手食指,在刀面上连绵不绝的写画了一阵。等他收手,菜刀``咣当''一声落了地。

落地的声音惊醒了顾基,顾基低下头,发现自己左手的小拇指被菜刀砍断了。下意识的呜咽一声,他骤然恢复了往昔的软蛋风采。英俊的五官皱成一团,他像个没成形的小孩子一样,开始连哭带嚎。

丁小猫并不肯声张菜刀作怪之事。关了房门拍拍顾基的肩膀,他安慰道:``少了个小指头,不算什么。你今天算是立了一大功,我不会忘记你的功劳!''

顾基已经熬过了最初的剧痛,此刻在小丁猫的抚慰下,他委委屈屈的一点头:``嗯,我知道。''

小丁猫故作轻松的又笑:``九个指头一样生活工作,不耽误吃不耽误喝,如果将来在个人问题上因此遇到了困难,我可以替你出面。我姓丁的说句话,总会有人买账的嘛!是不是?''

顾基还没想过``个人问题'',不过小丁猫大包大揽的豪爽态度,倒是让他有了一点安全感:``嗯,我知道。''

然后他抬起了头:``丁同志,菜刀是不是被敌人安装了遥控装置?要不然它怎么能说飞就飞?''

小丁猫深沉的一点头:``陈大光毫无谈判的诚意,居心险恶之极。不过今天的事情你不要对外说,我自有安排。''

顾基打了个喷嚏:``现在夜里冷了。''

小丁猫笑而不语,夜里不冷,是屋内的鬼魂让空气降了温度。像猎人贮存武器弹药一样,他学了岳绮罗的法子,贮存鬼魂。对于人类来讲,鬼魂是种看不见的力量,也许微弱,但毕竟是力量。方才他放出鬼魂困住菜刀,现在他抬起了手,正要效仿岳绮罗虚空画符收回魂魄。但是手指在空中顿了顿,他捂着心口背对了顾基。

岳绮罗的法子是不能常用的,用得多了,他会感觉岳绮罗正在自己的心中缓缓复活、东奔西突。

``顾基,你回房休息吧。别人问起你的伤,你扯个谎,别说实话。''他如是说道。

顾基乖乖的起身离去。而小丁猫锁了房门关了电灯,走到桌前撕下几条白纸。拧开一瓶墨水,他把指尖伸入瓶中蘸了蘸,然后在纸上龙飞凤舞的画符。

他的办法是繁琐了一点,使用时比不得岳绮罗潇洒自如,好在没有观众。纸符刀片似的斜飞出去,飞到鬼魂所在之处忽然一滞,随即飘然而落。小丁猫绕过桌子捡起一张张纸符,把纸符用胶水全粘贴在了菜刀上。菜刀上附着邪气冲天的鬼魂,不知是它斩杀过什么妖物。小丁猫以毒攻毒,用纸符里的鬼魂阻住了菜刀里的鬼魂。

小丁猫上辈子和鬼打了太久的交道,以至于他这辈子对于鬼神之流毫无兴趣。心思从菜刀转移向了陈大光,他认为还是人有意思。与人斗争,其乐无穷。忽然抄起桌上的电话,他找到了李作诚,让对方趁夜调兵,设法暗暗包围陈大光所住的二层旅社。

他忙着,陈大光也没睡。旅社楼后挖了深坑,因为他刚刚得知全县的电话线电缆都从他的脚下过。几名技术高超的工人守在地面,随时预备下坑施工,建立一个地下窃听站。

所有的人都很忙,即便身体清闲,精神也是紧张的。只有丁小甜的革命热情一落千丈,还在和苏桃唧唧咕咕的耍嘴皮子。苏桃死不认错,也不肯顺着她的意思和无心一刀两断;她去食堂打了一份土豆片炒肉,当成晚饭两个人吃,苏桃不思悔改,还把肉全挑着吃了,挂着满脸的眼泪也不擦。丁小甜被她搞得很疲倦,颇想再揍她一顿。

两人一宿无话,到了翌日清晨,丁小甜整理了身心,严肃了表情,勉强把思想境界恢复到了往昔的高度。把苏桃反锁在房里,她随着小丁猫杜敢闯出了发,要去机械学院和红总谈判。

苏桃趴在窗口向外望,眼看他们上车走远了,就开始在屋里转圈,想要逃走。忽然推开窗户又把脑袋伸了出去,她见招待所院内虽然安静,但是偶尔也有人来人往,是容不得自己顺着排水管子爬窗户下去的。

正当此时,一个影子立着脚尖横挪过来了,正是鲍光扛着拖把,要来擦拭水泥花坛的边沿。扬着脑袋一个亮相,鲍光正和苏桃对了眼。苏桃慌不择路,对着鲍光做了个口型:``救命。''

鲍光怔了怔,随即像没看见似的垂下头,继续干活。

\chapter{逃离招待所}

苏桃见鲍光不理睬自己,只好悻悻的缩回了脑袋。她总觉得自己和鲍光是同命相怜的人,文化大革命像是一部粉碎机,粉碎了她的家庭,也粉碎了鲍光的人生。她比鲍光强在不必装疯卖傻、劳动改造,而鲍光比她强在亲人俱全、家庭尚存。

鲍光用湿淋淋的拖把擦了水泥花坛,然后扭着大秧歌回到楼内冲洗拖布。他疯得很有分寸,一般只跳革命舞,唱革命歌——其实他本来也是投错了胎,男人壳子里藏着个能歌善舞的女人灵魂。先前碍于身份,他是不敢唱也不敢跳,如今好了,他身为疯子,可以明目张胆的捏着嗓子唱李铁梅了。

把拖布架到窗口晾在太阳下了,他暂时得了清闲,一路扭进了他的专用办公室。他的办公室乃是一间背阴的杂物间,里面放着无数笤帚拖布以及沦为抹布的破毛巾。关上房门对着墙角,他嘴里还在咿咿呀呀,但是表情严肃了,是个犹豫不决的模样。末了上前几步弯了腰,他巧妙的挪动了无数破烂,不知从哪个老鼠洞里掏出了沉甸甸的一大串钥匙。

能够舍了脸皮装疯自保的人,当然不会是傻瓜。在针对他的大字报贴出的第一天,他就耗子过冬似的藏起了体己,比如当时能弄到的钱,包括公款和私款;以及粮票,包括地方和全国;还有全招待所的备用钥匙。反正当时上下一团乱麻,谁也管不得谁了。从钥匙串上解下一枚小钥匙,鲍光又迟疑了一下,随即把钥匙揣进了裤兜里。把他的破烂重新一层层的安放好,他抄起两条大抹布,打开房门一路高歌而行,继续劳动去了。

苏桃在房内枯坐许久,中午吃了丁小甜留给她的一纸包饼干——她平时最爱吃饼干的,可是如今嚼的满嘴乌烟瘴气,木渣渣的毫无滋味。一颗心东跳一阵西跳一阵,让她慌得站不稳坐不住。

及至到了下午,她含着一块忘了嚼的饼干,开始直着眼睛发呆。走廊里响起了鲍光的歌声,招待所的墙壁全用油漆刷了半人高的墙围子,鲍光隔三差五的就要把墙围子擦拭一遍。歌声距离苏桃越来越近了,忽然``嗷''的起了个高调,高调之中夹杂着``咔哒''一声轻响。苏桃木然的扭头一望,却是发现门上的暗锁已然开了!

歌声越来越远,而苏桃站起了身,顺手抓起了丁小甜丢在床上的一只联指红袖章。走去拉开房门向外望了望,走廊里暗沉沉的没有人,只有鲍光在尽头干活。苏桃心里明白了,但是不敢道谢——无论自己能不能成功逃离,都不可以暴露鲍光的行为。鲍光是无处可逃的,他还得在招待所挣出自己的一日三餐。

转身关了房门,苏桃做了个长长的深呼吸。把乱跳的心脏压到胸腔最深处,她一边套上联指红袖章,一边昂首挺胸的走向楼梯口。平平静静的出了大楼,她目不斜视的直奔院门。守门的两名卫兵丝毫没有阻拦她的意思,因为她的服装与袖章、神情与态度,都是典型的``自己人''。

苏桃不喘气,一喘气心就要往乱里跳,心一乱,脚步也要乱。咬紧牙关走在光天化日之下,她头顶悬着一把剑,一步一步像是走在了刀锋上。身后忽然起了汽车声音,而且是小车。声音越来越近了,她闭了闭眼睛,心想难道是谈判已经结束了?身后的车里又坐着谁?

她的两只手变成了冰凉,手臂的关节都僵硬了。一辆黑色小轿车从她身边缓缓经过,里面当然坐着不凡的人物,但是和她没有关系。

冷汗顺着她的鬓角往下流,一直淌进领口里。盛夏时节,一声车响却是冻透了她的身体。她在路口拐了弯,一边往小路上走,一边摘了手臂上的红袖章。胳膊腿儿都是硬的,走不利落,于是她开始跑,朝着机械学院的方向跑。机械学院已经可以算作是红总的地盘,她只要见了红总的人,就一定能够打听出无心的下落。

在苏桃穿大街走小巷之际,陈大光和小丁猫已经在机械学院的大会议室里谈崩了。双方都是没诚意,都是狮子大开口。陈大光话里话外透出的意思,已经是在暗示小丁猫滚回保定。小丁猫涵养极好,一根接一根的吸烟,旁边的杜敢闯也是深藏不露。只有丁小甜听不下去了,借故出去独自散步。在她心目中,红总是彻头彻尾的反革命组织,和这样一个组织组成革命大联合,简直就是给联指抹黑。

到了傍晚,谈判毫无进展的告一段落。小丁猫和陈大光一团和气的起立握手,心里则是统一的在琢磨如何打响第一枪。无缘无故的动武,总像是有点儿理亏,将来上头派人下来调查了,说着也不硬气。陈大光恨不能恳求小丁猫给自己一个大嘴巴,而小丁猫也颇愿意承受陈大光的一记耳光。

两位大头目谈笑风生的出了会议室,与此同时,苏桃也到达了机械学院的侧门。联指的巡逻队走到此处就自动的向后转了,因为以侧门为界线,对面正站着红总的巡逻队。

苏桃和联指的队伍走了个顶头碰。队伍中的队长履行职责,立刻拦住苏桃,先让她背了一段毛主席语录,然后盘问她从哪来到哪去。苏桃做贼心虚,脸上红一阵白一阵的,又见几米之外的人员全带着红总袖章,自己面前横着的只有一小队联指战士。支支吾吾的答了几句,她瞅准巡逻队中的一处缝隙,忽然拔腿冲锋,一头撞破人墙冲向了前方。两边的人立时全都愣了,而苏桃一边飞跑一边喊道:``我找陈大光!''

此言一出,红总的巡逻队中有一个小伙子认出了她:``哎?你不是原来在革委会看大门的丫头吗?''苏桃气喘吁吁的停在了小伙子面前,急急的答道:``是我,我和无心走散了。我——''未等她把话说完,对面的联指战士起了吼声:``回来!你到底是什么人?是不是他们派出来的奸细?''

此言一出,红总立刻针锋相对的骂上了:``你说谁是奸细?她是我们红总的人,轮得到你们盘问?''联指方面立刻有了回应:``放你妈的屁!她是从哪边跑出来的?''

双方隔着一道侧门宽的距离,开始扯着喉咙对骂,本来就是生死仇家,如今虽然碍于谈判,不好动刀动枪,但是动动嘴皮子还是不成问题的。三五分钟之后,他们骂着进入石器时代,开始互相捡了石头投掷。苏桃得了小伙子的指示,撒丫子往前方继续狂奔。跑过了一条大街之后,她找到了被红总征用为司令部的二层旅社。一名军装整齐的干事从里往外走,抬头一见苏桃,登时开口惊道:``哟,你不是原来在革委会看大门的丫头吗?''

苏桃跑得直咽唾沫,否则心脏会一直跳到喉咙口:``我\ldots{}\ldots{}我从联指逃出来了,我要找无心\ldots{}\ldots{}''干事眼珠一亮:``你是从联指逃出来的?没人追你?''苏桃抬手向后指,语无伦次的答道:``他们在侧门正骂着呢。''

干事好像想起什么美事似的,无暇多听,拔腿就走。苏桃则是被门口的卫兵拦了住,不得入内。站在楼下向上望,她漫无目的的喊道:``无心!我来了。''一声过后,二楼上的一扇窗中立刻伸出了无心的脑袋。随即肩膀出来了,一条腿也出来了,无心从二楼窗户直接向下一跳,从天而降的落在了苏桃面前。两人对视一眼,无心笑了,苏桃也笑了,小声说道:``累死我了。''

无心拉着她的手转身往楼里走,一直把她带到了二楼的房间里。开了一瓶汽水送到苏桃手中,他又拧了一把湿毛巾。弯腰站在苏桃身边,他一手托着她的后脑勺,一手托着毛巾,给她仔仔细细的擦了一遍脸。然后苏桃接过毛巾,又把耳朵脖子也擦了擦。

气氛是不可思议的恬静,仿佛两个人一直在一起,从未分开过。苏桃脱了鞋,盘腿坐在小床上。白琉璃本来正在睡觉,这时受了惊动。从枕头下面探出了头,他很意外的看到了苏桃,立刻高兴的吐着信子凑上去了。

无心双手把他捧到了苏桃的腿上,自己也紧挨着苏桃坐下了。苏桃一手握着汽水瓶子,一手轻轻摸着白琉璃的圆脑袋。白琉璃天天守着一个愁眉苦脸的无心,一只一厢情愿的猫头鹰,烦得几乎要死。如今终于领略到了一点少女的柔情,他心里登时愉快了许多。

无心偏着脸,望着苏桃微笑,笑着笑着他下了床:``你等等,我出去一趟,马上回来。''不等苏桃阻拦,他已经开门走了出去。几分钟之后他真回来了,端着一只搪瓷茶缸,茶缸里面放着两支半融化的雪糕。雪糕比红豆冰棍贵了一倍,平时是不大买的。单腿跪在床上,他把茶缸递向苏桃:``赶紧吃,再不吃就全化没了。''

苏桃接过茶缸,拿起一支舔了一口,舔完之后抬头对着无心笑:``真好吃。''无心凑回她身边坐下了:``先吃,吃完了再说话。''苏桃把雪糕送到无心嘴边,无心小小的咬了一口。咬过之后苏桃不收手,无心只好小小的又咬了一口。苏桃收回雪糕一舔,低声重复了一句:``真好吃。''

在丁小甜身边,她是不敢轻易点评食物的。一旦她舔嘴咂舌的说好说坏了,丁小甜便要义正词严的说她``满脑子都是吃吃玩玩的资产阶级思想'',又让她``把嘴闭上,不许放毒''。如今回到无心身边,她像只小鸟终于抖散开了羽毛,周身都是清凉自在的风。变本加厉的把两支雪糕赞美了一顿,她由着性子吃鸟食,东啄一下西舔一下,最后像要对谁示威似的,她还唆了唆两根带着奶香的木棍。

无心握住了她的手,她歪头枕上了无心的肩。两人全都长长的伸了腿,无心听她讲述方才的历险记。当时险是真险,可事后回想起来,却又带了一点传奇色彩,仿佛不甚真实。讲完最后一句,两人都沉默了片刻。苏桃张开五指,和无心比了比巴掌的大小,同时小声说道:``以后,咱们再也别分开了。''无心合拢手指攥住了她的手:``好,不分开。''

苏桃感觉自己说的还是不够准确,所以加以强调:``我们一辈子、永远、总在一起。''无心留意的看了她一眼,看她还是孩子的脸。十几岁的小姑娘,真懂得什么叫做一辈子吗?无心想她是不懂的,但不管她此刻懂不懂,他都先答应着了:``好,总在一起。''

苏桃的心中还没有爱情的概念,她只是觉得无心最好,自己最想和无心在一起,在一起就安心,不在一起就惶恐。既然无心答应了她,她便心满意足的别无所求。欢欢喜喜的跪在床上,她开始和白琉璃玩。而白琉璃生前不曾恋爱,死后略微的开了点窍,刚才听了苏桃和无心的一番对话,他咂摸来咂摸去,感觉很有意思。

在苏桃拿着小手绢给白琉璃擦身之时,红总与联指之间的大决战,由两群百无聊赖的巡逻队员,在机械学院侧门外拉开了序幕。红总一方来了一名干事,很巧妙的激怒了联指的巡逻队长,被队长用板砖进行远距离打击,正好拍在了鼻梁上。干事立刻抹了自己一脸鼻血,倒在地上抽搐不止。

一旦有人挂了彩,这场嘴仗的性质就起了变化。双方越过界线开始对打,打到最后,红总一方出了人命,死了个十六岁的孩子。陈大光在旅社里听闻了这个消息,乐得一拍巴掌,仰天长笑。

\chapter{大决战}

在陈大光的彻夜调遣之下,红总的队伍无声无息的大集合了。来自石家庄等地的援兵也不显山不露水的暗暗抵达文县周边,最新式的武器全被藏在不见天日的隐蔽处,队伍口令每小时变化一次,严防联指的奸细刺探军情。

死于机械学院侧门的十六岁孩子被人收拾干净了,身上还覆盖了红总的旗帜。他成了红总的烈士,生的伟大死的光荣。当战斗准备全部就绪之后,陈大光大张旗鼓的为他举行了治丧游行。

游行以一辆崭新的解放牌大卡车开路,卡车上的乐队把一曲哀乐演奏的惊天动地。紧随其后的便是灵车。灵车被黑纱和白花装饰满了,车头悬挂着孩子的大幅遗像。孩子是个活泼孩子,在相片上笑得有牙没眼,让人没法把他和灵车上被旗帜包裹着的小尸体联想到一起。

灵车之后,跟着上万人的送葬队伍。前方哀乐凄凉,催人泪下;后方的口号声则是震天撼地,催人尿下。总而言之,一望无际的队伍直奔文县主要大街,杀气腾腾的要让联指``血债血偿''。

文县如今已经是联指的地盘,只不过是因为要和谈,才开了个口子放陈大光等人进城。如今红总的人居然蹬鼻子上脸的闹了游行,联指自然不能坐视。道路两旁的楼房,一色五十年代建造的苏联式建筑,如今窗户全开了,钢筋焊成的巨大弹弓立到窗前,接二连三的往外发射板砖。

弹弓力度惊人,一砖打到身上,能拍出人的内伤。游行队伍立刻乱了形状,而打头的大卡车看到前方来了一队联指战士要拦路,司机立刻一踩油门,``轰''的一声直冲向前,当场碾死了两个人。

在游行队伍被板砖打得满街窜之时,红总的武装队伍趁乱出现,而提前布置在附近楼顶的重机枪也亮了相,开始对着窗口进行扫射。在一锅粥似的大街上,两派的大决战开始了。

城内枪一响,城外也乱了套。陈大光切断了全县的电话线电缆,小丁猫暂时与城外战场失去了联系。但小丁猫也不是吃素的,电话线一断,他立刻启用了无线电台,调兵到文县外围,想要关门打狗,在文县内部解决掉陈大光。可惜他尽管想得美,陈大光却不肯束手待毙。红总和附近城市的军校有了联系,军校学生把坦克一路开上战场了。

红总在城外开了炮,联指在城里也开了炮。一门加农炮瞄准了陈大光所在的二层旅社,一天之内发射三百发炮弹,把旅社轰得渣都不剩。陈大光卫兵众多,行踪不定,不怕炮轰。无心和苏桃则是躲入一户民居。民居里的居民早逃出文县了,留下的空房子清锅冷灶,窗户被砖头砌了一半,目的是防流弹——如今坐在家里却被流弹打死,也不是很稀奇的事情。

两个人全都不敢上炕,因为一层砖头也未必挡得住子弹,走兽似的靠在炕边席地而坐,好在外面正是秋老虎肆虐之时,地面丝毫不冷。两人相聚的好日子刚过了两天,还没过新鲜劲儿呢,就遭遇了新的大战。如今别说吃雪糕了,棒子面窝头都没地方找去。

饥一顿饱一顿的熬到夜里,两个人从炕沿露出眼睛,透过半截玻璃窗往外看。夜空之中闪烁着一道一道的火光,是子弹飞或者炮弹飞。爆炸声昼夜不息,没有人能安心入眠。

无心怕苏桃害怕,把她搂到胸前捂着耳朵;苏桃怕白琉璃害怕,但是找不到他的耳朵,只好把他整个儿抱进怀里。白琉璃暂时倒是没有出去看武斗的打算,因为发现恋爱比武斗更有趣。他等着无心和苏桃再说几句``在一起''之类的话,可是两个人都没再提。

昏昏沉沉的混过一夜,到了天明时分,无心带着苏桃出去找食。商店全都关门了,好在他们所处的位置偏僻荒凉,走出不远有条小河。无心跑去河边抓了一串大肥蛤蟆。拎着蛤蟆回了住处,他关了院门,把蛤蟆宰了扒皮;苏桃则是把厨房里的土灶点燃了,提前烧起了一锅水。粉红色的蛤蟆肉扔进锅里,倒是煮出了油汪汪的一锅汤。

两人都饿极了,守着大锅连吃带喝。苏桃起初还有点儿犹豫,无心安慰她道:``放心吃吧,又不是癞蛤蟆。再说癞蛤蟆扒了皮,也是一样的能吃。''蛤蟆肉刚吃了一半,不远处轰然一声巨响,正是胡同外面落了炮弹。黑色硝烟铺天盖地的遮住了远方朝霞。无心一把抓起书包,一把扯住苏桃,出了门就往胡同尾巴跑。他先跑,其余各家的居民回过味了,慌里慌张的也跟着跑。

全县已经没有安全的藏身之处,仅有的防空洞也被武装队伍占据了。整条胡同的人心有灵犀,一起奔到了河畔野地。其中有嚎啕的有痴呆的,还有一个吓得发了疯,哭哭笑笑声震云霄。

无心和苏桃装着半肚子蛤蟆肉,半饱不饿的坐在河边耗时光。最后无心小声说道:``我们还是去找陈大光吧。陈大光虽然目标大,但是守卫也严,肯定比我们更安全。''苏桃一点儿主意都没有,跟着无心站起身溜过人群,两个人像是属黄花鱼的,贴着墙根悄悄逃了。

无心费了不少的劲,终于在一处废弃的火车隧道里找到了陈大光。陈大光一天一夜没合眼了,然而精神焕发,根本没空搭理无心。无心向人要了两个面包,和苏桃分而食之,然后很识相的抱着膝盖坐在角落里,不出声也不添乱,一坐又是一整天。

到了夜里,两人朦朦胧胧的要睡,耳边听得有人说道:``报告司令,第一批队伍开始进攻县招待所了!''陈大光声若洪钟的答道:``好!''无心似睡非睡,心想红总竟然已经打到了联指总部,难道小丁猫要完蛋了?

未等他想出眉目,头顶忽然起了一声怒吼。他打了个激灵,睁开眼睛就见隧道之中一片忙乱。连忙把苏桃也推醒了,他起身问道:``陈主任,怎么了?''陈大光往腰间挂了几枚手雷:``撤退!''

在卫兵的护送下,无心背起睡眼惺忪的苏桃,跟着陈大光往隧道外的荒野地里跑,一群人跑成了草上飞,两只脚恨不能不落地。一鼓作气跑到安全地带了,陈大光喘着粗气趴伏在草窠子里,通信员则是摆起电台迅速发报。无心蹲在陈大光身边,就听他气喘吁吁的自言自语:``妈的,想堵老子?门儿都没有!''

隧道方向很快起了枪响,陈大光纹丝不动,直到通信员向他通报了最新战况。带着刚被蚊子咬出的一身大包起了立,陈大光一马当先的返回隧道。企图进入隧道偷袭陈大光的联指战士已被红总的援兵尽数击毙。隧道通体成了黑色,烫如火炭,是被联指用火焰喷射器烧灼过了。

隧道是住不得了,无心随着陈大光换了地方。心惊胆战的熬过一夜,翌日依旧是战火连天。联指显然是真落了下风,因为红总的队伍里应外合,已经占据了大半文县。

陈大光有一点``狡兔三窟''的意思,每隔几个小时便要换一次阵地。无心跑不动了,带着苏桃在一处断壁残垣后休息。断壁残垣就在县招待所的后方,招待所里还有人在抵抗,但是据说小丁猫等人已经早撤了。

到了下午,无心饿得发昏,苏桃作为一个十几岁的孩子,更是耐不住饥。无心急了,又听招待所一带已经枪声疏落,便打算带着苏桃过去碰碰运气。如果招待所已被红总攻克,自己也能进去找点吃喝。他很谨慎,带着苏桃在瓦砾堆上匍匐前进。一点一点的挪到了招待所一侧,无心和苏桃从坍塌了的围墙中,意外的看到了丁小甜。

院内乱七八糟的垒着沙袋,丁小甜就趴在一堆沙袋后面,正在遥遥的对着院子另一侧的战友喊话。院子里已经没有多少人了,丁小甜的身边躺着两具尸首,都是中弹而亡的青年。而丁小甜喊完话后一回头,忽然看到了烟熏火燎的苏桃,登时愣了一下。苏桃也是一怔,但随即小声开了口:``你还打啊?你快跑吧!''

丁小甜本来就丑,如今涂了满脸烟尘,更丑了。抱着冲锋枪又狠狠的看了苏桃一眼,她开了口,嗓子哑得像个男人:``别往前走,前头还有我们的人。''苏桃知道她心肠不坏,甚至是个好人。忽然想起她对自己的好处,苏桃几乎着急了:``小丁猫都走了,你还留?你傻呀?''

丁小甜仿佛已经不屑于解释,对着苏桃笑了一下,她的神情堪称平静:``杀了我一个,还有后来人。''苏桃真急了,可是又知道自己说不动她,情急之下只会重复:``你傻呀?你跟着小丁猫他们跑啊!他们都走了,你还不走——你傻呀?''

丁小甜不看无心,只看苏桃。苏桃急得语无伦次,倒是让她又笑了。笑过之后她从腰间抽出了弹夹,一边上子弹一边收敛笑容说道:``我们的想法不一样。红总很快还会发动进攻,你快滚吧,不要给我添乱。''苏桃哭唧唧的还要劝,可丁小甜忽然变了脸,扭头对她吼道:``滚!不要蠢头蠢脑的烦人!''

无心把苏桃强行的往后拽。等到距离丁小甜足够远了,无心对苏桃说道:``别费劲了,她不能听你的。''苏桃想不通,对着无心嘀嘀咕咕:``小丁猫都跑了\ldots{}\ldots{}''无心忍饥挨饿的答道:``我看全文县的聪明人,只有陈大光一个。除了陈大光,其他的人全都多多少少的在发傻。''苏桃想了一想:``小丁猫也傻吗?''无心向她扬眉一笑:``你往后看吧!''

两人不敢再动,在瓦砾堆里趴了小半天,直到傍晚时分,才像两条野狗似的起了身,瘪着肚子往前溜达。招待所已经彻底被红总攻克了,正有红总的战士进了一楼餐厅找吃找喝。有吃饱喝足了的站在院内,仰着脑袋向楼顶张望。无心和苏桃试试探探的进了大院,有样学样的也抬了头,只见楼顶固定着一杆残破的联指旗帜,而下方跪着一个人,一动不动的搂着旗杆,正是丁小甜。

苏桃惊叫一声,随即抬手捂了嘴,不敢说自己认识联指分子。而无心问了旁边的观众:``楼顶上怎么还有人?''对方看了他一眼,认出他是陈司令身边的人,便大喇喇的答道:``死啦!死不悔改,有意思吧?人都往外逃,她往楼上冲,就为了保护她们那杆破旗。''话音落下,他举枪向上一扣扳机。丁小甜的尸体随着子弹的力道一跳,然而双臂死死的环住旗杆,硬是不倒。

苏桃垂下了头,由着无心把自己领入楼内。一根剥好的香肠送到她的手里,她略略清醒了,抬眼看无心也在吃香肠,才跟着张嘴啃了一小口。

到了入夜时分,持续了两日两夜的大武斗终于结束了。小丁猫和杜敢闯逃了个无影无踪,被俘虏的联指成员排成长队,被反绑双手关进了机械学院。大操场四周的探照灯彻夜雪亮,无心和苏桃跑去看了一次热闹,在无数黑沉沉的人影中,他看到了陈部长、李萌萌、还有武卫国。

陈部长的头脸被鲜血糊住了,歪着靠在李萌萌身上。几个月不见,李萌萌忽然长大了,看着比苏桃还成熟一点。她的脸上也缠着肮脏的绷带,绷带遮住了一只眼睛。武卫国横躺在人群中,两条腿被齐膝炸断,不知是死是活。无心不出声,轻轻走,走了一圈之后没有看到顾基,不知道顾基是死了,还是逃了。

带着苏桃进了一间空教室,他们在地面上躺了。苏桃枕着他的手臂,轻声问道:``是不是打完了?''无心叹了口气:``应该是打完了。''

\chapter{浪迹天涯}

大武斗结束之后,文县渐渐的恢复了平静,同时显出了一点儿劫后余生百废待兴的精神头。被炮弹炸成半截的楼房,修修补补的还得砌回原貌,被烈火烧过一遍的胡同,拆的拆建的建,也要重新拼成一串人家。副食品店又开始营业了,每天的顾客都能挤破了门,因为全被吓破了胆子,想要积攒粮草,为下一次战争做准备。

县革委会恢复了办公,陈大光自然还是说一不二的主任。大决战之后,他的威望达到了前所未有的高度,他的红总也随之成了毋庸置疑的革命组织。眼看他把文县慢慢的带回正轨了,上头顺水推舟,立刻把他树为典型、嘉奖了一番。

部下进行了论功行赏,他最后问无心:``你想要点什么?说吧!给你找个坐办公室的工作?''无心笑眯眯的答道:``我还看大门去。你要是念着我跟你一场,我就伸手跟你要点粮票吧!除了吃喝,别的我也不太在意。''陈大光听了,目瞪口呆,万没想到世上还有如此馋迷了心的笨蛋。

无心得了粮票和钱,马上带苏桃上街下了一次馆子。虽然饭馆黑洞洞的,服务员对人也是爱答不理,不过两个人对着一桌子的盘碗杯碟,还是吃了个欢天喜地。最后无心额外又要了两个馒头,和苏桃一人一个掰开了,用馒头去蹭盘子里的油吃。

他们都是太缺油水了,一天吃一锅窝头也还是饿。最后苏桃撑得坐不住,离开饭馆时须得微微弯着腰,扶着无心走路。秋老虎已经退了,风中略略透出了一点秋凉。苏桃走着走着,忽然问道:``无心,我们以后一直看大门吗?''无心压低声音答道:``桃桃,我想带你离开文县。''

苏桃好奇的看着他,等他的下文。无心继续说道:``在文县混了半年,真把我混怕了。现在小丁猫还没消息,谁知道他会不会卷土重来?趁着现在天下太平,火车站也开放了,我们做好准备,往外走吧!''苏桃想了想,自己笑了:``好,我们浪迹天涯去!''无心扭头对她做了鬼脸:``当盲流去!''

两人说说笑笑的往前走,迎面却是遇见了李萌萌和陈部长。联指分子落网之后,杀的杀关的关,另有一些毫无价值而又罪不至死的,则是被胡乱放了。陈部长的寡妇妈死了,自己头部受了重伤,白痴似的不知人事,李萌萌瞎了一只眼,倒是依然爱着陈部长,愿意继续照顾他。

和无心苏桃打了个照面之后,李萌萌面无表情的扶着陈部长走到了街道另一侧,对他们视而不见。而无心收回目光,心想李萌萌才十四,陈部长也未满二十——他们还都是孩子呢,但是人生已经毁在了自己制造出的战火里。

无心从李萌萌身上联想到了苏桃,苏桃也才十五。颇为悚然的握住了苏桃的手,无心越发的想要带她逃离文县。文县的青年都打野了心,如果再来一次战争,他们会更加的杀人不眨眼。

陈大光听说无心要走,深感莫名其妙:``为什么?''无心不敢说是怕他江山不稳,只答:``我和桃桃不是本地人,住久了就想回家。''陈大光上下打量着他:``回黑龙江啊?''无心点头答道:``对。''陈大光一皱眉头:``你家真在黑龙江吗?说老实话,其实我一直都感觉你来历不明。不过你别怕,闲事我不管。''

无心低头望着地面:``还有件事,想求你帮帮忙。''陈大光挺感兴趣的看着他:``说!''无心对着陈大光有一说一,听得陈大光啼笑皆非。无心也知道自己的话挺出奇,但是不说不行。顶着陈大光向日葵似的大笑脸,他咕噜咕噜的说了长长一串。最后陈大光哈哈笑了一阵,笑过之后告诉他:``行啊,我帮你了!''

陈大光发了话,让革委会的工作人员给无心和苏桃办了一套结婚证。结婚证一套两本,是大红的封面,翻开来第一页印着毛主席像和毛主席语录,第二页是正文了,把苏桃写成了二十岁,无心写成了二十三岁。

苏桃拿着结婚证看了半天,心里怦怦乱跳,嘴上问道:``无心,咱们算是\ldots{}\ldots{}结婚了吗?''无心拍了拍她的脑袋:``你才多大,结什么婚!有了它我们夜里就可以住一间屋了,白天在一起也没人拦着了,知不知道?''苏桃``哦''了一声,然后贼心不死的又问:``是不是得满二十岁了,才能领真正的结婚证?''

无心叹了口气:``想领真正的结婚证,也得有户口本和单位证明才行。麻烦着呢,好在都是将来的事,现在先不用想它。''苏桃的户口本是不能示人的,有了不如没有。抬头望向无心,她有点儿不好意思,感觉自己赖皮赖脸的问个没完:``你的户口本在家里吗?''

无心把两只手插到裤兜里,舌头在嘴里转了个圈:``桃桃,其实我没有户口,我\ldots{}\ldots{}我是个孤儿。''苏桃一下子心疼了他:``你怎么不早说呢?你比我还可怜。''无心笑出了一口白牙齿:``我已经长大了嘛!''

苏桃把结婚证放进了带着拉链的书包夹层里,心想我没有户口本,他也没有户口本,将来也是办不了结婚证的。手里的这一对红本本,怕是要用一辈子了。

结婚证刚刚放好,却又被白琉璃偷偷扯开拉链叼了出来。白琉璃还没有见过结婚证,十分好奇,也要瞧个新鲜,可惜没手没脚的,无法翻页。正是卷起结婚证胡乱揉搓之际,苏桃忽然发现了书包里的动静。打开书包向内一瞧,苏桃立刻就把白琉璃拎出来了。

苏桃一贯最爱白娘子,如今也忍不住在白娘子的圆脑袋上弹了一指头。把出了皱折的结婚证仔仔细细的压平整了,她咕咕哝哝的告诉白琉璃:``不许你再碰它了,这可是要命的东西,以后得用好几十年呢!''白琉璃不以为然,直条条的趴在床上。大开的窗户外面暮色苍茫,一个黑影左一闪右一闪,正是大猫头鹰在伺机寻觅他。及至无心叫苏桃去食堂吃饭了,大猫头鹰果然一头扎进房内,收拢翅膀落在了床上。白琉璃装死不理他。猫头鹰却是垂下头,把嘴里叼着的一只没毛小老鼠放到了他的面前。

白琉璃不吃白不吃,懒洋洋的张嘴吞了老鼠崽子。猫头鹰很高兴,用尖嘴轻轻啄了啄他的尾巴,又非常难听的叫了一声。白琉璃听了他的鬼哭狼嚎,登时烦得脱离蛇身,想要给他一点教训。

半小时后,无心和苏桃回到了收发室,一进门就发现了异常。苏桃弯腰去看地面:``哪里来了一地鸡毛?''无心看了白琉璃一眼,然后故意问道:``是鸡毛吗?不是有鸟飞进来了吧?''苏桃登时笑了,顺手拿起了笤帚:``那得是多大的鸟啊!你看这一地的毛,要是小鸟的话,非变成秃子不可。''

在苏桃扫地,无心喝水,白琉璃假寐之时,猫头鹰很孤独的站在收发室房顶上,用尖嘴整理自己一身乱七八糟的羽毛。白琉璃的怒气让他仿佛落在了冬天的龙卷风里,等到无处不在的冲击力消失之时,他发现自己已经没个鸟样了。

无心既然做了要走的打算,而且陈大光对他又是格外的好说话,他便厚着脸皮百般索要,给自己收拾出了一个结结实实的帆布旅行包。白琉璃听说他又要出发了,而且有苏桃同行,便很兴奋。这天夜里,他在无人处问无心:``接下来要去哪里?西南就不要去了,那些地方我都走过。你带我去东南看一看吧!''

无心手里拿着一只硕大的西红柿,一边吃一边答道:``你想得美!现在串联已经结束了,外面可没地方再让我白吃白喝白住了。我打算去东北,万一遇到了危险,也能进山躲一躲。''白琉璃很失望:``你要带我回家了?''

无心对着西红柿一口咬下,喷了满襟的汁水,连忙抬手去抹:``慢慢走,未必真的要回家。白琉璃,我不想带桃桃进山,我怕她在山里住久了,会变成野人。''白琉璃举头望明月:``做野人也不错啊,可以夏天看看花,冬天看看雪——''``呸!你自己都看腻了,还想哄别人陪你一起腻?总而言之,桃桃原来是好人家的女儿,我想把她的生活恢复原样。我不愿意让她跟着我混日子,更不愿意让她到山里干一辈子活、老了之后变成枯树精似的老婆子!''

白琉璃郑重其事的告诉无心:``你把她带到地堡里杀掉吧。我会保护她的灵魂。''无心听了白琉璃的高论,不禁有些头疼。吭哧吭哧的吃掉了西红柿,他没遮没掩的对着白琉璃打了个饱嗝,然后脱了汗衫走到水龙头前,一边去洗前襟的西红柿汁,一边说道:``白琉璃,你再敢胡说八道,我就不要你了。''白琉璃从天而降骑上他的脖子:``冻死你。''无心没吭声,因为他现在正在出汗,而脖子上的白琉璃好像一团凉阴阴的空气,真是让他舒服极了。

时光易逝,在过完了这一年的中秋节后,无心带着苏桃出发了。他背着一只双肩帆布包,苏桃挎着一只小书包。揣着陈大光开给他们的各种证明以及钞票粮票,他们在文县火车站挤上了火车。

上个月,中央发出了号召,让红卫兵小将们``就地闹革命'',使得一直持续着的串联活动宣告了结束。串联活动虽然结束了,但是出去的小将总要返乡,所以火车里面依旧是拥挤不堪。无心进入车厢之后,立刻变得十分烦人,在满车少男少女的叫骂声中强行硬是挤出两个座位。拎着衣领把苏桃扯到身边推到了靠窗的位子上,他随即也一屁股坐下了。

过道上的两个小红卫兵气呼呼的瞪着他——不要脸,那么大的人了,还抢他们的座位。苏桃在无心的遮挡下,不必面对小红卫兵如刀似剑的目光,心中倒是又兴奋又坦然。前方的第一站是辽宁,她还没去过辽宁呢!

比她更快乐的是白琉璃。白琉璃不挑地方,去哪里都可以,只要不回家就行。再说他在文县也住久了,如今一走,不但可以新开眼界,而且可以甩掉讨厌的猫头鹰,正是一举两得。从苏桃的手臂下面探出头,他对着车窗一吐信子,饶有兴味的欣赏窗外的秋日风光。

火车轰隆隆哐当当,一路越开越快,在山间的铁路上扭来扭去。车内的乘客们并不知晓在他们头顶上方,正有一只大猫头鹰背风而蹲,两只利爪死死的抓住了车顶末尾的小铁梯子。

\chapter{在路上}

苏桃守着无心的背包,缩着脖子坐在沈阳火车站内的候车室里。东北的秋天来得太快,说冷就冷。她记得自己从文县出发时还穿着一身单衣,如今在外面也没流浪多久,单衣却是已然换成了薄棉袄。

候车室门口的人群中挤进了一溜小跑的无心。无心双手捧着两只烤白薯,白薯刚出炉,烫得他几乎捧不住。苏桃连忙把旁边椅子上的背包抱到了怀里,而无心一屁股坐稳了,小声笑道:``快趁热吃,这两个烤得最好。''

苏桃接过一只烤白薯,掰出了一团又香又甜的热气,白薯的红瓤都快被烤成半融化的糖汁了,稀稀软软的要往下淌。她伸舌头舔了一口,食欲立刻蓬蓬勃勃的燃烧成了火:``真甜。''

无心被烫着了,张了嘴一伸舌头。而藏在他怀里的白琉璃从他的领口伸出了一个小脑袋,吐着信子向外看了看。天气一冷,小白蛇就有了要冬眠的趋势,白琉璃虽然精神永远焕发,可是既然此刻做了蛇,免不得就要受到自然规律的约束。眼看自己一天比一天懒怠动,他命令无心立刻设法拯救自己。无心没什么办法,只好给他换了个安身之处,让他从书包搬迁到了自己怀里。用一根长布条把他贴肉绑在自己身上,无心用自己的体温帮他过冬。

趁着旁人不注意,无心用手指头挖了一点烤白薯的红瓤,想要往白琉璃嘴里抹。白琉璃当即向下一躲,并不肯吃。

无心不理他了,转而去和苏桃说话:``桃桃,天太冷了,晚上带你去找家旅社住吧!''

苏桃啃着一块焦黑的白薯皮:``我还能忍一夜,明晚再住吧!''

无心望着她苦笑。自从踏上了北上的火车,仿佛出于女孩的天性一般,苏桃立刻就学会精打细算了。他们两个是明摆着的坐吃山空,全仗着手里的一点积蓄度日,所以苏桃能睡火车站就不睡旅社,吃烤白薯能吃饱就不吃正经饭菜。她无师自通的苛苦着自己,然而精神上很快乐,因为她的身心都自由了。

凭着陈大光开给他们的各种证明,他们暂时拥有了光明正大的合法身份。他们悄悄的游离在时代大潮之外,避开了无产阶级专政的铁拳。灰头土脸的赖在候车室里,苏桃用湿手帕擦了擦嘴角的黑灰,心中也有一点苍凉。如果真有家,谁愿在路上?

把手帕递给无心,她让无心也擦了手嘴,然后起身走去候车室一角的公用水龙头前,把手帕放在水流下搓了搓。

在候车室里又混了一夜,到了翌日上午,无心无论如何都要带苏桃去住旅社了。

两人在火车站外的小馆子里吃了热汤面,然后一起去逛大街。走过寒风萧瑟的红旗广场,他们看到了一座正处在施工中的巨型毛主席塑像。他们来的时间正好,沈阳城内的大武斗刚刚告一段落,市民生活也在逐步恢复正常。他们若是早到一两个月,正赶上武斗期间城里断粮,不要说热汤面,怕是连烤白薯都吃不上了。

苏桃已经走过了好几座城市,很是开了眼界。站在高高的脚手架下看了一会儿热闹,她忽然抬手一指:``无心,你看,猫头鹰又来了!''

无心仰起头嘿嘿的笑,一边笑一边把双臂环抱在胸前,勒了勒紧贴身的白琉璃。大猫头鹰正在空中盘旋,像个影子似的和他们若即若离。仿佛是知道自己不招人爱,大猫头鹰特别自觉,一路上只是偶尔亮相,绝不上头上脸的往他们身边凑。

苏桃把双手送到嘴边呵了一口热气:``无心,猫头鹰是不是认识我们,想和我们一起走?''

无心双手插兜:``这么大的猫头鹰,咱们没法带呀!让他自己飞去吧,他自在,咱们也省事。''

苏桃深以为然,跟着无心又走了一段路。最后在一处大众浴池附近,无心带着苏桃进了一家半大不小的旅社。进门之后见了服务员,无心开口说道:``农村包围城市,武装夺取政权。同志我想要间房。''

服务员打了个哈欠:``帝国主义都是纸老虎。拿介绍信!''

无心立刻翻出了陈大光开给他的介绍信,乖乖的送到了服务员面前的小桌子上:``没有调查就没有发言权。你看吧!''

服务员也不知道是有多犯困,一个哈欠接着一个哈欠,看过介绍信之后,她对着无心张嘴一露扁桃体:``没有正确的观点,就等于没有灵魂。结婚证呢?''

无心收起介绍信,拿出结婚证:``美帝国主义想打多久,我们就打多久。给你。''

服务员检查了结婚证,半闭着眼睛拿出一只大本子:``不打无准备之仗。你俩签字登记。''

登记完毕之后,无心和苏桃得到了一间小屋子。屋子里面倒是挺亮堂,左右靠墙各摆了一张小单人床。窗户下面的暖气管子已经颇有热度,苏桃高兴的脱了薄棉袄,露出里面一件火红火红的毛衣。毛衣是半个月前在本溪买的,虽然织得经纬稀疏粗枝大叶,但是没要票,价格也便宜。脱了鞋坐到床边,她伸长双脚去蹬暖气,又回头对无心笑:``脚都凉透了。''

无心也脱了棉袄,棉袄里面是一件泛了黄的衬衫。撩起衬衫解开贴身的布条,他把白琉璃放到了床上。一片阳光不知在床单上洒了多久,晒得床单暖烘烘。白琉璃惬意的盘起身体,仿佛受到了服务员的传染一样,也张开大嘴打了个哈欠。

无心伸展身体躺在了床上,酣畅淋漓的伸了个懒腰:``桃桃,一会儿我带你出去洗个澡。洗完澡了,我们买双棉鞋。''

苏桃蹬着暖气向后一仰,也躺下了:``又要花钱了。''

无心蜷起双腿,一双手打脚和白琉璃挤着分享阳光:``小守财奴,再由着你的话,我看你连吃喝都要省下了。''

苏桃枕着双臂,有点儿害羞:``舍不得嘛。''

然后她侧了脸去看对面床上的无心:``白娘子一个多礼拜没吃东西了,我们下午去给它买一小块肉好不好?''

无心点头应允:``好,天一冷,白娘子都没力气出去打野食了。''随即他一个鲤鱼打挺起身坐了,低头伸手拨弄白琉璃:``娘子啊娘子,你家的许仙怎么还不露面?你让他给我几斤肉票也是好的,没肉票我怎么给你买肉吃?''

白琉璃一动不动,决定目前姑且忍了,夜里再找板砖拍他。

无心装着一肚子热汤面,兴致很高,继续呶呶不止的撩闲,满嘴都是娘子许仙。白琉璃把嘴角向下一弯,心中暗暗骂道:``太贱了!''

无心快乐的耍贱完毕,转向苏桃畅想未来。下一站已经定好了是长春,无心想要趁着天气还暖,去长白山玩一玩。苏桃当即举了双手双脚赞成,袜子破了个大洞,亮出了她整个儿的脚后跟。

两人把一身的筋骨全都平躺着抻开了,肚里的热汤面也消化得差不多了,便统一的下床穿鞋。无心用个小塑料袋装了毛巾香皂,苏桃也翻出了一身干净的内衣。把苏桃一直送到旅社外的大众浴池门口,无心围着浴池开始溜达,一直溜达到苏桃焕然一新的走了出来。

毛巾香皂都是只有一份,所以苏桃洗过了,无心才能去洗。无心看苏桃的头发脸蛋都在冒热气,连忙把她带回了旅社——盲流可是没有资格生病的,所以苏桃万万不能感冒。

把苏桃送回房内安顿好了,无心才拎着小塑料袋去了浴池。苏桃一边晾着头发,一边整理了无心的帆布背包。忽然听到门外起了低低的敲门声,她以为是无心回来了。起身走去打开插销,她开门向外一望,面前却是一片空荡。正是狐疑的东张西望之际,不知是什么东西``呼''的蹭过了她的小腿。她低头一瞧,吓了一跳,原来是大猫头鹰从她的腿边挤进房里去了!

她下意识的抄起了立在门旁的秃头笤帚,虽然知道这大猫头鹰是只和善的动物,不过看着他的尖嘴利爪,心里还是隐隐的打怵。大猫头鹰站在地上,一个脑袋倏忽间向后转了一百八十度,苏桃看清了,发现他竟然叼着一条水淋淋的小鱼。

向苏桃展示了自己的猎物之后,大猫头鹰振翅飞到床上,开始去喂白琉璃吃鱼。白琉璃一张嘴就把小鱼吞了,猫头鹰拍着翅膀落到窗台上,雕塑似的一动不动了。

苏桃拿他没有办法,只好放了笤帚等无心回来拿主意,无心偏又久候不至。直过了两个多小时,无心才带着一身寒气进了门,手里拎着巴掌大的一块五花肉。

``没有肉票真不行。''他一边进门一边说话:``我为了这么点肉,快给卖肉的跪下了——''

话没说完,他一抬头看见了猫头鹰,当即惊讶的``哟''了一声。苏桃连忙告诉他:``猫头鹰给白娘子带了一条鱼吃。''

无心拎着肉走到窗前,抬手拍了拍猫头鹰的脑袋:``好孩子,谢谢你。''

猫头鹰没敢出声,生怕自己一叫,屋子里就要刮阴风。

无心在肉铺苦苦哀求,终于花高价买来了一点猪肉,没有浪费的道理。让苏桃自己上床休息了,他把白琉璃放到腿上,自己把猪肉咬成小块喂给他吃。白琉璃吃着猪肉,对猫头鹰是一眼不看。他对妖精向来没有兴趣,并且自视甚高,认为自己和一只猫头鹰没什么可说的。

手里剩下最后一块猪肉,无心把它喂给了猫头鹰,又告诉他道:``我知道你是好心肠,他不领情我领情。以后你常来,给他带点小鱼小虾老鼠蛤蟆什么的,他好养活,十天半个月喂一次就成。''

猫头鹰在窗台上横着挪了一小步,然后一扇翅膀飞到了床边。两只炯炯有神的大黑眼珠亮了一下,他知道自己以后可以公然的尾随他们了。得意的伸开一只翅膀盖在了无心的大腿上,他不假思索的想把白琉璃据为己有。然而白琉璃不耐烦的一昂头,对着他的身体就是一口。他吓得羽毛一乍,体积登时增大了一倍。翅膀也抬起来了,露出了白琉璃怒气冲冲的脑袋,以及一嘴灰扑扑的柔软羽毛。

无心最怕白琉璃发疯。乖乖的把猫头鹰撵出去了,他花了半个小时为白琉璃摘净嘴里的鸟毛。苏桃也在一旁帮忙,嘴里嘀嘀咕咕:``白娘子不喜欢它,你看它那大嘴像雕似的,多吓人啊。''

无心笑道:``别怕别怕,那鸟脾气挺好,就是个头太大。''

苏桃用手帕蹭去了白琉璃头顶的一点灰尘,低声抚慰他:``你别生气啊,无心已经把猫头鹰赶走了。''

无心弹开手指头上的一根鸟毛:``他都吃了人家的鱼,还好意思生气?桃桃,别擦了,趁着天亮上街去,咱们的棉鞋还没买呢!''

无心带着苏桃出去买鞋,白琉璃守着帆布包趴在床上,总算是得了片刻的宁静。

天擦黑时,无心和苏桃穿着棉鞋回来了。两人洗漱过后,各自占据了一张小床。因为明天就要买火车票去长春了,所以两个人很有话讲,一递一句聊个没完。说着说着又拐到了猫头鹰身上,无心开始拿着白琉璃和猫头鹰打趣,非说猫头鹰是许仙。

白琉璃本来盘在无心的被窝里,听到此处忍无可忍,悄悄的游下床去,要去投奔苏桃。无心的身上没有香味,手脚一动一动的不老实,而且满嘴屁话,句句气人。成功的攀上了苏桃的小床,他往对方的被窝里一钻,心中还在暗骂无心:``这个贱人,真是吵死了!''

到了第二天中午,买了票的无心苏桃,以及没有买票的白琉璃猫头鹰,各就各位的在火车内外找到了安身之处。经过了小半天的颠簸之后,他们在长春站下了火车。哪知长春并不比文县太平,火车站外皆是废墟瓦砾,遥遥的居然还有枪声。

无心和苏桃随着人潮往外走,出了站之后他们站住了,感觉情况不妙。在车上和他们对面而坐的乘客是个保定人,因为保定打得太厉害,局面彻底失控,所以吓得逃来东北避难。对着面前情景怔了片刻,保定人经验丰富的扭头进站,决定继续逃。

无心和苏桃也不傻,随便买了两张火车票,他们也换了方向。两个人漫无目的的游游荡荡,最终到达了长白山,他们却是在山下发现了一座奇妙的小村庄。

这个村子由几十个大小家庭组成,人口不少,但是不属于任何公社,在地图上也绝对找不到。因为它是由各地逃来的盲流组成的。其中有在老家吃不饱饭的穷苦人,有黑五类的狗崽子,还有一群戴着眼镜耍过笔杆的牛鬼蛇神。总而言之,全是为大时代所不容的分子。这一帮人陆陆续续的聚在了长白山下的大森林里,各显其能的从土地里刨食吃,也没人管他们。

无心没想到山里藏着这么一群人,周密的考虑了良久之后,他对苏桃说道:``天气越来越冷,我们不要走了,就在这里过冬吧!''

苏桃欢欢喜喜的看天看地,十分同意。

\chapter{山居生活}

盲流村里的大小盲流们,发现一夜之间村子里多了户人家。

村子里没有砖瓦,房子全是木头搭建成的,有个名字叫``木刻楞''。木刻楞要是讲究了,能用粗大原木建出小楼,不过盲流们显然无力讲究,有个木头房子遮风挡雨已然心满意足。在千姿百态的众多木刻楞边缘,很突兀的立着个尖尖的仙人柱,正是无心单枪匹马搭出的小帐篷。

目瞪口呆的村民们围住仙人柱,没想到还有比他们的木刻楞更简陋的房屋。冷不防仙人柱下帘子一掀,弯腰钻出了一个雪白脸子的青年。无心四面八方的点头微笑,又往几个小孩子手里塞了水果糖。一个戴着老花镜的老头子,文绉绉的做出评论:``你这个帐篷,很有游牧民族的风格。那个大兴安岭里的鄂温克人呀,就是像你这样\ldots{}\ldots{}''

没等老头子说完,老头子的小女儿跑过来了,说是家里没有了盐。老头子意识到自己的闲话不能调味,于是当即转身找盐去了。

全国的农村都公社化了,原始森林里的盲流村因为没人管,反倒是自种自得。黑土地肥的流油,只有肯出力,就绝对饿不死。如今到了秋冬之交,各家各户都多多少少的存了粮食预备过冬,唯有无心一无所有。苏桃坐在仙人柱里挖出的火塘前,一边烤老玉米一边忧心忡忡:``怎么办呢?我们现在种粮食也来不及了。''

无心一边翻动老玉米,一边满不在乎的答道:``来得及我也不种地。''

这是实话,他游手好闲的混惯了,让他本本分分的卖力气挣饭吃,他不耐烦。

苏桃心算了两个人的财产,然后就忧郁了:``那冬天我们吃什么呢?''

无心从火炭上捡起一根老玉米,双手倒着吹了吹:``我会打猎,你看四周都是林子,肯定够我打的。''

苏桃接过老玉米,一点一点的抠着玉米粒吃,虽然没有领会无心的意思,不过因为对他是无比的信任,所以也就不再多问。

到了下午,无心带着苏桃出门走了走,顺便昭告天下,表明了自己和苏桃的关系。村里的人见了苏桃,纷纷惊讶:``哟,真是个小媳妇。''

苏桃不好意思了,低着头不说话,无心则是不厌其烦的一遍遍重复:``其实我也是很年轻的,我们两个只差三岁。''

村民们当然承认无心的年轻,问题是苏桃年轻的过了火,根本还是一身的孩子气,像个正在成长的大丫头。众人看新鲜似的看着他们,看到最后都笑,认为小两口全很漂亮,倒是难得的相配,不知道他们生出的娃娃会有多美。

无心打听到了村里最有威望的领头人,特地带着苏桃过去坐了一坐,又送了一斤白糖做见面礼,算是取得了对方的认可。出了村子进了林子,苏桃双手扶着一棵树干仰头去看树冠,老树不知已经活了几百年,树冠是名副其实的高耸入云。苏桃深深的吸了一口气,心里生出了一种险伶伶的兴奋——只要有无心,自己也是在哪里都能活的。

她转身要找无心说话,不料扭头一瞧,她发现无心从地上捡起了一根比手腕略细些的桦树枝。手持匕首将树枝一端削尖,无心向上一扬头:``看见没有?到处都是松鼠,哪棵树上都有鸟窝。''

然后他收起匕首,开始去解棉衣。苏桃愣头愣脑的旁观片刻,忽然反应过来了,连忙去拢他的前襟:``天这么冷,你还脱衣服?''

无心拨开她的手:``你不懂,乖乖站在树下等着我吧。''

他把棉衣棉裤棉鞋尽数脱掉,把紧贴身的白琉璃也抻出来埋在了带着体温的棉袄里。扭扭脖子晃晃肩膀,他看准一棵老树纵身一跃,苏桃只觉眼前一花,他已经光着双脚上树了。

一手握着那根削尖了的桦树枝,一手搂抱着老树干,无心摇头摆尾,转眼间就爬到了高处。隔着稀稀疏疏的黄叶,苏桃就见他停在一处枝杈上,忽然一动不动了。

苏桃都要急死了,不知道他在上面是个什么情况,有心喊一嗓子,又不知道能不能喊,该不该喊。万一自己一嗓子吓着了他,罪过就大了。心急如焚的等了又等,就在她忍无可忍之时,无心在上面忽然动了一下。随即一个小灰影子在树枝间磕磕绊绊的坠下,最后``啪嗒''一声落在了她的面前,乃是一只脖子被扎穿了的大松鼠。大松鼠躺在落叶堆上抽搐不止,看得苏桃一阵心疼。可是没等她心疼过劲,头上又有猎物落下来了。

这回的猎物是一只带着黑色条纹的桦鼠,过冬前的动物都吃得足,这桦鼠足有小兔子大,肥得圆滚滚,也是脖子受了致命伤。一群黄叶簌簌而落,苏桃向上再望,就见无心握着染血的桦树枝,轻轻巧巧的滑下来了。落地之后转向苏桃,他鼓着两腮一低头,向手心里吐出两颗大鸟蛋。

``窝里一共五个蛋,我拿了两个,给老鸟留三个。''他告诉苏桃:``老鸟不识数,不能发现。''

苏桃对老鸟没兴趣,慌忙抖开棉衣要给他披。然而无心把鸟蛋交到她的手里,迈开步子向前跑了几步,又上树了。

如此忙了一下午,末了在暮色苍茫之时,无心一手领着苏桃,一手拎着四条毛茸茸的尾巴,尾巴吊着大头冲下的死松鼠,鲜血滴滴答答的流了一路。苏桃单手插了兜,兜里是暖烘烘的五枚鸟蛋,鸟蛋品种不一,有的大一点儿,有的小一点儿。

当天晚上,无心和苏桃围坐在火塘边吃晚饭。他们的日用品十分匮乏,容器除了水壶之外,只有一只大饭盒,以及进山前买的一只小搪瓷盆。无心暖了一盆水,让苏桃洗了手脸,同时把饭盒放在火上,用四个鸟蛋做了一盒蛋花汤。余下一个最大的,被他塞进了白琉璃的嘴里。

松鼠肉烤好了,他撕了一小块递给苏桃。苏桃没吃过松鼠,迟迟疑疑的不肯接。无心转而把肉送到了她的嘴边,自己张大嘴巴:``啊\ldots{}\ldots{}''

苏桃笑了,也跟着他``啊'',嘴刚一张,松鼠肉就被无心的手指塞进去了。无心随之一舔嘴唇:``尝尝,比什么肉都好吃。''

苏桃闭了嘴慢慢咀嚼,吞咽之后笑了:``是挺好吃。''

无心得意的又一舔手指上的油。火光自下而上的照耀着他,他成了个眉飞色舞的大男孩子,有着金红色的光滑皮肤和流光溢彩的黑眼睛。苏桃快乐的又咬了一口松鼠肉,心里喜欢死他了。

吃饱喝足之后,无心躺在火塘旁的一块帆布上,伸了手臂给苏桃做枕头。苏桃仰面朝天的向上看,能从仙人柱顶端的圆孔中看到星星。苏桃不明白无心为什么要把帐篷围成一把大伞的形状,也不明白伞顶为什么还要留个孔洞。不明白的太多了,她懒得一样一样细细的想,反正至少无心是明白的。

孔洞上方出现了一双亮闪闪的大眼睛,鬼头鬼脑的向下窥视。苏桃正要去推身边的无心,可是未等她出手,闭着眼睛的无心忽然吹了一声口哨。

大眼睛立刻消失了。曳地的帆布门帘却是动了一角。大猫头鹰探头探脑的钻了进来。无心从苏桃的脑袋下抽出手臂,坐起身把大猫头鹰抱到了怀里:``嘿嘿,就知道你丢不了!''他在大猫头鹰身上捏了捏:``肚子这么大,你又吃什么好的了?''

大猫头鹰不会说话,一肚子的花开不出来,只好低低的``嗥''了一声。白琉璃最听不得他怪叫,在地铺上猛一昂头,他对着大猫头鹰怒目而视。大猫头鹰吓了一跳,立刻耸起两只大翅膀捂住了尖嘴,一双大眼睛战战兢兢的看看无心,又看看苏桃,最后才偷偷摸摸的看了白琉璃一眼。

无心不理他的小心思。一歪身又躺下了,他很舒服的唠唠叨叨:``吃独食的,你吃饱了,倒也给我们带一点呀!你看你这大嘴大爪子大翅膀,抓个兔子抓个山鸡,还不像玩似的?''

然后他侧身把胳膊又伸出去了:``来,桃桃,你把手往他翅膀下面放,可暖和了。''

苏桃小心翼翼的把手搭上了猫头鹰的身体,发现猫头鹰看着威武,其实一身软毛。而猫头鹰是个绵羊脾气,活了一百来年,从来不知道什么叫做生气。挣扎着爬起来蹲在无心和苏桃之间,他睁一只眼闭一只眼,在周遭似有似无的阴气中,他忽然打了个冷战,周身羽毛中逸出一股子黑气。无心和苏桃都睡了,白琉璃藏在无心怀里也睡了,猫头鹰很孤独的变成了一个小男孩,两边腋下还分别夹着无心和苏桃的手。脚趾头抓了抓地,他感觉自己在平地上有点儿蹲不住。所以赶在惊动旁人之前,他又变回了猫头鹰的形象。

一夜过后,天光大亮。无心和苏桃还在火塘边睡眼惺忪的迷糊着,忽听帐篷外面起了一阵大骂。苏桃吓了一跳,登时瞪圆了眼睛。而无心起身一掀帘子,猫着腰钻出去了:``怎么回事?''

来人是个四十来岁的汉子,伸手指着帐篷外的一地细骨头怒骂:``我说你俩可是够不地道的,初来乍到就敢偷东西吃!''

无心反问:``我偷什么了?''

汉子大声吼道:``你还有脸问我?你看看地上是什么?你敢说你没偷我家的鸡?''

无心一点头:``我敢。你给我弯腰看清楚了,那是鸡骨头还是松鼠骨头。''说完他转身绕到帐篷后面,拎来了四条扒下的松鼠皮。抡起毛皮直抽到了对方脸上,他气势汹汹的逼问:``再看看这他妈的是什么皮?是鸡皮还是松鼠皮?''

他占着道理咄咄逼人,汉子的气焰立刻就落了。口中支吾几句,汉子落败而走。而无心转身回了帐篷,只见苏桃脸上颜色不定,身边还蹲着大猫头鹰。

无心把松鼠皮放在火塘边,开始琢磨着如何用皮子换钱。苏桃小声开了口:``他们怀疑我们是小偷吗?''

无心一摆手:``别管他们,可能看咱们是新来的人,想要讹诈一笔。''

苏桃抱着膝盖:``他们也欺负人啊?''

无心伸手摸了摸她的后脑勺:``没事,有我呢。要论单打独斗,我可谁都不怕。''

如此到了中午,两个人把存粮全部吃光了,准备再去林中打猎,出了帐篷才知道村里真闹了贼,昨夜全村一共丢了七八只大肥鸡。这年头油水匮乏,丢了鸡也能疼出人的血来,所以村中人心惶惶,养了家禽的全都加固围栏,生怕今夜再受损失。

\chapter{吃饱喝足}

无心用一根草绳把大猫头鹰捆绑住了,然后蹲在他的面前小声质问:``你说实话,村里的鸡是不是你吃的?''大猫头鹰一直过着鳏寡孤独的生活,导致交流能力有所退化。听了无心的问题,他愣了好一阵子才反应过来。一个脑袋向左一扭,再向右一扭,他摇头表示否认。

无心伸手摸了摸他的大肚子:``真不是你?''大猫头鹰怔呵呵的继续摇头,把无心摇糊涂了:``到底是不是你?''大猫头鹰有点儿傻眼,感觉自己是把头摇错了。睁着圆圆的大眼睛望向无心,他的尖嘴上下点了一点。

无心更糊涂了:``什么意思?是你就点点头,不是你就摇摇头。''大猫头鹰的黑眼睛里汪了一层泪光,楚楚可怜的一摇头。苏桃坐在一旁开了腔:``我看也不能是它。不是说丢了七八只鸡吗?它哪吃得了那么多?''

白琉璃从无心的领口中探出了头,因为感觉帐篷里的温度还不算低,便在无心的脖子上缠了一圈,瞪着两只黑豆眼睛审视大猫头鹰。大猫头鹰一见他露面了,一双湿润的眼睛越发勾魂摄魄。可惜白琉璃并不欣赏他的风采,十分冷漠的又缩回了无心怀中。

无心和猫头鹰交情尚浅,所以对于猫头鹰的回答,也是半信半疑。抬手捏住他的尖嘴,无心把眼睛瞪得比他还大,做出凶恶的神情恐吓他:``我们好不容易才找到了地方安身过冬,你要是敢连累了我们,我就把你吃了!''

猫头鹰吓得直挺挺的,从喉咙里``嗥''了一声,表示明白。无心松手解开了他身上的麻绳:``好啦,姑且信你一次。''猫头鹰得了自由,挪到角落里一缩脖子,缩成了一截短粗胖的灰色桩子。而苏桃把火塘边的厚鞋垫递给了无心:``我们别管闲事了,晚上的饭还没有着落呢。''

无心脱了棉鞋,把烤热的鞋垫放进鞋里:``其实我是怕有大野兽。大野兽吃鸡吃顺了嘴,非得常来不可,我们的小帐篷可禁不住大家伙拱。''说完这话,他抱着大猫头鹰起了身:``你也跟我们一起走吧。''

无心带着帐篷里所有的活物进了林子,直到傍晚才归。他一手领着苏桃,一手拎着草绳,草绳下面绑着沉甸甸的两只大肥野兔,猫头鹰却是不知所踪。

村庄里面已是炊烟袅袅,木刻楞的面积有限,许多人家没有厨房,在门口拢一堆火就能开伙。无心和苏桃也在帐篷外面搭了个石头灶。苏桃蹲在地上呼呼的吹旺火苗,无心则是快手快脚的在一旁剥兔子皮。兔子太肥了,一身厚厚的脂油。无心用匕首把脂油全刮进了饭盒里,余下的肉则是切成小块,用树枝穿了架在火上烤。

种地是力气活,打猎是技术活,来自五湖四海的村民们经过了半年的劳作,家家都能吃上棒子面,想要开荤却是难得。兔肉的香气弥漫开来,把人勾得直冒心火。一个小孩子跌跌撞撞的跑过来了,咬着一个手指头看着半熟的兔肉发呆。苏桃被小孩子盯得很不自在,想要给那孩子一块,然而无心轻轻一摁她的手,守株待兔似的不肯动。

片刻之后,一个老婆子拿着个热窝头追过来了,千哄万逗的要带孩子回家。孩子半蹲着身体死活不走,老婆子拽得狠了,小孩子便``哇''的一声开始嚎哭。无心这时才一团和气的扭头问道:``大婶,你家有白面吗?我这儿还有一只五六斤的兔子,野兔子,肥得像小猪似的,三斤白面就换。''

若从交换的角度来看,三斤白面换一只五六斤的兔子,绝对是合算;但是人人赤贫,三斤白面也不是能够轻易付出的。老婆子很为难的牵着孩子:``我给你三斤棒子面吧?好棒子面,今年新磨的。''无心笑呵呵的一摆手:``不要棒子面。''

然后他自顾自的去翻动架在火上的兔子肉。一味的以肉为食也是不行的,日子久了会营养不平衡。赛维最通这些理论,当年时常教导无心。无心被她吵得泼烦,烦着烦着却也都记住了。其实他自己是吃什么都无所谓的,但是很想给苏桃弄点粮食。粗粮就是粗粮,没有细米白面养人,所以他务必要给苏桃弄点儿好的。

老婆子家里没养鸡鸭,想拿谎话对孩子许愿都不能够。费了不少劲把孩子弄走了,片刻之后,她带着两斤白面和一斤棒子面回了来,想要换兔子。无心没挑剔,很痛快的把兔子给了老婆子。随即立刻和了一团白面,他笑眯眯的告诉苏桃:``给你烙饼吃吧。''

苏桃高兴极了:``往里搀点儿棒子面,只吃白面太浪费了。''无心固执的摇头:``不,我们穿没好穿、住没好住,就剩下一个吃了,还不吃点儿好的?你去把白糖拿过来,我们烙糖饼。''

苏桃在无心面前素来是没主意的,在理智上,她认为无心应该把白面和棒子面混起来吃;可在感情上,她对棒子面也是腻歪透了。狂欢似的钻进帐篷,她欢天喜地的拿出了白糖。

无心把兔子身上的脂肪炼成了油,又用这油烙了糖饼。因为是用饭盒当锅,导致糖饼也是四四方方,带着一点兔肉的膻味。吃饱喝足了的闲人们觅着香味过来看热闹,统一的认为这小两口太不会过日子,又有人问无心:``你从哪儿弄的兔子?''

无心吃饱了,喝着热水笑道:``偷的。''当然没有人信:``在谁家偷的?''无心答道:``在兔子家偷的,偷了兔子它爸它妈。''众人哄堂一笑,同时感觉新来的小白脸子够奸的,身手灵活能抓兔子,心眼灵活能换白面,说起话来避重就轻,不是个老实东西。

饭盒里残留着油和糖,苏桃加了点水进去,煮了一盒棒子面粥预备做明天的早餐。两只耳朵听着无心和村民闲聊,她并没有加入谈话的欲望。和无心相处得越久,她对于外人的兴趣就越淡。把煮好的棒子面粥端到地上,她听无心询问周遭众人:``明天谁还想吃肉?想吃的话就提前告诉我。无论是大米白面还是干菜咸菜,我都肯换。''

一个十几岁的半大孩子一边偷眼瞧着苏桃,一边大声说道:``哥,你明天也带我一个吧,我向你学习学习。我也不要兔子它爸它妈,能弄个兔崽子解解馋就行。''

无心笑模笑样的,倒是很好说话:``行,带你就带你。可是我只管带,不管教。''半大孩子乐了:``嫂子也去吗?''无心看出这个半大孩子和自家的半大丫头年龄相仿,当即起了提防:``她明天不去。''半大孩子暗暗的失望了:``哦\ldots{}\ldots{}''

等到夜色深了,众人各自散去休息。猫头鹰也悄悄的回来了,嘴里叼着一只没毛的小田鼠。无心在林子里让他帮忙抓野兔,可他见野兔和自己体积相仿,而且四肢有力,一腿能把自己蹬个跟头,心中立刻生出了以和为贵的思想。趁着无心不注意,他扇着大翅膀悄悄溜了。

臊眉沓眼的挤进帐篷,他要把田鼠喂给白琉璃吃。无心不和他一般见识,把白琉璃从棉袄下摆抻出来放在火塘边,他甩着闲话躺下了:``你啊,白长这么大的个头了。看着像只大雕似的,其实是个麻雀胆子。桃桃,过来睡觉。''

苏桃枕着无心的胳膊躺下了,帐篷看着单薄,其实真能挡风,把火塘的热量圈了个严实。无心用手背蹭了蹭她的脸蛋,发现苏桃的皮肤变粗糙了。苏桃也意识到了自己的变化,但是没太往心里去。现在不是个以美为美的年头,她也没到很把姿色当回事的年龄。她吃了一顿烤肉和糖饼,现在躺在无心的身边,已经满足到了无以复加的程度。

一夜过后,半大孩子早早的来到帐篷前,等候无心出猎。无心没有惊动苏桃,自己在外面生火热了饭盒里的棒子面粥。呼噜呼噜的喝了几口,他转身钻回帐篷说道:``桃桃,你睡吧,我走了。''苏桃睡眼惺忪的坐起来,嘴里答应了一声。

天气越发冷了,余下的半饭盒棒子面粥还放在帐篷外的石头灶上。苏桃出了帐篷,见周围也没有人,便端着个小水杯到帐篷后面刷牙。刚刚漱净了嘴里的牙膏沫子,她忽然听到身后起了``哧溜哧溜''的声音,像是有人正在自家门前吃喝。忽然想起了灶上的饭盒,她贴着帐篷慢慢的绕向前方。末了伸着脑袋一瞧,她大惊失色,只见一只火红火红的大狐狸蹲在石头灶前,正在偷喝自己的棒子面粥!

苏桃傻了眼——如果偷粥的是只灰狼,她或许能更有主意,直接撒腿逃跑;可狐狸到底值不值得一怕,她含含糊糊的不能确定。眼看搀着油和糖的棒子面粥越来越少了,她心中一急,弯腰把一只手从帆布帐篷的下方往里伸,手指头在地上划了几下,她摸索着先抓住了一只爪子,正是猫头鹰蹲在角落里睡大觉。放开爪子继续摸,她再没找到武器,于是缩回了手,就地捡起一根短粗的枯树枝。蹑手蹑脚走近石头灶,她呐喊一声冲上前去,对着狐狸脑袋就是一棒。

狐狸正在低头喝粥,猝不及防挨了一下子,当即一嘴扎进饭盒。而苏桃随即放声高喊:``快来人哪!有狐狸呀!''狐狸把头一抬,棒子面粥顺着嘴边的胡子往下滴答。对着苏桃一呲牙,它正要做出恶行恶相,不料附近村民已经闻声而来。狐狸最是狡猾,见状不妙,它扭头便逃,三步两步的窜进林子里不见了。

这回村子里的偷鸡案可以马马虎虎的告破了。被人看见的是一只狐狸,没被人看见的,不知还有几只。一大群狐狸进了村,鸡鸭可不是要遭殃?苏桃撅着嘴去刷饭盒——甜甜的棒子面粥,无心都没舍得多喝,专给她留着的,结果一眼没照顾到,全便宜狐狸了。

\chapter{傻狐狸精}

半大孩子亦步亦趋的跟着无心进了林子,满拟着能学几手做套下绊挖陷阱的巧本事,不料无心走兽似的埋伏在草丛里,竟然是手嘴并用的生擒活捉,比野物还野。半大孩子没见过他这么快的身手,紧赶慢赶的撵着他跑,差点跑丢了一只棉鞋。

最后到了中午时分,无心扛着一根粗树枝走出了林子,树枝一端削尖了,血淋淋的插着三只大肥兔子。半大孩子白白浪费了一上午的时间和体力,肚里的窝头消耗得丝毫不剩。眼巴巴的望着无心的猎物,他把一根脏兮兮的手指头塞进了嘴里。

无心头也不回的叫了他的名字:``小全,想不想吃兔子肉?''小全累得气息都弱了,垂涎三尺的低声答道:``想呗!''无心回头对他诡谲一笑:``你家不是有干黄花菜吗?拿黄花菜换。''小全咽了口唾沫:``我妈不能让我拿。''无心转向了前方:``那没办法,我回家吃我的兔子肉,你回家吃你的黄花菜吧!''

小全苦着一张脸,赖唧唧的尾随着他,知道自己想吃兔子肉的话,必须得冒险回家偷黄花菜,可他只想吃肉,不想偷菜。不知不觉跟到了小帐篷前,他一看到苏桃正在帐篷外面干杂活,立刻正了正眉目神情——在无心面前他是个小孩,在苏桃面前,他不由自主的想要做出小男子汉的模样了。

他不再纠缠无心,悄无声息的往家走。而苏桃完全没有留意到他的存在,径直跑到无心面前说道:``今天早上来了一条狐狸,偷喝咱们家的棒子面粥!''此言一出,附近木刻楞开了门,走出一个面黄肌瘦的大媳妇:``狐狸最奸了,肯定是上次偷鸡偷顺了嘴,昨天夜里就又进村了。无心,你那兔子是咋抓的?''

无心一手扶着肩上的粗树枝,一手叉着腰:``狐狸还喝棒子面粥?嫂子,兔子是我用手抓的。''大媳妇十分惊讶:``用手抓的?那你这手得多快?''无心一本正经的告诉她:``可快了。''
无心不回来,苏桃就觉得自己一个人没有开伙做饭的必要,生生饿了大半天。无心快手快脚的烤了一小块兔子肉给她,又问:``你说你是不是懒?我要是一天不回来,你就饿一整天?''苏桃用牙齿撕着兔子肉,烫的嘶嘶吸气:``我才不是懒呢!''

无心动作娴熟的扒下整张兔子皮,又把兔子开膛破肚清洗干净,切成小块晾在石头灶旁。附近的木刻楞又开门了,大媳妇伸出脑袋问道:``哎,无心,你那兔子肉是不是能用面换?''无心大声答道:``能!但是不要棒子面。''随即他站起了身:``嫂子,你家有没有用不上的锅?有的话就借我一口,我给你一只兔子,明年一开春我还把锅还给你。''

大媳妇一听,一扭身回了房,不过片刻的工夫,便拎出了一口小小的旧铁锅。把铁锅送到无心面前,大媳妇用海碗盛着满满一大碗兔子肉回家了。大媳妇刚走,小全用衣襟兜着一大包黄花菜回来了,眼看大媳妇端走了一大碗粉红的兔子肉,他吓得连忙去问无心:``还有吗?你没全给她吧?''

无心一丝不苟的清点了黄花菜的数量,然后剁了半只兔子给他。把小全打发走之后,他转身对着苏桃做了个鬼脸,又从衣兜里掏出了三枚大鸟蛋对着她一晃:``怎么吃?''苏桃想了一想,高兴的答道:``做疙瘩汤吃吧!''

苏桃找出面粉,张张罗罗的要给无心做午饭,然而刚一动手便显出了人仰马翻的趋势。无心连忙强行接管了她的事业。慢慢的用水调开面粉,因为面粉太可贵,所以无心慢条斯理,干得细致,又问苏桃:``看了狐狸怕不怕?''苏桃在一旁泡黄花菜:``不怕,我还给了它一棒子,把它打跑了。''

无心问道:``猫头鹰没帮忙?''苏桃听了,啼笑皆非:``它又不是看门狗,哪能帮我的忙?''无心骂了一句,意思是说猫头鹰是个吃货。猫头鹰在帐篷里似睡非睡,很偶然的听到了无心的批评,当即吓了一跳,六神无主的横着挪来挪去,两只爪子抓不住地,差点向后摔了个仰面朝天。

帐篷外面涌起了血腥气,他想定是无心在对着野兔子们大开杀戒。战战兢兢的展了展翅膀,他决定先行逃走,等到风头过了再回来。运足力气一振翅膀,他平地起飞冲向帐篷帘子。不料一个脑袋刚刚见了天日,蹲在帐篷前的无心猛然回身出拳,口中同时大喝一声:``哈!''

这一拳正好击在了猫头鹰的头顶,猫头鹰只觉一阵天翻地覆,待到他恢复清醒之时,外面石头灶上的疙瘩汤已经开了锅。面汤嫩嘟嘟的一颤一颤,里面煮着黄花菜和荷包鸟蛋。无心和苏桃围着石头灶席地而坐,直接用勺子对着锅吃。面汤太烫了,两人在冬日的太阳下面喝出了热汗。最后无心叹了一口气:``舒服。''

苏桃用勺子刮着铁锅:``还剩了一碗,晚上吃吧。''无心刚要说话,不料远方忽有一人急急跑来,却是前天早上污蔑无心偷鸡的汉子。那汉子生得五短三粗,本也有着几分威武样貌,然而此刻却是举止异常,夹着两条腿一路扭得飞快,一路分花拂柳的就飘过来了。

在距离小帐篷十步远的地方站住了,这汉子伸出两只大巴掌做了个兰花指,双双指向苏桃,口中尖声尖气的开始大骂,语言极其下流。苏桃端着一碗面汤愣住了,无心也扭头望向了他——望了没有几秒钟,无心起身绕过石头灶,弯腰一把捂住了苏桃的耳朵,同时就听汉子跳着脚的叫道:``你个不是人养的没汉子要的小骚×,姑奶奶喝你一口棒子面粥还要挨打,妈的姑奶奶今天非扯腿撕了你不可!''

周围的木刻楞全开了门,有见多识广的老人家开了口:``哎呀,你们听这不是王木匠的声音啊!王木匠这是怎么了?''无心紧紧的捂着苏桃的耳朵,站在原地腾不出手。王木匠骂得太牙碜了,最老辣的泼妇听了也要面红耳赤。他不允许这些脏话往苏桃的耳朵里进,一句也不行,只言片语也不行。

一个小脚老太太拿着一只大竹筐,东倒西歪的挪上去扣上了王木匠的脑袋。其余人等一拥而上摁住了他,其中一个老头子慌慌的从家里拿来一根马鞭子,抡圆了去抽王木匠头上的竹筐,一边抽一边骂:``你个狐狸精,你偷吃的你还有理了?你给我滚,马上滚,不滚打死你!''

马鞭子噼里啪啦的抽在竹筐上,带着呼呼的风声,听着颇有威慑力。王木匠渐渐的不挣扎了,然而脑袋在竹筐里继续哼哼唧唧的做女人呻吟。老头子抽了一身的大汗,末了喘息着停了鞭子,询问周遭众人:``你们说咋办?你们听他刚才说的那话,他不就是让早上那条大狐狸上身了嘛!''然后他转向了苏桃:``那个小丫头,是不是你早上给了它一棒子?''

苏桃还端着一碗面汤,彻底傻了眼,并且依旧被无心捂着耳朵。无心替她答道:``没错,是我们打的。''王木匠的老婆此时闻声赶来了,哭天抢地的扑向了无心:``你说你们招惹了狐狸精,怎么就连累到了我家木匠?你们得救他,他要是有个三长两短,我和你们两个小兔崽子没完!''

无心没理他,放开苏桃走到人群之中,接过鞭子继续去抽竹筐。一边抽一边又问:``谁是童子?快点脱了裤子对他撒尿!''众人这时才想起童子尿的功效类似黑狗血。看热闹的大小男孩全挤上来了,又惊恐又兴奋的解开裤子掏出家伙,对着王木匠就开始撒尿。王木匠的老婆又不干了,赶上来对着无心的后背连捶带打:``你个坏小子干啥呢?你支使小犊子们往我家木匠身上撒尿?''

无心顶住了她的攻击,低头问道:``王木匠,你清醒了没有?清醒了就回答一声!''王木匠的脑袋窝在竹筐里,一丝两气的不出声。无心拎着鞭子长叹一声:``唉,我听说大粪也能辟邪!''此言一出,虚弱的王木匠立刻拼命挣出了一声:``哎哟\ldots{}\ldots{}我好些了\ldots{}\ldots{}''

旁边的老头子一挑大拇指:``还是人家小伙子阳气足办法多,你看,一下子就把狐狸精打跑了!''王木匠的老婆扶起了一身臊的王木匠,嘟嘟囔囔的往家走。王木匠一走,其余观众也四散回家了。无心转身走到苏桃面前,低头向她一笑:``没事了。''苏桃站起了身,怯生生的问道:``无心,真有狐狸精吗?''无心微微俯下身,在她耳边说道:``有我在呢,不怕。''

苏桃望着无心的眼睛,一时忽然不知应该从何问起。迟迟疑疑的垂下头,她感觉自己是闯了大祸:``我早上不打它就好了\ldots{}\ldots{}''她扁了扁嘴,不知道自己该不该哭:``我不知道它那么厉害\ldots{}\ldots{}''

无心摸了摸她的脑袋:``它厉害个屁!它除了骂街还会干什么?几泡尿就把它浇跑了。这么个东西,也值得你怕?我告诉你,妖精也是分出三六九等的,你看它那副惨样,连棒子面粥都偷,混得还不如个盲流,它有什么可怕的?''苏桃被他说的笑了,自己抬袖子一抹眼睛:``对啊,它还是本地狐狸精呢!''

无心又拍了拍她的肩膀:``它都不如我们的猫头鹰体面。''苏桃心服口服了,小声嘀咕道:``猫头鹰从来都不偷嘴,还给白琉璃捉小田鼠吃。无心,你以后别打它了,它多好啊。''无心立刻钻进帐篷,抱着大猫头鹰亲了个嘴:``么——啊!''苏桃掀帘子看见了,忍不住笑出了声。而大猫头鹰怔怔的缩着翅膀,以为无心要把自己吃掉了。

下午太太平平的过了去,入夜之后,无心照例是带着苏桃在火塘边睡觉。猫头鹰则是彻底恢复了昼伏夜出的习性,溜出帐篷前去打猎。到了午夜时分,无心钻出帐篷撒尿,忽见白琉璃脱离了蛇身,东张西望的悬在了自己面前。

无心自顾自的打了个哈欠,然后轻声说道:``白琉璃,今天中午你应该出手帮我。你把那个小狐狸精赶走,王木匠就不会惊动那么多人了。''白琉璃心不在焉的答道:``那狐狸精像个傻瓜一样,我对它没有兴趣。''

无心系好了裤子:``这地方太荒凉,我在林子里面总能感觉到妖气,真怕小狐狸精会引来大狐狸精。白琉璃你不要飘了,你回到我这里来睡觉。这地方可没有人武斗给你看,你飘也白飘。''

\chapter{狐狸要报仇}

深山老林的偏僻村庄里,鬼狐精怪的故事最多。要是放到过去,找个跳大神的禳治禳治也就罢了,几乎不算了不得的大事。如今王木匠既然已经恢复正常,村民又联想起了狐狸精的所作所为,不由得将其当成一段笑谈,并不十分恐慌。

无心等了几天,不见狐狸精前来报仇,便略略的放了心。将这些日子积攒的灰鼠皮野兔皮用草木灰烧了烧,他潦潦草草的熟了一堆皮子。粗枝大叶的用针线把皮子连缀成一大张,他用它围了帐篷。这是他在大兴安岭向当地的通古斯人学得的方法,通古斯人的帐篷披上一层兽皮,冬天就可以不惧风雪了。

余下还有几张皮子,被他东拼西凑的做成了一张褥子。反正开春之后还是要走的,他和苏桃都无意去认真的建设家园。夜里两人躺在火塘边的兽皮上,苏桃枕着无心的手臂,仰面朝天的去看星星。细雪通过帐篷顶端的圆孔飘下来,融化在了火塘上方的温暖空气中。无心的声音低低的响在耳边,是他在给她讲故事。故事里面全都是山魈鬼魅,正配合着外面鬼哭狼嚎的风声。白琉璃从无心的领口中探出了脑袋,跟着苏桃一起听。

``最后,那位了不起的**师在胜利之后,就一个人下山去了。''他的气息轻轻扑上了苏桃的面颊,微弱的断断续续着。

苏桃好奇的扭头看他:``**师去哪里了?''

无心想了想,然后告诉她:``我也不知道,故事到这里就结束了。''

苏桃从他的黑眼睛里看到了火塘中暗红的光。他的眼睛真亮,闪烁了映在他眼中的光芒。她出了神,一直盯着他看,直到他抬手拂开了她脸上的凌乱碎发。

``明天烧壶热水,给你洗洗头发。''无心忽然说。

苏桃轻声开了口:``无心,你对我真好。''

白琉璃登时来了精神,睁着两只黑豆眼睛拼了命的倾听。然而无心并没有顺着苏桃的话头说出甜言蜜语,只对她笑了笑:``故事讲完了,你也睡吧。''

苏桃心满意足,果然转身背对着无心睡了。无心看着她那一头快要凝结成片的乱发,心里很不得劲,决定明天无论如何都要让苏桃搞一搞个人卫生。野人般的生活会很快让苏桃也变成野人,因为苏桃还小,而且是个随遇而安的性子。

一夜无话,到了翌日天明,无心果然跑去附近的小河边拎回了带着冰碴的冷水,又到邻居家借了几只大盆。他守着石头灶在外面用锅烧水,烧到烫了就倒进大盆里,把帘子掀开一线,他把大盆推进帐篷。帐篷里面弥漫着温暖的水汽,夹杂着香皂的芬芳。小全袖着双手溜达过来了,一看无心蹲在外面疯狂烧水,便是莫名其妙:``哥,你渴啦?''

无心满面尘灰烟火色:``我给你嫂子烧水洗澡呢——你离帐篷远点儿。''

小全听了,顿时有点儿不好意思:``水够用吗?我帮你到河边再拎一桶回来?''

无心立刻把桶递给了他:``好兄弟,辛苦你了。''

说完这话,他把手伸到棉袄里抓了抓痒,忽然发现怀里的白琉璃不见了。自从入冬开始,这白琉璃是从早到晚的贴在他胸前取暖,从来没干过不告而别的事情。东寻西找的低头看了两圈,末了他在帘子一角下面,发现了一条细细的白尾巴尖。

无心不动声色的捏住他的尾巴,慢慢的向外抻。直到把白琉璃彻底拽出帐篷了,无心将他重新往怀里一塞,一边捅火一边低声问道:``白琉璃,你老人家看什么呢?''

未等白琉璃回答,无心下意识的抬了头,发现猫头鹰不知何时回了来,竟然无声无息的落在帐篷顶上,也正低头向内窥视。

无心长长的吹了一声口哨,感觉自己十分端庄高洁。

苏桃在帐篷里大动干戈,费了许多力气,终于把自己收拾出了本来面目。坐在火塘边晾着头发,她正要细细享受这难得的一刻清爽,不料无心捡了许多麻雀粪回来,直接就要往她脸上涂抹。她吓得大叫一声,转身要逃。可无心连兔子都抓得住,何况一个她?小孩子似的被无心横着抱了,她瞪大眼睛呀呀叫着,眼看无心把一指头麻雀粪蹭上了自己的脸蛋。五官瞬间全皱到一起去了,她龇牙咧嘴的在无心怀里扭来扭去,紧闭双眼不肯面对现实。

无心雷厉风行,飞快的用麻雀粪敷了她的手和脸。片刻之后他放了手,用水为她洗净了手脸。

``麻雀粪嘛,又不算脏。''他安慰苏桃:``我们现在弄不到雪花膏,只好拿麻雀粪对付着用了。''

苏桃缩在角落里,自己摸着手背和面颊,感觉皮肤是比先前柔润滑溜了许多。冬季寒冷干燥的山风已经快把她的面孔吹出一层硬壳,手背也像干旱土地一样粗糙的皲裂了。颇为好奇的观察着无心的一举一动,她想无心真的是无所不知、无所不能。手托着下巴走了神,她又想起刚才无心只用一条手臂便箍住了自己的身体,真是力大无穷。

苏桃花了整整一上午的时间浮想联翩,直到无心把勺子塞进了她的手里,她才意识到要吃午饭了。

两人喝了一肚子肉粥之后,无心照例是出门打猎,顺路收集更多的麻雀粪来保护苏桃的脸蛋。

扛着一根削尖了的桦树枝,他一个人慢慢的往林子深处走。忽见一只野兔在荒草丛中一闪,他立刻四脚着地俯下了身。正是蓄势待发之际,不远处的一棵老树后面,突然响起了一声细细的呻吟。

无心觅声望去,林中地势不平,荒草长得乱七八糟,他依稀只见老树后面活动着一团白影。犹犹豫豫的起了身,他一言不发的慢慢走向老树,想要探个究竟。

及至将要靠近老树了,一张白生生的面孔忽然从树后伸了出来。面孔打着刘海挽着发髻,正是个旧式小媳妇的模样,而且还是个楚楚可怜的漂亮小媳妇,只是两道细眉蹙起,是个痛苦的神情。对着无心看了一眼,小媳妇开了口:``大哥救命,我刚扭伤了脚,现在疼得一动都动不得了。''

无心笑嘻嘻的绕到了她的面前,在一米远外稳稳当当的蹲下了:``你怎么扭的?''

小媳妇斜斜的伸出一只雪白的小手:``就在那边的草窝子里,我是一个不留神踩空了,嗳哟,可疼死我了。''

无心又问:``扭了你哪只脚?''

小媳妇向下一努嘴:``喏,左脚。''

无心上下打量着小媳妇,见她穿着一身干干净净的白布裤袄,眉目之间颇有几分动人的姿色。对着小媳妇点了点头,他笑眯眯的站起了身:``我知道了,再会。''

话音落下,他转身要走。小媳妇一见,登时急了:``单是知道有什么用呀,大哥,你得救我。你不救我,我非在林子里冻死不可。''

无心在她面前又蹲下了,慢条斯理的问道:``我怎么救你?''

小媳妇抿嘴一笑:``你背我走。''

无心一歪脑袋,唱歌似的答道:``我可不舍得费力气,背你走多累!''

小媳妇抬手作势对他一打:``你个小气鬼,大冷的天气,你权当是背张人肉褥子了。''

无心掏了掏耳朵:``就算你是褥子,也轮不到我睡。''

小媳妇向他一挤眼睛:``不要脸的,有本事你也抢着睡一觉。''

无心露出一脸傻相,对着她眨巴眼睛:``可我现在一点儿也不困,我今天早上刚睡醒。''

小媳妇格格笑了:``臭小子,你装什么傻?''

无心一立眉毛:``好哇!你敢说我傻?我今天饶不了你!''

话音落下,他冲上去一手抓住小媳妇的衣领,另一只手高高举起,一鼓作气连扇了对方十几个大嘴巴,把小媳妇的脑袋抽成了拨浪鼓。小媳妇先是一愣,随即反应过来了,当即开始挣扎。抬起双手挡住无心的大巴掌,她对着无心一张嘴,``呼''的喷出一团青雾。而无心当即还击,力道很足的``呸''了一声,把一口唾沫直啐到了对方脸上。

小媳妇气得目眦欲裂,张大嘴巴不换气的往外喷雾,无心则是接二连三的大啐不止。两人如此对战片刻,很快一起累得口干舌燥。小媳妇不住的做深呼吸,无心也是左看右看,想要找口雪来润喉。双方正是对峙之时,小媳妇忽然向后一仰头,换了个角度审视无心,同时口中做狐狸叫:``嗷?我怎么看你有一点眼熟?''

无心依然紧抓着她的领口:``妖精,少和我套近乎!信不信我对你先奸后杀再烧烤?''

小媳妇大叫一声:``操!这句话也很耳熟,莫非你是\ldots{}\ldots{}''

无心紧盯着她:``我是谁?''

小媳妇的嗓门降了一个调子:``莫非你是\ldots{}\ldots{}无心?''

无心吓了一跳:``你怎么认识我?''

小媳妇当胸给了他一拳:``乾隆年间你爱过我,你全忘了?''

无心影影绰绰的想起了一点皮毛,但是心惊肉跳,宁愿自己没想起来:``两百年前的事情你还记得?你这心胸也太不宽广了!''

手中的领口忽然一松,一只白毛红眼的大狐狸从袄裤中窜了出来,从天而降扑倒无心:``你他娘的少说风凉话!你敢说你不认识我?''

无心躺在地上,硬着头皮答道:``我好像是\ldots{}\ldots{}认识你\ldots{}\ldots{}一点点。''

大白狐狸一爪子摁住他的喉咙,张嘴说出流利的人话:``薄情寡义的东西,你敢说你只认识我一点点?''

无心非常了解对方的战斗力,所以一时反倒不敢妄动:``大白,你听我说——''

狐狸不听:``两百年前你还叫我小白,现在怎么成大白了?''

无心向她苦笑:``两百年前我是在恭维你,你看你这身材,比狼狗都大,在我心目中,一直都是大白。大白,刚才我没认出你,以为是个要害人的小狐狸精,就下了狠手。你要是生气,我让你打回来。打完之后你我就从此别过吧,你当你的妖,我做我的人。好不好?''

狐狸从头到尾扭成一股波浪:``不听不听不听!两百年前你不告而别,我还没有跟你算总账呢!''

无心苦着脸看着她,心想这么沉重的狐狸还要撒娇,简直快要压扁自己了。

与此同时,四周窸窸窣窣的起了响声,五六只大大小小的红狐狸从林子深处跑了出来,各自乖乖的围坐在了周围,一副徒子徒孙的乖模样。

\chapter{倒霉的白琉璃}

无心试试探探的抬起了一只手,去推身上的大白狐狸。大白狐狸的分量绝不小于一只普通灰狼,骨沉肉重皮毛厚,并且牙齿爪子都是极端的锋利。无心不敢太过明显的对她动武,因为怕她没轻没重的给自己一下子。虽然妖精们对于他的鲜血素来是敬而远之,不过有着两三百年道行的大狐狸精,总不会轻易死在他的血上,而他若是被狐狸咬断脖子抓烂了脸,晚上可怎么回家见苏桃呢?

一只手陷在了对方的雪白皮毛里,狐狸皮的手感果然是超过了猫头鹰的羽毛。无心把另一只手也伸向了白狐狸,轻轻的给她抓了抓痒:``大白\ldots{}\ldots{}''

白狐狸先是舒服的一眯眼睛,随即骤然变脸,对着无心亮出一口白森森的大獠牙:``我又不是狗,你挠我干什么?''

无心登时摆了个举手投降的姿势:``我又不能吃,你扑我干什么?''

白狐狸像匹小号骏马似的一挺身,两只前爪落在无心胸口,敲鼓似的一顿乱挠:``讨厌讨厌讨厌,你说我为什么扑你?''

无心慢慢的把眼睛越睁越大:``大白,你不会是\ldots{}\ldots{}还爱着我吧?''

白狐狸抬头想了一想,又张了张嘴,最后浪声浪气的告诉他:``我也不知道耶!''

无心听了她的娇音,忧愁得想要叹气:``大白,你放了我。我们有话坐着说,好不好?''

白狐狸果然从他身上撤了爪子。无心坐起了身,顺便环顾了周围的一圈大小狐狸,心中叫苦不迭。白狐狸倒是自顾自的挺欢喜,也不变个人形,直接就往无心身边一挤,无心猝不及防,险些被她挤了个跟头。一手撑地稳住了身体,无心扭头抱怨道:``大白,两百年不见,你越发力大无穷了。''

白狐狸一瞪眼睛:``不许叫我大白!''然后她从头到尾扭扭摆摆了一番:``你两百年前是怎么叫我的?''

无心做了个瞠目结舌的表情:``小白?''

白狐狸猛然怒视了他:``还有个更好听的,难道你忘了?''

如果白狐狸不出现,无心真就记不得两百年前的事了。然而白狐狸对于他来讲,总是一位出奇的伴侣,所以对方一做启发,他隐隐约约的,还真把往事记起了几分。对着白狐狸咽了口唾沫,无心又向后略躲了躲:``大白,你我两百多年没见面了,如今偶然重逢,是不是庄重一点更好?''

白狐狸当即任性的骂街:``操!我就不庄重!''

无心在狐狸们的包围下,无可奈何。蹙着眉毛一抿嘴,他露了个愁眉苦脸的笑容。缓缓转向身边的白狐狸,他捏着嗓子做鸭子叫:``狐狐宝贝儿!''

话音落下,他把脸扭向前方,不由自主的龇牙咧嘴,并且一吐舌头。可是还没等他收回舌头,后脑勺上已经挨了一大爪子。捂着脑袋向旁一躲,他大声叫道:``是你让我叫的,叫完了你又打我?''

未等白狐狸出言作答,周遭已然响起一圈低低笑声,叽叽咯咯的似人非人。无心恼羞成怒的把脑袋转了一圈,忽然伸手一指:``你是狐狸吗?黄鼠狼跟着凑什么热闹?''

一条细细长长的小黄鼠狼跟在一只红狐狸身后,本来也在偷笑,冷不防的被无心发现了行踪,立刻吓得往红狐狸身后一躲。无心和白狐狸讲不出道理,欺软怕硬的想要把矛头转向黄鼠狼,然而白狐狸急于叙旧,并不给他王顾左右而言他的机会。对着无心的脑袋又是一爪子,她开口骂道:``负心汉,你说你两百年前为什么不告而别?''

无心捂着脑袋转向了她:``为什么?因为我不想和你过了!''

白狐狸当场急赤白脸:``凭我的花容月貌,你凭什么不想和我过?''

无心不假思索,有一说一,开始对着白狐狸长篇大论。原来两百年前白狐狸刚刚得道修成人形,十分兴奋,一天三变化,三天九变化,今日做张,明日做李。无心早上出门去,晚上回家一定认不得老婆是谁。虽说夜夜做新郎是桩美事,可无心与众不同,只想找个固定的伴侣过生活。白狐狸终日千变万化,有时还变成个老爷们儿,在家里不男不女的吆五喝六,无心偶尔劝她几句,她嚣张惯了,反倒比无心脾气还大,丝毫道理不讲。

白狐狸没个准模样,日子也完全的不会过。她夜里不睡觉,坐在床上呼吸吐纳;白天不做饭,因为最爱吃水煮蛋,所以天天煮一大锅鸡蛋,自得其乐的吃出满屋子鸡屎味。无心想要劳她做一顿饭,真是千难万难,时常是十求九不允,臭骂倒是管够。如此生活了一个多月,无心实在是熬不得了,回家和白狐狸摊了牌,要和她大道朝天各走半边。白狐狸对他还没喜欢够呢,听闻此言,登时大怒,扔了他一身鸡蛋皮。无心一言不发的上床睡觉,翌日一早出了门,脚底抹油径直溜了。

无心自觉十分占理,倒要看看白狐狸如何回答。而白狐狸经过了两百多年的成长,虽然法力越发高超了,脾气越发暴躁了,但是并没有比两百年前更通情达理。无心说得她哑口无言,不以为然的抖了抖嘴上几根长胡子,她无言以对,忽然一歪脑袋枕上了无心的肩膀,一只三角耳朵直蹭无心的面颊。无心不为所动,抬手暗暗的去揉藏在胸前的白琉璃,想请对方出力帮忙,驱走白狐狸。然而白琉璃躲在他的怀里,正在饶有兴味的听他和白狐狸翻旧账,自作主张的不肯出手。

无心孤立无援,而白狐狸企图以柔克刚,在他身上蹭得正欢,忽然动了动鼻子,她直起身质问无心:``你身上怎么有一股子鬼气?''

无心一扬眉毛:``我身边不是人就是鬼,有鬼也不稀奇!''

白狐狸一亮獠牙:``好哇!那我现在已经回到你身边了,你快让你的鬼姘头滚蛋!''

无心把头一摇:``不行,我已经不爱你了,要滚也是你滚,我要和我的鬼姘头恩恩爱爱天长地久。''

白琉璃躲在一层大棉袄里面,听到此处不禁左思右想:``他说的鬼姘头,不会就是我吧?''

白狐狸一听,当场发飙:``嗷!有本事就把你那不得超生不入轮回的臭婊子带到我面前来,看姑奶奶不把她打成烟!''

无心一耸肩膀:``鬼嘛,看得见摸不着,你不打他也是一阵烟。''

白狐狸做怒目金刚状:``都成烟了还这么骚?看得见摸不着,不能亲不能抱,你找她图个什么?''

无心一手环抱膝盖,眺望远方咬着手指头:``我图他心灵美境界高,还图他不吃不喝不花钱好养活。''

白狐狸为所欲为惯了,没有无心她活得挺快乐,如今意外的见了无心,她一时春心萌动,忽然很想和他再续前缘;至于无心本人愿不愿意,则是根本不在她的考虑范围之内。听到无心对一只鬼心心念念的赞美不已,她胸中燃起一团妒火,张开大嘴做狐狸叫:``不管不管不管!你那鬼姘头在哪里?姑奶奶这就去撕了她!''

无心等的就是她这一句。歪着肩膀向白狐狸一转,他嬉皮笑脸的答道:``我那亲亲爱爱的鬼宝贝儿现在不在我身边,你有本事就今夜去找我,看我的鬼宝贝儿不打秃了你身上的毛!''

白狐狸把尾巴往身前一盘,盛气凌人的答道:``好!今夜就今夜!我若是赢了,你可得乖乖的跟我!''

无心连连点头:``那我可走啦?大白,咱们夜里见。''

白狐狸把嘴一伸:``亲一下再走。''

无心真不乐意和白狐狸亲嘴,可是如果不亲,少不得又要打许多嘴皮子官司。闭着嘴和白狐狸碰了碰嘴唇,他一挺身站起来,对着白狐狸抱拳拱手:``我真走了,你知道我家在哪里吧?别找错了!''

然后他捡起自己用来扎兔子的桦树枝,连跑带跳的逃了。

无心在离开狐狸的地界之后,并没有急着回家,而是上蹿下跳的继续打猎。末了捉到两只大尾巴松鼠,他收了手,开始满世界的找麻雀粪。一边找一边唤道:``白琉璃啊!''

咽喉凉了一下,仿佛有风从他的领口向外吹。白琉璃出现在了遮天蔽日的林子里,还在回味无心的情史:``什么事?''

无心心不在焉的说道:``你准备一下,今夜可能会有一只狐狸精来撕你。''

白琉璃怔怔的望着他,一时没听明白,直到几分钟之后才反应过来了,当即怒气勃发:``这和我有什么关系?''

无心蓄着满心的坏水,神情淡然的答道:``谁知道呢!''

白琉璃无论如何想不通:``为什么是我?''

无心仰起头,四面八方的寻找鸟巢,想要摸几个鸟蛋吃:``是大白要撕你,又不是我要撕你,我怎么知道?''

白琉璃听了半天热闹,最后听了个引火烧身。张着嘴望着无心,他简直不知从何说起:``你——我——她——''

无心蹲下来脱了棉鞋,开始爬树:``三百多岁的大狐狸精,论岁数够做你的祖奶奶了,绝对不是吃素的妖精。你可别轻敌,当心被她打出个三长两短。''

白琉璃气了个直眉瞪眼。三百岁的狐狸也是狐狸,让他和狐狸打架,他嫌丢人!再说他又不认识狐狸,为什么要和狐狸打架?

此时无心已经爬到树梢。伸手从鸟巢里掏出一只鸟蛋塞进嘴里,他一边往下溜,一边心中暗暗痛快:``让你不帮我,让你装死狗!这回好了,晚上你和大白闹去吧!''

无心用树枝扎着松鼠回了家,欢声笑语的磕碎了鸟蛋,和苏桃用荤油烙蛋饼吃。白琉璃已经和他成了仇人,不肯再紧贴着他取暖。独自爬进帐篷里,他钻到猫头鹰的肚子下,盘成一堆躲进了对方的羽毛中。猫头鹰正在睡大觉,丝毫没有察觉。

时光易逝,转眼间到了天黑时分。无心带着苏桃在火塘边的兽皮褥子上躺下了,苏桃好奇的抬头去看:``猫头鹰今晚怎么没出门?''

猫头鹰缩在角落里,两只大眼睛探照灯一样四处乱瞧。帐篷外面妖气逼人,他不敢出门。战战兢兢的乍起了羽毛,他用一只翅膀盖住了身边的白琉璃。

无心摁下了苏桃的脑袋:``他今天不饿,不用打猎。你睡你的,乖。''

\chapter{不眠之夜}

苏桃心无杂念,说睡就睡。而无心等到她的气息沉稳悠长了,便轻轻的抽出手臂,塞了个小包袱给她做枕头。趴在兽皮褥子上抬起头,他笑嘻嘻的对着白琉璃摇头晃脑。白琉璃从猫头鹰的大翅膀下伸出了脑袋,虎视眈眈的对他怒目而视。

无心满心都是幸灾乐祸的痛快,对着白琉璃先是一挑眉毛,随即一挤眼睛,最后一伸舌头。猫头鹰作为一只小小的妖精,对于妖气十分敏感,本来就要吓晕了,此刻欣赏了无心的鬼脸,越发的要站不住。而无心又对着帐篷外指了指,对着白琉璃做口型:``她来啦。''白琉璃一扭头,心想她来不来的关我屁事!

帐篷外面起了轻轻的响动,无心眼望白琉璃,同时抬手一指苏桃,又对帐篷外面一歪嘴巴。眼看白琉璃盘成一堆八风不动,他转而采取怀柔政策,对着白琉璃双手合什拜了拜。白琉璃没好气的瞪了他一眼,然后腾空而起窜出了蛇身。猫头鹰一哆嗦,被一股子阴森的鬼气狠狠一激,舒服死了。

无心没哆嗦,他爬到帐篷边沿,把帆布兽皮掀起一线,偷偷的向外窥视战情。白狐狸果然来了,变了个一身白旗袍的美女样子,一双眼睛滴溜溜乱转。忽然向上一抬头,她仿佛是见了什么稀罕物件,转身追着快走几步,她随即改走为跑,一溜烟的没影了。

无心知道是白琉璃把她引进了林子里。坐在兽皮上想了想,他灵机一动,把猫头鹰抱到怀里低低的嘱咐了几句,然后一掀帘子出了帐篷,一路尾随着观战去了。

再说白琉璃把白狐狸引到了林子深处。林中荒凉,阴气最重,正是妖魔鬼怪活动的好地方。白狐狸已然修炼出一双阴阳眼,此刻亭亭玉立的站在一丛荒草之中,她举目向前一望,就见白琉璃清清楚楚的飘在空中,果然是个如假包换的死鬼。上上下下的将白琉璃打量了一番,白狐狸心中有了自信,当即抬手指向白琉璃,口中骂道:``贱人!敢和姑奶奶抢无心!''

白琉璃又羞又愧的低下了头,没想到自己居然沦落到要和一只狐狸争风吃醋的地步,争风吃醋的目标还是无心。一辈子的脸,现在一瞬间全丢光了。白狐狸双手叉腰,继续大骂:``臭不要脸的,也不撒泡尿照照你那披头散发的死样子!你要身段没身段,要线条没线条,侧面像门板成精,正面像吊死鬼落地,凭你这种姿色,也敢在姑奶奶面前作乱?''

白琉璃没有受过如此猛烈的抨击,几乎被骂昏了头,但是没有生气,因为对于自己的形象不甚在乎,像门板也好,像吊死鬼也好,都没关系。

他不言语,导致白狐狸以为他城府极深,是位劲敌。深深的吸了一口气,白狐狸发动了第二轮攻击:``小婊子,不许装聋作哑!信不信姑奶奶暴脾气,打散了你让你去投个猪狗胎?老狐不发威,当我是病猫!连我的男人也敢抢,今夜姑奶奶非让你再死一回不可!''

话到此处,白狐狸妖气大盛,一双眼睛也隐隐的泛了红光。白琉璃先前生生死死几十年,只和猫头鹰打过交道,所以对于妖精的手段很不了解。莫名其妙的抬起头,他一脸好奇的望向白狐狸。而白狐狸和他对视片刻之后,眼中的红光忽然退了——在动武之前,她忍不住还想再骂几句:``瞧你这副德行,越看越像个男人,一点儿女人气都没有,真不知道你是怎么勾搭上的无心!我听无心说你什么美什么高,好的了不得!我倒想知道你哪里美哪里高?我怎么就一样都没看出来呢?''

白琉璃很认真的思索了一番,末了开口答道:``我也不知道。他大概只是随口一说,他是个骗子,经常说谎。''白狐狸后退一步,高声叫道:``哇操!你声音好粗,越来越像男人了!''白琉璃有些窘迫:``我的确是个男人。''

此言一出,白狐狸张大了嘴,足足安静了十分钟。十分钟后她做了个深呼吸,对着白琉璃怒道:``既然你是个男人,为什么不守男人的本分?''白琉璃很懵懂的歪着脑袋看她:``男人的本分\ldots{}\ldots{}是什么?''白狐狸不假思索的答道:``男人的本分就是离无心远点儿!''白琉璃想了想,随即一本正经的摇了头:``不。''

白狐狸没想到他敢公然违令,当即怒不可遏的向前一跃,在半空之中恢复原形,抖擞着一身雪白皮毛落到白琉璃面前。双眼亮成两颗剔透的火红珠子,她开始对着白琉璃发狠,口中一呼一吸,喷出的全是青色毒雾。而白琉璃缓缓飘落到一棵老树下盘腿坐了,弯腰垂头闭了眼睛。

无心躲在远处的草窝子里,目不转睛的静静观战。他只盼着白琉璃给白狐狸一个下马威,让白狐狸自己知难而退。然而术业有专攻,白琉璃的本领显然不适宜刀光剑影的真战场。普通的树枝石头伤不了白狐狸,而在白琉璃喃喃念咒的空当里,白狐狸仰头对着夜空张开大嘴,慢慢吐出了自己的内丹。

妖精的灵性出于日积月累,法力则全是凭着勤修苦练。躯壳为鼎炉,精神为药物,妖精无论大小,只要是真成了精,体内都藏有一颗修炼而得的内丹。此刻白狐狸吐出一团鲜红的烟雾,雾中一枚圆珠光芒闪烁,几百年的修为都凝结在丹中。白琉璃若是被她的内丹伤了,后果可是不堪设想。

白琉璃的咒术是个慢功夫,白狐狸内丹已出,却是随时可以给他迎头一击。无心大惊失色,起身就往前跑。跑了几步之后他一转身上了树,猴子似的抓着树枝向前悠荡。眼看就要到达战场上空了,一个黑影在他头顶盘旋一周,``嗥''的发出了一声猫头鹰叫。

猫头鹰是留在家里坐镇的,如果没有意外情况,绝对不会冒险溜出帐篷寻找无心。无心分身乏术,只能先救眼前的急。眼看白狐狸的内丹距离白琉璃越来越近了,他一狠心纵身一跃,想要从天而降压住白狐狸,暂时阻止她的攻势。不料树枝都被冻脆了,不能由着他拉扯借力。张牙舞爪的从天而降,只听``扑通''一声,他擦着白狐狸的鼻头着陆,把半空中的内丹给拍到土里去了。

双手撑地猛一抬头,他大声喊道:``白琉璃快回家,家里可能出事了!''白琉璃一闪身,登时飘了个无影无踪。而白狐狸猝不及防的受了一惊,此刻用两只前爪捂着鼻头,望着无心直发呆。无心把手伸到胸前一抓,抓到一枚热腾腾的浑圆珠子。攥着珠子一跃而起,他一转身,也撒丫子跑了。

白狐狸不怕他跑,可是内丹还在他的手里,如果丢了内丹,她几百年的修为就算是喂了狗,恐怕连变个人形都有困难。两只前爪保护着受了伤的鼻头,她迈动两条后腿,体态修长的追着无心也跑了。

无心的速度比野兔还快,不出片刻的工夫,已经回到了村子。村子里面没有灯火,然而无人入睡,全惶惶然的站在木刻楞外窃窃私语。无心再一细瞧,发现各家连行李都收拾得了,是个随时要走的模样。

他在帐篷门口找到了苏桃,苏桃挎着书包,抱着背包,一见他出现了,她当即狠狠一跺脚:``大半夜的,你上哪儿去了?''无心回答不出,只接过了她的背包背上,又低声问道:``发生什么事情了?''苏桃方才等他等得心急如焚,简直隐隐的快要就地发疯。如今吐出了一口气,她在劫后余生的轻松中小声答道:``有人说县革委会要派民兵来搜山,要把山里的人全都抓住遣回原籍。''

无心一听,连忙又去找了旁人细问。原来此言并非空穴来风,长白山下的原始森林里,如今已经有了好几处盲流聚集点。入夜之后他们刚得的消息,说是昨天夜里,真有民兵袭击了距离此处一百多里地远的一处盲流村,抓了好几百人。几百人中溜出了几条特别机灵的漏网之鱼,其中一条鱼逃来此处,让村里的人马上做出逃亡的准备。

无心打听清楚了,钻回帐篷看到了猫头鹰和白琉璃。白琉璃已经附回了蛇身,正在猫头鹰的翅膀下东张西望。无心把他抻出来往怀里一塞,然后扯起兽皮褥子把猫头鹰一裹,抱孩子似的抱在胸前。钻出帐篷拉住了苏桃的手,他算是把家里的活物都带齐了。

全村的人像桩子似的在外面站了一夜,随时预备着往山林里逃。白狐狸此刻没有内丹,法力消失了十之八九,导致她现在有点儿缺乏自信,一见人多,竟然没敢贸然进村。捂着鼻头在林子边缘也陪站了一宿。

好容易熬到了天亮,民兵并未出现,村子里随之渐渐恢复了往日的生机。众人不敢生火做饭,怕炊烟会引起民兵的注意,只用炭火对付着煮些稀粥。小全看无心抱了个毛茸茸的兔皮襁褓,大吃一惊,以为苏桃生了孩子。凑过去一瞧,他登时笑出了声,原来襁褓之中躺着个大猫头鹰。猫头鹰值了一夜的更,此刻闭着眼睛,竟是已然入睡了。

帐篷里的火塘是昼夜不息的,上面总吊着一壶热水。无心和苏桃钻回帐篷对付着吃喝了,无心看苏桃脸上灰苍苍的,几乎带了病容,就安慰她道:``民兵来了也没事,至多是把我们遣回文县。回文县就回文县,大不了到文县我们买张火车票,照样是想去哪里就去哪里。''

苏桃处在崩溃与麻木之间,要说怕,也没感觉很怕。自顾自的倒了一杯热水,她疲惫的嘀嘀咕咕:``住到山里了还不得太平,那些民兵真是吃饱了饭没事做!''无心仰起头,从帐篷的孔洞中看天色:``好像要下大雪了。一旦下了大雪,大雪封山,我们就安全了。''

无心这话说出不久,外面果然飘起了小雪。小雪落在地上就不化,慢慢的越积越厚。及至到了傍晚,无心和苏桃吃过晚饭,眼看天色越来越暗,苏桃便把兽皮褥子重新铺好,无心则是钻出帐篷,把小帐篷上的积雪扫了扫,免得帐篷被雪压塌。

下雪的时候,天气往往不冷。无心把帐篷扫干净了,回到火塘边烤火。正是惬意之时,帐篷帘子忽然动了动,同时一个声音模仿了敲门的声音:``咚咚咚。''无心望向门帘:``谁啊?''外面有人斯斯文文的回答道:``嗷,我是大白呀。''

无心摸着棉袄兜里的圆珠子,发现这大白狐狸没了内丹,倒是变得文明多了。

\chapter{冬日}

无心并不想请白狐狸进帐篷,怕吓着苏桃,就算吓不着苏桃,万一白狐狸胡说八道漏了他的底细,他也没法对苏桃交待。然而就在他左思右想之际,一只白爪子一挑帘子,白狐狸已然伸着脑袋亮了相。

一双红眼睛滴溜溜一转,白狐狸看清了帐篷内的格局。帐篷中央是一坑炭火,炭火上方吊着一只铁水壶。无心和一个小姑娘分别坐在火塘左右,无心端着一杯水,小姑娘穿着一身臃肿的棉袄棉裤,正张着嘴看她。

虽然在白狐狸的眼中,苏桃不过是个一分钱不值的小丫头崽子,但是如今人在矮檐下,不得不低头。礼数周到的对着苏桃一点头,随即她转向了无心,彬彬有礼的问道:``我可以进来吗?''

无心望着白狐狸,就见她为了保护鼻头,别出心裁的用一片大黄叶子折了个三角小帽扣住鼻子,黄叶子上还戳了好几个洞,以便她伸展胡须。而从黄叶子表面的一层白霜来看,她方才定是没少跑路。

偷偷溜了苏桃一眼,无心正要拒绝白狐狸的要求。然而白狐狸性情急躁,见无心盯着自己一言不发,她一时忍耐不住,索性自己摇头摆尾的挤进了帐篷。像条大白狗似的坐在火塘边,她把尾巴一盘,对着无心抛了个媚眼:``我想拿回我的内丹。''

无心单手插兜,攥着一枚大圆珠子。白狐狸一出现,大圆珠子似乎有所感应,在他手心里一跳一跳的升了温。沉吟着又偷看了苏桃一眼,他发现自从白狐狸出声开始,苏桃的嘴就没合上过。直勾勾的盯着大白狐狸,她的神情极其类似梦游。

无心浑身破绽,做贼心虚,以至于他默然无语,越想越慌。而白狐狸没有好性子陪他坐禅,眼看他一双眼睛东一转西一转,嘴唇也紧紧的抿成了一条线,她不禁一头扎进了无心怀里,开始大规模的撒娇:``无心无心无心,给我嘛给我嘛给我嘛!''

无心连忙向后一仰:``大白,君子动口不动手,你给我好好坐下!''

白狐狸非常识相,立刻一挺身坐稳当了:``无心,只要你肯把内丹还给我,我就承认两百年前是我错,我——''

无心听到这里,吓得险些把眼珠子瞪出眼眶。闪电一般的出手捏住了面前的狐狸嘴,他急急的斥道:``往事不许再提了,想要内丹也容易,但是我有条件。''

白狐狸张不开嘴,从鼻孔里面挤出``唧''的一声,表示自己愿意洗耳恭听。

无心开了口:``第一,不许你再来骚扰我,你我一刀两断,原来的事情,更是提都不许再提了!''

白狐狸本来也不是情种,之所以对着无心纠缠不休,无非是在林子里闲出屁了,想要找点乐子。听了无心的条件,她连忙点头,长胡子在无心的手心里蹭来蹭去。

无心紧接着竖起了两根手指:``第二,内丹不能白给你,你得拿粮食来换。''

话音落下,他松开了手中的狐狸嘴。白狐狸张大嘴巴活动活动下颚,然后反问无心:``我到哪里去弄粮食呀?我又不吃粮食!''

无心一扬脑袋:``我不管,反正你肯定有办法。''

说完这话,他又溜了苏桃一眼,发现对方依然张着嘴,眼睛都直了;而白狐狸舔了舔嘴唇,显然还想讨价还价。赶在白狐狸开口之前,无心起身搂住白狐狸,双臂运力大喝一声,把大灰狼似的白狐狸抱起来,一弯腰钻出了帐篷。

帐篷外面飘着小雪,白狐狸落了地,仰头还问无心:``帐篷里的丫头是谁?''

无心压低声音答道:``我女儿。''

白狐狸惊讶的一张大嘴:``哇!你还能生——''

无心一巴掌拍断了她的长篇大论:``你也快回窝里去吧,雪下大了可就不好走了。记住,说话算话,拿粮食换内丹。你敢耍赖骗我,我就和你对着耍。你知道我也很会耍赖的,我要是耍上了,你可赖不过我。''

白狐狸记得他是挺难缠,没想到如今仍旧是这么不好说话,居然得理不饶人,还把自己的内丹绑了票。对着无心呲了呲牙,她想识时务者为俊杰,自己活了几百年,要是在阴沟里翻了船可犯不着。欲言又止的白了无心一眼,她无可奈何,只好暂且颠着爪子跑向山林里了。

无心总算打发走了白狐狸,心惊肉跳的松了一口气,他转身钻回了帐篷,就见苏桃还是直眉瞪眼的张着嘴。

无心凑到她的面前,伸手为她一推下巴:``怎么发起呆了?''

苏桃如梦初醒的眨巴眨巴眼睛:``无心,家里是不是刚来了一只白狐狸\ldots{}\ldots{}还会说人话?''

无心严肃的一点头:``没错。狐狸精嘛,说句人话也不稀奇。''

苏桃继续眨巴眼睛,感觉自己是在梦里:``狐狸\ldots{}\ldots{}说人话\ldots{}\ldots{}''

无心一下一下的摩挲着她的头发:``深山老林里面,什么鬼神精怪都有。大狐狸会说人话,不是也挺有意思的?''

苏桃仰脸看他:``这里的狐狸都会说人话吗?''

无心低头笑了:``不是,非得狐狸精才行。''

苏桃又问:``什么样的狐狸才能成精呢?''

无心略一思索,随即答道:``桃桃,成精的狐狸都是爱学习的老狐狸。你想一只狐狸又要学说人话,又要学习法术,是不是也怪不容易的?''

苏桃顺着他的意思一想,倒是深以为然的点了点头:``嗯,够辛苦。''

无心饶有耐心的哄着她:``所以嘛,狐狸精没什么好怕的,它只不过是比一般的狐狸高明一点点而已。''

苏桃毕竟是年少,听了无心的一番言语之后,她越想越是有理,最后枕着无心的手臂躺下了,她竟然兴致勃勃的问道:``无心,人能成精吗?''

无心被她问住了:``人?不知道,我没见过人精。''

苏桃背对着他,摆弄着他的手指头:``我想成精——我要是有了法术,就谁都不怕了。''

无心怕她想邪了,连忙说道:``你算了吧。就算真有法术给你学,等你学成也得一百来岁了。到时候你老成了人瑞,想找人欺负你都找不到。''

苏桃听了这话,感觉还是很有道理。话从无心嘴里出来,怎么说都对劲。

如此过了三天,大雪当真封了山,村中众人松了口气,开始筹备着过年。日子再怎样动荡流离,一年中的大节日还是不能潦草敷衍的。无心和小全翻山越岭的跑远路,在最近的公社里找到了一处黑市。小全用粮食给自己的妹妹换了一块花布——妹妹活到十三岁,还没穿过一件鲜艳衣裳。无心则是用皮子换了糖和酒。两人带着收获踏上归途,路上累死了也不敢停,因为一旦停了,他们容易由累死变为冻死。

千辛万苦的回了家,小全自去向妹妹献宝不提,只说无心钻进帐篷,趴在兽皮褥子上一动不动,连气都无力再喘。苏桃在家等了他两夜一天,身边只有大猫头鹰作伴。望眼欲穿的把他盼回来了,她立刻像个小媳妇似的拧了一把毛巾,给无心满头满脸的擦了一遍,又蹲在地上为他脱了冰砣似的大棉鞋。

无心穿着一双虽有如无的破袜子,脚趾头接触到了帐篷内温暖的空气,让他打了个舒服的冷战,紧接着又呻吟了一声。身体缩在壳子似的大棉袄里,他闭上眼睛把脸在褥子上慢慢地蹭——这一趟走得太辛苦了,他下意识的想要找个人撒撒娇。

一只热腾腾的小手抚上了他的后脑勺,温度透过一层短短的头发,温暖了他冰凉的头皮。他扭头向上一瞧,看到苏桃跪在地上,一手撑在褥子边缘,一手摸着他的脑袋。无心看她,她也看无心,脸上烟熏火燎脏兮兮的,一双眼睛却是黑白分明的很干净,带着笑意和好奇注视他。

无心忽然就不好意思了,乌龟似的把头往领口一缩,闭了眼睛抿嘴微笑,又扬起双手放在两边,尽量的遮挡了自己的面孔。

苏桃笑出了声音,因为一直认为无心顶天立地无所不能,没想到他此刻累成小孩子了。

无心缓过一口气后,便爬起来脱了棉袄棉裤。厚重的袄裤藏着寒气,穿着比脱了更冷。坐在火塘边喝着热糖水,他从棉袄兜里掏出了一只小小的塑料发卡,形状是个扁扁的黑色蝴蝶结。苏桃立刻用它收拾住了垂在额前的碎发,又掰了一小块棒子面饼去喂白琉璃。白琉璃随着无心东跑西颠,日夜不得闲,此刻懒洋洋的趴在兽皮褥子上,他扫了饼子一眼,嫌伙食不好,把嘴闭了个死紧,胸有成竹的等着明日凌晨吃大餐。然而刚刚熬到傍晚,他便开了斋——白狐狸来了。

白狐狸带着一只胖大的红狐狸,二狐全打扮的像驴一样,脊梁上搭着结结实实的大褡裢。正所谓一回生二回熟,这次白狐狸没敲帘子,直接鬼鬼祟祟的钻了进来。对着无心和苏桃都打了招呼,她和红狐狸摇头摆尾的一晃,背上的褡裢当即顺着尾巴滑落到地。转身从褡裢里叼出一只红纸包送到苏桃面前,白狐狸谦逊的笑道:``小姑娘,拿去买糖吃吧。''

无心手快,抢着拿起红纸包,打开了向内一瞧,只见里面装着整整齐齐一小沓钞票,全是一角一张的,加起来足有两块钱。白狐狸因为平时太过无礼,所以偶然一旦懂事了,反倒让无心受宠若惊:``哟,大白,你也太大方了。她一个小孩子,给她三毛五毛的就够了,你何必还特地包个红包?''然后他一拍苏桃的后背:``快道个谢。''

苏桃面对此情此景,感觉新鲜死了,同时不假思索的一鞠躬:``谢谢狐狸。''

无心清点了褡裢里充作赎金的食物,对于白狐狸的所作所为十分满意。趁着白狐狸现在比较老实,他掏出了内丹,想要尽快把她打发走,免得她一时得意忘形,再对着苏桃胡说八道。而白狐狸一见无心掌中的内丹,一双红眼睛立时放了光。张嘴深深吸了一口气,内丹随着她的气息自动飞到半空,同时表面渗出一股子鲜红的雾气。雾气包裹着内丹缓缓移动,一直进了白狐狸的大嘴。及至它移动到喉咙口了,白狐狸``咔嚓''一声合了牙关,紧接着把脖子一伸,将内丹彻底吞落入肚。缭绕在口鼻之间的红雾慢慢散尽,她又做了一番呼吸吐纳,末了忽然把嘴一张,朗声大笑:``哈哈哈哈哈,终于又有自信了!''

无心对她一抱拳:``恭喜恭喜,看到你恢复了自信,我也很是快乐。本来想请你吃顿夜宵的,可是外面雪厚,路不好走,所以我们将来有缘再见,今天就不留你了。''

白狐狸把头一扭:``哼!你当姑奶奶稀罕吃你的饭?姑奶奶今年遇到了你,算是流年不利。你带着你的丑丫头吃吧,姑奶奶全当是喂了一次狗!操操操,黄鼠狼下崽,一窝不如一窝。''

话音落下,她一甩大尾巴,带着红狐狸向外就走。无心笑嘻嘻的不敢拦,而苏桃拿着红包,感觉自己和无心好像是被白狐狸骂了,不过看着无心嬉皮笑脸满不在乎,她便也跟着宽了心,没把白狐狸的话往脑子里放。褡裢里面有白面,有水果糖,有带着尖牙印的军用罐头。无心当场用刀子划开了一只罐头,里面盛着满满的红烧牛肉。白琉璃从他的棉袄下摆中探出了圆脑袋,睁着黑豆眼睛先是愣了愣,随即反应过来了,登时把嘴向上张成了瓢。苏桃捏起一块牛肉塞进他的嘴里,他闭了嘴,专心致志的吞咽。

无心低头看着他,就见他直挺挺的向上伸出老长,和自己组合出了个很不雅的形象。幸而苏桃是个无知的,还在专心致志的喂蛇。不动声色的伸出两根手指夹住白琉璃,他想把对方向下拽一拽,不要伸着个圆脑袋吃的那么来劲。然而白琉璃嚼的正酣,受了他的干扰之后,很不耐烦,当场剧烈的乱扭了一气。

\chapter{除夕}

大年三十这天早上,无心七分愉快三分惆怅的开始张罗着过除夕。七分愉快,是因为他弄到了足够丰富的食物,能让他和苏桃满足的吃上几天;三分惆怅,则是因为单单的有吃有喝还不算好日子,苏桃天天裹着一身桶似的棉袄棉裤,越来越像个野丫头了。

过年的好处在于好吃好喝,没吃没喝还叫什么过年?无心用热水化开了两只冻硬了的野兔子,又撩开了帐篷帘子,蹲在帐篷门口用匕首切肉和干菜。苏桃守在火塘旁边,一边揉着一团白面,一边等着一壶水开。白琉璃在带着两人体温的兽皮褥子上爬来爬去,一双眼睛灰蒙蒙的暂时失了明,因为过几天又要蜕皮了。大猫头鹰缩在角落里一动不动,闭着眼睛睡得无声无息。

小全带着他的几个小妹妹,嘻嘻哈哈的四处乱跑,引得一群大小孩子跟着他登高上远。孩子们都是衣衫褴褛,没个孩子样,然而心还是孩子的心,捡根树枝也能舞弄半天。经过无心的小帐篷时,一个小男孩停了脚步,好奇的问道:``哥,你家过年吃啥?''

无心抬头向他一笑:``吃饺子,你家呢?''

小男孩盯着无心脚边半融化的冻兔子肉:``我妈蒸了馍。''

小全家里最小的妹妹从前方跑回来了,笑嘻嘻的大声说道:``他家的馍就是棒子面发糕。''

小男孩当即抬手打了她一下:``不是棒子面的,是白面的!''

小妹妹很坚决的一口咬定:``是两掺面的!''

小男孩为了维护自家荣誉,开始认认真真的和小妹妹吵架,吵得无心脑仁疼。钻回帐篷摸出两颗水果糖,他一人给了一颗,想把两个小崽子一起打发走。小妹妹用舌头把水果糖推到腮帮子里,四脚着地的把脑袋往帐篷里伸。小男孩也跟着凑热闹,盯着苏桃手里的白面问道:``姐,你咋不出来玩呢?''

苏桃干活永远干不到点子上,一盆白面被她揉了个七零八落不成团:``我干活呢,没时间玩。''

小妹妹一脚把小男孩蹬出老远:``姐,你干完活也来玩吧!我哥总说你长得好看,他肯定愿意带你玩。''

苏桃对小孩子们笑了笑,同时手上不停,疯狂揉面。无心听了小妹妹的话,一边从兔子骨头上往下削肉,一边回头对着苏桃做了个鬼脸。苏桃从眼角瞥见了,但是只当不知,继续乌烟瘴气的和面。

擀面杖是无心用一截粗树枝削成的,一点儿也不圆,但是对付着也能用。接管了苏桃手中的面团,无心开始揉面揪面,擀饺子皮。苏桃在一旁端着小锅,低了头去嗅里面的馅子。馅子很粗,但是肉多油多,气味香的咄咄逼人。无心干来劲了,揪了一小块湿面捏成两只鹿角,黏黏的粘在了白琉璃的脑袋上。白琉璃正处在失明期,并不知道自己被无心打扮成了龙样子。

`

饺子皮太厚了,而且不圆,包成的饺子足有巴掌长,一个一个奇形怪状,在一块木板上列了队。无心心不在焉的哼着小曲,同时发现苏桃是真高兴了,不住的给饺子排队,话也是特别的多:``无心,饺子越包越大了。''

无心从白狐狸送来的褡裢中找到了一盒香烟。叼上一根低头凑到火塘里的炭火上吸燃了,他摇头晃脑的从嘴角挤出回答:``剩下的馅子和皮,改成包子得了。''

苏桃咳嗽了一声,伸手从他嘴角拔下了烟卷:``别抽了,怪呛人的。''

无心对着手里的厚皮大馅不以为然:``小丫头,还管起我了。''

苏桃用发卡夹住了前额的凌乱刘海,嘴里嘀嘀咕咕的讲道理:``抽烟不是好习惯,总抽烟的人,身上有味儿。''

话音落下,她不由自主的一皱眉头,心中想起了小丁猫。小丁猫是一杆面嫩的老烟枪,从头到脚都是烟油子味,像是烟草成了精,用纸一卷就能点火。飞快的把小丁猫从脑海中驱逐出境,她又对无心说道:``无心,我们只吃饺子吗?''

无心抬眼看着她微笑:``还想吃什么?你告诉我,我给你做。''


苏桃想了想,没想出结果。无心不再追问,自顾自的发了面,预备晚上再给她添点花样。

村中的炊烟从早飘到了傍晚,空气中弥漫着甜丝丝的气味,仿佛总有面食刚出锅。盲流们也不是从天上掉下来的怪物,在成为盲流之前,他们也大多有家有业。此刻像一切平常人家一样,木刻楞里点亮了油灯,虽然不敢燃放鞭炮,但是房门两边也都贴了红纸对联。对联是村中一位臭老九亲自书写的,纸不好,墨也不好,可毕竟是红纸黑字,能让人取个吉利,添些喜气。

满村野跑的孩子们各回各家了,就算尝不到饺子,也能饱饱的吃顿干饭。小全家里五个孩子,一个孩子分了两个饺子和一个白面馒头。小全三口两口的吃光了自己的一份,舔嘴咂舌的又去锅里掰了一块杂合面饼子:``无心今晚肯定吃得好,他家不缺肉。''

小妹妹抬起两只小手比划出一个长度:``他家包大饺子了。哥,你也抓兔子给我们吃呀!''

小全嚼着杂合面饼子,想起无心的家庭,一时出了神。而与此同时,正如他们所料,无心和苏桃缩在小帐篷里,的确是正在满嘴流油的大嚼。煮好的大饺子装在铁盆里,油渍麻花的铁锅里放着一摞烙好的发面糖饼。饭盒放在火塘边,里面是满满的肉炒干菜。另有一盒打开了的红烧肉罐头,和一瓶白酒作了伴。把一只盛着干玉米粒的空罐头盒子放在火塘上,无心一边连吃带喝,一般等着火塘的热度把干玉米粒烤成爆米花。

苏桃热出了一身的汗,脱了外面的厚棉袄。无心攥着酒瓶灌了一口,然后把酒瓶递给苏桃:``桃桃,大过年的,你也喝一口吧!''

苏桃没喝过酒。接过酒瓶嗅了嗅,她没闻出好气味。试试探探的仰头尝了一点,她当即张大嘴巴,很痛苦的``哈''了一声。

无心见状,连忙夹起一筷子炒菜喂了她:``得,怪我没正经。''

苏桃吃了他的菜,自觉着一张脸发了烧,红通通的胀成盆子大。滚热的气息从鼻孔呼出去,居然带出了一点酒香。小心翼翼的又喝了一口酒,她咂了咂嘴,抬头对着无心笑:``不好喝,是苦的。''

无心夺过了她的酒瓶:``尝尝味道就行了,不爱喝就不喝。''

苏桃垂下眼帘点了点头,在温暖的帐篷中忽然感到了一阵眩晕。脸越来越大了,头越来越沉了,无心一眼没注意,她竟然抄起酒瓶子又喝了一口。要喷火似的对着火塘呼出一口长气,她随即抬头告诉无心:``哈!我学会喝酒啦!''

无心看了她面红耳赤的德行,心中暗暗感觉出了不妙。强行夺过她的酒瓶子放到角落里,无心拦着她不让她抢:``小姑娘不许学喝酒,你乖乖坐着,一会儿给你吃爆米花。''

苏桃出了一头一脸的汗,脖子都红了:``我不是小姑娘,我二十岁了!''

无心用一条旧毛巾给她擦了擦汗:``好好好,再过四年你就二十岁了。''

苏桃认认真真的要和他讲道理:``我真的不是小姑娘,我都结婚了!''

无心摸了摸她的脸蛋,发现她的体温已经高到烫手:``对对对,你不是小姑娘,你是小媳妇。''

苏桃东倒西歪的绕过一盆饺子一锅油饼,蹲到了无心面前。眨巴着大眼睛凝视了他良久,她忽然张开双臂向前一扑,热腾腾的扑进了他怀里。潮湿的汗气透过绵软的旧衬衫,活泼泼的升着温;双臂环住无心的脖子,苏桃和他贴了贴脸。无心的皮肤总是光滑温凉的,所以她贴得放心大胆,不怕会有胡茬扎她。腾云驾雾的闭了眼睛,她从鼻子里哼出一声,声音懒洋洋而又软绵绵。

无心怔了一下,先是手足无措。抱火炭一样虚虚的抱着苏桃,他很快发现自己是想多了。苏桃的举动中仿佛并没有复杂的深意,纯粹只是小丫头要撒娇。拦腰把苏桃抱稳了,无心想要哄她入睡,哪知苏桃另有主意。一手抓住无心的衬衫前襟,她像只小牛似的一头抵上他的胸膛,拼了命的开始顶。

无心莫名其妙,因为被她揉搓的坐不住,所以只好双手撑地稳住身体:``桃桃,干什么呢?''

苏桃一言不发,专心致志的顶他,顶得摇头晃脑。末了披头散发的抬起了头,她气喘吁吁的咕哝道:``我要进去。''

无心哭笑不得的单手推了她的肩膀:``你要往哪里进?''

苏桃抬手敲了敲他的胸膛:``我要进去。''

无心没想到她存着如此怪异的想法,不禁追问道:``进去干什么?''

苏桃继续顶他,力气大方向偏,几乎一头滑到他的腋下:``进去\ldots{}\ldots{}就再也不出来了。''

无心运力抱住了她:``傻丫头,外面这么大的世界,你都不要了?''

苏桃挣出了一身热汗,鬓角打湿了,弯成一个俏皮的小卷:``不要了\ldots{}\ldots{}我不喜欢它,我不要它了。''

无心用手臂箍住了她的身体,随她翻滚挣扎,就是不肯松手。一切都是事与愿违,他是那么的想在社会中给苏桃找到一处体面的立足地,可是苏桃小小年纪,已经``不喜欢'',``不要了''。

低头望着苏桃头上的廉价发卡和身上泛了黄的白衬衫,他因为爱她,所以感觉眼前情景分外刺目。那么厚密乌黑的好头发,那么苗条亭匀的好身体,不该被这么一堆破烂玩意儿装饰遮掩。

如果时光倒退几十年,他作孽挣命也要给苏桃挣下一份家业。苏桃愿意跟他,他会让苏桃做一名舒舒服服的小少奶奶;苏桃不愿意跟他,他也会擦亮眼睛,给她找个好小伙子相配。可现在不是先前的世道了,不是靠着勤劳聪明挣饭吃的时代了。让他去效仿陈大光一步登天,他做不出。

哄着苏桃在自己怀里入睡了,无心望着火塘浮想联翩,怎么想怎么感觉不对劲。罐头盒子里噼啪乱响,是干玉米粒正在一粒一粒的爆开。

一夜过后,便是大年初一。大猫头鹰全年无休,除夕夜还要出去打食。清晨无心和苏桃一起醒来之时,他已经喂了白琉璃一只小田鼠。

苏桃把自己昨夜的所作所为忘了个一干二净,兴致勃勃的扯出一条红布带子,她在猫头鹰的脖子上围了个红领结。白琉璃头上的白面鹿角只剩了一个,因为眼盲,所以悻悻的趴在火塘边不肯动。

无心热了剩饭。和苏桃吃饱喝足之后,他袖着双手钻出帐篷。先是打扫净了帐篷上的积雪,然后他仰头望天,心想天气一暖,自己就立刻带苏桃出山。在山里与世隔绝的生活久了,苏桃很有变成隐士的危险。

\chapter{不速之客}

正月十五的上午,白琉璃缠在一捆枯枝上,摇头摆尾的蹭啊蹭。大猫头鹰看出了他的痛苦,所以难得的没有白天睡大觉。一动不动的守在一旁,他成了白琉璃的大卫兵。可惜白琉璃完全不能理解一只鸟的苦心,自顾自的只是蹭,直到无心端着一盆热水钻进了帐篷。

无心把白琉璃放进热水盆里,亲自为他蜕皮,一边退一边又唠唠叨叨,全是在对猫头鹰说:``你不要天天喂他了,你看他长得有多快。万一他真长成大蛇了,我可怎么带着他到处走?''

猫头鹰张了张嘴,真想说句人话,然而本领有限,实在是不会说。

无心小心翼翼的撕下了长长一条蛇蜕。把蛇蜕提到眼前看了又看,他叹了口气:``白琉璃,你看你现在肥成了什么样子。原来你细的好像一条小水蛇,如今可好,成擀面杖了。''

白琉璃游出水盆,在兽皮褥子上很舒服的盘成一堆,无意理睬无心。

正月十五也是个大节日,虽然村中各家都做不出元宵,但是多多少少也得预备些许饭菜意思一下。小全捏着一块杂合面发糕走过帐篷,忽见帐篷帘子是大开着的,便很好奇的弯腰向内张望。他往里看,苏桃正好也往外看,两人打了个照面,小全自惭形秽的藏起了手里的发糕,硬着头皮打了招呼:``嫂子,无心哥呢?''

苏桃正在发散帐篷里闷了一夜的热气,此刻拿着一张油饼坐在兽皮褥子上,她垂了头,也是感觉自己形象不好:``他去捡柴禾了。''

小全舍不得走,没话找话的强问:``你家又吃烙饼啦?''

苏桃耷拉着眼皮,对着火塘里的红炭火点头:``嗯。''

:

小全讪讪的直起了腰,恋恋不舍的往家走,刚进家门就受了母亲的质问:``你是不是跑到无心家门口说话去了?''

小全没想到母亲有着鹞鹰般的眼力,人在家中坐,能知天下事:``是,我说话了,怎么啦?''

母亲也是一身的寒气,拍打着身上的雪花骂道:``不许你再和他家的小媳妇凑近乎!那无心的手多狠哪,天上飞的地上走的没有他抓不着的。万一闹出误会了,他非把你当兔子扎了不可!''

小全被母亲说中心事,当场恼羞成怒的闹起了脾气,毛驴一样大尥蹶子。他娘治不住他,于是他爹登场,抡着一根木棒追得他满村逃窜。无心抱着一大捆枯枝败叶走出林子,见了小全父子飞檐走壁的功夫,很觉有趣,笑容可掬的旁观许久,直到小全落网才罢。

无心如今只有白面,所能吃的也只有面食。元宵节里没元宵,于是他傍晚煮了一大锅热面条。面条七长八短有粗有细,面汤也是浓稠得类似糨子,滚烫得让人无法下嘴。苏桃拿着一只白铜勺子,蹲在锅边想要捞肉吃。大海捞针似的在面汤里找了半天,她最后一无所获的收了勺子,送到嘴里试探着舔了舔。

无心被蒸汽熏得满脸泛红,有心脱了棉袄晾一晾汗,可是手忙脚乱的腾不出工夫。正是又热又饿之际,远方忽然起了一声枪响,``叭''的一下子,又轻又脆。

无心愣了一下,心想谁这么大胆,半夜在林子里玩枪。苏桃也抬了头,本来也是热汗涔涔的,然而此刻脸上骤然竖起了一层汗毛:``谁来了?''

无心笑了一下,认为苏桃是兔子的胆子,简直可以和猫头鹰媲美,可未等他开口说话,帐篷帘子被人从外面猛的掀开了,小全的脑袋伸进帐篷,带着哭腔大声喊道:``哥,嫂子,快跑啊,民兵来了!''

话音落下,他调转回头冲向自家的木刻楞。无心知道自己没有时间细问了,和苏桃对视一眼,两人无须交流,直接心有灵犀的起身找出背包,开始手忙脚乱的往里面塞东西。

仿佛是在一刹那间,村子里面就乱套了。不是每个人都能像无心一样洒脱,他们在这片土地上劳作了一年,房屋粮食都在这里,让他们空着两只手往外跑,他们会茫茫然的找不到方向。而且,他们存着侥幸的心思又想,自己在无主的土地上卖力气刨食吃,应该不算犯大罪吧?

未等他们把头脑中的思路整理清楚,民兵进村了。

民兵进村之时,无心刚用一壶水浇灭了火塘里的炭火。黯然一片的帐篷里,两双眼睛在他身边闪闪发光,一双是猫头鹰的眼睛,天生就是这么亮;另一双是苏桃的眼睛,苏桃挎着书包,抱着背包,胸膛里憋着一股子劲儿,仿佛随时可以生出尖牙利爪,和人同归于尽。

无心的小帐篷位于村子边缘,是民兵到达的第一站,可是因为它上尖下圆形状古怪,而且里里外外无声无光,所以民兵根本没把它当成房屋来看,直接绕过它进了村子内部。叮叮咣咣的踢门声音响起来了,女人和孩子们也高高低低的哭起来了;伴随着零零落落的枪声,民兵们开始大声的呵斥叫骂,让全村的盲流们都滚出来集合。

无心背上了帆布背包,为了稳妥起见,又用绳子把它五花大绑的固定在了自己身上。把苏桃拉到身边,他低声想要对她耳语几句,可在开口之前,帐篷外面忽然起了人声:``哎?这是个啥玩意儿?''

有人笑道:``看不出来!帐篷?''

``哪有这样的帐篷?帐篷都像蒙古包似的,你没见过蒙古包吗?''

``那这到底是个啥?你看,这儿还有个帘子,是不是仓库?''

回答他的是一阵抽气:``不对,你闻闻,这周围挺香,好像刚炖了肉。''

在肉香的诱惑下,两名青年民兵大喇喇的走到帐篷前,弯腰扯了帘子一角便是一掀。在帘子掀起的一瞬间,一大锅滚烫的热面条直飞而出,兜头泼了民兵一脸。沸腾面汤的杀伤力是不容小觑的,而在两名民兵捂脸惨叫之时,两个黑影一前一后疾冲出去,一溜烟的消失在了林子里面。民兵抹着满脸的稀烂面条鬼哭狼嚎,正是瘫在地上挣扎着爬不起,冷不防帐篷里又飞出了一只目露凶光的大猫头鹰。大猫头鹰叼着一条白蛇盘旋而上,迎着一轮明月飞向森林,只给民兵留下了一声难听的嗥叫。

无心和苏桃快要跑疯了。

帆布背包被他移到了胸前,他背起苏桃跑得上蹿下跳。背一阵子背不动了,他放下苏桃带着她跑,跑一阵子她跟不上了,他再把她背起来。两人一口气逃出了几里地,后来估摸着民兵们一定追不上了,才双双的停了脚步。

苏桃扶着一棵大树弯了腰,喘得死去活来。无心倒是没有喘的意思,可是不喘也不好,只得陪着她也做了几个深呼吸。大猫头鹰准确的落在了他们身边的树枝上,嘴里叼着扭来扭去的白琉璃。无心怕白琉璃受不了冻,伸手要去抓他;大猫头鹰还很不愿意,把个脑袋左一转右一转。直到无心在他的脑袋上凿了个爆栗,他才乖乖的松了口,把白琉璃放回了无心的手中。

把沉甸甸的白琉璃贴肉放好了,无心打了个哆嗦,感觉自己是揣了一大块冰。接下来怎么办,他一时没有主意,幸好他在林子里混惯了,总不会眼睁睁的任由自己冻死。冒着暴露目标的危险,他收集树枝拢起了一堆火。火烤胸前暖,风吹背后寒,单有一堆火还是不够,于是他把苏桃搂到胸前,让她坐在自己的腿上等天亮。

苏桃在经过了最初的慌乱之后,如今已经渐渐镇定。仰头向后枕上无心的肩膀,她因为在过去的一年里已经是见多识广,所以此刻麻木不仁,并不绝望。本来这里也不是他们永远的家,本来开春之后他们也要继续流浪,只要别落到民兵的手里,其它问题就都不是大问题。把无心的一只手掖进自己的棉袄袖口里,她还是很安心,很知足。

两人围着一堆火坐一会儿,走一会儿,保持身体不被冻僵。将到天亮之时,无心从背包里取出了两张冻硬了的油饼。把饼放在火上烤了烤,他和苏桃狼吞虎咽的填饱了肚皮。找到一片未经踩踏的雪地,无心俯身拂去最上面的一层白雪,然后伸手抓了洁净雪团往嘴里送。看到苏桃有样学样,他开口说道:``少吃,当心吃坏了肚子。''

苏桃含着一口雪,站直了问他:``无心,我们接下来去哪里啊?''

无心弯腰捧起一把雪,满头满脸的搓了搓:``我打算回村里看一看,如果民兵走了,我们就还回去。''

苏桃立刻说道:``我跟你一起走。''

无心也不放心把苏桃一个人留在林子里。领着她踏上来路,在距离村子半里地远之处,他找到一棵歪脖子老树,把苏桃塞进树洞里去了。

然后他继续前行。悄无声息的摸到了村边,他攀在树上向下一瞧,发现村子中央的一处空地上,正有队伍在分批解散。队伍是由村中全体男人集合而成的,每三个人算是一组,用根麻绳拴成一小串。在队伍之中,无心看到了小全。

小全半张脸都被鲜血糊住了,显然是挨了一顿狠揍。他和他爹以及王木匠归为一组,三个人的裤腰带被没收了,一起提着裤子往一间木刻楞里走。民兵端着步枪来回巡逻,是要留在村中大动干戈的架势。

无心一声不吭的盯着小全的背影,心里想要救他。他不信县革委会真有耐心把这帮盲流们``遣回原籍'',对于小全等人的下场,无心几乎能够想象得出——他们的性命,已经被民兵攥在手里了。

苏桃听说无心想要救人,当即表示同意。

虽然她对小全等人毫无兴趣,不过很乐意和民兵们对着干。在她眼中,正月十五进山扫荡的民兵,和联指的革命小将们并无区别。对于他们,苏桃已经反感痛恨到了无法言喻的地步。

两个人在林子里混了一天一夜,到了翌日上午,无心又回了村子,结果发现民兵押走了全村的男劳力,只留下了一群老弱妇孺。追着民兵的足迹走出几十里,最后无心和苏桃停在了一处大农场外。原来县革委会只让民兵去抓盲流,抓到之后怎么办,却是根本没做指示。民兵们不可能长久的留在山里看守盲流,民兵队长一时福至心灵,竟然把盲流送去了附近的农场,盲流在农场劳动所得的工钱,自然也就被民兵们私下分了。

大冷的天,无心不可能总在农场外围转悠。入夜之后他找地方安顿了苏桃,单枪匹马的想要潜入农场去找小全。带上他那套打猎用的装备,他鬼鬼祟祟的进了农场地界,专走僻静小道。眼看将要接近前方一排平房了,他往荒草丛中一匍匐,正要秘密前进,哪知刚刚爬了不到一尺,忽有四只爪子点上他的小腿,一路小跑的向上直踩过了他的后脑勺。他猝不及防的往下一趴,两条小腿上又踩了爪子。翻着眼睛向前一看,他当即气得要骂街——他看到了白狐狸!

白狐狸威风凛凛的充做前锋,带着一溜五只红狐狸和一只小黄鼠狼,昂首挺胸的踏过无心,直奔农场鸡棚。

\chapter{逃离农场}

农场的鸡棚不属于集体财产,是场里工人们自己搭建出来的,目的是能够偶尔改善生活开开荤。棚子里的鸡也不出售,养来纯粹是为了吃。新年过后,鸡们并未死绝,鸡棚里面依旧弥漫着热烘烘的鸡屎气味,勾引得狐狸们垂涎三尺,闻着味儿就过来了。

无心不敢招惹白狐狸,怕她翻起旧账,公然骂街。眼看狐狸们排着队伍走远了,他继续匍匐前进,悄悄的摸到了前方平房附近。四脚着地的弓起了腰,他走兽一样蹲到了后窗户下面。闭着眼睛侧耳倾听,他发现平房里面十分安静,不像个有人居住的模样。蹑手蹑脚的绕过平房,他茫茫然的继续寻找工人宿舍。农场坐落在山麓,大而荒凉,他无声无息的越走越远,最后停在一座灯火通明的大院外,他抽抽鼻子,嗅到了一股子淡淡的酒气。``吱嘎''一声开了房门,有人走到院子角落里哗哗撒尿。透过密集的木栅栏向内窥视,无心发现来人包了一头一脸的白纱布。忽然想起自己当初泼出的一锅沸腾面汤,他暗自点了点头,认为自己虽然没找到盲流,但是找到了民兵,至少可以顺藤摸瓜。

民兵撒过尿后,转身要往屋子里走,可是还没走到门前,房内有人亮开了嗓门:``我说,今天晚上该轮到你了吧?''

民兵嘻嘻哈哈的笑道:``我不去!我是伤员,得养上十天半个月!''

屋子里的人十分不满:``不就是烫破你一层油皮吗?他们农场的人不管,咱们也不管,万一盲流趁夜逃了,你说最吃亏的是谁?''

民兵一边进屋一边骂了一句。片刻之后房门又开了,他背着一杆步枪往院外走,且走且抬起手,去解头脸上的纱布。及至出了院门,他的面孔终于见了凉风。很舒服似的晃了晃头,他大踏步的走向了前方一片小树林。

无心悄悄的跟上了他,一路距离他不远不近,生怕露了形迹。农场正在四处开荒,林子迟早也是要被彻底砍伐铲除的,在林子边缘的一排棚子里,民兵打了个大喷嚏,然后和棚子外面的一名战友打了招呼。战友拄着步枪将要冻死,见他来了,当即骂骂咧咧:``你不养伤吗?你还知道来啊?''

两人开始斗嘴,斗得嘻嘻哈哈。而无心藏在一棵大树后面,抱着肩膀蹲成了一块石头。抽着鼻子吸了吸冷空气,他忽然感觉周遭很臭。从树后露出一只眼睛,他真想派白琉璃上前侦察一番,可是白琉璃最近和他总是别别扭扭,此刻冰凉的缠在他的腰间,显然是无意出手相助。

``怎么会这么臭?''无心想不通了:``他们总不会把人关到茅房里吧?''

正当无心疑惑之际,棚子周围发生了两件事。一是两个民兵走了一个,只留下自称伤员的青年继续看守棚子——他大概也是嫌冷,所以独自钻进了棚子里;二是棚子后面伸出了一个雪白的大脑袋,正是鬼鬼祟祟的白狐狸。

猛然和无心打了个照面,白狐狸登时把嘴一张,欲言又止的露出了舌头。无心则是吓了一跳,因为白狐狸一贯狂放不羁,万一呱呱的和自己翻起旧账,自己可是受不了。双手合什对着白狐狸拜了一下,无心乖乖的服了软。

白狐狸的大脑袋左转一转右转一转,随即一个箭步窜向了无心。一人一狐在大树后面会合了,无心悄声问道:``大白,你来干什么?''

白狐狸仿佛是很困惑:``我来吃鸡呀!''

无心又问:``鸡呢?''

白狐狸立刻出口成脏:``妈的鸡全没了,鸡棚里面关满了人!''

无心听她嗓门不小,连忙伸手攥住了她的长嘴:``嘘\ldots{}\ldots{}你小点儿声,里面的民兵可是带着枪呢!''

白狐狸把头一扭,甩开了他的手:``你来又是干什么?''

无心压低声音答道:``我是想救棚子里面的人。我在他们的村子里住了一冬天,他们都不是坏人,现在被民兵关到农场里卖苦力,太可怜了。''

白狐狸非常任性的一晃脑袋:``我不管,我要吃鸡!''

无心脑筋一转,忽然有了主意。亲亲热热的给白狐狸抓了抓痒,他小声说道:``鸡在哪里,非得农场里的人才能知道。你去把棚子里的民兵揪出来,逼他说出鸡的下落。凭着你的道行,吓唬他还不是小菜一碟?''

白狐狸深以为然,当即颠颠的跑向棚子。在一扇破柴门旁停住了,她细着喉咙开了腔,娇声娇气的呻吟道:``哎哟\ldots{}\ldots{}哎哟\ldots{}\ldots{}有人吗?救命呀\ldots{}\ldots{}``棚子里面立刻起了回应:``谁啊?''

白狐狸一卷大尾巴:``我是工人家属,刚才在林子里把脚扭了,谁来送我回场里宿舍呀?''

她的话音落下,只听``哐啷''一声,破柴门被人亟不可待的推开了。年轻的民兵听到了外面的嫩嗓子,十分亢奋的想要助人为乐。借着身后一只小火盆中的炭火微光,民兵向外一瞧,没瞧到人;而白狐狸仰头看清了他,只见他一脸干瘪的水泡,当即粗声叫道:``哇操!丑得像个鬼!''

此言一出,民兵觅声低头,正和白狐狸对了眼。目瞪口呆的怔了几秒钟,民兵随即发出了一声惊叫:``狐——''

未等他把话说完,白狐狸一跃而起,把他扑了个仰面朝天。一只利爪摁住他的喉咙,白狐狸双目射出红光,龇牙咧嘴的大喝:``快说!鸡在哪里?''

随着她的逼问,一串口水向下落到了民兵的脸上。民兵瞠着眼张着嘴,惊得气都不喘了,直勾勾的望着白狐狸发痴。与此同时,白狐狸身边黑影一闪,正是无心像箭似的溜进了棚子里。

棚子里面光线黯淡,鸡屎味直冲鼻孔。在满地的烂干草中,一堆黑黢黢的人形或躺或坐,正是盲流村里的男劳力们。人们受了惊动,嗡嗡的起了一层疑声。角落中忽然爬起了一个破衣烂衫的少年,颤巍巍的问道:``哥,你来了?''

无心听出了小全的声音,心中登时松了一口气,知道自己没有搞错目标:``我来救你们了,起来快和我走!''

盲流之中爆发出了一阵低低的欢呼。一个牵一个的站起了身,他们开始手忙脚乱的去解身上的麻绳——白天他们可以自由的分散劳动,可一旦到了夜里,民兵还是要用麻绳把他们绑成一串。

无心掏出匕首,接二连三的割断绳结。等到几十个人全都行动自如了,他领头带队出了棚子,发现白狐狸还在摁着民兵发狠。盲流们自动排队络绎走出,万没想到棚子外面吱哇乱叫的女人,竟然是只大白狐狸。无心怕他们只顾看热闹,耽误逃生的时间,便回头疾言厉色的说道:``不要看,那是山里的狐狸大仙。''

大仙自然是不能亵渎的,盲流们在山里生活久了,对于鬼神之事都是宁可信其有、不可信其无。听了无心的警告,众人连忙恭恭敬敬的垂头经过了白狐狸,连声大气都不敢喘。无心领头快走,同时发现白狐狸一点儿也不给自己做脸。自己把她抬举成了大仙,可她压着个民兵大呼小叫,满嘴里就只有一个鸡!

盲流们深知逃生机会来之不易,而且全体不老不小,脑子清楚。有组织有纪律的排成了长队,他们无需无心嘱咐,很自觉的沉默疾行。飞快的走到了灯火通明的大院外,无心领头停了脚步,不敢再公然前进。没想到他们虽然谨慎,后方却是追上了一支无所畏惧的狐狸小队。大白狐狸依旧是打前锋,嘴角的白毛上还染着点点鲜血。无心怀疑她是刚对民兵行了凶,如今要去新鸡棚开斋了。

人的身手可是比不上狐狸灵活,所以无心不敢像白狐狸那样大摇大摆。等到狐狸一行通过大院了,他又观望片刻,见院子里当真是毫无动静,才带着盲流们高抬腿轻落步,一路悄悄的经过了院门。

农场的夜里十分安静,无心本来提防着有狗,然而一路上也并没有遇到狗影。农场太大了,有界碑没围墙。无心带着一大群人逃出老远,末了在一处山坳里停了脚步,他转身说道:``你们的家人还在村里,粮食也都还有。你们派个人回村里送个信报声平安,然后就到林子里躲一阵子吧!''

小全上前一步:``哥,你呢?你还和我们回去吗?''

无心摇了摇头:``我不回去了。本来我也不是长住,现在天气一天天的暖了,我也该继续上路了。''

盲流们乱七八糟的给他鞠了躬,小全则是拉着他的手不松开。无心仰头看了看星星,发现时间已经不早,便催促众人上路,让他们成群结伴的逃进山里去了。

只要进了山,这帮人就算是有了活路。无心长吁了一口气,认为自己起码是对得起小全。一屁股坐在雪地上,他累极了,想要歇一歇,可是未等坐稳,远方忽然出现了影影绰绰的光点。伴随着光点闪烁,人声狗声也一起响起来了。

无心一跃而起,以为是农场里的工人民兵有所察觉,现在要来捉拿盲流。盲流刚刚进了大森林,没有再落网的道理;自己孤零零的站在这里,却是太有危险。他六神无主的正要上树,不料火光在半路拐了个弯,原来目标并不是他。

无论目标是谁,无心都不敢再做停留。转身冲进茫茫夜色,他找苏桃去了。

\chapter{风雪夜}

无心一路攀援跳跃,在林子里东一转西一转,末了在一棵老树下面停了脚步,仰头对着树上的苏桃轻轻唤了一声。

苏桃缩在厚棉袄里,怀里搂着大猫头鹰。大猫头鹰虽然是个以和为贵的好妖精,但是嘴若金钩目如明灯,一脸凛凛的凶相。苏桃提心吊胆的蹲在树上,头脸全被包裹严了,唯有双手没有手套,只能掖在大猫头鹰的翅膀下。忽听树下有了动静,她低头望下一瞧,一颗心登时一轻,在围巾里面闷声闷气的叫道:``无心!''

无心站在树下一拍巴掌,然后向上张开了双臂。苏桃放开大猫头鹰,两条腿蹲久了,统一的僵硬麻木,两只套着大棉鞋的脚也成了冰砣。险伶伶的横向挪到一根粗树枝上,她气喘吁吁的做出预告:``我要跳了啊!''无心对她招了招手:``快!''

苏桃闭了闭眼睛,蒙在脸上充作口罩的一层棉布外面凝了一层白霜。自言自语的又咕哝了一句,她说:``我真跳了啊!''然后不等无心回答,她张牙舞爪蜷着腿,一头向下栽去。而无心高估了自己的胸怀与力量,苏桃从天而降,当场把他砸了个四脚朝天。合拢双臂抱住了怀里的苏桃,他先是狠狠一闭眼睛,随即呼出一口白色雾气,对着上方满天的星辰笑道:``桃桃,我成功了!''

苏桃下意识的想要挣扎起身,可是背着夜空对着雪地,她犹豫了一下,忽然想在无心身上再趴一会儿:``他们都逃了吗?''无心伸手去推苏桃:``逃了,全进山了。桃桃,起来,我怀里还藏着一条白娘子呢,别把他压扁了。''苏桃这才意识到了白琉璃的存在,立刻连滚带爬的起了身,又使出吃奶的力气扶起了无心。无心抬手抹去了她眉毛睫毛上的冰霜,然后攥住她的一只手,匆匆的继续前进。

一阵夜风掠地而来,卷起了一层白雪沫子;林中的树木随之打起了哨,声音如同鬼哭狼嚎。苏桃抬手扯下遮住口鼻的棉布,一路喘得呼哧呼哧,本来林子里已经天寒地冻到了极致,可是她在齐膝深的积雪中奋力调动着两只沉甸甸的脚,竟然走得头上热气腾腾。一只手伸出去和无心十指相扣了,手心也是汗津津的总不干爽。

大猫头鹰不消吩咐,自动的盘旋在他们上空。飞翔的速度自然大大的快于行走,他在前方飞飞停停,末了等得不耐烦,竟然试试探探的蹲上了无心的一侧肩膀。无心一手领着笨手笨脚的苏桃,一手拎着他的武器,怀里还暖着一条冷冰冰的白琉璃。肩上平白无故又加了好几斤分量,气得他一边走一边发牢骚:``你是只小鸟吗?你比老猫都重,装什么小画眉?''

大猫头鹰睁一只眼闭一只眼,装听不见。还是落在无心的肩膀上更安逸,否则飞快了不是,飞慢了也不是,还得时常东张西望,生怕半路跟丢了。无心知道他是个温吞性子,从来不受刺激,所以不得不多说几句:``你怎么还学会偷懒了?你又不是鸡,为什么非要让我扛着你?''

大猫头鹰稳稳的抓住了他的棉袄,钢勾似的爪子戳破了外面一层粗布。无心没发觉,他也不提醒。无心叹了口气,知道自己即便吵破了天,也只是一场独角戏。大猫头鹰天生的没脾气,而自己的双手都被占用,又没有余力把他从自己身上摘下去。

无心挣命似的往前走,先还遥遥的偶尔听到一两声枪响,后来周遭只剩了风声雪声,显然他们已经彻底远离了农场。苏桃实在是撑不起自己这一身装备了。弯着腰低着头,她恨不能走成四脚着地。身边的无心刚一停顿,她便一屁股跌坐在了大雪地里,上气不接下气的告诉无心:``累死了\ldots{}\ldots{}心都要跳、跳出来了\ldots{}\ldots{}''

无心跪在了她面前,先是摸了摸她的头脸,见温度不算低,便转而去脱了她的大棉鞋。苏桃的脚已经冻得没了知觉,摆成什么样是什么样,没了鞋袜也不知道冷。无心抓起一把雪放在手里搓了搓,然后握住了她的一只赤脚。搓过冰雪的手掌升了温,再去抓雪也不为难。

苏桃静静的望着他,心想他知道自己平时不怕冷不怕热,只有一双脚总是缺少热量。知道,也记得,自己都不记得了,他还记得。一只脚被他用雪搓热了,另一只脚又进了他的手中,一切都像是理所当然,无心微微低着头,搓着搓着忽然抬眼向她一笑:``热了没有?''苏桃也跟着笑了:``热。''

无心拿起鞋袜为她重新穿上,然后拍了拍手上的残雪。苏桃收回双脚系了鞋带,同时小声问道:``你呢?''无心撵走了肩膀上的大猫头鹰,同时发现这只坏鸟抓出了自己的棉花:``我?我不冷。''

苏桃用雪洗手,洗得手心发烧。起身走到无心身后,她用滚热的双手捂住了无心的耳朵。无心愣了一下,可也没有躲闪。苏桃的手,暖烘烘的,脏兮兮的,眼巴巴的,是非要为他做点什么的架势;掌心带着潮气,潮气又有温度又有力度,活蹦乱跳的温暖着他。

无心本来没打算在雪地上久坐,可是因为苏桃献宝一样伸出的两只热巴掌,他在雪地上跪出了两条小腿深深的形状。最后仰头转向身后的苏桃,他看到了一双黑白分明的大眼睛,原来苏桃弯着腰探着头,一直在居高临下的俯视着他,目光直勾勾的,几乎带了傻气。

无心收回目光,东倒西歪的站起了身:``不走了,我们找个背风的地方等天亮。''苏桃跟上一步,心中忽然很有话说。可是那话千头万绪,却又不知从何说起。说什么呢?说无心好?无心当然好,不用她说,说了倒显得生分。谈谈未来?未来自然还是流浪,况且大半夜的,也不是个畅谈的时候。

苏桃思来想去,想到最后抬头看了看身边的无心。无心正在数着星星辨认方向,一个脑袋仰到了极致,从耳根到下巴,是一道清晰柔和的线条。无心除了一双眼睛有些阴森,其余部分全都长得恰到好处。苏桃默默的凝视了他许久,倾诉的欲望渐渐消失了。

一切尽在不言中,她很笃定的相信在她和无心之间,早已存在了契约,虽然他没提起,她也没挑明。契约关乎着他们的一生一世,即便他不提起,她也不挑明。

在一道杂乱的灌木丛后,无心就地捡了几根枯枝,放心大胆的生起了火。农场的民兵们只要存有半分理智,就不会在深更半夜里追进森林深处,所以他们满可以尽情的点火取暖。大行李藏在山下,要等天亮才能去取,苏桃从怀里摸出两个棒子面饼子,放在火上慢慢的烤。饼子冻得好像石头,然而也能烤出一点甜香气。

饼子的表面略略焦糊了,表明这道夜宵已经可以入口。无心从苏桃的手中接过饼子,因为食欲澎湃,所以对着饼子张大嘴巴,还额外深吸了一口气。然而就在他要狼吞之时,大猫头鹰忽然慌慌的降落到苏桃身边,挤挤蹭蹭的往她怀里钻。无心咬着饼子抽抽鼻子,结果嗅到了浓郁的妖气。

灌木丛中起了沙沙的响动,一个皮毛蓬松的大白脑袋从一株矮趴趴的榆树后面伸出来了:``呀!你俩还吃上啦?''无心含着一口饼子,愁眉苦脸的把头一扭。而苏桃放眼一瞧,这回不需无心吩咐,很自觉的打了招呼:``狐狸好。''

大白狐狸龇牙一乐,满嘴鲜血,牙缝里还嵌着几根羽毛:``无心,你真是没个正经,有闲心去救别人,没闲心管管自己的丫头。瞧我大侄女多可怜,都冻成这个×样了。''无心把手一挥,恨不能一拳捶扁了她:``有事你请说事,没事好走不送。''

大白狐狸摇头摆尾彻底钻出了灌木丛,态度非常的好:``你吃不吃鸡?''无心不假思索的答道:``白吃当然吃!''话音落下,大白狐狸身后挤出了一只红狐狸。这红狐狸一嘴叼了两只大公鸡,鸡脖子全被咬得半断不断,两个鸡脑袋随着红狐狸的动作晃晃荡荡。

大白狐狸得意的瞟了死鸡一眼,然后自报佳绩:``今天算是过了瘾,现在农场里面只剩鸡崽子了!''无心盯着大公鸡,口水开始充沛:``大白,两只鸡都是给我们吃的?多谢多谢,我早就看你不是一般狐狸。这鸡可够肥的,算你豪爽大气。''大白狐狸一瞪眼睛:``想得美!姑奶奶这里没有白食给你吃!想要吃鸡,就得帮忙!''

无心眼里有了鸡,嘴巴就不思念饼子了:``看在鸡的面子上,我能帮一定帮。''此言一出,大白狐狸的身后热闹了,一只红狐狸驮着一只细条条的小黄鼠狼,闪电似的从灌木丛外飞跃过来。

原来大白狐狸素性嚣张,在农场鸡棚里由着性子作乱,既非正经偷鸡,也非正经吃鸡,而是肆意祸害,咬得遍地死鸡。农场里的工人受了惊动,叫了民兵出来救鸡,大半夜的也摸不清情形,只知道农场受了大损失,鸡棚内外到处都是鸡血。

大白狐狸是不怕人的,带着部下公然逃窜。红狐狸们也机警,唯有小黄鼠狼最弱,不但落了后,而且还被民兵用鸟枪打伤了后腿。一队狐狸中,只有大白狐狸法力高强,能够化成人形,可是心不灵手不巧,并不能充当医生;于是她灵机一动,决定追踪无心,让他出手去救小黄鼠狼。

正如她所料,无心看在鸡的面子上,很愿意帮这个小忙。把匕首放到火上燎了燎,他把细细长长的小黄鼠狼抱在腿上,用刀尖去挑它伤口中的铅弹。在他忙碌之时,大白狐狸不甘心安静旁观,没话找话的要和他聊:``无心,你明天去哪里?''无心大睁着眼睛低了头,攥紧了小黄鼠狼的细腿:``明天?明天我想下山,到县里去。''

大白狐狸把嘴一张:``你要走啦?''无心刀尖一颤,挖出了一枚小小的铅弹:``没错。总在山里住,非活成野人不可,再说现在山里也不算安全。''大白狐狸把嘴合上了:``嗷,我还挺舍不得你哩!''

无心发现小黄鼠狼的肉里还藏着一枚铅弹,于是聚精会神的继续去割伤口,疼得小黄鼠狼三个爪子乱蹬,口中咔咔乱叫。无心不为所动,专心致志的对着第二枚铅弹使劲:``大白,我不信。''

大白狐狸啐出一根鸡毛,顺便检讨了内心,感觉自己的确是没什么诚意。面前的无心忽然一抬头,鼻子里又低低的``嗯''了一声,正是第二枚铅弹顺着刀尖的力道弹入了火中。俯身把嘴唇贴上小黄鼠狼的后腿,无心连泥水带鲜血的吸了一口,紧接着扭头吐到火里。小黄鼠狼长条条的瘫软了身体,叫都不叫了。

从棉袄的破洞处开始撕,无心撕下了一条棉布,缠裹了小黄鼠狼的伤腿。红狐狸放下公鸡走过来,叼起小黄鼠狼一扭头,把它放到了另一只红狐狸的脊梁上。无心转身对着大雪地又吐了几口唾沫,然后笑眯眯的爬过去拽过了大公鸡。公鸡肥极了,而从现在开始到天亮,时间正够他和苏桃大嚼一场。

大白狐狸无意停留,临行前告诉无心:``其实有没有你我都是一样的过日子,所以我实在是装不出悲痛的样子来挽留你。你要滚就滚吧,兴许哪天我一高兴,也下山去逛一逛!''无心一边拔鸡毛,一边对着大白狐狸连连点头:``好,我就欣赏你这坦白的性格。桃桃,还不道别?''苏桃抱着大猫头鹰,很听话的出了声:``狐狸再见。''

大白狐狸扬长而去,留下无心和苏桃吃鸡。虽然缺油少盐,但是肉毕竟是肉,总比饼子香。两人很细致的啃出一地鸡骨头,然后在天亮之后下了山。从一眼老树洞里取出双肩背包,无心带着苏桃走出山林上了大路,凭着两只脚直奔县城火车站。

没有走出多远,无心和苏桃一起停了脚步,就见眼前路上平铺着一条挺新的小棉被,大猫头鹰收拢翅膀,睁着两只大眼睛站在小棉被上向他们行注目礼。无心弯腰细看小棉被:``哟,你还学会偷了?''大猫头鹰实在是懒得飞了,所以直挺挺的向后一仰,脑袋正是对准了棉被一角。

无心啼笑皆非,并且不想理他,然而苏桃福至心灵,却是领会了他的用意。把小棉被包裹成了襁褓形状,她抱起了大猫头鹰,又对无心说道:``抱就抱吧,权当是报答他给白娘子找鼠崽儿吃了。''无心不以为然:``哼,这夜猫子奸着呢,咱们谁也别想甩了他。''

作者有话要说:祝大家春节快乐,今天是一更啦O(∩\_∩)O\textasciitilde{}

\chapter{一路向北}

无心总是记不住自己所在的县城名字。长白山下本来是没有这个县的,是建国后才开发了这一片土地。县名非常的具有时代性,不是叫做团结,就是叫做建设,也可能叫做互助或者友爱。无心记不住,也懒得记,因为很快就要从县火车站出发,继续北上了。

带着苏桃走进县里唯一的招待所,两个人因为在山里生活久了,所以几乎忘记了山下是个什么样的世界。结结巴巴的背诵了一段毛主席语录,无心亮出自己的所有证明,登记之后得到了一间小屋子。

苏桃刚刚确定自己生了虱子,正在满头满身的做痒。生虱子本也不是稀奇事情,盲流村里的大小孩子全都有虱子,纵算其中有个别肯讲卫生的,也逃不脱外界的传染。苏桃与世隔绝的日夜缩在帐篷里,自以为可以出淤泥而不染,没想到防着防着还是没防住。当无心从她的头皮上捏起一粒虮子时,她先是吓了一跳,随即面红耳赤,身体像条独立的芯子似的,开始在棉袄壳子里乱动。

无心一派平静,没笑话她,也没安慰她,直接出门买回了药粉和篦子。解开苏桃的两条大辫子,他坐在床边,挑起一绺长发慢慢的篦了又篦。苏桃背对着他蹲在地上,听闻自己生了虱子,她从头到脚一起瘙痒:``无心,我会不会把虱子也传给你啊?''

无心轻声答道:``不会,我从来不生虱子跳蚤。''苏桃认为他是误会了自己的意思:``不是,虱子跳蚤是能传染的。''无心拧着一条眉毛,挑着另一条眉毛,因为知道好歹,无论如何不会认为虱子可爱。但是没办法,有些事情他不得不管,比如温暖着白琉璃不让他冬眠,比如整治处理苏桃身上的虱子。

``不让你抱夜猫子,你偏抱。''他喃喃的埋怨苏桃:``那夜猫子到处飞到处落,你知道他身上会有多脏?兴许虱子就是从他身上传过来的!''猫头鹰蹲在角落里,本来正是昏昏欲睡,忽然听到无心迁怒到自己身上了,便很委屈的睁开一只眼睛,偷偷的睃了他一眼。

苏桃不怕无心,不服他的话:``我和夜猫子之间还隔着一层小棉被呢,我又没直接抱他。''无心咬牙切齿的梳通了苏桃的发梢:``那小棉被也是来历不明。''苏桃抱着膝盖,随着他的篦子摇头晃脑:``是你先让我搂着它暖手的!''无心``嗯''了一声:``还嘴硬。''苏桃的头皮被他牵扯痛了,龇牙咧嘴的做鬼脸:``没嘴硬。''

白琉璃从无心的领口中伸出了脑袋,撕着大嘴打了个哈欠。本来他是一个无所谓饥饿疲惫的游魂,可是如今既然附上了蛇身,免不了就要受到躯壳的影响。昏昏欲睡的盘上无心的脖子,他对于外界的一切都不大感兴趣,懒洋洋的就只是想睡。角落里的猫头鹰打了个冷战,骤然睁大双眼望向了他;而他缓缓缩进无心的怀里,蹭皮贴肉的又睡了。

无心和这样一群活物混在一起,本来就胸无大志,现在越发的眼里只有虱子虮子。苏桃表面上和大猫头鹰很有共同之处,闷头闷脑的仿佛没脾气,然后大猫头鹰八风不动自有主意,苏桃像只猫似的叽叽咕咕,也是很会顶嘴,一边顶嘴一边又侧了脸用眼角余光瞄着他,怕自己说话说过了火,真激怒他。在外面出生入死风风雨雨的混了一年多,她自认为见多识广,已经很有一点小心眼了。

两人淡而无味的嚼了半天舌头,最后无心不言语了,专心致志的给苏桃抓虱子。苏桃稳稳当当的蹲在他的双腿之间,忽然有了主意:``无心,我把头发剪了吧!''无心受了白琉璃的影响,困得一双眼睛半睁半闭:``剪了?这么长的头发,剪了怪可惜的。''

苏桃抬手在耳朵下方比划出了一个长度:``就剪到这么长,不可惜,我头发长得快。''无心弯腰扭头,去看苏桃的侧影:``真剪?小姑娘还是留着长头发好看。''苏桃转向了无心,用手掌在脸蛋边缘一切:``我还没剪过短头发呢,剪到这里行不行?要不然就再留一点,你说该留多长?''

无心的黑眼珠半遮半掩的藏在眼皮后面,湿润而又迟钝的一转:``剪到下巴吧,到时候披散着也行,梳羊角辫子也行,还能经常换个样子。''苏桃笑了,嘴角弯弯的向上翘。无心是懂``美''的,而且是传统意义上的美,和她所受的家庭教育不谋而合。她越发感觉无心和自己是契合的了,契合,而又全新,因为家里常年的没男人,无心从天而降,在她面前把一切角色都扮演了。

无心找到了招待所的服务员,利用甜言蜜语借来了一把大剪刀。很谨慎的对着苏桃下了手,他剪羊毛似的为苏桃理了发。早就知道苏桃头发多,可是没想到吃了一冬天的野物之后,兴许是营养充足了,头发居然厚密到了不可收拾的地步。无心对于大事总是有一搭没一搭,对于苏桃的脑袋却是认真至极,从中午修剪到了傍晚,越剪越短,最后还是苏桃感觉出了不妙。趁着耳垂尚未露出,她起身强行逃走了。

带着无心给她买的药粉去了一家澡堂子,她含羞带愧的洗了许久。末了赶在天黑之前,她随着无心回了招待所。猫头鹰站在房间内的一张破桌子边缘,正在筹划着出去打猎。冷不防看见苏桃随着无心摸黑回来了,他睁圆了探照灯一样的大眼睛,就见苏桃脑袋特别大,仿佛是细脖子上挑了个大蘑菇。对于大猫头鹰来讲,这就算是怪物形象了。心惊胆战的横着挪了一小步,他一爪踏空,未等展开大翅膀,已经``咕咚''一声摔在了水泥地上。

房间里没镜子,无心开了电灯回头一看,也是强忍着没对苏桃咧嘴。若无其事的低下头,他催促苏桃快些上床睡觉。床是两张单人床,被褥全都又凉又潮不干不净,并且其中一张床还有残疾,一条腿东倒西歪的立不住。无心让苏桃和自己睡一张床,等到苏桃先钻进被窝里了,他便背对着苏桃盘腿坐稳,翻检着苏桃脱下的衣裤,想要除去残余虱子。

苏桃躺在被窝里,歪着脑袋看他的背影,看他像只大猴子似的端着肩膀缩着脖子,胳膊腿儿全是特别长。他穿的戴的都不好,因为不知道珍惜衣裳,导致形象比苏桃更像盲流。服装虽然糟糕,破烂冬装下面的身体却是比谁都好。苏桃受了母亲的影响,审美观总和主流格格不入。在当今这个如火如荼的革命大时代里,她还是坚定的认为小白脸才算美男子。

苏桃对着无心审视了许久,末了忽然发现了问题:``无心,你的头发怎么总也不见长呀?''无心没回头,是个要忙死的架势:``我家里人都这样,头发长得慢。''苏桃侧卧着打量他:``那也不能一点儿都不长啊!''无心头不抬眼不睁,快要把脸埋到苏桃的棉裤裆里:``我天生就这样,头发胡子都不长,汗毛也轻。正好,省了理发的钱。''

苏桃对他没有刨根问底的心,所以糊里糊涂的笑道:``刮脸的刀片也不用买了。''无心腾出一只手,从怀里抻出了昏昏欲睡的白琉璃:``我忙着呢,你和白娘子玩,玩累了就睡觉,不用等我。''苏桃接了白琉璃,其实还是糊里糊涂,不过真要让她细问,她也不知从何问起。白琉璃看了苏桃的新发型,惊得一吐信子,还以为自己是看到了蘑菇精。

无心嘴上不说,心如明镜,硬着头皮在招待所里住了足足一个礼拜。一个礼拜之后,他见苏桃的头发有所生长,看着不那么像蘑菇了,才把行李重新收拾了一遍,带着苏桃去了县里的火车站。火车站太小了,只偶尔会有过路的火车停留个一分钟半分钟。

无心和苏桃提前换上了一身春装,蛮不讲理的跳上火车,往罐头似的车厢里横冲。苏桃挎着书包,一手和无心相握,一手拎着一只网兜。无心后面背着帆布背包,前面捆着一只襁褓,拉扯着苏桃在车厢里开天辟地。他挤火车挤出了经验,行动如风,嗓门也大,一路且骂且走,将挡路的什物一概踩到脚下,气得一个老太太捧着一篮子鸡蛋左躲右躲,对着无心和苏桃的背影怒骂:``这两个玩意儿,真他妈缺德!''

火车的终点站是吉林市。无心和苏桃在吉林市住了小半个月,将当地的好风景看了个饱。及至在吉林市玩够了,他们漫无目的的上了火车继续北上。将沿途城市一座接一座的走了个遍,最后在这一年的六月,他们到达了哈尔滨。

同样是省会城市,哈尔滨就比去年的长春太平得多,打归打,但是没有打到天翻地覆的程度。无心和苏桃穿着利利落落的单衣单裤,除了永不离身的大包小包之外,苏桃身上又额外多了一只铁壳水壶;蘑菇头经过了无心的几次修剪,瞧着倒是比先前顺眼多了,只是前额留了一排齐齐的刘海,让她总像是与众不同。至于大猫头鹰,因为身体毛茸茸热烘烘,所以在这个夏天里彻底失去优待。他给自己预备的小襁褓,也被无心丢在火车站里了。

哈尔滨火车站是个大站,来自东南西北的几列火车一起到站,出站口几乎有了点人山人海的意思。无心照例是扯着苏桃披荆斩棘往外冲锋,苏桃牛似的低着头,恨不能头上长角顶出一条大路。好容易挤出了出站口,无心找个角落站稳了,见苏桃在,苏桃和自己身上的行囊也在,行囊里的白琉璃更在,这才松了口气,用手背给自己擦了擦额上的热汗。

未等他把汗擦净,苏桃望着远方开了口:``无心,你看,那边有个卖冰棍的。''说完这话,她拿眼睛去看无心,嘴里没提要吃冰棍,可是等待的姿态已经做出来了。无心紧了紧身上的背包,又抄起苏桃身上的水壶喝了一大口自来水:``没看见。''

苏桃在他面前,不是特别的要脸。他没看见,她就伸手指给他看:``要是有奶油雪糕就好了。''无心不大舍得在奶油雪糕上花钱,但是有些钱不得不花。十六岁的苏桃还可以归于孩子一类,他不想让个孩子活得无欲无求。领着苏桃走向前方的冰棍推车,他一边走一边和苏桃说话。

苏桃侧脸仰头看他:``你也吃一根。''无心摇摇头:``我不吃,我不爱吃。''苏桃告诉他:``你不爱吃奶油的,就买根绿豆冰棍。绿豆冰棍一点儿也不腻。''无心思索着答道:``我问问有没有红豆的,要是有红豆的,我就买一根。''

两个人认认真真的扯着闲话,把通往冰棍推车的一段路途说得津津有味。及至停在了推车的遮阳伞下,无心从衣兜里掏出一小沓整整齐齐的零钱,正要数出几张买雪糕,不料未等他把钱递出去,忽有一只大黑巴掌横空出世,把几枚脏兮兮的分币托到了推车后方的大婶面前。无心和大婶都吓了一跳,同时发现黑巴掌别有特色,居然只有四根手指,小拇指头齐根没了。

然后,一个熟悉的声音在无心身后响了起来,居高临下瓮声瓮气:``兵民是胜利之本,我要两根绿豆冰棍!''无心和苏桃一起回了头,近距离的仰视到了一张挺好看的黝黑面孔。而顾基莫名其妙的迎着目光一低头,当即对着无心和苏桃大叫了一声:``呀!''大婶本来正在开箱子拿冰棍,被他这一嗓子震得一哆嗦,气得大发牢骚:``这孩子怎么虎了吧唧的?买个冰棍吓我两跳!''

顾基对于大婶的抱怨充耳不闻,单是六神无主的后退一步,又求援似的回头往后看。无心和苏桃顺着他的目光望过去,就见在一带铁栅栏下蹲着个小老农似的青年,正在用一小条报纸卷旱烟末子。卷好烟卷叼住了,他一边伸手往衣兜里掏,一边抬起了头。遥遥的和无心打了个照面,他显然也是一愣。不过随即取下烟卷往耳朵上一夹,他撑着他那一身旧军装站起身,弱不禁风的对着无心点头一笑。

无心没出声,就见小丁猫瘦了一圈,本来是白白净净的娃娃脸,如今脏兮兮的花里胡哨,变成花狸猫了。大婶气哼哼的把两根绿豆冰棍直杵到了顾基脸上。顾基接了冰棍撒腿就跑,惊弓之鸟似的直奔到了小丁猫身边。把一根绿豆冰棍送到小丁猫手里,他畏首畏尾的往对方身后一缩,仿佛大狗熊躲在了小树苗后面。

小丁猫咬了一口冰棍,脸上隐隐露出了一点笑模样:``无心,巧哇!咱们可是好久都没见面啦!''然后他一边咔嚓咔嚓的大嚼冰棍,一边快步走到了无心面前。无心上下打量着他,只见他单薄成了十五六的半大孩子模样,一身的军装也是不干不净,腕子上虽然还带着一块手表,然而却是穷得买不起烟。

无心一味的看,一言不发,于是小丁猫笑眯眯的先开了口:``哎,你有钱吗?''无心十分狐疑,不懂小丁猫的用意:``干什么?你不会是想打劫我吧?''小丁猫把冰棍杵进嘴里,闭嘴撸下最后一块褐色的冰:``想什么呢?我看你还是不了解我。''

顾基颠颠的跑上来,把另一根冰棍也送到了他面前,原来顾基纯粹是个跑腿的,两根冰棍全归小丁猫一个人。无心趁机抢着问了一句:``你现在离开文县了?''小丁猫唆着冰棍一摆手:``别提文县,我跟那边早没关系了!你有没有钱?我有粮票,你要是有钱的话,咱们凑合着下顿馆子去!''

\chapter{谈话录}

小丁猫不知道从哪里弄来了那么多粮票,本地的全国的都有,是五颜六色的一沓子。无心看他和自己一样也是刚下火车,没有理由会存着一大把黑龙江粮票,心中就起了狐疑:``你是从哪里过来的?''小丁猫掀起宽宽展展的军装下摆,因为身体已经瘦到抽象,所以衣服特别的像旗帜:``我们是从齐齐哈尔过来的。''

无心怀着千言万语,不知从何问起:``你去齐齐哈尔了?''小丁猫从耳朵上取下了烟卷,叼到嘴上掏火柴:``我去?我是住!你不知道吧?我下乡了。''

旧报纸卷成了烟卷是个圆锥形,上宽下窄没有指头长,根本不禁抽。小丁猫三口两口吸到了头,扭头啐出了被唾沫浸湿的烟蒂,他吊儿郎当的笑嘻嘻,继续热情邀请无心和自己合作下馆子去。嘴上说着话,他一双眼睛躲在眼镜片后,不住的去瞟苏桃。

苏桃倒是很坦然,因为知道他是自己的手下败将,顾基虽然个子大,但也未必是无心的对手。作为占据上风的一方,她有种王者般的宽容。小丁猫看她,她不在乎;如果小丁猫敢蹬鼻子上脸,她想象了一下,耳朵里起了``砰''的一声空响,是她的双拳击中了小丁猫的两扇瘦排骨。

无心和苏桃没有户口,最缺粮票。小丁猫热情洋溢巧舌如簧,把他说动了心。转身从推车后面的大婶手里买了一根奶油雪糕,他决定和小丁猫合作一次,打一顿牙祭。

奶油雪糕冻得梆硬,为了彰显高级,外面还包了一层半透明的蜡纸。苏桃揭了蜡纸,在舔雪糕之前先舔了蜡纸上的残余奶油。无心扫了她一眼,看她舔得津津有味万分珍惜,于是第一次感觉苏桃变得像个野丫头了。

苏桃并没有留意到无心的目光,对她来讲,吃雪糕是种难得的享受,她小心翼翼的左舔一口右舔一口,无论如何舍不得真咬,一边舔一边又东张西望的跟着无心走,因为无心正在和小丁猫寻找饭馆。小丁猫显然不是第一次来哈尔滨,轻车熟路的走出火车站地界,他不吃则已,要吃就去大馆子里开斋。

三个人跟着他一个人走,先是步行了长长一段路,又乘了一段公共汽车,末了他们一起挤下汽车,到达了中央大街。中央大街是过去的老名字了,文革开始之后已经更名为反修大街。小丁猫兴致勃勃的踏上大街,把身后三人带到了一家大餐厅门前。

此餐厅本名叫做华梅西餐厅,如今顺应潮流,改名叫做反修饭店。名字改了,体面的外表可没改,无心随着小丁猫往里走,怀疑这小子是要趁机吃大户。钱要是自己的,他就不说什么了,小丁猫要吃就让他吃去;可钱是苏桃的,花一个少一个,他可不能拿着苏桃的小财产胡乱大方。

四个人捡了一处僻静位子坐下,小丁猫依旧是百事通,大刀阔斧的点了一桌子中餐。等到服务员走了,他才压低声音说道:``现在这里的好厨子都被打成苏修特务了,西餐味道不行,还是来几样炒菜合算。''隔着一张桌子,无心向他伸出了脑袋:``你说你下乡了?''

小丁猫翘着二郎腿,一手插在裤兜里。脑袋向后一仰,他枕着椅子高高的靠背点头微笑:``没错,我下乡了,现在就在那个——''他转向顾基:``叫什么名字来着?前几天不是刚有了个新名字吗?''顾基似乎是对于自己的存在深感不安,耸头耸脑的不看人:``生产建设兵团。''

小丁猫的细脖子在破烂了的领圈里转了转:``对,其实就是开荒种地。我刚去了没几天,可是你看我的手。''话音落下,他把一只苍白的巴掌伸到了无心和苏桃面前。巴掌薄薄的,掌心结着几片鲜红的血痂。``你看我是干活的人吗?''他摇头叹息:``可怜我这一身细皮嫩肉啊,妈的全葬送在扁担上了。''

无心捻了捻他的手:``你干什么活?''小丁猫翻了个白眼:``挑大粪。''无心盯着他看,满脸的不相信。顾基忽然机灵了,瓮声瓮气的为小丁猫作证:``他真是挑大粪,我也挑大粪,我天天帮他挑,他没劲儿,挑不动。''无心登时笑了,一双眼睛眯成细长:``真挑大粪啊?''

小丁猫收回了手,以一种很欣赏的神情审视着自己的掌心:``你控制一下,不要当着我的面幸灾乐祸。''无心勉强正了正脸色,然后告诉小丁猫:``好,我尽量控制\ldots{}\ldots{}嘿嘿嘿嘿嘿!''小丁猫听了他的笑声,登时抬手捂住了眼睛:``哎呀妈呀。''

顾基看了无心的反应,十分不忿,还要辩解:``现在挑大粪是好活儿,比种地强。挑大粪能偷懒,挑到半路还可以找地方休息。''无心忍住了笑,继续又问小丁猫:``文县的事业完了,你还可以回保定嘛!你当初不就是从保定来的吗?''

小丁猫清了清喉咙,又见神见鬼的环顾了四周,见天下太平,才嘁嘁喳喳的讲述了自己这下乡的原因。原来在他去年逃出文县之时,保定的联指总部也受到了新一轮的冲击,罪名是一号勤务员反对林彪。联指在几次三番的风雨中一直屹立不倒,可是如今这顶帽子实在太大,终于把他们压趴下了。

联指总部中的十常委,被解放军抓走了五个,其中就包括了小丁猫和杜敢闯。余下的五名常委之中,除了一号二号跑了个无影无踪之外,余下三人一直存着外心,此刻当即宣布和联指决裂,重起炉灶另开张,并且抢走了联指的大批军火。

这三人风云再起,姑且不提,只说落网的五常委算是倒了大霉,大热的天被关进仓库,吃喝拉撒都在里面,生活环境还不如蛆,而且天天挨揍。小丁猫是好汉不吃眼前亏,一打就服,让交待什么就交待什么,毫无保留的把罪行全推到了旁人身上,并且宣称自己患有精神分裂症。

军方的人万没想到联指十常委中还藏着一个精神病,当即对此展开调查,把小丁猫的父母拘了来。小丁猫的父母都是工人,出身是绝对没有问题的,家里除了小丁猫这个长子之外,还有个胖墩墩的次子丁小熊,娇滴滴的三女丁小鸽。惶惶然的坐在专案组人员面前,丁家父母有一说一,不敢隐瞒:在自家老大刚上初中的时候,他们的确是带着孩子去过医院,诊断结果也真是精神分裂症。不过老大越长越大,越大越正常,他们还以为孩子已经自动痊愈了。

专案组里的军人擦亮双眼,追着问道:``丁小猫平日有什么异于常人的特点吗?''小丁猫的母亲是个瘦长条的妇人,满脸都是心力交瘁贤良淑德。悠悠的叹了一口气,她开口答道:``哎呀妈呀,这孩子小时候可瘆人了,一点儿孩子样也没有,就像让鬼上身了似的,刚上小学就学会抽烟喝酒了。反正我和他爸都不爱管他,我们把他养大成人就算对得起他了。''

专案组没能从丁家夫妇身上打开突破口,转而去审问初中生丁小熊和小学生丁小鸽,也依旧是一无所获,因为他们的大哥一直不爱搭理他们。再去传唤了丁家的左邻右舍,他们所得的信息全都十分有利于小丁猫——老邻居们统一的认为小丁猫是个怪坯。

专案组几乎相信了小丁猫的病情,然而无论他是否真疯,事实摆在眼前,他的确是联指中的三号人物,对于文县大血战,他是要负责任的。然而就在专案组将要给小丁猫定罪之时,变故又发生了。

杜敢闯突然站了出来,表明小丁猫只在联指成立初期活动过,从去年年初开始,他就因病不再参与联指事务了。从六六年夏天到现在,小丁猫没有动手打骂过任何人,没有单独组织过任何一场武斗。至于文县大血战,陈大光应该负主要责任。是陈大光先动手,她才着手筹划反攻的。

情形陡然发生变化,让专案组措手不及。杜敢闯那一身横肉快速的熬干了,年轻的脸皮因为毫无准备,所以显出了松垮的老相,一颗颗痘子却是暴得此起彼伏,是一种脏兮兮的灼灼其华。丑陋而又坚定的站立在审讯室里,她调动出了最后的精气神,大包大揽的承担了所有罪名。虽然小丁猫不在场,可是她铿铿锵锵的高谈阔论,又是一次飒爽英姿五尺枪,又是一次天翻地覆慨而慷。

她挡在小丁猫的前面,替他往死路上走去了。

在这一年的冬天,小丁猫回家了。落网的五常委中,只有他一个得了自由。他瘦得像个鬼一样,狼吞虎咽的霸占家中有限的粮食。丁小熊是个老实孩子,大哥既然想多吃一口,他就毫不计较的少吃一口。丁小鸽则是对大哥有些崇拜,认为大哥是个落了难的革命英雄。至于丁家的老两口子,则是别有心肠。自家的儿子自家清楚,想起小丁猫的所作所为,他们算是怕了长子。长子要吃,就让他吃吧。

小丁猫在家里养了一个冬天和半个春天,养出了一身薄薄的膘。新的一年有了新的声音,上山下乡的号召渐渐响亮起来。小丁猫在保定一直活得心惊肉跳,生怕自己的老底不知哪天会再被人翻出来。所以躺在家里思索了几日几夜,他一挺身下了地,宣布自己要下乡当知青了。此言一出,老两口子登时乐翻——小丁猫早走早好,他们实在是供不起大儿子的烟酒糖茶了。

小丁猫主意一定,当即开始行动。听闻上海已经走了几十万人,山东的青年也是成千上万的往边疆去,他在家里对着地图盘算了一番,认为自己是早走早好,越远越好,能和保定一刀两断才妙。于是在这一年的春末时节,他作为一名知识青年,披红挂彩的来到了北大荒。

在和往昔岁月一刀两断的同时,他和大粪结下了新的情缘,并且意外的遇到了顾基。自从联指覆灭之后,顾基便一个人四处流浪。文县他是不想回了,街里街坊都知道他一枪毙了他父亲,虽然现在子女和父母决裂是潮流,可是人人心里都有一杆老秤,秤上的准星并不会随着时代轻易变化。

家乡没脸回,衣食住行也都没着落,他和小丁猫一样,迫切的要逃。在千里之外的异乡骤然见到小丁猫,他百感交集六神无主——照理,他现在满可以一拳捶死这个灭他满门的仇人,可他把小丁猫当惯了主心骨,见了小丁猫,他一颗心都落了地。

顾基没法子清清楚楚的去恨,只好糊里糊涂的去爱。和小丁猫在一起,他永远不吃外人的亏;而小丁猫一边保护他一边使用他,仿佛他是一匹好驴好马好骡子。

第一道菜上来了,小丁猫夹了一筷子肉往嘴里送:``无心,我不能总和大粪较劲。我得改变现状。''无心正在思索苏桃是否拥有上山下乡的资格,思索到了最后,他认为就算是有资格,也不能让苏桃去。他不能让苏桃挑大粪,也不能让苏桃干农活。与其让她去卖苦力,不如把她留在自己身边,自己至少还能给她一个自由自在。

``你怎么改变?''无心先给苏桃夹了菜:``不挑大粪,改挑别的?''小丁猫不置可否的笑了一声,起身走到服务员面前要了一瓶白酒。咬开瓶盖倒了一杯,他吱喽一声抿了一口,然后咂了咂嘴,颇为销魂的长吁了一口气:``我吧,就是不安于现状,明白吗?''无心看着苏桃吃菜,苏桃每吃一口,他心里就舒服一下:``明白。你要是能够安安生生的挑大粪,才叫奇怪。''

第二个菜也上来了,小丁猫伸长筷子,高兴的笑道:``哈哈,葱爆里脊!吃了一个多月的窝头咸菜,我掉了三斤肉,不过吃粗粮也有一个好处,就是让我拉得痛快!''无心把筷子往桌上一拍:``你是不是离了大粪吃不下饭?好不容易下次馆子,你说你——''

小丁猫满不在乎,连汤带水的往嘴里填肉:``吃不下没关系,我替你们吃。老实告诉你,我现在是有点儿后悔,我不该往北走,我应该南下去云南的。''无心来了兴趣:``南下干什么?你们不是到哪里都得种地吗?''

小丁猫伸手一指顾基,仿佛是要让他给自己作证:``我弄到了一台收音机,可以听到外国的电台\ldots{}\ldots{}''他把声音压成了耳语:``缅甸那边的华侨学生也在闹革命,反正我在国内也是担惊受怕,不如往远了跑。在联指混了两年,我也积累了许多经验,如果让我重新再来一次,我肯定不能弄得这么一败涂地。''

无心听了他的话,感觉是在听天方夜谭:``你就不能安稳几天吗?''小丁猫一摊双手:``我稳不住,我就喜欢玩人。如果这次闹革命还是不成,我想南洋那边又不破除封建迷信,凭我的本事,怎么着都能混口饭吃。''

无心吃了一口肥嫩的里脊:``你是挑大粪,还是闹革命,还是挑着大粪闹革命,我都没意见。''小丁猫对他眉飞色舞:``你跟不跟我走?我正需要你这样的人才。''无心大摇其头:``我不跟你走。你我志不同道不合,万一走到半路打起来了,也不好收场。老实告诉你,在外面混了一年,我也积累了许多经验。我当盲流当得挺舒服。''小丁猫用筷子一指他和苏桃:``你俩一起过上了?''无心摇了摇头:``我俩相依为命。''

第三道菜上来了,是白菜炒木耳。小丁猫见它是道素菜,便没急着去吃:``挺好,我和顾基也是相依为命。你有没有兴趣和我换一换?顾基一身的力气,可上九天揽月,可下五洋捉鳖。''顾基充耳不闻的咯吱咯吱嚼白菜。

无心挑了一块硕大的木耳给了苏桃:``少和我扯淡。咱们今天吃过这一顿饭,往后还是各走各路。我看你天生就是个惹是生非的货,怪不得你上辈子——你身上还有多少本地粮票?卖给我几十斤好不好?''

小丁猫端起玻璃杯,美滋滋的喝了一口白酒:``别提卖,我白给你。另外你再考虑考虑,反正你也没地方去,不如和我一起走。苏桃,你说呢?''苏桃没理他。

小丁猫是以看病为名请假跑来哈尔滨的。肥吃海喝的混了个醉饱,他心满意足的出了饭店,还要在街上来回散一散步。无心领着苏桃跟在后面,一边走一边低头清点粮票。正是入神之时,一辆吉普车忽然在前方刹住了,车窗打开,一个四五十岁的军人脑袋伸了出来:``是苏平平吗?''

此言一出,小丁猫和顾基不以为意,无心和苏桃却是一起钉在了原地——此时此地,怎么会有个陌生军人知道苏桃的学名是苏平平?

\chapter{新希望}

吉普车的车门开了,军人像要进一步作出确定似的,弯着腰跳下了车。手扶车门转向苏桃,他开口又问了一遍:``是苏平平吧?''

苏桃茫茫然的睁大了眼睛,答也不是不答也不是。无心握住了苏桃的手,一头雾水的看看军人又看看苏桃,末了他微微俯下身,在苏桃耳边问道:``认识他吗?''

苏桃咽了口唾沫,虚虚的反问道:``你是田\ldots{}\ldots{}叔叔?''

军人笑了一下,露出两颗可以媲美獠牙的大虎牙:``我说我不能看错么,还真是你个小丫头。''

苏桃没有笑,把头低下了。走在前方的小丁猫带着顾基停了脚步,饶有兴味的退到一边旁观。而军人上前一步又道:``你家的事情,我后来都听说了。你现在住在什么地方?怎么来了哈尔滨?''

苏桃的嗓子细成了线,说起话来嘤嘤嘤嗡嗡嗡,仿佛是存心让谁都听不清楚:``我也是刚下火车。''

军人一亮虎牙,很关切的又向前迈了一步:``来哈尔滨是有事?''

苏桃幅度很小的摇了摇头:``没事\ldots{}\ldots{}''

军人发现苏桃像只柠檬,不拧不出汁:``老苏出事之后,你有着落了吗?''

苏桃闭了嘴,因为不知道应该如何回答。说她没着落,可她有无心和一张做了假的结婚证,简直算是个终身有靠的人;但若说她有着落,她居无定所,差一点就是吃了上顿没下顿。流浪的生活,无论如何不能算是一种着落。

军人没有得到答复,于是收回虎牙,顺便看清了苏桃和无心握在一起的手。目光从苏桃转移向了无心,他和无心对视了一眼,然后感觉自己什么都明白了——老苏的丫头在外面混了一年多,可能是学坏了。

军人转身一指身后的吉普车:``平平,如果没地方去的话,可以和叔叔走。叔叔现在\ldots{}\ldots{}形势还行。''

这回未等苏桃做蚊子哼,无心先把她拉到一旁站住了。弯腰看着苏桃的眼睛,他郑重其事的问道:``他是什么来头?''

苏桃凑到无心耳边,嘁嘁喳喳的答道:``他是我爸爸的老部下。去年年初,他被人揪到北京去批斗了。''

无心的大黑眼珠在微凹的眼眶里滴溜乱转,是个心神不定的模样:``你信得过他吗?''

苏桃特地想了一想,末了告诉无心:``他是好人,当初救过我和爸爸。''

无心听到这里,就扭头再次望向了军人。军人饶有耐性的站在吉普车旁,本来当无心也是个东游西荡的野小子,然而冷不丁的被他盯了一眼,竟是心中一寒。那一眼的力道太足了,冷飕飕的往他脸上扎,简直就是霜刀雪剑。

无心一望即收,对着苏桃低声打商量:``他要是肯招待我们,我们就去吧。省一夜住宿费也是好的。''

苏桃现在已经很会精打细算了,虽然依旧是怕生,不过看在钱的面子上,她同意了无心的建议。抬眼望向军人,她扭扭捏捏的小声说道:``田叔叔,您能不能给我们找个地方住几天?我们\ldots{}\ldots{}我们初来乍到,没有地方安身\ldots{}\ldots{}''

军人竖着耳朵听清了她的言语。他去年自身难保,没能救成老苏,所以如今对待老苏唯一的一点骨血,他是有求必应:``好,好,上车吧,叔叔安排你们。''

小丁猫和顾基瞠着眼睛站在路边,看到无心和苏桃上了军人的吉普车。吉普车绝尘而走,让小丁猫十分艳羡的叹息出声:``莫非他们是攀上高枝了?''

顾基扬着一张晒黑了的脸,浓眉大眼高鼻梁,一脸男子汉式的好看。他显然不是小丁猫的知音,小丁猫盯着吉普车的后影,一双眼珠子快要突破眼镜片飞出去,而他站在一旁一言不发,只隔三差五点缀几声饱嗝。

吉普车流星一样在大街上疾驰,穿过了一世界的艳阳高照红海洋。末了停在一处不挂牌子的招待所门口,军人率先推开车门下了车。

无心没有再和苏桃手拉手,改用眼角余光牵着她扯着她。招待所外表看着不起眼,进入院内才发现里面风景优美,有花有草,通往楼内的大玻璃门太干净了,嵌在玻璃上的不锈钢门把手好像是飘在了半空中。有整洁利落的服务员从里面为他们拉开了大玻璃门,无心和苏桃跟在军人身后往里走,鞋底踏着厚实的地毯,一步一步软绵绵。

军人把他们领上了二楼。在一间窗明几净的屋子里,他们坐在一圈小沙发上,有勤务兵无声无息的端茶倒水。及至勤务兵退下去了,房门一关,房内无端的寂静了片刻。

最后,还是军人先开了口,他想知道老苏到底是怎么死的,也想知道苏桃是如何熬过了这一年半载的光阴。而对着田叔叔这么一张不甚熟悉的面孔,苏桃彻底成了个瑟缩乏味的丫头,把一切惊心动魄的故事都讲了个干巴巴,丝毫渲染形容都没有,纯粹只是讲述,并且是一场置身事外的讲述。军人对她是一边倾听一边审视,发现和去年相见时相比,她基本没变模样,要说变化,也就是黑了一点,不过大夏天的,人人都黑,不算稀奇。老苏的女儿其实一直是有名的,因为老苏长得不怎么样,女儿却是个水灵灵的小美人。女儿的大照片悬挂在老苏的办公室里,一年一换,由于父女二人对比强烈,导致往来的人都忍不住对着照片看了又看,私底下一致怀疑老苏让他老婆扣了顶绿帽子。

懒和尚念经似的喃喃完毕,苏桃没话说了,直着眼睛去看茶杯中的茶叶沉浮。茶是好茶,茶汤碧绿,一片茶叶在里面缓缓舒展,铺满了整个茶杯底。田叔叔原来并没有被真正打倒,当初看他摇摇晃晃的最危险,最终却是比父亲强,不但活着,而且穿住了一身军装,住在闲人免进的高级招待所里,``形势还行''。

可是对待这样一位堪称人物的叔叔,她一点眼色也没有,一句好话也不会说。冥冥之中似乎有所预感,她无欲无求的只想走。田叔叔当然是有办法把她从飘萍一样的生活中拯救出来,可是她回首往昔岁月,知道自己是回不去了。

她对于这个世界,对于这个世界上的人,已经是彻底的没有兴趣。她只想和无心在一起,有多远走多远,能走多远算多远。

她不说话,军人舔了舔大虎牙,也是沉吟。短暂的沉默过后,军人开始盘问无心的来历。苏桃静静的倾听着,听无心一口流利的谎言,假得天衣无缝,就像真的似的。等到无心自我介绍完毕了,军人起身走出门去,良久过后才又回了来。一屁股坐到苏桃和无心对面,他虽然也是昂首挺胸的摆出了军人姿态,可是后背微微的有些驼,肩膀也微微的有些塌,显然是大大的伤过元气。字斟句酌的开了口,他慢吞吞的分析了当今的天下大势,然后给苏桃画出了两条大路——在城里消磨光阴是肯定没有前途了,想要求生存求发展,只能另辟天地。凭着苏桃的岁数和资历,第一可以参军,第二可以下乡。他现在虽然是比不得先前有权力了,但是毕竟没倒,把个子弟安排进军队保险箱还是不成问题的;不过和参军相比,生产兵团里更像是广阔天地大有作为,如果真想干出一番大事业的话,倒是去北大荒更合适。

苏桃听愣了,万万没想到田叔叔竟然热心到为自己画好了人生蓝图。慌里慌张的看了对方一眼,她下意识的问道:``那无心呢?''

军人对着无心一点头:``小伙子,你有什么想法?''

无心俯下了身,把两边胳膊肘架在了膝盖上,是个埋头苦思的形象。双手十指交叉了,他抬起头,用一双大眼睛去看军人:``田叔叔,现在\ldots{}\ldots{}小姑娘去当兵,是不是\ldots{}\ldots{}也不算坏?''

军人听了他的问题,也说不出是哪里不对劲,总之听着就是很怪:``当兵是很光荣的事情嘛!这哪里要分什么男女?''

无心点了点头:``是,是,我知道现在和过去不一样了,现在当兵是好事。''

军人欲言又止的轻轻一呲虎牙,发现这个大眼贼说起话来居然老气横秋。

无心谁也不看,自己犹犹豫豫的又道:``反正那个生产兵团,我是绝对不会让她去的。''

军人发现无心年纪虽轻,可觉悟不是一般的低:``那个,我说一句。让娇生惯养的学生去农村接受再教育,也是很有必要的事情。再说一个青年人,应该到革命最需要的地方去,应该和工农相结合\ldots{}\ldots{}''

无心一边听一边点头,等到军人结束了长篇大论,他接着方才的话头继续说:``我和桃桃再商量商量,毕竟她是个小姑娘,无依无靠的,还是给她找个安稳地方最好。要是当兵不吃苦的话,去当兵也行。''

苏桃听他说得头头是道,越说越真,视自己为无物,终于忍无可忍的插了嘴:``田叔叔,无心能不能也和我一起去当兵?''

军人也是年轻过的,而且苏桃又是老苏的女儿,可以当成自己的孩子看待,所以没有绷着面子讲大道理:``平平,办法都可以慢慢想。''

这话说出了口,军人心中有些自得,认为自己总算对得起了老战友,不但负责了老苏的女儿,而且负责了老苏的女婿。哪知无心轻声说道:``田叔叔,我不当兵。''

苏桃睁圆了眼睛,下意识的作了回答:``你不当我也不当!''

军人紧随其后,一嘴的牙全见了太阳:``你个大眼贼,让你当兵你都不去,你这小子是不是缺心眼儿?''

无心抬了头,一个脑袋有千斤重:``田叔叔,我想和你单独说几句话,可以吗?''

苏桃被一名勤务兵领到了隔壁空屋子里,留下无心和军人相对而坐。无心像是累得挺不起腰了,含胸驼背的低声说话。他和军人之间当然是没什么交心之言,他所想知道的,无非是军中生活的模样:苦不苦?累不累?新兵进去受不受欺负?受了欺负能不能找到伸冤报仇的地方?像苏桃那样三棍子打不出一个屁的,进去之后能不能活?没有当兵当一辈子的道理,当完兵了有什么出路?苏桃能不能得到一份不受风吹日晒的工作?能不能活成个干净体面的小女人?

长达一个小时的询问结束之后,无心出门领走了苏桃。军人给他们另找了住处,距离招待所不远,一旦他们定下主意了,可以随时过来向他报告。

苏桃懵里懵懂的跟着无心走,一边走,一边摇晃着他的手臂:``要是咱们不能一起参军的话,我就不去。去了干嘛呀?不参军我不也是一样的生活?再说我也不想当兵,我妈最烦当兵的了,她要是活着,肯定不能让我往军队里进。你怎么了?你累啦?''

无心像乌龟驮碑似的驮着背上的帆布背包,一段路让他走得一步一顿。眼皮耷拉着遮住半只眼珠,他拖着苏桃和自己的两条腿,且走且呻吟了一声:``嗯,是累了。''

苏桃踮着脚去解他身上的背包:``我来背。''

无心一晃肩膀:``不用,马上就到旅社了。''

旅社是家大旅社,服务员提前接了军人的电话,所以只让无心一个人在簿子上登了记,也没检查证明。无心进了三楼的房间,卸下背包脱了鞋,要死似的往床上一趴,闭了眼睛就开始睡,一觉睡到了大天黑,一个梦都没有做。

最后朦朦胧胧的清醒了,他睁开眼睛向房内看,就见苏桃站在窗前,正在隔着一层纱窗往外张望。忽然撅嘴吹了一声口哨,她轻手轻脚的打开纱窗,放进了一只双目炯炯的大猫头鹰。猫头鹰收拢翅膀落在地上,有一点闲庭信步的意思,东张西望的寻找白琉璃。

白琉璃盘在枕头上,现在他长成了一条中等大小的胖蛇,放在书包里已经快要坠人的肩膀,所以时常也在背包里安身。虽然他一贯没什么人味,不过今天作为旁听者,他隐隐约约的也猜出了无心的心事。他和无心素来是志不同道不合,无心的一切作为他都不赞成,包括今天这一场。睁着两只黑豆眼睛凝视了无心,他看无心一口气都不喘,真是要累死了。

苏桃笑嘻嘻的站在床前,笑得不甚稳定:``无心,旅社里有公共浴池,能冲热水澡呢!一会儿是你先去还是我先去?''

无心闭着眼睛,一咬牙坐起来了:``你先去吧,我不着急。''

苏桃偷偷的瞟着他,同时从背包里翻出了香皂和毛巾。换上床底下的拖鞋,她像只怕被遗弃的家猫家狗一样,悄悄的开门出去了,脸上还带着一点儿笑意,笑给四面八方看,漫无目的的想要讨好卖乖。

房门关好之后,白琉璃像一朵云似的,飘飘忽忽的升到了无心面前:``无心,你不会是\ldots{}\ldots{}''

无心凝视着他,一言不发。

白琉璃略一思索,另起话题问道:``你不喜欢她了?''

无心轻声开了口,不知怎么搞的,嗓子还哑了:``我喜不喜欢她,你还看不出来吗?''

白琉璃看他情绪不好,所以难得的通情达理了,不和他一般见识:``那你还让她去当兵?我记得有句俗话,大概是`好男不当兵,好铁不打钉',你——''

无心一转身背对着他躺下了,气哼哼的抱怨道:``行了,你什么都不懂,还一直说说说!都什么时代了,现在当兵是美事,平常的人想当还没有资格呢!''

白琉璃看他给脸不要脸,居然还和自己耍起了脾气,就对着旁边的大猫头鹰一挥手:``去,啄死他!''

大猫头鹰迟迟疑疑的飞上床头,向下瞄着无心的一只脚,不知道应不应该马上出击。无心连着一天一夜没脱过鞋,一双穿着破袜子的脚看起来可是够有味的。未等他作出决定,房门忽然开了,苏桃蹦蹦跳跳的跑了进来,嘴里笑道:``嗬!哪是热水淋浴呀!放出来的都是冷水!''

白琉璃``嗖''的一下消失无踪,大猫头鹰则是松了口气。苏桃水淋淋的坐到床边,脸上笑得格外喜气,喜得不自然,像是生怕会有谁不喜。

无心东倒西歪的坐起来了,看了苏桃一眼。苏桃正在歪着脑袋擦头发,明眉大眼粉脸蛋看得无心一阵心疼。忽然又累了——他无涯的人生整个儿就是一场迎来送往,无休无止,无尽轮回。再爱也停不下,再好也留不住,累死他了。

``桃桃啊\ldots{}\ldots{}''他一下子上了岁数,足有成百上千岁,黑眼珠子停留在了蛮荒时代,历尽沧海桑田的望着苏桃:``你当兵去吧!''

苏桃没言语,擦头发的动作越来越慢。末了把潮湿的毛巾揉成一团放在桌子上,她言简意赅的答道:``不。''

无心垂头望着自己撂在大腿上的双手,一双手雪白雪白的,不见风雨不显光阴:``当兵挺好的,起码能让你活得堂堂正正。''

苏桃的预感成了现实。极度的恐惧转化成了愤怒,她一声不吭的下床出门,跑去卫生间里长长的撒了一泡尿。然后回到房内坐上床,她为了表示自己对于当兵一事的深恶痛绝,开始安安稳稳的赌气——她把自己里外都打扫干净了,现在不冷不热不渴不饿,满可以在床上直挺挺的坐上一夜。从来没和无心耍过小脾气,她决定今天要耍上一次,让无心知道他的念头有多无情多荒谬,自己有多难过多生气。

\chapter{交锋}

两张单人床相对着靠墙放了,一张床上坐着无心,另一张床上坐着苏桃。墙壁和床头栏杆构成了角落,正能让苏桃舒舒服服的嵌在角里,纹丝不动的在床上坐出个坑。她是个安静性子,装聋作哑以柔克刚是她的天分。她披头散发的垂着脑袋,目光隔着湿头发向外一扫一扫,倒要看看无心作何反应。

房内开着电灯,招来了一纱窗的大小蚊虫。纱窗半新不旧,并不能做到严丝合缝,于是无心走去关了电灯,只要窗外路灯的一点光明。黑黢黢的站在地上,在苏桃的眼角余光中,他成了个怯生生的大影子,欲言又止,欲走又停。

苏桃眨了眨眼睛,把前因后果来龙去脉重想了一遍,想到最后还是很坦然、很硬气:你还知道怯呀?你还知道不好意思呀?我还以为你要理直气壮到底呢!都说好了的,都约定了的,你说不算就不算了?你说推翻就推翻了?反正我不同意,我不干。我也是经过风见过雨的人了,我不是傻瓜。你要替我做主吗?我不听!

她越想越对,有理到了委屈的程度。压下一波泪水,她无声的做了个深呼吸,然后心平气和的放松身体,踏踏实实的窝进了角落中。她不是急性子人,必要的话,她可以开展持久战。

与此同时,无心像只心虚的猫狗一样,蹑手蹑脚的走到了她的床前。

``桃桃啊。''他俯下身,嗓子还是哑的:``你听我说——''

不等他讲出下文,苏桃直接从湿头发后面啐出三个字:``我不去!''

无心双手撑在床上,面孔距离苏桃已经很近。心力交瘁的低下头,他挣命似的发出声音:``桃桃,你应该去。你现在还小,不把流浪当成一回事,等你将来长大了,你会——''

苏桃根本不想领教他的高论,直接躲在湿头发后面放冷箭:``就不去!''

无心闭了眼睛,感觉自己的力气正随着语言向外流失。再说下去,他真能把自己活活说死:``桃桃,我都不知道今年冬天带你到哪里过冬。''

苏桃沉默了一瞬,末了答道:``我不怕冷。去年冬天能过,今年冬天一定也能过。''

无心的脑袋垂到极致,留给苏桃一副端端正正的肩膀和一后脑勺茸茸的短头发:``桃桃,当了兵,你就有了合法的身份,你就再也不必怕人了。''

苏桃盯着他,声音几乎堪称冷酷:``我谁也不怕。''

无心的手臂开始打颤,是终于撑不住了的模样。如果时光倒退几十年,除非苏桃自己愿意,否则谁也别想从他怀里抢走她。因为凭着他的小本事,他总能让苏桃安安然然的活过一生,他总能对得起她一世的年华。

可现在不行了,他没有户口,没有工作。在当今这个坦白从宽、抗拒从严的大时代里,他到了哪里都是异类,到了哪里都是行踪不定、来历不明。

流浪的日子,十天半月好混,一年半载也好混,一辈子,不好混。

撩起沉重的眼皮向前看,他看苏桃青春正盛,是一株含苞待放的花,太鲜艳了,太美丽了。所以他得给她找一处安身的温室,他不能让她再生冻疮和虱子。

慢慢转身坐到床上,他向后退到苏桃身边。靠着墙壁仰起头,他长长的叹出了一口气:``你必须去。''

苏桃冷笑一声,表示自己根本不拿无心的话当话听。

无心把脸转向了她,忽然不耐烦了:``笑什么笑?难道你还真想当一辈子盲流?''

他一变脸,苏桃也睁大眼睛抬起了头,万没想到他会舍得对自己发火。两人虎视眈眈的对望片刻,无心伸手一拎她的衣领,压低声音逼问道:``你看看你每天穿的都是什么?你再想想你每天吃的都是什么?我没本事,养不活你,什么都给不了你。你真跟我过一辈子,死了你都闭不上眼!桃桃,你别对我上心,没有用,不值得!''

苏桃猛的一晃肩膀,从他手中扯出了衬衫领子。衬衫还是去年穿过的,没型没款没颜色,和``美''有着十万八千里的距离。抬手一撩滴着水珠的刘海,她把脸扭向纱窗。气息颤悠悠的在鼻端打了个转儿,她从牙关之中挤出了含糊的一句话。

无心没听清楚,于是靠近了她问道:``你说什么?''

苏桃不看他,对着一纱窗的蚊虫蛾子开了口,声音夹了眼泪伴了哭腔:``当初都定好了的\ldots{}\ldots{}''

她用手背狠狠的一抹眼睛,咬牙切齿涕泪横流:``总在一起,不分开,都定好了的,还带反悔的?''

她不会嚎啕,再气愤再伤心也是喃喃自语,是谁爱听谁听的架势:``我没反悔,你先反悔了?你比我大了好几岁,还说话不算数?说好了的,说了好几遍,原来都是假话?''

她的眼泪迅速汹涌了,开始吭哧吭哧的又抽泣又哽咽,面红耳赤的对着满窗夏虫控诉:``苦不苦的我自己知道,你说苦就苦了?好端端的,非得让我当兵,不当还不行,凭什么啊?我不当,就不当。你爱当你当去,反正我不当。''

白琉璃无声无息的游上了床,盘到了苏桃的大腿上。苏桃伸手拢着他,谁也不看,只对着纱窗流泪。什么叫做``没有用''、``不值得''?无心说话太伤人心了。

无心抱着小腿,把下巴抵上了膝盖。太累了,他连花言巧语都说不动了。抬手揽住苏桃的肩膀,他要把人往自己怀里搂。第一下没搂动,第二下搂动了,他用袖子去擦对方滚热的眼泪。苏桃在他怀中抽抽搭搭,天大的委屈,委屈透了。歪着脑袋枕上无心的膝盖,隔着一层旧裤子,膝盖骨头的形状清清楚楚,硌得她太阳穴疼。无心真瘦,平时只看他东跑西颠活力无限,苏桃忽然发现其实他吃的不足喝的不足,所有的好吃好喝都被他填到自己嘴里去了。

苏桃一闭眼睛,眼泪又来了。

无心弯了腰,像条蛇也像只鸟,把苏桃卷着罩着护到怀里,面颊蹭过苏桃半干的头发,头发蓬松松的又厚又密,没有洗发膏,有香皂用香皂,有肥皂用肥皂,实在是什么都没有了,火碱也行——这么好的头发,给它用火碱!

无心不再说话了,双臂环住苏桃,他使劲的搂她抱她勒她,勒得她有了进气没出气,勒得她断了骨头连着筋。她是他偶然遇到的一线春光,她是他眼中花一样的小姑娘。他舍得让她去当兵?他舍得让她一个人出去闯世界?他舍不得,他最舍不得,可是这话,他没法说。

两个人一起侧身一倒,成了个相拥的姿态,双方的胳膊腿儿都嵌得合适极了,苏桃的脑袋正落在他的臂弯里。他轻轻的拍着对方的后背,低低的一句话让他说得声嘶力竭老气横秋:``桃桃,睡吧,有话明天再说。''

苏桃没吭声,把一张热气腾腾的面孔埋进了他的胸膛。

一觉醒来,天光大亮。苏桃肿着眼睛坐起身,发现无心已经出门买了油条豆浆回来。白琉璃盘在对面床上,一双黑豆眼睛定定的望着她。猫头鹰照例是蹲在角落里,灰扑扑的像一截矮木桩子。

她揉着眼睛往窗前的小桌上看,发现豆浆里面居然加了打散的鸡蛋花和红糖,简直稠成了粥。这时房门一开,无心端着水杯和牙具走了进来。

``来。''他嬉皮笑脸的开了口:``先刷牙,然后趁热吃油条。油条是用香油炸的,现在还脆着呢!''

苏桃从他手里接过挤好了牙膏的牙刷,心中有些恍惚。无心看起来太若无其事了,让她感觉昨夜的交锋不过是一场梦。无心把水杯也递给了她,顺手从床底下拉出了一只大痰盂。在她低头对着痰盂刷牙时,他又出去一趟,把湿毛巾也拧回来了。

苏桃擦过了脸,自己下床在桌前坐了。拿起一根油条咬了一口,她尝出了好滋味,立刻回头去看无心:``你吃了吗?''

无心走到床边坐下,紧挨着桌子答道:``吃了。''

苏桃现在不大相信他,捏着油条又问:``真吃了?''

无心笑了:``真吃了,在楼下的油条摊子上吃的,豆浆也喝过了。''

话音落下,他对着苏桃一掀身上的单衣,向对方展示自己的白肚皮。苏桃用手背又在他的胃部轻轻摁了一下,摁过之后心里有了数,知道他肚子里是真有食。

收回手喝了一口热豆浆,苏桃烫得一伸舌头。豆浆太甜了,内容太丰富了,让她不假思索的感到了痛心:``加鸡蛋和糖不得多花钱吗?日子不过啦?''

无心坐在一片明媚的阳光里,半张面孔被阳光照耀得要透明了。美滋滋的对着苏桃一笑,他开口说道:``等你当了兵,咱们的日子就好过了。''

苏桃一愣,舌头上的甜味立刻消失无踪。原来持久战并未结束,她怒发冲冠的想,他还想用糖衣炮弹哄我呢!

``谁说我要当兵了?''她粉嘟嘟的脸蛋瞬间冷成了苍白:``谁要当兵你找谁去!我不是兵,我是盲流。我没家没钱,我也吃不起豆浆油条。''

无心还是笑,笑出了一副没脸没皮的孩子相:``桃桃,昨晚的话我还没说完呢,你一哭,吓得我把下文都忘了。今天你给我一点儿时间,听听我的话到底有理没理,好不好?''

苏桃听他换了口风,和昨夜那副死气活样的德行大不一样,便起了好奇:``你说。''

无心清了清喉咙,又下意识的伸手抻过了白琉璃的尾巴尖捏来捏去:``桃桃,我是这么想的,凭着你现在的身份,唯一的出路就是去参军。昨天你那个田叔叔告诉我了,说是从军队里出来的人都会有户口和工作,而且还是好工作。桃桃,你自己说,是工作好,还是流浪好?''

苏桃不理他的话茬,直接问道:``那你呢?我去参军了,你怎么办?你干什么?''

无心答道:``我?我一个人总不会饿死。你到哪里当兵,我就到哪里生活。你能出军营,我就和你见面;你出不了军营,我也给你写信。等到将来你退伍了,要是不嫌弃我的话,我还跟着你。''

苏桃因为从不在他面前藏奸,所以此刻听他说得有鼻子有眼,脑筋不由得有些不够用:``真的假的?''

无心一点头:``我没户口没工作,谁要我谁吃亏,我骗你干什么?''

苏桃想了又想,没想出头绪,可心中像是松快了一些似的,让她能够低头喝下一口热豆浆了:``那你怎么不和我一起去参军呢?听田叔叔的意思,他肯定是能帮忙的。''

无心大摇其头:``我不干。我自由惯了,受不了约束。就算进了军队,不出一个月我也得当逃兵。''

苏桃开始咬起了油条:``那咱们都不当兵,咱们下乡去那个什么兵团吧!在兵团里不就是干活吗?我想干活的地方,纪律肯定不会太严。你看小丁猫和顾基不是说请假就请假了?''

无心把脑袋摇成了拨浪鼓:``桃桃,饶了我吧,我一不想当兵,二不想种地,我懒啊!你要是真心对我好,就乖乖的快去参军。我还指望着你以后有了出息给我养老呢!''

苏桃不置可否的连吃带喝,热得满头大汗。无心眼巴巴的看着她,不知道她会给自己一个什么样的答案。白琉璃长长的瘫在床上,颇为痛苦的一吐信子——尾巴快被无心揪断了!

苏桃喝光了最后一口豆浆,然后放下大碗一抹嘴,顶着一鼻尖汗珠告诉无心:``要不然,咱们还是一起下乡吧?北大荒是不是和长白山差不多?也有松鼠和狐狸吧?''

无心听闻此言,一拍大腿:``桃桃,你怎么又说回来了?我刚才的话全白讲了?''

苏桃舔了舔嘴唇,嘴唇都是甜的:``无心,只要我们能够常见面,干农活也没什么了不起的。''

无心把头一低:``不!''

苏桃叹了口气:``你好懒啊!''

端起大碗舔下碗边的一片蛋花,苏桃向他发出了最后通牒:``一会儿我就去找田叔叔,问问兵团到底怎么样,如果条件不是很差的话,我们就下乡去。当兵得当好几年呢,我不愿意和外人在一起过集体生活。''

无心快要哭了:``下乡不也是要过集体生活吗?难道你以为到了北大荒,我们还能搭座帐篷继续过小日子?''

苏桃忙忙碌碌的开始梳头:``白天干完了活,晚上见一面也是好的。''

\chapter{前途}

苏桃忽然来了精神头,豆浆油条在她的肚子里转化成了勇气与力量,她牵羊似的牵着无心往外走,一直走到了田叔叔所在的招待所。无心被她牵成了个别别扭扭的小男孩,走一步退两步,从头到脚全透着不情愿,又不敢实说内情——怎么说?说什么?想要吓唬小姑娘吗?

及至见到了田叔叔,苏桃的气焰略微有所低落,但是字字句句咬得清楚,是只口齿伶俐的大蚊子。现在苏家除了苏桃之外,其余人等已经基本死绝,老田对苏桃的提携照顾因为不求回报,所以格外显出了一种纯粹的赤诚。苏桃问一他答一,呲着虎牙心平气和,还给她抓了一把奶糖。苏桃接了奶糖,一直用双手捧着不肯放,等到把话说尽了,她彬彬有礼的起身告辞,顺便把奶糖全塞进了自己的衣兜里。

一出招待所的大门,她欢天喜地的高兴了:``无心,你听见没有?到了兵团还有工资呢,一个月三十二块钱!''

无心没言语,从她的口袋里掏出一颗奶糖剥糖纸。苏桃又扯了扯他的衣袖:``去兵团不比去农村当农民强?虽然都是干活,可兵团战士听着更好听呀!''

无心把奶糖塞进嘴里,因为苏桃满嘴都是理,所以他简直不知从哪里开始反驳:``冬天能冻死你。''

苏桃连吃奶糖的心思都没有了,一肚子的话是非说不可:``我又不傻。我自己不想冻死,谁还能把我绑在外面?正好田叔叔肯帮忙,我们办不出的手续,他全能帮我们办。''

奶糖粘在了无心的牙齿上,让他很不自在的舔来舔去:``听说还得体检,万一我体检不合格\ldots{}\ldots{}''

苏桃气得打了他一下:``人家有肺病肾病的都照样下乡了,你能有什么不合格的?''

无心把双臂环抱在胸前,愁眉苦脸的咽下了奶糖。真是小看苏桃了,他想,小丫头原来是茶壶煮饺子,心里很有数,平时不说而已。一夜一天之中她对自己围追堵截,自己现在除非耍横使蛮,否则完全不是她的对手。

无心不能对着苏桃耍横使蛮。唉声叹气的过了一天,翌日上午他们从田叔叔手中得到两张表格,坐上了吉普车前往医院接受体检——现在他们要什么没什么,连户口都不知所踪,自己唯一能做的,也就是先体检了。

医院位于市中心,距离招待所并不远,还没等吉普车开出速度,已经到了目的地。医院里面十分热闹,长长的学生队伍从楼里排到楼外,尾巴快要甩到医院大门口,人人手中都有表格,正是一大队接受体检的青年学生。学生们的表情有喜有忧,以喜居多。开车的司机沿着队伍来回走了一趟,末了见缝插针,把无心和苏桃塞进了队伍中央,好让他们少等一阵子。苏桃捏着表格,回头对无心说:``你看,楼里面是分成男女两队的,咱们还不能在一处体检。''

话音落下,她格外留意的看了看无心的面孔:``你怎么了?''

无心的脸白到泛青,阴森森的没热气,眼皮薄成了半透明,两只黑眼珠子在薄眼皮下光芒闪烁:``我\ldots{}\ldots{}桃桃,你说体检到底都检查什么?''

苏桃小声答道:``可简单了,就是听听心肺,走个手续。''

无心还要继续询问,可是后面有人不耐烦的推了他一下,他抬头向前一看,才知道队伍向前移动,自己也要进楼了。

楼是老楼,暗沉沉的没有生机,并且弥漫着强烈的消毒水味。体检果然只是一场形式,无心排在男生队伍里,缓缓穿越一间空荡荡的大办公室,前门进后门出。办公室里摆着几张办公桌和几只体重秤。医生坐在办公桌后,潦草的在体检表上大写草书。

无心心惊胆战的尾随在一名高大青年身后,按照顺序递上表格,张大嘴巴让医生看了自己的牙齿舌头嗓子眼。在体重秤上站了一秒钟之后下了来,他坐到了一张办公桌旁。神情倦怠的老大夫把听诊器往他胸前一贴,倾听片刻之后出了声:``听诊器坏了?''

没人回应老大夫,于是他转而把听诊器摁上了自己的胸膛。两道花白的眉毛皱了皱,老大夫自言自语:``没坏呀!''

然后他一抬头,发现面前的椅子空了,一张填了一半的体检表还留在他手边的桌面上。

苏桃在女生的队伍中走得脚不沾地,一转眼的工夫就拿着体检表出了楼。在楼前的一棵老树下,她找到了无心。上下将无心打量了一番,她开口问道:``你的体检表呢?''

无心低下了头:``桃桃,我们不去兵团。''

苏桃怔了怔,随即猛然一甩手中的体检表,当众把嗓音拔了个尖:``都说好了的,你又反悔!''

无心面无表情,淡定的像是故意要气人:``不去兵团。既然能当真正的兵,干嘛还要去兵团种地?桃桃,你听我的,去当兵。''

苏桃把手里的体检表一下一下甩得哗哗作响,恨不能把无心一并甩到万里之外:``无心,你真讨厌!你就知道落户口找工作,别的什么都不想!''

无心像块干干净净的顽石,在树下站得十分安然:``你要是有了户口工作,我的确是什么都不用想了。''

苏桃本来怀着一团火苗般的热情,结果无端的被无心兜头泼了一桶冷水,大夏天的,她冷成了个透心凉。把体检表狠狠的揉成一团,她真想再也不理他了。

无心和苏桃没有再坐吉普车。在步行回旅社的路上,无心给苏桃买了一根奶油雪糕。雪糕快要凑上苏桃的鼻子尖了,苏桃只装看不见。天热,雪糕眼看着在融化,浓郁奶汁滴滴答答的往下流。无心伸舌头舔了一口,然后告诉苏桃:``再不吃就化没了。''

然后他又舔一口,把勉强还算完整的雪糕往苏桃手里塞。苏桃松着手指头不肯接,无心便笑着逗她:``怎么不要?嫌我舔了两口?''

苏桃快要被他气死了,望着前方硬是不出声。

无心连陪小心带陪笑:``桃桃,别生气了。你我至多分开两三年,再说你在军营里,我在军营外,离得又不算远。等你当完了兵,我们的日子就好过了。到时候你有工作,我也能挣钱,我们找间小房住下,不怕人抓不怕人查,想吃什么就吃,想穿什么就穿。你自己想想,是不是好生活?是不是比到北大荒种地强?''

苏桃迟疑着接了雪糕,一口舔下去了小半根:``我发现\ldots{}\ldots{}你可会骗人了。''

无心把双手揣进衣兜里,扭头对着她笑:``不相信我啦?''

苏桃没言语,因为雪糕化得一塌糊涂,再不吃就吃不成了。

苏桃对无心言听计从惯了,在无心面前,她始终是精明的有限——没和无心耍过小心眼,如今让她现耍,她耍不出。

兵团是肯定不去了,她讪讪的回到了田叔叔面前,表示自己想要参军。老田听了,坦然的问道:``你当然是可以,但你的对象\ldots{}\ldots{}''

苏桃垂头嗡道:``他不想当兵。''

此言一出,老田虽然是省却了解释的烦恼,但是心中却也有些遗憾。如果无心真是狗皮膏药一样贴上他硬要参军的话,他看在苏桃的面子上,也是可以再想想办法的。

参军自然也是要体检的,而且是十分严格的体检,相比之下,上次在医院里的体检真是简单成了胡闹。大白天的,无心独自留在旅社里,数着时间等苏桃归来。抱着膝盖蜷成一团,他直着眼睛长久的发呆。

白琉璃在阳光不可及之处现了形。他依然保持着死后的形象,头发眉睫都带着寒冷的水意。歪着脑袋凑到无心面前,他轻声说道:``真的要让桃桃走吗?''

无心微微的一点头。

白琉璃又道:``她走了,谁陪我睡觉?''

无心气若游丝的吐出一个字:``我。''

白琉璃生前没有领略过异性的风情,死后却是明白了女子的好处。苏桃是香的甜的,软的绵的,偶尔慢吞吞赖唧唧,也别有一种趣味。想象着生活中再没有了苏桃,白琉璃一阵沮丧。

``她像夏天的花,冬天的雪。''白琉璃字斟句酌的对无心说道:``她没什么用处,可是因为有了她,风景才好。''

抬手作势去拍无心的肩膀,他一本正经的下了命令:``不要让她走。三个人过日子比较好,两个人太无聊了。你这张老脸我看了几十年,现在真是懒得再看。''

无心一挥手:``那你就滚回山里去!''

话音落下,他身边桌上的搪瓷杯子凌空飞起,开始在他的后脑勺上敲鼓。他一动不动的硬挺着,对于白琉璃是既不驱赶也不求饶。下意识中,他也认为自己是该疼一疼的。

苏桃天天出门,直奔走了一个礼拜,才算过了体检一关。

她在体检当中一直是不大配合,暗暗的希望自己会被淘汰下去,可谁知道她竟会有那么标准的身高和体重,那么结实的骨骼和皮肉——凭着她的条件,上天入地都够了!

政审的事情她插不上手,只能住在旅社里等消息。其实也不必等,因为田叔叔已经拍了胸膛做了保证,必能让她穿上一身崭新军装。

苏桃茫茫然的,有时候往远了想,想到两年三年之后,心里有一点快乐;有时候想得近,想到两月三月之后,又恨不能痛哭一场。

无心既不回首往昔,也不展望未来,天天只是琢磨着给苏桃弄点好吃的,一副``不过了''的气派。苏桃唉声叹气的吃胖了,脸蛋白里透红的饱满着,一双眼睛也是黑白分明。她买了一条新手帕,天天把白琉璃擦成玉雕。白琉璃夜里把脑袋挤到她的颈窝里,苏桃轻轻摸着他的脊梁,在黑暗中去问对面床上的无心:``蛇的寿命很长吧?''

无心答道:``长。''

苏桃又问:``等我当完兵了,白娘子是不是就长成大蛇了?''

无心受不了她的畅想,把脸埋在被窝里答道:``是。''

苏桃又道:``我走了之后,你别欺负夜猫子。它通人性的,你总打它,它不伤心吗?''

无心在被窝里喘气,喘得像是在哭:``嗯。''

苏桃不问了,噙着眼泪看窗外星月流转。看着看着,一个月过去了,两个月也过去了,到了第三个月,这一年的冬季征兵正式开始,老田也把她又带了出去。这时她已经从田叔叔那里得到了全新的身份,混在大批应征入伍的青年男女之中,她把先前走过的步骤重新又走一遍,然后顺顺利利的得到了一张入伍通知书。拿着入伍通知书,她知道除非有人翻尸倒骨的去刨她的祖坟,否则任谁也挑不出她的问题了。她脱胎换骨重新做人,已经成了一名光荣的解放军战士。

拿到入伍通知书之后不久,她又得到了一身新军装。军装尺寸正好,无心第一次看她穿正合身的衣服,单是一个合身就让她好看了许多。鼓着腮帮子站在无心面前,她嗫嚅着说道:``田叔叔说今年入伍时间早,下个礼拜他就要带我走了。''

无心说不出别的话来,弯腰为苏桃抻了抻军装下摆,他没话找话的问道:``用不用再剪一次头发?去理发店,让人剪得好看一点儿。''

苏桃把脸一扭,嘟嘟囔囔:``花那钱干什么?进了军队会有人给免费剪的。''

无心硬着头皮扯闲话:``别给你剪成秃小子。''

苏桃垂下了头,从喉咙里咕噜出声:``秃就秃吧,反正也没人看。''

无心苦笑了一下:``是,至少我是看不到了。''

然后他微微弯腰,失控似的狠狠抱了苏桃一下。苏桃现在用洗发膏洗头发了,头发香喷喷的又黑又亮。无心把鼻尖蹭进她的头发里吸了一口气,也说不出对她是怎样的一种爱,总之她还没离开他,他已经惦念的要死了到了临行前的最后一夜,苏桃和无心挤在了一张小床上。旅社的暖气烧得不好,夜里尤其更凉。苏桃像往常一样背对着无心侧身躺了,睁着眼睛不睡觉。她忽然想起自己已经和无心同床共枕了许久许久,并且下定决心要一辈子都在一起了,可是双方居然连个嘴都没亲过。

她从来没想过要和无心亲嘴,脑子里根本就没有过那个念头,然而此刻她挤挤蹭蹭的翻身面对了无心,发现无心也是同样的没有睡。

隔着一层衬衫,她试试探探的抬手摸了摸无心的胸膛。这胸膛被她依靠过无数次了,或是休息或是取暖,已经完全没有了神秘色彩。左手张开五指抚上他的心口,她没有留意到手掌下的平静,只是仰头对着无心的眼睛出神。

无心向她笑了一下:``怎么不睡?明天不是还要起早出发吗?''

苏桃轻声答道:``咱们说定了,你等我两年,不能再反悔了啊!''

无心在枕头上点点头:``嗯,不反悔。''

苏桃鼓足勇气,伸头在他嘴唇上啄了一下,啄完之后躺回原位,她的面孔开始缓缓升温。眼看无心逼近自己了,她没有躲,只是闭上了眼睛。

无心张嘴噙住了苏桃的嘴唇,没伸舌头,只吮了一下。他总感觉苏桃还小,是个小丫头。对于小丫头,他只舍得亲到这个程度。亲了一下,再亲一下,他忽然起身用棉被裹住了苏桃,然后把她紧紧的抱了个满怀。

苏桃嵌在了大号襁褓之中,不明所以的去看无心。无心仿佛是正在忍受着某种痛苦,双臂将她越箍越紧,双腿也是死死的夹住了她。白皙的额头不住的磨蹭着棉被边沿,无心发出了一声缠绵的叹息,然后搂着她翻了个身,又翻了个身。

苏桃感觉到了他的热度,并且看见他出了汗。她腾不出手去为他擦汗,只能莫名其妙的看着他压着自己辗转反侧。末了无心停了动作,走兽似的把她护在怀里。一动不动的沉默了,他无声无息,只是偶尔一抽搐。苏桃试着挣扎了一下,挣不开,原来无心并没有松劲。

苏桃喜欢无心的拥抱,躺在棉被卷里闭了眼睛,她也喜欢无心的温度与重量。她枕着无心盖着无心,想要睡了。

无心将苏桃搂抱了整整一夜。凌晨时分,他的热血渐渐冷却了,可是依旧不肯放手。他像一只无依无靠的大野兽,栖息在了小小的苏桃身上。侧脸凝视着苏桃的睡颜,他可怜兮兮的抿了抿嘴,想要再亲她一下,又怕惊动了她。

\chapter{两相思}

清晨时分,天还没有亮,苏桃就被无心叫醒了。无心钻进了她的棉被卷,把她搂到怀里抱了又抱。苏桃朦胧着一双睡眼没醒透,半睡半醒之中,就感觉有冰凉的鼻尖凑到自己耳根不住的嗅,然后是柔软的嘴唇贴上她的面颊,贴住之后长久不动。

她很安然的仰卧在无心的怀里,暖烘烘热腾腾的没睡够。连着闲了好几个月,她懒惯了,而且外面大冷的天,尤其让人留恋房内的被窝。灵魂一飘,她沉沉的又要入睡。无心的手臂横撂在她的肚子上,手指抓着床单,强忍着不妄动。

和苏桃朝夕相处了将近两年,无心仿佛今夜才第一次意识到了她的性别与年华。她在他身边一直活得像只猫,他几乎忘记了她不会永远都只是个小丫头。为什么会忘记?大概是因为她那怯生生的一脸孩子气,因为她那嘤嘤嗡嗡的一嘴孩子话,因为她的破衣烂衫永远比她的身体大一号。

其实最初他是怕她长大的,他怕她长大了,会引得狂蜂浪蝶来争来抢。她是个多好看的小姑娘啊,长大之后怎么了得?手指拧绞了床单,绊住自己不往上也不往下。苏桃真睡了,睡得呼哧呼哧有滋有味,还是小孩子的架势。无心仰脸望着窗外的天色,天边泛出一点寒冷的鱼肚白,时间不多了,真该起床了。

手指迟迟疑疑的松开床单,轻轻拍上了苏桃的腰间:``桃桃。''无心听见自己的声音在阴暗房间之中回荡:``你忘了?今天我们\ldots{}\ldots{}我们得起早啊!''苏桃在梦中听到了无心的呼唤。冷不防的打了个哆嗦,她睁开眼睛,忽然想起今天不是寻常日子。

苏桃没说什么,像个小影子似的起了床。五官面目全模糊了,她佝偻着腰低垂着头,小小年纪却是上了岁数,被一生的心事压矮了一截。无心比她动作快,洗漱过后下了楼,他给苏桃端上了豆浆油条。豆浆里搅了鸡蛋加了白糖,是给苏桃的特别优待。苏桃昨天洗了头发,一夜过后,正好蓬松得很有分寸,只是后脑勺上翘起了一撮。无心用梳子蘸了水,一遍一遍的给她梳头发,又说:``你吃你的,趁热吃。''

苏桃不吭声,吸吸溜溜的喝热豆浆。豆浆喝光了,油条也吃光了。其实她毫无食欲,然而不喝强喝,不吃强吃,豆浆油条在她胃里堵成了个大疙瘩。无心为她预备的这最后一顿早饭,足够她消化整整一天。

吃饱喝足之后,她扭头对无心说:``把白娘子也带上吧,它通人性的,我想让它也送送我。''无心看了白琉璃一眼,虽然嫌他是沉甸甸的一大堆,不过苏桃既然开了口,他便好脾气的点了头:``好,我带着他。''然后他把白琉璃拎起来塞进了书包里。

大猫头鹰一拍翅膀飞上了床尾栏杆,睁着两只大眼睛看看无心,又看看苏桃。苏桃伸手摸了摸他的脑袋:``我要走啦!''大猫头鹰什么都知道,对着苏桃一张嘴,他强忍着没有叫。苏桃不看无心,只对着大猫头鹰说话:``他要是再欺负你,可没有人救你了。''

大猫头鹰深以为然的闭了嘴,一双大眼睛滴溜乱转。无心斜挎书包,一手握住房门把手:``桃桃,走吧。''苏桃站着不动,垂头不语。无心静等片刻,末了拉起她的手,他一言不发的领着她往外走。

在步行前往招待所的路上,无心一直在说话,唠唠叨叨的,他也上了岁数。受了欺负怎么办,生了病怎么办,吃不饱穿不暖了怎么办\ldots{}\ldots{}他装着一脑子狡猾对策,此刻恨不能全部传授给苏桃。军营位于郊县,距离哈尔滨不算远,于是他最后又告诉苏桃:``你不是说三个月的集训过后,就能休礼拜天了吗?我不走,在哈尔滨等你三个月。三个月后我们见一面。''

他对着苏桃笑:``三个月,很快的。''苏桃扭头问他:``要是军营里一点儿也不好,我挺不过三个月呢?''无心默然无语的微笑片刻,片刻之后他答道:``我每天下午都会去一趟东方红百货商店,你要是当了逃兵,就到那里找我。''

用力攥了攥苏桃的手,他踏过满地白霜:``桃桃,别怕,我离你不远。''苏桃转向前方,气息哽在喉咙里,她费了天大的劲,才发出了一声含着泪的``嗯''。

在招待所门前,他们见到了老田,以及老田的警卫员和吉普车。老田去年大难不死,现在是个独善其身的状态,不显山不露水的享受着自己那点小特权。他家里没女儿,只有三个虎背熊腰的大小子,统一的继承了他的利齿,乍一看宛如三只猛兽;如今来了个娇滴滴的半大姑娘让他关怀,他还关怀得挺有兴致。

苏桃和老田打了招呼。看到吉普车敞开的车门,她知道自己这回是真的要走了。白琉璃从书包中伸出了个小脑袋,偷偷摸摸的去看苏桃。无心也放开了苏桃的手,轻声催促道:``桃桃,上车吧。''

苏桃随着老田走向吉普车,开头几步走得很乖,是一去不回头的架势,可走着走着就不对劲了。停在吉普车前一转身,她忽然对着无心一咧嘴,眼泪瞬间淌了满脸。漂亮的脸蛋走了形,她把小嘴咧成大嘴,没遮没掩的哭出了声:``不想去了\ldots{}\ldots{}''

十六岁的姑娘哭成了六岁,是最笨拙的一种哭法,是最难看的一种哭法,她没什么有理的话可说,只能躲在涕泪后面耍赖:``无心,我不想去了\ldots{}\ldots{}''

无心不动,因为害怕自己一旦迈了步,会将苏桃一把扯回自己身边。老田替他动了手,摆弄小崽子似的把苏桃往吉普车里推。苏桃身不由己的上了车,一手死死的扒住车门,她探出脑袋,这回真是一句话都没有了,她遥遥的望着无心,发出了一声尖利的嚎啕。

无心被她震得一颤——那是婴儿才有的哭声,没心没肺而又撕心裂肺,存在于一切语言之前,是最原始最赤诚的悲怆。下意识的上前一步,他看见老田把苏桃那四处乱攀的手脚全收拾进了车里,随即一弯腰也上了车,老田彻底堵住了她。车门``咣''的一关,吉普车哇哇的哭着走了。

无心慢慢的走回了旅社。进房之后关了房门,他摘下书包随手一扔,然后一屁股坐在了床上。俯身用手捧住了脸,他沉默良久。末了抬头向上望去,他看到了飘在面前的白琉璃。白琉璃面无表情,和他对视。大眼瞪小眼的静了片刻,无心直起腰,忽然一笑:``你看,现在又只剩我们两个了。''

白琉璃似乎是懒得理他,一转身穿墙而出,溜了个无影无踪。无心望着他消失的方向,大声问了一句:``这怪我吗?你忍心让她人不人鬼不鬼的和我混一辈子?你忍心我还不忍心!''白墙上隐隐浮现出了一双蓝眼睛,是白琉璃在对他怒目而视:``为什么不忍心?你又不是没找过女人!''

无心弯腰去解鞋带,感觉自己和白琉璃说不通。而白琉璃从墙壁中伸出了脑袋,居高临下的俯视着他:``你对桃桃到底是特别喜欢,还是特别不喜欢?''

无心脱了鞋,然后抬头对着对面的单人床怔了一瞬。苏桃白天总爱在那张床上躺躺坐坐,她是个安静性子,一条手帕也够她摆弄个小半天,玩都玩得没气魄。现在床空了,只摆着一只书包一只背包,曾经是他和苏桃的全部财产。

无心不看了,抬腿上床往下躺。白琉璃是真迷惑,所以从墙壁中探出了上半身,不依不饶的追问:``你为什么不喜欢她?''无心翻身背对了他,闭上眼睛轻声答道:``白琉璃,别吵了。你让我睡一会儿,我快累死了。''

无心睡了整整一天。傍晚时分他搬了家,随着老田派来的警卫员离开了旅社。在哈尔滨工业大学附近的一幢老楼里,无心得到了一套空屋子。警卫员传达了老田的意思,说是他可以在这里随便住。

无心道了谢,又问警卫员:``桃——苏平平今天哭了多久?''警卫员答道:``她进了军营之后就不哭了。''无心又问:``是她让田叔叔给我找的房子吗?''警卫员一点头:``是。''

无心不再问了,等到警卫员离开,他巡视了自己的新领地——一共是里外两间屋子,先前的主人应该是个不俗的人物,因为仅存的几样家具都是精致东西。里屋是抄家没抄干净的模样,墙角堆着一座乱七八糟的书山,按照当今的标准来看,全是毒草,而且还是外国毒草,书页上印着的都是外国字。照理来讲,毒草应该早被付之一炬,之所以留存至今,也许只是因为小将们革命革得虎头蛇尾,把它忘了。

寒风吹透夜色,刮得楼外墙壁上的大字报哗哗作响。楼内楼外没有人声,无心出门走了一圈,没看到几户人家亮着灯。老楼被大字报糊成了白色,他一张接一张的慢慢读,得知此楼曾经住满了资产阶级反动学术权威,如今权威和权威的家人哪里去了?他想不出。

无心不饿。回到二楼房内,他锁严了门,然后抱着膝盖坐在了角落里。不知道桃桃晚上吃的是什么,他默默的想,也不知道军营里发的被褥够不够厚。小丫头们厉害起来可是了不得的,他真怕苏桃会受欺负。

在无心胡思乱想的同时,苏桃已经钻进了宿舍床上的冷被窝。一间宿舍里面睡着六名小女兵,除了她之外,其余五人都是戴着大红花乘火车来的。六个人从上午开始相处,此刻到了夜晚,苏桃还认不清她们谁是谁。

认不清,也懒得认,爱是谁是谁,和她没有关系。仰面朝天的躺在上铺,她只感觉四野茫茫,自己是躺在了无边无垠的荒原上。她想无心,想得心里一抽一抽的痛,早上把眼泪哭尽了,于是她现在痛得干巴巴。忽然抬手摸了摸脸,她仿佛刚刚彻底清醒,记起了无心曾把嘴唇贴上自己的面颊。

在宿舍里低而兴奋的窃窃私语声中,她自顾自的回首往昔,想起来的全是美事。悄悄的向旁边挪了挪,她想象着无心还在身边,自己给他留出了一人多宽的地方。

似乎只是一闭眼的工夫,一夜就过去了。翌日凌晨天还没亮,一宿舍的小姑娘已然全被班长唤醒。松软的新棉被被拖到了地上,她们开始了今天的第一课:和班长学习叠被。

棉被带着女孩子们的体温,东一条西一条的摆了一地——床太小,非得在地上才能铺开。有人端着一盆冷水回来了,在班长的命令下,六个小姑娘一起撩水往棉被上洒,因为棉被只有潮了重了,才能叠成棱角分明的豆腐块。苏桃知道自己动作慢,所以一刻不停,忙忙碌碌细细致致,力求不领先也不落后。一个小姑娘一边叠被一边起了疑问:``班长,晚上被子能干吗?不干的话,怎么盖呀?''

话音落下,她挨了班长一顿臭骂。至于问题本身,则是没有得到答复。一天的军事训练过后,六个小女兵东倒西歪的回了宿舍。棉被果然还是潮湿不堪的,不盖被比盖被更舒服。苏桃已经学得很能对付,在军营里对付着吃对付着穿,对付着训练对付着睡觉,一颗心不是飘在过去就是飘在将来,唯独不看当下。

新兵训练进行了一个礼拜之后,开始有人挨揍。苏桃是田首长亲自送到军营里的,连队的干部心里有数,所以和旁人相比,苏桃还算是受了优待。穿着解放鞋站在初冬的大操场上,她一边随着号令踢腿练习正步,一边望着天边的太阳出神。下午了,无心一定正在东方红百货商店门口游逛。

东方红百货商店本名叫做秋林公司,坐落在一处很繁华的十字路口。商店门口总有买冰棍的小推车,自从决定参军之后,她时常会对着无心耍小脾气,一耍脾气无心就给她买奶油雪糕。她吃得太慢了,一根雪糕够她从大街舔回旅社。

苏桃心里一想无心,就感觉训练的时光也不算太难熬,冻僵了的双脚狠狠跺在地上,也不是疼得不能忍受。前方起了一声脆响,是班长用皮带的铜头抽打了一名女兵的小腿。苏桃心里一惊,立刻昂首挺胸抬高了腿。好汉不吃眼前亏,她犯不上自己找打。

\chapter{光阴}

大猫头鹰在凌晨时分回了家。收拢翅膀落在二楼窗台上,他从窄窄一道窗缝里挤进了房。一屁股把窗扇拱成严丝合缝,他振翅落上了窗户旁边的破衣帽架。屋中地上摆着一本书,书页正在缓缓的自行翻动。一身羽毛乍了一下,他很舒服的低低嗥叫一声,知道那是白琉璃在读书。

白琉璃不抬头,读书读得入了迷。眼前忽然掠过一只雪白的手,他发现无心不知何时走了进来。无心已经连续一个礼拜没吃东西了,黑眼睛陷在了大眼眶里,鼻子和下巴都显得异常尖削。把手里的英文书哗哗翻了一遍,他看不懂,把它依照原样又摆回到了白琉璃面前。

``我饿了。''他慢吞吞的转身扶了墙壁,摇摇晃晃的往外屋走:``我要出去找东西吃。''白琉璃现在不大关心他。百无聊赖的垂下头,他有一搭没一搭的继续读书。

无心穿着一双来自黑市的翻毛皮鞋,顶着寒风出了门。城市大,市场多,总有地方能让他空手套白狼的打食。苏桃参军之前,他们一共剩了一百多块钱。苏桃说在军营里无处花钱,所以只拿走了零头,余下的钱全给了他。他舍不得花,因为三个月的期限还没有满,他不知道苏桃到底能不能在军营里呆住。如果在军营里真被人欺负狠了,他想着,自己还得带着苏桃走。

他是早上六七点钟空手出的门,九点多钟顶着一头小雪花回来了,手里多了一只来历不明的小菜筐。进门之时他咳嗽了几声,想要咳出体内的冷空气。关闭房门进了里屋,虽然里屋也没什么好,不过盘踞着一只鬼魂一只妖精,总能让他感觉自己并非孤家寡人。把小菜筐放在地上,他随之一屁股也坐了下去。掀起菜筐上盖着的几大片冻白菜叶子,他从里面掏出了三枚红皮鸡蛋。白琉璃伸了脑袋向内瞧,发现筐里还藏着一截很鲜嫩的肉骨头。

无心掂着手里的鸡蛋,首先想的是它富有营养,应该留给桃桃吃,随即他意识到桃桃已经不在身边了,以后自己再弄到了好吃好喝,也都不必留了。

把鸡蛋往墙壁上一磕,他仰起头,直接把蛋清蛋黄打进了自己的嘴里。低头闭嘴咽了鸡蛋,他从筐里捧出了那一大块肉骨头。国营肉铺的营业员一定想不通这块肉是怎么没的,因为他在肉摊前面根本连停都没停。没人知道他的手有多快,他连松鼠野兔都能徒手捕捉。

望着肉骨头愣了愣,他又出了神——加几碗水就能煮成一锅好汤了,够桃桃喝好几顿的。苏桃在,他就不怕辛苦不怕麻烦,愿意把日子过得复杂繁琐有滋有味;苏桃不在,他做出花来也是独自欣赏,做不做的又有什么意思?牙齿衔住鲜肉向下一撕,他的嘴唇蹭上了淡淡的鲜血。一边咀嚼一边望向窗外,小雪下得越来越急了,他只希望今年冬天不要太冷。

一截肉骨头被无心啃得斑斑驳驳。吮净最后一点油水之后,他扬起骨头向前一掷,正好投中了落在衣帽架上的大猫头鹰。大猫头鹰正在打瞌睡,猝不及防的受到袭击,当即一头栽倒在地。仓皇的拍着翅膀飞上窗台,他不明所以的睁开眼睛,就见无心虎视眈眈的盯着自己,下半张脸布满斑斑血迹。一颗心在胸膛里翻了个跟头,大猫头鹰吓得爪子一软,当场从窗台边沿滑下,``咕咚''一声在地上摔成了个光屁股小男孩。

一本英文书骤然飞到了半空中,是无聊至极的白琉璃被他逗笑了,撒着欢儿的扔起了书。大猫头鹰不知道自己是怎么变成小男孩的,尖嘴利爪全消失了,他惊恐的张开了嘴,露出一条尖尖的鸟舌头:``嗥!''白琉璃听了他的叫声,越发哈哈大笑。无心也跟着他笑,笑着笑着忽然不笑了,转向白琉璃问道:``你在笑什么?''

白琉璃抬手指着大猫头鹰,笑得前仰后合:``他真像你!''无心想了一想,没想出这有什么可笑的。不过他早就认定白琉璃的脑筋有点问题,所以此刻也不和对方一般见识。起身走到战战兢兢的小男孩面前,他摸了摸对方的黑头发,然后背对着他向下一蹲:``上来!''

小男孩张开双臂一扇,两条细胳膊没能带动自己的身体。意识到了自己如今已成人形,他六神无主的向前一蹦,一下子蹿进无心的手里了。无心背着小男孩,屋里屋外的来回走。走到白琉璃面前停了脚步,他低头问道:``当爹就是这样吧?''白琉璃抬起头:``我不知道。我的儿子没有长大,我没背过他。''

无心换了个问法:``我像爹吗?''白琉璃审视着他那半脸血,感觉他今天格外的没人样:``不像。''无心托了托背上的小男孩:``叫我爸爸。''白琉璃莫名其妙的向后一飘:``爸爸?''无心不耐烦的叹了口气:``我没有和你说话,我是让他叫我爸爸!我何德何能,会养出你这样的货?''

白琉璃张着嘴对他眨巴蓝眼睛,片刻之后终于出了声:``第一,他不会说话;第二,你是不是想挨打?''无心并不想挨打,尤其里屋堆着一座书山,导致白琉璃的武器十分充足。背着小男孩走向外屋,他且逃且怨:``我和你们真是过不下去了!''白琉璃没理他,因为感觉他嘴贫人贱,一打便跑,真是不值一理。

无心从背包里找出一身苏桃穿过的旧衣,套在了小男孩的身上。背着小男孩出了门,他继续装爹,从一条街外的小商店里买了纸笔。及至冒着小雪回了来,小男孩已经冻得没了热气。

他把小男孩放到了白琉璃身边,然后自己在外屋的地面上摊开纸笔,跪趴在地上开始给苏桃写信。白琉璃听外面半天没有动静,忍不住穿透墙壁探头去瞧,结果就见无心握着一根花花绿绿的长铅笔,屁股撅得比头还高。一手托着脸蛋,他歪着脑袋抿着嘴,一边写一边把两道眉毛皱成八字,仿佛随时预备着要哭一场。

小男孩也从门口伸出了脑袋窥视。看过一眼之后缩回了头,他抱着手臂蹲稳当了,认为无心好可怕。

无心在地上撅了一个多小时,写出了一封长信。下午出门把信投进了邮筒里,他独自走去了东方红百货商店。多少年没给人写过信了,他也不知道自己的写法对不对,信件能不能到达苏桃所在的军营。总在商店内外乱走也不是长久之计,革命群众无处不在,并且全把眼睛擦得雪亮,真要是有好事之徒盘问了他,兴许真能盘问出事。无心沿着大街来回溜达,心里知道其实自己徒劳无功是好事,万一真是大白天的等来了苏桃,才叫糟糕。

下午三四点钟的时候,他回了家,拎着他的小菜筐又去了菜市场。国营菜市场规模很大,临近下班时间,里面人头攒动,买点什么都要拼命。无心在人群里东一钻西一钻,袖口拂过熟食摊子,他在一笸箩大馒头前踉跄了一下。大冬天的,蔬菜稀少,他扶着一摞大白菜站直了腰,收回手再拎菜筐时,菜筐表面就被白菜叶子盖严实了。

拎着脏兮兮的菜筐回了家,家里没人搭理他。白琉璃和小男孩模样的大猫头鹰一起从里屋门口探出了头,看到无心盘腿坐在暖气管子旁,正在往掰开的热馒头里夹猪耳朵。现在他是放开手脚做贼了,原来当着苏桃的面,他总想做个好榜样。苏桃懂得什么?万一跟着他学成了女飞贼可怎么办?

他还是想苏桃,热馒头和猪耳朵配在一起,滋味香得让他心痛,先前苏桃若是能吃上这么一顿,就算是上好的大餐了,都能一顿顶两顿了。

无心吃得没滋没味,不过总好过苏桃现在没得吃。面无表情的坐在连部办公室里,她是刚被人从食堂叫过来的。女兵们经过了一个多月的训练,现在已经变得如狼似虎,全有着小伙子的饭量。苏桃不知道是哪个领导要找自己,只晓得自己今晚必定是要挨饿了。

办公室的房门开了,连部领导很客气的引进了一名青年军官。苏桃毫无兴趣的扭头看了对方一眼,虽然是素未谋面,不过一眼就认出了来者的身份——凭着他那一对虎牙,必定和田叔叔有血缘关系。

青年军官除了虎牙之外,再无特色,堪称是不丑不俊,个子虽高,然而没有军人的英姿,倒有点纨绔子弟的意思。单手插兜走到苏桃面前,他先是上下把她打量了一番,随即呲牙一笑:``是苏平平同志吧?''苏桃起身打了个立正,耷拉着眼皮告诉对方``是''。

连部领导关门退出去了,青年把手里的一只大网兜放在了大写字台上,然后搓了搓手,笑微微的做了自我介绍。原来他乃是老田的次子,大名叫做田兴邦。田家满门从戎,他也早早的参了军,如今常驻在附近的空军基地里,是名半大不小的军官。

田家本在沈阳,老田前些日子回了家,忽然想起老苏的姑娘不知在军营里过得怎么样了,便让家里老二前去瞧瞧。老二一听是瞧小女兵,当即欣然同意。拎着些许食品坐上吉普车,他翩翩而来,及至和苏桃打过照面之后,他的虎牙和目光彻底失控,统一的全收不回来了。大豆芽似的往写字台边一靠,他站没站相的笑眯眯:``苏平平,我爸爸让我给你带些零食和营养品。他回沈阳了,一时半会儿的不能再来哈尔滨。''

苏桃站得笔直:``谢谢田叔叔,也谢谢你。''田兴邦笑得豆芽乱颤,语气越发亲切:``平平,不要客气。这也是我做哥哥应尽的关怀。''苏桃没言语,直勾勾的盯着网兜里的食品,在军营里吃独食是不成的,但是一味的搞共产主义也是不智。她得去芜存精,分享一批私藏一批。在食堂里吃不饱,女兵们常有偷馒头当夜宵的。

田兴邦抬手挠了挠新剃的短发,露出了腕子上的上海牌手表,同时语气越发温柔:``平啊,在军营里生活了一个多月,还习惯吗?''苏桃翻了他一眼,然后答道:``习惯。''田兴邦自作主张的红了脸,虎牙尖端反射了阳光:``那个\ldots{}\ldots{}要是有什么难处的话,就和哥说。哥帮不了你,还有爸呢!''

苏桃的脸上看不出阴阳,是城府三丈高的样子:``谢谢你,我知道了。''然后当着田兴邦的面,她伸手打开了网兜。先把里面小块的压缩饼干全掏出来塞进军装里面,她紧接着用牙齿咬开了一瓶糖水琵琶的铁皮盖子。举起玻璃瓶子往嘴里倒——军营里面到处都有眼睛,倒是此时此地更安全。

她早就想吃点儿甜的了,一瓶糖水琵琶喂饱了她肚里的馋虫。田兴邦看直了眼睛,看着看着开了口:``平,你性格真好,豪迈大方,像个女将军似的。''苏桃放下空玻璃瓶,抬起袖子一抹嘴,继续去掏大网兜。

田兴邦没有和女兵久处一室的道理,及至把话说到山穷水尽了,他便摇摇晃晃的告辞离去。苏桃拎着网兜找到班长,闷头闷脑的直接说道:``班长,有人给我捎来几盒罐头,你也尝尝。''班长是位五大三粗的女杰,见了一网兜肉罐头,自然是喜不自胜:``哎呀,全是给我的?苏平平,你家是高干吧?''苏桃嗫嚅着没说出什么。班长也未追问,因为苏平平是一贯的无话可说,问也白问。

入夜时分,苏桃蹲在了厕所里不露面。厕所用矮墙分成了一个个格子,她找了个僻静位置蹲稳当了,开始往嘴里塞压缩饼干。压缩饼干里面有糖有油,还有一点芝麻香。她一边大嚼一边东张西望,至于环境的香臭,则是不在她的考虑范围之内。不少女兵都生病了,她不能病。参军之前无心对她嘱咐了又嘱咐,她不能让无心说了白说。她想自己三个月后若是能够健健康康的去见无心,无心一定很高兴。

夜里填饱了肚子,苏桃睡得舒服。到了翌日中午,又有好事,新兵们迎来了第一批家信。小女兵们乐得欢天喜地,只有苏桃淡然,因为知道自己没有家。然而班长亲自叫住了她,高声大嗓的嚷道:``苏平平,你的信!''苏桃在看清信封上的第一行字之后,一颗心便开始狂跳了——她认得无心的笔迹!

撕开封口倒出信纸,她爬上上铺,做贼似的读信。信一共有两页,第一页被她读过之后揣进了口袋,因为无心没有在开头敬祝伟大领袖毛主席万寿无疆。第二页倒是写得没毛病,她反复读了又读,再看落款日期,原来是此信是昨天邮寄出来的。

``真是不远。''她用手指去摸信纸上的铅笔字:``昨天寄信,今天就到。''然后她以着和无心相同的姿势,撅着屁股跪在床上,开始抓紧时间写回信写好的回信交给通信员,不定什么时候才能发出去。苏桃依旧是每天下午做白日梦,双脚走着正步,喉咙吼着军歌,心里想的却是东方红百货商店。她天天下午会和无心见一面,看无心在商店门口游手好闲笑微微,看得清楚极了。

回信久候不至,田兴邦却是又来了一次。苏桃笑纳了他的礼品,不苟言笑的在他面前连吃带喝。吃饱喝足之后,她苦大仇深的抬起头,严肃而又诚恳的说道:``谢谢你。''

田兴邦感觉她这派头十分冷艳,于是通过长途电话联系到了沈阳的父亲,开诚布公的表明自己想和苏平平搞对象。老田听了,大吃一惊,又不好明说苏平平和个野小子在外面混了一年多,只得言简意赅的告诉儿子:``去你妈蛋!''

田兴邦十分不解,很有礼貌的反问:``爸爸,为什么呢?苏平平不好吗?''老田握着话筒,直说苏平平不好,他感觉自己对不起死去的老苏;要说苏平平好,他又昧了良心。短暂的沉吟过后,他作了答复:``滚犊子。''

田兴邦作为田家三子之中最为荏弱的老二,不是很敢和父亲抗衡;而三天两头的往新兵基地跑,影响又不好。打开一瓶苏桃最爱的水果罐头,他吃得唉声叹气,算是害起了单相思。

苏桃心中完全没有田家的豆芽少爷,成天单是琢磨着偷吃和偷懒,仿佛周围全是敌人,导致她必须想方设法的保存实力。时光易逝,转眼间又过了两个月,新兵训练结束。苏桃人如大名,成绩平平的通过了考核,然后下了连队,开始学习专业知识。

照理来讲,既然正式下了连队,她就有资格休礼拜天了,虽然只是半天而已,但至少够她和无心见上一面。然而新兵头上压着老兵,单有资格还没用。苏桃天天琢磨着去申请周末外出的名额,可名额都被老兵和士官占了,她急得直上火。忽然想起了吊儿郎当的田兴邦,她心思一动,决定另辟蹊径,走走后门。

她不再腆着脸去请假了,转而排队打了个电话,找到了田兴邦,想请他帮自己说句话。虽然田兴邦和她不是一个系统,然而毕竟是一名混久了的高干子弟,她想他总会有点四面八方都通用的面子。田兴邦果然是视纪律为无物,热情洋溢的表示自己愿意带苏桃去哈尔滨玩几天,可惜立刻遭到了拒绝。

放下电话又过了几日,苏桃得到了为期半天的假期,不过她得到了一点暗示,知道自己可以偷偷的早出晚归,不按时归队也可以。提前把一封信发给无心,她在周六的晚上跑步出了营门,搭乘最后一班长途汽车进城去了。

\chapter{挥剑一斩}

四月的傍晚,已经有了一点暖意。一身军装的苏桃坐在长途汽车上,引来无数艳羡的目光。解放军战士多光荣啊,谁敢不高看她一眼?

她一路急得坐立不安,汽车距离长途汽车站还有老远的距离呢,她已经抓心挠肝的挤到了车门口。及至汽车到了站,她毫不维护解放军战士的体面,在车门打开的一瞬间,她一个箭步先蹿出去了。踉跄着站稳了一抬头,她看到了前方的无心。

和当今的大部分青年一样,无心穿着一身半新不旧的军装,周身干干净净利利落落。站在原地没有动,他仿佛是不好意思了,拎着一只保温桶只是笑。于是苏桃也笑了,笑得扭扭捏捏没个大人样儿,吼军歌吼哑了的嗓子也细了,她的长进付诸东流,倒退回了三个月前的模样。

天黑,夜色正好成了无心苏桃两人的幕布。掩人耳目的走到了一起,苏桃先开了口:``车开得可慢了,你等了多久?''

无心低头拧开了保温桶的圆盖子,然后把保温桶往苏桃面前一送:``吃。''

苏桃借着路灯的灯光低头一瞧,发现保温桶里插着三根奶油雪糕。连忙伸手拿出一根,她催促无心:``快点盖好,冷气都跑了。''

无心拧好盖子:``饿不饿?肯定饿了。''然后他抬手一拍苏桃的后背:``怎么没见长?''

苏桃舔了一口雪糕:``不长也够了,我在新兵班里算中等个头呢!''

无心又拍了她一下,拍不够,可是长拍不止也不好。转而又摸了摸她的头发,他有无数的话要问:``头发也涩了,是不是营养不足?几天能吃一顿肉?''

苏桃高高兴兴的往前走:``那得看你够不够厉害。反正一盆炖白菜里面就几片肥肉,谁能抢到谁就吃呗!''

无心居高临下的看她:``你能抢到吗?''

苏桃想起自己在军营里磨炼出的那些小本事,不禁生出几分得意:``一般都能抢到,我手快。''

无心不说话了,让苏桃专心致志的吃雪糕。两人沿着大街往前走,最后绕过一座大学校园,无心把苏桃带回了家。里外两间屋子都被他提前收拾整齐了,一张靠墙的单人床也是铺得平平整整。白琉璃盘在枕头上,大猫头鹰蹲在床角,两个活物也被无心搞了卫生,看着别有一番新气象。门旁角落处有个小洋炉子,炉子旁边堆着一小堆煤。一口小铁锅坐在炉子上,锅盖缝隙中热腾腾的溢出米饭香。

苏桃森严壁垒的过了三个月,如今颇有一种卸甲归田的感觉。转身把房门关好上了锁,她下意识的长吁了一口气,然后跑到炉子前弯了腰,揭开锅盖深深一吸:``好米,真香。''

不等无心回答,她起身走到床边坐下了,把鞋一脱把腿一盘,又将白琉璃整个儿的抱到了自己怀里。捏着对方的圆脑袋亲了一下,她忽然想起保温桶里还存着一根雪糕。单脚踩着鞋面下了床,她从床尾地上拎起了保温桶:``无心,我全吃了啊!''

无心站在地上,向左一转向右一转,是个从头到尾一起骚动的模样:``吃吧吃吧,家里好吃的多着呢,够你明天吃足一天了!中午我从饭店里买了两样炒菜,再炖一锅排骨,可以吧?''

他一边说一边蹲在床边,从床底下拽出一只竹筐。筐里装着大包小裹,全是各色零食,甚至还有软糖和巧克力。苏桃跪在床上,伸了手去翻翻捡捡:``无心,你不过啦?''

她的脑袋正是探到了无心面前,无心一时忍不住,在她头顶心的发旋儿上亲了一下:``吃你的吧,劳军的钱我总有。''

他的嘴唇很软,软得让苏桃一哆嗦,手里的雪糕都快要捏不住。一张脸藏在蓬松的齐耳短发里面,她垂着头继续嘀嘀咕咕:``我用你劳呀?我在队伍里有吃有喝的\ldots{}\ldots{}''话音未落,她忽然直起了腰,从衣兜里掏出了十八块钱:``给你。三个月的津贴,我全攒下了——我要钱没用,没地方花。''

无心接过了钞票,一张一张的整理好后卷成一卷,重新塞进了她的口袋里:``别给我钱,我怕我攒不住。''

苏桃看着他,怀疑他是和自己生分:``我要钱真没用。''

无心在她头上弹了一指头:``知道你不花钱,所以才要把钱交到你手里。你好好攒着,将来咱们用钱的时候多着呢。''

苏桃一听,又乐了:``也对,我比你能攒钱。当两年兵的话,我吃喝穿戴都不要钱,总能攒下一两百块。''

无心弯腰把篮子拎到了床上:``我去炖肉,你吃你的,别给白娘子吃糖。看他肥成什么样了,越肥越馋,全是夜猫子把他惯的!''

苏桃从篮子里挑出了一块巧克力:``你别总说白娘子,白娘子通人性,什么都听得懂。''

白琉璃把脑袋搭在苏桃的大腿上,因为的确是什么都懂,所以心里一点儿也不快活。屋子里渐渐弥漫了肉香,没有桌子,米饭和热过的炒菜全摆在了地上。最后一锅炖肉也登了场,苏桃向无心展示了自己的新饭量——她用大饭盒盛了米饭泡了肉汤,吃完一盒再来一盒。前额的碎发被汗水打湿了,她酣畅淋漓的连吃带喝。无心见了她的食量,几乎有些害怕:``别吃了,肠胃受得了?''

苏桃握着筷子向他摆手,鼓着腮帮子告诉他:``我还能吃。''

无心没话找话,想要转移她的注意力:``你和田叔叔还联系过吗?''

苏桃的嘴唇果然暂时离开了饭盒:``半个月前通过一次长途电话。他让我好好干,说以后他能想办法让我上军校。''

无心的眼睛亮了一下:``上军校?从军校毕了业,是不是一辈子都有着落了?''

苏桃点了点头:``军校毕业生都能留在军队里当干部。可是我不想去。''

无心一团和气的问她:``为什么?''

苏桃忙着说话,不再狼吞虎咽的猛吃了:``我不想一辈子都在军队里。在军队里不自由,结婚对象都要受审查,我怕他们不让我和你在一起过日子。我想好了,我先在部队里当两年卫生兵,将来退伍之后要么进工厂,要么进医院,反正工厂医院也都是挺好的地方,你说呢?''

无心不置可否的微笑,心想军队干部和工人护士怎么会是一回事?

但是他也没有多说,只道:``我看田叔叔倒真是个好人,对你很照顾。''

苏桃伸了筷子,从锅里捞出一块油汪汪的肉骨头:``他对我是好,还让他家老二给我送过几次营养品呢。无心,可有意思了,他家老二也有大虎牙。''

无心随口又问:``他家老二多大了?''

苏桃被他问住了,思索着猜测:``不知道,看着是比我大,比你小。他和田叔叔不一样,田叔叔一本正经的,老二可不正经,总是黏黏糊糊的,还特别爱现。上次他戴了只进口手表,在我面前捋了十几次袖子。嘁!我没见过进口手表呀?''

无心低着头,心事重重的吃菜:``老二在什么单位?''

苏桃预备鲸吞肉骨头,在鲸吞之前,她忙里偷闲的作了回答:``也是当兵的,是空军。''

无心抬头想要再问,可是已经没了机会。苏桃吃得太投入了,他不舍得打断她的好兴致。

清洗过了锅碗瓢盆之后,苏桃照例上了单人床。白琉璃盘在床头栏杆上,是个冷眼旁观的姿态。房内关了电灯,无心坐在床边,窸窸窣窣的也脱了衣服。仰面朝天的躺好了,他伸出手臂,给苏桃当枕头。苏桃的脑袋热烘烘沉甸甸,厚密短发摩擦着他的臂弯。他翻身面向了她:``桃桃,下了连队之后,有没有人欺负你?''

苏桃枕着他靠着他,暖融融的摊开了胳膊腿儿:``老兵最欺负人了,我们天天都得给她们洗衣服,她们还抢我们的东西吃。''

无心在被窝里抬起了手,试试探探的想要落,可是不知该落到哪里:``她们打人吗?''

苏桃并没有意识到他的胆怯与渴望:``打!打得可狠了。不过我只挨过一次——她们冲进宿舍让我们站成队,轮流抽我们的嘴巴。我忍不住还了手,拿牙刷柄去扎她们的眼睛。其实只是吓唬吓唬她们,不能真扎,可是她们害怕了,一边退一边还说要整死我。''

虽然知道苏桃所说的都是往事,可无心还是悬起了心:``然后呢?''

苏桃没有再笑,望着黑暗的天花板答道:``然后?然后她们没再找过我。''

无心叹息一声,伸手扳着苏桃的肩膀,把她搂进了自己怀里:``桃桃,没有我的话,你自己\ldots{}\ldots{}行不行?''

苏桃闭上眼睛,把额头抵上了他的胸膛:``你放心,我能行。新兵训练最苦了,我不是也平平安安的熬满了三个月?再说田叔叔也经常关照我,连里的领导都对我挺和气的。''

无心仰起脸,用下巴去磨蹭苏桃的头顶。苏桃被他磨蹭成小猫小狗了,他一下一下抚摸着她的肩头后背,恨不能把她抚摸到融化,再吮了她、吃了她。

他喜欢她,特别的喜欢她。他为她扮演了可依靠的一切角色,她要他是父亲,他就是父亲;她要他是兄长,他就是兄长。把脸埋在苏桃的头发里,他还想去做她的丈夫,可惜在当今的大时代里,他没资格。

微微抬头凑上了苏桃的面孔,他用睫毛刷过了对方的脸蛋鼻尖。嘴唇颤抖着张开了,他避重就轻的吻了她的眉心。

他吻她,她稚气十足的撅了嘴,也要亲他一下。亲是真亲,``叭''的一大口,响亮得让人想笑。于是无心就真笑了,一边笑一边低声唤道:``桃桃啊!''

苏桃睁眼看她:``嗯?''

无心没有话说。用一侧胳膊肘撑起身体,他悲怆而又凄凉的注视着她:``桃桃,你怎么还不长大?''

苏桃向上迎着他的目光:``我不想长大。我怕我变了,你也会变。''

她认真的对无心说:``我们都不要变啊!''

无心的手指穿过了她的头发:``我不变,永远不变。''

苏桃抬手去摸他的脸,朦胧夜色之中,无心的面孔像是深潭之中浮出的白玉,不知是被清水黑泥浸了多少年,白得潮湿而又寒冷,不带丝毫活气。周身汗毛忽然竖起一片,苏桃发现自己还没有刨根问底的追究过无心的出身来历。他生在哪里长在哪里,自己全不知道。

掌心贴着无心的皮肤,苏桃无端的恐慌了,怕他毫无预兆的来,又毫无预兆的走。

``两年——再过两年。''她语无伦次的出了声,几乎类似哀求:``你不要走,等我两年好不好?''

无心躺好了,做苏桃的枕头苏桃的被褥:``睡吧睡吧,我才不走,我还等着两年之后你给我养老呢!''

苏桃得了保证,放心的睡了。无心平静的搂抱着她,搂抱一刻是一刻,搂抱一刻少一刻。其实当初只不过看她是个可怜的小丫头,他没想到她会活成自己的心头肉。

仿佛只是转眼的工夫,天光大亮了,无心起床给苏桃弄吃弄喝。苏桃没有机会再对他长篇大论,因为嘴不闲着,饮食从早供应到晚。及至快到傍晚时分了,无心把两条巧克力塞进了苏桃的衣兜里,苏桃坐在床边长吁短叹:``唉,下次不知道什么时候才能请下假了!''

无心手脚不停,很巧妙的往苏桃身上藏糖果。末了蹲在床边地上,他抓住了苏桃的一只脚踝,为她穿上了解放鞋。苏桃看他忙得一言不发,心里倒是过意不去,有心让他歇歇,可他拎着保温桶出了门,片刻之后回来说道:``桃桃,该走了,再不走的话,赶不上长途汽车了。''

苏桃向白琉璃和大猫头鹰道了别,然后随着无心下楼上街。保温桶里放着三根雪糕,够她一路且行且吃。

苏桃心里有盼头,所以走得有劲。及至到了长途汽车站,她从无心手中接过最后一根雪糕,随即转身挤上汽车,在最后一排抢到了一个靠窗的座位。无心站在外面,隔着车窗向她挥手。

一切如常,毫无异样。汽车发动起来了,苏桃打开车窗,探出头去喊道:``我走啦,下个月想办法再请假,你回家吧!''

无心站在一盏要亮未亮的路灯下面,没有回答,只是定定的凝视着他。苏桃吮着雪糕回望过去,看他距离自己越来越远,影子越来越小。

疾风扬起她的短发,售票员高声吆喝着让她把脑袋收回去。她那魂游天外的劲儿又上来了,充耳不闻的一边吃雪糕,一边盘算着下次怎么请假。

无心一直等到长途汽车开得无影无踪了,才慢悠悠的走回了家。

这回他真放心了,原来桃桃过得挺好,起码能够吃饱穿暖,还有点小本事小主意,不是个白受欺负的软蛋。这么漂亮的一个小姑娘,背后又有一位田首长撑腰,将来再读上几年军校,毕业之后成了干部,岂不是一生一世都妥了?

长痛不如短痛。无心对自己说:``你老人家狠一狠心吧,可不要再害人家了。小姑娘不懂事,你也不懂事吗?''

然后他在初春的夜风中自嘲一笑——迟早都会是这样的,他有他的宿命。

在归队后的第五天,苏桃收到了无心的信。

她白天忙忙碌碌,不舍得潦草的读信。把信贴身揣好了,她预备留着晚上闲了再慢慢读,又想无心一定是思念自己了,要不然怎么刚见完面就又来了信?

\chapter{天涯陌路}

苏桃走进阅览室,在一份《人民日报》的掩护下打开了信封。抽出信纸平铺到报纸上,她大模大样的低头看,神情姿态都十分自然,任谁也瞧不出她是在守着报纸阅读私货。

慢吞吞的把信读完了一遍,苏桃抬起头望向前方愣了愣。说老实话,她没读懂。

无心的字,每一个她都认识,可是长篇大论的连成行组成段之后,却成了一片模模糊糊的陌生面孔。在信纸上,他说他要走了。

他走,一个人走,要和她走成天涯陌路,她过她的阳关道,他过他的独木桥。为什么要走?因为现在她有着落有前途了,离了他也能活好了,他放心了。

她可怜,小小年纪已经受过了无数的罪,没有家,没有亲人,没有依靠。所以军校还是要上的,不容易上都要争取上。他走了,她得学着自己活了。

苏桃在阅览室呆坐了许久,直到阅览室将要关门了,她才梦游似的回了宿舍。慢慢坐到下铺床上,她听见自己年轻的关节瞬间上了千年的锈,随着动作吱嘎作响。站不动了,也坐不动了,她上不着天下不着地的僵在了时间洪流之中。无心走了?无心真走了?无心怎么能走?不是都说好了吗?不是都约定了吗?他又反悔了?

她没哭,也没闹。低头看自己搭在大腿上的双手,手指蜷曲,指甲青紫。她的血全壅在了心口,四肢百骸都冷硬了。扶着床栏缓缓站起身,她拖着两条腿往外走。有人问她:``苏平平,你还不洗漱?快熄灯啦!''

她听见自己说了一句什么,嗡嗡隆隆的不知道是声高还是声低,但应该是很合理的答案,因为对方立刻闪身为她让出了路。她推门进了走廊,向左望又向右望。长长的走廊里走着那么多的兵,走廊两边的宿舍里又坐着卧着那么多的兵。她难以置信的抱住双臂,忽然要被自己满心的疑惑逼疯了:自己怎么会落到了这么一个陌生的人窝子里来?这些人都和她有什么关系?眼前浮现出了一片盛开着波斯菊的废墟,阳光由明转暗,波斯菊消失了,取而代之的是一片温暖的火塘。长白山的夜风卷着雪花掠地而过,她躺在兽皮褥子上,一边是火,一边是无心。

那些地方才是她的家,她想回家。一定是什么地方出了差错,她咬着嘴唇往前走,一边走一边在心里苦苦哀求:``老天爷,到底是哪里错了?你告诉我,我改!''

在渐渐寂静下来的卫生间里,苏桃进了最里面的格子。稳稳当当的蹲好了,她掏出信,从头到尾的又读了一遍。

然后她捋起袖口,一口咬住了自己的手臂。疼痛让她保留了些许清醒,她想无心也许不会真走——他对自己那么亲那么好,怎么会说走就走?他也许是藏起来了,藏到暗处不露面,他还以为他这样做是为自己好呢!对,肯定是藏起来了,藏到哪里去了?不好说,他总像是无所不能。哈尔滨这么大,天气又暖和了,能让他对付着生活的地方可是太多了。

苏桃松了口,脑子里浮现出了一张路线图。和无心一起流浪了小半年,她知道自己应该先去哪里后去哪里。折好信站起身,她若无其事的回了宿舍,衣袖垂下去,遮住了她小臂上的深刻齿痕。

凌晨时分,宿舍里的女兵发现苏平平不见了。苏平平的被窝里鼓起了一个人形,掀开被子一看,原来里面放了个小铺盖卷。

全连队的人都因此起了个绝早。而在上午八九点钟,逃兵苏平平在火车站落了网。

领导们挠了头,不知道怎么处置她才合适。她是田首长送来的孩子,怎么处置都是要打田首长的脸。直眉瞪眼的打电话去问田首长的意思,似乎也嫌冒昧。无可奈何之下,领导们联系到了田兴邦。田兴邦终于得到了英雄救美的机会,当即大包大揽的把苏桃罩到了自己的羽翼之下。在禁闭室里单独见了苏桃,他一团和气的问道:``平,你为什么要逃呢?是不是遇到了什么困难?有了困难可以和哥说嘛,哥一定会帮助你的。''

苏桃端端正正的坐在一把椅子上,一张脸白中透灰,眼皮耷拉下去,眼尾挑出老长。老气横秋的开了口,她告诉田兴邦:``我对象跑了,我是想去找他。''

田兴邦把嘴一张:``你有对象啊?''

苏桃一点头,人成了木雕泥塑,脸上皮肉纹丝不动:``有。''

田兴邦又问:``他\ldots{}\ldots{}跑了?''

苏桃继续点头:``嗯,跑了。''

田兴邦双手插兜,不知道自己是该哭还是该笑:``跑了\ldots{}\ldots{}平啊,他跑就跑了吧。你年纪还小,将来还会\ldots{}\ldots{}还会\ldots{}\ldots{}你知道哥的意思吧?''

苏桃冷静的回答:``知道。''

事情并没有闹大,被领导消化在了连队内部。苏桃被关了禁闭,静静的坐在禁闭室里,她把自己这十几年的人生从头到尾细细回想。小屋子里安静得让正常人发疯,然而她却怡然。她不喜欢人,不见人的禁闭生活,其实正合她意。

抱着膝盖坐在角落里,她始终感觉无心并未走远,甚至在将来的某一天,他还会再回来,回来看她是不是真上了军校,是不是真像他在信里嘱咐的那样成家立业,是不是真活成了个体体面面的军队干部——一定是这样的,他对她那么好,怎么可能一走了之,不再惦念?

这个念头越来越强烈了,她终于信以为真。怨恨随之而生,她想无心真狠,真自以为是。他凭什么要这样摆布指点自己的人生?

十七岁的苏桃暗暗的下了决心。她要等待无心回来,无论是一年十年还是一百年,她都要等。她要用事实向无心证明,证明他一厢情愿的离去有多错多失败!

在苏桃蹲禁闭之时,无心已经在齐齐哈尔下了火车。

他背着背包,挎着书包,怀里抱着大猫头鹰。下火车后没往远走,他站在告示板前看了一遍列车时刻表,然后挤到售票处,买了一张前往海拉尔的火车票。

此刻正是上午八九点钟,距离车票上的开车时间还有七八个小时。无心出了火车站,想要找个小馆子吃碗热汤面。不料在站前熙熙攘攘的人群中,他猛的被人一把抓住了后衣领。连忙回头向后一看,他和小丁猫打了照面。

距离他们上次相见,已经过了将近一年的光阴。小丁猫的娃娃脸上笼罩着一层沧桑而又油滑的笑意,看起来又老又小的,让人摸不清他的年纪。无心万没想到自己还会再次遇见他,不由得问道:``你不是要逃吗?逃了一年还没成功?''

小丁猫把手指竖到唇边,``嘘''了一声,又问:``苏桃呢?''

这个问题让无心又伤心又自傲的笑了一下:``她当兵去了。''

小丁猫艳羡的睁大了眼睛:``这么好?''

无心以一种父亲的心态,忍不住要捕风捉影的吹嘘几句:``将来她还会进军校——她叔叔是大首长,已经替她把路都铺好了。''

小丁猫上下打量着无心:``她叔叔这么厉害,怎么没顺便提拔提拔你?''

无心被他问住了。抱着大猫头鹰顿了顿,他低声答道:``因为我不想。''

小丁猫穿着一身堪用军装,宽宽大大的极不合身,让无心又想起了苏桃。苏桃以后再不必穿这些破衣烂衫了,刚十七岁,美的日子在后头呢,自己总算是没太耽误她的好年华。

小丁猫又问:``有钱吗?有钱就请我吃顿饭。''

无心做了个深呼吸,然后答道:``好,我请你!''

小丁猫听闻此言,当即握着拳头一伸脖子,爆发似的大吼一声:``顾基!''

远方遥遥的有了回答,顾基抱着一只大网兜穿越人海,飞快的挤到了小丁猫面前。无心和小丁猫一起扭头看他,只见他的大网兜里装满了成卷的卫生纸。

无心不明就里,小丁猫也愣了:``你买这么多卫生纸干什么?''

顾基气喘吁吁一头大汗:``给你路上用。你不是嫌报纸太硬吗?''

小丁猫抬手扶额:``哎呀妈呀\ldots{}\ldots{}''

随即他抬头怒视了顾基:``我一路上也用不了这么多啊!''

顾基手足无措的搂着大网兜,倒也十分有理:``慢慢用呗,这卫生纸质量可好了,又软又结实。''

小丁猫和他谈不下去了,转向无心一笑:``走,咱们找饭店去。有日子没见故人了,我还真想和你聊聊。''

话音落下,他一马当先的开了路。无心和顾基紧随其后,一人捧着猫头鹰,一人捧着卫生纸,黑白双煞似的跟住了小丁猫。

在一家小馆子里,三个人围着一张油渍麻花的小桌子坐住了。小丁猫叼上香烟,直接点了三个油重肉多的炒菜,又要了两瓶啤酒。忽然对着顾基一拍桌子,他一脸嫌恶的斥道:``把你那卫生纸给我放下!''

顾基吓了一跳,立刻弯腰去放网兜;无心不劳小丁猫出声,很自觉的也让大猫头鹰蹲上了自己的大腿。大猫头鹰睡得双眼朦胧,一只尖嘴勾上桌面,也是无知无觉。

小丁猫对于野物没有兴趣,手指夹着香烟深吸一口,他对无心轻声说道:``我这回是真要走了。为了这一走,我们准备了大半年。''

无心也把嗓门压到了最低:``还是去南边吗?''

小丁猫一点头:``南边一是有机会,二是我没去过。就算去了之后事业不成,开开眼界也是好的。现在好时候已经过去了,我们这帮让人当枪使的傻×没了用处,除了上山下乡卖苦力之外,再没其它前途了。''

无心想了想,又问:``户口什么的\ldots{}\ldots{}你也都不要了?''

小丁猫嗤之以鼻:``我要它还有什么用?为了每个月那点儿吃不饱饿不死的粮食?没意思!''然后他看了看无心的打扮:``你呢?你上哪儿去?''

无心摸了摸大猫头鹰的脑袋:``我?我找个地方过日子去。''

小丁猫热情的建议:``你往西北走,西北地方大,容易混饭吃。''

无心摇了摇头:``不必。我往深山老林里一钻,也是一样的。''

小丁猫思索了一番,末了表示同意:``是,你和我们不是一个品种。你的日子更好过。''

炒菜出了锅,顾基起身走去通往厨房的小窗口,把三个炒菜依次端到了桌上,又用牙齿咬开了啤酒瓶盖。小丁猫抄起一瓶仰头咕咚咕咚灌了一气,末了抬手一抹嘴,低头打了个响嗝。很痛快的又长吁了一口气,他出了一会儿神,突然冷笑了一声。

``你真不跟我走?''他问无心。

无心心不在焉的吃着炒肉,只是摇头。

小丁猫又问:``再加个菜行不行?''

无心点了头——小丁猫虽然不讨人爱,可毕竟是个活人。他不知道过了今天,自己又要孤独多久。加个菜就加个菜吧,反正他以后要钱也没什么用处了。

小丁猫和顾基像吃大户似的,闷头大嚼不止,是要一顿吃出一天的量,最后又要了几个杂合面馒头,把盘子里的油汤蹭了个干干净净。无心默默的看着他们连吃带喝,脑海中一幕幕的放映着文县的电影。

中午时分,小丁猫和顾基背着行李抱着卫生纸,鬼头鬼脑的走了。他们要赶南下的火车,去走出一条新的人生道路。无心望着他们的背影消失在检票口,忽然感觉他们两个都是浪漫派,为了一个虚无缥缈的目标,兴致勃勃的说走就走了。

在候车室坐了半个下午,他什么也没想。及至将要检票进站了,他被检票员拦在了外面:``哎?你怎么上车还带了只鹰?这是鹰还是雕?''

对面的检票员见多识广:``是夜猫子。''

无心抱着大猫头鹰不松手:``你看他们还带活鸡活鸭了呢!都是鸟,我为什么不能带?''

检票员不耐烦的立起眉毛:``谁知道你这玩意儿伤不伤人啊?你赶紧把它处理了,反正带它上车就不行!''

无心被检票员搡到了一旁。臊眉耷眼的转身离去,片刻之后他回来了,臂弯中坐了个懒洋洋的小男孩。小男孩缩成小小的一团,一看就是要免票的。这回没人拦他了,他急匆匆的挤上火车。找到座位坐下了,对面的老太太笑道:``嗬!这小爷俩儿也太像了!''

小男孩搂着无心的脖子,睡得呼哧呼哧,脚上没穿鞋,脚趾头蜷缩着蹬在无心的腿上。无心对着老太太笑了笑,无话可说。

无心下了火车改乘汽车,又搭了一段马车。最后凭着两只脚翻山越岭,他回家了。

穿过一片遮天蔽日的林子,他越走地势越高。恢复了原形的大猫头鹰在树梢之间盘旋飞舞,忽然猛的打了个冷战,他感觉自己像是进入了一个异世界。看看周遭环境,还是普通的山林,然而作为一只上百岁的妖精,他嗅到了一股子浓郁的阴寒邪气。没想到世上还有这样的地方,简直就是鬼神精怪的乐园。

无心继续走,走了整整一天。末了在一片斜坡上停了脚步,他弯腰搬开一块生满青苔的大石头。猫头鹰听到一阵刺耳声音,正是无心拉开了嵌在地下的一扇小铁门。小铁门已经锈蚀的不成样子了,然而依旧坚固。铁门一开,露出了个小小的幽黑洞口。无心把身上的大包小裹扔到地面,然后大头朝下的钻进洞里去了。

地堡里还是老样子,处处都是伸手不见五指的黑暗,墙壁上用油漆画着的日本字依然清晰。无心靠墙坐了,双手搭在膝盖上。仰起头闭了眼睛,他开口问道:``白琉璃,我们在外面走了两年,这两年里,你玩得高不高兴?''

白琉璃在他面前也坐下了,影子清晰至极,几乎像是真人:``开始很高兴,中间也很高兴,最后不高兴。''

无心沉默良久,末了答道:``我也不高兴。''

\begin{quote}
作者有话要说:
\end{quote}

\begin{quote}
第三部即将完结O(∩\_∩)O\textasciitilde{}
\end{quote}

\chapter{他们的岁月}

对于无心来讲,时间是没有意义的。

天气热了又冷,冷了又热。山外的知青们来了又走,走了又来。机器与刀斧的力量终究是有限的,无心在山里活得安静而又安全。起伏的密林与恐怖的传说,为他隔离出了一个孤独的小世界。

山中有一条小河,不知源头在哪里,总之春天汹涌,夏天平缓,入秋之后河水渐渐干涸,到了冬天,便冻成了一条薄薄的冰带子。小河两岸盛开着鲜花,花朵颜色新鲜浓烈,美得怪异,惊心动魄。无心的赤脚趟过牵扯勾连的花草丛,初秋的阳光晒热了他的屁股脊梁。

他活成野人了,甚至省略掉了衣裤鞋袜。在足够暖和的天气里,他直接赤身露体的东跑西颠。停在一片野葡萄藤前,他咽了口唾沫。野葡萄四处攀爬,结成了一面郁郁葱葱的绿墙。紫色的果实垂垂累累,其中大部分都酸,不过只要熟透了,酸也酸得有限。

无心摘了一串葡萄,想要坐到旁边的大石头上慢慢吃,可是未等坐稳,他猛然向上一窜,开始捂着屁股骂骂咧咧。原来大石头被太阳暴晒了一天,如今的热度已经可以媲美火炭了。

无心拎着葡萄向林子里走,一侧屁股蛋被烫红了,红了一路总不见好。他素来怕疼,此刻自然满心牢骚。然而自怜自艾不耽误他觅食。大猫头鹰在林子里找到他时,他已经收获颇丰,虽然依旧红着屁股。

大猫头鹰还是没有学会说人话,对着无心高一声低一声的嗥叫了一阵,无心大概听明白了:``白琉璃又下山去了?''

然后他举起手中的一根树枝,张嘴去吃结在树枝上的野果子:``他要去就让他去嘛!''

大猫头鹰的羽毛中溢出了隐隐的一团黑雾。黑雾渐渐笼罩了他,他不见了,取而代之的站起了一个小男孩。小男孩围着无心团团乱转,一手抓住无心的腕子,一手往山下的方向指,是非让他把白琉璃找回来的架势。无心不去,不但不去,而且不耐烦,弯腰一口咬上了小男孩的咽喉。小男孩吓得一闭眼睛,一动不动的老实了。

小男孩逃离了无心的牙齿,自己跑向山下去找白琉璃,跑着跑着他变成了猫头鹰,飞着飞着他落了地,又变成了小男孩。连跑带飞的没走多远,他和白琉璃来了个顶头碰。他还没有修炼出一双阴阳眼,看不见白琉璃的影踪,可是出于妖精的直觉,他闭着眼睛都能找到对方。``扑通''一声跪在草地上,他张开双臂抱住了眼前的大白鹅,又很快乐的叫了一声:``呼!''

附在大白鹅身上的白琉璃愣了一下,随即一嘴把他啄开了。

白琉璃当蛇当得百无聊赖,于是转而做鹅。心安理得的把大白鹅交给小男孩,他溜出鹅身,一路高高兴兴的先飘向前了。在林子边缘,他啼笑皆非的遇到了无心。

无心一手倒拎着一只死鸟,一手举着一枝结满野果的绿树枝。不知道是刚刚想起了什么美事,他下面通红的支起了一根棒槌,棒槌上面缠着葡萄藤,坠着沉甸甸的两大串野葡萄。嘴里一左一右含着两枚大鸟蛋,他对着白琉璃眨巴眼睛,意思是``你回来了?''。

白琉璃被他的形象逗笑了,笑得上气不接下气,恨不能就地打滚。满山的生灵死灵加在一起,谁也没有白琉璃活得欢乐。生前藏而不发的活泼劲儿全施展在死后了,他时常笑得像个疯子。等到由着性子笑够了,他才飘到无心身边,像个活人似的陪着他并肩走:``你知道吗?山下的知青都撤走了。''

无心想要找到一块平整地方吃东西,于是一边走一边东张西望。

白琉璃又道:``知青在闹事,说是要回城。''

无心把手里的果树枝和死鸟放在了一棵老树下。自己坐在凸起的老树根上,他先吐出嘴里的鸟蛋,然后低头解开了命根子上的野葡萄藤。白琉璃为了表示自己也是通人情的,特地问道:``你想女人了?''

无心``嗯''了一声,摘了葡萄往自己嘴里送。

他已经沉默寡言了许久。白琉璃记得他死了上一个老婆之后,虽然在地堡里也哭丧了几天,但是几天之后就又嬉皮笑脸了。疑团终于有了答案,白琉璃想,越来(原来)他是特别的喜欢苏桃。

无心吃了葡萄野果,又撕开死鸟生吃了它的肉。最后带着两枚鸟蛋爬上了树,他舒舒服服的躺稳当了,不知道什么时候才能再落地。白琉璃在枝叶之间飘来飘去,想让无心带自己再下山逛上一圈。无心用一片大树叶挡住了眼睛,低声答道:``我不去。''

白琉璃告诉他:``山下有很多女知青,你可以捉一个陪你睡觉。''

无心叹了口气,不想理睬白琉璃。他和白琉璃的感情全迸发在久别重逢的一刹那,千万可别相处久了。一旦过上了朝夕相对的生活,他们迟早是要相看两相厌,比如现在,他真想把胡言乱语的白琉璃一指头弹飞。

无心躺在树上不言不动,缓慢的消化着肚中的食物。一周之后他落了地,半死不活的再次觅食。

花草渐渐凋谢了,小河渐渐消瘦了。季节周而复始的变换着,山外的知青也彻底走光了。山中才一日,世上已千年。无心长久的坐在树上,看月亮升太阳落,看星星排着阵法,一夜一夜的划过漆黑天幕。桃桃现在长大了吧?桃桃现在毕业了吧?桃桃现在结婚了吧?一滴很大的眼泪凝结在了他的腮上,是透明的胶质,最后风干,如同一颗琥珀。

在一个寂静的夜里,他又想:``桃桃现在生小孩子了吧?''

桃桃和他最初相遇的时候,也是个小孩子,孤苦伶仃,哭得上气不接下气。无心从来不做梦,可是此刻第一次体会到了做梦的感觉——他和苏桃相处的两年,就是一梦。

当无心算到``桃桃的孩子也长大了吧''的时候,苏桃已经在河北文县的县医院里工作了将近二十年。

她没有读军校,因为还是嫌军队里不自由,怕有朝一日无心回来了,组织会不同意自己和他结婚。退伍之后她主动要求分配到了文县,其实文县也不错,地方不大不小,既不落后闭塞,也不繁华喧闹。县医院是个好单位,她在医院里熬成了护士长,工资比上不足比下有余,够她活了。

她始终是没有结婚,在军队里,田兴邦曾经惊天动地的追求过她;后来到了医院,她也成了不少年轻医生的水中月镜中花。无数天作之合一般的好姻缘都被她冷漠的斩断了,她活成了医院里面有名的老处女。

她白白的美丽了一世,对于她所处的大世界,她永远是冷若冰霜、心如铁石。

在晴朗无风的周末午后,苏桃会一个人出门散步。文县越来越大了,她沿着街道慢慢走,要走好久才能到达一中门口。一中所占的还是二十年前的老楼,校园对面的破厂房成了三不管的地界。她的身体已经不复少年时代的轻盈,又顾忌着脚上的一双新皮鞋,所以在厂房废墟之中走得磕磕绊绊。最后她坐在了半截砖墙上,在阳光下举目远眺,去看砖石堆中生出的一丛丛野草闲花。

她没有读书,没有提干,没有结婚,没有生子。她以自己的人生为筹码,对无心赌了二十年的气。她坚信无心总有一天还会从天而降,就像他第一次出现时一样。到时候他老了,她也老了,她要让他读读自己一生的故事,她要让他知道他有多错!

与此同时,千里之外的无心睡在树上,很难得的做了个梦。

他梦见了一大片随风摇曳的波斯菊,盛开在那年炮火纷飞的春天里。

\begin{quote}
作者有话要说:
\end{quote}

\begin{quote}
\end{quote}

\begin{quote}
第三部到此结束,感谢大家的喜爱和阅读。
\end{quote}

\begin{quote}
\end{quote}

\begin{quote}
接下来我休息两天,如果一切顺利的话,两天之后我开始写第四部。第四部的时间背景为二十一世纪,敬请期待O(∩\_∩)O\textasciitilde{}
\end{quote}

\begin{quote}
祝大家元宵节快乐。
\end{quote}

\part{廿一世纪}

\chapter{精神病人}

在一个晴朗的四月午后,攀附在大货车顶的无心被交警发现了。当时他被牵连不清的绳网牵扯纠缠了住,否则凭着他的身手,他绝不会趴在车上束手就擒。大货车满载货物,长宽高已经几乎相等,跳车等于跳楼。交警费了老大的劲,蹬着梯子往车上爬。司机早下了车,手搭凉棚往上望,一边望一边和身边的交警解释:``我真不认识他,我能把我认识的人往车顶上放吗?哎呦我操,你们说他是怎么上去的?''

爬上车顶的交警解开了无数半死不活的大绳扣,让无心的胳膊腿儿得了自由。无心跪坐在了大货箱上,怔怔的望着面前的小交警。小交警有恐高症,一边四脚着地的往后倒退,一边怒道:``你是猴儿哇?''

话音落下,交警眼前一花,无心没了。

然后小交警在自己的惊叫声中,看到一个灰扑扑的人影斜刺里穿越国道,刹那间冲入路旁树林,从此消失无踪。

无心一路狂奔,在穿越了一片小树林后,他上了一条柏油路。路边立着个大铁牌子,上写六个大字:火星镇欢迎您。

无心仰头望着牌子,又发了半天的呆。简化字在他眼里总像是缺胳膊少腿,怎么看怎么不对劲,六个字让他翻来覆去读了好几遍。末了心里明白了,他惶惶然的迈开步子,向前走入了火星镇。在大兴安岭的深山老林里隐居了将近四十年,如今骤然回归人间,他发现人间竟然大大的变了模样——变化之剧烈,简直要让他惊恐了。

山外的人们已经不认得他手中仅有的几张旧人民币,粮票也成了天方夜谭般的往事。他的假介绍信假证明更是一分钱不值,现在的人可以随便走随便住,而且都有身份证。他穿着一身几近褴褛的旧军装走在人群中,引得人们纷纷对他行注目礼,看一个浓眉大眼的小白脸子,竟然穿戴成了乞丐模样,而且还是怪模怪样的乞丐,像是从革命时期穿越而来的。

他难得的懵懂怯懦了。扒着一辆运输木材的火车走了一段路,火车到站,他茫茫然的也到了站。在火车站外爬上一辆大货车。货车司机无知无觉的上了路,带着他疾驰了将近一天,直到交警发现了他。

无心此刻饥肠辘辘,决定去火星镇打食。千变万化的新人间虽然吓得他左一跳右一跳,但还是要比山里强。白琉璃彻底被大猫头鹰哄住了,一鬼一妖合作欺负他一个,横竖知道他死不了,所以下手格外狠辣。大猫头鹰当年一脸忠厚老实相,原来也不是个好东西。山中日月成全了一个他,几十年中他妖术大有长进,已经敢和无心蹬鼻子上脸了。

于是无心自作主张的下了山,不和他们过了。

无心沿着柏油路往前走,路是好路,路两边有田地有房屋,乃是火星镇外围的一处大村庄。此时正是四月时节,待种的田地都被翻过了,黑土被晒了一整天,此刻已经干爽松软。无心一边走一边东张西望,心想野地里不会有野菜野果,自己还是得往人的身上打主意。要说人,眼前倒是有现成的一个,看背影是个青年人,打扮得西装革履,然而双臂环抱在胸前,腰也弓着,显然是在搂抱着什么。青年人步伐匆匆,越走越快;无心连跑带跳的追上了他,侧着脸想要和他搭话,然而定睛一瞧,他心中一惊,原来青年双眼通红,满面泪痕,嘴唇紧紧的抿成了直线。西装前襟只系了一枚纽扣,下摆偶尔随风飘起,无心瞪大了眼睛,怀疑自己是看到了一圈炸弹。

看到的是一圈,看不到的,被青年双臂环绕着的,不知还有多少。一条穿着桃红背心的白哈巴狗从前头颠颠的来了,伸着舌头且颠且喘,又对着青年``汪''了一声。

未等白狗闭嘴,柏油路上爆发出了惊天动地的巨响。无心、青年、白狗瞬间化为乌有,道路两边的大树也被气浪摧成了骨断筋折。附近的房屋玻璃全起了共鸣,连远方一座小楼内的史高飞都被震得打了哆嗦。一哆嗦,手里的面巾纸失了准头,他上面望着电脑屏幕里的南波杏,下面一波接一波的射了一裤子。

一惊之后,史高飞慌忙低了头。裤子被他退到了大腿处,如今前门拉链已经被他的万子千孙彻底糊住。匆匆忙忙的用纸擦了,他心怀鬼胎的提了裤子往窗口跑。``哗''的一声拉开拉窗,他探出上半身向外张望,想要查看巨响的来源。然而窗外风景一如往常,只有一只大灰雀趁虚而入,扑啦啦的飞进了房内。

史高飞来不及驱赶鸟类。转身出了房门穿越客厅,他推开向外的楼门,几大步蹿进了院子里。院子是大院,一半铺了水泥地,一半种了花花草草。另有一棵吃里扒外的老果树紧挨院门,每年都要无私的向院外奉献出几枝子沙果。史高飞别有心事,一味的只往大门口跑。然而未等他打开左右合拢的黑漆铁栅栏门,他的眉心之间忽然落了一滴暖暖的雨。下意识的抬手一摸,他随即对着手指头直了眼——不是雨,是血!

猛然抬头向上望去,在老果树的密集枝杈之间,他看到了一只白色的狗头。狗头保持着龇牙咧嘴的神情,脖子往下一无所有,只垂了丝丝缕缕的几条鲜红筋肉。狗嘴毫无预兆的上下一张,一小块粉红色的肉垂直落到了黑土地上。

在和狗头对视了片刻之后,史高飞和狗头一样龇牙咧嘴了,恶心得恨不能就地呕吐一场。举起一根竹竿捅下狗头,他薅着狗耳朵将其扔到了院外。随即跟着狗头一起出了门,他一路小跑的看热闹去了。

史高飞本名史鸿鹏,乃是本镇首富之子。他幼年兼生了倾国倾城的貌以及多愁多病的身,把他上面的一个姐姐比得狗屁不如。不过一个男孩子一味的娇弱也不是长久之计,后来经过高人相看之后,他换汤不换药的改了名字——由具体的``鸿鹏'',改成了抽象的``高飞''。

名字一改,果然立竿见影,史高飞改头换面,从小病秧子变成了一名高大英俊的精神病患者。从十五岁疯到了二十五岁,他坚信自己是一名外星遗孤,有朝一日必将回归母星。他妈赵秀芬为他嚎得肝肠寸断,并且在丈夫史一彪心中彻底失宠——当年在赵秀芬年轻貌美之时,史一彪忘了赵秀芬的妈和妹妹曾经先后声称自己是狐狸大仙和九天神女。赵家八辈贫农,全国劳苦大众都翻身了他家也没翻身,留给子孙后代唯一的遗产就是精神病。史一彪重男轻女,恨不能练就神功,把儿子的精神病转给姑娘。姑娘三十了,生得花容月貌,袅袅娜娜,曾经是火星镇的林黛玉,还念过三年大专,可如今硬是没人敢娶,因为都怕她会随了她妈,再养出个疯儿痴女。

史一彪对于家庭彻底失望,尤其恨老婆恨得牙痒,常年不肯回家。他身为本镇的娱乐业巨头,经营着今夜星辰夜总会,明日之星KTV,快乐时光咖啡屋,以及酷龙连锁网吧三家。既然拥有如此可观的家业,他自然不会无处落脚。而赵秀芬进入更年期,天天在家要死要活,专跟着女儿较劲。女儿名叫史丹凤,既没事业也没爱情,连她妈都不肯高看她,甚至认为她一个人也挺好,将来正好照顾儿子一辈子。反正儿子疯得全镇出名,想必也找不到媳妇伺候他一生。史丹凤看她妈把心偏到了胳肢窝里,自然也有意见。总而言之,史家全体成员之中,只有史高飞的痛苦程度较轻——他一心等待母舰降临接他回家,对于家中三个地球人,他一般懒得搭理。

在柏油路上的村民群中凑了半天热闹,因为警察封锁了现场,所以他也没看到什么,只知道路面被炸出了一个大坑。傍晚时分,观众们纷纷回家做饭,他也跟着回了自己所住的小楼。小楼一共有二层,当初史一彪想在农村发展一点副业,才盖起了小楼大院。后来副业胎死腹中,小楼空着没人住;而史高飞去年年末被家人强行送进精神病院住了一阵子,出院之后和地球人越发势不两立,索性独自进了村,要安安静静的过几天田园生活。

没滋没味的锁了院门进了楼,他穿过客厅往卧室里走,一边走一边自己叹息:``我还以为是飞船来了呢!''

电脑屏幕上的视频已经播放完毕,不速之客大灰雀也早没影了。他牢牢骚骚的蹲到电脑桌下,想要清理白天乱扔的面巾纸团。不料在一团半干半黏的面巾纸下,他意外的发现了一枚大豆子。此豆十分古怪,竟然是个心形,如果把它比作人的话,必定是个连体婴。史高飞四体不勤五谷不分,不知道豆子也会畸形。捏着豆子端详了半天,他扪心自问:``我白天射豆子了?''

随即他把裤子一脱,仔细检查了自己的先天条件,最后认定这应该是不可能,因为他的那条播种的道路长而狭窄,不足以孕育出尺寸如此壮观、形象如此美好的种子。拈着豆子站起身,他忽然打了个激灵,心里又生出了邪主意:莫非方才自己的卧室内有人来过了?莫非这豆子承载着母星传递给自己的信息?光天化日的,总不会无端的发生大爆炸,必有玄妙在里面!

可他马上又犯了难:母星的使者也太不体谅人了,他在地球过了二十多年,现在哪里还能和同类心有灵犀?掂着豆子出了许久的神,他坐卧不安,实在是揣摩不出豆中的深意,又不敢贸然把豆子剖开或者嚼碎。抓心挠肝的熬到午夜,他终于浮想联翩的思索出了眉目:``这是一颗种子啊!''

午夜时分,众人皆睡,唯有史高飞独醒。站在土质最为肥沃的老果树下,他挥舞着一把大铁锹,挖了个半米多深的圆坑。恭而敬之的把心形豆子放入坑底,他双膝跪地,亲自伸手捧土填坑,一边填一边又默默祈祷:``种子啊,你快长大快显灵吧。他们都不相信我的话,还丧心病狂的诬陷我,说我是精神病。你一定要长成个了不起的宝贝,好向他们证明我的身份!''

虔诚的撒下最后一把土,他双手合什又拜了拜。最后意犹未尽的站起身,他垂着两只泥手仰望苍穹,心想满天的星星有明有暗,不知道哪一颗才是我的家。人在异星,没个知音,真是遭罪啊!

村口柏油路上的爆炸案上了各大网站的头条,捎带着火星镇一起出了名。一个月后,案子基本破了,原来是场未遂的情杀——一男一女搞对象搞出了仇,男方是个亡命徒,绑了一身炸药往女方家去,本意是要趁着傍晚女家人齐全,点燃导火索来个一锅端。没想到炸药本身出了问题,走到半路,自行炸了,炸得什么都不剩,导致警察须得四处走访调查,一点一点的拼出事实真相。

村里常年太平,近几年连去世的老人都少有,所以一桩爆炸案足以让村庄沸腾许久,唯有史高飞极其冷静,满眼满心只装着他的种子。在等待种子发芽的期间里,他连爱情动作片都没心思下载了,成天无欲无求的蹲在树下,直勾勾的只盯着土地使劲;饭也时常是一顿管一天,饿得他一米九的身高只有一百五十斤,扛着宽肩膀垂着头,他支起后背两大片肩胛骨,乍一看好像一只秃毛又折翼的大天使。

勤勤恳恳的浇了两个月的水,他天天对着一片土地望眼欲穿。如此熬到了七月,头顶的果树已经结出了累累的小绿果子,可是他的种子依旧毫无动静。

他等不得了。在一个狂风大作的夜晚,他欲哭无泪的蹲在树下,预备对种子做出一番控诉,然后把它挖出来就地踩扁。然而在他顶风开口之前,空中忽然裂过一道闪电。随即在震天撼地的雷声中,史高飞睁大眼睛,发现一贯平坦的地面竟然隐隐鼓凸,仿佛是有什么东西将要破土而出了!

颤巍巍的伸出一只手,史高飞轻轻的拨开了最表面的一层浮土。浮土之下露出了一小块粉红的皮肉,皮肉中钻出几根东倒西歪的白毛,正在暴雨来临之前的疾风中微微抖动。

史高飞忽略了地上的风与天上的雷。他屏住呼吸张大了嘴,用十根手指又挖又掘。末了在第一颗大雨点子砸向他时,他从土里刨出了一只半人长的大毛毛虫。``扑通''一声跪在泥水之中,他激动得又哭又笑,又捶大腿又甩泥巴。原来母星的同胞并没有忘记他,原来同胞所给他的,真是一粒种子!

脱下身上的T恤裹住大毛毛虫,他在大雨之中站起了身,抱着毛毛虫趿着人字拖,他一路噼里啪啦的跑进楼里去了。

\chapter{虫宝宝}

史高飞盘腿坐在卧室内的大床上,一件衬衫被他当成围裙系在了腰间。大毛毛虫刚被他送到浴缸里洗干净了,此刻正长条条的横在他的大腿上,大腿瘦成了两根粗大的骨头棒子,越发衬得大毛毛虫粉嫩嫩软颤颤,仿佛一把能掐出水。只虫体表面凹凸不平,并且白毛丛生。

史高飞认为它很可爱,连它身上甜腥的气味都忽略不计了。

从头到尾的摸了一遍,史高飞没有找到它的头也没有找到它的尾,同时感觉毛毛虫软中带硬,仿佛嫩肉里面也有骨骼。手指划过虫身,史高飞的动作忽然一滞,因为感觉大毛毛虫仿佛在他的腿上抽搐了一下。

慢慢的俯下身去,他几乎把鼻尖凑上了一丛白毛:``宝宝,你怎么了?疼了?还怕了?你不要怕,我你的爸爸。我已经在地球上生活了二十五年,个老地球人了。以后我会保护你的,不过你打算长住呢?还要带我回家?''

话音落下,他感觉自己说的没毛病。从把毛毛虫抱进楼内开始,他的脑筋就像上足了发条一样,一直没停转:卧室这么宽敞,豆子落到哪里不好,非要挤到脏兮兮的卫生纸下面?可如果把豆子想象成一颗来自母星的卵子,一切问题就都迎刃而解了——非得如此不凡的大号卵子,才能自行找到他的卫生纸受精,并且在两个月内长成半人多长。

所以他封了自己为毛毛虫之父。虽然他这样的,他的毛毛虫宝宝那样的。

史高飞彻夜不眠,想要找到毛毛虫的嘴。没有嘴,他怎么给它喂食呢?

徒劳无功的忙了一夜,他一个哈欠都不打,脑筋继续高速运行。既然实在找不到嘴,那索性就把它当成花花草草来养。把它埋回土里舍不得的,于他无师自通的开始进行无土栽培。蓄了一浴缸的温水,他找出家中所有的维生素药片,全磨碎了溶入水中。自认为一缸温水已经十分富有营养了,他调动了他的大长胳膊大长腿,颤巍巍的把大毛毛虫放进了浴缸里。

然后他不走,捧了一台笔记本电脑坐在浴缸旁,他为大毛毛虫播放钢琴曲,权当迟来的胎教。纹丝不动的从早坐到晚,他直到饿得眼前发黑了,才东倒西歪的起了身,想要找点食吃。家里已经没有存粮,他把楼门院门里三层外三层的锁严实了,草上飞似的跑去村口超市,买了许多饼干泡面。气喘吁吁的回了家,他进门之后先往浴室跑,见大毛毛虫还怡然自得的躺在水里,才长长的吁出一口气,一颗心也落回了腔子里。

史家的大门又关上了,院里无论昼夜,永远清静的连个人影都没有。村民们知道史高飞的底细,平日恨不能绕着史家走路,他死活,自然也无人关心。如此过了一个多月,史家门外终于有人驻了足——史丹凤来了。

史丹凤穿着一身雪纺连衣裙,为了防晒,头上又戴了一顶大黑檐遮阳帽。上下活动的帽檐比脸还大,放下来把脸扣了个严丝合缝。窈窈窕窕的推着一辆小电动车,她看身体飘飘欲仙,看脑袋神秘莫测,正史高飞最瞧不上的人类形象。抬手连摁了几下大门门铃,她单手扶着电动车,车后座上捆了个大纸箱,箱子里她给弟弟带的援助物资。长姐如母,虽然史一彪赵秀芬二人偏心骗得人神共愤,但她身为大姐,并没有迁怒于弟弟的打算。好好一个弟弟,男明星似的英俊潇洒,偏偏疯头疯脑的不说人话,她看在眼里,疼在心里,连自身的痛苦都暂时淡忘了。

门铃响了一长串,楼内丝毫没有回应。史丹凤从身上的小皮包里掏出手机,正想给弟弟打个电话;不料未等她开始按键,身后忽然起了一串叮叮铃铃的响动。回头一瞧,她看到了一头大汗的史高飞。而史高飞骑着自行车猛一捏闸,见了鬼似的瞪着他姐,也不打招呼,只欲言又止的张了张嘴。

史丹凤收起手机,张口就牢骚:``小飞,你刚跑哪儿去了?我还当我扑了个空。大热天的,我来一趟容易的?我告诉你啊,现在爸妈惯着你,我可不惯着你。有本事你滚回太空去,否则我作为你姐,我就敢揍你!''

史高飞握着车把,支支吾吾的不肯靠近她:``我\ldots{}\ldots{}姐,你来干什么呀?''

史丹凤从镇子骑到村里,快被晒得融化喷火。此刻伸手一拍车后座的纸箱,她躲在黑面罩后面急赤白脸:``你快开门!速冻的饺子快要化了!''

史高飞下了自行车,犹犹豫豫的推车上前,一边走一边把手伸到短裤口袋里掏钥匙。史丹凤一眼看清了他挂在车把两端的大包装袋,立刻又起了高调:``你买婴儿奶粉了?''

史高飞停好自行车,慢吞吞的去开大门锁头:``嗯\ldots{}\ldots{}''

史丹凤拥有贤妻良母的一切素养,从经济的角度出发,她当即针扎火燎了:``你多大了还喝婴儿奶粉?十五块钱一袋的不够你喝吗?婴儿奶粉一桶得一百多吧?''

史高飞开了院门,转身去推自行车:``两百多呢,我挑了最好的买。''

史丹凤双臂运力,把电动车推入院内:``小飞,你个不听话的,气死我了。''

史高飞也跟着他姐进了院。摘下车把上的两只大纸袋,他把他姐带入楼内。史丹凤记得弟弟一贯很讲卫生,然而此刻进了门,她猝不及防的吸了一鼻子怪味——又甜又腥的,不算臭,然而越闻越不舒服。客厅里摆着旧沙发和旧茶几,她一边催促史高飞把纸箱里的冷冻食品往冰箱里放,一边摘了遮阳帽坐上沙发。低头摸了摸皮沙发的表面,她摸到了几根细长的白毛。

``小飞!''她高声质问:``你养狗了?''

史高飞离开厨房进入客厅,意意思思的站在沙发一旁:``没、没有。''

史丹凤一抬手,向他展示白毛:``你养狗我不管你,可千万别让狗咬了。''

史高飞心神不宁的看着她,鼻子里``嗯''了一声作答。

史丹凤不和他一般见识,起身往卧室里走,要给他收拾房间,顺便洗洗涮涮。虽然家里有洗衣机,但史丹凤对洗衣机信任的有限。来都来了,她总要给弟弟出把子力气。然而未等她走到卧室门前,史高飞已经背靠房门,阻住了她的去路:``姐\ldots{}\ldots{}不用打扫了。''

史丹凤看他脸上红一阵白一阵的,不个正经颜色,不禁起了疑心:``小飞,你紧张什么?屋子里有什么怕人看的?''

史高飞义正词严的正视着她:``没有!''

史丹凤伸手拽他,拽了一下没拽动:``真没有?你让我进去瞧瞧!''

史高飞提高声音:``不行!''

他嗓门大,中气十足的吼出一声,把史丹凤吓了一跳。吼声过后短暂的寂静,史丹凤的耳朵忽然一动,仿佛听到房内有活物在唧唧的叫。

心随耳动,史丹凤不问了,转身坐回沙发上,她开始和史高飞扯闲话,话里话外的设了钩子,想要勾出他的真话。史高飞警惕的望着她,忽然问道:``他们派你来的吗?''

史丹凤太好奇了,灵魂恨不能突破躯壳穿墙而去,看看弟弟的卧室里到底藏了什么:``没人派我,不过妈很想你,想让你回家住几天。''

史高飞严肃的望着她:``我很理解他们想要禁锢我的心情,但我对他们来讲,毕竟只过客。或许他们当初根本就不该收养我——他们为什么不收养一个同种族的地球婴儿呢?姐,我看你在地球人中还算个善良的,所以对你有一说一。''

史丹凤忍不住了:``你有一说一个屁!你告诉我你屋里到底有什么好东西不能让我看?''

史高飞昂首挺胸:``姐,你走吧。也许在不久的将来,我就要回家了。希望你趁着现在我还在,多信任我,少骚扰我,少给我买特价卫生纸和特价牛奶。让我在走的时候,还能保留一点儿对你的美好回忆。''

史丹凤被他生生的气跑了。

史丹凤一走,史高飞立刻锁了大门二门。笑嘻嘻的进了卧室,他掀开床上的大毛巾被,低头去看床上的外星宝宝。大毛毛虫逆着他的预想飞快成长,一身的骨骼越来越硬,并且分化出了潦草的四肢。人形的首端也有个脑袋,脑袋圆圆的,分布了五官的雏形。昨晚史高飞发现它有了真正意义上的嘴,登时欣喜若狂,以至于今天起了个大早,特地到镇上去给它买奶粉——它只有几个月大,当然应该喝奶粉的。

快手快脚的冲了一杯奶粉倒进奶瓶里,他先在胳膊上试了试奶水温度,然后把一身白毛的宝宝抱到了大腿上。把奶嘴塞到对方嘴里,他只听``咯吱''一声,拔了奶瓶一看,他发现橡胶奶嘴已经被对方的利齿咬破了。

史高飞很有做父亲的自觉性,饶有耐心的换了个奶嘴,继续去喂。一边喂一遍又自言自语:``宝宝,爸爸没想到你入乡随俗,也长成了一个地球人。不过这样更好,给我减少了许多麻烦。''

他的宝宝,无心,一边窝在他的怀里咕咚咕咚喝奶,一边在心里窃笑。没想到这次下山走了邪运,如无意外的话,他自己琢磨着,很可以在这疯小子手里混上几年的好吃好喝了。

用大号奶瓶喂了四瓶奶后,史高飞打开卧室墙上的电视机,抱着他人模鬼样的宝宝看电视。电视正在播放相声,无心听高兴了,想要大笑,然而器官发育不全,声音失控,只会尖着嗓子唧唧乱叫。史高飞以为他又饿了,连忙起身再去烧开水冲奶粉。无心快乐的在床上爬来爬去,所过之处一层白毛。虽然美中不足的给人当了儿子,不过当儿子当得如此舒服,他认了。

史高飞忙忙碌碌关起大门做奶爸,在第二次喂奶之时,他顺便又确定了宝宝的性别——一眼没留意,他的虫宝宝竟然连鸟带蛋的长全了家伙。而他的姐姐史丹凤回了家,对着他们的妈窃窃私语:``妈,小飞好像出事了。''

赵秀芬正在家里唉声叹气,听闻此言,吓得一激灵:``他怎么了?''

史丹凤见神见鬼的压低了声音:``他好像在屋里藏了个婴儿。''

赵秀芬常年闹病,终日打嗝。这时直瞪瞪的盯着女儿,她惊讶的屏住了呼吸:``小飞有孩子了?''

史丹凤也犯嘀咕:``不知道哇!小飞买了那么多婴儿奶粉,还死活不让我进卧室。我听他们卧室里有东西叫,唧唧喳喳的就像小孩!''

话音落下,母女二人一起对了眼。凭着史高飞的好皮囊,只要他自己肯,诱骗几个无知少女还不成问题的。如果真弄出一条小人命了,她们娘儿俩不能坐视史高飞自己胡闹,至少也得把孩子抢回家里来抚养。

良久之后,赵秀芬开了口:``小凤,过几天你再去一趟,悄悄的去,别让他发现。看准了回来告诉我。小飞要真有了我的孙子,我告诉你爸,让他去抢。''

史丹凤听闻此言,有些后悔,心想弟弟虽然疯,但能顶着大太阳去买奶粉,可见还知道疼孩子的。妈和爸说抢就抢,万一把弟弟惹急了,非出大事不可。然而不去也不行,弟弟的精神好一阵歹一阵的,若哪天把小地球人掐死埋了,到时自己岂不悔之晚矣?

史丹凤左右为难,不由得步了她母亲的后尘,坐在房内津津有味的长吁短叹。火星镇上的姑娘只要过了二十五,哪怕天好,都得算老姑娘,而她今年已经满了三十,并且连着两年都没人给她介绍对象了。如果弟弟真鼓捣出了个小婴儿,她满心惆怅的想,自己倒可以帮他养育——老天保佑,可千万别个小疯子。

一个礼拜之后,史丹凤用黑色遮阳帽和雪纺连衣裙再次武装了自己,骑着电动车往村里去了。大正午的烈日高悬,热出了她一头一屁股的汗。为了防止迎风走光,她把裙摆全压在了身下,一边颠颠簸簸的高速前进一边心疼,因为今年夏天只买了这么一条裙子,非得出门时才舍得穿,结果今天下乡跑了长途,非把裙子压出无数褶子不可。

\chapter{姑侄相见}

在距离村庄一里地外,史丹凤提前下了电动车。村里的新幼儿园就修在了路旁,一座五颜六色的二层楼被一圈五颜六色的铁栅栏围了个严实。冒充家长把电动车停到了幼儿园大门口,史丹凤轻装上阵,开始步行前进。村子不是现代化的大村,民居还以平房居多,所以史家的小楼在村边鹤立鸡群,十分醒目。一身的褶子抖索开了,史丹凤顶着烈日骄阳走成草上飞,倒是感觉比骑车更舒服些,因为走得胯下生风,别有一番凉爽。

鬼鬼祟祟的靠近了小楼,史丹凤踌躇了,不知应该如何打探。明公正气的往里闯,自然是闯得进,不过至多进入客厅,想进卧室恐怕是不可能,弟弟虽然瘦如刀螂,但是毕竟有高度,自己一介女流,单打独斗必定不占上风。不进入内部,在外围活动也是个办法,可问题又来了:史家小楼的格局类似缩小版的幼儿园,一圈铁栅栏围住小楼,让她除非翻墙,否则根本无法靠近卧室后窗户。史丹凤身量苗条,翻墙也是翻得动的,然而院后的栅栏外生了一大排苍耳,形成荆棘防线,既防猫狗也防贼,顺便还防了今天的史丹凤。史丹凤虽然身负重任,但也没有为了重任扎死自己的道理。裙角飘飘的站在院后踱来踱去,她两只眼睛盯着左侧的后窗户——窗户挂了窗帘,窗帘一动一动的,显然是卧室里的人不老实。史高飞没有演默片的内涵,既然不老实,就应该同时发出动静。史丹凤在一大片苍耳后面抻了脖子,拼命倾听,听得耳朵都长了,然而一无所获。脸上忽然红了一下,她浮想联翩:``莫非是小孩的妈来了?''

史丹凤冰清玉洁的活了三十年,虽然在读大专时也谈过恋爱,然而始终没走到最后一步,导致她总存着一层不合年龄的羞涩。扭扭搭搭的退了一步,她转念又想:``弟弟是个不通人事的,如果孩子的妈明白道理,自己不如和孩子妈谈一谈。万一谈出了成绩,也不枉自己汗流浃背的跑来一趟。''

思及至此,她当即改变战术。估摸着又过十分钟了,她转到院子正门,抬手去按门铃。一边按铃,她一边看清了院子里堆积如山的奶粉罐子。奶粉的牌子不完全相同,罐子却是统一的漂亮。史丹凤快速的数了一遍,心中大惊:``小飞这是养了几个孩子?开幼儿园也吃不了这么多呀!''

铃声响成一串,片刻之后楼门开了,史高飞拧着眉毛撅着嘴,一脸不情愿的走向史丹凤:``姐,你来啦?''

史丹凤等他给自己开了门。不动声色的走入院内,她问史高飞:``家里有别人吗?''

史高飞立刻摇头:``没有。''

史丹凤飞快的瞟了他一眼,偏巧他也正在瞄着她。两人对视一眼,随即立刻把脸扭开,全是心怀鬼胎的样子。一前一后的进入楼内客厅,史丹凤摘下她的大遮阳帽,同时发现地面瓷砖上一片牵牵连连的细软白毛,屋子里的怪味倒是几乎消失尽了。

走到沙发前放下帽子,史丹凤抬手把一头波浪长发挽成了利落的圆髻,同时闲闲的问道:``小飞,冰箱里有没有雪糕?''

史高飞不知有诈,老老实实的告诉她:``有棒冰。姐你不生我气了?''

史丹凤转身往厨房的方向走,仿佛是要去找冰箱。然而走到半路她一个向后转,以着迅雷不及掩耳之势发足狂奔,``咣''的一声直撞进了卧室里去。史高飞站在客厅中央,只觉眼前一花,卧室房门已经大敞四开。大叫一声追了上去,他在卧室门口撞上了他姐的后背。而史丹凤本在呆站,冷不防从后向前受了冲击,当即顺着力道飞起,结结实实的拍上了正前方的大床。直眉愣眼的一抬头,她的面颊生出毛刺刺的温热触感,正是和床上的无心贴了个脸。

猛然翻身向旁一躲,她彻底看清了面前怪物的全貌。无心此刻似人非人,正处在一个最不招人看的时期。披着一身细软的白毛,他塌着肩膀东倒西歪,细瘦的四肢蜷缩着抱住圆滚滚的大肚皮。至于面孔——虽然骨骼轮廓基本成形了,但是眼睛还不能睁。粗线条的大眼眶里,乌溜溜的大眼珠子在半透明的眼皮下转来转去,让人想起一枚巨大的胚胎。

史丹凤瞪着他,一声没吭,气都不喘了。一条毛巾被从天而降展成幕布,她看见她弟弟手忙脚乱的包裹了面前的怪物,又很怜爱的把他整个抱起,藏宝似的背对了自己:``姐,你不要吓到他。''

史丹凤冷笑一声,心想凭着我和它的形象,要吓也是它吓我,我怎么还能吓到它?

然后她双眼一翻,嗓子里``嗝喽''一声,晕过去了。

史丹凤做了个短暂的噩梦,噩梦的背景和情节都很杂乱,集她所看过的恐怖片之大成。后来她在哭天抢地之中骤然苏醒了,发现自己躺在弟弟的大床上,脚上的高跟凉鞋已经脱了,额头上搭着一条冷冰冰的湿毛巾。

``飞啊\ldots{}\ldots{}''她哼哼的叫唤:``小飞\ldots{}\ldots{}''

床尾传来了史高飞的回答,声音还挺温柔,是难得的有人味:``姐,没事,我在这儿呢。''

史丹凤慢慢的抬手扯下毛巾,然后歪了脑袋往下看。第一眼她没看到史高飞,看到的是史高飞腿上的毛巾被大包袱。包袱上面才是史高飞的面孔,而毛巾被里又伸出了一个白茸茸的脑袋,脑袋很亲热的枕在史高飞的宽肩膀上。

史丹凤一言不发的闭了眼睛。定神片刻之后睁眼再看,看到的还是包袱和史高飞。攥着毛巾坐起了身,她彻底的认清了现实。

``小飞啊\ldots{}\ldots{}''她恹恹的开口问道:``你这猴儿是从哪儿逮的?''

史高飞从来没见他姐闹过毛病,今天说晕就晕,导致他十分关怀。然而他姐刚一苏醒就不说好话,导致他瞬间变脸,不但嘴角下垂眉梢上扬,甚至连鼻孔都呼扇呼扇的扩大了些许:``不许你说他是猴儿!''

史丹凤苦口婆心的要和他讲道理:``小飞,你想养宠物,姐不拦你。养个小猫小狗都行,还能给你解个闷。但是你不能养这东西,这东西太吓人了。市里不是有个动物园吗?我回去查查号码,给动物园打个电话,问问他们要不要这玩意。要是人家肯接收的话,小飞,你听姐一句话,赶紧把它送走吧。再说报纸上都写了,看什么像什么,你总对着这么个东西,时间一长,你也得长成它这模样。''

此言一出,白毛脑袋自惭形秽似的向下缩了缩。而史高飞十分怜爱的轻轻拍了拍他的后背,然后抬头对着史丹凤长叹了一声:``姐,你不知道前因后果,所以我不生你的气。对你说句老实话吧,姐,其实他是我的儿子。''

史丹凤看到弟弟病情陡然加重,真是快要落泪:``就算它是你的儿子,可是谁给你生的它呢?''

史高飞傲然扬眉:``姐,我给你看几张照片。看完照片,你再判断我是不是胡说八道。''

史高飞力大无穷的抱着毛巾被包袱起了身,走到电脑桌前坐下。弯腰摁了电源开关,他一边等待开机,一边用双臂环抱着怀里的无心。及至电脑打开了,他打开了一个层层加密的文件夹,然后起身说道:``姐,你看吧。宝宝是在两个月大时被我挖出来的,你看他当初是不是个猴儿?''

文件夹里存放着上百张照片,一天一张的记录了无心的生长过程。史丹凤坐在电脑屏幕前,一张一张的仔细看过一遍——看完一遍,再看一遍;看完两遍,她魔怔了似的,从头开始看第三遍。

末了她松开鼠标转向史高飞,垂死挣扎的问道:``是你PS的吧?''

史高飞不理她了,在床上展开了毛巾被,自得其乐的喂无心吃手指饼干。无心靠在两只摞起来的大枕头上,脑袋向后仰着,吃完一根等下一根。一只发育未完的手搭在史高飞的膝盖上,手背指缝全是白毛,指尖红通通的没有指甲。

史丹凤讪讪的:``也可能是电影截图,你当我什么都不懂?重口味电影多了去了,你随便找一部来骗我,我也不知道。''

史高飞已经听不见她的嘀嘀咕咕了。一双眼睛望着无心,他兴致高昂的笑道:``叫爸爸。啵——啊——叭!爸爸!''

房内随之响起了两声怪叫,第一声极其高,第二声极其低:``爸——爸——''

史高飞幸福死了,拍着大腿哈哈大笑。而披毛戴角的无心在眼皮下面转动了眼珠,一个脑袋也悄悄的转向了史丹凤的背影。四十来年没闻过女人味了,刚才史丹凤汗津津的落到了他身边,把空气搅得暗香浮动。回忆起双方面颊的一蹭一贴,无心忽然兴奋了,为了让对方能够回头再看自己一眼,他撒欢似的一跃而起,一头扎进了史高飞的怀里,又可着嗓子大嚎了一声,吓得史丹凤当场出溜到了电脑桌下。

史丹凤留下没走,关起房门和史高飞密谈了小半天。及至到了傍晚时分,她亲自去村口超市买了菜肉,给弟弟做了一顿有凉有热的好饭菜。

史高飞要带着无心一起上桌,史丹凤坐在两人对面,左一眼右一眼的一共看了无心两眼,随即愁眉苦脸的说道:``小飞,你能不能把它送回卧室里去。我一看它就头晕。''

史高飞堪称通情达理,先把无心送回房内,然后又往卧室运去了半锅米饭以及半桌子菜。最后和他姐相对而坐了,他老气横秋的摆出了家长派头:``姐,你看看,真是不养儿不知父母恩。只要他能吃饱,我饿着都愿意。''

史丹凤身心俱疲的抄起了筷子:``既然它能吃饭,以后就别给它买奶粉了。奶粉多贵啊,有钱得计划着花。爸可连着一个月没回家了,万一哪天他真跟着小狐狸精跑了,我没工作,你还不如我,妈的退休金也少,咱们可怎么办?到时候你还想养猴——儿子?恐怕你连自己的嘴都糊不住了。''

史高飞充耳不闻,不能理解地球人的烦恼。

史丹凤看了他这个德行,一颗心越发悬在了半空。和这弟弟是争论不出是非黑白的,她这弟弟可是经过官方认证的妄想症患者。

史丹凤没吃饱。饭菜全被史高飞拿去喂儿子了,以至于他们姐弟两个竟是不够吃。

她洗了碗筷,整理了冰箱,又拖了地板。悻悻的走去幼儿园门口骑上电动车,她回了家。到家之后面对着赵秀芬的盘问,她一时也不知从何说起,只讲自己上次是听错了,弟弟还是一个人过,家里并未多出下一代。

赵秀芬听了,十分失望,唉声叹气已经不足以抒发她沮丧的心情,于是她开始哐哐的打嗝,每一声都是气运丹田,发自肺腑,如同一口酸菜缸在翻江倒海的冒泡。史丹凤绝望的看着她妈的今天,宛如看到了自己的明天。

史丹凤心事重重的在家熬了整半个月。到了第十六天头上,她实在是放心不下史高飞,并且担心史高飞会被猴儿夜里吃掉,于是跨上她的坐骑又下乡了。

这回她光明正大的在院门外下了车。门铃刚响了没几声,史高飞便喜气洋洋的跑出来了:``姐!给我带吃的了吗?''

史丹凤看了他的喜色,不由得怀疑自己是皇上不急太监急,愁得毫无必要。推着电动车进了院,她还没来得及发话,就被史高飞抓住了手臂:``姐,你进来,我让你看一样好东西,你看了一定高兴。''

史丹凤实在是想不出弟弟会给自己什么惊喜。强打精神的跟着他进了客厅,她站在原地,连坐都懒得坐;史高飞则是松开她冲进卧室。不过几秒钟的工夫,卧室房门轰然而开,史高飞拦腰抱着无心冲进客厅,对着史丹凤高声笑道:``哈哈!姐,看哪!我儿子帅不帅?''

史丹凤把眼一瞪,手里的车钥匙当场落了地——猴儿没了,她看到了一个有模有样的青年人。

史高飞继续高声大笑:``姐,我儿子有名字的,你猜他叫什么?他叫无心?无是没有的无,心是人心的心,无心就是没心。可惜我不是他哥哥,我要是他哥哥,我就改名叫无肺。哈哈哈哈哈,你看我儿子多体贴,知道我初中没毕业,直接自己把名字想好了,省得我还得费心思。哈哈哈哈哈!''

在他连说带笑之时,无心已经溜出他的臂弯,双脚一起落了地。穿着史高飞的大短裤大T恤,他那还未最后定型的身体显出了几分少年气。一双阅人无数的黑眼珠子盯住了史丹凤,他看出对方是个美人,而且处在美的巅峰,是一枚果子熟透了,不知在接下来的哪一刻会过了季节。灵机一动的亮了眼睛,他踉跄着向前走了几步,然后张开双臂一扑:``姐。''

结结实实的,他扑进了史丹凤的怀里。史丹凤鼓溜溜软颤颤的胸脯贴上了他,他那还未收缩回去的大肚皮也老实不客气的顶向了她。史丹凤莫名其妙的被他搂了个密不透风,眼睛顺便看清了他一头刚刚破土而出的厚密黑发,以及耳根颈窝处留存的几根白毛。

史高飞撵了上来:``不对不对,她是我姐,你得叫她——姐,他该叫你姑姑还是大姨?''

史丹凤梦游似的看着弟弟:``应该叫姑姑吧?''

史高飞把无心从史丹凤身上扒了下来:``宝宝,听话,叫姑姑。''

无心心怀鬼胎,不肯认她做长辈,抿着嘴只是对她笑。装疯卖傻的机会不是常有的,他得把机会利用住了。和白琉璃猫头鹰搭伙过了四十年,现在他一想起那二位就要吐,岂止是审美疲劳,简直疲劳出了内伤。如今总算落回了人窝子里,单是守着个疯小子混吃混喝也不算有前途,要是能和面前的美人勾搭上,生活才叫有滋有味。

无心现在站得还不大稳,然而身残志坚,依靠着史高飞坚持微笑,左一摇右一晃,笑得摇曳生姿。史丹凤被他连看带笑,心里乱七八糟的直发毛。也许对于这个先是虫子中间是猴最后变人的东西,弟弟的那一套奇谈怪论都是真的,可如果都是真的,未免过于不可思议。应该把这个东西交给政府,让科学家好好研究研究,不过想想而已,不能真做。弟弟疼他疼得像眼珠子一样,管他是什么怪物,留下来能给弟弟作伴也是好的。也许弟弟心情一好,病情也能有所好转呢!

史丹凤浮想联翩,站在地上出了神。忽然身边起了声音,她低头一瞧,见无心给她搬了一只小圆凳:``姐,坐。''

史丹凤把嘴一咧,对着他``呵''了一声,是想笑而没笑出来。

无心开始献小殷勤,凳子面明明没有灰尘,可他偏要用手掌擦拭一遍,结果在干干净净的凳子面上留下了几根白毛。等到史丹凤坐稳当了,他又搬了个更矮的小塑料凳,自作主张的坐到了史丹凤身边。史高飞没想到他会骤然吃里扒外,连忙上前拉他:``宝宝,你怎么跟她好上了?走,爸爸带你回屋去!''

无心一晃肩膀,从他手里抽出了手臂。史高飞拽了个空,当即迁怒到了史丹凤身上:``姐,全怪你!你把我儿子教坏了!''

\chapter{偶遇损友}

史丹凤独自走在镇中心的火星大街上,大街两旁商铺林立,其中有一座大厦最为出众,乃镇里第一高级的百货大楼。百货大楼共有五层,楼内的服饰以落后北上广一个季度的速度,紧追慢赶的随着潮流变化。大楼的一楼面朝南北分别开了肯德基和麦当劳,另有许多花花绿绿的招牌入口,通往地下的超市和大电玩城。

史丹凤没有伴,只有一个妈和她朝夕相处。然而跟着赵秀芬出门实在太遭罪,自从史一彪不肯回家之后,赵秀芬就落下了打嗝的毛病,兴起之时不分场合,咕咕嘎嘎的说打就打。史丹凤随着这样一位有声有色的母亲上街,时常要羞臊的无地自容,所以宁愿独来独往。拎着一只飘飘摇摇的空布袋子进了地下超市,她在货架之前来回穿梭,想要给弟弟和弟弟的儿子各买几条内裤。无心一天一个模样的变化着,终于在前天彻底变成了人,而且个挺好看的人。既然成了人,就得穿戴成个人模样。不过他到底能不能算人呢?史丹凤也说不好。

几十分钟之后,她拎着布口袋重返地面。正要过马路去停车场取电动车,不料站在街边抬眼一瞧,她忽然看到了弟弟和无心——两人一高一矮的站在冷饮店的玻璃柜台外,正在等待小店员给他们制作蛋卷冰淇淋。一只冰淇淋从里向外递给史高飞,冰淇淋顶端甩出了个摇摇欲坠的尖。史高飞接过冰淇淋,先一舌头把尖舔去,然后转身拉起无心的一只手,小心翼翼的把冰淇淋塞进了他的手里。无心背对着史丹凤站立,从史丹凤的角度眺望他,只能看到他握着冰淇淋一低头,随即他昂首挺胸,一边鼓着腮帮子东张西望,一边用手指把冰淇淋的蛋卷尾巴摁进了嘴巴里。

史丹凤看得好生心痛——弟弟从土里刨出了个大吃货!

现在和弟弟见面无话可说的,她决定还先自行回家。

无心穿着史高飞的大短裤和大衬衫,在冷饮店前一口一支的吃冰淇淋。冷饮店内也有几张小桌子,食客们偷眼观瞧,见史高飞一手托着无心的后脑勺,一手捏着一张面巾纸,正在很细致的给他擦嘴;一个中学女生暗暗的摸出手机,瞄准他们偷拍了一张照片。

史高飞的眼里除了儿子之外,什么都没有;正喂儿子哄得愉快之时,忽有一只手拍上了他的肩膀。他回头一瞧,没看见人,视线向下一扫,他和来人打了照面。对方名二十岁上下的青年,头顶竖着一溜鸡冠子似的冲天红头发,两鬓的头皮却剃得发青,越发显得一张国字脸硕大无朋。又因为他脸大,所以修饰的余地也足,眉钉鼻环唇环一应俱全。仰头对着史高飞嘻嘻一笑,他虽然形象不羁,言谈倒客气:``飞哥,好长时间没看见你了,听说你搬到村里住去了?''

史高飞认出了他:``李光明。''

原来李光明乃他幼年时的邻居,少年时的学弟。此人初中辍学,横行于火星镇的中小学校,企图在校园之内发家致富。史一彪他的人生偶像,连带着他对史高飞也高看了一眼。虽然史高飞常年不说人话,但他坚持着维护住了二人的友谊,隔三差五的必定和史高飞见一面,顺便向对方借个三百五百的零花钱。自从史高飞进了村,李光明对他遍寻不着,导致手头十分拮据。如今终于又见了面,他喜气洋洋,恨不能当街载歌载舞:``你有新朋友了?怎么不给我介绍介绍?''

史高飞一本正经的把无心揽到了怀里:``光明,他不我的朋友,他我的儿子,已经六个月大了。等他满周岁了,我想给他摆桌酒庆祝一下,到时候请你一个,你要来哟!''

话音落下,他低头在无心的额头上亲了一下。无心麻木不仁的咀嚼着嘴里的蛋卷,又抬手舔了舔手指上的奶油。

李光明直了眼睛,感觉自己今天开了眼界:``哦\ldots{}\ldots{}飞哥,我说你不找女朋友呢,原来\ldots{}\ldots{}你说他你儿子?''

史高飞一点头:``对啊!''

李光明讪讪的笑:``好,好,你俩还挺有情趣。飞哥你说你真的,原来还总跟我装性冷淡,说你一辈子不找地球人,没想到其实你口味比谁都重。''

史高飞没听明白,微微低头看他:``什么意思?''

李光明退了一步:``没、没什么,我说你俩感情好。''

史高飞把无心搂到了身前,带着他左右的摇晃:``我们当然感情好。父子你知道吗?他我生的,我和他天下第一亲。''

李光明一抖朋克头:``肯定的呀!''

李光明虽然名叫光明,其实一贯向往黑暗,还在后脖颈上刺了个蝙蝠精似的撒旦纹身,自以为酷得要死。和史高飞分别了半年多,没有一见面就开口借钱的道理,于他决定先和对方培养培养感情。真没想到他的飞哥不但精神病,而且同性恋,堪称双料的变态。不过看在钱的面子上,李光明决定伏低做小,如果史高飞出手够大方,自己也可以捏着鼻子硬着头皮让他占点便宜。

把史高飞和无心带进了地下电玩城,他热情洋溢的要请史高飞父子的客。火星镇的电玩城个藏污纳垢之所,里面黑灯瞎火乌烟瘴气,良家的少男少女从来不会光顾。李光明给史高飞买了一小口袋游戏币,然后化身为尾巴,亦步亦趋的跟着他想要搭话。哪知史高飞如今做父亲的人了,视野变得极其狭窄,除了儿子再无其它。无心则大开了眼界,抬腿跨坐上电脑屏幕前的摩托车,他抬手去拽史高飞的前襟,在震耳欲聋的游戏声中问道:``爸,怎么玩?''

李光明听在耳中,心想叫爸叫得这么痛快,不知道史高飞给这小白脸花了多少钱。精神病的钱都要骗,真没人性啊!

然后等到史高飞为无心投了游戏币后,他拽了拽对方的后襟:``飞哥,跟你商量个事,最近你手里方便吗?''

史高飞疯归疯,但不白痴。很警惕的扭头看着李光明,他大声反问:``干什么?又想和我要钱?''

李光明扯着嗓子否认:``哪要哇?借!''

史高飞把手伸进裤兜,掏出了乱七八糟的一大卷子钞票。李光明看得眼都直了,心想这回精神病可能要大放血,不料史高飞慢条斯理的把钞票整理了一遍,然后只抽出了一百块钱:``我现在上有老下有小,手里也不宽裕。喏,给你一百,不用还了。''

李光明接过一百块钱,心中无比失落。悻悻的向后退了一步,他感觉脚下有异,脚跟用力碾了几碾,他莫名其妙的转了身,发现自己方才竟踩上了大人物的脚面!此人物横行于火星镇长途汽车站一带,乃李光明前途道路上的劲敌。两人大眼瞪小眼的对视片刻,大人物心平气和的开了口:``找死啊?''

李光明攥着一百块钱反问:``操!你对谁说话呢?''

大人物当即伸手搡了他一下:``我对你说话,有问题吗?''

李光明冷笑一声:``我看你人话说的不怎么样啊!用不用我重教你一遍?''

话音落下,双方动了手。李光明满以为史高飞看到自己挨揍,即便不大呼小叫的出面阻拦,也该跑出电玩城,到街对面的网吧里召集自己的小弟前来对战。哪知他这边人脑袋都要打成狗脑袋了,史高飞那边却岿然不动。和无心并肩坐稳当了,他们两个拿着鼓槌,正在梆梆梆的玩太鼓达人。无心听后面叫得如同屠宰场一样,忍不住的回头要瞧。然而史高飞接二连三的把他的脑袋扳向前方,又撅嘴在他脸上亲了一大口:``不看不看,宝宝看了要害怕的。''

无心听了这话,倒心有所感——好些年没听过这么充满爱意的话了,乖乖的把脸转向电脑屏幕,他又问史高飞:``爸,姑姑怎么连着好几天都不来了?''

史高飞一边挥着鼓槌敲敲打打,一边不以为然的答道:``宝宝,我看你特别的喜欢姑姑。其实姑姑不怎么好,她总对我唠唠叨叨,说你吃得太多,还说你个怪物。哈哈,在她们地球人的眼中,我们当然怪物了。''

一只臭球鞋飞过了两人的头顶,惊叫声在电玩城中此起彼伏,一群花枝招展的小女生蹦蹦跳跳的逃向了通往地面的大门。电玩城乱成了一锅粥,唯有史高飞和无心无比的淡定。警察都来了,他们还在跟着节奏敲鼓。

警察几个新警察,没见过派头这么大的,于决定给这二位来个下马威。史高飞和无心被警察吆喝着押出电玩城,先人一步的上警车了。

留守在派出所里的老警察认出了史高飞,当即为难的开始搓手。一个精神病,而且经过调查,还个无辜的精神病,抓了没用,随便放了又怕担干系,非得把他亲手交回史家才妥当。抛下史高飞再去审问无心,无心睁着两只奇大的黑眼睛,一言不发的四处乱看。老警察长叹一声——又一个精神病。

下午时分,老警察联系到了史一彪。十分钟后,史一彪驾到。

史一彪身为火星镇以及周边五村三屯的首席,自然气度不凡。坐着一辆道上专用的丰田霸道,大吉普鸣着喇叭停在了派出所门前。车门一开,史一彪闪亮登场,颠着一身肥肉喘下了车。他发福发得早,十几年中积累了一身的脂肪,热得夏天永远光着膀子,身上肥肉一圈一圈,人送外号米其林。戴着墨镜仰头看了看前方,他脖子上的金链子立刻被后脖颈的肥肉掩埋了一半。双手抓住裤腰向上提了提,他迈动着两只穿着白皮鞋的大胖脚,开始向派出所的正门行进,一边走一边又喊:``小飞,爸来了!''

所长和老警察守着两个精神病,正度日如年,如今忽然听到了史一彪豪气干云的大叫,立刻松了一口长气。恭而敬之的把史高飞和无心全送出了门,他们感觉自己算过了一关。

史一彪对于史高飞一直感情复杂,爱爱的,可养他养得毫无指望,自己简直像在家里供了个高达一米九的大花瓶。看儿子安然无恙,他放了心,随即再看儿子一手领着个白脸青年,他傻了眼,一颗肥腻的心又提到了喉咙口:``小飞,这谁啊?''

史高飞对他很宽容的笑了一下,决定忘记他曾经把自己强送进了精神病院:``爸,他我的儿子,血统和我一样,很纯粹。''

史一彪目瞪口呆,而所长走上前去对他使了个眼色,随即小声说道:``什么证件都没有,来历不明,好像精神上也有点问题。有人说他和小飞同性恋关系,不知道真假。''

史一彪听得血压升高,压低声音问道:``谁说的?''

所长答道:``李光明。''

史一彪不想在派出所门前丢人现眼,于把史高飞和无心一起带上了车。一路疾驰回到了家,他把家里的史丹凤和赵秀芬一起痛骂了一顿,说她们吃闲饭的货,小飞往家里藏了个陌生人,而且藏了好几个月,她们竟然腆着大脸一无所知,真欠揍。

然后他把史丹凤单拎出来,说自己``一看见你就愁得慌'',``挺大姑娘没人要'',``一脸倒霉模样'',``和你妈一模一样''。

史丹凤和赵秀芬一声不敢吭,老老实实的挨骂。而史一彪颇有计策,并不肯和儿子正面交锋,只对家中两位女性发威:``我今天晚上要去县里,一个礼拜之后回来。到时候小飞如果还这样,你们还干吃饭不管事,你们等着,老子把你们全轰出去!''

眼神凌厉的又横了无心一眼,他闭了嘴,肉山一样大踏步的冲向门外,走得太有速度了,迎面都起了风,导致乳头上的几根毫毛风中凌乱。一堆麻烦扔给老婆女儿,他预备着一个礼拜之后再回家,到时家里恢复原样,自己还可以在儿子面前做个好人。

赵秀芬被丈夫恐吓了一顿,一时间忘了打嗝,只逼问无心:``你哪儿来的人呀?你怎么还赖上我家小飞了呢?你走吧,你没听见小飞他爸刚才的话吗?唉,你说你可真的,这不给我们家添乱吗?你为什么不说话?你不想要钱?''

史高飞旁听至此,终于怒不可遏了:``妈!你别说了!他我的儿子,千真万确我儿子!谁敢撵他,我就和谁没完!''

赵秀芬吓得一哆嗦:``小飞啊,不我撵他,你爸撵他。''

史高飞把无心拽到自己身后,然后对着他妈他姐做狮子吼:``谁撵也不行!你们再敢提这话,我就和他一起走!''

史丹凤知道内情的,但又不能实话实说,因为怕被人也当成精神病。眼看弟弟气得青筋迸出,她见缝插针的开了口,说要先送弟弟和无心回村。有话可以慢慢说,横竖还有一个礼拜的时间呢。

带着给他们买的新内裤出了门,三个人乘坐长途汽车出镇进村。史高飞一直不发一语,及至进了家门,无心从冰箱里拿出两瓶矿泉水,先拧开一瓶给了史丹凤:``姐,喝水。''

史丹凤愁容满面的看了他一眼,不知拿他如何好。本来她那长相就个林黛玉的风格,如今眉毛一蹙脸一沉,更好看了。

她不喝水,把一口袋新内裤交给了史高飞。史高飞向内一看,见内裤连个塑料包装带都没有,就抬头说道:``姐,你又给我买便宜货!这大裤衩,老头子都不爱穿。你再来摸摸,料子又粗又硬的,非把我儿子的屁股磨破了不可。''

史丹凤叹了一声,暂时无计可施,只好去赶末班车回家。等到她走远了,史高飞关了层层房门,却把无心拽到了自己面前:``宝宝,我们有危险了!''

无心仰头喝了一口矿泉水,忽然想起史高飞进门之后还滴水未沾,便连忙把瓶子递给了他。然而史高飞心事沉重,无意喝水:``现在我有了你,我们的人数多了,力量也大了,所以地球人一定起了警惕心,又要迫害我们了!去年他们曾经干过一次这样的事情——他们把我骗进医院里,故意让我讲明自己的身份,然后大剂量的逼我吃药,想要拿我做实验。幸好我够聪明,装成合作的样子,哄他们放我出了院。宝宝,医院实在太恐怖了,我无论如何都不能让你再去受我受过的罪。既然他们已经露出了凶恶的真面目,我们也必须行动、不能坐以待毙了!''

无心观察着他的表情,看他一脸杀气,不个好惹的势头。伸出舌头舔了舔嘴唇,他小声问道:``爸,你要干什么?''

史高飞伸手抓住了他的手腕,一字一句的告诉他:``爸爸要带着你逃!''

\chapter{新房客}

无心如今已经成长发育完全,既能见人,也能远行。对于外面光怪陆离的新人间,他真太有兴趣出去大开眼界了。

他极力赞同了史高飞的逃跑计划,虽然史高飞根本没打算征求他的意见。对于没满岁的外星儿子,史高飞认为自己作为父亲,很有资格为他做主。打开电视调到动画片频道,他让无心看电视,自己则楼上楼下的四处乱转,想要开始收拾行囊。行囊的内容有的,然而行囊的躯壳却缺乏。空荡荡的小楼里先前没人过日子,什么存货都没有。史高飞忙了一晚上,只翻出一个破了洞的蛇皮袋子。把蛇皮袋子往地上一掼,他抬手抓了抓头发,然后出门奔向了村口超市。

无心一直留意倾听着他的动静。听他真出大门了,无心溜下床去打开了电脑。双手抓着鼠标,他左调右试的移动屏幕上的箭头,根据记忆找到了史高飞多年积攒的精神食粮。随便挑选了一部片子打开,他先直着眼睛看,看着看着张了嘴,及至史高飞拎着两只小书包回家时,他已经流了口水。史高飞没想到儿子如此早熟,把他从电脑屏幕前扯开之时,裤裆里竟然都支了帐篷。三下五除二的关了电脑,他不耐烦的训斥无心:``地球人有什么好看的?你和我一样没出息!''

无心坐在床上,面红耳赤的仰头看他,眼睛睁得又圆又大,自知羞愧,所以下意识的模仿了大猫头鹰。和大猫头鹰过了四十年,他学会了不少装模作样的把戏。

史高飞老气横秋的又叹一声,然后继续收拾行装。

如此过了一夜,翌日凌晨天刚刚亮,史家小楼就有了动静。史高飞穿着一身长衣长裤,一手扯着同样装扮的无心。仔仔细细的锁好大门之后,他们蹑手蹑脚的出了巷子踏上柏油路,肩并肩的走向了朝阳升起的地方。两人的背影一高一矮,各自背着村口超市出售的小学生书包。书包粉红色的,印着美羊羊。

出了村子继续往东,走出不远便火星镇长途汽车总站,终点唯一的,除了县城哪儿也不去。史高飞和无心上了首班车,车空调大巴。无心占据了靠窗的座位,因为过于兴奋,所以坐立不安东张西望。抬手拨弄着头顶上方的空调出风口,他想起了蹲在山里的白琉璃和大猫头鹰,心中别有一种幸灾乐祸式的喜悦,同时又得意的对他们做出了新评价:``两只卑鄙的土鳖。''

时间一到,大巴发动,沿途随叫随停,一路捡客上车。史高飞心跳如擂鼓,生怕下一位拦车人会史一彪或者史丹凤。双臂横撂在前方座位的靠背上,他向前俯身做睡觉状,把一张脸藏了个严严实实。

大巴一旦驶出了繁华的火星镇地界,因为沿途荒凉,速度自然加快,一路黄烟滚滚,跑了个无影无踪。两个小时之后到了站,史高飞带着无心下了大巴,心中依旧不安,因为史一彪如果昨天没撒谎的话,此刻应该也在县里。他爸位不显山不露水的高手,虽然一身五花三层的好膘,然而高人大隐隐于肉,一旦出手,必定打出严重后果。史高飞高得飘飘摇摇,无论如何不敢和他爸单练。一手死死的攥着无心,他鬼鬼祟祟的带着儿子走小路,直接奔了火车站。

史高飞没有真正出过远门,一时间也茫然无目的。看到最近的一班列车正好能到三百里外的江口市,而江口市他已经去过好几次,于他没犹豫,当即买下两张车票,带着无心再次出发了。

史高飞说走就走,连个招呼都不打,导致他都走了三天了,史丹凤才发现弟弟和侄子一起没了。

她敲不开史家小楼的院门,给弟弟打电话,弟弟又一直关机。亲自翻墙进入院内,她连拍窗户带踢门的叫了一气,一边叫一边预感不妙。等到确定了弟弟的失踪之后,她吓出了一身冷汗,连滚带爬的骑上电动车回了家,又连哭带嚎的告诉赵秀芬:``妈!小飞跑了!''

赵秀芬一口气没上来,在胸口憋成了一个极大的嗝。哆哆嗦嗦的给史一彪打了电话,她没敢哭,只说:``你把小飞给吓跑了!''

史一彪事业缠身,百忙之中回了家,先把母女二人大骂一顿,然后愁得满地乱走,因为近来正要办大事,无暇去找儿子。待到他走出家门了,史丹凤灵机一动,却追上了他:``爸,你别急,我闲着没事,我去找小飞。''

史一彪扭头看她:``你上哪儿找去?''

史丹凤答道:``我先去县里,县里没有再去市里。小飞也没出过远门,他可能只一时害怕,想找地方躲一躲。''

史一彪皱起眉头:``行啊,去吧。''

史丹凤紧紧跟着他:``爸,我下午就出发,你给我点儿路费。''

史一彪问道:``你要多少?''

史丹凤不假思索的答道:``有个三万五万也就够了。''

史一彪一瞪眼睛:``你要包机啊?''

史丹凤平心静气:``爸,我不知道我得在外面住多少天啊,穷家富路嘛。''

史一彪被她说的发懵,糊里糊涂的点了头。史丹凤平白无故的得了四万块钱,把钱尽数存入自己的账户里,她因为嫌长途大巴太贵,所以下午走长路去了镇火车站,花三块五毛钱买了一张硬座火车票,往县城去了。

在史丹凤成本低廉的浪迹天涯之时,史高飞已经和无心在江口市安了身。史高飞活了二十五年,第一次品尝到了拮据的滋味——离家出走时手里只有万把块钱,到了江口市之后连吃带住的玩了几天,他也没感觉自己如何挥霍,但已经连间像样的房子都租不起了。

租不起像样的,只好租不大像样的。市内有一条百年风情老街,老街两边都奇巧的老店,而老店之间偶尔会有幽深的胡同,狭窄细长,不像给人走的,倒像给蛇走的。小胡同两边开着大大小小的院门,有的院子洁净点,有的院子污秽点,无论形象如何,房屋本身统一的老旧残破。这样的位置,民工不肯来;这样的环境,白领也不肯来。所以老房子的房租很便宜,想要租,随时都能找到空房。

史高飞不想在大街上风餐露宿,所以很果断的进入百年风情老街。老街毫无风情,游人稀少,也正合了他的心意。在一间还算整齐的小院子里,他和无心得到了一间屋子。屋子里家具一应俱全,只差餐具和被褥。

他们不唯一的租客。院中一共有三间平房,余下两间,据房东说,属于一对父女。其中正房窗明几净,门口挂了块牌子,写着``易经研究所''五个大字。厢房垂着花窗帘,看不清房内情形。

房东走后,史高飞带着无心也出了门。买到一床被褥回了来,他们刚刚进院不久,院门一响,却另两位房客出现了。

史高飞正在和无心合力铺床,房门开着,一个大女孩子哼哼唧唧的含糊歌声传进屋内。史高飞警惕的向外望了一眼,看到了一个穿着花裙子和平底凉鞋的女生——说不准十五六还十七八,露出的小腿和手臂都圆滚滚的白皙。手里拎着一篮子青菜,她也好奇的去看史高飞和无心,脸圆,眼睛大,忽然傻乎乎的对着房中二人一笑,她``嘿''的笑出了两个酒窝。

没等她笑完,院里响起了圆润的男子声音:``佳琪,别乱看。''

佳琪很听话的拎着菜回屋了。史高飞转向了无心,低声说道:``她长得有点儿像林嘉欣。''

无心走到床边坐下了,环视着房内的旧桌子旧椅子旧纱窗。还村里的小楼好,又宽敞又明亮,但自己不能埋怨史高飞的。史高飞对他太好了,于他决定和史高飞一起疯一疯。

``爸。''他很自然的对史高飞说:``我饿了。''

史高飞弯腰抹平床单皱褶,然后起身往院外走,要去给儿子找吃的。无心坐在房内,听到他在院子里和男子声音搭起了话。男子有一副华丽的好嗓子,说话时带着隐隐的膛音。三言五语的交谈过后,男子和史高飞一起出了门,同去胡同口买酱肘子。及至把酱肘子买回家,两人已经开始谈笑风生。无心站在窗前往外看,只见男子能有个四五十岁的年纪,穿一身干干净净的灰色唐装,看容貌堪称美男子,鼻高眼大,面孔圆白,又架了副金丝眼镜,富富态态的十分体面。他让他家的佳琪给史高飞切碎了酱肘子,又把自己饭锅里的米饭挖出半锅请客。史高飞碗筷一概没有,索性隔着院子大喊道:``宝宝,来吃饭!''

话音一落,无心推门露了面。

院内摆起一张小圆桌,四人团团围坐,不出一顿饭的工夫,已经互相摸清了底细。原来中年男子名叫白大千,挂着``易经研究所''的牌子,他神棍;摘了``易经研究所''的牌子,他无业游民。他女儿白佳琪已经十九岁了,虽然他自己不承认,但连史高飞都看出了她有点儿傻。

白大千也认定了史高飞和无心全有问题,而且还精神方面的问题,但问题不大,即便疯子,也属于文疯子。他带着个傻女儿,不敢招灾惹祸,所以满面春风,不肯多问。及至吃完了一顿饭,他摇着蒲扇进了正房。打开电脑开始搞事业——为了淘到第一桶金,他在各大论坛注册了马甲无数,使用各种办法自炒。每天晚上频繁的换马甲打广告,生意还没上门,他自己却快要累得精神错乱。

佳琪在院内的公共水龙头下洗碗。史高飞走到她面前,一本正经的说道:``喂!你长得像林嘉欣。''

佳琪惊讶的抬头看他,一脸傻相:``啊?''

史高飞扭头走了,一边走一边告诉她:``我要哄我儿子去了。我儿子还小,要过很久很久才能长到像你一样大。等他长大了,我就可以享清福了。''

史高飞躺在床上,用一柄蒲扇给无心撵蚊子。无心百无聊赖的仰卧着,忽然扭头去看他:``爸,我们买个电视机好不好?''

史高飞起身拽过粉红色的小书包,从里面翻出一大把钞票。把钞票整理了一遍,他重新躺下了,拿起蒲扇叹了口气:``爸爸的钱不够啊。''

无心又道:``那我们去想办法挣钱吧!''

史高飞刚要回答,冷不防正房忽然传来一声大叫。随即院子里起了啪嗒啪嗒的脚步声,正白大千穿着拖鞋在疾行。敲开了史高飞的房门,他一脸喜色的低声说道:``老弟,我想求你和你——你儿子帮个忙。一会儿有个客户要来见我,你俩能不能给我当一晚上的徒弟?''

史高飞莫名其妙:``当徒弟?怎么当?''

白大千笑道:``很简单,很简单。有人敲大门的时候,你去开门,然后问他:`来找我师父的吗',他说,你侧身往正房的方向一指,告诉他`师父在等你'。''

史高飞又问:``我儿子呢?我儿子干什么?''

白大千笑得满面红光:``他的工作更容易了,站在正房门口,负责给客人开门。''

史高飞想起他家的大米饭和林嘉欣,决定和地球人合作一次。白大千得了承诺,激动的满院乱转,又自言自语的暗笑:``终于要开张了!''

白大千换了一身新衣,扛着放光的大脸,坐在正房专候贵客。入夜之后,果然有客来访。史高飞和无心规规矩矩的把客人放进了正房。一个小时之后,客人匆匆离去,白大千也露了面。呼吸过几口新鲜空气后,他转向史高飞和无心,笑眯眯的问道:``你们二位,明天有时间吗?''

不等他们回答,他搓了搓手,心痒难耐的又道:``不瞒你们说,我刚接了笔大生意,明天要出一次门。单枪匹马的去不好看,想要带两个人装装门面。我看你俩形象还不错,要愿意的话,明天跟我走一趟,我决不让你们白走,有好处的。''

史高飞怀疑的望着他:``去哪里?''

白大千心旷神怡的答道:``也不算远,江口市不挨着江吗?明天我们过江,到江对岸去的度假村。对方车接车送,还管一天三顿饭。我不用你们干什么,等到了度假村,你们自己到处玩玩也行。但当着外人的面,你们不能——不能暴露父子身份。''

史高飞想了又想,最后感觉白大千应该不邪恶一派,故而点头答应了。

第二天上午,一辆小汽车停在胡同外,接走了气派俨然的白大千以及客串徒弟的史高飞和无心。两个徒弟还各拎了布口袋,里面装着白大千的罗盘和法器。

汽车一路开到江边,白大千等人下车上船。小船突突突的开到江心,无心举目远眺,已经看到了对岸一片精致洋楼。旅游的旺季刚过,他们到岸之时,度假村里已经游人稀少。昨天夜里拜访白大千的来客,自称黄经理,如今又出现了,引着白大千一行人走上度假村外围的林荫大道,一边走一边压低声音说道:``我上个礼拜已经把E区封锁了,对外说要装修;好在夏天刚过了,没太耽误营业。原来的服务员都辞职不干了,我好容易才又招了几个保安。白大师,拜托了,想想办法帮帮忙吧。''

白大千微微一笑,笑而不语。及至绕过了一大片红顶白房子,他们停在了一处富丽堂皇的大门前。大门里面不别墅式的小洋楼了,而一幢酒店式的五层楼。楼不算很大,然而造型别致,突出的阳台栏杆上缠着牵牛花藤,应该本来要走田园风的,但因为太久无人打理了,以至于花藤疯长,勾结连环的绿成了一面墙。

黄经理向楼一指:``白大师,您看,E区不大,主要建筑就它。''

白大千面沉似水:``进去瞧瞧。''

黄经理掏出手机打了个电话,很快从楼后跑出了三名保安,为首一人方面大腮,步伐矫健。史高飞看在眼里,惊在心中——对方竟然李光明!

李光明跑到半路看到了史高飞和无心,险些当场刹了闸。及至黄经理带着白大千走入E区了,史高飞落了后,低声询问李光明:``你怎么来了?''

李光明抬手扶了扶帽子,朋克头和一脸的环全没了:``我坐长途车来的。''

史高飞又看了他一眼:``我还以为你被人打坏了。''

李光明嘁嘁喳喳的耳语:``我装的,没真受伤。我怕他们找我报仇,所以一出医院就跑了。今天上午刚找的工作,当保安,一个月八百,管吃管住。我想我先干着,过一阵子我再回家。飞哥,你呢?你怎么也来了?''

史高飞感觉李光明智商极低,自己没有必要和他推心置腹,故而言简意赅的答道:``不知道!''

李光明又问:``你俩还在一起哪?挺好,爱情不分公母,感情好比什么都强。我让我女朋友跟我一起走,她死活不同意。我一生气,临走前把她甩了!''

史高飞拉着无心的手,李光明的话总让他懵懵懂懂:``什么意思?''

李光明转移话题,亲亲热热的又道:``我明白了,你们来玩的。富二代就命好,我比不了。可我告诉你们啊,出去住小别墅去,别往那楼里进。上午我刚换完制服就听人说了,这楼里不太平,好像闹了半年的鬼,没看里面既没有游客也没有服务员吗?''

李光明一片好心,然而史高飞并不领情,扬着脑袋就要去追白大千。一条腿刚刚作势要迈,他手臂一紧,却无心把他拽住了。

``不去。''无心双手一起拉扯着他,脸上一本正经的没笑容:``爸,我们不进去。''

史高飞愣头愣脑的看着他:``宝宝,爸爸昨天都答应白大千了,现在不进去不好吧?''

无心微微下蹲,用身体的力量坠住了他:``我们在外面等他也一样的。''

史高飞很听儿子的劝。李光明被他的保安同僚叫走了,他和无心呆站了片刻,感觉十分无聊,又见白大千和黄经理始终不出来,便自作主张的出了E区。沿着林荫大道走出不远,他忽然停了脚步,指着路边一家报刊亭上的海报说道:``宝宝,看,她就林嘉欣,你说她像不像佳琪?''

无心看了看海报,然后问道:``爸,你喜欢佳琪?''

史高飞犹犹豫豫的摇了头:``不,我不喜欢地球人。''

\chapter{异象}

白大千在楼内上下走了一圈,一路昂首挺胸不回头,导致他一直没发现身后少了两个伪徒弟。乘坐电梯下到一楼,他把脑袋转向黄经理,慢条斯理的问道:``盖楼之前,这片地方做什么用的?''

黄经理答道:``原来游泳池。游泳池度假村刚开业那些年挖的,年头太多,功能跟不上潮流了,所以就把池子填上盖起了楼。''

白大千淡淡一笑:``这就对了。这里本来蓄水的地方,阴气最重。盖起楼后,楼内大厅里还砌了小喷泉,正地下阴气未竭,地上阴气又起。还有,从二楼开始往上,我看卫生间的房门怎么全开到了走廊两端?''

黄经理变脸失色:``大师,我们不犯了什么忌讳?''

白大千叹了口气:``本来就不好风水,而你们又错上加错,终于把一座楼变成了凶宅。罢了,我先给你两张五行八卦福贴一贴,如果压得住,你们运气好;如果压不住,怕就要破土动工,把整座楼改建一番了。''

话音落下,他想从自己的布袋里拿五行八卦福,然而回头一瞧,他发现史高飞和无心竟然无影无踪。

李光明四处奔波,费了偌大的力气找到史高飞和无心,让他们马上回E区。而白大千从史高飞的袋子里取出两张印着福字的红纸片,轻描淡写的递给了黄经理,又漫不经心的索要了两千块钱。黄经理病急乱投医,大师肯要,他就肯给。而白大千在话中留了个小尾巴,正色告诉黄经理道:``如果五行八卦福无效的话,你一定要立刻去找我。我一来帮人帮到底,送佛送到西,二来我慈悲为怀,也不希望有人因它受害。''

黄经理看他神情凝重,语气坚决,心中不禁悚然,连连点头答应。

中午在度假村吃了顿饭,下午白大千等人先乘船后乘车,顺顺利利的回了家。白大千虽然在事业上常年失败,可的确个称职的好父亲。他一上午赚了两千块钱,然而只取出十分之一送给史高飞做酬劳。在风情老街口的饮食摊子前买了许多样小吃,他欢欢喜喜的要拎回家给女儿。史高飞和无心又落了后,无心拿着一张一百元,看画似的看了半天,末了停了脚步,对史高飞说道:``爸,我要吃汉堡。''

史高飞答道:``可爸爸想吃牛肉面。''

无心知道史高飞可以由着自己欺负欺负的,所以停了脚步:``我不想吃面条。''

史高飞忽然怀疑自己把儿子惯坏了:``不行,一定要吃。''

一个小男孩在老街中央撒泼打滚,给了无心些许启示。和史高飞对峙了片刻,他一屁股也坐在了地上:``你不给我买汉堡,我就不回家!''

在行人的注视下,史高飞老鹰抓小鸡似的把他拎起来往肩上一扛,随即一拐弯进了胡同。无心悔之不及,心想自己光顾着学习撒泼,就没想到小男孩的撒泼对象个老掉渣的爷爷;自己的撒泼对象却一条好汉。

无心在半路落了地,并且服了软,告诉史高飞:``爸我再不敢了。''

史高飞像个真正的父亲一样,大模大样的拍了拍他的后脑勺,然后领着他进了院子。佳琪穿着一身旧运动服,正站在院子中央吃油炸臭豆腐。臭豆腐盛在一只小纸杯里,佳琪吃得满嘴都红油。对着史高飞和无心咧嘴一笑,她把纸杯递向了他们:``臭豆腐,爸爸买的。''

史高飞被臭豆腐熏得闭了气,但对着面前这张圆白甜美的面孔,他没好意思逃。佳琪并不能体谅他的痛苦,笑得露出了牙缝里的碎辣椒:``哥哥你吃,宝宝也吃。''

哥哥快要窒息而死,推辞不吃;宝宝用牙签一次扎穿三块臭豆腐,一下子全塞进嘴里去了。嘴里嚼着臭豆腐,无心又提醒佳琪:``我知道你爸还给你买了猪肉脯和奶酪。''

佳琪没心眼,得意的承认,随即跑回房里拿出她的存货,要和他们分享。正房中的白大千站在窗前,一边看着无心往嘴里塞猪肉脯,心里一边犯嘀咕——如果无心真疯子的话,未免疯得过于狡猾。给自家女儿买的食,全被他吃了。

而如果无心不疯,那他和史高飞又到底什么关系?白大千想了又想,无论如何想不明白。

如此过了两天,院内平安无事。倒天气变化明显。按节气看,早就入秋了;按温度看,秋意却刚刚到来。

史高飞取出有限的一小笔存款,给自己和无心各添了一件外套御寒,又在小店里给佳琪买了个发卡。发卡上面粘着一枚硕大的蝴蝶结,谁戴上了都会像米老鼠。把发卡送给佳琪,他板着脸说道:``喏,给你。其实你长得挺好看的,就头发乱七八糟。你就不能把头发好好梳一梳吗?''

佳琪欢天喜地的接受了礼物,当场戴到了头上,又跑去正房让白大千看。白大千看得心事重重,怀疑史高飞对自己女儿图谋不轨。正踌躇着不知该不该让女儿退回发卡之时,黄经理忽然又来了。

黄经理来了,在正房里和白大千密谈了一个小时,然后惶惶然的又走了。待他走远之后,白大千一如既往的搓着手出了门,满面红光的在院内乱转。正心潮澎湃之际,他身旁忽然响起了一个声音:``白叔叔,你真的会降妖除魔吗?''

白大千一扭头,看见无心不知何时端着一杯牛奶进了院。另一只手捏着一片奥利奥夹心饼干,他把饼干放到牛奶里浸了浸,然后往嘴里一塞。

``这个\ldots{}\ldots{}''白大千忽然有些不安。要说降妖除魔,他毫无疑问的绝对不会,但翻过一本图解易经,虽然最终还没看懂。不过话说回来,`不会'不问题,不会可以装会。凭着他的服装、气度、年龄、以及与年龄十分相衬的美貌,他自认为别说装大师,倒退几百年装皇帝都够了。

白大千思来想去,末了没有正面回答,而问无心道:``你真小史的儿子吗?''

无心认认真真的点头:``嗯,!''

白大千一笑:``那我也真会降妖除魔。''

无心仰头喝了一口牛奶:``你如果再去度假村的话,把我带上吧。''

白大千不置可否的望着他,感觉他话里有话。如果他真的只文疯子的话,带着也行,权当给自己壮声势了。不过要带他的话,也得带上史高飞,不能让史高飞和自家女儿同处一个小院儿。可三个人都走了,留下女儿一个人也不行\ldots{}\ldots{}

白大千对着全院宣布自己又有了新的生财之道,并且生大财。有意给他当徒弟顺便分一杯羹者,可以速速到正房报名。等到史高飞父子报名完毕了,他带着佳琪出了门。坐上出租车直奔城南的金光寺。原来他不家中独子,虽然父母亡了,但还有一位常年不相往来的在金光寺当住持。他与在年轻时统一的英俊潇洒,为了佳琪的妈争风吃醋。最后白大千胜出,他哥则万念俱灰的出了家,法号汇丰。

转眼之间过了二十年,白大千混得穷困潦倒,汇丰却名利兼收,成了全省有名的大和尚。白大千人穷志短,每逢穷得要断顿了,便要去向汇丰化缘,汇丰看在佳琪的面子上,也只好捏着鼻子施舍。如今白大千要去做大事了,无处安顿女儿,情急之下索性把女儿送到金光寺,让她和女居士们先住几天。

及至把女儿安顿好了,他一身轻松的回了家。到了翌日下午,一辆小汽车把他和两个徒弟送去江边。和上次的路线一样,他们直接进了E区。

进入楼内的时候,已经傍晚时分。黄经理又出现了,先请他们吃了晚饭,然后把他们一直送入三楼客房。走廊两端的公共卫生间已经被封锁了,门前还各自摆了一座屏风。黄经理已经不敢在楼内久留,白大千也不需要他陪伴。待黄经理离去了,三个人各归各位。白大千占据了一间客房,隔壁则住着史高飞和无心。

史高飞始终犯着糊涂,糊里糊涂的来,糊里糊涂的住。进入客房之后,他先打开了电视机:``宝宝,你不要看电视吗?''

无心坐在床上环顾四周,发现客房格局很简单,进门条短短的过道,过道一侧开了门,通往洗手间。客房本身方方正正,有着大吊灯和曳地的窗帘,床也双人大床。对着大床的电视柜,电视柜短短的一截,紧挨着电视柜还有立柜。楼新,家具也新,空气中几乎还存留着一点油漆味。

无心让史高飞和自己一起看。两人挤着半躺半坐,不过片刻的工夫,史高飞歪着脑袋先睡了。无心扶他躺好,然后自己关了电视和吊灯。

屋中黑到了伸手不见五指的程度。无心静静的站在黑暗之中,抬手咬破了食指指尖。这一阵子他营养充足,血液也充盈。湿漉漉的指尖划过史高飞的眉心,他噙着手指一动不动。口中弥漫了甜腥滋味,他闭上眼睛,只感觉怨气正在源源不断的从下向上蒸腾,如果他个正常人的话,现在必定已经毫无缘由的心烦意乱了。

房内没有完整的鬼魂,所以他也找不到可消灭的对象。隔壁静悄悄的,似乎还不必让他亲自过去查看。抬腿上床躺下了,他提醒自己不要睡。

一小时后,在史高飞的鼾声中,他不由自主的睡着了。

与此同时,白大千却冷不丁的睁了眼睛。

仰面朝天的躺在被窝里,平心而论,客房的条件要比家里好。他睁着眼睛愣了愣,不知道自己怎么会无缘无故的醒。把手伸出被窝,他一边去摸床头柜上的眼镜,一边心中暗想:``不管怎么样,只要我能在这里住得平安无事,黄经理就不能说我法力不高。至于我走了之后这里再闹鬼,和我可就没有关系了。所以无论如何我得在这里熬过一个礼拜,要不然明年的房租还没着落呢!''

他想得有条有理,同时急着下床撒尿,然而手在床头柜上摸来摸去,就摸不到眼镜。他一时急了,欠身想要开灯,可在他抬起头的一刹那,他忽然用力挤了挤近视眼,怀疑自己看到了什么。

与此同时,他的手指肚有了冷硬触感——他摸到了自己的眼镜镜片。

抓起眼镜戴上了,他望着斜前方一哆嗦。随即摘下眼镜用枕巾擦了擦,他重新戴上再看,眼前一片漆黑,却又没有异常景象。

想起黄经理的所言所欲,白大千的一颗心开始在腔子里怦怦乱跳了。打开壁灯坐起身,他鼓起勇气下了床,走去卫生间尿了一泡。然后战战兢兢的回到床上关了灯,他缩在蓬松的大被窝里,闭上眼睛暗想:``我刚才\ldots{}\ldots{}好像看到了一排手指头。''

一只手的四根手指头,自内向外惨白的扒在了立柜门边,立柜的门要开不开,手指头也要露不露,只显出了指尖。

``立柜我打开过的,里面有股子甲醛味,所以我没往里面挂衣服。''他自己盘算:``立柜肯定空的,我当时看得很清楚。''

他在被窝里想要翻身,然而脑袋一动,才意识到自己还戴着眼镜。试试探探的抬起头,他想要给自己一个保证,让自己确定刚才一瞬间的所见全幻觉。

然而在他举目向前的一刹那,他明显感觉自己的汗毛竖起了一层——在一片漆黑之中,他再次看到了扒在柜门上的鬼手!和上次相比,鬼手已经露出了第一指节。房内分明没有光,可白大千却能看清手上青紫破碎的长指甲。

白大千一声不吭,直直的躺回了原位,告诉自己:``眼花了,睡觉!''

在眼镜片后闭了眼睛,他极力的想要入睡。午夜时分万籁俱寂,他怎么躺都不舒服,同时听到自己的心在咚咚大跳。脑袋失控似的悄悄歪向一旁,他不由自主的又望向了立柜。

四根手指一起露出来了,微微蜷曲着想要抓挠柜门。立柜里忽然发出``咕咚''一声闷响,像有人在里面敲击壁板。里面``咕咚''一声,房间门外竟然有了回应,由远及近的一串脚步。脚步沉滞,仿佛鞋底始终没能离开地面,一步一步疲惫不堪的拖着走。

白大千没有再躺,直勾勾的盯着立柜发傻。脚步声越来越近了,不知道最后会走来一个什么东西。扒在柜门上的鬼手似乎一动不动,可只在他一眨眼的工夫里,鬼手的姿势骤然起了变化。手掌慢慢向上举起扶住柜门边缘,最后就听``吱嘎''一声,柜门竟然被那只鬼手猛然推开了!

白大千和柜子里的东西打了个照面。脑子``嗡''的一声刮起龙卷风,他往床上一栽,屎尿齐流的晕了。

天将亮时,白大千悠悠醒转。缓缓的睁开眼皮,他先嗅到一鼻子恶臭,还以为自己躺在了公厕里。及至慢吞吞的起身一瞧,他才找到了臭气源头,顺便把昨夜的惊魂一幕想了起来。

他下意识的想要逃,可逃了就没钱赚,没钱赚就要连累女儿和自己一起挨饿。摘下眼镜揉了揉眼睛,他发现立柜柜门关得严丝合缝,并无异状,而阳光透过窗帘射入房内,可见外面还个秋高气爽的艳阳天。

白大千把心一横,决定不逃

\chapter{鬼迷人}

白大千换了内裤洗了澡,又把脏床单卷成一卷扔进走廊。梳起分头擦亮眼睛,他敲开了隔壁的房门。

史高飞和无心也已经醒了,两人正在睡眼惺忪的坐在床上说话。无心让史高飞以后不要当着外人叫自己``宝宝''。史高飞不以为然,还问无心:``你进入叛逆期了?''

不等无心回答,他张开大嘴,气吞山河的打哈欠:``长得真快,你还没满周岁呢!''

无心没听懂``叛逆期''三个字,还要继续和他讲道理,然而白大千不请自来,站在床边问他们:``你们昨夜\ldots{}\ldots{}睡得怎么样?''

史高飞向后一躺,闭着眼睛又要睡:``挺好,比家里舒服。''

白大千又问:``夜里\ldots{}\ldots{}什么都没看见?''

无心东张西望的环顾了房内:``睡觉嘛,闭着眼睛,能看见什么?''

白大千抬手摸着下巴,心中十分疑惑,暗暗的自问:``莫非我夜里产生了幻觉?或者噩梦做得太真?''

白大千没敢妄言,怕把史高飞父子吓跑了。史高飞和无心虽然不大正常,但毕竟活生生的带着热气。白大千昨夜惊魂一宿,如今看到人类,感觉十分亲切。

到了上午八点钟,李光明出现了,手里捧着高高一摞方便饭盒,正要送给白大师及其弟子享用的早餐。找机会和史高飞搭上了话,他似乎也颇为寂寞:``哎?你怎么又来了?我明白了,你还真不来玩的,可你怎么和白大师混上了?史叔不管你了?''

史高飞不爱搭理他:``别和我说话,我现在没钱。''

李光明夜里在网吧过了一宿。如今黑着眼圈,精神不济:``飞哥,我问你件事,你们昨晚睡觉的时候,楼里没别人吧?''

史高飞摇头:``没有。''

李光明心中一动,笑嘻嘻的不再说话了。

自从白大千等人进驻E区之后,原本留守的保安们就清闲了,唯一的工作便轮班给楼内的半仙们送饭。时光易逝,转眼间一天过去,又到了天黑时候。白大千在外面散步完毕了,背着手慢慢的往楼内走,心里一边惦念着远在金光寺的女儿,一边惶惶的不知今夜又会出什么幺蛾子。本来他自诩为无神论者,以为凭着自己这样光辉的形象与气质,就算真有鬼魅魍魉,也会被自己的气场镇住,自惭形秽的退散。没想到事情不那么简单,如果今晚还一个惊魂夜,白大千在电梯门前停住脚步,心想自己也许应该去史高飞的房里挤一挤。

电梯门一开,白大千迈步进去,随即转向门口,伸手一摁控制面板上的数字三。电梯门缓缓合拢,白大千对着锃亮的电梯门照了照,感觉自己太帅了。

电梯轿厢微微一颤,随即下方发出``喀喇''一声巨响。轿厢内的灯光忽然灭了,白大千惊叫一声,就感觉电梯正在直线下降——可他此刻人在一楼,电梯还能降到哪里去?

他向后一步靠在了壁上,慌乱的安慰自己:``应该有地下负一层——负一层停车场,仓库\ldots{}\ldots{}什么都合理!''

不知过了多久,电梯依然保持着下降的状态,而且高速下降。白大千在黑暗中抱住了头,想要拼命的吼叫,然而在极度的惊恐之中,他的呼吸暂停了,声音也哽在了喉咙里。肠子忽然一绞一绞的做了怪,白大千呜咽一声,心想自己真人间奇葩,都到了这般时候了,居然还有心思闹肚子。

他弯腰抱头夹腿提肛,浑身肌肉一起抽筋。周遭漆黑如墨,他想要掏出手机照照亮再看看时间,可手脚统一的不听使唤,而且依着他的胆量,他其实也不大敢弄出动静。大眼睛在眼镜片后快闭没了,他简直无法面对此刻的现实。

不知过了多久,他的双脚忽然感觉到了震动——电梯停了!

强大的惯性让他浑身一哆嗦。黑暗之中响起了一线细声,十分高亢的拐了个弯。白大千缓缓的睁开了眼睛,听自己把屁放成了小调,不禁有些脸红。

电梯门无声的开了,迎面吹入一阵冷风。他向外一瞧,登时大吃一惊。原来电梯之外竟一片连绵的荒冢,荒冢上方的夜空中,还悬挂着一轮苍白的明月。

``怎么搞的?''白大千捂着肚子想:``电梯把我送到荒郊野外来了?''

这样的情形,单单的想想不通的,况且白大千还有更急迫的事情要做。一只脚迈出电梯,他在外面找到了一块颇为沉重的大石头挡住了电梯门,然后一侧身出了电梯,他掏出一小包香喷喷的面巾纸,解开裤子蹲到了一处小土丘后。度假村提供的伙食实在好,而且一人给了两人的量。他认为自己在外面多吃点,回家就可以少吃点,所以仿佛将要冬眠似的,也不管能否消化,一味的只往嘴里填。

声势浩大的拉了一通,白大千提着裤子起了身,顺便看清了电梯的位置。电梯居然开在了一座巨大的山岩表面。两扇电梯门正在一下一下的磕打着大石头。神清气爽的揉了揉肚子,白大千略略镇定了,开始思考现实问题。

在头顶月光的照耀下,他环顾四周,发现这片荒野无边无际,远远近近的鼓着小坟头,有的坟头竖着残破墓碑,有的坟头则光秃秃的一个土馒头。白大千见最近的坟头距离自己不过几步远,便壮着胆子走上前去,弯腰去看碑上文字。碑青灰色的石碑,斑斓破败,字迹模糊。他一时好奇,把怕给忘了,从裤兜里掏出了手机。此手机外壳金光灿烂、设计清新霸气,虽然价格只有九九八,但汇集天下手机功能之大成。手机屏幕上已经没有了信号显示,电量倒满格。他轻轻一摁手机侧面的摁钮,手机顶端立刻射出一束白光,强度可以和普通手电筒媲美。白光自下而上的照过碑文,白大千一边看一边不出声的念:``墓之奶三王母慈。''

念完之后他愣了愣,心想:``奇了怪了,我妈也叫王三奶。''

他想再去看看立碑人的姓名,可未等他调转灯光,碑文上方嵌着的小小遗像已经映入了他的眼中。鼻孔瞬间哼出两道凉气,他手指一紧,几乎攥碎了手机。

他看到了他妈的黑白照片!

他妈年轻时不美人,老了之后越发堕落,奔着丑陋的方向一去不回头。阴惨惨的苍白背景前,他妈微微低头面对前方,一脸怪笑,松弛的大厚嘴唇横贯下半张脸。眼睛部位却模糊了,看不清黑白眼仁,只能看到眼角翘得喜气洋洋。

``不可能啊\ldots{}\ldots{}''白大千此刻肠子里没了存货,所以只勉强在裤裆里挤出了几点尿:``我妈葬在老家的,电梯再怎么快,也不至于一下子把我送回老家吧!''

双掌合十夹住他的手机,他慌里慌张的对着墓碑拜了拜,决定还回电梯里想办法上楼去。可就在他将要抬脚之时,身后忽然掠过一阵凉风,吹来了一个苍老的女声:``大千啊,你个不孝子,怎么来了就走?''

白大千一回头,差点没把肠子拉出来——不知何时,他妈出现了!

他妈还穿着当初下葬时的装裹衣裳,模样和表情都和墓碑上的遗像一模一样。一头白发在风中飘乱了,她笑眯眯的白眼上翻,竟然已经没了黑眼仁。对着白大千伸出一只手,手指腐烂得露出了白骨。一条黏腻的长虫瞬间隐入她的袖口,她一步一步磕磕绊绊的走向了白大千:``我儿,你不来,你也不来,妈想你们啊!''

白大千向后退了一步:``别,妈,让爸陪您也一样的,我们兄弟两个\ldots{}\ldots{}四十年后再来找您吧!''

他妈缓缓的摇了摇头,嘴角流出了乌黑的口涎:``不行,妈等不得了。''

白大千眼看他妈越逼越近,自然不肯老老实实的坐以待毙当孝子。转身一大步越过一座小坟头,他撒腿要往电梯里跑,哪知脚踝忽然一紧,他低头一看,只见土中伸出了两只白骨手掌,死死的锁住了自己的左脚。

白大千吓得一晃,抬起右脚狠狠踩向了探出地面的骨头腕子。几脚踩出一地零碎骨头,他无暇往电梯里躲了,只能慌不择路的背对着他妈逃命。前后左右的地面无端的一起颤动了,远处一座坟头忽然裂开,一具不成人形的腐烂尸首爬上地面,东倒西歪的也追向了白大千。

白大千攥着手机,终于爆发似的喊出了声。汗湿的右手死死的握着手机,也不知碰到了哪个按键,手机背面的八个喇叭骤然爆发出了一阵激昂音乐,随即一个女声唱道:``谁在唱歌,温暖了寂寞,白云悠悠蓝天依旧泪水在漂泊\ldots{}\ldots{}''

身后的腐尸越来越多了,在王三奶的带领下紧追白大千。白大千来不及关手机,只能合着节奏东逃西窜。正走投无路之时,前方忽有一道白光从天而降,他猛然刹住脚步,只见无心出现在了自己面前。而未等他开口求援,无心一手抓住他的衣领,另一只手高高扬起抡圆了,``啪''的抽了他一记大耳光!

白大千被他打得脑袋一晃。短暂的眩晕过后他回了神,眼前骤然大放光明。月夜、野地、荒坟、厉鬼、乃至他妈,全不见了。他发现自己正坐在三楼光滑的走廊的地面上,身后靠着墙壁。旁边哐哐的总有声音,他扭头一看,看到了挡在电梯门间的铁皮大垃圾桶,以及走廊正中央的一泡屎。

气若游丝的张了嘴,他对着无心开了口:``你怎么来了?''

无心一指他的手机:``我被它吵出来了。''

白大千虚弱的关了手机,脑子里还一团乱麻:``我怎么了?''

无心迟迟疑疑的摇了头:``我\ldots{}\ldots{}我不知道。你不看到什么了?''

白大千不敢再隐瞒了。一手抓住无心的手臂,他哼哼唧唧的答道:``我糊涂了\ldots{}\ldots{}我刚才一直以为我在乱坟岗子里被鬼追杀,直到你给了我一个大嘴巴。到底怎么回事\ldots{}\ldots{}我不知道,真不知道。''

无心小声嘀咕道:``不会真的有鬼吧?''

白大千做了个深呼吸,想要说话。可没等他开口,无心又嘀咕道:``听说\ldots{}\ldots{}鬼能迷人的。''

白大千问道:``什么迷人?''

无心不看他,垂着脑袋低声答道:``让人产生幻觉\ldots{}\ldots{}其实我随便说的,我也不懂。''

白大千带着哭腔开了口:``妈的,绝对啊!我一定被鬼迷住了,我明明人在三楼,怎么会一直以为自己在坟地里?幻觉,一定幻觉!呜呜呜\ldots{}\ldots{}无心,幻觉好可怕噢!''

无心扭头看了地面一眼,随即继续嘀嘀咕咕:``白叔叔,你随地大便也好可怕噢!''

无心把白大千带到了自己的房内。史高飞正坐在床上看一部奇长无比的老韩剧,白大千见了他的大个子,忽然感觉很有安全感,当即抛弃无心,主动挤到了他身边坐下。

史高飞十分擅长把自己和外界隔离,想要把谁忽略,就真能对谁一眼不看。但白大千泪水澎湃,一定要和他讲一讲自己方才的历险记。及至白大千哭哭啼啼的说完了,史高飞漠然的答道:``你们地球人真麻烦,几个死掉的同类也能让你们又哭又叫,一群精神病。''

白大千愣了愣,随即试试探探的问道:``史老弟,你\ldots{}\ldots{}你在自嘲吗?''

史高飞不看他,心不在焉的反问:``自嘲什么意思?''

无心怕史高飞一旦明白了自嘲的意思,会把白大千暴打一顿。于走到两人中间,左撅右挑的用屁股拱出了座位。三人并肩坐在大床上,白大千不住的打激灵,无心一言不发的装傻,只有史高飞一派恬然,对着电视嘿嘿发笑。

笑着笑着,他忽然不笑了。他眼中的电视机自动关了,四面的墙壁上却缓缓渗出了鲜红的血字。

他看见了,白大千也看见了,两人一起瞪圆了眼睛。只有无心还处在一如往常的世界里。只在天花板的角落处,他发现了一张铁青的鬼脸。

鬼脸个女人的形象,轮廓面貌十分清晰,可见她颇有力量,绝非新鬼。鬼脸瘦骨嶙峋,一双眼睛深深陷在了黑眼窝里,一侧颧骨裂开一道伤口,能够隐隐看到外翻皮肉下的白骨。对着床上三人,她满意的咧嘴一笑,露出了一口断裂残缺的牙齿。

无心的目光一放即收,也效仿史高飞和白大千,做了个瞠目结舌的惊讶表情。正要观察女鬼下一步的动作之时,冷不防身边的史高飞忽然先动作了。

史高飞一跃而起。高高大大的站在床上,他双手叉腰,怒不可遏的大声吼道:``谁弄脏了我的墙?!''

他真生气了,一边晃脑袋一边跺脚,大脚丫子踩得床垫悠悠乱颤:``我要在这间屋子住满七天的,今天刚到第二天,就被你们把墙壁画得乱七八糟!到底谁?给我站出来!你们这帮低素质地球人,老子今天非揍扁你们不可!''

话音落下,他眼前一花,随即看到了一只漂浮在半空中的人头。他看到了,无心和白大千也一起看到了——这回不幻觉,女鬼借着夜间阴气盛,自己现了形。

白大千嗓子里咕咕嘎嘎,个随时要晕厥的样子。无心咬着手指头,蓄势待发想要驱鬼。而史高飞抡起拳头向前击去。拳头在鬼影之中打了个空,他气得大骂:``操!3D效果挺好哇,我还以为个真人呢!''

无心正要咬破手指放血,万没想到史高飞会发此高论。张着嘴仰望了史高飞,他和女鬼一起傻眼了。

与此同时,一个黑影鬼鬼祟祟的进入了E区大门。

一身保安制服的李光明走得蹑手蹑脚,一边走一边东张西望,生怕被人发现行踪。听闻E区现在日夜都没人管了,他决定趁夜潜入楼内,偷点值钱的东西做福利。至于鬼不鬼的——反正他没亲眼见过鬼。和鬼相比,穷更可怕。

\chapter{破土}

李光明蹑手蹑脚的穿过院子,穿过了一楼半开半掩的大玻璃门。一楼大厅的喷泉早停水停电了,他怕乘坐电梯会有声音,所以想要绕过喷泉,走正前方的主楼梯步行上楼。微微弯腰踮了脚尖,他无声无息的走到了楼梯口。一只脚刚要踏上第一级台阶,他忽然听见身后起了``嗤嗤''的水流声音。

他停了脚步,一边回头一边心想:``莫非喷泉的水管子漏水了?''

一楼没开灯,三楼住着半仙,或许该有光亮的,然而亮度有限,决计照不到楼下。借着玻璃门外的月光星光路灯光,李光明伸着脖子眯着眼睛细瞧,先没瞧出什么,随即感觉空气变了味道,血腥气渐渐的浓了。

``怎么回事?''他心里打了鼓:``莫非我还赶上了凶杀案?不会飞哥把大师宰了吧?妈呀,飞哥精神病,杀人不犯法,大师白死了!''

正胡思乱想之际,状如莲花的喷泉底部忽然``咕咚''一声冒了泡。随即莲花之中充当花蕊的一束水管汩汩的开始向外喷涌液体。液体流淌得沉重而又平稳,不像水。李光明奓着胆子转了身,走上前去想要细看。然而未等走近,他忽然脚步一顿,感觉自己已经看明白了。

喷泉之中,正在喷血!

李光明站在原地做了个深呼吸,并且狠狠的咽了口唾沫,顺便把快要蹦出口的心脏一并咽入肚中。解开制服衣领扯了扯,他露出脖子后面的撒旦纹身,紧接着双手合什对着喷泉低声说道:``神仙鬼神,不管你什么吧,总之你忙你的,我忙我的,咱们大路朝天、各走半边。我不耽误你闹,你也别耽误我偷。好不好?实不相瞒,兄弟我十三岁混社会,江湖上人送外号赛撒旦,一贯心狠手辣狼心狗肺。和我合作只有好没有坏。你自己想想吧,我先走了。''

话音落下,他硬着头皮做了个向后转,明明两条腿都硬了,可还坚持着往楼上走。楼里闹鬼,据说,也不一天两天的事情了。他今天不下手,明天再来恐怕也没有好果子。择日不如撞日,既来之则安之吧!

抬手抓住楼梯扶手,他走不动了,想要借力把自己往上拽。楼梯扶手木质的,手触之处一片光滑。握着扶手合拢手指,他手臂刚一运力,手指却一滑。扶手在他手中蜿蜒的动了,他扭头一看,吓得险些把眼珠子瞪出眼眶——不知何时,扶手上竟然缠满了蛇!

他猛的收回了手,同时发现在无数下垂蠕动的蛇尾之中,隐约仰着一张苍白的大脸。此脸阔大如盆,一时间也看不清五官。对着李光明张开了嘴,大脸露出一口锯齿獠牙,颗颗都青中透紫,齿间还挂着黏涎。

李光明大叫一声,慌不择路的撒腿狂逃。因为楼下正有一座血喷泉,所以他索性头也不回的冲向了上方。脚下的地板变成了粗糙肮脏的崎岖土路,他连哭带叫的一路往上跑,跑着跑着忽然脚下一滑。脑袋``咣''的一声磕上地面,他痛得眼冒金星,仿佛脑浆都被震成了碎豆腐脑。晕头转向的爬起身,他眨巴眨巴眼睛,又一怔——土路没了,灯光有了,看着雪白墙壁上的指示牌,他发现自己已经上了三楼。手扶墙壁起身低头再瞧,他登时苦了脸——怪不得滑了一跤,原来一脚踩到屎了!

他想不出谁能在走廊里大便。白大师风度翩翩,没有在走廊里方便的可能性;无心应该也不会,凭着他能把史高飞骗得团团转,他的智商也不该低到随地大小便;唯有史高飞最可疑。现在楼里没有勤杂工了,如果有了大事小情,全得由保安负责,比如处理一堆屎。

李光明拼命的蹭了蹭鞋底,上楼不敢,下楼也不敢。思来想去的,他决定还暂时终止行动,找个人类陪伴自己等天亮。他知道白大千等人所住的客房号码,一边试试探探的往前走,他一边左思右想:``我找谁比较好呢?照理说应该去找飞哥,可现在大半夜的,万一他和骗子正在互爆菊花,我在一边看还不看?不看的话不自然,看的话又太尴尬。虽然我外表狂野,但内心还很清纯的。''

思及至此,李光明决定去白大千的房内躲一夜。白大千气度不凡,又一身温文尔雅的派头,必能大发善心收留他。

一步一步的走到了白大千的客房前,他本来想要敲门,可后脖颈忽然掠过一阵阴风,吓得他慌忙抬手推门。一推之下,他发现门没有锁,只虚虚的掩上而已。

一只脚迈入门内过道,他为了显得自己有礼貌,故意轻声呼唤:``白大师,睡了吗?你别怕,我不坏人,我楼下的保安哪。''

房内没有开吊灯,只在卫生间里亮着一盏小灯。过道墙壁上钉着几枚衣钩,其中一枚钩子上挂着个锦绣灿烂的布口袋。李光明抬手捏了捏口袋一角,感觉里面装的仿佛都书籍一类,不有钱的样子。往前再走几步,他发现大床上一片狼藉,并没有人,心中不禁又生了邪主意。暗想大师吃得好穿得好,必定穷不了。悄悄转身走到电视机旁的立柜前,他想要在大师身上发点小财。

手指刚刚攥住立柜的门把手,李光明还未用力,耳朵却听到走廊外面有了门响。他吓了一跳,本意想放手,然而手比心快,居然先人一步的拉开了柜门。端端正正的面对了立柜里面,他倒抽一口冷气,只见立柜内的一层搁板之上,正放置着一只半腐的人头。

人头披头散发,眼睛似闭非闭,屋中黑暗,越发显得人头肤色泛青,嘴唇却鲜红,两道眉毛也黑得奇异,竟像死后多时重新出了土,又被人粗略的化了个妆。收回开门的手送到嘴边,李光明咬着手指,转身要逃。与此同时,白大千摸摸索索的从外面走入过道,正要抬手去摘他的布口袋。两方来了个顶头碰,未等李光明出声,白大千先嚎叫了:``嗷!''

然后他拎着布口袋转身冲出门外,走腔变调的高喊:``来人啊!救命呀!有个鬼要用板砖拍我啊!''

隔壁房门一开,无心伸头拦住了他:``鬼在哪里?''

白大千一侧身,``刺溜''钻进了无心的房内。李光明紧随而至,嘴里还咬着自己的手指头,呜呜噜噜的说不出话。无心莫名其妙的将他打量了一番,然后问道:``你拿板砖了?''

李光明疯狂的摇头。

白大千听无心语气淡定,忍不住转身走回了门口。上一眼下一眼的看了李光明半天,他恍然大悟:``哦\ldots{}\ldots{}我看错了。他脸方,我还以砖呢!''

李光明求援心切,所以特别自觉。在门口把踩过屎的皮鞋脱了,他光脚挤入房内,一边挤一边哭哭啼啼:``飞哥,不得了,白大师的柜子里有个人脑袋!''

白大千提前经过了一场风浪,所以倒比李光明镇定许多:``我知道,昨夜我已经见过一次了。''

李光明问白大千:``大师,现在我们该怎么办?''

白大千长叹一声:``唉\ldots{}\ldots{}我想回家。''

为了防止再次坠入幻境,四个人盘腿坐上了床,分别面朝东南西北四个方向。史高飞正对着电视机——墙壁上的血字无端出现又无端消失了,他解释不了,也懒得深想。

紧挨着他的无心,无心摸着下巴若有所思,不知道如何出手才既能助人,又不暴露自己的秘密。

无心的另一边坐着白大千。白大千把他口袋里网购的纸符以及图解易经全摆到了床上,想要找到速成的驱鬼大法。而李光明一言不发,只匀速的打着哆嗦。

如此过了良久,史高飞忽然大叫:``妈的!又在我的墙壁上乱涂乱画!''

无心回头狠捶了他一拳。他后背一痛,定神再瞧,墙壁上的血字果然不见了。十分得意的笑出了声,他扯着大嗓门叫道:``谁要再见了鬼就出声,我儿子会把人打清醒!''

此言一出,李光明颤声哼道:``蛇来了!''

无心回身对他也一拳。李光明被他打得向前一栽,眼前的蛇随之暂时消失了。

白大千察觉到了问题,转过脸问无心:``你没有幻觉?''

无心一脸茫然的摇头:``我\ldots{}\ldots{}我什么都没看到。''

然后他又用胳膊肘一杵白大千的肋下,抬手捂了嘴嘁嘁喳喳的耳语:``白叔叔,活不好干,你该让黄经理加钱了。''

白大千惊讶的压低了声音:``你认为我还有实力把活干完吗?实不相瞒,我现在看你只女鬼的造型,只因为不想挨打才一直没出声。''

无心没打他,只在他的胳膊上狠狠掐了一把:``来都来了,空手回去多不合算?''

白大千来了精神:``你有办法?''

无心对他伸出一只手掌:``我不会被鬼迷,肯定比你们有优势。我可以帮你的忙,但赚了钱我们得五五分。''

白大千揉着胳膊,心中算了笔账,末了认为五五分也行,五五分总好过一分没有。

一夜过后,天光大亮。史高飞等人东倒西歪睡成一团,只有无心依然坐着。

无心知道史高飞真没用,而自己既然做了他的儿子,少不得就要为他负起责任。活了无数年,好像还真没给谁当过儿子,无心发现父爱也挺动人,如果能够平平安安的当上几十年儿子,倒不失为一段幸福生活。

然后他又想起了史丹凤。史丹凤一身熟透了的女人味,从波浪卷发到高跟凉鞋,无一不美。无心抱着肩膀笑了一下,忽然浑身骨头做痒,想跑到史丹凤面前讨好卖乖的贱一贱。可惜在史丹凤面前又暴露的太早了,给了她一个奇丑无比的第一印象。无心感觉史丹凤好像一直不愿意搭理自己,虽然还给自己买过好几条粗糙的便宜内裤。

坐在清晨第一束朝阳光芒之中,无心盘算着自己的小心事,想得津津有味。还人间好,他想,在山上隐居了四十年,他几乎活成了一块石头。好些惊心动魄荡气回肠的往事,都被他一点一点的忘怀了。

他的历史无人记诵的,忘了,就没有了。即便有过,也像没有过一样了。

白大千本来没把无心当人看,不过不由自主的,他还听从了无心的建议,跑去找了黄经理抬价钱。他别的本事没有,唬人的派头却天生就的,死的能说成活的,有的能说成没的,何况这回不信口胡说,他当真受了大惊。

黄经理决定同意他的要求。并且两人正式签了合同。而在白大千运动三寸不烂之舌时,无心带着史高飞在楼内散步,史高飞不知道自己为什么要楼上楼下的走来走去,不过儿子要走,他身为父亲,自然应该跟着。无心一边照顾着他,一边猎犬似的到处又看又嗅。最后停在了楼下的干涸喷泉前,他抬手拍了拍大莲花瓣。

午饭过后,黄经理带着一队工人赶来了E区。一楼大厅变成了乌烟瘴气的工地,工人各显其能,想要把喷泉拆掉。黄经理站在院子里,问白大千:``大师,问题真出在喷泉下面吗?''

白大千也上午受了无心的撺掇,此刻答得并无底气:``到时便见分晓。黄经理,你刚才说你也不知道喷泉的来历\ldots{}\ldots{}''

黄经理答道:``你们外人不知道,度假村的名字虽然十几年没变,其实里面人事变动很大,都被转卖过好几手了。我去年来的,来的时候E区已经动工了,听说前一任钱不够,把E区盖成了烂尾楼。现在的董事长看楼还不错,所以把工程承包出去,接着先头的基础,把楼盖完了。谁知道白搭钱,盖了一座凶楼。我们用的图纸,都原来的老图纸,格局和喷泉都没变,只把装修的风格换了一下。''

白大千点了点头,然后趁着黄经理不留意,他火速找到无心,把黄经理的话复述了一遍。不知不觉的,他把无心当成主心骨了。

到了傍晚时分,一楼大厅的喷泉已经无影无踪,光亮的地面被凿出了一个黑漆漆的大坑。喷泉下方纵横的水管该截的截该堵的堵,也都收拾利落了。从黄经理到工人一起垂手肃立,对着大坑发呆——修喷泉不复杂工程,尤其装饰性的室内小喷泉,何至于还要在地下挖个无底洞?

黄经理最先清醒了,听到白大千问自己:``黄经理,你原来不知道喷泉下面有洞吗?''

苦思冥想的沉默良久,黄经理最后摇了头:``烂尾楼到我们手里时,喷泉已经修好了。我们当时还以为占了便宜——''

话未说完,他继续摇头,显然知道自己彻底吃了亏。

白大千刚才又受了无心的嘱咐,所以此刻胸有成竹:``黄经理,不要惊慌。事情还有挽回的余地。我想洞内必有玄机,但亲自下洞的话,未免风险太大\ldots{}\ldots{}''

他沉吟不语,同时肋下被无心捅了一下。不动声色的抬眼望向黄经理,他摇了摇头,一脸悲天悯人的庄严相:``黄经理。''

黄经理正在感受洞内喷出的阵阵凉风:``大师,怎么?''

白大千笑而不语。

黄经理明白了,大师又要加价了。

黄白二人忍着一洞的阴风,非常含蓄的讨价还价。末了两人都把话说到山穷水尽,总算勉强达到了双赢的局面。先给工人全下了班,黄经理带着李光明等保安留在了现场,倒要看看白大千的本事。

不料白大千做佛陀状,脱了鞋在楼内尚存的前台桌子上盘腿打坐。他的两个徒弟却一人腰间系了一根绳子,一前一后的下洞了——依着无心的本意,不让史高飞插手。但史高飞哪能允许儿子单身赴险?

洞口小肚大,要说深也不很深,坑底积着浅浅一层水。史高飞落地之后两眼一抹黑,什么也看不见。在他伸手摸索之际,无心已经弯腰开始动了手。手掌没入积水,他开始去刨水下的稀泥。等到史高飞渐渐适应眼前的黑暗了,无心已经水淋淋的抱着一捆东西直起了腰:``爸,上去了!你先上,别碍手碍脚。''

史高飞小声问道:``当着黄经理的面,我可以叫你宝宝吗?''

无心答道:``不可以。等黄经理走了你再叫。''

史高飞失望的``哦''了一声,拽着绳子开始往上爬了。

史高飞一露头,白大千立刻下了桌子。走到坑旁向内一望,他打了个激灵,怀疑自己看到了一条摇头摆尾的大蜥蜴。摘了眼镜揉揉眼睛,他定睛再瞧,蜥蜴没有了,只有无心带着一身淋漓的泥水,抱着一堆白骨爬上了地面。

把骨头往地面一放,无心坐在地上脱了鞋,倒出了两鞋的积水。黄经理大惊失色:``白大师,这怎么回事?这都什么骨头?''

白大千被他问了个哑口无言,只好背着手叹道:``天机不可泄露。趁着太阳没落山,黄经理,请你速速退下,免得惹火烧身!''

黄经理一听这话,吓得屁滚尿流,也不客气了,带着保安即刻撤退。楼内瞬间恢复寂静。白大千站在夕阳余晖之中,怯怯的把目光投向地上的骨头。骨头长而笔直,类似动物的腿骨,然而表面凹凸不平,仿佛浅浅的刻了纹路。

\chapter{骨神}

无心盘腿坐在大坑旁边,手里拿着一根骨头翻来覆去的研究。史高飞和白大千分别蹲在他的左右,直着眼睛看热闹。看着看着,白大千开了口:``上面刻的不甲骨文吧?''

无心摇了摇头——骨头上面刻的应该某种咒文,不过他没见过,也读不懂。

史高飞也出了声:``宝宝,你饿不饿?爸爸给你拿酸奶喝呀?''

无心继续摇头,心想此刻若有白琉璃在场就好了。白琉璃藏着一肚皮稀奇古怪的知识,正常人懂的道理他全不懂,正常人闻所未闻的谜题,他则基本全能解答。把骨头和自己的大腿比了比,他确定了骨头的来历——全人的大腿骨。

大腿骨虽然粗壮,但经了雕刻,自然结实得有限。无心看了看骨头的颜色,又掂了掂骨头的分量,开口对白大千说道:``我看不明白,你能看明白吗?''

白大千把两只手缩进了袖口,一个脑袋摇得左右乱晃。

灰蒙蒙的玻璃门外,天色越来越快的黯淡了。太阳将要下山,大厅内的黑洞中,则喷涌出了越来越强的阴气。史高飞和白大千一起打了个寒战,随即听无心说道:``你们把眼睛闭上。天快黑了,鬼也快要出来了,闭上眼睛才不会被它们迷住。如果听到了怪声音,也千万不要睁眼。我不怕鬼,我会保护你们的。''

白大千立刻就把眼睛闭紧了,还用两张面巾纸缠住了眼镜片。史高飞却不以为然:``宝宝,要保护也爸爸保护你。爸爸也不怕鬼。''

无心无暇理睬他,继续摆弄手里的骨头。因为实在找不出头绪,所以他攥住骨头一端,随手向地面敲了一下。只听轻微的一声脆响,无心大吃一惊,万没想到骨头竟然应声断为两截。一段中空的骨棒落在无心面前,骨中迅速逸出几团惨绿鲜红的光芒。无心仰起头,就见光团在半空中聚为一体,竟迅速融合成了一个顶天立地的巨大鬼魂。

鬼魂渐渐显出人形,朦朦胧胧的一身赤金颜色,形象类似一位练过健美的金身罗汉,其中头脸眉目最为清晰,从腰部往下开始渐渐模糊,及至过了大腿,干脆模糊成了一抹红绿交错的绚烂光芒。一双眼睛骨碌碌的左转右转,此鬼先看了看白大千,又看了看史高飞。白大千双目紧闭,史高飞还在对着儿子絮絮叨叨,于金色鬼魂最终把目光射向了无心。

无心顾不得旁人了,忍不住开口问道:``你\ldots{}\ldots{}一直被封在骨头里的?''

金色鬼魂生前想必个威风的相貌,大眼睛高鼻梁厚嘴唇。听了无心的问题,他张开大嘴哈哈大笑:``没错,你给了我自由。所以我要报答你。''

无心又问:``你\ldots{}\ldots{}''

金色鬼魂傲然答道:``你可以叫我骨神。''

无心站起了身:``骨神,你打算怎么报答我?''

骨神居高临下的笑道:``我会满足你一切愿望。''

无心不出声了,同时一脸狐疑的拧起了两道眉毛,心想你以为我没读过《天方夜谭》吗?

正当此时,骨神忽然对他做了鬼脸,随即高声笑道:``哈哈哈,骗你的!三个小东西,让我吃掉你们的魂魄吧!''

未等他话音落下,无心一弯腰抱起了散落在地的完整骨头,紧接着一手扯起史高飞,大声叫道:``爸爸,拉住白叔叔!''

史高飞猝不及防的被他拽起了身。不假思索的握住了白大千的手,他一边跟着无心往玻璃门的方向跑,一边吼着问道:``宝宝,你刚才在和谁说话?''

白大千晕头转向的调动了双腿:``还用问?你儿子肯定也被鬼迷了!''

眼看就要到达大玻璃门了,无心忽然刹住脚步,一个急转身逃向了通往楼上的主楼梯。他身后的尾巴也跟着东倒西歪的做了大转弯。在绕过大厅中央的深坑之时,无心一边狂奔一边斜眼一瞥,只见骨神的双腿也在缓缓的现形。而坑底发出了此起彼伏的惨呼,一双双的鬼手向上扒住了坑沿,可见正有冤魂在争先恐后的往地上爬。

无心加快了速度,抓着史高飞发疯的跑。在一行三人全部踏上楼梯之时,后方忽然起了一声巨响,竟大玻璃门无端的爆炸,尖锐的大块碎玻璃四处飞溅,白大千只觉脊背一痛,正被一块碎玻璃浅浅的划破了衣服和皮肉。史高飞也意识到了危险,迈开两条长腿拼命往上蹿:``霸天虎来了吗?''

白大千下意识的睁开了眼睛,可眼镜片被面巾纸缠住了,导致他眼前一片白茫茫:``霸天虎个屁!你看我们谁像汽车人?''

史高飞随着无心拐了弯,三步两步的跨上了二楼,同时还有闲心气喘吁吁的嚷道:``其实我比较喜欢狂派!宝宝,你喜欢哪一派?''

无心一手抱着骨头,一手死死的抓着史高飞往上跑:``我喜欢香芋派。''

白大千的胳膊快被史高飞拽脱臼了。直着眼睛张着大嘴,他喘成了一只蹦蹦跳跳的风箱:``我们要、要往哪里去?''

无心首当其冲的踏上三楼地面:``回房间!''

白大千一侧眼镜片上的面巾纸随风飘落,一只眼前骤然清晰,他和一名现了形的女鬼来了个顶头碰。怪叫一声直冲向前,他骤然开始了爆发式的狂飙,以惊人的高速超过史高飞和无心,一马当先的把两个晚辈带进了房内。无心莫名其妙的殿了后,``咣''的一声用后背顶住房门,他急急忙忙的做出指挥:``爸,快搬电视机堵门!''

客房虽新,电视机却四四方方的老款式,而且还大屏幕,十分沉重。史高飞毕竟年轻,能有余力继续干活。白大千坐在床上喘了几口粗气,挣扎着问道:``堵门干什么?鬼还走门啊?''

未等无心回应,摆在电视柜上的遥控器忽然凌空飞起,狠狠拍向了白大千的面门。白大千当即抬手把遥控器抓了个正着,一张白脸开始泛青:``怎么回事?你们看见没有?遥控器自己会飞了!''

无心先帮着史高飞把电视机放好了,然后走到立柜前拉开柜门。柜子里藏着个小小的鬼娃,正在试图操纵一副衣架,显然还想继续行凶。

无心没言语,抬手咬破指尖弹出了一滴血。鲜血穿透鬼娃的身体,随即凭空消失。鬼影僵在半空闪闪烁烁。鬼娃做出了一副狰狞面目,作势要去抓咬无心,然而鬼影迅速变得疏淡,最后化为几点魂魄光芒,流星似的散逸了。

为了防止白大千和史高飞被鬼附身,无心用自己的鲜血在两人眉心间分别抹了一指头。两人面面相觑,然后一起问他:``你干什么?''

无心仿佛很惊讶:``辟邪的,你们不知道吗?''

史高飞立刻把手指头塞进嘴里,狠狠的咬了一口,然后连口水带鲜血的在无心额头上也画了一道。又扭头告诉白大千:``下次轮到你咬啦!''

白大千摘下眼镜擦了擦,带着哭腔嘀咕道:``我说回家,你儿子说不回。现在可好,闹鬼闹大发了,想回也回不了了。我要出师未捷身先死了,佳琪可怎么办?白大万不会让她当姑子去吧?''

无心不理他,只让史高飞和他一起坐好了,不许乱动。自己捡起带上楼的一堆骨头,他席地而坐,心想原来骨头之中封着恶灵。一个骨神已经十分难缠,自己无论如何不能给他再添帮手。背对着床上二人垂下头,他把手指上的伤口压上锋利牙齿,忍痛来回锉了几锉。

用力挤了挤加深扩大的伤口,他用鲜血涂抹了骨头表面。鲜血实在充足,通红的渗入了骨头表面的浅浅纹路之中。一根骨头涂完了,他再涂第二根骨头。

五个手指肚很快全有了伤口,他拿起最后一根骨头,刚想要设法再挤一点血,哪知未等他行动,正前方的房门上方忽然探进了一个金光灿烂的大脑袋,正骨神。

骨神望着整齐摆在地上的一排骨头,先一惊,随即对着无心怒吼了一声。无心手中的骨头登时断裂,然而逸出的红绿光团经过了无心伤痕累累的血手,升到半空之时虽然也试图融合,但光团交错,总不能汇成一体。不出片刻的工夫,红绿光芒渐渐消失,光团黯淡成了普通的零碎魂魄,萤火虫似的没了踪影。

骨神怒不可遏的瞪着无心:``巫师小子,你杀了我所有的朋友!''

无心对着他微微一笑,同时抬起了一只手。垂下眼帘望着身前的一排骨头,他忽然大喝一声,运足力量向下拍去。手掌触及之处,骨头竟腐朽极了一般,沿着带血的纹路破碎成了骨渣。无数零散魂魄平地飞升,同时骨渣硌破了无心的掌心皮肉,给他放了淋淋沥沥的一手血。扬起手挥向骨神,无心满拟着一掌把他打成魂飞魄散。不料骨神动作更快,瞬间消失在了半空中。

在他消失的同时,门窗一起嗡嗡的发生了震动。无心起身冲回床前一抖被褥,把自己和史高飞白大千一起盖了住。近在咫尺的爆炸声骤然而起,碎玻璃溅上了柔软的厚棉被,只有露在外面的双脚受了害,但史高飞和无心穿着运动鞋,白大千穿着皮鞋,都有厚度和硬度,倒抵御住了玻璃碴的袭击。

窗子一碎,无心立刻钻出棉被,在满地的碎玻璃前蹲下身,他扬起双手满地乱拍,东一个西一个的留下血手印。史高飞见了,几乎心痛欲死,慌忙冲上去阻拦他:``宝宝,你干什么?''

无心站起了身,心知碎玻璃已经沾染了自己的血,旁的作用不敢说,至少有力量的鬼魂无法操纵碎玻璃来伤人了。两只手滴滴答答的流着鲜血,无心想要趁着血多再画一道符保护史高飞和白大千,可绞尽脑汁的想了又想,他发现自己把画符的学问忘光了。

窗户已经没了玻璃,房门却又发生了状况。一声闷响过后,雪亮的刀刃突破门板,无心用血手在史高飞和白大千的脸上身上又分别乱抹了几把,紧接着跑向窗口,一个箭步飞身而出。

史高飞和白大千望着窗口,一起傻了眼。史高飞先清醒了,踩着一地碎玻璃跑向窗口往外望,又对着夜空纵声喊道:``我儿子超人!''

楼下一片空荡,无心已经不知所踪。楼上的白大千抱住了想要跳楼的史高飞:``你儿子超人,你不超人!史老弟你不能死啊,你死了我怎么办哪?我不死也要吓死了!''

无心带着两手未干的鲜血,悄悄的走大门进入了一楼大厅。

他发现自从自己毁了那一大堆骨头棒子之后,楼内阴气好像渐渐的有所消退。游荡着的鬼魂也像失了元气似的,只飘忽,不再伤人。

他伸了手,一边走一边打鬼。打出身后一溜明明暗暗的小光点。凭着直觉向上找到了五楼,他越走越感觉不对劲。阴森森的鬼气重新变得浓重了,然而一路上却几乎不见完整的鬼魂。心中忽然一凛,他想莫非骨神真的开始吃鬼了?

自己来驱鬼的,骨神却要闹鬼的,双方无论如何都不可能讲和。无心弯腰脱了鞋,穿着袜子贴着墙根快走。走着走着脚心一痛,他低头望去,发现不知哪扇窗户被骨神震碎了,地上散了一大片极其细碎的玻璃渣。

无心本想绕道,然而心思一转,他把牙一咬,俯身竟在玻璃渣中抓了两大把。蹑手蹑脚的继续前行,他最后在通往天台的楼梯拐角处,看到了悬浮在半空中的骨神。

骨神盘腿而坐,双手扶着膝盖,个垂头沉思的模样。无心望着走廊一侧的大玻璃窗,心中十分犯难——一旦骨神发现了自己的行踪,对方非用碎玻璃把自己崩成筛子不可。但自己此刻距离骨神又太远,不个发动攻击的好位置。静静的将骨神又打量了一番,他忽然发现对方的造型很眼熟。

意意思思的咳嗽了一声,他惊动了骨神。骨神抬起头,虽然只鬼魂,然而目光如电,恶狠狠的望向了无心:``怎么?小巫师,还不甘心等着死吗?''

无心不动声色的走向了他:``你生前哪里人?''

骨神看着他脚上被血染红的白袜子,脸上显出一丝狞笑:``怎么,想用你奇怪的血消灭我吗?''

无心摇了摇头,攥着碎玻璃的手垂在身体两边:``不,我看你很像我的一位老朋友\ldots{}\ldots{}''

攥着碎玻璃的手指合拢进了,鲜血顺着他的指缝流淌。两条手臂蓄势待发的运足了力气,他慢慢的把话说到最后:``你认不认识白琉璃?''

骨神一瞪眼睛,脸上露出了惊恐神情。而与此同时,无心已经向前撒出了手中的玻璃渣。细碎的玻璃渣被鲜血浸透了,红色雨点一般穿透了骨神的鬼影。无心满以为骨神在魂飞魄散之前至少嚎一嗓子。不料骨神的金色光芒只剧烈一闪,随即连光芒带鬼影一起消失了。

无心忍着疼痛站在原地,自己咧了咧嘴,感觉骨神应该彻底死了,不过死得未免太安静,辜负了他光芒万丈的形象和山崩地裂的脾气。

无心一瘸一拐的回了房。门板上嵌着一把未砍透的大菜刀,不知哪位厉鬼行凶未遂。满脸血的史高飞和白大千见了满身血的无心,史高飞自不必提,连白大千都瞠目结舌的痛心了。

三人进了隔壁白大千的房间,白大千的房内除了床上没有床单被褥之外,其余一切都齐全。史高飞把无心拦腰抱到了卫生间里。用毛巾蘸了水为他擦拭伤口。白大千战战兢兢的站在门口,探进脑袋问道:``无心,我有点儿糊涂——你刚才干什么去了?我怎么感觉现在好像天下太平了呢?''

无心疼得直发昏,咧着嘴不住的吸冷气。于史高飞一把将白大千搡出去了。

史高飞脱下自己贴身的长袖T恤,又用小刀割布料挑线头,将T恤撕成了绷带,将无心的两只伤手缠成熊掌,一只伤脚也捆成了大粽子。光着膀子穿了外套,他背着无心走入房内。房中的吊灯和电视全开了,白大千惶惶然的仰头望着无心:``无心,你说句话,到底怎么回事了?我们不安全了?''

无心低低的``嗯''了一声。

白大千登时有了喜色:``真的?你把鬼给灭了?''

无心沉沉的又一点头:``嗯。''

然后他枕着史高飞的肩膀,开始装睡。史高飞弯腰把他往上掂了掂,随即对白大千说道:``你不要缠着他说话,他刚才流了好多血。''

白大千立刻闭了嘴,心想自己真走了天大的运,居然真把买卖干成了!不管事实如何,反正成绩归了自己。以后要对史高飞和无心好一点,他想,等到自己把无心的本事学会了,以后大发其财,也去金光寺做一次大施主,让汇丰老秃驴傻眼!

\chapter{大事业}

史高飞不肯放下自己未满周岁的儿子,宁愿背着无心满地走。白大千直挺挺的坐在床边,因为受了大惊吓,所以也半晌不言语。无心垂着长胳膊长腿,在史高飞的背上打了个盹儿。清醒之后他来了精神,感觉自己有必要继续做完善后工作,否则凭着白大千的本领,一问摇头三不知,明天还没法向黄经理交差。

史高飞从随身携带的粉红色小书包里摸出了酸奶和蛋黄派,一口一口的喂儿子吃。他也饿了,但很有父亲的自觉,绝不肯和儿子抢食。等到无心把一杯酸奶喝光了,他才叼着吸管,很用力的又吸了吸杯中残余,吸出了呼噜噜的一片空响,顺带着把白大千震得回了神。

白大千心想自己不能干坐着发傻,起身挪到了无心身边,他压低声音问道:``你\ldots{}\ldots{}你会法术?''

无心含着半块蛋黄派,一双黑眼珠子慢慢的转向了他:``我\ldots{}\ldots{}''

蛋黄派落了肚,他把答案也想清楚了:``我外星人嘛!''

白大千不动声色,顺着他的话头往下问:``可你爸爸也外星人,他怎么什么都不会?''

无心抬起熊掌似的大手蹭了蹭头皮:``他在地球生活得太久了,被你们地球人同化了。''

白大千暗暗的感慨,心想上帝果然很公平。比如自己,一辈子又帅又穷,脑筋也不笨,然而永远不发财。再比如史高飞和无心,脑筋明明百分之百的搭错了线路,可也都各有一点天赐的邪本事——无心会捉鬼,史高飞则不怕鬼,并且给无心当稳了爹。

于他改变作风,单刀直入的问道:``明天黄经理来了,我怎么向他解释?''

无心让史高飞背起自己,然后带着白大千出了门,一边楼上楼下的走,一边如此这般的嘱咐了一通。

一夜无话,到了翌日清晨,彻夜未眠的黄经理带着保安出了场,躲躲闪闪的在E区大门口探头缩脑。白大千洗了个热水澡,又换了一身洁净的新衣,把一张面孔也刮得干干净净。带着两个徒弟走过满地碎玻璃,他款款的出了一楼大厅,一边走一边扬起手,对着黄经理招了招。

黄经理如同见了真仙一般,登时被他的气场给震住了。颠颠的跑到白大千面前,他满含希望的仰望对方:``白大师,我昨夜听见这边好像发生了几次爆炸,您没事吧?''

白大千点头微笑:``谢谢关心,我没什么。''

黄经理看着碎成渣的大玻璃门,一时间来不及心疼,只感觉不可思议。偌大的两块钢化玻璃,居然在一夜之间碎了一地,可见昨夜大师必定大动了干戈。

白大千转身向楼内做了个``请''的手势:``黄经理,邪祟虽然已经除了,但我还有几句话要交待给你。''

黄经理立刻跟上了他。白大千指东点西,滔滔不绝,先把此楼的风水描述的极其凶险,及至吓得黄经理要拆楼了,他才话锋一转,自吹自擂的提出了破解之法。黄经理被他说得晕头转向,一时间也想不得许多,唯唯诺诺的只点头。

如此到了下午时分,白大千志满意得的带着史高飞和无心回了市区。进入市区之后,他直接去了金光寺接女儿,顺路又到一家自助银行查了查账户余额。余额数目本来十分可怜,然而如今再看,数目赫然拖出了一条长尾巴,堪称亘古未有的盛况。白大千瞬间有了自信,一路昂首挺胸的奔了金光寺。

说来也巧,在金光寺的一扇侧门外,他和汇丰大师打了个照面。汇丰大师的俗家姓名白大万,自从皈依佛门之后,已经修行得万念俱灭、四大皆空,唯独不能见白大千。一旦见了白大千,他必定大犯嗔戒。

此刻一僧一俗四目相对,当场互相嗤之以鼻。白大千一转念,心想自己如今已经有资产的人了,应该表现出一点气度,于耐着性子主动开口:``出门啊?''

汇丰从鼻孔里``哼''了一声。

白大千回头看了一眼,又道:``换车了?你原来不坐奔驰吗?''

汇丰忍无可忍的迈步向前走去:``奔驰?哼,我丢不起那人!''

然后他一弯腰钻入车内,坐着他的新宾利走了。

白大千的热血登时有所降温,发现自己和汇丰比阔,简直自寻死路。

白大千带着佳琪往家走时,天上飘起了半大不小的秋雨。白大千一路走得心事重重,想要把史高飞和无心利用住了,自己也好做出一番大事业。在风情老街的街口买了一些卤菜和热烧饼,他一手撑着一把雨伞,一手领着女儿。佳琪提着一塑料袋烧饼,且走且问:``爸,对门的哥哥也回来了吗?''

白大千光顾着思索心事了,没听见女儿的问话。

胡同最怕下雨,小雨稍微下久了,胡同里面就能积出一条细长的泥水河。白氏父女踩着水中的碎砖,一路险伶伶的跳跃腾挪,好容易才进了家门。把佳琪和食物一起送入房内,他独自去找了史高飞和无心。

史高飞和无心躺在床上,正在看一只摆在桌子上的旧电视机。电视机史高飞下午在旧货市场中买的,除了岁数大长得丑之外,再无缺点。忽见白大千进了门,未等史高飞开口,无心先出了声:``白叔叔,钱到了吗?''

白大千没找到坐的地方,自己原地转了个圈,末了在床边挤着放了屁股:``到了。黄经理又不傻,赖谁的账也不能赖我的。万一我真有本事的,他惹了我,我不报仇?''

无心伸出了一只手,手上的绷带已经除了,手指手掌的创口也已经愈合成了深深浅浅的粉色印记:``五五分,给我一半。''

白大千把他的手往下一摁,然后郑重其事的说道:``我想和你们商量一件事情,关乎我们的前途命运,你们要不要听?''

史高飞忽然大叫一声:``白大千你别摁我儿子的手,他的手压到我的蛋了!''

白大千一抬手:``唉呀我的史老弟,满床都你的腿,你能不能好好躺别劈叉?你看你像把大剪刀似的——你把腿合上!''

史高飞盯着电视屏幕,真把腿合上了。白大千得了清静,继续对着无心说话:``我想既然我们手里已经有了点钱,不如以它为资本,开一家真正的公司。''

无心来了兴趣:``真正的公司?做什么生意?''

白大千一拍手:``我们能做什么生意?当然还降妖除魔看风水啰!史老弟你把电视的声音调小一点,别影响我和你儿子谈大事,蜡笔小新有什么好看的。''

无心把眼睛瞪得溜圆:``现在\ldots{}\ldots{}捉鬼的都能开公司了?''

白大千抹平了膝盖上的一道皱纹,踌躇满志的望向窗外:``除了降妖除魔看风水之外,我想我们将来还可以开展改名转运以及景观设计等新业务,顺便出售五行八卦福和太极八卦镜等辟邪利器。''

无心``哇''了一声:``你说的好像真的一样。''

白大千扶着史高飞的大腿转向了无心:``什么叫`好像真的'?你当我在和你开玩笑吗?''

无心摇了摇头:``不,我以为你想赖账不给钱。''

白大千拍着史高飞的大腿痛心疾首:``胡说八道,狗眼看人低!''

史高飞``腾''的坐起了身,对着白大千怒道:``你说谁狗?再敢骂我儿子一句,我宰了你!''

白大千审时度势,当即服软:``唔,不骂了。''

史高飞``咣''的一声躺回原位,继续看动画片。

白大千压低声音,继续和无心嘁嘁喳喳。两人商议良久,及至到了入夜时分,他们移师正房,继续长谈,直到午夜方罢。

翌日中午,佳琪煮了一大锅大米粥,盛了半锅给史高飞吃。史高飞和无心正对着一口小钢锅吸吸溜溜的喝粥,白大千忽然推门走进来了,手里拿着一张写满铅笔字的信纸。围着史高飞和无心走了一圈,他见二人猪吃食似的一味喝粥,头都不抬,便高声喝道:``停一停,住嘴!''

两人果然抬了头,汗涔涔的抬头看他。

白大千抖了抖手中的信纸,然后恢复了正常音量,笑嘻嘻的说道:``我给公司拟了几个名字,你们听听哪个好。第一个`大千世界易经研究中心',第二个`大千国际周易风水研究院',第三个`大千风水命理预测馆',第四个`大千国学研究室',第五个`大千文化企业管理咨询有限公司',第六个`大千国际易经研究协会',第七个\ldots{}\ldots{}''

等他念到第十个名字,两名听众一致表示``哪个都行'',然后开始低头继续喝大米粥。

白大千很寂寞,感觉两个吃货全不自己的知音,女儿虽然惹人疼爱,然而究其本质,也个女吃货。捏着信纸回了正房,他洗脸梳头换衣服,然后独自出了门。施展轻功穿过遍布泥水的脏胡同,他直奔工商局去了。

下午他离开工商局,开始四处找房子。不能把公司开在龙潭虎穴似的贫民窟里,他得另找个体面地方。地方若体面了,房租自然一定可观。他掂量着手里的几万块钱,越挑选离市中心越远。几天之后,他终于在城郊的一幢大公寓里找到了心仪之处。

大公寓刚刚竣工不久,原址一片古老的乱坟岗子。公寓楼共有十几层,一到三层写字楼,四层往上才住家。因为周边地区还未开发,所以公寓楼也卖不出高价。白大千深一脚浅一脚的进入楼内,发现外面虽然乌烟瘴气,楼内却窗明几净,装潢也堪称时尚。一层二层已经没有空写字间,于他在管理人员的陪同下上了三楼。三楼的写字间有大有小,最小一间不过六七十平方米。白大千在房内转了一圈,心中又惊又喜,当场签了合同交了定金。

在接下来的一个月里,白大千孤军奋战,奔波于事务所银行以及工商局之间。及至拿到执照之时,已经到了深秋时节。公司共有两名股东,一位白大千,一位史高飞;另有一名普通员工无心。佳琪也不闲着,负责全公司的后勤工作,主要业务蒸大米饭。

白大千选了个阳光明媚的吉日,率领全体人员喜迁新居。小小的写字间被他用屏风分隔成了两间。前面一间对着两扇一尘不染的玻璃门,又摆了一副精致桌椅,算前台,后面一间则白大千的办公室。

一间办公室容不下四个人安身的,所以白大千又在办公室正上方的四楼租了一套房子,房子三室一厅,并且粗略的装修过,只要八百块钱一个月。房子里要什么没什么,他买了三张席梦思床垫,让大家集体打地铺。

史高飞和无心没意见。他们把床垫摆到铺了大块瓷砖的地面上,又将旧电视机放到了一只矮墩墩的木板凳上。还有最后一点家当,两只粉红色的小书包,被无心放在了墙角。

放过了长长的一串鞭炮之后,``大千国际周易风水研究院''正式开业。先前白大千的`易经研究所'坐落在贫民窟里,自然不招人问津;如今虽然改头换面的入驻了写字楼,然而酒香也怕巷子深,不做广告还不行。打开电脑连起网线,白大千重操旧业,将自己的无数马甲全部穿起,日夜出没于各大论坛,对自己的公司正炒反炒混合炒,想要打个不花钱的广告。这天上午,他一时不察穿错马甲,闹精分时被人抓了个现形。正恼羞成怒的要和人对喷之时,佳琪忽然进来了,笑嘻嘻的说:``爸,我想上楼去看电视。''

白大千在百忙之中看了女儿一眼:``电视有什么好看的?爸爸告诉你啊,你大姑娘了,以后对于男孩,能不搭理就不搭理,能少搭理就少搭理。尤其对待小史——他有什么好的?你总看他干什么?''

佳琪微微的红了脸:``我不要去看哥哥,我想看金三顺。''

白大千恨铁不成钢的一指女儿:``看金三顺也不行!以后你跟着爸爸,爸爸不上楼,你也不许上楼!''

佳琪开始左摇右晃:``爸爸呀,金三顺已经演了,我想看电视。''

她并不臃肿的身段,然而动作笨拙,摇晃了个东倒西歪。白大千如今没时间教训女儿,只好放了她上楼。而佳琪得了自由,先下楼去给史高飞买了薯片,又给史高飞的儿子买了油炸臭豆腐。自己叼着一根雪糕,她欢欢喜喜的上楼了。

白佳琪走后,白大千成了办公室中的孤家寡人。关了电脑站起身,他慢悠悠的绕过屏风踱到前台。一手摁在前台桌子上,他忽然发现公司里还缺少了一位前台。

``可以再雇个人,试用期工资八百,转正之后一千二,供吃不供住,应该能招得到。''他沉沉的思索:``要求形象好气质佳,长得丑可不行。不知道史高飞有没有意见,其实小史倒好打发,难缠的他儿子。好在他儿子个黑户,没法出面管理公司。不过话说回来,他儿子到底不真疯?''

白大千走回办公桌前坐下,上网发布了招聘信息。与此同时,楼上三人其乐融融,正在一起看金三顺。在插播广告的间隙中,史高飞扯过了无心的一只脚,扒了袜子给佳琪看:``当时玻璃把他的脚都要扎透了,喏,你看,就从这里扎进去的,扎得那么深。可还不到一个礼拜,就愈合得看不出来了。你能看到疤痕吗?看不到吧?''

佳琪四脚着地的跪在床垫上,低了头仔细瞧:``看不到。''

史高飞很得意:``我们母星上的人,都他这样的。等我以后回家了,我也会变得和他一样。''

佳琪笑了,感觉史高飞说话太玄,但玄得有趣,她喜欢听:``真的?你家在哪里呀?''

史高飞一本正经的告诉他:``我家在天上,很远很远,比赛博坦星球还远。我们星球的人都粉红色的大毛毛虫,不过来到地球之后入乡随俗,就长成人的样子了。''

说到这里,他扭头征求无心的意见:``宝宝,对吧?''

无心正在偷吃史高飞的薯片,听了问话,他不大好意思的收回了脚丫子,又羞答答的点了点头。

佳琪很认真的为史高飞规划人生:``那你还别回去了,毛毛虫不好看。''

史高飞刚要回答,然而电视屏幕吸引了他的注意力:``别说话,广告结束了。''

佳琪和史高飞并肩坐着,直着眼睛看电视。无心蹲到两人身后,低了头咔嚓咔嚓的吃薯片。

\chapter{欢聚一堂}

大千国际周易风水研究院斜对面的大写字间被租出去了,租客一家半大不小的贸易公司,不知道具体经营什么业务的,总之男女职员全都精神利落,面貌十分整齐。空荡荡的三楼立刻有了生机,同时把大千国际周易风水研究院衬托成了一方孤零零的小豆腐块。

贸易公司中的红男绿女们立刻对小豆腐块产生了兴趣,先在经过之时左一眼右一眼的向内张望,及至过了三天五天,在白大千开了玻璃门流通空气之时,开始有活泼的青年和他搭话。可惜白大千忙着上网自炒,无暇交际;史高飞和佳琪在楼上忙着看金三顺,也无暇交际,唯独无心无所事事,个饱食终日的闲人。自作主张的占据了前台的桌椅,他在桌子上摆了一盘子高级薄荷糖。到了午休时分,他倚着门框笑迎八方客,见了谁都打招呼:``吃饭去?吃完了?''

他像姜太公钓鱼似的,希望可以用自己的热情勾引个大姑娘小媳妇,然而不知怎的,大姑娘小媳妇不上他的钩,反倒青年小伙子们时常对他连说带笑,并且把他的薄荷糖全吃光了。

于无心把糖盘子收进了抽屉里——他的笑脸和薄荷糖不喂男人的。

白大千面试了几位刚毕业的女大学生,全不满意,嫌人家长得丑。他自认为满门英俊,连麾下的两个半疯子都一表人才,所以决不能找个歪瓜裂枣装点门面。然而真正的美女又犯不上到他的小公司里低就,所以前台的位置一时无人,便被无心稳稳的坐住了。

隔着一层大玻璃门,无心从早到晚的对着走廊发呆。照理来讲,他如今的生活堪称幸福至极,然而饱暖思淫欲也人之常情,他垂着双手俯身向前,把下巴撂在了桌面上,睁着两只大眼睛定定的往外看。偶尔看到个漂亮的,他来了精神,立刻一挺身,眼珠子追着对方的身影,能够一直斜到眼角里。如此又过了一个礼拜,他有了目标,盯上了公司中的一位卢。

卢正处在青春年华,生得美而多姿,终日花枝招展的在无心眼前往来。无心天天直着眼睛看她,她自然有所知觉。可惜恋爱并非一厢情愿的事情,无心越像个鬼似的天天窥视她,她越感觉无心鬼头鬼脑的不招人爱。如此过了几天,这一日中午公寓停电,外面又个浓浓的阴天。卢独自下楼买了一份有汤有菜的午餐上来,穿过黯淡的长走廊往公司走。走着走着她一抬头,忽见前方靠墙悬浮着一个苍白的人头,正在死不瞑目的盯着自己——脸特别白,眼睛特别黑!

扯着嗓子尖叫一声,卢当场把午餐抛了个天女散花。而前方的脑袋随之一哆嗦,却无心将玻璃门开了一道缝,单单的只伸出了一个脑袋看人。

卢最爱美的,如今被自己的饭菜洒了满头满脸,一时间几乎怔住。随即新仇旧恨涌上心头,她大踏步的走向玻璃门,要去找无心的上司论理一番。无心不明就里,把脑袋向后一缩,心中生了不妙的预感。玻璃门随即又开了,正卢和他擦肩而过,直接去找了白大千。

白大千正在网上和人联络生意,冷不防办公室中来了女客,并且女客还顶着一头两肩的米饭炒菜鸡蛋汤。惶惶然的起了身,他未等开口,已在气势上大大的落了下风。而卢伶牙俐齿的开了腔,声音极其尖锐,语言极其犀利,狠狠的控诉了无心近日来装神弄鬼的恶行。白大千的生意没谈完,又没有勇气把卢推出去,情急之下抄起电话打往楼上,让史高飞下来给他儿子善后。

史高飞到达之时,卢骂过了瘾,已经离去。白大千和无心在办公室内相对而站,见史高飞进了门,白大千沉着脸怒道:``我告诉你,无心学会耍流氓啦!人家小姑娘刚刚骂上了门,羞得我一张老脸都没地方放!真食色性也,我当你们两个只知道吃呢,原来还有更高的追求。该通的人事不通,不该通的人事无师自通。哼!你说你们算什么东西!''

无心无话可说,站在角落里垂着头。而史高飞站在原地,身体一动不动,心中却翻江倒海:儿子学坏了,要不要教训?应该要的。养不教,父之过。自己虽然很爱他,但不能没有原则的溺爱。溺爱等于害。做爸爸的,怎么能害儿子?

思及至此,史高飞把心一横。转向白大千,他先当胸挥出一拳:``老混蛋,你敢骂我儿子流氓!''

白大千猝不及防,当场向后飞出了一米远,后背结结实实的撞上了墙壁。捂着胸口落了地,他万没想到史高飞居然如此不分青红皂白。然而正在他打算推开屏风逃命之时,史高飞转了方向,又气势汹汹的杀向了无心。

无心在他面前做久了宝宝,以至于对他毫无防备之心。此刻见他红着眼睛直奔自己而来,无心还想劝他讲点道理,不要再去追打白大千。哪知史高飞骤然出手,大巴掌一把抓住了他的衣领:``为什么要去骚扰女生?爸爸都二十五岁了,还没有谈过恋爱,你呢?你知不知道你刚满半岁,还个小孩子?''

无心比他矮了大半个头,登时被他拎成了脚尖点地。翻着一双乌溜溜的大黑眼珠子,他缩着肩膀仰视史高飞:``爸\ldots{}\ldots{}你不会连我也要打吧?''

史高飞拽着他一转身,不由分说的把他搡向了白大千的大办公桌:``打你也为了你好!''

无心身不由己,脚不沾地的被他拎着走:``爸,不要啊。我、我\ldots{}\ldots{}我以后再也不骚扰女生了\ldots{}\ldots{}爸,饶命啊\ldots{}\ldots{}''

史高飞不为所动的硬了心肠,非要履行严父的职责。而无心被他摁着趴在了大办公桌上,屁股骤然凉飕飕的见了天日。摇头摆尾的挣扎了一气,他开始向白大千求援:``白叔叔,救命!''

白叔叔捂住心口蹲在角落里,带着哭腔答道:``我自己都要被他打断气了,我还救你?''

无心还要饶舌,然而时不待人。史高飞抡圆了巴掌,已经对着他的屁股下了狠手。一声脆响过后,无心干打雷不下雨的发出一声长嚎,响彻了整层写字楼。

史高飞吓了一跳,以为自己把儿子拍死了。而屏风外的大门口也起了响动,一双高跟鞋由远及近的叩击了地面。脚步声停在咫尺之外,短暂的静默过后,一个女声怯生生的穿透了屏风:``请问\ldots{}\ldots{}贵公司要招聘文员吧?''

此声一出,屏风后的三个人一起愣住了。白大千侧了脸,通过屏风的缝隙向外窥视,只见外面站着一位苗苗条条的佳人,化了妆烫了发拎着包,正经有着十分的姿色。而无心趁机活鱼似的一挣,提着裤子下了桌子。

意意思思的绕过屏风,他和来人打了个照面。双方都一惊,随即他先笑了:``姐?''

史高飞追上了他,闷声闷气的也喊:``姐。''

史丹凤本站得亭亭玉立,冷不防的和弟弟见了面,她不知怎的,脸上登时灰了一层,本披肩的大波浪,也忽然有了点披头散发要发疯的意思。肩膀上的小皮包一下子滑到了腕子上,她知道自己自投罗网,把逍遥快活的好日子生生结束掉了。

一番相认过后,史丹凤被白大千让到了办公室内落座。白大千万没想到史高飞那样的弟弟,竟然上面会有个史丹凤这样的姐姐。一时间淡忘了胸口的创伤,他把自己的杯子用开水烫了烫,然后沏了一杯热茶送到史丹凤面前,又满面春风的搭话道:``史真他的姐姐吗?哈哈,看着不像啊,不像姐姐,像妹妹。''

然后他紧挨着史丹凤坐了:``虽说不应该细问女士的年龄,不过你们年轻小姑娘,想必不会在乎。小史今年二十五,你应该二十六吧?''

史丹凤听了,心中暗喜:``我都三\ldots{}\ldots{}都奔三了。''

白大千朗声大笑:``哈哈哈哈哈,越年纪小的女孩子,越爱把自己往大了说。二十六岁也可以算作奔三嘛。我虽然不年轻了,但也经常爱在外面倚老卖老,装个年过半百的样子。其实,嘻嘻,我刚四十出头。''

话音落下,史高飞身后忽然传出了无心的声音:``白叔叔你不四十七了吗?''

白大千当即恼羞成怒:``放狗屁!''

无心躲在史高飞背后放冷箭:``哦,我还以为你真四十七了呢。''

史丹凤有一肚子的话想问弟弟,可在发问之前,她起身绕到弟弟身后,加意的先将无心审视了一遍。无心穿着一身柔软的衣裤,从头到脚都彻底的人模样了。史丹凤对他看了又看,一颗心始终悬着的——他既然能从一条大毛毛虫长成人,自然也可能从人再长成其它怪物。反正他本身就个超自然的现象,又有谁能预料到他的变化?如果他有朝一日成了怪物,被人抓了或者被人杀了,弟弟必要闹个天翻地覆。史丹凤为了弟弟,不得不对无心特别关注。

无心被她静静的看了良久,几乎害羞。歪着脑袋靠在史高飞的后背上,他忽然抬眼对着史丹凤抿嘴一笑。

史丹凤叹了口气,心想他真和人一模一样。试探着抬手在他脑袋上摸了一把,她坐回原位,抬头对着史高飞说道:``算你有本事,不但没和他一起饿死,好像还都胖了。''

史高飞眼巴巴的望着他姐:``你来抓我回家的吗?姐,我不想回家,我一个人也能照顾他。''

史丹凤摇了摇头:``不回家,你不回,我也不回。''

白大千当场拍板,录用了史丹凤为自己的女秘书。又把自己的房间让给她住,自己夜里暂时在办公室内安身。史丹凤看白大千仪表堂堂,言谈举止之中毫无疯气,便暗暗诧异,不明白他怎么会和自家弟弟合伙做了生意,而且弟弟居然能拿出几万块钱入股。

她强忍着没有明问,随着弟弟上楼谈话。及至进入房内关了门,她坐在白大千的床垫子上,又把高跟鞋也脱了。想当初她打着寻找弟弟的名义出了远门,又有钱又有自由,很过了几天舒服日子。游山玩水的逛到了江口市,她花钱花得心痛,有心找份工作赚一点零花,不料甫一登场面试,便和弟弟会了师。

先前不见弟弟的时候,她活得很清静;如今见了弟弟,她明明知道自己的好日子要结束了,可让她抛了史高飞不管,她又实在不忍心。瞪冤家似的瞪着史高飞,她心中荡漾了一点母爱,同时又颇想掐死他。

无心爬到床垫一端,把白大千的新枕头拍了拍:``姐,躺下休息一会儿吧!''

他不出声倒也罢了,他一出声,史丹凤就忍不住要回头去端详他。端详过后她问史高飞:``无心还在变吗?''

史高飞一脸懵懂的摇头,不知道儿子还能变成什么样。

史丹凤继续发问:``你那公司到底干什么的?''

史高飞终于有话说了,并且说得巨细无遗。史丹凤目瞪口呆的从头听到尾,最后心中暗打鼓:``白大千会不会个骗子,哄得小飞做了他的同伙?''

思及至此,史丹凤认为自己真得留下做女秘书了——独自走,她担心弟弟会受骗子的连累;带着弟弟一起走,她又舍不得弟弟入股的几万块钱。

史丹凤身心俱疲,回头一看枕头摆得端端正正,便下意识的仰卧在了床上。一个哈欠没打完,她忽然发现弟弟和无心分列自己的左右,做瞻仰遗容状,正在一起低头对自己行注目礼。无可奈何的扫视了二人,她愁眉苦脸的问道:``我要休息了,你们到底走还留?''

史高飞想了想,一歪身倒在了床垫上:``现在金三顺已经演完了,我也睡吧。''

史丹凤刚要侧身给他让出地方,哪知未等她动作,无心也挤挤蹭蹭的躺在了她的身边:``姐,爸刚才打我了。''

史丹凤的动作停顿住了,不知道自己应不应该把无心当人看待,如果当人看待,当男孩看待还当男人看待。要男孩的话,自己无须避嫌,可以继续休息;要男人的话,自己再躺下去可就太不像话了。

对于无心,史丹凤始终充满了疑问,可若让她开诚布公的提问,她又不知道从何问起。接着方才的话头,她决定和无心谈一谈:``小飞为什么打你?''

史高飞带着睡意答道:``他不学好。''

无心不理会他,自顾自的把声音压到极低:``姐,你不要走了。''

史丹凤略略的生出了一点兴趣:``为什么不让我走?''

无心嘁嘁喳喳的和她耳语:``我能养家,我会养你和爸爸。我们在一起吧,不要走了,好不好?''

史丹凤笑了:``好,那我就暂时不走。''

无心抽了抽鼻子,嗅到了一点淡淡的脂粉香。女人的颜色气味他生活中的花,无须多,有一株可爱的就好。如果史丹凤肯留,他便可以把努向外界的一双眼珠子收回来了。

当天晚上,无心和史高飞一起出门,跟着史丹凤去了市区内的一家小旅馆里,取了一旅行袋的简单行李。

背着弟弟和无心,史丹凤偷偷的和史一彪通了电话,说自己已经找到了弟弟。弟弟执迷不悟,依然陷在同性恋的漩涡之中不能自拔,并且死活不肯回家。而她作为姐姐,无可奈何之下,只好留在江口市照顾弟弟了。

史一彪肉山一样坐在家中,握着电话左思右想,末了认为与其让儿子回家丢人现眼,不如由着他在江口市疯。自己眼不见心不烦,还能落个清静。

他刚刚定了主意,史丹凤在电话另一端又开了口:``爸,我和小飞都没钱了。''

史一彪不能由着亲生儿女在外饿死,只好出钱。出的钱全被史丹凤转入了自己的秘密账户,史高飞丝毫不知,还大包大揽的告诉她道:``姐,你不要嫌工资低,有我呢,我会给你发奖金的。白大千说了,只要有鬼闹,我们就有钱赚。妈的鬼在哪儿呢?我都等急了。''

无心笑微微的跟着他们走,心里很有底气。现在的世界可真一个有趣的好世界,他决心认认真真的做一番事业,正正经经的做一世人。

三人乘坐通往城郊的公共汽车回了写字楼。走到三楼办公室门前了,无心见办公室内灯光通亮,正有客来访,便拦住了史高飞和史丹凤,直接带他们上了四楼。

再说白大千坐在大办公桌后,又惊又喜的面对着前方两位客人。客人之一黄经理,白大千在网上百般造作,也没能骗来一单生意;不料黄经理骤然出现,却给他带来了一位大客户。

\chapter{重创}

史丹凤在批发市场里精挑细选,给自己置办了一身职业装。职业装被她带回家中又洗又熨又剪线头,末了穿上身一看,倒也颇能对付一阵子了。

批发市场成了她的乐土,因为天气越来越凉了,所以她又给史高飞买了一件略含几根羊毛的羊毛衫。给史高飞买了,自然也得带上无心的一份。好歹他个人样子,史丹凤不好不把他当人看待。回家把羊毛衫给了史高飞,史高飞嗤之以鼻:``姐你又买便宜货,我不穿!''

史丹凤气得开始唠叨,冷不防无心走了过来,:``姐,爸不穿,我穿。''

史丹凤立刻得到了一点安慰。等到无心把自己的一件羊毛衫穿好了,她围着他扯扯领口拽拽袖口,嘴里嘀嘀咕咕:``这不挺好看的吗?''

史高飞大喇喇的答道:``好看什么呀!你把我儿子都打扮成小老头了!''

无心没言语,只对着史丹凤微笑摇头,又无声的做了口型:``好看。''

史丹凤看了他的言谈举止,心中不知怎的,竟然一酸——在个鸡飞狗跳的家庭里长到三十岁,没被人疼爱过,疼爱别人也没得到过回报,未曾想弟弟从土里刨出的小怪物却知道哄人,乖巧得让她百感交集。

史丹凤恨起了史高飞,绵里藏针的甩闲话:``人长得好,穿什么都好。吧无心?''

无心点了点头:``嗯。''

史高飞没能领会到史丹凤的语言锋芒,自顾自的穿上外套:``宝宝,走,白大千要带我们去工地。''

出了写字楼,四面八方全工地,工地之间夹着几条刚刚竣工的水泥路。白大千的新客户位建筑公司老板,老板听黄经理对白大千百般推崇,故而恭而敬之的来请他出山。白大千为了保持神秘形象,所以也无须对方陪同,自己一路溜达着走去了工地。

工地距离写字楼并不远,已经起了几幢高高矮矮的楼,将来会一所大学的分校校园。坟地上建学校,本来最平常不过的事情,十分合理。然而如今出了意外之事,导致施工无法继续进行了。

白大千进入工地范围,先不惊动旁人,只问无心:``你看看这片地方,有没有什么不对劲的?''

无心东张西望,然后答道:``没看出什么不对劲的,只鬼多。''

白大千打了个冷战,不知道他实话实说还开玩笑:``鬼多?''

无心百无聊赖的踢开了脚下一块碎砖:``大路朝天、各走半边。反正它们也不害人,多少和我们有什么关系?''

白大千转而又问史高飞:``史老弟,你看见鬼了吗?''

史高飞一摇头:``看不到,我可能真的退化了。''

白大千停了脚步,伸手一指前方的大坑:``你们看到没有?整座工地一起开的工,别的楼都起了好几层了,只有这一片地方,连地基都打不成,一动土就出事,前些天还死了个小工。''

无心听了,仰头看了看天上的大太阳,随即答道:``现在正中午,阳气太重。我们回去吧,夜里再来一趟——我自己来。''

三人大白天的无计可施,只好班师回了公司。人在归途,白大千起了闲心,笑容可掬的问史高飞:``老弟啊,你姐姐一个人跑到我们这里上班,你姐夫放心吗?''

史高飞莫名其妙的看了他一眼:``我姐没结婚。''

白大千的脸上刮起了春风:``哟,没结婚?怎么没结婚呢?''

史高飞心不在焉的答道:``因为我们镇上的人都说她生的孩子会得精神病,没人肯娶她。''

白大千听得心潮澎湃,心想史丹凤的问题到了自己这里,全都不成问题,自己已经有了佳琪,并不想再要小孩子。可史丹凤没了问题,自己倒又有了问题——双方年龄似乎相差的略大了一点,当然,年龄的差距可以用金钱来弥补,然而自己的事业刚刚处在起步阶段,虽然有心弥补,却又心有余而力不足。

无心瞥了白大千一眼,心中暗暗说道:``别和我抢,你抢也抢不过我。''

白大千回了写字楼,进入办公室时正好和史丹凤打了个照面。史丹凤无所事事,正在读一本过了期的《读者》。见白大千回来了,她起身笑着打了个招呼,叫老板不合适,叫经理也不对头,于她自己忖度着喊了一声:``白大师。''

白大千走得衣角飘飘,眼镜片也光芒闪烁:``我们个小公司,关起门就算一家人,你不要客气,叫我大千就好。''

史丹凤只笑,万万不肯直呼他为大千。随后史高飞晃着大个子进来了,吊儿郎当的喊了她一声:``姐。''

史丹凤看了他的德行,懒得理他。

最后出现的人无心。进门之后他直接走到了史丹凤身边,低头解拉链脱外衣。史丹凤若有所思的望着他,忽然想道:``如果他也我的弟弟就好了,他比小飞通情达理得多。我要有这么个好弟弟,也算我没有白白的当一辈子好姐姐。''

目光追着无心的背影走,史丹凤又想:``土里刨出来的\ldots{}\ldots{}老天保佑,可千万别让他再变化了。''

她正浮想联翩,不料无心忽然回了头。双方毫无预兆的对视了,史丹凤不假思索的问了一句:``冷不冷?''

无心笑着摇头:``不冷。''

然后他搬了一把椅子坐到史丹凤旁边,又拿了史丹凤的旧杂志低头翻看。史丹凤没有撵他的打算,屏风后面的白大千却有点坐不住——无心有前科的,他敢狗胆包天的去骚扰对面公司的卢,孰知不会纠缠本公司的史呢?

隔着一座屏风,白大千开了口:``无心,你下楼去买三杯珍珠奶茶,一杯给我,一杯给丹凤,另一杯送上楼给佳琪,记得我那杯不要珍珠哟。过来,我给你钱。''

屏风后面响起了无心的回答:``我不去。''

白大千碰了个硬钉子,颜面尽失,气得咬牙切齿,又不敢耍出老板的威风,因为无心属于公司的骨干,而且再过几个小时,自己还要派他去夜探工地呢。

无心在史丹凤身边坐了一下午,到了傍晚时分下班了,他还陪着史丹凤去了附近的农贸市场。史丹凤花钱如放血,在市场里来回走了几圈,最后只买了一堆丑陋的苹果。

苹果拿回家里,被史丹凤狠狠的洗了一通。无心一直等着要吃,可直等到天黑要出门了,史丹凤才把苹果彻底洗好。用干净毛巾擦出两个相貌最美的苹果,史丹凤对无心说:``给你,路上吃。''

无心答应一声,把苹果往外套口袋里揣。然后趁着史高飞未察觉,偷偷的出门走了。

夜风很凉,幸而他在山里野人似的熬了四十年,已经熬得寒暑不侵。多少年没穿过毛衣了?他简直想不起。掏出一个苹果啃了一口,他忽然很想念白琉璃。他想告诉白琉璃自己在人间找了个爸爸,还想告诉白琉璃人间有个漂亮芬芳的女人,给自己买了内裤,买了毛衣,还有苹果。她好像总在为自己担着心,偶尔还摸自己的头。为什么要担心?也许因为自己的来历。所以自己很小心的接近着她,要让她知道自己不妖怪,和人一样。

无心在心里默默的和白琉璃说话,说着说着忽然想起了大猫头鹰。心底骤然泛起一股子怒气,他恨不能把大猫头鹰拔毛扒皮,烤了吃掉。

走过一条名为天仙东路的崭新大道,他到达了目的地。广袤的工地并未赶夜工,此刻四面八方基本全熄了灯,夜色一片深深浅浅的黑。一只小鬼围着无心转了一圈,因为力量太弱,所以只转成了一道模模糊糊的鬼影。

无心拿着大半个丑苹果,继续往前方的大坑走。小鬼不转了,开始在他面前张牙舞爪,仿佛要作势阻拦。他只作不见,一边吃苹果一边前进。及至将要靠近大坑之时,小鬼停了动作,开始慢慢的向后退。

因为刚刚动土就出了人命,所以此地暂时停工,挖出的大坑也不算深。无心停在了大坑边沿,发现沿途虽然鬼魂众多,然而鬼魂们的阴气加起来也没有此刻坑中的阴气重。可奇怪的坑里干干净净,并未见到力量异常强大的游魂。沿着大坑边沿兜了圈子,无心白天对此处只远观,看得不清不楚;如今身入其境了,才发现这地基实在打得太匆忙,远的不提,大坑附近便有几座未迁的孤坟。孤坟大概无主的,因为地面遍布了连环陷阱,可见有主的坟都已经早被迁走,所以会留下无数未填的深浅土坑。

高抬腿跨过一片腐朽的棺材板子,无心在一处坑里发现了一只圆圆的红萝卜。迁坟的规矩处处不同,也许此地的习俗就要在旧坟坑里留个萝卜。无心看着萝卜,下意识的想要捡起来吃。然而拿着苹果的手指动了动,他意识到自己已经今非昔比,不能像只野兽似的见什么吃什么了。

一大步跳过坟坑,无心又咬了一口苹果。正打算找块石头坐下来歇一歇时,他眼前一亮,却见到一只鬼魂从暗处仓皇飘出。那鬼魂个老人家的模样,衣着穿戴堪称古色古香,显然已经死得有年头了。这样的老鬼都该有些本领,不知为何会被人撵成兔子。无心正要看个究竟,哪知一堆沙子背后忽然金光一闪,那老鬼嚎了一声,登时消失无踪。

无心含着一口苹果张了嘴,不知道沙堆后方埋伏着何方神圣,对于老鬼居然能够说吃就吃。蹑手蹑脚的向前迈了一步,他发现沙堆后方像着了火似的,金光越来越盛,光芒之中一个大脑袋猛然向上一窜,无心登时傻了眼——骨神!

骨神飘在沙堆之后,也没想到自己刚一亮相就能遇见熟面孔。此刻逃来得及,不过未免偏于丢人现眼。蒙着一层光晕越升越高,他故态重萌的向左一转眼珠,又向右一转眼珠。看清无心单枪匹马了,他居高临下的探了头,声音很柔和的说道:``小巫师,你好呀!''

无心弯腰捡起了一块棱角尖锐的碎石,随时预备着给自己放血。和初次相见时相比,骨神的形象朴素了许多,起码不再朝阳似的发出一身刺眼光芒。一手拿着苹果,一手拿着石头,他颇有底气的问道:``你还没死?''

骨神咧嘴一笑,笑出一口方方正正的大白牙:``我已经死过一次了,你还想让我怎么死?''

无心上下打量着他,有感而发的说道:``你很厉害嘛,不会真和白琉璃有关系吧?''

骨神倏忽间飘到了他的眼前:``你想不想知道我和白琉璃的关系?''

无心连忙点头:``想!''

骨神在半空中盘腿坐好了,然后俯身对他说道:``其实,白琉璃我儿子!''

话音落下,他笑微微的俯视着无心。眼看无心惊讶的睁大眼睛了,他忽然扬起双手狠狠一拍膝盖,同时仰天长笑:``哈哈哈,我骗你的!''

随着他双手的起落,周遭的土木砂石瞬间暴起,如同遭了龙卷风一样急速盘旋飞升,劈头盖脸的尽数砸向了无心。不过一刹那的工夫,骨神消失在半空中,地面则多了一座一人多高的小垃圾山。

天地恢复黑暗寂静,只偶尔有风掠地而过。不知过了多久,垃圾山顶忽然伸出了一只血淋淋的手。手本攥成了拳头,直直的在风中伸了片刻,手指一松,攥在掌中的石头滑落到了垃圾山上。小臂像蛇一样动了动,那只手开始搬运压在山巅的一小块水泥板。及至水泥板被掀开了,无心在一团铁丝之中抬起了头。

一根手指粗的钢条扎进了他的左眼,鲜血顺着他的面颊往下流,一直流到了下巴尖。抬手握住钢条,他用力向外一拽。钢条被他拔出来了,上面穿着他的眼珠。没了眼珠的眼窝空空荡荡,显得眼眶很大很深。无心把钢条横着送到嘴边,用牙齿咬住了自己的眼珠。晃着脑袋一抽钢条,他把自己的眼珠咽进了肚子里。

闭了眼睛歇了歇,他挣扎着继续向外爬。垃圾山随着他的活动渐渐瓦解,他最终蠕动着得了自由,从头到脚已经灰蒙蒙的肮脏成了一色。另一只手里的苹果早没了,手背上的皮也被蹭掉了厚厚一层,露出了几根雪白的掌骨。

艰难的站起了身,无心怒不可遏的睁大了完好的右眼。骨神实在太过分了,他现在简直不知道自己应该如何去见史高飞和史丹凤。

周身的剧痛让他战栗不止,他一边抹着脸上的血,一边随着直觉跑向前方——今夜他饶不了骨神!反正骨神能看见他,他也一样能看见骨神,双方势均力敌,非常适合决一死战。

骨神躲在坑边一座空板房里面,极力想要隐藏自己身上的金色光芒。他没想到无心居然没死——再高明的巫师也**凡胎,他没料到无心会个例外。

外面响着无心的脚步声音,远一阵近一阵的,表明他正在疯跑。骨神上次在度假村里已经伤了元气,如今没有力量痛打落水狗,只能缩成一团小太阳暂避风头。如此避了良久,脚步声音却不知何时消失了。骨神听了又听,始终只能听到风声。忍不住把个脑袋穿墙而出,毕竟耳听为虚,他想要眼见为实。哪知伸头这么一看,他虽然个鬼,竟然也吓了一跳。

他看到无心趴在了坑底正中央,一个脑袋正在往土里钻。长条条的身体如蛇一样盘旋扭曲,末了他竟向地下扎入了一米多深。短暂的停顿过后,他开始缓缓的向上退。最后双膝跪地直起了身,他的头脸全被泥土糊住了,两只手却捧了一只小陶坛。低头和陶坛贴了贴脸,他仿佛怔了一下,随即大头冲下的重新入了土,把陶坛又送回了地下。

最后将出入的孔洞填埋了,他起身爬出大坑,低着头往远走,一边走又一边从口袋里掏出一只苹果,个边走边吃的样子。

骨神缩回了脑袋,认为自己逃过了一劫。

无心没有地方可去。等到吃完了一个苹果之后,他发现自己已经不由自主的走回了写字楼下。城郊荒凉,连路灯都隔三差五才亮一盏。无心的左眼窝还在针扎火燎的疼着,有心找个地方躲几天,可两只脚钉在路面上,他真舍不得离开史高飞。

十分钟后,他上四楼进了家门。屋子里一片漆黑,该睡的都睡了,只有史高飞的房间开着门。史高飞盘腿坐在床垫上,听见门响,连忙伸了腿找拖鞋:``宝宝,你回来了?''

无心钻进了卫生间里:``爸\ldots{}\ldots{}你睡你的,我要上厕所。''

史高飞答应一声,转身爬到床垫子上展开棉被。无心关严了门,打开电灯细看自己——脸真没法看了,左眼的眼皮都被被钢条扎豁了。

他窸窸窣窣的洗漱了一番,然后用贴身的衬衣包住了脑袋,只露出一只右眼。手背的伤一时处理不及了,他索性不管,又把脏衣服全扔进了一只大盆里。

无声无息的回了房,他钻进了被窝里。史高飞朦朦胧胧的看着他,十分好奇:``你怎么了?''

无心背对着他躺下了,不知道明天如何见人,又恨骨神恨得要死:``爸,我受伤了。现在\ldots{}\ldots{}现在看起来不大像地球人,我怕你见了会怕。''

史高飞连忙欠身要去看他:``受了什么伤?让我看看!''

无心犹豫了一下,仰面朝天的抬手解开了头上的衬衫:``爸,你给我一点时间,我还可以再长一只眼珠,长好之后就和原来一样了。''

史高飞看清了无心的空眼窝,登时倒吸一口冷气,眼睛鼻孔和嘴巴一起张大了。

无心面对着他这副见了鬼的神情,心中几乎怕了:``爸,眼珠那么小,长起来很快的,一个礼拜就够了。你别怕我,大不了我这一个礼拜躲起来不见你。还有\ldots{}\ldots{}爸你多给我一点东西吃,我吃得多就会长得快。你别不要我,也别告诉姐,好不好?''

史高飞依旧瞪着他不言语。

无心真恐慌了:``爸,原来我做毛毛虫的时候你都不嫌弃我,现在我只少了一个眼珠而已,总比毛毛虫强吧?我已经在工地找到闹鬼的线索了,我们又可以赚到钱了。我不要钱,我把钱全给你和姐,好不好?''

史高飞气息一颤,终于有了反应——他没头没脑的死死抱住无心,张着大嘴嚎道:``嗷\ldots{}\ldots{}我的宝宝啊\ldots{}\ldots{}哪个狗养的欺负了你\ldots{}\ldots{}爸爸要去杀了他\ldots{}\ldots{}''

无心伸手捂住了他的嘴,急得小声说道:``你别嚷,再嚷全屋子的人都醒了\ldots{}\ldots{}求你别哭了\ldots{}\ldots{}你藏到被窝里哭行不行?

\chapter{潘多拉的罐子}

史丹凤夜里做了个噩梦,梦见无心长了满头满脸的白毛,又变回了一只非人非猴的怪物,并且还满屋里乱窜着咬人。一头冷汗的睁了眼睛,她掀开棉被坐起身,捂着胸口喘了半天的粗气。

佳琪起得早,下楼买了煎饼果子和豆浆给众人做早餐。白大千也上了楼,四个人相聚在空无一物的客厅里,各自坐了个小板凳吃吃喝喝。史丹凤天生一个好坯子,又用有限的几样化妆品将自己美化了一番,引得白大千不住的细端详她。史丹凤倒不把他往眼里放,只环顾了面前三人的面孔之后,开口问史高飞道:``无心呢?''

史高飞本来生着单薄清晰的双眼皮,如今却双目红肿,连双眼皮都肿没了:``他生病了,不想吃饭。''

此言一出,听众集体纳罕,万没想到无心还会有``不想吃饭''的时候。纳罕完毕,白大千开口问道:``不昨夜在外冻着了?''

史高飞垂着眼皮,闷头闷脑的``嗯''了一声。

史丹凤放下手里的豆浆和煎饼果子,进房取了一盒感冒药。回来把感冒药递向了史高飞,她犹犹豫豫的问道:``他\ldots{}\ldots{}能吃药吗?''

史高飞并不接药,只声音很响的吸了吸鼻子,个欲哭无泪的模样。

因为无心始终不肯露面,而且又被史高飞描述成了重病患者,所以白大千的三人小队只好精简成了两人。史高飞夜里已经受了无心的指点,如今鹦鹉学舌一般把无心的话尽数转述给了白大千。白大千心里略略有了数,正巧他的大客户又打来电话,恭而敬之的催促他尽早动手,扭转工地的乾坤。于他勉强摆出大师的派头,带着史高飞出发了。

白大千一去不复返,留下史丹凤一个人看守公司。中午她上了楼,见佳琪蹲在史高飞的房门前,正把耳朵往门板上贴。莫名其妙的站住了,史丹凤小声问道:``佳琪,干什么呢?''

佳琪抬起了头,声音比史丹凤更小:``姐姐,我听电视呢。哥哥说了,宝宝病了,怕吵怕闹,这一个礼拜都不许我进屋看电视了。我说我不吵不闹,我只看电视。哥哥说只看电视也不行。''

史丹凤听得啼笑皆非,凑到门旁侧耳听了听,房内一个男声侃侃而谈,正电视机在播放午间新闻。

抬手敲了敲房门,史丹凤问道:``无心,你饿不饿?中午想吃什么?''

电视机忽然没了声音,无心的回答则半晌过后才传出来的:``我不饿,我不吃。''

史丹凤早上就存了疑心,如今听他依然``不吃'',越发感觉不对劲。用力推了推门,她提高了调门:``无心,你开门,让我看看你。''

房中彻底安静了,直过了好几分钟,无心才又结结巴巴的作了回答:``我\ldots{}\ldots{}我睡了。''

史丹凤常年和家人斗智斗勇,自然有办法对付无心的消极抵抗。先找个借口把佳琪支走了,她在外面虚张声势:``无心,这种破门我一脚就能踹倒。你自己开,还我给你开?''

无心听闻此言,吓得头发都竖起来了,左眼窝里新生的嫩肉随之抽筋似的一蹦一蹦。手足无措的在床垫子上爬了一圈,他无计可施,只得起身走过去将房门打开了一道缝。小心翼翼的露出右眼,他惶恐的望着史丹凤:``姐,我真的要睡了。''

史丹凤笑面虎似的对他嘘寒问暖:``你真没事?''

不等无心回答,她一晃薄薄的肩膀,已然侧身挤入了门缝。无心猝不及防的向后一退,门户算彻底失了守。慌忙用手捂住面孔,他在史丹凤面前低头缩肩的蹲成了一小团。史丹凤见他并没有变成怪物,先松了一口气,顺手关严了房门。

也在无心面前蹲下了,她伸手去拽对方的腕子:``手怎么了?让我看看。''

无心听天由命的松懈了身体。一只手被史丹凤摊平到了她的膝盖上,手背上明显缺了一大块皮肤,然而也并没有结痂。一层薄薄的粉色肉膜覆盖了白色的纤细指骨,薄膜不算平整,依稀可见表面生了几根七长八短的白毛。

史丹凤又去拉他另一只手,拉了一下没拉动,第二下她用了力气,一把扯下了无心挡在眼前的手掌。无心闭了眼睛,低声说道:``姐,我昨夜在外面受了伤,左眼\ldots{}\ldots{}没了。''

史丹凤瞪着眼睛看他:``没了?''

无心立刻补充了一句:``还能再长出来——我不会变残废的。''

史丹凤伸手抬起无心的下巴,让他仰脸面对自己。无心始终闭着眼睛不肯睁,于她用另一只手的手指拨开了无心的左眼眼皮。左眼眼皮凹陷着的,拨开之后空无一物,只在眼窝底部隐约有嫩肉鼓凸。

史丹凤屏住呼吸,半晌过后才松了气又松了手:``你可吓死人了。昨天夜里我就梦见你变成了猴儿,没想到梦的还挺准,你这模样比猴儿也好不到哪里去。你怎么受的伤?自己不小心,还被人打了?你说你要只小猫小狗就好了,偏偏长成了个人模样。既然像个人,就得把你当人看待。唉,你知不知道疼?应该知道吧?厨房有鸡蛋,我给你做碗蛋炒饭?给小飞当姐姐我上辈子做了孽。小飞干什么不好非要刨地?刨出来个什么不好非要刨出来个你?算我胆大,当初没让你活活吓死,现在你又来吓我一跳。你说你到底不被人打了?这地方这么荒凉,小飞就不该让你夜里出门\ldots{}\ldots{}''

史丹凤语无伦次的长篇大论,把话说得东一句西一句,仿佛也要发疯。最后她又问了无心一句:``疼不疼?''

无心点了点头:``疼。''

史丹凤叹了口气,起身之前又在他头上摸了一把:``烦死人了。''

史丹凤心乱如麻的去厨房做蛋炒饭。用电磁炉炒出热腾腾的一大锅,她先给自己和佳琪盛出了两碗,然后把锅端进了无心房中。史丹凤有一手好厨艺,做蛋炒饭时能用一只鸡蛋炒出十只鸡蛋的盛况,看着满锅金黄,其实全假象。然而今天她没有施展厨艺,把鸡蛋老老实实的炒成一大块藏到米饭下,她全给了无心。

无心来得蹊跷,伤得恐怖,让她生出了一种惶惶然的伤悲,仿佛无心随时可能消失,自己对他也``喂一顿少一顿''了。

下午回了公司,史丹凤对着几本过期杂志,独自枯坐到了傍晚时分。上下三层写字楼中的大小公司都下班了,走廊里面空无一人。史丹凤正要锁门上楼吃晚饭,不料未等她起身,白大千却满面红光的回来了。

史丹凤起身向他打了招呼,因见他孤身一人,故而又问:``小飞呢?''

白大千正人逢喜事精神爽,如今又找到了和她独处的机会,越发喜上加喜:``他没和我一路走,下午直接进城去了。哈哈哈,丹凤,今天对我来讲,个大日子啊!''

将手里的一只圆滚滚鼓溜溜的白布口袋放在桌子上,白大千打了个酒嗝,然后意犹未尽的对着史丹凤摆了摆手:``丹凤,你不知道,我今天太帅了。''

史丹凤将他上下审视了一通,倒承认他仪表堂堂,但不知道他做了何等大事,以至于自夸自赞到了这般地步。

白大千装着一肚子暖洋洋的酒肉,一边回忆着晚上的盛宴内容,一边向史丹凤大肆渲染了自己今天的大成绩。原来他上午在工地里装神弄鬼、百般做作,吓得客户与围观民工们一惊一乍。直到表演得差不多了,他在大坑之中停住脚步,猛然伸手一指地面,高声喝道:``给我挖!''

几名民工扛着铁锹当即上前开挖,挖了一米多深时,挖出了个小陶罐。陶罐一看就不古董,分量还挺重。白大千见陶罐的模样和史高飞所说的丝毫不差,立刻仰天长笑:``就这个妖孽在作祟了!''

把陶罐放在一只贴了符的白布口袋里,白大千命人填了深坑,又利用新近学习的知识,当众做了一场法事,震得观众们面面相觑。工地下午开了工,果然一切顺利。客户对白大千崇拜得五体投地,不但奉上丰厚酬金,而且设了丰盛宴席款待大师。于,白大千很意外的扬名立万了。

根据无心的指示,白大千把陶罐带了回来。陶罐带着个盖子,四周不知糊了什么,脏兮兮的很严密。白大千举着罐子摇摇晃晃,感觉里面似乎有水,有心开了封看一看,可无心不在场,他又不大敢动手。

史丹凤感觉白大千说话有点云苫雾罩的意思,不值一听,故而在他换气的间隙之中告辞而走。白大千瞬间成了孤家寡人,颇为扫兴的坐回自己的大办公桌后,他开始饶有兴致的摆弄陶罐。

史高飞不许他打开陶罐的,要问为什么,却也没有明确的原因,只说``无心不让''。虽然大家个有财同发的关系,但白大千藏了心眼,并不十分信任无心。无心,按照老话来讲,可以说生了一双阴阳眼,个能通阴阳的人。对于这种玄之又玄的货色,白大千真探不明看不透。陶罐里的东西,可能好可能坏。但无论好坏,无心总该心里有数的,既然有数,为什么不说?莫非里面藏了宝贝,他想带着疯子独吞不成?

白大千思及至此,骤然醒了酒。侧着脸把耳朵贴上陶罐,他忽然一哆嗦,感觉陶罐里面好像有活物——小小的,软软的,轻轻在搔陶罐的内壁。一下子一下子,声音很软,似有似无。

白大千抬了头,用指甲轻轻去刮罐口的污渍。刮了几下,他心中悚然,暗暗的想:``别急,万一真个邪东西,我可整治不了它。再等等吧,看看无心怎么说。''

白大千上了楼,希望和无心谈谈。然而无心把房门关得死紧,只说自己要病死了,拒绝和他交谈。

白大千感觉他病得太怪,十分狐疑。偏巧史高飞带着一身寒气回来了,双手各拎着一只大塑料袋,里面装的全汉堡。原来他想起无心仿佛很爱吃汉堡,可城郊偏僻,肯德基麦当劳一概没有,于他特地因此进了一趟城。敲开房门之后一闪身,他头也不回的挤进了房内。

白大千冷眼旁观,越看越疑。史高飞的饭量,他知道的。既然史高飞不会对着无心吃独食,那无心这位病人的胃口,未免过于可观了。

如此过了几天,无心依然没有痊愈。史高飞出出入入都像贼一样,若有谁胆敢向他房内张望,他必定怒不可遏的咆哮许久,好像他儿子一不小心就会被人看死。

白大千心事重重的坐在办公室里,从早到晚的对着陶罐发呆。陶罐被他擦干净了,比骨灰罐大,比他的脑袋小,圆溜溜的一身大酱色。白大千几次三番的把耳朵往罐子上贴,越听越感觉里面真有活物。心痒难搔的熬到了第五天,他终于忍无可忍了。

中午时分,白大千决定上楼和无心见一面,开诚布公的解决罐子谜团。在他上楼之时,史丹凤和史高飞正在一起研究无心。三个人站在窗前,史丹凤扒了无心的左眼皮细看。新生的眼珠子黑白分明,湿润润的十分灵动。史高飞坐在窗台上,用四肢把无心缠到了自己身前,又低了头,在他头顶上不住的亲。

史丹凤长长的吁出了一口气,虽然感觉无心的存在个大麻烦,可看他变回了人样,也没来由的感觉出了轻松:``小飞,你要养他就好好的养。以后大半夜的不要放他一个人出去——当然,也不许你一个人出去。''

无心靠在史高飞怀里,对着史丹凤嘻嘻的笑。史丹凤被他笑了个哭笑不得,忍不住又要去摸他的脑袋:``你笑什么?你知不知道你七天吃了多少钱?''

史高飞不以为然的一挥手:``姐,你不要这么吝啬好不好?他怎么说也你的侄子,你怎么说也他的姐姐,能不能别什么事都扯到钱上去?''

史丹凤一扬头:``怎么着?他七天吃了我一个月的工资,我还说不得了?''

史高飞不屑于和他姐一般见识,低声嘀咕道:``恶俗。''

史丹凤听了弟弟对自己的评价,登时起了杀心。然而未等她反唇相讥,白大千上来了。进门之后和无心打了个照面,他见对方三人聊得热火朝天,心中不禁一别扭。皮笑肉不笑的咧了咧嘴,他开口问道:``好了?''

无心笑道:``好了。''

白大千单刀直入的奔了主题:``好了就好,那个罐子一直在我手里,我不知道怎么放置它才合适。既然你已经好了,我们就研究研究怎么处理它吧!''

无心说道:``把它给我,你不用管。''

白大千听了,登时火起:``让我不管可不行,谁知道罐子里藏着什么呢!''

无心不动声色的瞄着他:``里面肯定不什么好东西。''

白大千压了火气,勉强保持了平静:``无论好坏,我作为老板,总该有知情权。现在大天白日的,就算罐子里有鬼怪也不能作祟,你们跟我下楼,我们把罐子打开,好东西我们分了,坏东西我们扔了,无论好坏都别瞒人!''

无心万没想到白大千会闹起脾气。挣开了史高飞的胳膊腿儿,他向前走了一步:``白叔叔,我真不知道罐子里的东西到底什么。但看罐子的样子,它不应该古货,倒像近些年被人埋进地下的。埋他的人总该有个目的,要藏它还要扔它,我说不准。总而言之,罐子阴得很,最好不要碰。''

白大千一摊双手:``好哇,那我们把它扔了吧!''

无心连忙向前又追了一步:``怎么扔?扔到垃圾箱里?我告诉你,那种东西放到哪里都会害人,我得想办法毁了它!''

白大千冷笑一声:``好,别等着了,我们现在就去毁吧!''

白大千把无心和史高飞带进了楼下办公室。把陶罐捧在手里,他上下晃了晃,然后对无心说道:``鬼怎么回事,我现在已经知道了。鬼无形的,看不见摸不着。可罐子里面至少有半罐子水,这就证明里面应该不会有鬼。没鬼我就不怕了。实不相瞒,我已经预备了一瓶强效杀虫剂,没牌子,小作坊里配的,奇毒无比。无论罐子里面有什么活物,都受不住我这一喷。我——''

无心听他还要打开罐子,立刻向他伸出了手:``把它给我!我这就点火烧了它!''

白大千侧身一躲:``不给!''

史高飞见他敢和儿子作对,立刻揎拳掳袖的上前应援。白大千骤然受了围攻,急得左右腾挪。忽然脚下一绊,他大叫着向前仆去。无心和史高飞拽他不及,只见他结结实实的迎面拍在地上,而手里的陶罐``哗啦''一声,也在地上摔成了几片。粘稠腥臭的黑色液体流淌开来,白大千一跃而起,随即对着地上情景傻了眼。

在黑液之中,竟然蜷缩了一个小小的婴儿。

准确的讲,不婴儿,更像胎儿。小小的身体大大的脑袋,一身的皮肤皱巴巴的青灰色。大脑袋上有着模糊的五官雏形,一双眼睛忽然缓缓的睁开了,白大千惊叫一声,发现婴儿的眼珠居然通体腥红,没有白眼仁。

无心第一个有了反应,猛扑上去要抓怪婴。哪知怪婴手脚一动,竟然藉着液体的粘滑向旁一躲,随即四脚着地的冲向了大门口。怪婴的动作快,无心的动作也快,一把抓住了怪婴的一只小脚。小脚滑不留手,无心感觉它将要脱逃了,索性用指甲向下狠狠一掐。只听一声尖利的怪叫,他竟然把怪婴纤细的脚腕掐断了。

无心绕过屏风,发现地面上长长一道黑色污迹直通半开半掩的玻璃门外,幸而史丹凤此刻不在,逃过了一吓。低头再看手中的小脚,他发现除了表面皮肤青白色之外,皮肤下的肌肉骨头,居然全漆黑的。再用手指一搓小脚,他蹭了一手薄薄的油脂。低头嗅了嗅油脂的气味,他抬头变了脸色,对着追赶出来的白大千和史高飞说道:``尸油。''

白大千的脸也青白了,又悔又怕的望着无心,他一时吓得哑口无言。倒史高飞还能出声:``刚才罐子里的东西什么?新物种吗?够臭的啊,看那又小又挫的×样,肯定不我的同胞。宝宝你手脏了,快去洗一洗。记得用香皂哦,不用香皂洗不干净的。''

\chapter{怪婴}

史高飞从前台桌子的抽屉里翻出一块香皂,一心一意的要带无心去洗手。然而无心一闪身溜出办公室,顺着地面的黑迹直接冲向了走廊尽头的公共卫生间。大白天的,走廊两端的大窗户透入日光,把整条走廊照了个通透明亮。地面黑迹越来越淡,最后断断续续的无法辨认。无心四脚着地的跪伏了,探头去嗅黑迹的气味。气味腥臭微咸,停留在空气中长久不散,把无心引到了卫生间里。

卫生间分成男女两部,房门相对而开,因为写字楼内的保洁人员工作不力,所以门口的空气永远是淡臭微臊。无心的追踪受了干扰,站在两扇门间踌躇了一下,他先走入了男洗手间。

虽然已经是深秋时节了,可卫生间还开着窗户。无心手里还攥着怪婴的小脚,此刻低头看了看它,他发现小脚正在变色,从青白变为紫黑。皱巴巴的皮肤却是饱满透亮了,用手指轻轻摁一下,触感是一种浮肿式的柔软。漆黑的液体随着他的挤压从创口涌出,顺着他的指缝流成滴滴答答。

小脚的肿胀越来越明显了,五个小小的趾头分了开,圆圆的脚背也高高隆起。无心一眼不眨的盯着它,只见脚背皮肤的颜色深浅不一,深深浅浅的竟是渲染出了一张人脸。人脸有着巨大的额头和模糊拥挤的五官,正是怪婴的模样。

卫生间的窗户不朝阳,永远带着点阴风惨惨的意思。无心咬破指尖,将一滴血滴上了脚背的人脸。血滴落处,立时蚀出了细细的孔洞。鲜血渗入孔洞,将孔洞蚀得越发深了,而孔洞周围的皮肤一收一缩,隐约的人面随之扭曲变形。忽然向后一回头,无心没有看到什么,只捕捉到了一股阴冷的风。

无心不再找了,罐子里的小东西邪得很,应该不是他想找就能找得到的。握着小脚回了办公室,他对着白大千和史高飞低声说道:``我说我处理不了罐子里的东西,你不信。现在好了,它跑了。''

白大千一手扶着桌子,脸上一点血色都没有了,说起话来也哑了嗓子:``无心,我向你道歉。早知道是这么个后果,我死也不会打罐子的主意。我是一时财迷心窍了\ldots{}\ldots{}不知道是怎么搞的,自从把罐子带回来后,我这心里就一直七上八下的不安定,总忍不住要以小人之心度你们的君子之腹,其实凭着我们同生共死过的交情,我真是不该\ldots{}\ldots{}''

他知道自己是闯大祸了,把一席话说得哆哆嗦嗦,几次三番的要咬舌头,直到无心对他摇了摇头,他才暂时住了口。眼巴巴的望着无心,他揣着一肚子惊恐的疑问,简直不知要从何问起。

无心不许白大千和史高飞碰触地面的罐子碎片和黑色污渍。自己下楼找了个僻静地方,他用一块石头把小脚砸成黏腻的一摊黑糨子,然后划了一根火柴扔向它。火苗立刻在尸油上面生了根,在阳光下燃烧成了一团黄中透绿的光焰。片刻之后,火苗熄灭,地面无灰无烬,只留下了小小一块黑斑。

上楼回了办公室,无心撕了一本旧杂志,把地面擦了个干干净净,又把罐子碎片也全部运出写字楼,在太阳下用火将它烧了一遍。

最后用香皂彻彻底底的洗净了手,他关了玻璃大门,绕到屏风后去见白大千和史高飞。史高飞天真无邪,正坐在白大千的沙发椅上玩电脑游戏。而白大千自知理亏,不敢和无心之父抗衡,乖乖的站在了办公桌旁。

三人会了面,白大千和无心全没了精气神。白大千试试探探的问道:``那玩意\ldots{}\ldots{}不会是藏起来了吧?''

无心靠着窗台半站半坐,垂着头答道:``不藏起来才叫怪了。''

白大千又问:``那它如果再出来的话\ldots{}\ldots{}会不会害人呢?''

无心抬头望向了他:``你看它的德行,会不害人吗?''

白大千腿肚子抽筋,站不住了,全凭一双手撑着桌沿借力:``无心,你说它到底会是个什么东西?属于妖魔鬼怪中的哪一种?''

无心对他一笑:``我不知道。我说过我不知道。你不信,非要把罐子打开看个究竟。现在罐子开了,罐子里的东西也逃了。你现在急了?可惜,我还是不知道。''

白大千听无心阴阳怪气不是个好态度,越发惶恐:``别啊,你得知道,你要是不知道的话,世上就没人知道了。史老弟,你别玩了,你快哄哄你儿子,你儿子闹脾气了。''

史高飞正在对着电脑屏幕入迷,忽然听到无心``闹脾气了'',当即起身揪住了白大千的衣领:``妈的你又惹我儿子生气了?''

白大千一愣,随即吓得四肢瘫软下垂,一双眼睛泪汪汪的:``我已经年过半百了,你还要对我动粗?''

史高飞不为所动,还是捶了他一拳。

当天晚上,白大千带着佳琪进了城,把女儿又安顿进了金光寺。夜里回到写字楼内,他爬楼梯要往四楼走。然而刚刚走到三楼,身后便响起了咚咚咚的一串脚步声。回头向下一望,他看到了一名西装革履的小伙子。小伙子拎着一份盒饭往楼上跑,经过白大千时点头一笑:``白大师刚回来?''

白大千认出他是对门公司里的职员,便也和和气气的做了回应:``今晚加班?''

小伙子匆匆答了一声,迈开大步继续往楼上跑。公司是一间几百平米的大写字间,此刻一眼望过去,正是黑茫茫的一片,只有几名留下加班的同事头顶还亮着灯。放下盒饭扯了一条卫生纸,他在大嚼之前,转身先奔了卫生间。

在小伙子进入卫生间时,白大千也回了家。佳琪幼年曾经遇过一场惨烈车祸,虽然大难不死,然而被大卡车撞得智商与烦恼齐飞,从此永远笑嘻嘻的长不大。白大千知道凭着女儿的头脑,见了鬼都不懂得跑,只有把她远远的送去寺里避难才最保险。

``今晚我住佳琪的房间。''他讪讪的对无心说话:``现在让我一个人睡办公室,我还怪害怕的。''

无心正在吃一条烤鱿鱼,嘴唇淋淋漓漓的蹭着鲜红的辣椒酱,脸却是雪白,看着和鬼也差不多。大眼珠子横了白大千一眼,无心的脑筋转了一转,发现自己和史高飞其实还真离不得这个小心眼的老混混。呲牙咬下一口鱿鱼须,他的脸上露了晴天:``你吃饭了吗?姐晚上煮了一条胖头鱼,肉被爸吃光了,留下一碗汤可以泡饭。''

白大千长吁短叹的去了厨房,吃了剩鱼汤和大米饭。虽然今天闯了大祸,但他半生失败,已是身经百战,所以倒还没有影响胃口。回到佳琪的房间关了门,他先把女儿的凌乱物件收拾整齐了,然后心事重重的上了床。辗转反侧的睁着眼睛,他直到半夜也不能入眠。门外隐隐有了响动,他侧耳一听,怀疑是有人正在房内走动。

客厅里只有几个小板凳,不怕贼人光顾。白大千走兽似的翻下床垫爬到门口,想要偷偷的向外窥视。房门还是开发商留下的伪劣品,门板尺寸不合门框,关严之后下方正有一道缝隙。白大千撅着屁股伸着脖子,用一只眼睛向外看。然而外界迎接他的,却是一只腥红的眼睛。

白大千张了嘴,身体僵硬成了一座四脚兽似的木雕泥塑,喉咙口憋着一声惊吼。而腥红的眼珠子仿佛对着他转了一下,随即光点似的迅速熄灭消失。

白大千慢慢的抱着肩膀坐起了身,``嗷''的一声开始狼嚎,把全屋子的人都震起来了。

听说怪婴来过客厅之后,无心立刻跑去了卫生间和厨房。卫生间和厨房未经装修,通往外界的孔道有好几处。用史高飞的旧衣服死死堵住孔道,他又让白大千拿了几张五行八卦福,用大米粥当浆糊,将其尽数糊在了孔道表面。白大千愁眉苦脸的说道:``无心,不行的,那都是骗人的东西,我自己买的我还不知道吗?''

无心没理他,自顾自的依旧是贴。史丹凤抓了史高飞问道:``怎么?你又刨出怪物来了?''

史高飞伸手一指白大千:``不是我,是他。''

史丹凤抬头看了无心一眼,随即问道:``又刨出来个什么?''

史高飞眉飞色舞的答道:``哎哟,这回不是埋在土里的,是坛子里泡的,一个小婴儿,身体这么小,脑袋那么大,丑死了。和我的宝宝相比,简直就是一坨屎。''

无心忙碌一场,然而除了堵塞怪婴可以出入的通道之外,也再没有别的办法斩草除根。他想趁夜出去转一转,寻找怪婴的行踪,史高飞却又坚决不允。

如此过了一夜,翌日清晨众人起床,各自洗漱。无心跑下楼去买烧饼和豆腐脑,然而刚刚下到三楼便看到了警察的影子。

他不知道自己应不应该躲避警察。犹犹豫豫的走到一楼,他发现写字楼的一楼大门也被警察封锁了。在寒风中买齐了三人份的早餐,他开口去问买烧饼的小贩:``楼里出什么事了?怎么来了那么多警察?''

小贩一脸看好戏的神情,很向往的望着楼门:``听说是夜里出人命了,死了一个。''

无心拎着烧饼往家里走,结果在楼下被史丹凤拦了住。史丹凤一手拽着史高飞,急急的告诉无心:``你别上去,警察把白大千叫去问话了。''

中午时分,整座写字楼恢复了平静,白大千接受了一番问询,问过之后也就罢了。神情不定的坐在办公室里,他压低声音对无心和史高飞说道:``你们听说了吗?死的人是对面公司里的职员。我昨晚上回家时还和他打过招呼。''

史高飞和无心一起点头——他们不但听过,而且已然听过了好几个版本。但是各个版本万变不离其宗,总而言之,死者死在了卫生间里,不但被人咬破动脉吸了血,而且半边面孔也被啃了个稀烂。如今流言四起,凶手的身份横跨妖魔鬼怪四届,已经吓得女职员们白天不敢上厕所了。

白大千的心理压力大了,喃喃的自语:``会不会是那个\ldots{}\ldots{}那个东西在作怪?''

无心一侧身坐上了写字台:``我更想知道是谁把它埋进土里的。''

白大千抬头面对了他:``有道理。罐子不会自己钻进土里,好比度假村里的骨头也不会自己长出花纹。骨头里的鬼很可怕,罐子里的婴儿更可怕,在最开始的时候,是什么人胆子这么大,敢和这些东西打交道呢?''

无心沉默片刻,忽然跳下了地,转身对着白大千说道:``这地方原来是坟地,埋罐子的人,我们一定是找不到了。不过我们应该能找到埋骨头的人。有这种邪本事的人不会多,我们如果能打听到其中的一位,兴许就可以顺藤摸瓜的再找到新线索了。''

白大千深以为然的点了头。放出的怪婴闹出了人命,虽然死的不是他,但他一颗心扑腾扑腾乱跳,颇有一点不肯承认的负罪感。抄起电话联系了黄经理,他匆匆出门,直奔度假村去了。

白大千一走,史高飞就又占据了他的位置,不但打开电脑看动画片,还让无心坐到自己腿上一起看。欢天喜地的看完一集,他高声大嗓的喊道:``姐,我想喝冰镇可乐。''

史丹凤坐在一扇屏风外,并没有伺候他的意愿:``自己买去!''

无心起了身,颠颠的下楼穿过一条街,去写字楼对面的一家小超市里买零食。片刻之后回来了,他不但给史高飞买了可乐,还给史丹凤带了一本杂志。史丹凤很意外的被他``伺候''了,感觉十分不习惯。抬眼将无心审视了一通,她心中暗想:``该给他添置冬衣了,又要花一大笔钱。''

无心绕过屏风,回到了史高飞身边。史高飞抓起他的手翻来覆去的看,像真正的父亲在欣赏自己的小儿子。看够了之后,他张嘴去咬无心的手,嘴张得太大了,一口咬了无心半只手。

无心随他研究自己,魂游天外的思索着如何对付怪婴。而史丹凤侧身透过屏风缝隙,很好奇的也在观察无心——不知怎的,她特别喜欢看无心,喜欢看他像个人似的行动坐卧。凭着直觉,她认定他是个有情的活物,对史高飞有情,对自己,显然也有情。她也想像史高飞一样去摸摸无心,可无心毕竟是个男人样子,自己贸然的动手动脚也不好,所以,算了吧。

白大千傍晚打了电话回来,说是自己受了黄经理的盛情款待,今晚要在度假村过夜了。无心放下电话,立刻让史丹凤下了班。三人上楼草草吃了饭,无心以着去买烤鱿鱼的借口,独自溜出了家门。

现在写字楼里没有职员敢再加班了。贴着墙壁站在黑暗的三楼走廊里,无心闭了眼睛,决定守株待兔。

窗外的夜色渐渐浓重了,半空中忽然响起了一声婴儿的啼哭。无心觅着声音缓缓移动,最后走到了走廊尽头,他停了脚步,只见尽头的大窗台上,赫然躺着那只小小的怪婴。

怪婴像所有婴儿一样张牙舞爪,只是一条短腿少了脚丫,是根光秃秃的小棒槌,棒槌顶端还带着丝丝缕缕的筋肉骨茬。残肢向上一直伸到脸上,怪婴张开嘴巴吮住创口,随即面无表情的扭了头,青白的小脸上鼓凸着两只腥红的眼珠。

娇嫩的啼哭声音又响起来了,怪婴松开了自己的残肢,露出了口中上下两对尖锐的獠牙。无心看了它的牙齿,心中立刻全明白了。

``昨夜是你吸了人血?''无心轻声问道:``是谁把你埋到地下的?''

一步一步逼向怪婴,他的语气十分柔和:``别怕我,我不会再埋你。只要你乖乖的,我会找处深山老林把你放掉。''

怪婴的脸上没有表情,然而啼哭声音依然低低的回响在走廊里。在无心将要动手的一刹那,怪婴忽然凌空向上一窜,瞬间消失在了黑暗之中。

无心捕了个空,同时知道怪婴起了戒心,自己一时半刻是不可能再见到它了。

\chapter{无心与骨神}

白大千在度假村里过了一夜,翌日中午启程渡江,不料刚刚上岸便赶上了初冬第一场雪。雪是大雪,落地即融。天地之间一片茫茫,路面又是水又是冰又是泥,交通一下子就堵塞了。

白大千瑟瑟发抖的在外面奔波了小半天,到达城郊写字楼时,天色已经见黑。逆着下班的人流往楼上走,刚到二楼周围就没了人。袖着双手低着头,他忽然抬起头深吸了一口气,鼻孔之中一阵奇痒。张大嘴巴正是要打喷嚏之时,他偶然向上一抬眼皮,喷嚏立时没了。

他看见了一双腥红的光点。

光点悬于天花板下,借着楼道中黯淡的灯光,他认出了那光点的主人——怪婴!

怪婴的身体贴在天花板上,只将一个脑袋大头冲下的后仰着垂了,一双红眼睛定定的盯着白大千。它眼睛大,白大千的眼睛更大,几乎快要瞪出眼眶。嘴唇颤抖着张了张,他最后只呻吟似的``啊''了一声。

他以为自己今夜是必死无疑了,佳琪唯一的出路也只能是当姑子去了。下腹一松裤裆一热,他叉着双腿站在楼梯上,情不自禁的尿了一泡。天气冷,穿得多,他的内裤,秋裤,毛裤立刻全湿透了。两条腿各自为政的颤抖着,已经快要支撑不起他的身体。

正当此时,怪婴动了。

它的胸腹仿佛带了吸盘,能够稳而迅疾的在天花板上移动。四脚着地的骤然爬到了白大千上方,它忽然抬起两只小手用力一拍天花板,小身体应声而落,直直的掉到了白大千怀里。白大千下意识的一抬双手,正把怪婴托进了自己的臂弯。颈关节吱嘎作响的低了头,他近距离的面对了怪婴。怪婴扳起一条短短的残腿,张大嘴巴吮吸着少了脚丫的光秃脚踝。一双大眼睛正视着白大千,它从喉咙里发出了一阵叽叽咕咕,类似一串僵硬的笑声。

白大千晃了一下,先是放了个响屁,然后身体横着一栽,晕倒了。

午夜时分,白大千悠悠醒转。

身下起伏坚硬,硌得他从头到脚一起疼痛,两条腿也是冰凉的,冷到了彻骨的地步。哼哼唧唧的抬起头,他发现自己正趴在楼梯上。

冷不丁的打了个寒战,他抬手摸了摸自己的前胸后背,没有摸到怪婴。掏出手机看了看时间,他发现此刻已经是十二点多了。连滚带爬的站起身,他单手扶着墙壁,东倒西歪的开始向上疯跑。及至到了四楼回了家,他哆嗦着敲开了史高飞的房门:``完了,完了,我告诉你们,我被那东西盯上了!''

史高飞开了卧室电灯,然后哈欠连天的发出疑问:``啊?''

白大千用力推开了他,直奔房内床垫上的无心。一把掀开无心身上的棉被,他强行把无心拽了起来:``我刚遇见它了,它像蟑螂一样可以到处爬,还掉到我的怀里要我抱。吓死我了,妈的,吓死我了!''

无心穿着史丹凤买给他的老头汗衫和三角裤衩,因为房里暖气不热,所以冻得抱了肩膀:``它没伤害你?''

白大千重新将自己审视了一番,随即惶恐答道:``目前看来好像是没有。它那么小,想必也不会趁机非礼我。''

无心嗤之以鼻:``那你真是想多了。''

白大千无暇和无心斗嘴,忙忙的又问:``我放在办公室里的杀虫剂,你们拿上来了没有?''

这话倒是提醒了无心,盘腿坐直了身体,无心问白大千道:``你大哥不是一位得道高僧吗?他有没有什么辟邪的法器?我们借来抵挡几天也是好的。''

白大千一挥手:``别求他,他属于腐朽落伍学院派,除了念经什么也不会。''

无心抬手敲了敲脑袋,想要捡起自己那点画符施咒的学问,然而绞尽脑汁,硬是回忆不起来。怪婴其实已经不能算是鬼魅一类了,倒像是被巫师炮制成的妖魔一流。对待妖魔应该怎么办?他搜索枯肠想了又想,末了感觉自己在过去的四十年里活成白痴了。

他一时没了办法,只好转移话题:``你在度假村里都打听到了什么消息?''

白大千一屁股坐在了地上:``黄经理告诉我,说上一任董事长是个南洋华侨。岁数不小了,想要回国投资发大财,可惜经营不善,大财没发成,最后只好撤资走了。''

无心没想到他如此言简意赅:``就这些内容?没了?''

白大千连连摇头:``没了,黄经理只知道这么多。据说那华侨就是个挺普通的老头,还总往南洋跑,一年在中国住不了几个月。''

无心听了,感觉白大千是白跑了一趟。忽然皱起眉头抽了抽鼻子,他满含疑惑的上下审视了对方:``你怎么这么臭?''

白大千听闻此言,当即起身逃跑,在床垫旁的地面上留下了一个潮湿的屁股印记。

史高飞关了房门和电灯,睡眼惺忪的钻进被窝继续睡。无心也躺回了原位,但是没有睡,手指一直在枕边画来画去。末了一掀棉被又坐起来了,他小声对史高飞说道:``爸,我下楼去公司拿杀虫剂。''

史高飞打着小呼噜,根本没听见。这倒是正合了无心的心意。穿好衣裤推了门,他无声无息的溜入走廊,蹑手蹑脚的直奔三楼。

在距离三楼还有几米远处,通过两扇玻璃门,无心看到了隐隐的金光。停在原地愣了一下,他随即反应过来——形象如此光辉的鬼魂,他只见过骨神一位!

他不声张,一边掏钥匙一边继续前进。及至走到门外了,他骤然出手打开暗锁,一阵风似的冲进了办公室内,正和骨神打了个照面。骨神悬浮在墙角落里,仿佛先前正在研究身下的一架大地球仪。地球仪乍一看像是铜制的,其实是座以假乱真的塑料品。猛的抬头面对了无心,他一挑眉毛,周身全是七长八短的光焰。

无心立刻把手指头塞进嘴里去了,呜呜噜噜的问道:``你来干什么?''

骨神的眼珠子骨碌碌乱转:``我一直很惦念你,想来看你的伤好了没有。''

无心暂时吐出了手指头:``你当初伤我太重,现在我阳寿无多,已经快死了。''

骨神把两道浓眉全扬起来了:``真的吗?''

无心把手指头又含进了嘴里:``嘿嘿嘿,骗你的。我已经去通知了白琉璃,他很快就要过来和你叙旧了。''

骨神的眉毛快要飞出去了:``真的吗?''

无心的舌头和手指直打架:``哼哼哼,骗你的。说实话,你到底来干什么?''

骨神心神不定的向下指了指地球仪:``我想回家去报仇,可是又不认识路\ldots{}\ldots{}''

无心走过去拨了拨大地球仪:``你家在哪里?''

骨神看他手上没有见血,这才放心大胆的告诉他道:``我一直四海为家,不过近几十年一直住在这里——''他用一根灿烂的手指头指向中泰边境:``这里的人把我当成金光佛来崇拜,让我感觉十分温暖幸福。为了回馈他们的好意,我也经常显灵让他们乐一乐。''

无心带着口水的手指穿过了骨神的指尖,向上划出了一道长长的线:``那你为什么会被人埋到了这里?''

骨神的金色面孔登时显出了沮丧神情:``呜,别提了,一个阴险狡诈的老巫师捕获了我和我的手下,还把我封进了一根人骨头里。你知道,像我这样伟大的鬼魂,如果常年生活在阴气太重的地方,力量会越来越强的。''

无心好奇的看着他:``那你住在度假村里不是正好?''

骨神一撅厚嘴唇:``可我是个热爱自由的灵魂。奴隶再强也是奴隶,那老巫师很会折磨我呢!而且骨头里的生活很寂寞,虽然度假村里也经常有些鬼魂慕名前来膜拜我,但是——''

话说到此,骨神扭头望向窗外,语气苍凉的唱了两句闽南语老歌:``心事那没讲出来,有谁人会知。有时阵想要诉出,满腹的悲哀\ldots{}\ldots{}''

无心一句也没听懂,双手合什对着骨神拜了拜,他很怕骨神会抒情不止:``唱得好,我很理解你的心情,真是举头望明月,高处不胜寒。不过我想顺便问一句,度假村里闹鬼的事情,和你有没有关系?照理你当时被人封在了骨头里,应该不能兴风作浪才对。''

骨神一耸肩膀:``和我没有关系。是那些鬼魂感知到了我的存在,不由得有些亢奋。这就是领袖的魅力,我也没有办法。''

无心听他一味的自吹自擂,不禁暗暗的有些鄙视。不过他作为一只鬼魂,本领的确是出乎其萃、拔乎其类,而且闲得要死,看他身上的光明程度,似乎元气也已经恢复了。略略的思忖了一下,无心转而问道:``骨神,我把你从骨头里放了出来,你却恩将仇报,几次三番的想要杀我。你说你对得起我吗?''

骨神很痛快的摇头:``对不起,只是我现在见了巫师就生气,十分想拧断你的细脖子再吃了你的灵魂。听你的意思好像是不打算找我报仇了,怎么?难道你是想让我帮你去杀那个小崽子?''

无心仰头望着骨神:``你全都知道了?''

骨神微微一笑:``我闲来无事,也经常到这座大楼里逛一逛,找个阴气重的地方,悠闲的度过一个下午或者一个晚上。''

无心忽然警惕了:``你一般都在什么地方?''

骨神答道:``女厕所。''

无心点了点头,心中暗骂:``妈的,姐姐肯定被他看光光了。我还没有看过呢,他先看了。''

骨神一边说话,一边用手轻轻去拍膝盖,拍一下,地球仪转动一点。对着地球仪瞧了半天,他皱着眉毛自言自语道:``我还是应该去找一张地图看一看。''

无心用手指摸着地球仪的表面,心里的念头一个接一个的乱转,同时心不在焉的敷衍了一句:``其实也不用地图,你直接往南飘就可以了。''

骨神饶有兴味的问道:``从这里往南飘,第一站会是哪里?''

无心随口答道:``城郊废品收购站。''

骨神的手指头在膝盖上来回敲起了鼓,犹犹豫豫的想要教训无心一下,然而无心忽然倒吸了一口气,随即蜷缩着蹲在了地球仪旁。斜着眼睛望向窗外,他看到了一只倒吊着的脑袋。

骨神也扭头向窗外看了一眼,随即口中``哟''了一声,仿佛是被怪婴贴在玻璃上的面孔吓了一跳。随即转向无心,他满不在乎的开了口:``这东西的怨气好重。''

无心怕被怪婴发现行踪,闭了嘴不肯说话。而怪婴向室内窥视了良久,末了用两只小手拍上玻璃,扬起脑袋向下爬去。

无心闭了眼睛,感觉它真是走远了,才开口去问骨神:``你知道它是什么吗?我认不出。''

骨神也特地思索了片刻,然后才答道:``想要养出这么一个小妖怪,必须先找一个有六七个月身孕的孕妇。这孕妇不能是壮年妇女,要么极老,要么极小,如果是乱伦所怀之子,就更好了。找到孕妇之后,就要剖开她的肚子取出胎儿。如果胎儿见了天日之后死了,还是用不得,非得活的才行。这就很难,也许剖了许多肚皮,也未必能找到一个活胎。''

无心开了口:``你这话我听着很耳熟。接下来是不是要用人血代替母乳,把婴儿喂养到足月?''

骨神点了点头:``是的。''

无心彻底明白了——这种炮制胎儿的方法,还是白琉璃无意中讲给他听的。总而言之,繁冗非常,把一个婴儿改造成非人非鬼的毒妖怪,几乎是件碰运气的事情。而巫师一旦成功,这小妖怪也足以供巫师使用几十年了。

无心又问:``它现在算是死了,还是活着?''

骨神莫测高深的答道:``半死不活。''

然后他告诉无心:``在我的领地里,如果人们捉到了这种东西,一定要先请大法师念三天经,再挑个好时辰在太阳下把它烧成灰。烧过它的地面,几年之内不生寸草。''

无心听到这里,越发感觉事情难办。拢了拢身上的外套,他站起身,对骨神说道:``你和我回家吧,我穿得少,现在好冷。''

骨神没意见,一马当先的往前飘:``我还没见过像你这么娇气的巫师。除了冷,你还怕什么?''

无心翻出杀虫剂,一路轻轻的往外走:``你上次把我打出了后遗症,现在我不仅怕冷,还怕渴怕饿,怕疼怕累。脾气也变坏了,总想放了自己的血和别人同归于尽。''

骨神冷笑一声,心想你还敢恐吓我,可随即回味起对方鲜血的滋味,他又有些毛骨悚然。直接向上穿透楼板到了四楼,他一边高升一边丢下一句话:``小巫师你不要胡说八道了,没有信任的友谊是不会长久的。''

无心冻得脸都青了,一边拎着杀虫剂跑楼梯,一边暗想谁和你是朋友?你若不帮我把怪婴收拾了,我非把你打成魂飞魄散不可。

无心回房之后,先把杀虫剂放好了,然后脱了衣服钻进热被窝。史高飞照例睡成一把大剪刀,两条长腿左一条右一条的叉开来,占据了整张床垫。无心在他身边缩成一团,然后对骨神招了招手。

骨神一歪身,也在半空中摆了个侧卧的姿态,毫无预兆的问道:``白琉璃现在怎么样了?早死了吧?''

无心先是一点头,随后压低声音说道:``他做鬼做了几十年了,住在一片与世隔绝的山林里,和一只妖精一起生活。''

骨神一笑:``哈?是什么妖精,居然愿意和他一起生活?''

无心嘁嘁喳喳的告诉他:``是一只猥琐丑陋龌龊的猫头鹰。''

骨神被他说愣了,想了又想,想象不出猫头鹰精的真面目:``嗯?到底是什么模样的妖精?''

无心一提起猫头鹰,就气得脑筋要短路:``懒得说,反正看着和我差不多。''

骨神笑了:``你太谦逊了。''

无心开始语无伦次的骂街:``谦逊个屁!我要睡了。你不嫌冷就出去找妖怪吧,如果找到了,别忘了拍拍大腿替我弄死它!''

然后他缩进被窝,一头拱到了史高飞的肋下。史高飞在梦里哼了一声,抬手夹住了他。

无心一觉醒来,骨神已经无影无踪。

他把杀虫剂给了史丹凤,想让她以后不要在三楼的公共卫生间里上厕所,可是这话又不好出口。史丹凤始终是不知道他们在闹什么,大冬天的白得了一瓶杀虫剂,也算不得占便宜。

四个人在客厅汇聚一堂,照例是捧着煎饼果子大嚼。客厅里拉了一根绳子,上面晾着白大千的内裤,秋裤,毛裤,外裤,袜子,鞋垫。史丹凤看了白大千的装备,忍不住在吃饱之后,伸手去摸了摸无心的腿。

无心坐在暖气管子旁边,还没有吃完。史丹凤摸过之后问道:``冷不冷?''

无心对她一点头:``冷。''

眼看白大千走去卫生间了,史丹凤低低的对史高飞说道:``你要养他就好好的养,不想养了就早早挖个坑把他埋掉。大冷天的,你忍心让他这么冻着?''

史高飞怔了怔,随即一把将无心扯过来搂进了怀里:``我忘了!姐,听你这么一说,我也感觉好冷哦!''

此言一出,卫生间里忽然响起了白大千的大叫:``啊呀!无心你快来看,你贴的五行八卦福裂开了,是不是那东西夜里又来了?''

未等无心回答,白大千骤然换了话题:``天哪!快来看呀,楼下又来警察了!''

写字楼内的保安们集体提出了辞职,因为一名保安昨天夜里死在了一楼走廊中,死状与三楼公司中的职员是一模一样。

消息并未立刻扩散出去,起码是没有上报纸。白大千吓得抱着脑袋不敢出门,倒是史丹凤跑去看了热闹——尸体已经被抬走了,半条走廊都是干涸的血。

把热闹看完了,史丹凤上楼回了公司。白大千不肯下楼,公司里就再没了别人。她守着电话和杂志,正是百无聊赖之际,玻璃门忽然开了,走进了一名西装革履的男子。史丹凤抬头一瞧,发现来者看着是三十岁上下的年纪,颇有一点自家弟弟的意思,不但有副人高马大的好身材,面貌也堪称端庄英俊。

此人对着史丹凤一笑,开口说了话:``请问,白大师在吗?''

因为美男子当前,所以史丹凤不由自主的要脸红:``白大师\ldots{}\ldots{}我可以马上去联系他。请问您找白大师是有什么事情?''

美男子笑了一下,没有回答。而史丹凤一边说话,一边抄起电话打给了白大千。白大千正在裹着棉被发抖,只说自己身体有恙,连生意都不肯做了。史丹凤无可奈何的挂断电话,还觉得自己挺对不住美男子:``白大师有事外出了,今天可能都不会回来。要不然您——''

未等她把话说完,美男子从怀里摸出了一张名片放到了桌面上:``既然白大师不在,那我就先告辞了。明天我还会再来——或者今晚等白大师回来了,你按照名片上的电话通知我也好。''

史丹凤接了名片,一团和气的目送美男子离去,然后低头一瞧名片,登时哑然失笑,原来美男子姓丁名丁,名叫丁丁。

\chapter{意外来客}

史高飞鸠占鹊巢,霸占了白大千的电脑玩游戏。无心陪着史丹凤坐在外间,两人对着吃话梅。吃着吃着,史丹凤忽然伸手托住无心的后脑勺,用一条香喷喷的小毛巾给他擦了擦嘴。无心的头皮热烘烘的,短发毛茸茸的刺着她的手心,一个脑袋随她摆弄,扭向什么方向是什么方向。史丹凤被他的乖巧激发出了一点母性,几乎想给他花点钱,买点小玩意儿哄他高兴。这种冲动在她的三十年人生之中极为少有,她没有什么依靠,安全感等于零,全凭着手里的私房钱撑着精气神。因为吝啬得太久了,竟然苦中作乐的成了习惯,以至于她素来是连自己都不哄。

``前一阵子不是又挣了钱吗?''她对无心说:``你让小飞去趟银行,取个千儿八百的出来给你买衣服。''

无心被她擦得摇头晃脑:``姐,你要不要添衣服?我让爸多拿些钱。''

史丹凤松手放了他:``别给我买,也别给小飞买。我早给家里打过电话了,妈会把我们的厚衣服邮过来。''

无心歪着脑袋凝视史丹凤,看她的相貌和史高飞是一个模子,也有着清清楚楚的双眼皮,眼尾微微的往上挑,少年时代应该像是狐媚子版林黛玉,然而青春易逝,如今两道眼尾挑不起了,眉宇之间总缭绕着一团百无聊赖的寂寞气。这点寂寞气时常让她显出了一副褪色的旧相,仿佛快要一个人活成老照片似的。

无心望着史丹凤出了神,史丹凤先还不察觉,后来意识到了,就莫名其妙的问他:``看什么呢?''

无心开口答道:``姐,你是个美女。''

史丹凤叹了口气,并无喜意:``这我知道,不用你说。''

正当此时,玻璃门忽然开了。白大千迈步进门,正赶上了无心和史丹凤的最后对话。心里别扭了一下,他没想到无心居然还有成为自己情敌的潜质。而史丹凤站起了身,将一张名片递向了他:``白大师,上午来了一位先生找你,听你不在,给我留了一张名片。''

白大千在楼上活成孤家寡人,故而心惊胆战的下了楼寻找同伴。接过名片看了看,他刚要说话,不想外面又来了一名快递员,送了一大箱子护身符吉祥物。史丹凤和无心有了工作,开始把箱子里的小玩意儿分类放置。而白大千好言好语的哄走了史高飞,坐在写字台后开始望着名片思索——依着他的心思,他真有心把生意停了。然而财路一断,手中的存款又实在是支持不了多久。

下午时分,史丹凤给丁丁打了电话,告诉他白大师已经回了公司。电话放下不久,玻璃门开了,来者正是丁丁先生。

和上午来时一样,丁丁依然是西装革履,进门之后未语先笑,笑出一口将要反光的白牙齿。史高飞和无心正围着前台桌子小声聊天,忽然听得有人来了,史高飞扭头看了一眼,看完之后很笃定的告诉无心:``一只鸭。''

写字楼二楼有一家名气不小的演艺公司,常有俊男美女出出入入。白大千私自给楼下的男女们定了性,认定他们皆是失足青少年。史高飞从来不把白大千的话当话听,唯独这一句记住了,从此见了略有几分姿色的地球人,就必要将其归到鸡鸭一类。

丁丁听了他的评语,似乎是没听明白,还特地回头向后看了看。史丹凤不敢当众教训弟弟,迎着丁丁的笑脸,她以笑还笑,把他请到了屏风后面,等到他在写字台后坐下了,又找出纸杯,给他倒了一杯热水。

然后退到屏风外坐回前台,她听到丁丁用很柔和的声音和白大千寒暄了几句,随即开门见山的问道:``我听说白大师前些天曾经在附近挖出了一只陶罐。''

此言一出,房内立刻静了一瞬。白大千沉吟着微笑:``嗯\ldots{}\ldots{}是的。怎么?丁先生对那个罐子有兴趣?''

丁丁讲一口略带口音的普通话,口音很淡,基本可以忽略不计:``与其说我对罐子有兴趣,不如说我对白大师您更有兴趣。''

白大千莫测高深的微笑看他,心中直打鼓,怀疑对方是同行来砸场子:``噢?对我感兴趣?为什么?''

丁丁微微的一昂头:``因为我没想到白大师的本领如此高明,不但能够找到它,而且敢于挖出它。''

白大千淡淡一笑:``仅此而已吗?''

丁丁答道:``仅此足矣。说老实话,我万没想到这个地方会藏龙卧虎,有您这样道行高深的大家。''

白大千微微颔首,脸上神情不变:``多谢夸奖,但我看丁先生也是有意而来,所以不如省略客套,我们开诚布公的直奔主题好了。''

丁丁一点头:``好,那恕我冒昧,我有一个问题想要请教白大师。''

白大千向他一伸手,是个八风不动的派头:``请。''

丁丁望着他的眼睛开了口:``我想知道那只罐子,现在是否还在白大师的手中。''

白大千一时哑然,不知应该如何回答。正当此时,无心无声无息的走到了白大千身边:``在,怎么样?不在,又怎么样?''

丁丁加意的审视了他,而白大千立刻淡然的介绍道:``他是我的弟子,可以代我说话。''

此言一出,丁丁点了点头,随即答道:``如果在,我愿意出钱把它买下来;如果不在,那请告诉我它为什么不在,是丢了,还是毁了。''

白大千听了一个``钱''字,登时悔恨交加。早知道罐子能卖钱,他又怎会打碎罐子放个妖怪出来?而他身边的无心想了一想,却是答道:``罐子还在,但是我们不打算卖。你既然肯出钱买,想必也是知情人。罐子里面的东西,谁能控制就是谁的。凭着我师父的法力,兴许可以对它试一试。''

丁丁站起了身:``可它是我的!''

办公室内的温度不算高,于是无心把两只手揣进了衣兜里:``不可能,除非你是天赋异禀的神童,或者是保养太好青春永驻。否则凭着你的年纪,你绝对没有制造出那种小妖怪的本事。''

丁丁显然是急了,两道眉毛拧了起来:``它是我家的!''

无心笑了:``你家的?你家里的谁?是谁的让谁来要,总之我不会把罐子直接给你。''

丁丁站起了身:``我付钱!你们开价好了。''

白大千眼含热泪面如死灰,悔得想要撞墙。无心则是满不在乎:``我们不要钱。''

丁丁把牙一咬,方方正正的额头上青筋直蹦。对着无心瞪了半天,他忽然扯着喉咙吼道:``给我!''

屋子里的人全被他低音炮似的华丽嗓音震了一下。无心摇头:``不给。''

丁丁攥了拳头,从头到脚一起使劲,想要旱地拔葱似的抻长了脖子:``给我给我!''

无心斜眼瞄着他:``就不给就不给。''

丁丁原地做了个向左转,战车一样碾向大门,轰隆隆的撞门便走。史高飞和史丹凤并肩而坐,此刻他冷飕飕的发表了评论:``这鸭子脾气好大,疯了吧?''

史丹凤第一次发现无心是个狡猾东西,话里话外带着股子气死人不偿命的阴险劲儿。而屏风后的白大千则是欲哭无泪的扭头仰视了无心:``贤侄,我们这样骗他好吗?为什么不实话实说呢?''

无心俯身去看电脑屏幕,发现桌面被史高飞换成了自己的照片:``我还不是想让他们多来几趟。万一听说罐子碎了妖怪跑了,他们也跟着消失了,我们怎么办?谁来管我们?丑话说在头里,那妖怪爬得太快,我可抓不住。''

无心把话说得斩钉截铁,导致白大千信以为真。然而如此过了一天一夜,丁丁却是再无音信。

写字楼内人心惶惶,因为白大千最近有了名气,所以许多人就近上楼,到他的公司里购买护身符等小物件,想要辟邪。白大千发了一笔小财,然而怏怏不乐。说来也奇,他独自坐在公司里时,一颗心总是冷冰冰的酸楚,时常悲观得想要自杀。然而无心等人一出现,兴许是气氛变得热闹了的缘故,他又立时好转许多,感觉自己还可以对付着活下去。这日到了礼拜六,史丹凤带着无心和史高飞进城购物,白大千想去看看女儿,又懒得动弹。一个人静静的坐在办公室里,他算了算公司的经济账,发现自己正处在一生中最富有的时期,手中的存款虽然买不起宾利,但买辆夏利还是不成问题。

窗外一片阴霾,白大千也是长吁短叹的不快乐。后背忽然一凉,他不由自主的向前一扑,仿佛是被一股力量撞了一下。连忙回头向后瞧,后面除了椅背就是白墙,却又并无异常。

他坐正了身体,掏出手机想要听歌。空荡荡的办公室里忽然弥漫开了音乐声音,他闭上眼睛向后一靠,百无聊赖的跟着唱道:``找个好人就嫁了吧,虽然不是我心里话,纵然情到深处——''

一句歌没唱完,他瘫在沙发椅上的身体骤然一抽搐。四肢猛的僵直伸开了,他神情痛苦的张嘴想要呼喊,可是一口气噎在胸中,他在天旋地转的眩晕和激荡之中狠狠转身撞向墙壁,只听一声闷响,他四仰八叉的滑下了沙发椅。金丝眼镜滑落在地,他的视野由朦胧变成了黑暗。

赶在天黑之前,史丹凤带着两个弟弟回家了。这一趟购物经历堪称惊心动魄,因为史高飞在大街上发表了许多高论,没有一句是正常人能说出来的话。史丹凤作为他的姐姐,虽然已经丢脸丢习惯了,然而江口市毕竟不是火星镇,她在镇里可以破罐子破摔,在江口市繁华的大街上,却是不能不要点脸面。

她费了天大的力气,总算哄住了史高飞的一张嘴。可史高飞别有一种出其不意的捣乱本事,嘴老实了,改为动手,缠着无心又亲又抱,几乎引起了路人的围观。史丹凤撕撕扯扯的想要把无心从他怀里拽出来,正是努力之时,几个大学生模样的男生从一旁经过,其中一人把嘴一撇:``我就看不惯这些腐女!''随即他的同伴表示异议:``不对,是三角恋,那个女的和那个男的抢那个男的。''

史丹凤除了硬着头皮装听不见之外,别无他法。而无心处在史家姐弟二人之间,因为感觉自己很抢手,所以十分得意,嬉皮笑脸的任由他们撕扯自己。

史丹凤一路讨价还价,一路丢人现眼。及至购物完毕之后,她感觉自己以后再也没脸进城了。乘坐一辆黑出租车回了城郊写字楼,她提着大小购物袋率先上了四楼。掏钥匙开了房门之后,她迎面正是见到了客厅中的白大千。

颇为意外的点头一笑,她开口说道:``白大师,吃了吗?小飞和无心买了晚饭,没吃的话正好一起吃。''

白大千背着双手站在客厅中央,鼻梁上架着一副略显歪斜的金丝眼镜。似笑非笑的摇了摇头,他低而迟缓的答道:``不了,我很累,要去休息了。''

说完这话,他转身进了卧室。

史丹凤对他一贯的没兴趣,转而去照顾弟弟和无心。无心的新衣服并没有穿上身,导致他一路把身体冻成了冰凉。进屋之后他饭都不吃,直接脱了衣裤跳上床垫,钻进了从来不叠的乱被窝里。

史高飞开了电视,盘腿坐在床垫一角看动画片。史丹凤换了拖鞋,先是大声催促史高飞去吃晚饭,见史高飞不动,她转而走到无心面前蹲下了,想要把他驱赶起床。然而抓住被子一角一掀,她骤然看到了两条光溜溜的手臂和一双蜷缩起来的光腿赤脚。

``哟。''史丹凤不知道自己该不该不好意思:``脱得这么干净呀?''

无心耸着肩膀抱着小腿,是小小的一团:``被窝里暖和。''

史丹凤拍了拍他的脑袋:``先吃饭,吃饱了再睡。''

无心不情愿的推开棉被,而史丹凤双手撑地站起了身,就感觉后脖颈过凉风,可见房内的暖气真是热度有限,不怪无心要往被窝里钻。

史丹凤意识不到骨神驾到,还打算再去劝史高飞吃饭。无心也不理睬骨神,起身弯腰去找裤子。而骨神不知去哪里逛了一天,如今兴致勃勃的看了看屋中三人,随即抬手一拍膝盖。

``啪''的一声,无心的三角裤衩应声而落,向下一直滑到了脚踝。史丹凤和史高飞若有所感的一起回了头。史丹凤愣怔怔的没看清,史高飞则是捂着心口大叫一声,仿佛受到了极大惊吓。及至叫过了,他吐出一口气,紧绷的面孔却是恢复了先前的松弛。

``看错了。''他拍着胸膛自言自语:``我还以为他又受了伤,肠子流出来了。''

无心手忙脚乱的提起裤衩,又回头去怒视骨神。骨神幸灾乐祸的哈哈大笑,同时双掌合十又是一拍。无心只觉手中一松,裤衩已经再次滑落到了脚踝。

史丹凤终于是看了个清清楚楚——第一次见了活人的真家伙,她``腾''的红了脸。真家伙和她在图片影片里看到的还不大一样,粉嘟嘟的软垂着,随着他下蹲的动作一甩,比她印象中的器官要温柔得多。

心跳如鼓擂的转身出了门,她一边走一边说道:``还闹?再闹饭菜全凉了,没人给你们热。''

无心害了羞,恨不能去打骨神一顿。骨神笑够了,对他抬手一指地面:``去看看你们的老板吧,他的灵魂正在楼下男厕所里游荡。他很笨,飘都不会飘,但是哭得很响,简直吵死了。我保护了他一下午,保护得很烦啊!''

\chapter{亲昵}

无心听了骨神的小报告,脸上不动声色。史丹凤在客厅里连绵的呼唤,力逼着弟弟和弟弟刨出的儿子快来吃饭,声音温柔婉转,带着以柔克刚的劲儿,显然史高飞等人若是不及时的出来填饱肚皮,她便能柔情似水的一直催促唠叨到深夜。

史高飞把饭菜端进卧室里,心不在焉的边看电视边吃。无心穿上裤子,端着饭碗蹲到了他的身后。史丹凤站在一旁,食不甘味的大嚼。慌慌的把一顿晚饭对付过去了,她心乱如麻的回了屋。抖开棉被钻进被窝,她有滋有味的回忆起了无心的光屁股。怎么想怎么觉得有趣,并且认为自己当时毫无准备,以至于虽然看了两次,但还是没有彻底看清楚。

史丹凤回了房,史高飞也长在了电视机前。无心向骨神递了个眼神,然后走去敲响了白大千的房门。骨神笑眯眯的跟在他身后,算是他的临时保镖。

房门一敲便开,
白大千没有更衣,一身齐整的站在了门口。对着无心点头一笑,他没言语。

无心一言不发的一侧身,从他身边挤进了房内。白大千随手关了房门,然后回头问他:``有事?''

无心在屋子里转了一圈,最后转回了白大千面前。不是什么鬼魂都能操纵人身,一个活人的身和心乃是天配的一对,想要拆开了重组,总是比不得原装货。无心迎着白大千木然的眼神和僵硬的表情,感觉他身上虽然破绽不少,但毕竟灵魂和身体只磨合了一个下午,能够默契到这种程度,已经实属难得了。

骨神鬼在门外,只让一个金光灿烂的大脑袋穿墙而入,饶有兴味的窥视房内情形。无心瞥了他一眼,随即对着白大千瞪了眼睛:``我听说你去联系了那个姓丁的!怎么,你见利忘义,想要出卖我们了?''

白大千的脸上没有表情,只在瞳孔之中有光芒流转:``我\ldots{}\ldots{}不明白你的话。''

无心一把揪住了他的衣领,两道眉毛一起拧着立起来了:``我什么都知道了,你他妈的还敢对我装傻?白大千,你听清楚了,我给你一夜的时间,你好好考虑一下。如果还想活命,明天早上就把罐子交给我保管!否则别怪我翻了脸,找出罐子摔个鱼死网破!''

白大千任他揪着自己的衣领,直挺挺的毫不挣扎:``怎么?你真能控制它?''

无心对他咬牙切齿的狞笑:``能不能不是问题,问题是敢不敢!我敢,即便控制不了它,我也有办法消灭它!所以你死了心吧,我是不会允许把它卖掉的!''

松开白大千的衣领,无心又威胁似的指了指他的鼻尖,末了一甩手走了。出门回了自己的卧室,他把房门关了,对尾随而来的骨神吩咐道:``你下楼去,把白大千带上来。''

骨神很勤快的向下一沉,没入地面。同时史高飞回了头:``宝宝,你让爸爸干什么?''

无心走到他身边坐下了:``没什么,我刚才是在对一只鬼说话。''

史高飞安心的转向电视屏幕,不再问了。

片刻之后,骨神携着一团微弱黯淡的灵魂上了楼。灵魂影影绰绰是白大千的形象,骤然见了无心和史高飞,他做了个抬手抹泪的动作,抽抽搭搭的开口问道:``无心,我是不是\ldots{}\ldots{}是不是\ldots{}\ldots{}死了?''

然后他的影子在空中一飘,飘成了头上脚下的姿势。乌龟似的将四肢划动了一气,他没能把自己调转向上,只好倒栽葱的认了命,继续哼哼唧唧的哭诉:``没想到我这样命苦,辛辛苦苦的熬了大半生,刚刚赚到了一点小钱,就莫名其妙的丢了命。我死了,佳琪怎么办?汇丰老秃驴狼心狗肺铁面无情,还不送她当姑子去?呜呜呜,我可怜的丫头啊,再也没人疼没人爱了\ldots{}\ldots{}''

骨神伸出援手,把白大千的脑袋顺时针拨向了上方。无心不等他说完,也低声开了口:``白叔叔,安静,你听我说,你现在只不过是灵魂出窍了而已,你的身体还活着,被一个陌生的鬼魂占据了。他现在可能正躺在你的被窝里睡觉呢!''

白大千抬袖子抹眼睛:``呜!气死我了。''

无心继续说道:``你现在要做的就是找个地方躲起来,千万不要被别的鬼吃掉,也决不能魂飞魄散。一旦魂魄散了,我也救不得你了。''

白大千放下手,抬起了一张模模糊糊的大脸:``什么是魂飞魄散?说老实话我现在感觉很困,只是内心太痛苦,所以睡不着觉。''

无心警告似的向他竖起了一根手指:``千万不要睡,如果你睡了,佳琪就没有爸爸了。现在你可以想一些最能让你愤怒怨恨的事情,怨气重的灵魂总是存在得比较长久。''

白大千很老实的点了头,又嘟嘟囔囔的含泪说道:``好,我这就开始想汇丰。''

骨神把白大千带回了楼下的公共卫生间里。史高飞关了电视睡觉了,无心却是一直睁着眼睛。隔壁卧室里总有窸窸窣窣的响动,可见一墙之隔的假白大千一定没闲着,大概正在翻箱倒柜的寻找那只陶罐。可惜房里空空荡荡,没有足够的箱柜供他研究。忽然听到``吱嘎''一声,是客厅里有房门开了。一串沉重的脚步声移向了卫生间和厨房,无心缩在被窝里,倒要看看这个假白大千能找出什么宝贝。然而等了不过片刻,厨房里忽然有了大动静,仿佛是有人撕裂了一张干脆的厚纸,``嗤啦''一下子,紧接着天花板的一角响起了一阵叽叽咕咕的怪叫,叫声单薄而又低沉,是个诡异的婴儿声音。

无心一掀被子起了身,忍着寒意迈开大步,打开房门溜了出去。一转身站到了厨房门口,他望着眼前情景,不禁发了呆。

他看到了怪婴。

怪婴突破了贴着五行八卦福的排风口,一个青白色的脑袋从墙壁中突兀的探了出来。腥红的大眼睛死盯着假白大千,它的小脸虽然没有表情,但是一腔怒火全从眼中喷射出来了。漆黑的口涎顺着嘴角滴滴答答的流下,它忽然一张嘴,露出了口中上下四枚锐利的尖牙。

假白大千站在厨房中央,显然也是愣住了,甚至没有留意到身后来了无心。于是在被怪婴发现自己之前,无心横着挪了一步,往暗处又躲了躲。可未等他站稳,假白大千忽然向后一仰,在厨房地上摔了个四脚朝天。而一道光芒一闪即逝,正是一只鬼魂冲出了他的躯壳。

无心暗暗算计着时间,倒要看看怪婴是何举动。怪婴的眼睛盯着鬼魂消失的方向一转,紧接着它向下钻出了排风口,大号爬虫一样飞快的蹿到了白大千身边。围着白大千的脑袋转了半圈,它扬起小手一拍对方的胸膛,同时张大了嘴,从喉咙里发出了一声婴儿特有的娇嫩啼哭。

无心本想趁机将它捕获,可没想到它和白大千越来越亲,最后竟是撅着屁股跪伏到了白大千的肚皮上。无心深知它的毒性——哪怕只是它的尖牙碰破了白大千一点油皮,白大千的躯壳便能立刻硬成一具僵尸。

螃蟹似的慢慢移动到了厨房门口,无心忽然不知道怎样对待怪婴才好了。而怪婴本来还在拍打白大千,忽然抬头见了无心,它当即走兽一样向后撤了一步,随即猛的向上一跃贴了墙壁,蚰蜒一样瞬间钻回了排风口。

无心下了楼,在男厕所里找到了怨气冲天的白大千。骨神飘在小便池上方,正在盘着腿似睡非睡。白大千坐在小便池里,独角戏似的讲述白大万如何卑鄙的勾引佳琪她妈。忽见无心来了,他抬头说道:``无心,我现在很生气,也感觉自己很有力量。明天我打算去趟金光寺,好好的吓一吓汇丰老秃驴!''

无心对他招了招手:``别想美事了,除了我之外,根本没人能看到你。你的身体刚被我抢回来了,现在你赶紧跟我回去,希望你还能活过来。''

白大千一听,立刻不骂了,兴奋的抬头恳求骨神:``大神,求您帮忙带我走一程,我陷在小便池里出不来了。''

白大千上了四楼,被骨神摁进了躺在地上的身体之中。灵魂渐渐和躯壳重合为一体,末了地上的白大千眨了眨眼睛,一挺身坐了起来。

抬手捂着后脑勺,他开始感觉身体疼痛虚弱,仿佛大病初愈一般。摘下胸前衣襟上的一缕灰尘,他发现自己身上气味古怪,又咸又腥。

``妈的。''他嘀嘀咕咕的骂道:``好臭啊,莫非我下午是被一条带鱼附体了?''

无心略一犹豫,没有说出怪婴曾在他身上爬了好几个来回。

白大千拧了一把热毛巾给自己擦了擦身,然后自作主张的溜进了史高飞的卧室。在床垫上靠边躺了,他把无心一直挤到了史高飞的怀里。史高飞朦朦胧胧的抬起手,劈头盖脸的摸了无心一把,摸完之后确定了这的确是自己的儿子,便闭了眼睛又睡了。

如此过了一夜,翌日清晨,史丹凤起了床,照例是下楼去买早餐。白大千额外向她提供了住处,她额外付出一点劳动,也属正常,况且在史高飞身边,她向来是偷不到懒的。

从写字楼到早餐摊子,一段路让她走得浮想联翩——做了一夜的梦,梦里无心光着屁股在客厅里跑来跑去,□那条命根子甩过来又甩过去,甩得她眼花缭乱。忽然对方的脸孔变了模样,从无心变成了前些天光临过的丁丁先生。

史丹凤拎着一口袋油饼,在寒风之中走得面红耳赤。真没想到自己会一下子梦到两个光屁股男人,可惜两个脑袋配了一个身体。她想丁丁实在是长得帅,和自己年纪也差不多,不知道结婚了没有,当然,他结不结婚都和自己没关系,自己也只是想想而已。

及至上楼进了门,她眼里有了无心,思绪随之换了内容:``有意思,真和人是一模一样。不知道他懂不懂得恋爱,也不知道将来会不会有女孩子肯接受他。''

思及至此,她忽然生出了一股醋意:``我弟弟把他刨出来的,我弟弟把他养到这么大,凭什么最后要把他给别人?谁要也不给!''

然后她横了无心一眼,想他身上的一丝一缕都是自己亲手买的,心中便有了几分霸王般的得意。仿佛为了彰显主权一般,她故意把正在吃油饼的无心叫到自己面前,抬手给他理了理身上衣服,又想没话找话的训他两句,表明自己是说一不二的老大姐。哪知未等她开口,无心忽然微微俯身,很认真的和她贴了贴脸。

史丹凤登时被他雏鸟一般的举动打动了,一颗心融化成水,软得提不起放不下。真说不清无心是男孩还是男人,似乎男孩的成分占了上风,而且还是个小男孩,无依又无靠,乖得不得了。

史丹凤摸了摸无心的脑袋,嘴里无话可说,心里却是恨不能咬他一口,像新妈妈咬婴儿的小手小脚一样,轻轻的咬一口,让他疼一下,笑一下。

无心察觉到了史丹凤的爱意,心中立刻得寸进尺的有了想法。

白大千的身心受了重创,一整天都是怏怏的没精神,然而让他独自留在卧室休养,他又死活不肯,非要投身于人海中才有安全感。写字楼里是没有人海的,所以他裹着一件旧羽绒服,垂头丧气的还是坐进了办公室内。

在办公室内坐了不久,前台的电话座机响了。史丹凤接了电话一听,对方竟然是丁丁先生。把电话转到办公室内,白大千抄起手边的电话话筒,无精打采的``喂''了一声。

丁丁的态度堪称有礼,恢复了起初的翩翩风度:``白大师,我想,我们还是有必要再谈一谈上次的交易。''

白大千的精神瞬间紧张了,肉体却是依旧松懈:``哦\ldots{}\ldots{}''

丁丁很好听的笑了:``昨天我们小小的试探了白大师一次,起初见白大师完全不设防,还以为您是浪得虚名。没想到一夜过后您安然无恙,才知道您是真有本事,竟然已经驯服了罐子里的小东西。白大师,坦白的讲一句,我们很佩服您。''

白大千感觉自己气息微弱,仿佛随时都要眩晕:``嘤\ldots{}\ldots{}''

丁丁又道:``对于白大师您,我想再谈交易就不恭敬了。我们不谈交易,改谈合作如何?毕竟那个东西凝结了我们的心血,如果任由它逃了,终归是一笔大损失。实不相瞒,我们本来是想把它带走的,但是既然一时半会不能成行,那么我索性对白大师实话实说。这个东西,制出来就是为了用的,我们既然忙着要它,自然也是有急用。如果白大师这回肯配合我们的行动,我们不但愿意付您一笔酬金,而且还可以把它留在您的身边,只要在我们需要用它的时候,您能出手相助便可以了。''

白大千满头满脸的出冷汗:``嗯\ldots{}\ldots{}怎么相助?''

丁丁答道:``这个\ldots{}\ldots{}恐怕要劳您的大驾,和我们一起出趟远门。''

白大千有气无力的答道:``我不要钱,也不出门。那个东西你爱抓就来抓,抓走最好。再见。''

把电话一挂,白大千趴在了桌子上,哼哼的呻吟:``丹凤,你打电话给帝豪皇宫食府,定个晚上的包间。我现在虚得很,一点力气都没有,得吃顿大餐补充一□力才行。''

史丹凤绕过屏风,好奇的看了看他,见他真是面无人色,便给他沏了杯热茶。白大千常年穷困潦倒,许久没有得到过女性的关怀。如今小口呷着热茶,他赖唧唧的说道:``丹凤呀,来,坐到我身边,反正外面也没事情,我们正好谈谈心。你来公司也有一个多月了,生活工作都习惯吗?毕竟是一个女人离家在外,身边除了弟弟之外也没有别的亲人,会不会偶尔感觉空虚寂寞冷?''

史丹凤没有坐,站着答道:``习惯,挺好的,也不冷。白大师你先养一养神,我去给饭店打电话定包间。''

然后她绕过屏风,急急的溜走了。一边溜一边想我弟弟也是公司的老板,难道你还真想拿我当女秘书消遣?

一个电话打完,史高飞和无心从外面进来了。楼中保安队长养的大狼狗夜里死于非命,乍一看没有伤,仔细一找才从狗脖子上找到了小小的伤口。去围观的人不少,踩着满地狗血欣赏保安队长嚎啕。大狼狗直直的伸着四条腿,据说是一身的血全淌光了。观众们一边看,一边称赞白大师的护身符真灵。因为戴了护身符的保安队长安然无恙,没戴护身符的队长之狗则是死了。

一个下午的工夫,白大千又卖出了无数护身符。四人晚上出门肥吃海喝了一顿,夜里醉醺醺的回了家。白大千依然不肯独处,非要挤到史高飞的床垫上睡觉。史高飞有子万事足,并不管他。只是史高飞和白大千虽然睡得酣然,但无心被他们夹在中间,别说翻身,甚至连动都都不得。身上压着史高飞的胳膊腿儿,面孔贴着白大千的后背,他在此起彼伏的鼾声之中睁大眼睛,无论如何睡不着。脑筋一个圈接一个圈的转着,末了他忽然起了贼心,小心翼翼的起身下床,推门进了客厅。

轻轻的敲响了史丹凤的房门,他压低声音唤道:``姐,是我,无心。''

史丹凤的房内亮着灯,一阵轻微响动过后,房门开了,史丹凤穿着睡衣伸出脑袋:``干什么?''

无心还是一身短打扮,抱着肩膀小声说道:``姐,白叔叔和爸睡一张床了,没给我留地方。你带我睡吧,好不好?''

史丹凤一听,立时红了脸:``你怎么不上白大师屋里睡?''

无心冻得皮肤蜡白,仿佛快要打哆嗦:``他屋里不干净。''

史丹凤刚要说出一个``不''字,哪知无心动作极快,竟然在自己开口之前摇头摆尾的向内一钻,直接钻进了房内。蹦蹦跳跳的跑到床边跪坐下去,他很自来熟的掀开被子躺下了。

史丹凤回头看着他,虽然知道他不算个人,可还是感觉不大对劲。迟迟疑疑的把门关了,她转念又想:``反正我是单身,没人管得着我,我怕什么?''

走回床垫边蹲□,她也上了床。倚着一个竖起来的大枕头靠墙坐了,她拿起方才翻到一半的杂志继续读。眼睛盯着书页,神经末梢却是伸展向了四面八方。两条腿直直的靠边放了,她生怕自己会不慎碰到无心。一直没想过给无心买睡衣,以至于无心现在光溜溜的,夜里离了被窝就要害冷。

心不在焉的翻了一页,她又意识到了新的一点:其实她很少单独的和无心共处一室,几乎少到了没有。试试探探的扭了头,她发现无心侧身对着自己,脑袋已经快要拱到自己的腰间。

``你好好睡。''她拍了无心的头:``别往我这边挤。''

无心仰了脸看她:``姐,你怎么不睡?''

话音落下,他在被窝里换了个姿势,动作之际,小腿蹭过了史丹凤的脚趾头。史丹凤一惊,差一点就要抬腿躲闪,然而强忍着没有躲,因为自己心里明白,那不值一躲。

她的心还没跳匀,无心又出了声:``姐,别看了,睡觉吧。''

史丹凤放下了杂志,目光沉重而迟钝的又扫了他一眼,扫得结结实实,把无心的小白脸子和大黑眼睛全印在了眼里心里。再扫一眼,鼻梁和嘴唇也记住了,直鼻梁,红嘴唇,皮肤嫩得阴森森,是个好看的家伙。

史丹凤收起杂志,关了电灯,摆好枕头往被窝里一沉:``睡觉。''

然后她大着胆子推了无心一把:``往那边去,咱俩一人一个枕头,谁也不许挤谁。''

无心果然乖乖的躺到``那边''去了,可是过了不过片刻,他磨磨蹭蹭的翻了身,又凑回到了史丹凤身边。

史丹凤感觉到了他的目光,于是微微侧身去看他。窗外正好邻着路灯,史丹凤借着灯光,能够隐约看清无心的脸。无心很专注的凝视着她,一双眼睛睁得奇大。看了良久,他缓缓垂下眼皮,同时从被窝里抬起了一只手。手是雪白的,干干净净,在空中停顿了一瞬,随即深思熟虑似的慢慢下落,一直落到了史丹凤的胸脯上。手掌贴着一层睡衣,无心又抬了眼睛望向史丹凤,目光非常懵懂,非常无辜,同时又是非常的欢喜。

垂眼再次看了自己的手,无心的手指轻轻合拢了一下,额头也向前触碰到了史丹凤的面颊。脑袋微微摇晃着,他用最小的力气去顶去蹭:``姐\ldots{}\ldots{}''

史丹凤在度过了最初的惊愕之后,胸腔里燃起了一团火。低头望着自己胸前的手,她下意识的来了一句:``干什么?我又不是你妈。''

无心欠了身,把脸贴上了史丹凤的胸脯。胸脯波涛起伏,柔软芬芳,让他联想起了一切温暖香甜的所在。贴了一下,随即抬头,他依旧是大睁了眼睛望着史丹凤,仿佛两个人中,受惊的是他。

于是史丹凤又问了一句:``知道什么是妈妈吗?''

无心摇了头。

他把史丹凤摇得立刻不忍心了。短暂的对视过后,史丹凤把他拉扯了上来:``好好睡,别乱动。''

无心贴着史丹凤躺好了,一只手依然抓在对方胸前。史丹凤犹犹豫豫的扯开了他的手,然而扯开之后她刚一松手,那只手就又回来了。

拉锯战似的撵了又来,来了再撵,最后史丹凤抓起无心的手,当真是在那手掌上不轻不重的咬了一口。

这一口疼得无心出了声。及至她松了口,那只手鬼鬼祟祟的,又奔着目标去了。

史丹凤不撵了,无心遂了心愿,也不动了。

翌日清晨,史丹凤照例早早醒了。睁眼向旁一看,她发现无心搂着自己的腰,还在大睡。

屏声静气的挣开了对方的束缚,史丹凤回想昨夜情形,感觉还是不对劲——不该收留无心的,不管他本质上是个什么,至少看起来是个男人。然而坐起身低头又看了看无心,她含羞带愧之余,又藏了一点小小的窃喜。还是那句老话:不管他本质上是个什么,至少看起来是个男人。自己老大不小的,无论如何,喜欢男人总不能算错。

她轻手轻脚的抱着衣服去了卫生间,关了房门悉数穿好。洗漱过后下了楼,她照例是去买早餐。等到她带着肉馅饼回来了,正赶上史高飞在卧室里发出了一声怒吼:``啊!好恶心哪!!''

随即房门``咣''的一声开了,史高飞光着膀子穿着裤衩,跌跌撞撞的跑出了卧室:``宝宝!宝宝!''

史丹凤的房门也开了,无心揉着眼睛走进客厅:``爸,怎么了?''

史高飞先是一把抱住了他,紧接着转身指向了站在门口的白大千:``我、我、我以为他是你,居然搂着他睡了一夜!早上我还亲了他的鼻尖!''

白大千的金丝眼镜歪挂在耳朵上,用手背把高鼻子擦了个东扭西歪:``我不嫌你就不错了,你还敢嫌我!你说,我怎么恶心了?''

史高飞气得问道:``你为什么冒充我儿子,还到我的房里睡觉?''

不等白大千回答,他转身又问无心:``宝宝,你夜里到哪里去了?是不是白大千把你赶走了?''

无心张了张嘴:``我\ldots{}\ldots{}你们两个都挤我,所以我就到姐姐房里睡了。''

此言一出,白大千立刻瞠目结舌。而史高飞怒不可遏的抬手指点着白大千:``姓白的,你凭什么把我儿子挤到我姐房里睡?你自己怎么不去呢?''

白大千听了他的奇思妙语,越发张大了嘴。而史高飞还要叫骂,冷不防史丹凤卷起一本旧杂志,``唰''的抽上了他的后脖颈:``放你的疯屁!''

史高飞捂着后脖颈,还和史丹凤嘴硬:``姐,你怎么胳膊肘往外拐?你说白大千坏不坏?你不打他你打我?''然后他转向白大千,坚持把话骂完:``以后不许你再到我房里睡觉!要睡找我姐去,我姐一个人睡一张床,我们两个人睡一张床。你放着宽敞地方不去,非得挤我们,真是又愚蠢又讨人厌!''

史丹凤对于他是身经百战了,此刻用杂志卷指着他的鼻尖,她横眉怒目的质问:``还说?还说?我给你脸了是不是?信不信我打电话让爸来抓你回家?''

史高飞很不忿的闭了嘴,又抬手指了指白大千,是个意犹未尽的样子。

\chapter{相约}

白大千死了一回,自觉长了许多见识。翘着二郎腿坐在办公室里,他信口开河,开始讲述自己灵魂出窍之时所见的众鬼。史丹凤拿着一份娱乐小报,坐在一旁半听不听半信不信。无心端着一碗方便面,不早不晚的给自己加餐。唯有史高飞听得认真,不时发问,把白大千的讲述搅成了一团乱麻。最后白大千急了,对着史高飞怒道:``你还要我说多少遍才能明白?我见的是鬼,不是外星人,和霸天虎更是没有半分钱的关系!''

史高飞听到这里,触动心事,当即转向无心一拍大腿:``哎呀宝宝,爸爸很久没有给你买过香芋派了。''

无心从方便面碗里抬起了头:``爸,姐中午给我买了栗子饼。''

史高飞又转向了史丹凤:``又是要过期的便宜货吧?''

史丹凤恨不能活活掐死他:``又不是给你买的,怕有毒你别吃!''

白大千冷眼旁观,很希望史丹凤能够大发淫威,把史高飞揍一顿。然而史丹凤富有理智,不到忍无可忍的时候,绝不会轻易对弟弟动手。把一份小报卷成了卷,史丹凤无可奈何的去看无心。无心终日大嚼垃圾食品,尤其是把方便面当成美食,不但吃了面,而且还喝汤。要是再有地沟油炸出的油炸肉串佐餐,就更合他的心意了。史丹凤不知道他是个什么体质,所以时常暗暗担心,怕他会被垃圾食品毒死。

无心哧溜哧溜的吃面,呼噜呼噜的喝汤,导致满办公室都是方便面的气味。天气太冷,不宜开窗,白大千只好走去开了公司大门,又皱着鼻子回头说道:``无心,别吃了。不是我说,你有点儿影响公司形象。''

史高飞攥起了一对大拳头,在动武之前特地问了一句:``你是说我儿子长得丑吗?''

白大千立刻摇头:``不是不是,我是说方便面味儿太大,熏得我坐不住。''

史高飞听闻此言,通情达理的松了拳头。

白大千拿了一张脸盆大的硬纸板,满屋里扇动空气,想要让方便面的气味快速流出办公室。史丹凤打开一本旧杂志,也张牙舞爪的帮忙。两人最后移到门口,将武器合力向外一挥。只听``啪''的一声轻响,旧杂志和硬纸板重叠出击,正好拍到了一位来客的脸上。

两人吓了一跳,连忙一起放下了手。来客高高大大的站在门口,脸上笑容不变,却是丁丁先生。

几日不见,丁丁剪短了头发,穿着带有裘皮衣领的短大衣,胸前挂着一排牛角扣,不但相貌英俊依旧,而且还比先前增添了几分青春气息。史丹凤见了他,不禁心中暗赞:``太帅了。''

白大千也承认丁丁的帅,问题是对他来讲帅不值钱,他看自己也十分帅。带着一点敌意堵在门口,他开口打了个招呼:``丁先生。''

丁丁满面春风的向房内一伸手:``白大师,我可以进去和你谈吗?''

白大千犹犹豫豫的侧身让出了通道。而史丹凤先人一步的绕过屏风,没收了无心的方便面。史高飞起身踮脚,目光越过屏风往外看:``哟,鸭子又来了!''

趁着丁丁不留意,史丹凤把史高飞和无心全带到了外间前台,又低声呵斥弟弟闭嘴。丁丁对着无心点头一笑,然后随着白大千进了后方办公室。史高飞占据了前台的椅子,兴致勃勃的对史丹凤说道:``鸭子今天还扮嫩呢,是不是他觉得自己有点儿老,怕白大千不要他?''

史丹凤听了个莫名其妙:``你说什么呢?''

史高飞理直气壮的告诉他:``白大千说了,鸭子和人好,都是为了钱。你看他来了又来,肯定是知道白大千发财了。''

史丹凤抽出一张面巾纸,给无心擦了擦嘴上的油,然后说道:``无心,你带着他出去逛逛。再由着他胡说八道的话,客户能被他得罪光了。''

无心很听话的起了身,带着史高飞出门下楼。在路边摊里吃了几串烤鱿鱼之后,他们回了公司,发现丁丁还在和白大千扯皮。董来覆去的劝白大师和自己合作,白大师口干舌燥的拼命推脱。小小的写字间里,丁丁富有磁性的低沉声音回荡不已,充分显示出了他的男性魅力,听得史丹凤如痴如醉,可惜内容略显空洞乏味,因为白大千始终是不动摇。

无心听出了白大千没有还手之力,于是脱了外面的厚衣服,径自走进了里间的写字间:``丁先生,你们没有诚意。''

丁丁一看又是他来了,登时有些头痛:``我们没有诚意?何以见得?''

无心站到了白大千身边:``我和我师父已经全在你面前了。可是你们呢?你们的人躲在幕后,只派了你一位说客露面。你说你们有诚意,我们会相信吗?''

丁丁一听他说话就要生气:``怎么?在你们眼中,我只是一位说客?''

白大千迟疑着没有回应,无心则是很痛快的点了点头:``对,在我们的眼中,你只是一位说客,和我师父讨价还价,你不够资格。''

丁丁一跃而起,一脸要吃人的怒容:``我也是有身份的,我——我阿爸——你们真是看低了我!''

无心把手插到裤兜里,向他一探身,笑微微的又问:``你是哪里的人呀?''

丁丁的下巴在裘皮领子的包围之中向前一抬:``哼!我是哪里的人不关你事!''随即他低头望向了白大千:``白大师,恕我直言,你的徒弟很讨人厌,你应该尽早把他逐出师门!''

白大千扶了扶金丝眼镜:``我感觉他还可以,也不是特别讨厌。''

丁丁对着白大千说得嗓子都哑了,结果不但徒劳无功,还被白大千的徒弟狗眼看人低、侮辱了一通。抬手系好领口的圆盘大纽扣,他用裘皮领子保护住自己的脖子,紧接着愤然转身,炮弹似的直接轰向了玻璃门,一边走一边又道:``敬酒不吃吃罚酒,好,你们等我阿爸亲自出面吧!''

白大千最近卖护身符也挺挣钱,又被鬼魂和怪婴吓破了胆子,所以对丁丁提出的合作毫无兴趣。史丹凤坐在前台,则是偷偷的去问史高飞:``无心到底是和谁学的说话?我看他说话说得比你好。''

史高飞答道:``有其父必有其子,其实我也很厉害的,只是一直生活在地球人的家庭里,被你们拖累了。''

史丹凤转向前方,心想自己得把这么个弟弟照顾到死,真是上辈子做大孽了。

白大千穿衣戴帽,拿着一副皮手套出了门,要趁着天早进城去看望佳琪。他这个女儿倒是放在哪里都不会招灾惹祸,然而毕竟是脑筋太慢,让他永远不能彻底放心。

傍晚时分,他披着一身的雪花回来了。将一只十字绣钱包扔给史高飞,他不大高兴的说道:``喏,佳琪给你绣的。''

钱包上面绣着几片绿叶和两只花狗。史高飞看了看,也不道句辛苦,直接将一沓子脏兮兮的零钱塞进了钱包。白大千看在眼中,气得要死,恨不能把钱包再要回来。

与此同时,无心正在厨房里给史丹凤帮工。史丹凤买了一些又丑又小的黄瓜,想要切成丝做凉拌菜。无心站在一旁,碍手碍脚的帮她拿东递西。史丹凤切着切着忽然停了菜刀,挑出一片黄瓜喂给了无心。

无心吃了黄瓜,然后张嘴还要。史丹凤切了一小块黄瓜头塞进他的嘴里:``不给了,再给你就不够做菜的了。''

及至把凉拌菜做好了,史丹凤走到电饭锅前打开了锅盖。在骤然腾起的热蒸汽中,她正要回身去拿饭碗,不料把身一转,她正和无心打了个照面。

``姐\ldots{}\ldots{}''厨房关着门,电灯被蒸汽熏得朦胧了,无心也像是站在了云里雾里。很忸怩的望了史丹凤一眼,他垂下眼帘小声说道:``摸一下。''

然后不等史丹凤有所反应,他抬起一只手,当真在对方的胸脯上轻轻摸了一下。摸完之后他抿嘴笑了,一边笑还一边点点头,是个心满意足的模样。

史丹凤不知道自己应该作何表示,想了一想,也没想出答案,于是只低声说道:``当着别人的面,不许和我闹。''

随即她自顾自的去拿饭碗,在盛饭的时候又后了悔,感觉自己话里有破绽——当着别人的面不许闹,难道背了别人就可以了?

史丹凤心里有些乱,本来想再炒个鸡蛋的,一乱,也就没炒,只打开了一个鱼罐头充数。马马虎虎的将一顿晚饭打发过去了,她在史高飞的卧室里看了一会儿电视,然后就想回房休息。临走之前她下意识的瞟了无心一眼,结果发现无心正在直盯盯的看着自己。

她没言语,穿了拖鞋往外走。穿过客厅进了卧室,她没锁门,独自坐在了床垫上。昨天带着无心睡了一夜,虽然纯粹只是睡,但也像是有种隐秘的快乐在其中。想起无心的温度,气味,眼神、身体;史丹凤出了神,同时感觉心中空落落的,满怀的爱意无处发散,简直快要过保质期了。

正当此时,房门一开,无心悄悄的伸进了头:``姐。''

史丹凤冷不防的见了他,情绪居然堪称惊喜:``你来干什么?''

无心轻轻巧巧的溜入房内,还是一身裤衩汗衫的短打扮。转身把房门锁好了,他欢天喜地的跳上床垫,一头滚到了史丹凤的怀里。枕着史丹凤的大腿仰卧了,他掀起对方的睡衣下摆,用一只眼睛从下往上看。史丹凤没先到他如此胆大手快,刚要出言阻拦,然而无心猛一抬头,一个脑袋已经钻进了睡衣里面。史丹凤``咝''的吸了一口冷气,双手托着无心的肩膀后背,本来运了一股子向外的力气,如今力气引而不发,潺潺的化于无形了。

最后她强定心神,硬是把无心推离了自己。无心没有远离,依然枕着她的腿,嘴唇湿漉漉的泛着殷红颜色。

史丹凤把双臂环在了胸前,面红耳赤的说道:``好了,不许闹了。真把我当你妈了?''

无心起了身,张开双腿跪坐在了史丹凤的大腿上。史丹凤看他撒娇撒来了劲,正要撵他,哪知他伸了手臂向前一扑,正把她抱了个满怀。柔软的嘴唇凑到史丹凤耳边,她听到他轻声说:``我喜欢姐。''

史丹凤双臂似抬非抬,不知道自己对待无心是该推还是该抱。一只手漫无目的的下落了,落的地方非常不合适——无心身上的劣质裤衩已经松松垮垮的没了形状,方才在他的动作中越发变了形,竟然尽数偏向一侧,让他露出了半个屁股。史丹凤托着他的屁股蛋怔了半天,半天过后猛一抬手,感觉自己的心都要跳出喉咙口了。

然而无心却又老实规矩了,单是紧紧的抱着她,也不动,也不松。

等到电视剧演完了,无心自动的回了卧室。白大千抱着个枕头,站在门口探头缩脑。史高飞一边铺床,一边很警惕的瞄着他,无论如何不许他进门。

无心侧身躺在床垫正中央,闭着眼睛浮想联翩。想着想着,他美滋滋的缩成了一团。史高飞蹲在床尾,忽然抬头看了他一眼,随即展开棉被,向上盖住了他。

白大千在床垫旁强行打了个地铺,对付着过了一夜。翌日清晨无心睡了懒觉,起床之后他坐在床垫边穿袜子,耳边听到史高飞在客厅里对史丹凤嗡嗡的说话:``姐,以后不许你给宝宝买衣服了。你看你给他买的破裤衩,越洗越大,比面口袋还松。昨夜他在床上一躺,我抬头一瞧,发现他连□都露出来了。''

话音落下,史丹凤没回应,倒是响起了白大千的笑声。

无心登时抱着脑袋往床上一滚,感觉自己无颜走出卧室见人了。正是无可奈何之时,客厅里忽然又起了异常的陌生声音:``白大师,白大师\ldots{}\ldots{}''

无心先以为客厅里来了客人,可是转念一想,又感觉不对劲。穿上衣裤起了身,他打开房门向外探身,结果在吃早餐的三人身边,发现了一只女鬼。

此女鬼生得高颧骨大腮帮,一张脸堪称骨骼清奇。虽然不知道她当初是因何而死,但是死相挺干净。无心观望之时,她悬浮在白大千面前,一边呼唤一边手舞足蹈。可白大千捧着一套煎饼果子大嚼,除了感觉有些寒冷之外,并无其它不适。女鬼显出了无计可施的沮丧相,突然意识到了无心的目光,她当即扭头望向了无心。

无心没言语,只向她递了个眼神,然后轻轻的关了房门。果然,女鬼在几秒钟之内出现在了他的面前:``你是谁?难道你也能看得到我?''

无心小声反问:``你又是谁?为什么大白天的骚扰我师父?''

女鬼扬起骨感大脸,眼中流露出了崇拜的神色:``我的主人让我来给白大师送信,可是我没想到白大师的修为如此之高,定力如此之深,居然对我听而不闻,视而不见。我从凌晨他去厕所撒尿时开始尾随他,忙到现在也有好几个小时了,可他我行我素,硬是不肯鸟我。''说到此处,女鬼一挑大拇指:``真是大师风范,真男人!''

无心张了嘴,眨巴眨巴眼睛又问:``我看你也是有些法力的老鬼了,你没试过在他面前现形吗?''

女鬼不屑一顾的一撇嘴:``人家是大师,一双阴阳眼,两手乾坤术,还能看不到我吗?我何必还要多此一举的现形?''

无心深吸了一口气,感觉女鬼的智商绝不比佳琪高:``你怎么认定他是大师的?''

女鬼答道:``主人说的。''

无心指了指自己的胸膛:``我是白大师的大弟子,有什么事情你先对我说。如果师父今天心情好的话,我再把你的话转达给他老人家。说吧,你主人是谁?找我师父有什么事?''

女鬼颇为喜悦的点头同意:``好,好,其实我主人找白大师也没什么大事,只是久仰他法术高明,想和他会一会面。其实我家少爷已经和白大师见过好几次了,但总是谈不拢。白大师太超脱了,淡泊名利,连钱都不往眼里放。搞得少爷拿他没办法,主人只好亲自出山了。''

无心对着女鬼笑了:``行,我记住了。如果见面的话,时间地点谁来定?''

女鬼想了一想,随即答道:``谁定都行。''

无心挥了挥手:

``你们定吧,定好了来告诉我。记住,找我就好了,别惹我师父。我师父一生气,把你打成魂飞魄散就不好了。''

女鬼很赞同:``对,那的确是不大好。''

女鬼清晨离去,下午又回来了,和无心约定了时间地点。

白大千听说无心代替自己做了主,要去赴怪婴主人的鸿门宴,当场吓得瘫在沙发椅上不能起立。史高飞在秋天里长了几斤肉,如今身大力不亏,索性把他背出了写字楼。写字楼前的大街上从早到晚总停着一排黑出租车,上车前要先讲明价钱。白大千落了地,不情不愿的先讲价后上车,带着两名伪徒弟直奔市区。

出租车开到半路,司机一踩刹车靠边停了:``你们到地方了。''

白大千奇道:``到地方了?不对呀,还有一半的路没有走呢!''

司机摇下车窗,点了一根香烟:``你给的钱只能开到半路,要不然不够油钱。''

白大千着了急:``事先都说好了的,你怎么——''

司机悠然的吐了个烟圈:``事先你说要进市区,现在已经进市区了,我不往前走也不算错吧?''

白大千正要争辩市区边缘不算市区,然而未等他开口,眼前骤然一花。耳中听得``咚''的一声,却是坐在副驾驶座上的史高飞不耐烦了,一记直拳正中司机的下颚。随即抓住司机的头发,史高飞将对方的脑袋接二连三的撞向车窗窗框:``王八蛋!你是活不起了还是怎么的?居然抽这么次的烟来熏我!''

白大千和无心一拥而上,七手八脚的去拦史高飞。

半个小时后,一辆无牌黑出租车缓缓停到了市中心商业区一隅。车门开处,史高飞趾高气扬的下了车,而驾驶座上的司机一头乱发,抽抽搭搭的含泪扶着方向盘。白大千从后方探过脑袋,柔声问道:``要不然,我还是给你多加五块钱吧?''

司机看了车外的史高飞一眼,随即恐慌的摇了头:``不,我不要。''

白大千和无心也下了车。其中白大千最认路,而且身已至此,别无选择,只能硬着头皮安心赴宴。领着史高飞和无心向前转了一个弯,在一座黑色的店面门前,他们停住了脚步。

店门口的空地上立着一座雕塑,是只又像马又像驴的卡通动物,穿着西装做侍者状。三个人一起抬了头,只见店面招牌上亮着七个大字:我爱骡主题餐厅。

\chapter{丁阿爸}

我爱骡主题餐厅共有二层楼,楼中装潢得有趣,处处都带有一点卡通风格,连门童的服装造型都和门前的骡子塑像一致。白大千带着史高飞和无心进了餐厅,由迎宾小姐引领着上了二楼。迎宾小姐拥有魔鬼身材,打扮成了兔八哥的模样,可惜是只穷困潦倒的兔八哥,因为脑袋上的兔子耳朵耷拉了一只,翘屁股上的短尾巴也脱了线,随着她的步子一甩一甩。

在二楼一间名为``蘑菇村庄''的包房门前,迎宾小姐停住了脚步。白大千见包房半掩着门,便试探着伸手轻轻一推。房门顺着力道开了,白大千身后的史高飞和无心一起伸了脖子向内张望,结果只见房内站着一高一矮两个人,正在动作一致的搓手呵气,看样子也是刚刚到达,甚至连身上的厚重外衣都还没脱。忽见白大千等人来了,高个子的丁丁登时抬手扶住了身边矮个子小老头的肩膀,不由分说的把人往前一推:``喏,白大师,我阿爸来了,我不配和你谈,我阿爸总能入你的眼吧?''

白大千猝不及防的和丁丁的阿爸打了照面,上下略一端详,他发现对方其实并不算矮,堪称中等身量,只是受了丁丁的衬托,才显得小了一号。看他一脑袋浓密的花白头发,应该得有个五六十岁了,可是头发虽然沧桑,一张脸却挺年轻,是个慈眉善目的娃娃脸老头。抬手一扶滑到鼻尖上的半框眼镜,小老头的眼珠从左至右转了一圈,随即瞄准无心,``嗤''的一笑。

无心打了个冷战,感觉对方有一点眼熟,可要说是谁,却又死活想不起。畏惧似的往史高飞身边躲了躲,他当儿子当得正幸福,真怕来个知情人戳穿了他的身份。

而在此时,丁丁爸爸已经笑呵呵的对着白大千伸出了手:``久仰白大师的大名了。免贵姓丁,丁思汉。''

白大千立刻满面春风的和他握了手:``丁老先生,你好你好。大师二字我怎么敢当,叫我的名字就好。''

丁思汉和白大千四手交握,上下摇动:``称你一声老弟不介意吧?''

白大千配合着他的动作:``不介意不介意,叫老弟正合适。听丁老兄你的口音,也是北方人?''

丁思汉嘻嘻哈哈:``少小离家老大回,乡音未改鬓毛衰,说的就是我了。''然后他歪着脑袋,目光越过了白大千的肩膀:``后面两个小伙子是白老弟的高徒吗?''

白大千松了手,侧身让史高飞和无心也亮了相。要说体面,史高飞的形象最好,走T台都够资格了,只可惜状态太不稳定,未等白大千开口,他已经自顾自的走到一旁,看起了墙壁上的卡通画。无心孤零零的站在原地,走也不是留也不是。飞快的又瞄了丁思汉一眼,他怎么看怎么感觉对方似曾相识。正是心中打鼓之际,白大千已经将他介绍完毕。

勉强对着丁思汉笑了一下,他无话可说。然而丁思汉却是开了口:``好,真年轻。''

此言一出,无心的脑子里拉了警铃。盯着丁思汉的眼睛怔了一瞬,他忽然认出了对方。

``你\ldots{}\ldots{}''他颤悠悠的打了结巴:``你\ldots{}\ldots{}你好。''

丁思汉哈哈大笑,一边笑一边抬手一指无心的鼻尖:``小伙子,我不认识你,你认识我吗?''

无心刺猬似的竖起了一脑袋短毛,脊背直冒凉气:``不认识,完全不认识。''

丁思汉又搓了搓手,随即上前捧住无心的脸用力一挤:``年轻貌美啊,不错不错。''

未等无心反抗,包房内部忽然响起一声惊叫。丁思汉回头一瞧,只见史高飞不知何时转到了丁丁身后,一手掐着丁丁的脖子,另一手捏住了丁丁的面颊。双方遥遥的对视一眼,史高飞怒道:``老家伙,再不放开我的儿子,我就捏死你的鸭子!''

丁思汉吓了一跳,立刻松了手。而丁丁被史高飞捏得嘴唇突出,有口难言。史高飞伸头看了看他的侧影,紧接着抬头对白大千和无心笑道:``哈哈哈,你们看,他真的很像鸭子耶!''

白大千讪讪的转向丁思汉,同时抬起手指点了点太阳穴。丁思汉登时了然,了然之余又很惊讶,不知道白大千作为一名新兴的大师,为何会收这么一位脑筋短路的徒弟。

白大千派无心出马解救了丁丁,包房之内暂时恢复了和平。外面天寒地冻,众人一起宽衣落座。丁思汉虽然上了一点年纪,然而并未发福。指挥丁丁把自己的棉外套挂到衣帽架上,他露出了里面的雪白衬衫和天蓝背心,绒线背心的领口镶着一道花格子边,胸前还织出了一张金黄色的笑脸图案。和白大千推让了一番过后,他理所当然的坐到了首席。抬手摸了摸自己花白的短头发,他得意洋洋的环顾四周,神情与形象都很像一名老男童。

丁丁本来就不甚高兴,在被史高飞捏痛了脸之后,越发的郁郁寡欢。在等待上菜的间隙里,丁思汉用一侧胳膊肘撑上桌面,托着面颊含笑发问:``你们两个,谁是谁的儿子?''

无心怕史高飞胡说八道,只得主动答道:``我是儿子,他是爸爸。''

此言一出,白大千羞愧的低下了头,心想对方一定以为自己两个徒弟全是疯的。

丁思汉点了点头:``有趣,虽然我还是没听懂。''

白大千一伸脖子,用自己堂皇的大白脸阻挡了丁思汉的目光:``丁老兄,晚辈的事情姑且不要管它,我们既然有缘相见了,不如好好的谈一谈正事。''

丁思汉刚要回答,包房房门一开,却是到了上菜时间。史高飞当即抓起无心的一只手,将一副筷子塞进了他的手中,口中兴高采烈的欢呼道:``宝宝,吃饭了!''

白大千和丁丁一起叹了口气,无心则是将要脸红,喃喃的想要抽回手:``我不饿。''

史高飞知道自己的儿子一贯嘴馋,所以根本不信:``不饿?不爱吃吗?要不然爸爸带你去买香芋派?''

无心凑到史高飞耳边,压低声音急道:``爸!你不要当着外人叫我宝宝。''

史高飞恍然大悟,深深的一点头:``哦,爸爸忘了!''

史高飞由着性子连吃带说,无心想要倾听白大千和丁思汉的谈话内容,然而虽是近在咫尺,可因史高飞吵个不停,以至于他竟是一句也没听清楚。后来史高飞终于暂时安静了,无心在他专心致志剥虾壳之际,只听白大千笑道:``原来丁老兄还是在江口市长住过几年的。''

丁思汉点了一根雪茄,回答之前先深吸了一口:``当初来的还是太仓促了,以为可以在生意上发点小财,挣一点养老钱,没想到我不是经商的料,对于本地的情况也了解不深。加之身体不好,总要回家养病,所以才半途而废,没能把度假村经营到底。''

白大千做出同情神色:``丁老兄有什么病?''

丁思汉坦然的答道:``小病,精神分裂症。''

白大千的心一哆嗦,暗想难道我今年命犯精神病?身边的两位已经是不正常了,来了个对头更是凶险,居然还会分裂。温柔如水的笑了一笑,他问丁思汉:``现在已经痊愈了吗?''

丁思汉咬着雪茄,先是抬手摁了摁胸膛,随即笑道:``没事,有事的话,我也不能坐在这里同你讲话。''

白大千很勉强的继续微笑:``好,老兄吉人自有天相。''

然后他招架不住似的看了无心一眼,无心随即看了丁思汉一眼。丁思汉接收到了他无线电似的目光,当即一手夹着雪茄,一手拿着餐巾,微微的一点头。

因为史高飞剥虾壳剥得入了神,所以无心得了机会,以去洗手间为名暂时离席。他一走,丁思汉把雪茄交给丁丁,紧跟着也出门包房。两人在外面大厅里会了面。厅角一株巨大的凤尾竹下摆着沙发圆桌,正好能供他们相对着坐下。周遭没了听众,无心立刻开了口:``你当年叫什么来着?丁小猫还是丁小狗?几十年不见面,我记不清楚了。''

丁思汉很舒服的向后一仰,又对着无心翘起了二郎腿:``丁小猫,也有人叫我小丁猫,不过我改名了,现在我叫丁思汉。思想的思,好汉的汉。''

无心缓缓的眨了一下眼睛:``对,我记起来了,小丁猫。你为什么要改名字?''

丁思汉垂下眼皮,看自己穿着黑色马丁靴的脚在圆桌下面摇来晃去:``你不觉得那个名字听起来很可笑吗?''

无心一扬眉毛:``你刚知道?不过丁思汉也马马虎虎,好像是说你在想汉子,幸好你是个男人。''

丁思汉抬了头,脸上现出愕然神情:``是吗?''

随即他挥了挥手:``算了,随便吧,反正我已经老了,下辈子再换个新名字就是了!''

无心对于故人的情绪总是难拿捏,有点高兴,也有点害怕:``你的儿子真够帅的,比你年轻的时候强多了。''

丁思汉恨不能啐他一口:``他不是我的儿子,他是顾基的儿子!顾基你还记不记得?一个傻大个儿!''

无心费了很大的力气,才影影绰绰的想起了顾基这么个人。原来当初丁思汉胸怀祖国放眼世界,当真是带着顾基逃离北大荒,跑去缅甸闹起了革命。可惜事与愿违,他们跟着缅共越混越惨,最后实在是没活路了,只得作鸟兽散。而顾基虽然一贯是干啥啥不行,吃啥啥不剩,可是在革命最艰苦的时候,他竟然和一位云南女知青搞出了个私生子。私生子先是由女知青抚养着,后来女知青病死了,私生子只好转给顾基接手。顾基不会养孩子,不想要;而丁思汉见孩子生得可爱,却是动了恻隐之心,顾基不要,他要。

私生子从此跟着丁思汉姓了丁。而丁思汉脱离缅共自力更生,凭着先天的本事,居然在阴阳一道上发了大财。丁丁十岁回国,一直在昂贵的国际学校里读书,越长越大,越大越像他亲生父亲,是标准的金玉其外、败絮其中。丁思汉在他身上耗费金钱心血无数,可他最终连个野鸡大学都没能读完。除此之外,他本事不大,脾气不小,很会对着他的阿爸发飙。丁思汉把他从小幼童养到了三十岁,展望前途,前途漫漫,只要丁思汉不死,恐怕还得把他养到四十岁五十岁。

``报应啊。''丁思汉把被烟草熏黄了的手指插到一头花白短发里面,缓缓的向后梳去:``当初我拿顾基当奴才,现在轮到他儿子把我当奴才了。''

无心听得很有趣味:``顾基呢?''

丁思汉轻描淡写的答道:``他中了风,偏瘫,我把他送到老人院里去了。''

无心不知道老人院是个什么环境,想了一想,还是想象不出:``老人院\ldots{}\ldots{}好吗?''

丁思汉一摊双手:``里面全是一些没人要的老家伙,应该是不怎么样。不过我花了足够的钱,让他能够住单人间,吃的好喝的好,也不会受护工的虐待。''

无心不再问了,改为专心审视丁思汉的面孔。对方眼角的皱纹和松弛的皮肤都让他感觉新鲜。他很久很久没有陪过一个人从小活到老了,对着面前这个一身娇嫩颜色的小老头,他双手做痒,恨不能去拍拍对方的脑袋。

丁思汉用鞋尖轻轻磕打着圆桌桌腿,对无心说道:``你想办法帮帮我的忙,让白大千把我的小儿子还给我。不肯还,借给我也行。我在云南揽到了一笔生意,急着用它。''

无心盯着他的脸,忽然问道:``精神分裂症真的好了吗?''

丁思汉没言语,只抬手又用力的按了按胸口,仿佛随时会有活物破胸而出。望着上方一排极长的凤尾竹叶子,他沉默了片刻,末了答道:``她很麻烦。我真怕她会伤害我的家人。''

然后他望向无心:``我是说丁丁。我没有正式结过婚,只有丁丁一个儿子。''

然后他将双手十指交叉着放在了大腿上,悠然神往似的去看远处:``我需要很多很多的钱,没有钱丁丁会饿死。幸好我还有下辈子下下辈子,否则做一个永远都不能退休的老家伙,真是太可怜了。''

无心低头对着桌面,看桌面倒映出了大厅天花板上的点点灯光:``我也不能退休,习惯就好了。''

丁思汉放下了腿,手扶膝盖向无心探过了身:``无论如何,尽快把我的小儿子给我。我拿到了它,好马上启程回云南。别让我在你身边停留太久,我对你没有恶意,可是她就不一定了。''

无心听到这里,忽然发现事态严重:``说实话,那个东西真不在我们的控制中。它行踪不定,根本没法子去找。''

丁思汉把双臂抱在胸前,很怀疑的看着无心:``你知道,我曾经让鬼魂上过白大千的身。他可在你家的厨房里见过它!''

无心前倾身体,因为极力想要压低声音,所以快要把嘴凑到了丁思汉的脸上:``谁知道它那天晚上为什么会出现?我只知道它不会伤害白大千!它就像——就像——就像已经认了白大千做主人似的!''

丁思汉一拍大腿:``这个吃里扒外的小崽子!''然后他一指无心:``我们好像真是天生的冤家,过去的事情不提了,据我所知,我埋在度假村里的骨符也是被你们破坏的,你知不知道我当初为了抓住那几只鬼,费了多大的心思?''

无心毫无诚意的辩解道:``不知道是你埋的嘛!你也是的,既然要走,为什么不把它们一起带上?''

丁思汉咬了咬牙,脸上有了一点怒容:``我那个时候\ldots{}\ldots{}状况不大好,顾不上它们了。''

无心和丁思汉在凤尾竹下密谈许久,末了初步达成协议,一前一后的回了包房。

史高飞给无心剥了一大碗虾肉。白大千则是在和丁丁闲聊。见他二人回来了,白大千继续谈笑风生,根本不往正题上靠。及至酒足饭饱了,双方一团和气的出了餐厅,各自离去。

在他们坐上出租车回家之时,史丹凤正坐在史高飞的卧室里看电视。电视里演的是什么,她全然不知道。手里抱着无心的枕头,她正在魂游天外,可要说是在思念无心,也不准确。

她也不清楚自己到底在想什么,只隐隐的感觉自己恐怕是要思春。二十来岁的时候,她也曾有过许多玫瑰色的美梦,不过那都是多少年前的事情了,她在近几年里一直活得心如止水。好端端的,怎么就又活回去了呢?

把怀里的枕头又搂紧了一点,她忽然生出了一个新想法:``三十如狼,四十如虎,莫非我要变成狼了?''

史丹凤并没有做母狼的意愿,不过缓缓揉搓着怀里的大枕头,她的内心的确是十分骚动。

正是心乱如麻之际,房门一响,白大千等人回来了。白大千一边往里走,一边回头对无心说话:``什么?让他亲自来抓?当然,我倒是没意见,可他会不会影响我们做生意?''

无心手里攥着一根奇长的糖葫芦,同时口中反问:``要不然能怎么办?他像狗皮膏药一样,你不让他抓,他一定缠着你不放。''然后他转向史丹凤,笑眯眯的喊了一声:``姐。''

史丹凤迎到客厅里站住了,舔了舔嘴唇,不知道说什么才好。而无心把手里的糖葫芦向前一递:``姐,给你买的。''

史丹凤接了糖葫芦:``你吃了吗?''

无心站在墙边,脱了外套往简易衣帽钩上挂:``没吃。''

史丹凤听了,就一直拿着糖葫芦不动口。等到无心换了家常衣服,随着史高飞进卧室了,她才跪坐在床垫一角,举着糖葫芦对无心招了招手:``过来,我一个人吃不完,你先吃。''

无心四脚着地的爬到了她身边,伸长了脖子张大了嘴,咬下了一枚山楂。嚼着山楂扭头望向史丹凤,他见史丹凤没有收回手的意思,便张开嘴,又咬一枚。

史丹凤近距离的望着他的额头、鼻梁、嘴唇、下巴,望着望着,忽然想起了小红帽的故事,并且感觉自己是故事中的狼外婆。

``怎么搞的?''她沉沉的思索,因为一直以为自己已经变成了性冷淡,所以对自己此刻的不冷淡,她感觉很是不可思议:``我总不会是看上他了吧?可他连个人都不是,而且刚出生了半年,品种和年龄都不合适呀!我要是真和他好上了,是不是得算人兽?我的娘啊,我口味好重!遗传的力量真强大,可能我也要疯了。''

\chapter{反噬}

无心夜里不睡觉,蹲在厨房里守株待兔,想要尽快捉到怪婴送还给丁家父子。

凌晨时分,客厅里同时开了两扇门。史高飞和史丹凤一起裹着厚衣服出了来。两人在光线暗淡的客厅里打了个照面,史丹凤怔了一下,随即问道:``你睡醒了?''

史高飞把一根手指竖到唇边,鬼鬼祟祟的对她``嘘''了一声。

然后两个人心有灵犀似的,一起走向了厨房。并肩站在厨房门口,他们看到了睡在墙角里的无心。无心穿着单薄的新睡衣,整个儿的蜷缩在一件羽绒服里,只斜斜的伸出了一只雪白赤脚,睡裤的裤管微微卷了,露出的脚踝已经冻到白里透青。

史高飞蹑手蹑脚的走上前去,小心翼翼的弯腰把他托抱起来。等他转身走出厨房门口了,史丹凤迈步跟上,伸手为他拢了拢羽绒服的前襟,又顺便摸了摸他的脚。脚凉如冰,简直不是活人的冷法。

姐弟二人静悄悄的进了史高飞的卧室。无心睡得很沉,身体软绵绵沉甸甸,摆成什么样子是什么样子。史高飞把他送进了热被窝里,同时听到史丹凤嘁嘁喳喳的低声嘀咕:``要是真把他冻病了,我看你把他往哪家医院送!''

史高飞把无心身上的羽绒服放到床垫边上,因为摸他的头脸也很凉,所以扯过一条枕巾蒙了他的额头耳朵。史丹凤见他忙得一言不发,忍不住又添了几句:``我发现你现在是越来越傻了,白大千还没怎么样呢,你倒是先把无心贡献出去熬夜受冻了。到底谁是你刨出来的?亲疏远近都不分了?有活儿全让无心去干,有钱可没见分给无心多少,都让你们两个吞了。你这算盘可打得真精,明天我也回家刨地去,万一再刨个无心出来,我下半辈子都有依靠了\ldots{}\ldots{}''

她轻声细语唠唠叨叨,没有一句话是中听的,最后她做了总结陈词:``你要养就好好养,不爱养了挖个坑把他埋回去!''

史高飞打了个哈欠,终于做了回应:``姐,你烦死人了。''

然后他俯身低头,在无心的脸上亲了一下,亲过之后他问史丹凤:``姐,他好可爱,你要不要也亲他一下?''

史丹凤抬手把长头发掖到耳后,犹犹豫豫的答道:``行,那就亲一下吧!''

跪在无心身边深深弯腰,史丹凤用嘴唇轻轻触碰了他的眉心,一触即收,不作停留,因为怕惊醒了他。

无心一觉睡到大天亮,睁眼之时已是日上三竿。屋子里只剩了史高飞陪着他,史丹凤和白大千早下楼到公司里去了。

无心抱着棉被呆望窗外,看夜里下了一场大雪,覆盖出了一个起起伏伏的白世界。他不知道怎样才能抓到怪婴,抓不到怪婴,就打发不走丁思汉。丁思汉口中的``她'',到底是谁来着?想不起来了,只记得``她''是个危险人物,很危险。

无心的心中素来很少有恨,因为在无涯的时间面前,他的敌人们下场统一,迟早都会化为一抔黄土。死去元知万事空,人家死都死了,没都没了,他还恨什么?不过他想自己肯定是恨过``她''的,而且恨得要命。几十年上百年过去了,往事全模糊成梦里的影子了,``恨''却还在,可见是真恨,至少当初是真恨。

极力的伸长了一条腿,他蹬了前方的史高飞一脚:``爸,还有我的早饭吗?''

史高飞盯着电视屏幕答道:``厨房里有热粥,自己喝吧!''

无心慢吞吞的穿起了衣裤:``姐煮了粥?''

史高飞心不在焉的答道:``她说你夜里冻着了,今天应该喝点儿热粥驱寒。''

无心听了这话,心中一阵欢喜。

粥煮得稠而滚烫,无心捧着饭碗喝出了一头的热汗,不知怎的,他忽然想起了白琉璃。白琉璃的不分好歹一度让他伤透了心,不过毕竟是老朋友了,哪怕在一起时是相看两相厌,分开久了,还是要惦念。喝着史丹凤给他煮的热粥,他格外想要献宝似的让白琉璃看看自己现在的好生活。

正当此时,骨神出现了。

骨神横眉怒目,光芒万丈的从天花板向下沉,经过无心时他没有暂停的意思,看势头是要继续往下穿透楼板。无心汗涔涔的抢着问了一句:``干什么去?''

骨神翕动着鼻孔,做无敌金刚状:``去报仇。''

无心愣了一下,随即追着说道:``怎么着?你的仇人来了?不行,你现在可别去添乱。你的仇人有精神分裂症,一旦你把他惹毛了,他兴许会发疯!''

骨神的大脑袋缓缓消失于地面,只留下一句气冲冲的怒吼:``不把他宰掉我也会疯的!''

无心留不住骨神,于是放下饭碗,他一转身冲出厨房,穿过客厅也开门下楼去了。

骨神虽然可以直线下降,但是因为怒火攻心,一时失控,直接降到了写字楼地下一层。他在地下迷了方向,气急败坏的向上一窜,结果瞬间窜上了六楼。而无心目标明确,反倒先他一步的进了公司。

公司里果然是来了客人,然而白大千不在,只有史丹凤一人负责招待寒暄。无心进了里间办公室一瞧,只见丁思汉父子坐在靠墙的一排沙发上,史丹凤一边给他们斟茶递水,一边微笑着解释道:``白大师早上接了个电话,去市里给一家公司看风水去了,说是半天之内肯定能回来。两位先生要是不急的话,就请坐下稍等一会儿吧。''

丁思汉上身穿着一件花格子羊绒外套,□配着卡其色裤子和低帮皮靴,头上戴着一顶圆圆的小礼帽,乍一看像个富有英伦风情的女学生。笑眯眯的对着史丹凤一点头,他随即转向了门口的无心:``来了?早上好。''

史丹凤放下茶杯直起了腰,认为丁思汉虽然造型奇特了一点,但依然不失为一个可爱的小老头。给无心也倒了一杯茶放到办公桌上,她静悄悄的走回外间坐了。

无心望着花枝招展的丁思汉,下意识的要冒冷汗:``今天\ldots{}\ldots{}开始?''

丁思汉从丁丁手中接过了一只扁扁的牛皮书包。把书包放在腿上,他开始从里面一样一样的往外掏东西。沙发是新购置的,沙发前的小茶几也是新购置的,配着沙发上的丁家父子,正是鲜艳成了一团。把一沓黄纸端端正正的放在茶几正中了,丁思汉随即又掏出了两只精致的木头盒子,分别放在了黄符两边。最后从书包表面的小口袋里抽出一条丝绸手帕,丁思汉擦了擦手,恭而敬之的打开了两只盒子。原来两只盒子里面并无玄机,其中一盒是香烟,另一盒是红色的印泥。

无心侧身退到床边站住了,倒要看看丁思汉的本事。丁思汉摘下眼镜又擦了擦,一边擦一边说道:``丁丁,给阿爸点根烟。''

丁丁依言点了一根香烟递给他。而他把烟叼进嘴里,正要伸手去摸黄纸,房间之内却是陡然卷入一阵寒风。无心看得清楚,正是骨神携着雷霆之怒来了。

光芒万丈的悬浮在丁思汉正前方,骨神歪着脑袋怒视了他,同时高高的抬起了双手。丁思汉漫不经心的向前扫了一眼,随即伸出右手食指,在印泥盒子里捺了一指头。暗红色的指尖落上黄纸,他龙飞凤舞的画了一道符,在骨神的双手将要落下之时,他抄起黄符向前一甩手,薄薄的黄符平行飞出,正中了骨神的鬼影。鬼影瞬间闪烁了一下,骨神大喝一声落下双手,只听半空中一声轻微爆响,黄符竟然自行破碎成了无数纸屑。

未等纸屑落地,第二张黄符飞向了骨神。骨神怒目圆睁,双手用力一拍。黄符悬在他的双掌之中,``啪''的一声又成了碎屑。然而未等骨神松手,第三张黄符又来了。

史丹凤坐在前台,只听办公室内噼噼啪啪响成了串。一片纸屑飘飘摇摇的落到了她的头发上,抬手摘了一瞧,纸屑一面是黄色,另一面是红色,带着股子甜腥的怪气味。她起了好奇心,正要起身去窥视一眼,可未等她动作,无心忽然发出了声音:``丁思汉,放了他吧!''

丁思汉咬着香烟低着头,充耳不闻的继续画符。将最后一道黄符向前猛地一挥,骨神向后一仰,要躲而没躲开。周身的金光骤然暗了,他求救似的扭头去看无心。张了张嘴没说出话,他的光芒越来越微弱,不过片刻的工夫,他的影子彻底消失了。

半空中的纸符飘然而落。丁丁起身绕过茶几,想要去捡。不料无心忽然弯腰出手,在他头里抢到了纸符。

丁思汉用丝绸手帕擦净了手指,然后夹着烟卷深吸一口:``你要他有什么用?他很不听话的。''

无心攥着黄符不松手:``把他给我吧!''

丁思汉若有所思的看着他:``白送给你?未免太便宜你了。''

无心把黄符揣进了紧贴身的衣兜里:``我不白要,以后有我帮你的时候。''

随即他抬了头:``你没感觉你现在有点儿奇怪吗?''

丁思汉盯着他看了良久,末了缓缓的一点头:``无心,我当然感觉到了。我自己的事情,我还不知道吗?''

抬手向外挥了挥,他又说道:``丁丁,你和史小姐回避一下,我有话要和无心说。''

等到丁丁和史丹凤都出门了,丁思汉站起身,开始在办公室内来回的踱步:``无心,我看起来是不是很像小丑?''

无心摇了头:``不像小丑,像小姑娘。''

丁思汉狠狠吐出了口中的烟蒂:``妈的,不说了!我已经想出了捕捉小崽子的办法,现在只需要一个诱饵。你不是说小崽子很喜欢白大千吗?好,让白大千做诱饵吧!''

无心紧张了:``你不能伤害白大千!''

丁思汉走到茶几旁边,弯腰又给自己点了一根烟:``伤不到他,只是要劳他出手,给小崽子加点料而已。''

无心想起白大千的手艺,心中暗道不妙。哪知未等他开口,公司门外响起一阵爽朗的谈笑之声,正是白大千外出归来了。

白大千带着一身寒气和一沓钞票,眉飞色舞的和丁思汉打了招呼。然而三言两语的交谈过后,他傻了眼。

``我?''他吓得快要站不住,扶着写字台坐到了沙发椅上:``我不行吧?我\ldots{}\ldots{}我最近身体不大好,精神也不大好,见了太恐怖的小动物,会害怕的。''

此言一出,丁思汉不禁愣了一愣,不知道白大千是大智若愚,还是大愚弱智。

``白大师。''他不客气了,加重了语气说道:``一张纸符而已,凭着你的修为,贴张纸符总是不成问题。''

白大千暗暗的捂了肚子,感觉自己的肠子在咕噜噜作响:``纸符?丁老兄,实不相瞒,纸符这东西,我公司里有的是,各种图案一应俱全,每张纸符的成本只有几分钱。你让我拿着几分钱的东西去收拾妖怪,未免太强人所难了。''

丁思汉叼着香烟一耸肩膀:``白老弟如果不肯合作的话,就别怪老哥哥我翻脸无情啰!''

话音落下,他从胸前的小口袋里摸出一只小小的黄色纸鹤。手指夹着香烟烧灼了纸鹤的脑袋,一股青烟袅袅而上,不过片刻的工夫,一只面青唇红的吊死鬼凌空现了形。

白大千吓得瘫在了沙发椅上,裤裆之中隐隐有了湿意:``无心,怎么回事?救命啊!''

无心也急于捉住怪婴交差,所以此刻眼望窗外,装听不见。

在吊死鬼的注视下,白大千同意充当诱饵捕捉怪婴。丁思汉松了一口气,想让丁丁回到自己身边。不料出门一瞧,他发现丁丁和史丹凤一起没了。

如此又过了一个多小时,丁丁和史丹凤才施施然的回了公司。丁丁神色如常,史丹凤却是垂头丧气。原来她对丁丁很有好感,陪着丁丁下楼散步。可是两人相谈不久之后,丁丁似乎是看出了史丹凤对自己存有几分爱慕之心,竟像一只公孔雀一般,浪头浪脑的一边耍帅,一边开了黄腔,表示自己愿意屈尊和史丹凤来一场一夜情。史丹凤本来看他是一尊美男子的标准像,没想到他其实是个绣花枕头,一肚子乌七八糟的野草。满腔的爱意付诸臭水沟,她感觉自己是受了侮辱,一路上强忍着没有翻脸。回到公司迎面见了无心,她不动声色的做了个深呼吸,忽然感觉无心好纯洁。

众人在办公室里消磨了半天的光阴。及至晚上写字楼内的大小公司都下班了,无心也下楼去买了一摞盒饭充当晚餐。白大千打开自己的盒饭,发现里面额外加了一根火腿肠和一只荷包蛋,和旁人的晚餐相比,丰盛的如同断头饭一般,不禁泪如涌泉。

慢吞吞的吃下最后一口饭,他从丁思汉手中得到了一张符。符不知是由什么材料制的,脏兮兮的又薄又韧,上面印着古怪花纹。

``白老弟,你放心,我不会拿你的性命开玩笑。''丁思汉压低声音说道:``我和无心会埋伏在附近保护你。你只要想办法把它贴到小崽子的脸上就可以,记住,一定要把它的五官全部盖住。''

白大千抖了抖手中的纸符:``丁老兄,你这是什么纸做的?手感还挺好,成本不低吧?''

丁思汉小声答道:``其实\ldots{}\ldots{}它不是纸,是人皮。小崽子生身母亲的皮。''

白大千听闻此言,``哇''的一声,弯腰把一肚子盒饭全呕出来了。

漱了漱口擦了擦脸,白大千一个人站在了三楼走廊里。保安已经刚刚巡逻过了这一层,走廊内的大部分电灯也都熄灭了。白大千背靠了墙壁,右手用拇指和食指的指尖捏了人皮符。他希望怪婴不要出现,一旦出现了,还得劳烦自己行凶。他活了四五十岁,一直是连只鸡都不敢亲自杀的。可怪婴若是不出现,他明天晚上恐怕还是得站在走廊里值更。总而言之,不是短痛,就是长痛。

正在他心惊胆战的浮想之际,他的头顶忽然受了轻轻一击。心脏猛然一个大跳,他慢慢的仰起了头。

在他的上方墙壁上,他看到了大头朝下的怪婴。

怪婴先前不知是藏到了哪里,如今一身淋淋沥沥的臭水,头顶还粘着一片烂菜叶。对着白大千一咧嘴,它露出了上下四颗尖锐的小獠牙。喉咙里叽叽咕咕的响了一阵,它抬起小手拍了拍墙壁,紧接着张大嘴巴,``吧''的叫了一声。

一声过后,它小小的胸腔里传出一阵颤抖的怪笑。向下爬了一步,它的腥红眼睛发出光芒,小走兽似的弓起后背,它作势要往白大千的怀里跳。而白大千是无论如何都不肯触碰它的,趁着它的力量引而未发,白大千一闭眼睛一咬牙,抬手扬起人皮符,没头没脑的向上便是一拍。拍过之后睁了眼,他借着走廊里黯淡的灯光,发现自己把符拍歪了!

怪婴的眼睛和鼻子全被人皮符盖了住。那符像有黏性一般,立刻和它的面孔溶为一体,撕不开扯不下。一刹那的愣怔过后,墙壁上黑影一闪,怪婴发出一声啼哭般的尖叫,纵身扑向了白大千的头脸。白大千毫无还手之力,当即摔了个仰面朝天。随即咽喉一凉,正是怪婴趴在他的胸前,已经张嘴咬上了他的喉头。

白大千吓得彻底痴傻了,流着眼泪预备等死。然而怪婴的尖牙轻轻点在他的皮肤上,却是始终不肯用力刺入。两只小手愤怒的抓挠着他的衣襟,怪婴的身体颤抖成了一块腐臭的嫩肉。

正在这时,无心和丁思汉从怪婴的身后包抄而来。从他们的角度望过去,白大千的手法堪称完美——他们并没有发现白大千把符贴歪了。

丁思汉弯着腰伸着手,姿势类似在捉鸡。无心起初落后了一步,此时加快速度,想要和他齐头并进。可是未等他们靠近白大千,怪婴忽然抬头转身,一瞬间凌空而起,直扑向了无心。无心见势不妙,迎头飞出一脚把怪婴当成了球踢。而怪婴当即顺着力道横飞,竟是撞到了丁思汉的怀里。丁思汉抱住怪婴,先是暗喜,然而低头一看,他清清楚楚的看到了怪婴漆黑的口腔与四颗扯着黑涎的尖牙。大惊之下他一松手,想要把怪婴远远扔开,可是怪婴把头一扭,已经咬中了他的右手手掌!

丁思汉哀鸣一声,然而左手托着怪婴却又不肯放了:``无心,快!''

无心会意,迅速转到丁思汉面前,用手去扒怪婴的嘴。怪婴的小身体里像是藏了一条蠕虫,顶着它的皮囊挣扎扭动不止。丁思汉的右手刚刚得了自由,立刻从裤兜里摸出一把小刀,只听噗噗两声,他用刀尖扎向了怪婴的眼珠位置。腥红汁水从创口中喷射而出,洒上了蒙面的人皮符。丁思汉把怪婴交给了无心,自己则是腾出左手抓了满手红汁,飞快涂抹了已经泛青僵直的右手。

怪婴周身的液体都是黑的,唯有眼珠含了两泡红血。丁思汉垂着血淋淋的右手,低声骂道:``妈的,养不熟的东西,敢反噬我!''

无心看了他的举动,料想他不会有生命危险,便开口问道:``你杀了它?''

丁思汉沉着脸答道:``它施的毒,只有它的命能解。''

无心低头再看怪婴,见它上半张脸都被红血浸染透了,四肢却还在微弱的抽搐着。两只小拳头攥紧了,它一只小脚往外蹬着,另一条残腿蜷缩向上。一张嘴越长越大,最后它从喉咙里发出了一声悲怆的啼哭,非常稚嫩,非常凄凉。

无心看了它的反应,忽然怀疑它是有思想的。

正当此时,丁思汉摇晃着依靠了墙壁,身不由己的缓缓坐下,右手在电灯的照耀下血光闪烁。忽然打了个极大的冷战,他抬起左手抓住了无心的裤管。手指用力使劲的拧绞了一下。

无心蹲下了身,把濒死的怪婴放在了地上,同时问道:``你怎么了?''

丁思汉闭上眼睛,摇了摇头。

\chapter{迷茫}

无心用白大千扔在办公室里的一件旧羽绒服包裹了怪婴。怪婴已经不动了,小小的胳膊腿儿也有了僵硬的趋势,显然是它作为小妖怪的一生,已经走到尽头了。

白大千捂着脖子爬起了身,在确定自己是安然无恙之后,他和无心合力,把丁思汉搀扶到了四楼家中。进门之时,史丹凤正在陪着史高飞看电视,丁丁独自坐在客厅里的小板凳上,垂着头自得其乐的玩手机。忽见丁思汉被人搀进门了,他连忙起身问道:``阿爸,你怎么了?''

丁思汉紧咬牙关,不说话只摇头。客厅里连张像样的椅子都没有,无心只好让他席地而坐。史丹凤出了卧室,见丁思汉一手鲜血,吓了一跳:``哟,我有云南白药,你们用不用?''

丁思汉把个脑袋摇成了拨浪鼓,然后也不要人照顾,自顾自的倚靠在了墙壁上。闭着眼睛喘息良久,他忽然低声说了一句:``不要怕,我死不了。''

丁丁人高马大的蹲在一边,此刻吓得嘴唇都白了:``你是不是被你养的怪东西咬伤了?阿爸你真是个老糊涂,我早就说让你改行,你偏不听!万一哪天被鬼吃了,也是你活该!''

丁思汉听惯了养子的妙语,故而根本不生气:``不会的\ldots{}\ldots{}''他气若游丝的说道:``阿爸不会的\ldots{}\ldots{}''

丁丁把双手搭在膝盖上,像一只英俊的大猴子,满脸的恨铁不成钢:``什么会不会的,好像你能说了算似的!你如果遭殃了,还不是要拖累我?你不为你自己着想,也该为我想一想呀!''

丁思汉虽然早就看透了养子的本质,可是此刻听了他□裸的心声,还是第无数次的寒了心。然而寒心归寒心,寒心也没办法。身体僵硬的瘫在角落里,他只感觉五内俱焚,自己使用了五六十年的身体忽然变得陌生笨拙了,他仿佛变成了孤魂野鬼,暂时藏匿在一具无主的躯壳之中。

丁思汉闭着眼睛,足足养了一个小时。右手的鲜血没有干涸,而是缓缓渗入了皮肤纹理之中。最后他扶着丁丁站起了身:``无心,把它给我,我要走了。''

无心把包着怪婴的羽绒服包袱给了丁丁。白大千追问了一句:``我说\ldots{}\ldots{}以后我们是不是算两清了?''

丁思汉没言语,拖着两条腿往外走。白大千眼看他出了门,心中猛的一阵轻松,精神也有了,扯着大嗓门叫道:``无心,去,下楼给丁老先生叫辆出租车。''

无心果然出了门。不出片刻的工夫,他顶着一头小雪花回了来:``白叔叔,他们上车走了。''

白大千越想越喜,感觉自己是度过了人生一大关:``好啊,现在天下太平,我也可以把佳琪接回——''

话说到这里,他心中一动,忍不住往史高飞的卧室里瞥了一眼。佳琪若是回了家,必定又要和姓史的小子狗扯羊皮。白大千虽然在理智上也知道自家女儿有些问题,可是理智往往退居二线,慈父的思想占了上风,他认为女儿的迟钝和笨拙叫做``敦厚有福''。敦厚有福的女儿不是一般小子可以消受得起的,所以他已经做好了养女儿一辈子的准备。

白大千闭了嘴,不知道要不要立刻把女儿接回家。不过自己除了一块心病,明天无论如何都要进城去看女儿一眼。掏出手机翻了翻日历,他发现明天乃是周六,进城的人潮必定十分汹涌,自己须得提早出发才能抢到出租车。思及至此,他忙忙的洗漱了一番,回房睡觉去了。

客厅里关了灯,史丹凤也自去休息了。无心仔仔细细的洗净了手脸,然后回到卧室说道:``爸,别看了,我想睡觉。''

史高飞兴致勃勃的盯着屏幕:``再等一会儿,快演完了。''

无心自己铺开棉被,钻进了被窝里躺下:``我累死了,你还用电视吵我。明天再看不行吗?''

史高飞不假思索的答道:``宝宝别闹,今天晚上是大结局,让爸爸把它看完。''

电视里不是嚎啕大哭就是吱哇乱叫,烦得无心躺不住。一掀被子坐起身,因为史高飞对他素来是百依百顺,所以他得寸进尺的有了一点小脾气。一脚蹬向史高飞的后背,他气冲冲的大耍威风:``明天还有重播!''

史高飞猝不及防的挨了一脚,登时向前一仆。以手撑地坐稳了,他直起腰自己思索:``儿子打老子,这不对吧?''

思索很快有了结果,他回身揪住不孝子的一条光手臂,把无心摁在床上打了一顿屁股。一阵响亮的噼里啪啦过后,无心提起裤衩起身便逃,一溜烟的穿过客厅,逃进了史丹凤的卧室。

史丹凤还在回忆着白天丁丁的一言一行,越想越是睡不着觉。开门把无心放了进来,她小声问道:``大半夜的不睡觉,你们又闹什么呢?''

无心脱了拖鞋,一步跳上了床垫:``爸打我。''

史丹凤想起了弟弟的大手大脚大力气,登时担了心:``怎么打的?打你哪儿了?''

无心背对着史丹凤,一脱裤衩一撅屁股:``打我这儿了!''

史丹凤冷不丁的看了个清,下意识的厉声喝道:``你给我穿上!''

无心被她这一嗓子震得一哆嗦,立刻就把裤衩又提上了。

史丹凤自从察觉到了自己的狼化趋势开始,对于异性的一举一动便都留了意。此刻望着嬉皮笑脸的无心,她忽然感觉这个家伙有色诱自己之嫌。

``不回去啦?''她问无心。

无心钻进了被窝里,又将她摞起来的两个枕头并排放好:``不回去了,回去要挨打的。姐,快来睡觉啊!''

史丹凤扭头望向窗外,窗帘很薄,可以看到天边一轮圆月,以及月光下高高矮矮的楼房与脚手架。这幅荒凉风景触动了她的神经,让她脑海中莫名的浮现出了一只对月长嗥的大灰狼。

关灯上床躺到了无心身边,她知道无心又在睁着大眼睛凝视自己。有心翻身把他抛到脑后,可是在她翻身之前,一只手忽然轻轻扳了她的肩膀:``姐。''

史丹凤扭头看他,看他对着自己的胸脯微微垂下头,一脸认真的说道:``摸一下。''

然后那只手便自作主张的移下去了,带了一点好奇和莽撞,抓她一下,揉她一下。忽然停了手,他抬起头小声说道:``我喜欢姐。''

史丹凤迎着他的目光问道:``有多喜欢?''

无心探头枕上了她的肩膀,一粒一粒去解她的睡衣纽扣:``我想和姐结婚。''

史丹凤叹了口气:``结婚是男人女人的事情,你连人都不是,又怎么能——''

这句话没能说完,因为无心扭过脸注视了她:``姐,我和人是一样的。人懂的,我都懂;人能做的,我也都能做。''

钻出被窝站起了身,他望着史丹凤的眼睛说道:``姐,你看看我。''

他穿得简单,脱了汗衫裤衩之后便是□。高高的站在床垫上,他在月光中面对了欠身而起的史丹凤。静静的站了片刻,他转过身,又给了她一个清清楚楚的背影。

最后跪坐回了史丹凤身边,他拉起了她的一只手,侧了脸往自己的肩膀上放。而史丹凤在一瞬间的失神过后,发现自己已经把无心搂到了怀中。

``好了,好了\ldots{}\ldots{}''她轻声的说:``以后姐再也不提你的来历了,姐知道你是人。''

无心歪着脑袋偎在了史丹凤的颈窝里。短暂的沉默过后,他仰起脸,开始亲吻对方的耳根。

史丹凤没有躲闪,她想如果自己再拒绝的话,无心一定要伤心了。

她心一软,无心就狗胆包天了。

翌日清晨,史丹凤起了个绝早,自己溜到厨房里煮大米粥。手指摁下电饭锅的煮饭键,她盯着小星星似的电源指示灯发了呆。长达三十年的黄花大姑娘生涯已于昨夜彻底结束,结束就结束了,这倒没什么的,根本早就该结束了。处女又不是专家和中医,总不会越老越值钱。问题是她现在有些糊涂,不知道自己对于无心到底有着什么样的感情。

她总觉得自己不能爱上无心——首先年龄上就不般配,其次,共同语言也没说出过几句。守着电饭锅思来想去的,她承认自己对无心的确是有独占欲,怜爱之情也不缺少,偶尔还想化身为母狼吃了他。但这就算是爱情了吗?

大米粥都熟了,史丹凤还没想明白。史高飞推门出来了,似乎是刚刚意识到自己昨夜打跑了儿子,此刻慌里慌张的往史丹凤屋里冲。随即白大千也露了面,兴致勃勃的准备进城看女儿。房子里的人气立刻兴旺了,史丹凤以卖煎饼果子为名,匆匆的躲着人出了门。

在白大千洗漱之际,史高飞正在摆弄儿子。无心也是刚醒,醒来之后伸手一摸,发现身边没人,不禁一愣。随即史高飞进来了,大呼小叫的宝宝长宝宝短。无心懒洋洋的不吭声,由着他又亲又抱。似睡非睡的闭着眼睛,他忽然一笑,害羞似的往史高飞怀里拱了拱:``爸,姐呢?''

史高飞想都不想:``不知道。''

无心认定自己是又要有家了,美滋滋的微笑不止。然而等他在客厅里和史丹凤见面了,史丹凤的态度却是平平淡淡,不但没有额外的高看他,甚至还带了一点不爱搭理他的意思。

无心有些傻眼,等到众人喝完了大米粥之后,他借着刷碗之机,跑到厨房里和史丹凤凑近乎。把厨房门一关,他小声问道:``姐,你怎么不高兴了?''

史丹凤还在翻来覆去的想着心事,越想越乱。多少年没有为情所困过了,没想到今天被它困了个走投无路。刚才在楼下被寒风一吹,她忽然发现其实无心比丁丁还要不靠谱。丁丁是个明摆着的草包,让人一览无余;而无心——无心在她面前像个小男孩,在白大千身边却又像个老油条。真不知道他那些本事都是从哪里学来的。

史丹凤忙着思考,无暇理睬旁人。而白大千早上出门,下午回了公司,正和无心相遇了。

无心知道白大千回来之后必定要到公司里玩一会儿电脑游戏,所以坐在办公室里守株待兔。好容易把白大千盼到眼前了,他扭扭捏捏的开口问道:``白叔叔,你说一个男人和一个女人,如果已经\ldots{}\ldots{}已经有关系了,是不是就算夫妻了?''

白大千听得一头雾水:``关系?什么关系?睡啦?''

无心点了点头:``对,睡了。''

白大千又问:``领证了吗?''

无心摇了摇头:``没有,只是睡了。''

白大千一扬眉毛:``那怎么能算夫妻呢?睡一觉就算夫妻——想讹人啊?''

无心满以为自己已经领会了新世界的精髓,直到听了白大千理直气壮的回答,才知道自己和这个时代依然格格不入:``那怎么才能算是夫妻呢?非得领结婚证吗?''

白大千脱了大衣挂上衣帽架:``哎哟,那可复杂了,首先男女双方得自愿吧?男女自愿了,两边家庭也得同意吧?然后彩礼,嫁妆,房子,车子\ldots{}\ldots{}太多了,麻烦着呢!一个环节出了错,兴许就能把一桩婚姻搅黄了。''

说到这里,他忽然来了精神:``你问这个干什么?你把谁给睡了?还是谁把你给睡了?''

无心慌忙摇头:``不是我,和我没关系。''

告别了白大千之后,无心悻悻的一个人下了楼。在楼前的小推车上买了一根糖葫芦,他找了个背风的角落里站了,开始唉声叹气的吃。

原来根本就没有大功告成,原来史丹凤依然是想不要他就可以不要他。无心低头吐出一粒山楂核,心里虚得很,脑筋也有些不够用,忽然又怨恨起了白琉璃——和白琉璃在一起生活久了,他简直快要活成白痴。先前积攒的那许多经验智慧也不知道全丢去了哪里,他一枚接一枚的吃着山楂,吃得大脑一片空白。末了手中只剩下一根又尖又长的竹签子,他没想出头绪,不想回家,于是蹲了下来,开始百无聊赖的用签子掘地面的冻土。

掘着掘着,他的眼前出现了一双大脚。仰起脸向上看,他看到了史高飞。

史高飞牢记自己昨天打了他一顿,所以十分愧疚不安。虽然不孝子的确是应该受教训的,可是教训的程度也分深浅,自己那大巴掌显然是抽得有些过火。弯下腰伸手捧了他的脸,史高飞向他咧嘴一笑:``乖宝,干什么呢?''

无心没想到他会找到自己,张了张嘴,一时不知应该如何回答。而史高飞随即看到了地面浅浅的小坑,当即脸色一正,手指地面悄声问道:``有宝藏啊?''

无心摇摇头,同时发现自己虽然有了一个爸爸,有了一个不知能否兼任妻子的姐姐,但还缺少一个朋友。他不明白史丹凤为什么会忽然冷淡了自己,她见过自己的真面目,一直照顾着自己的衣食住行,容许着自己对她的亲昵与亲热,可是亲近到了最顶峰,怎么却又冷了呢?

无心重新垂下头,在寒风之中呼出了一团苍白的雾气。

\chapter{承诺}

史丹凤无论如何想不出头绪,于是在前往农贸市场的路上,她顺便到银行查了查自己的秘密账户。拿着一张明细单站在街边,她饶有兴味的数着数目字后面的零,一时数了个如痴如醉。末了仰起头望着天,她忽然像通了任督二脉似的,豁然开朗的呼出了一口长气。

她有钱,有自由,有时间,还有了一个送货上门不包退换的俊俏小情郎。人生如此,夫复何求?何必还要钻着牛角尖自找不痛快?

史丹凤把自己劝解高兴了,一高兴,她斥巨资买了一扇排骨。拎着排骨往家走,她一边走一边掂量着排骨的价钱,掂量到了最后,她感觉手里的排骨不像是猪的,倒像是自己的,想一想都要心痛。

拎着排骨回了家,她没有见到无心和弟弟,只看到了白大千。白大千心情很好,又闲得无聊,几乎起了一点调戏妇女的余兴。看了史丹凤手里的排骨,他大加赞赏:``好,今天开荤了。''

史丹凤冻得脸红耳朵红:``今天的排骨钱不从伙食费里出,我请客。''

白大千挽了袖子:``不不不,你一个月能挣多少钱?要请也是我请。''

史丹凤脱了外面的短大衣,抬手把头发挽到了脑后,忍着心痛强装爽朗:``几斤排骨的客,就让我来请吧!白大师要请的话,得请我们去吃大餐才行。''

话音落下,大衣口袋里的手机响了。史丹凤掏出手机接了电话,那一头的说话人却是她妈赵秀芬。母女二人对起话来,先是一团和气,然而谈着谈着就变了味。几十分钟之后,史高飞带着无心回了来,进门之时正赶上史丹凤咆哮出了最后一句话。通过大开着的卧室房内,史丹凤把手机遥遥的掷向了自己的床垫,随即气冲冲的对着史高飞嚷道:``过年不回家了!反正我不回!''

然后她脸红脖子粗的拎起排骨进了厨房,叮叮咣咣的又切又剁。无心本来就是心虚,如今见识了史丹凤的雷霆之怒,越发吓得贴了墙壁,不知道她恨的是不是自己。

白大千本来预备着一肚皮的俏皮话,想要和史丹凤攀谈攀谈,如今也哑巴了。

当米饭和排骨全出了锅时,史丹凤恢复了平静。大概是感觉自己有必要对方才的震怒做一番解释,她一边给众人盛饭,一边牢牢骚骚的出了声。话没说出几句,史高飞插了嘴:``怎么?妈又要给你介绍对象啦?''

史丹凤答应了一声,忽见无心像个贼似的站在角落里,正在眼睁睁的盯着自己,便对他招了招手,让他把盛好的米饭端进客厅:``气死我了——是县里钢厂的工人,都四十三了,妈说四十三,肯定说的是周岁,也许是四十四五岁,带一男一女两个孩子,两个孩子全都是十几岁,一个读初中,一个读职高。那个男的好像是十年前死了老婆,之后找了好几个女的,全都没成,因为那男的爱喝酒,一喝醉就耍酒疯。''

史高飞听到这里,摇了摇头:``条件是不怎么样。''

史丹凤端着一碗米饭,拿着一把筷子出了厨房:``就这样的货,妈还当宝贝呢。让我把这边的工作辞了,赶紧回家相亲,要是相成了,我下半辈子就有依靠了。我呸!我宁吃仙桃一口,不吃烂杏一筐!我一个人活了三十年,也没活丢了一块肉!再说我为什么不好找对象?是我自己的错吗?我是奸懒馋滑还是嘴歪眼斜了?还不是因为——''

史丹凤并不打算迁怒于弟弟,所以及时把话打住了。装着排骨的精钢小锅摆在一张矮矮的简易小方桌上,四个人团团围坐,开始动了筷子。一阵似有似无的咀嚼声中,史丹凤忽然又来了一句:``妈说那男的还有个酒糟鼻子!''

史高飞神情漠然的啃着排骨:``别说了,怪恶心的。''

史丹凤忍不住:``妈骂我骂的可难听了,说我嫁不出去给她丢了人,还说我是存心要把她活活气死!''

白大千叹息一声:``恕我直言,这就是家长的不对了。为人父母的,不能为了自己一时的虚荣,拿儿女一生的幸福当儿戏。丹凤,我支持你。过年你留下来吧,我和佳琪给你作伴。''

史丹凤支吾着道了谢。没滋没味的吃了半碗饭,手机忽然又响了。

她吓了一跳,以为是她妈要和自己打持久战。不料起身走去拿了手机一看,屏幕上显示的却是丁丁的号码。接了电话一听,原来丁丁是特地要向她道一声别——丁思汉的身体状况始终是不见好转,所以他要带着他阿爸回云南去了。

史丹凤始终是不知道自己应该如何对待丁丁。高看他一眼?他不值得自己高看。不理他?他也没差劲到不值一理的程度。客客气气的闲聊了几句,她挂断电话转向小饭桌,忽然发现无心捧着满满一碗白米饭,面前桌上竟然一根骨头也没有。

弯腰拍了拍无心的脑袋,她惊讶的问道:``你怎么不吃呀?''

无心摇了摇头:``我不饿。''

史丹凤骤然紧张了:``不饿?是这一顿不饿,还是一整天都不饿?''随即伸手一摸无心的额头:``是不是病了?有没有感觉哪里不舒服?''

无心意外的回头看了她一眼,心想你到底是想不想和我好?一天没搭理我了,我凑到你眼前你都不看我;现在发现我不吃饭,却又一惊一乍的,仿佛是真害怕。

``我没生病。''他告诉史丹凤:``爸下午带我吃了汉堡。''

话音落下,他的后脑勺挨了史丹凤一巴掌:``你吓死我了!我告诉你我现在正闹心着呢,你别给我添乱!''

然后她直起了腰,嘴里嘀咕一句:``真烦人。''

无心糊里糊涂的,还是一头雾水。

无心不知道自己是不是真的很烦人,有心当面去问问史丹凤,可是史高飞父爱大发,缠着他不让他离开卧室。

一夜过后,无心讪讪的瞄着史丹凤,想要和她说话。可史丹凤忙忙碌碌洗洗涮涮,始终不闲着。他静静的等了良久,最后却是等来了一笔生意。

生意是白大千出面接洽的,客户是位小富豪,自称家里闹鬼,愿请白大师出手为他这个人类主持正义。白大千摆出大师派头,居高临下的细问详情。小富豪被他的气质所折服,毫无保留的吐露出了自己那点不得见人的家庭隐私。原来小富豪并非富二代一流,乃是靠着勤劳的双手和聪明的头脑致了富。致富之后一回首,他忽然发现家中的糟糠之妻十分糟糠,简直不堪入目。为了提高自己的生活质量,他分别在外面发展出了二奶小三等若干情人。可是温柔乡中的好日子没过多久,他后院起了火——糟糠之妻上吊自杀了。

``大师你到我家里看看吧,我总感觉不对劲。尤其是到了夜里,我在床上一闭眼睛,就能听见外面叮叮咣咣的有人做家务。你不知道,我那个死了的娘们儿有个毛病,每次和我吵完架,都气得不睡觉,在外面又擦又洗的胡折腾。自从我们家搬进小楼里了,她更来了劲,楼上楼下的乱走,一闹能闹一宿。''

白大千淡然一笑:``小问题,我可以先派个弟子去看你那里看一下。如果不是厉鬼恶灵的话,凭我弟子一人之力,便足能降妖除魔,保你家宅平安了。''

然后他抄起电话打往楼上,把差事推给了无心。无心干的就是这个生意,所以并不怕苦怕累。可在出发之前,他偷偷的问白大千道:``他能给我们多少钱?''

白大千不敢骗他,向他比了个巴掌:``五万。''

无心心里有了数。下楼上了小富豪的汽车,他直接奔了市区去。

小富豪住在一座园林式的社区之中,社区住宅一部分是二三层的小楼,另一部分是三十层左右的大厦。小富豪的汽车在自家小楼门前停了,无心一路无话,下车之后他原地转了个圈,然后说道:``这里风景很好。''

小富豪立刻炫耀:``是,值它的房价。''

无心指着小富豪的小楼问道:``买一座这样的房子,要多少钱?''

小富豪笑道:``我买的时候是一百二十万,房价涨得太快,现在得要多少钱,可不好说了——还不得超过两百万?''

无心暗暗的计算,发现自己捉一次鬼能从白大千手里分到两万块,两万块还是归史高飞所有。如果自己能和史高飞五五分账,那么想买一座好房子给史丹凤当家的话,至少要让自己不吃不喝的去捉两百次鬼。抬手摁了摁胸前口袋里的纸符,他想起被封在纸符里的骨神,心中几乎起了恶念,想要和对方联起手来为非作歹,好好的发一笔横财。

迈步进入楼内,他迎面看到了他的猎物。

他的猎物是个灰扑扑的鬼影子,藏在一株翠绿的发财树后面,面无表情的盯着他和小富豪。无心若无其事的向前走,同时盘算着如何故弄玄虚,让小富豪看出自己是真有本事和真卖力气。然而上到二楼时,他忽然嗅到了一股子新的鬼气。

停住脚步望着前方,他背对着小富豪问道:``你有几个老婆?''

小富豪被他问愣了:``一夫一妻——一个啊!''

无心不问了,在二楼各房中来回的又走了一圈。末了停在楼梯口,他望着楼下的发财树说道:``不是一个。''

小富豪傻了眼:``我这可是新房子,难道除了我老婆之外,还背着我死过别人?''

无心神情凝重的叹了口气:``不好说。给我一夜的时间,我要再看一看。''

小富豪把家里的值钱什物都收藏好了,然后带着保姆撤离小楼,住到了附近一家宾馆里面。小楼大门一关,无心坐到了客厅的角落里。发财树后的女鬼形容枯槁,是个典型的郁闷黄脸婆形象,必定就是吊死了的前妻。无心感觉这位前妻身上并无凶气,似乎没有夜里作祟的道理。而那前妻可怜巴巴的扭头望了他一眼,似乎是看出他是个与众不同的,所以有了一点求援的意思。

无心一动不动,只作不见。窗外天光渐暗,转眼间过了傍晚,屋里屋外全黑成了一片。无心正是坐得昏昏欲睡,不料身边忽然有光一闪。他扭头望去,只见一只女鬼探头探脑的穿墙而来。双方四目相对,女鬼登时乐了:``哟,你不是白大师的徒弟吗?几日不见,还是这么帅啊!''

无心认出了她:``你是丁思汉的——的——鬼奴?''

女鬼扬着电视机似的大方脸,摇曳多姿的对着无心一甩手:``什么鬼奴,这么难听!我是出于崇拜才自愿追随了主人他老人家。哎,帅哥,你家白大师有没有意向收些非人类做弟子?我这一阵子对他也有些崇拜。''

无心立刻摇了头:``我们师父很挑剔的,一般的人类都不肯收呢,何况你这非人类了。不过你家主人不是要回云南了吗?怎么你没跟着回去?''

女鬼在他身边向下一沉,也摆了个抱膝而坐的姿势:``后天的机票,他老人家正在宾馆里哄丁丁少爷呢。丁丁少爷不想回云南,把主人的箱子摔了个底朝天,还把他老人家的小礼帽从十六楼扔出去了。''

无心深深的一点头:``于是你就趁机溜出来了?''

女鬼扬手对他作势一打:``什么趁机,这么难听。主人很信任我的,我可以随便出入。我告诉你啊,这间房子我这一个月是天天要来一趟的,有两个女鬼在这里打架,每夜一打,十分准时,比电视剧好看多了。''

话音落下,楼上飘飘忽忽的出现了一个新的鬼影。大凡鬼魂的面目,都是临死之时的模样。后来的鬼影显然是死得够惨,因为一张面孔血肉模糊,五官已经基本分辨不出,只能通过衣着身材来判断她的情况。无心见她披着一头浓密长发,身上穿的短裙上露肩膀下露大腿,虽然上半身沾染了大片的殷红血迹,可是双腿肌肤光滑饱满,可见她死时的年纪必定不大。缓缓飘到了发财树下,年轻女鬼扬起了头,额前几缕长发想必是被鲜血打湿了,湿漉漉的贴在她那张惨不忍睹的烂脸上。

对着树下的前妻扬起手,年轻女鬼作势挥下。这一巴掌若是打在人身上,兴许不会怎样,可是落在了鬼身上,结果就不同了。那前妻随着她的巴掌向旁一倒,口中哀哀的发出鬼哭,年轻女鬼抬起了脚,继续对她乱踢乱踹。

无心看了片刻,发现这是一场独角戏,前妻完全不还手,只是依依呀呀的哭,倒是年轻女鬼真卖力气,兴许是怨气太足的缘故,拳头过处,居然能够拂动树叶,树上挂着的几枚小铃铛也跟着发出了隐隐的响声。

这样的打戏,对于无心来讲,实在是毫无趣味可言。从暗处忽然起身走了过去,他开口说道:``两位,停一停,请问你们为什么要打架?''

年轻女鬼暂停了卷脚,转过身抬头去看无心:``你他妈的是谁?怎么会看得见我们?''

无心答道:``那个\ldots{}\ldots{}我也是鬼,偶然路过,看你们打得热闹,所以比较好奇,想要问问原因。''

年轻女鬼上下打量了他,随即疑惑道:``你是鬼?我怎么看你不像?''说着她撩起短裙飞出长腿,对着无心的脑袋就是一个回旋踢。无心立刻抱着脑袋一躲:``哎唷,好疼呀!''

年轻女鬼飘稳了,还是对着无心审视不止:``是我见识少还是怎么的?难道你这样的也是鬼?喏喏喏,你看你还有影子呢!''

无心双手合什对她一拜:``美女,别打了,我真是鬼。''然后他侧身对着角落里的方脸女鬼一指:``不信你问她。''

方脸女鬼没有动,扯着嗓子笑道:``哈哈哈,他是鬼,只不过与众不同了一点。''

年轻女鬼感觉对方二人很不正经,所以沉吟着不肯回应。这时树下的前妻爬起来了,轻声哭道:``是我对不起你,可我也一命偿一命了,你还想怎么样?''

无心立刻转向了前妻:``大姐,你说说吧,我看你一脸忠厚相,肯定诚实理智,不会胡说八道。''

此言一出,年轻女鬼当即骂道:``放你娘的狗臭屁!这个老×最他妈阴险狠毒了!''

前妻不理会她,开始自顾自的说起了话。原来

她不但看着像前妻,实际上也是真是前妻。她说自己苦熬苦挣的帮助丈夫发了财,丈夫在外面却被小狐狸精迷了心。小狐狸精杀到家里来逼着她自动离婚滚出去,她不肯,两人就厮打起来。当时小富豪出了远门,保姆也放假也回了乡,前妻一时失手,就把小狐狸精给打死了。

``她说我要是不离婚,她就要一直闹下去,饶不了我更饶不了我丈夫,我当时气急了,才动了刀子\ldots{}\ldots{}''

前妻说到这里,捂着脸做哭泣状。而旁边的小狐狸精高声怒骂道:``你把我的脸砍成了这个样子,你还敢说你是一时失手?后来你在厨房里把我大卸八块喂了狗,也都是一时失手不成?''

前妻哭道:``我只是不想给他惹麻烦。反正你已经死了,我拿我的命赔给你就是了。''

小狐狸精听到这里,继续高叫。无心大概弄清楚了来龙去脉,转身走到角落里问方脸女鬼:``我问完了,你想不想吃掉她们?想吃就去吃,不想吃我可要动手了。''

方脸女鬼一摆手:``我从来不吃丑鬼。''

无心刚想给自己放血,可是一转念,他从胸前的口袋里取出了一张纸符。``嚓''的一声撕开纸符,他的手中缓缓升起了一团昏黄的光芒。光团渐渐分化成了人体形状,朦朦胧胧的正是骨神。一脸倦容的睁开眼睛望向无心,他慢吞吞的盘起了腿。

无心向着发财树下的两名女鬼一指:``骨神,我请你吃夜宵。''

骨神一言不发的飘向发财树。只听树下两名女鬼一起惊呼出身,随即鬼影闪烁着消失了,骨神面无表情的转向方脸女鬼,眼睛却是忽然睁大了些许。

方脸女鬼欣喜的迎上前去:``米奇,好几年不见了,你还是这么高大威猛器宇轩昂!还记得我吗?我是玛丽莲啊!''

骨神向无心一歪身,低声解释道:``米奇,我的英文名。''随即坐正身体,面对了女鬼正色骂道:``滚你的蛋!我和你们势不两立!你去告诉丁思汉,让他等着受死吧!''

无心向方脸女鬼递了个眼色:``别告诉你家主人哦,他吹牛的。''

方脸女鬼挨了骂,但是丝毫不生气,还问骨神:``你是不是要追随白大师了?''

骨神嗤之以鼻:``我是自由职业者,没老板!''

然后他疏忽之间消失无踪,不知溜去了哪里。方脸女鬼悻悻的也想走,可是未等她一张方脸没入墙壁,无心忽然追上了她:``我说——我想问你一个问题。''

方脸女鬼的下半身嵌在墙壁里,只转过了上半身看他:``什么问题?''

无心思索着问道:``如果有一只男鬼\ldots{}\ldots{}爱上了你,那么他应该怎么追求你,你才能动心呢?''

方脸女鬼不假思索的答道:``我是颜控。如果男方够帅的话,我可以直接倒搭。''

无心下意识的抬手摸了摸自己的脸:``那要是他\ldots{}\ldots{}他不大帅呢?''

方脸女鬼想了一想:``那就得看他能否让我开心了。''

无心若有所思的点了点头,忽然感觉自己的确是有些太消极了。自己在史丹凤面前活得好像儿子一般——哪个女人愿意要这种男人呢?

一夜过后,无心出城回了郊区。白大千得知他已经从根本上解决了问题,便打扮好了,又跑去小富豪家中做了半天的法。不料下午刚刚回了公司,他便被无心截住了。

无心问他:``钱拿到了吗?''

白大千连连点头:``拿到了。''

无心又问:``是现金吗?''

白大千一摇脑袋:``打到公司账户里了。''

无心立刻推着他往外走:``给我两万块。''

白大千扛不住无心的纠缠,迫不得已跑了趟银行,取出两万块钱给了他,因怕账目不清,史高飞会饶不了自己,他又让无心写了一张收条。

无心拿着钱在外面野跑了一天。晚上天要擦黑的时候,他回家了。鬼头鬼脑的把史丹凤拉到卧室里,他从衣兜里掏出一只皮面小方盒子:``姐,送给你的礼物。''

史丹凤挽着袖子系着围裙,正在盘算晚餐内容。此刻接过小盒子打开一看,她登时笑了:``谢啦,还挺好看。''然后她仔细看了看外面的盒子:``这么个小东西,倒是配了个好盒子。我看这盒子都得比戒指贵。''

无心愣了愣,随即小声说道:``姐,戒指\ldots{}\ldots{}是真的。''

史丹凤抬头看了他一眼,随即低头又去看盒子里的钻戒。看完钻戒,她一抬头:``真的?你花了多少钱?证书和发票呢?''

无心解开羽绒服的拉链,从怀里摸出一只压扁了的纸袋:``都在里面。''

史丹凤打开纸袋,从里面摸出一沓子纸票。一张一张的看过了,她骤然尖叫一声,随即弯腰抄起用来扫床的塑料刷子,对着无心就是一击,同时嘴里连珠炮似的叫道:``疯啦?不过啦?要死啊?一万九买个戒指,你欠揍吧?''

无心被她打得很疼,抱着脑袋退到了角落里:``姐,我能挣钱\ldots{}\ldots{}''

史丹凤常年吝啬,如今骤然得了一枚货真价实的大钻戒,她只感觉面红耳赤,心怦怦跳,把一柄塑料刷子舞出了风:``你能挣钱?你能挣几个钱?你能挣钱了不起啦?''

无心被她打得抬不起头,只能结结巴巴的低声辩解:``姐\ldots{}\ldots{}我们\ldots{}\ldots{}结婚\ldots{}\ldots{}''

史丹凤立时停了刷子:``你说什么?''

无心直起了腰,怯生生的抬眼看她:``结婚戒指。''

史丹凤这才明白了他的心意。扔了刷子双手叉腰,她先是啼笑皆非的哼了一声,随即一吸鼻子,扭了脸对着墙壁说道:``去你的吧,还没有一条狗的岁数大呢,你知道什么叫结婚?''

说完这话,她从围裙口袋里掏出卫生纸飞快的一擦鼻子,又用手背匆匆一抹眼角。无心看了,试探着问道:``姐,你哭了?''

史丹凤没理他,低头把小盒子往纸袋里装。从来没有人对她这么好过,她受不了。把纸袋递向无心,她先是做了个深呼吸,然后含糊的说道:``明天去看看能不能退了它。以后不许这样乱花钱了。''

无心背过了手,低着头不肯接。

史丹凤弯腰把纸袋放在了地上,然后自顾自的出门去了厨房。屏着呼吸炒菜做饭,她眼里总像是含着一泡眼泪,非得控制再控制,一秒钟都不敢松懈。忙忙碌碌的把饭菜端进客厅了,她先呼唤了白大千和弟弟出来吃饭,然后一边解围裙一边进了自己的卧室。

进门之后,她第一眼就看到了站在角落里的无心——从她方才离去开始到现在,他连站立的姿势都没变过。

她心里疼了一下,伸手拽他的胳膊:``走,吃饭了。''

无心迟缓的抬起眼皮,看了她一眼,然后重新垂了头,又微微的一侧身,把胳膊从她手中抽了出来。

史丹凤的手在半空中停了一瞬,心里难过死了,后悔死了。活蹦乱跳的一个无心,被自己欺负成了什么样子?

``吃饭了。''她一把又抓住了他:``先把外面衣服脱了,再去洗洗手。乖啊。''

无心靠在角落里,紧闭着嘴摇了摇头。

史丹凤上前一步,停在了他的面前。仰起脸看着他,她抬手摸了摸他的脸,又扳了他的后脑勺,让他向下枕上自己的肩膀。

``不生气了\ldots{}\ldots{}''她拍着他的后背:``姐跟你闹着玩儿呢。知道你是好心\ldots{}\ldots{}不许生气了,男子汉不能跟女人耍小脾气。''

无心喃喃的说道:``我不知道该怎么办才好了\ldots{}\ldots{}我想对你好,想让你喜欢我。''

史丹凤下意识的搂了他的脖子,忽然发现自己不会谈情说爱——谈不出口也说不出口,只是很想咬他一下,让他疼一疼。

``戒指我收下了。''她在无心的耳边说道:``以后不许私自花钱了,听见没有?''

无心用面颊蹭了蹭她的肩膀:``嗯。''

史丹凤一下一下的抚摸着他的后脑勺:``我们的关系,现在还不能公开,不怕别的,怕小飞闹,对家里也不好交待。你乖乖的,反正我又不会跑了,我们天天在一起,不也和结婚是一样的?''

无心熬到如今,终于从她嘴里听到了一句准话。

\chapter{丢失}

无心想要养家糊口,想像老鹰似的把史丹凤和史高飞一起收到自己的羽翼下。可是鬼们仿佛窥透了他的心事,忽然一起奉公守法的老实了。他望穿秋水的等了一个礼拜,只等来了一笔看风水的生意。看风水自然也赚钱,但是所赚的钱归白大千一人独有。白大千抱着一本风水大全日夜研习,虽然水平可疑,可因他有个大名声,所以出场费是两千元起,上不封顶。

无心隐约记得自己仿佛也是会看风水的,不但会看风水,还会算命。拿了白大千的风水大全摊开了,他蹲在地上一个字一个字的读,读到最后抬起头,张大嘴巴打了个哈欠,顺势把方才学得的一点知识全哈出去了。

他把风水大全送回到了白大千房内,心想自己还是应该依靠先天的本事去捉鬼。问题在于他现在无鬼可捉。从厨房里切了一片绿皮红心甜萝卜,他一边吃,一边起了邪念。

他把骨神召唤到了面前,让他速去楼下的演艺公司闹鬼,要闹得大一点,别出人命就行。骨神听了,嗤之以鼻,根本不听他的话。无心在家闲了一个礼拜,本来就闲得心中烦躁,如今见了他的德行,不禁新仇旧恨涌上心头,想要把他打成魂飞魄散。

骨神不肯坐以待毙,但也不肯真逃,故意流星似的满屋乱窜。无心捏着半片萝卜咚咚的追他,追着追着脚下一绊,他顺着惯性凌空飞起,正好扑向了前方的史高飞。史高飞一把抱住了他:``宝宝,你乱跑什么?''

无心把萝卜塞进嘴里,无话可答。史高飞摸狗似的把他从头到脚摸了一遍,又很亲热的搂了搂他:``爸爸带你出去玩,好不好?''

无心含着萝卜答道:``不去,没心情。''

傍晚时分,公司结束了一天的营业。史丹凤先白大千一步上了楼,想要看看厨房里的青菜还够不够吃一顿。自从得了一枚钻戒之后,史丹凤开始闹起了心慌病。戴着钻戒怕丢了,摘了钻戒藏好了,依然怕丢了。钻戒的尺寸正合她的中指,戴着钻戒坐在前台,她先看看钻石,再照照镜子,末了只觉自己面貌衣着都黯淡,简直和钻戒不相配。

为了不辜负钻戒的光彩,她在检查过了青菜数量之后,走出厨房对史高飞说道:``明天礼拜六,我想去市区逛一天。我不能像个灰老鼠似的过新年。你怎么样我不管,我得给我自己添一身新衣服。你去不去?你爱去不去,反正无心得跟我走。便宜羊毛衫是不禁穿,我再给他买件新的,和旧的换着穿。''

史高飞本来已经过了进城的兴头,然而听闻史丹凤要给无心买衣服,立刻答道:``我去,我自己给他买,不用你。''

史丹凤看他紧张的异常:``我买怎么了?''

史高飞老实不客气的答道:``你总买处理品。''

史丹凤深吸了一口气,因为没法和弟弟一般见识,所以只好拉着无心,忍气吞声的进厨房了。

厨房的房门一关,自成一统。史丹凤站在水池边慢条斯理的洗菜,无心贴在她的身后,双臂环着她的腰,凑到她的耳边小声嘀咕:``姐,你夜里别锁门。等爸睡了,我去找你。''

史丹凤侧过了脸,压低声音答道:``你少闹了,万一让白大师撞见了,算是什么事情?''

无心把下巴搭上了她的肩膀:``那我干脆搬到你房里睡。''

史丹凤想了一想,也感觉出了苦恼为难,牢牢骚骚的低声说道:``小飞能同意吗?他不同意,谁敢惹他?唉,你说我是个什么命?该找对象的时候,一个合适的也没有;如今好容易有了,可倒好,还得由着弟弟先霸占。早知今日,当初我先住到村里去好了。反正是先到先得,谁刨出来的归谁。我要是抢了先机,现在也不用天天看小飞的脸色了。''

无心被她说得没了话。沉默片刻之后,他很不甘心的又道:``姐,摸一下。''

史丹凤本来专心做饭,没想理他。然而他毛手毛脚的紧粘着她,不是摸一下就是亲一下。导致最后她在往锅里倒油之前,忍不住回头一扳无心的脑袋,在他嘴唇上狠狠亲了一口。

然后在变成大灰狼之前,她强行把无心赶出了厨房。

午夜时分,无心蹑手蹑脚的出了卧室。史丹凤的房门果然没有锁,他轻轻巧巧的溜入房内,无声无息的滚上了床。随即史丹凤一掀棉被,瞬间把他裹进了芬芳温暖的黑暗之中。无心紧紧的搂住了她,手臂硬如铁箍,嘴唇和腰却是活得让人把握不住,几乎要在被窝里游拱成了一条龙。

一时事毕,史丹凤从被窝上方伸出了脑袋,满头满脸热汗涔涔。隐隐约约的心火全熄灭了,她仰面朝天的枕了无心的胳膊,先是默然无语,良久之后才轻声说了一句:``真好。''

无心笑了,史丹凤的评语让他十分自得。

翌日上午,史家姐弟和无心一起出了门。起初倒也逛得融洽,然而到了下午时分,史高飞在一家电子大世界门前停住了脚步,无论如何都要进去。史丹凤一路挑挑拣拣,还有许多该买的东西没买,实在是没有时间和他在电子大世界里穷耗。无可奈何之下,他们只好兵分两路。史高飞像牵驴一样,牵走了不情不愿的无心。

大世界是座五层楼,五楼乃是游戏机专场,十分热闹有趣,不买看看也是好的。史高飞带着无心在五楼混了一下午,等到下楼重见天日之时,天已经黑了。

市中心是不怕黑的,天色越暗,越显出霓虹的明亮。史高飞一手插在衣兜里,一手领着无心,攥得很紧,生怕他丢了。因为路上始终没有空出租车经过,所以他一边东张西望,一边慢慢的往前走。不知不觉的经过了我爱骡主题餐厅,他正打算站到路边专心等车,哪知未等他站稳,身后忽然有人捶了他一拳:``飞哥!''

史高飞回头一看,只见对方穿着西装打着领结,剃着干干净净的小平头,却是他的老乡兼老友李光明。李光明换了个规规矩矩的造型,颇有清水出芙蓉的效果,一张大脸也方得顺眼许多。对着史高飞和无心嘻嘻一笑,他冻得抱着肩膀乱跳:``你俩挺浪漫啊,大冷天的压马路。''

史高飞莫名其妙的看着他:``度假村开到市里来了?''

李光明笑道:``不是,我早离开度假村了。''说着他抬手一指身后一座五光十色的大门面:``看见没有?夜色撩人Club。我现在在里面当服务生,比当保安赚得多。飞哥,你俩没事的话,进去玩玩吧,里面全是美女——哦,你俩不需要。''

史高飞扭头问无心:``进去吗?''

无心不知道什么叫做Club,故而立刻点了头:``去。''

因为史一彪是靠经营夜店发的家,所以史高飞见惯不怪,对于此地毫无兴趣。带着无心在吧台上坐定了,他要了两杯甜甜的鸡尾酒。一杯推到无心面前,他一本正经的说道:``宝宝,慢慢喝,别喝醉了。''

无心捧着玻璃杯抿了一小口,随即抬头对他说道:``爸,下次也带姐来喝吧!''

史高飞被周遭的舞曲声音吵得心乱如麻,忍不住在高脚凳上左摇右晃:``她?有她在的话,咱们谁也别想花钱买酒。''

话音落下,一只手搭上了他的肩膀。他回头一瞧,见来者是名浓妆艳抹的女子。女子对他粲然一笑,眼睛一眯,蓝色眼线快要延伸到了太阳穴上:``帅哥,请我喝杯酒吧!''

史高飞坐怀不乱的答道:``一边呆着去!''

然后他转向无心,继续说话:``姐太小气了,我们不和她在一起玩。''

无心的感官十分发达,一只耳朵听着史高飞的话,另一只耳朵听到蓝眼线在后方骂道:``操!这个大傻×,真他妈没情趣!''

偷偷笑了一下,无心捧着高脚酒杯,低头伸了舌尖去舔鸡尾酒。舔着舔着,舞曲的节奏忽然越发激昂了,一名造型绚烂的长发青年跳上了后方舞池中央的台子,一手握着麦克风,另一只手蜷起中指和无名指,很来劲的向下东戳西戳,同时双腿半蹲,一边合着节奏向前走,一边高声叫道:``药!药!切克闹!黑喂狗!闹太套!''

歌手在台上叫,听众在台下叫,一边叫一边跳。无心回头翕动嘴唇,跟着歌手默念歌词。史高飞伸手扳过了他的脑袋,不让他往舞池里看:``难听死了,不要学。''

无心照例是很听话,捧了酒杯又要喝。而史高飞在吧台前坐得百无聊赖,便起身落地说道:``宝宝,爸爸去上厕所。你乖乖坐着,等爸爸回来带你回家。''

无心点了点头。而史高飞探头又亲了他一下,随即迈开大步去找卫生间了。

卫生间里人不少,是两拨人马正在对骂,颇有在小便池旁火拼的意思。史高飞等了半天,始终是进不去门。一转身忽然看到卫生间门口摆着一盆半死不活的万年青,他没犹豫,解开裤子瞄准万年青的大花盆,哗哗哗的尿了一大泡。尿过之后收回家伙,他高人一头的晃出走廊,直奔吧台。然而吧台前面只有一群女孩子在嘻嘻哈哈,无心竟是不见了。

史高飞的脑子里闹起了电闪雷鸣。一大步跨到了吧台前,他问调酒师:``我儿子呢?''

调酒师没听明白:``先生,你带小朋友进来了?''

史高飞慌乱的摇了头:``不是不是,他看起来不像小朋友,刚才还和我一起喝酒来着!''

调酒师一下子就懂了:``啊\ldots{}\ldots{}您的朋友吧?刚往舞池去了。''

史高飞原地向后转,一头冲进了不远处的舞池。狼入羊群似的打断了台上台下的歌舞,他翻江倒海乱推乱搡,扯过一个人看看不是,甩开了再扯下一个。在众人的惊叫声中,他大声喊道:``宝宝!宝宝你哪儿去了?你怎么没了?宝宝!''

夜店之内立刻乱了套,保安和服务生则是冲入舞池,七手八脚的去抓史高飞。史高飞无端的没了儿子,又见天上地下光芒闪烁,狰狞的面孔四面八方的包围了自己,全是一副要吃人的凶恶相。

他瞬间崩溃了,嘶吼着左奔右突,李光明闻讯而来,伸手想要抱他,结果被他撞了个四脚朝天。

混乱直持续了半个多小时,史高飞才被众保安们押进后方的办公室里去了。

李光明搜出了史高飞的手机,正要打电话通知史一彪,可是手指一乱摁错了按键,他一看手机屏幕,发现自己把电话打给了白大千。忽然想起史高飞和白大千也是有关系的,他将错就错,和白大千通了话。

白大千听说史高飞忽然在夜店里发了疯,吓得心一哆嗦。一边穿衣服下楼往市区赶,他一边又把电话打给了史丹凤。史丹凤提着大包小裹,正坐在回家的出租车上。接了白大千的电话之后,她直接让司机调了头。她不是第一次去收拾疯弟弟惹出的烂摊子了,所以颇为沉稳,并不十分惶恐。

可是通过夜店后门进了办公室后,她望着被人摁在椅子上的史高飞,不由得也愣了。史高飞怔怔的直了眼睛,嘴里轻不可闻的念念有词。史丹凤在他面前蹲下了,仰着脸去看他的眼睛:``小飞,姐来了。''

熟悉的声音似乎是唤醒了史高飞。他木然的眨了眨眼睛,然后哑着嗓子说道:``姐,宝宝丢了。''

史丹凤也跟着他眨了眨眼睛:``宝宝——丢了?''

史高飞骤然把嘴咧到了极致,呜呜噜噜的哭道:``我只是去撒了一泡尿\ldots{}\ldots{}我没有带上他\ldots{}\ldots{}他喜欢喝酒,我想让他留下喝酒,喝完酒好回家\ldots{}\ldots{}宝宝没有了\ldots{}\ldots{}''

李光明站在一旁,做了一番解释:``姐,他好像是说他自己出去上了趟厕所,上完厕所回来一瞧,发现他的\ldots{}\ldots{}人,没有了。再找也找不到了。''

李光明认为史丹凤是很理智的,必定能和自己做一番分析。没想到史丹凤勃然变色的起了身,嗓子都尖了:``小飞!''

把手里的大小袋子向下一掼,她又狠狠的一跺脚:``我说让你们跟着我走,你偏不听!那么大个人都能让你弄丢了,你还能干点儿什么?你去把他给我找回来!''

史高飞一跃而起,抹着眼泪往外冲:``我找他去!''

史丹凤忽然感觉不对,拎起袋子往外追:``你上哪儿找去?回来!现在轮不到你疯跑!你要是也丢了,我就没法活了!''

史家姐弟一前一后的跑出办公室,留下的经理和众保安面面相觑。紧接着他们也追了出去,追到一半刹住了,他们又怀疑自己追得没有意义。毕竟史高飞是很明显的精神有问题,而且方才他除了吓人之外,并没有造成其它的损失。对于这种人搞索赔,难度之大,实在令人望而生畏。

史丹凤在夜店门口拼命揪住了史高飞,并且正好遇上了刚下车的白大千。她手脚打着哆嗦,人快要吊在史高飞的身上:``白大师,无心丢了。''

白大千立刻白了脸——如果公司里没了无心,那他的财路也就基本走到了尽头。

``怎么会丢了?''他想不通:``是不是你们走差了路?''

史丹凤环顾着周遭的冰天雪地灯红酒绿,一颗心像是被油煎一样,恨不能也发疯撒泼的哭一场:``白大师你了解他,他不是不懂事的人,不会私自乱跑的。他\ldots{}\ldots{}他\ldots{}\ldots{}''

她两片嘴唇在寒风中颤抖得说不下去了,无心无端的来,自然也可能无端的去。她不敢深想,只感觉一颗心是被人生生的摘了去。

\chapter{追寻}

白大千和史丹凤左右夹攻的牵着史高飞,在市中心的商业区内整整走了一夜。清晨时分他们回了家,身体不但累透了,而且也冻透了。

白大千提议报警,然而史高飞立刻又有了发疯的征兆:``不行!他们会把宝宝抢走的!''

白大千已经大概掌握了史高飞的思维方式,所以顺着他的话头劝:``你不报警,无心也已经被人抢走了。你知道我最怕什么?我怕是有坏人绑了他去贩卖人体器官!要不然谁会平白无故的拐个大小伙子?丹凤你说呢?''

史丹凤的脑子里似乎结了冰,寒冷沉重的不能运转。她不肯把无心的来历告诉给白大千,抬眼去看史高飞,她希望弟弟也不要说。

史高飞抱着膝盖坐在自己的床垫上,闭目垂头一言不发。

屋子里忽然寂静了,静得让人要窒息。白大千走去打开了电视机,想要增添几分人声人气。电视里面正在播放早间新闻,某地警察破获了一个人贩团伙。史丹凤无意中撩了电视一眼,正看到一群丢了孩子的父母在撕心裂肺的哭。盯着屏幕愣了神,她一抽鼻子,也跟着落了泪。原先也知道丢了孩子的父母苦,可是直到此刻她才明白了苦有多苦,才明白了怎么会有爹娘砸锅卖铁找孩子,一找一辈子。

床垫上面乱七八糟,史高飞和无心的衣服混作一团,和从来不叠的被褥搅在一起。史丹凤用右手摸了摸左手中指上的钻戒,触感迟钝,进屋这么久了,手指还是麻木着的。天太冷了,这么冷的天,他跑到哪儿去了?

身体一栽,她坐在了史高飞身边。长长的深吸了一口气,她对着史高飞狠狠捶出一拳:``我打死你得了!''

史高飞被她捶得一晃。弯腰把额头抵在自己的膝盖上,他闷声闷气的说道:``我一定要找到他。''

史丹凤恨死他了。平时他再怎么混蛋,因为他有病,他不是故意的,所以她从来不怪他。但是今天,此刻,她恨死他了。恶狠狠的瞪了他一眼,她不屑于再和他说话。

白大千沉重的叹了一口气,嘴上不肯承认,但心里也知道无心丢的不对劲。

``报警吧!''他单问史丹凤:``让警察再去搜搜那家夜店也是好的,万一能查出点儿蛛丝马迹呢?我早就听说那一片夜店都挺乱,经常出事。''

史丹凤别无选择的点了头。手扶膝盖慢慢的起了身,她听见自己冻僵了的关节在吱嘎作响。第一次感觉自己老了,老胳膊老腿的,一压就垮,什么都承受不住了。

在史高飞魂游天外之际,史丹凤和白大千去了公安局。江口市是个大城市,人员流动性太大了,丢失个把人不算稀奇事。警察平平淡淡的给他们备了案,又平平淡淡的去了一趟夜色撩人Club,最后是平平淡淡的一无所获。

史丹凤没过过几天艳阳高照的好日子,但是也没经过大风大浪。离开公安局之后,她又去了一趟市中心。坐在步行街边的长椅上,她看面前往来的年轻小伙子,看着看着,便是泪眼婆娑。和无心朝夕相处的时候,没觉出他多重要,如今他骤然没了,她才想起了他所有的好处。不是一对爹娘生的,又不是一起长大的,他没了,她却能难受的没法活。于是她明白了:她爱他。

再不可思议,再莫名其妙,她也是爱他。哪怕他品种不明,还没有一条狗的岁数大。

史丹凤歇够了,起身继续沿着大街小巷走。一边走,她一边盘算着如何找人。或许可以在市电视台发一条寻人启事,赏格定的高一点,白大千如果不肯出钱,自己出也可以,也出得起。

她有钱。

史丹凤和白大千搭了伴,开始着手去登寻人启事。而史高飞独自在家,慢吞吞的收拾了床铺。把无心穿过的衣服一件一件叠好放在枕头上,他找出了自己的粉红色小书包。

将一件被无心穿薄了的汗衫放进书包底层,他又往书包里放了一瓶水,一包饼干,一只从家里带出来的数码相机。严丝合缝的拉好了书包拉链,他把佳琪给他绣的钱包揣进怀里,然后拎着书包走进了客厅。

穿上羽绒服,戴上鸭舌帽,他背起小书包,正要往门外走。然而在他伸手推门的一刹那间,他眼前忽然金光一现,一尊金身罗汉从天而降,不高不低的飘在了他的正前方。

史高飞不感兴趣的看了对方一眼,然后低了头继续去推门。骨神万没想到他会如此淡定,不禁大叫出声:``嗨!你去哪里?''

史高飞头也不回的答道:``我去找宝宝。''

骨神连忙后退,想要阻住史高飞的脚步:``慢!你到哪里去找?''

史高飞扶着门把手,红肿着眼睛扭头望向了他:``不知道。我没办法离开地球,要找也只能是在地球上找。''

骨神听了他的回答,有些傻眼:``地球\ldots{}\ldots{}很大的。''

史高飞伸手在他头上一挥,手臂穿过幻影,他明白了对方的身份:``你是鬼吗?''

骨神沉吟了:``我\ldots{}\ldots{}''

他有点不甘心承认自己是鬼,总认为自己如此金光万丈的现了身,史高飞虽然焦头烂额,但也至少应该小小的对自己顶礼膜拜一下。不料史高飞一脸肃杀,仿佛是根本懒得理他。

史高飞没有得到答案,对于答案也没有兴趣。垂下眼帘望着地面,他自顾自的继续说道:``我一个人在地球上生活了二十五年,我知道孤独的滋味,不能让我的儿子再尝一遍。现在我要出发了,我一定要把他找回来。鬼,再见。''

话音落下,他开了门就往外走。骨神万没想到他性子这么急,连忙一路飘着追到了走廊:``无心往南去了!''

史高飞本来正在锁门,听闻此言,骤然抬了头:``往南去了?''

骨神抬手往北一指:``昨夜他在酒吧里乱走,被人绑架了。绑匪是什么人我不知道,总之他们把他缠得好像木乃伊一样,装进皮箱里塞进了汽车。汽车开到城北的配货站,他们又把皮箱送上了一辆大货车。''

史高飞当即立起眉毛:``你怎么不早来告诉我?''

骨神也一拍大腿:``我没去过城北,我在城北迷路了!''

史高飞无暇再问,拔了钥匙撒腿就往楼下跑。骨神被丁思汉伤了元气,至今也没恢复力量。短暂的现形已经让他感觉出了疲惫。慢慢收拢金光消失在了半空中,他快马加鞭的追着史高飞下楼去了。

史高飞的脑筋虽然路数奇异,但是有着自成一派的体系。听闻骨神说无心是被人从配货站往南运走的,他在楼下路口的报刊亭里买了一张全市地图,一张全省地图,一张全国地图,以及一张世界地图。先是摊开全市地图和全省地图看了一分钟,随即他拦下了一辆出租车,坐进去后说道:``去配货站!''

司机回头问道:``哪个配货站?''

史高飞答道:``城北的。''

司机发动了出租车:``哟,那可远了。''

市区白天总是堵车,在史高飞的出租车一点一点向前蹭时,无心已经在皮箱里蜷缩了一夜半天。回想起昨夜往事,他肠子都悔青了。可他其实也没有错,他只是在史高飞上厕所的时候,好奇的溜到了舞池旁边看了看热闹。震耳欲聋的乐曲声中,频闪灯光把人的动作分解成了不连贯的画面。旁边忽然有一只手狠拽了他一把,他猝不及防的一歪身,直直的跌进了两名大汉的手里。

他下意识的叫了一声——也许是叫了,也许是没叫,因为他的耳中除了音乐的巨响之外,再无其它声音。然后仿佛只是在一瞬间,他被人捂着口鼻拖进了黑暗处的一扇小门。

未等他挣扎看清周围环境,气味刺鼻的厚胶布已经狠狠的压到了他的眼睛上。与此同时,一只大手将纱布塞进了他的鼻腔与口腔,动作是训练有素的快,一直把纱布推进了他的喉咙里。厚胶布一圈一圈的缠下去了,密不透风的封住了他的七窍。一双手缠胶布,另一双手扒他的衣服。除了这两双手之外,还有手。七手八脚来自四面八方,摁着他拗着他,简直快要捏碎了他的骨头。

他知道不好了,发了疯的又踢又打,直到双臂被厚胶布缠在了身体两侧,直到双腿也被缠成了一条长长的鱼尾巴。

胶布缠了不止一层,最外面又捆了几道绳子。最后那些大手把他抬进了箱子里。他蜷缩得好像回了娘胎。皮箱合拢上了暗锁。``咯噔''一声,他被那贴在层层厚胶布下面的耳朵动了一下,听得清清楚楚。

后面的事情,他就糊涂了。大虾米似的被禁锢在黑暗之中,他难受得像是落进了炼狱里。敢于这样炮制他的人,必定是知道他的底细。否则想要杀人直接杀就是了,何必还要活活的把人闷死——思及至此,无心心中忽然一亮:也许是自己无意中惹了人间的仇家,对方真的只是想闷死了自己再抛尸呢!

但他随即又暗暗的摇了头。自己刚回人间不久,哪里会有仇家?

知道他的秘密的人,再加上鬼,一个巴掌就能数清。白琉璃没有嫌疑,猫头鹰就算有嫌疑也没本事。只有丁思汉最值得怀疑,可真正的丁思汉和自己并没有深仇大恨,不至于要出手绑架自己,除非是\ldots{}\ldots{}

无心不愿再往深里想了。

身下一阵一阵的有颠簸,除了颠簸之外,他再感觉不到其它。人被封在厚胶布里,起初只是难受,后来竟是痛苦到了生不如死的程度。他是不怕黑暗的,即便是在地下深处也能生存,他怕的是束缚与憋闷。头顶抵住箱子一侧,他一动都不能动。想要大喊大叫,也是根本不可能。只有睡眠能让他得到暂时的轻松,然而处在与世隔绝的黑暗中,他的睡眠很快变成了断断续续的片段。扭曲着的四肢不会麻木,只是恒久的酸痛疲惫。不知道多久没有吃喝过了,他清楚的感觉到自己的血液在减少。

身体受苦,心里更苦。他想史高飞和史丹凤一定为自己急死了。史丹凤头脑清醒,倒还好些,可史高飞在太平岁月里还要疯头疯脑,如今自己突然没了,他会不会闹到天翻地覆?他要是发起了疯,可没人能治得了他。

渐渐的,无心连思考的力气都没有了。

他冻得通体冰凉,紧贴箱壁的皮肉已经从外向内结了冰。颠簸时断时续,停的时候越来越长。死心塌地的放松了身体,他此刻的感觉只有冷与痛。这一秒仿佛已经是难熬到了极点,哪知下一秒来势汹汹,铺天盖地的让他无处躲无处藏。在山里也没受过这样的罪,偏偏在受罪之前,老天特地让他过了一段蜜里调油的好日子,先把他养了个身娇肉贵。

在一个寒冷的白昼——他感觉应该是白昼,因为冷归冷,但是阳气旺盛,源源不断的从下向上升腾,骨神追上了他。

他的感官迟钝了,依稀感觉到了身边是有鬼魂萦绕,然而到底是谁,他分辨不出,只能依稀听到对方的声音:``无心,我是骨神,你还活着吗?''

无心想动一下给他看,可是胳膊腿儿全被缠了绑了,动弹不得。奋力的向上一抬头,他并没能真把头抬起来,但的确是微微的动了。

骨神看了他的反应,当即继续说道:``你现在是在一辆大货车上的集装箱里,你的周围全都是\ldots{}\ldots{}''他特地向上环顾了四周:``冻硬了的大鲑鱼。''

然后他向下沉入了装着无心的硬壳大皮箱:``我一直在追你,可惜方向感不大好,总是追丢。今天运气好,高速公路堵了车,我一共找了十里地长的大货车,终于找到了你。可惜我现在没有力量救你了,不过你不要怕,我马上就回去给你那个神头神脑的爸爸报信。''

话音落下,他调头便走。飘出老远之后他停在半空,发现自己又把方向搞错了,当即来了个向右转。

与此同时,史高飞抱着他的粉红小书包站在火车站售票大厅里,正在很不耐烦的和史丹凤通话:``姐,我昨天手机没电忘记充了,你找我又有什么事?''

史丹凤五天前得知弟弟离家出走,险些当场昏死,哆哆嗦嗦的拨通了弟弟的手机,然而话没说了三两句,电话便是自动断了。再重新拨号,那边已经自动关了机。如今她人在江口市郊的出租屋里,感觉自己真有要疯的可能性:``你跑到哪里去了?''

史高飞答道:``我在山东呢!''

史丹凤扯起了泼妇的调门:``山东哪里?!''

史高飞直接答道:``不知道!''

史丹凤在五天之内愁出了一嘴的火泡:``你赶紧给我回来!凭着你那个没头苍蝇的找法,你能找到个屁!''

史高飞对于他姐的一切意见都是不屑一顾:``姐你少管我!本来现在春运不好买票,我就够闹心了,你还跟着添乱!好了,不说了,拜拜!''

\chapter{百年情仇}

史高飞在骨神的指引下,走了无数冤枉路,同时花了无数冤枉钱去黄牛党手中买火车票。后来随着春节的临近,他实在是连黄牛党都抓不到了,只好换了交通工具,有什么车坐什么车。抱着他的小书包蜷在一辆黑大巴的行李舱里,他满面尘灰烟火色,从脏兮兮的羽绒服的领口里挑出细脖子,又瘦成了一只大刀螂。

骨神也很着急,并且第一次发现自己是个路盲。满载鲑鱼的集装箱大卡车的确是往南走的,然而往南的道路太多了,道路上的大货车也太多了。骨神终日飘来飘去,做鬼做了几十年,第一次比活着的时候还要忙。后来他疲惫至极,简直不想再管这档子破事,但是无心从丁思汉手中救过他一次,骨神扪心自问,感觉自己还是不能半路开溜。

在除夕这一天的上午,无心身下时有时无的颠簸终于彻底停止了。

他还清醒着,感觉自己是平地悬了空,耳朵也依稀听到了人的话语声,口音浓重,依稀是在抱怨天冷路滑。声音此起彼伏的,可见护送皮箱的人并非少数。

他还是冷,骨神很久没有出现过了,让他怀疑对方是跟丢了。跟丢了倒也罢了,横竖他只是一只无牵无挂的鬼,和无心没有太深的关系。无心惦念的是史高飞,因为骨神几次三番的告诉他史高飞到了这里、史高飞到了那里——史高飞越走越远,距离江口市已经有了千里之遥。

凭着史高飞对他的种种好处,他现在宁愿让史高飞无情的呆在家里。

身体时而向上升,时而向下沉,可见外界不是个平坦的地势。人声渐渐的停止了,忽然听到铿铿锵锵的几声响,紧接着他朦朦胧胧的感觉到了光明。上方有人含糊说道:``锁眼里面都结了霜。''

回应他的是个一团和气的男子声音:``今年冻雨下得太厉害了。''

无心的耳朵动不得了,甚至脑浆都已经结了冰。然而尚存的意识告诉他:回答的人是丁思汉!

丁思汉的小别墅,位于云贵交界处的山林中。说是别墅,其实不甚恰当,因为周遭尽是穷山恶水,距离最近的村庄也有几里地的路程。由于环境条件都不好,故而他只有在万不得已的时候,才会前来居住几日,譬如此刻。

坐在空荡荡的小客厅里,他把带着毛线手套的双手撂在了大腿上。南方的冬天越来越冷了,他此刻的衣着并不比在江口市时单薄。命令保镖抬起了大皮箱,他抬手向下一翻,跟了他好几年的保镖们心领神会,当即将大皮箱也向下一翻。箱中的白色人形``咕咚''一声砸在了地面瓷砖上,声音很响,堪称清越,因为人形是冻硬了的,重量与硬度都和一块石头差不多。

最外层的尼龙绳子是可以解开的,厚胶布层层的冻在一起,则是需要暖一阵子。丁思汉很有耐性的盯着地上人形,看他的表面渐渐凝出了一层薄霜。薄霜缓缓融化了,一名保镖开始试着去揭厚胶布。胶布缠得很整齐,一圈一圈的由下往上揭。揭完一层还有一层。一层一层的揭到最后,里面终于露出了皮肤颜色。

无心依然是一大块从里冻到外的冰砣,动是不能动了,感觉却是依然敏锐。厚胶布和他的头发眉毛粘成了一体,随着保镖的撕扯,他的脑袋在剧痛中变成了光溜溜的模样,甚至连睫毛都没能幸免。他疼极了,冻硬了的眼皮似睁非睁,眼珠滞涩的转来转去。未等他熬过头顶的疼,厚胶布揭到□,他又狠狠的疼了一下。

最后,他终于彻底的见了天日,从头到脚覆着一层黏黏的不干胶。一只眼睛的上下眼皮被粘住了,他睁大了另一只眼睛向上看,正遇到了丁思汉居高临下的俯视目光。

在双方相视的同时,保镖扯出了他口中鼻中的纱布。纱布冻在了咽喉鼻腔里面,保镖没轻没重的用力一扯,扯出的纱布表面粘了丝丝缕缕的粉色黏膜。无心疼极了,眼珠随着保镖的拉扯向外一努,随即``啊''的叫出了声。

丁思汉没言语,手扶着膝盖对他微微一笑。

无心不叫了,张着嘴巴直着眼睛往前看。看着看着,他慢慢的闭了嘴。喉结艰难的上下滑动了几下,他又张开嘴,用舌头推出了一块粉红色的血冰。

保镖显然是特别的尊敬丁思汉,不但恭恭敬敬的一口一个``先生'',而且言谈举止都是轻轻巧巧静悄悄的,仿佛是怕吓到先生。在丁思汉的命令下,他们用酒精擦净了无心身上的不干胶。天气再冷,温度也在零度之上。无心体内的冰一点一点融化了,而在他的身体彻底软化之前,小丁猫起了身,命令保镖把他拖进了地下室。

地下室像个水泥盒子,天花板吊着日光灯。进门之后迎面的墙壁前立了一根钢筋焊成的十字架。十字架上面长长短短的缠了铁链。无心被保镖摁倒十字架上绑好了,不但手脚被锁了铐子,甚至连脖子都被铁环箍在了十字架的上端。无心的另一只眼睛也睁开了,定定的望着丁思汉。丁思汉一手环在胸前,一手托着下巴。花白头发梳得很整齐,眼镜片后的眼睛也很亮。及至保镖把无心五花大绑的固定在十字架上了,他先是向外一挥手,随即对着无心一歪脑袋一扬眉毛,又笑了一下。

保镖退出去了,房门也关上了。丁思汉微微一点头,短短一叹息:``时光荏苒,无心。''

苍老的声音回荡在空空荡荡的地下室里,带着一点不怀好意的笑意。一切恐怖的预想都成了现实,无心垂死挣扎似的问他:``你是谁?''

丁思汉摇了摇头:``我不知道。''

然后他摊开了一只手,垂下眼皮望着掌心,语气幽幽的很温柔:``他中有我,我中有他。我们都不是纯粹的灵魂了,我不是我,他不是他。''

合拢五指抬眼向前,他清清楚楚的说道:``无心,你杀了真正的我。''

无心又疼又冷又渴又饿,各种痛苦一起发展到了极致。伸出舌头舔了舔枯萎的嘴唇,他的舌头刚刚脱了一层皮,一舔之下,给他的苍白嘴唇染了一层粉红颜色。

``我不是无缘无故的杀你。''他几乎是瘫在了铁链的束缚之中,声音也是有气无力:``我从不滥杀无辜。''

丁思汉对着无心摇了头:``不,我认为我很无辜。你当年竟然为了一个最平凡不过的女人杀我,你多么荒谬,我多么无辜。''

无心呼出了一口带着血腥味的凉气,静静的思索回忆了片刻。片刻之后他开了口:``不对,当初你杀了我爱的人。你看她平凡不过,我看她却是天下第一。你杀了我的天下第一,我找你报仇,没有错。''

丁思汉留意到了他方才的迟疑,于是忽然改换了话题:``无心,我是谁?''

无心抬起了头,头发眉毛睫毛全没有了,本应覆着毛发的皮肤呈现出了清晰的青色。虚弱的目光扫过了对方的面孔,他低声答道:``算你是丁思汉吧!''

丁思汉凝视着他:``你一定是忘了我的名字。百年光阴,天大地大,你有自由,我没有。我很寂寞,只能想你。和你相逢真是一件太不容易的事情,幸好我还没有太老,还有力气和你谈一谈上辈子的往事。''

话音落下,他抬起了自己的一只手,真正的丁思汉一生不干重活,所以一双手糙得有限,老得也有限。胸膛里活动着一股子不安分的力量,是真正的丁思汉要伺机造反。他活动了手指,一边体会着自己身体的灵活,一边在心中说道:``安分一点吧,老兄。你已经痛痛快快的活了几十年,现在也该轮到我了。''

``上辈子很糟糕。''他盯着自己的手指说道:``我只真正做了十四年的人,然后就是一百年的封禁。清清醒醒的一百年,难熬极了。一百年后我见了天日,不知变成了个什么邪祟,反正已经不能算人。所以我怕你,怕你的血。很喜欢你,可是不敢靠近你,就因为你流着一身可怕的血。''

话说到这里,他从裤兜里摸出了一把瑞士军刀。亮出刀锋走向无心,他抬起刀尖点上对方的眉心,虚虚的一路向下划。刀尖在咽喉处横着拐了弯,忽然斜斜的切进了皮肤。无心猛的一闭眼睛,颈部的血管已经被丁思汉割开了。

丁思汉一手依旧握着刀,另一只手则是狠狠挤压了他的伤口。血液都在路上熬干了,丁思汉只从翻开的伤口中挤出了几滴淡淡的凉血。把淌着鲜血的手背伸到无心眼前,他忽然神情欢愉的露齿一笑:``看看,现在我是人,我不怕它了。''

然后收回手送到嘴边,他伸出舌头舔了一口。舔过之后咂了咂嘴,他摇了摇头,依然是笑:``不好,不好,又甜又腥又涩。''扭头对着地面啐了一口唾沫,他双手扶着膝盖弯下腰,毫无预兆的笑出了声音。

无心看着他乐不可支的模样,知道自己是落到了任人宰割的境地。天下太平的日子过得太久了,他只记得自己曾经在很久很久之前被人当成妖怪放火烧过。火烧毕竟是场短暂的酷刑,虽然痛苦,但总能忍受;可是如今落入了老仇家的手里,恐怕自己的刑期就不只是``一阵子''那么简单了。

``你想怎么报复我?''他问丁思汉:``我死不了,不可能偿你上辈子的命。''

丁思汉没理会他,单是抬手抚摸了自己的脸,同时喃喃自语道:``奇妙,我还从来没有这样衰老过。我老人家,哈哈,我老人家。''

他调门很高的笑了几声,笑过之后抬起双手向后一拢头发,他对无心露出了整张面孔:``上辈子我是个小姑娘,对你有爱,也有恨。没办法,小姑娘嘛,免不了要喜欢男人。不过如今我是个老头子了,对你也没什么爱了,恨倒还是蛮恨。把你从北运到南,花了我很多的心思和工夫。现在应该怎么炮制你呢?你可以给我一点建议。''

无心始终是平静的,平静到了冷淡的程度:``把我剁碎了喂狗吧。''

丁思汉抬起腿,对他当胸踹出一脚:``去你的!我正计划要吃掉你呢,你是不是故意想要骂我?''

无心被他踹得一晃,脸上却是没什么表情:``老伯,你年纪大了,还是庄重一点为好。''

丁思汉愣了一下,随即阴阳怪气的又笑了:``无心,你是一句接一句的骂我啊!我恨死你了。''

正当此时,地下室的房门被人敲响了,有人隔着门板说道:``先生,小丁先生来了电话。''

丁思汉开门走了出去,从保镖手中接过卫星电话。电话中丁丁的声音怯生生的,试试探探的问道:``阿爸,你最近身体好些了吗?''

丁思汉沉了沉声音:``阿爸还好,你不必担心。''

电话那边的丁丁又小声说道:``阿爸,上次你突然对我发脾气,吓死我了。''

丁思汉仰起头,望着通往地面的狭窄楼梯:``阿爸心情不好,以后你要懂事。''

丁丁立刻答道:``我知道了。阿爸啊,你什么时候回昆明呢?我\ldots{}\ldots{}我一个人过新年,钱不大够用了。''

丁思汉冷淡的答道:``再等等吧,阿爸还有点事要做,最近大概都在山里。''

然后他挂断了电话。他对丁丁的关怀,完全是出于一种惯性,丁丁是自己另一半灵魂的宠儿,被宠了足足三十年。尽管现在的丁思汉并没有什么耐心拿他当大宝贝哄,不过若是突然铁面无情的翻了脸,似乎也不大合适。

把沉重的卫星电话扔给了保镖,他让保镖锁好地下室房门,随即自行踏上了楼梯。保镖锁了门后转过身,看到丁先生一步一步上得蹦蹦跳跳,要到楼梯尽头了,他忽然纵身一跃,``咚''的一声蹦上了地面,颠得花白头发一颤。

无心很绝望的委顿在十字架前,全凭双臂吊着身体。正是木然之际,地面向上悠悠的飘出了一张骨感大脸,却是女鬼玛丽莲。

无心和玛丽莲打了个照面,玛丽莲开了口:``不要客气,你忙你的。我前些日子听主人说你是个妖怪,十分好奇,今天特来瞻仰一番。''

无心半死不活的歪着脑袋,翻开了脖颈一侧的新鲜伤口。盯着女鬼看了一眼,他突然问道:``玛丽莲,你喜欢骨——米奇吗?''

玛丽莲爽朗的答道:``我对他一直是以暗恋为主。他那华丽的造型和不羁的性格,都深深的吸引了我。要不是他脾气过于火爆总想杀了主人,我非向他告白不可。''

无心没有力气点头了,只能闭了闭光秃秃的眼皮:``那我求你一件事,如果你在这附近看到了米奇的话,告诉他快带着我爸回家,千万不要过来救我。丁思汉出了问题,恐怕再见了米奇,会直接把他打散。''

玛丽莲一口答应,又对无心说道:``我们有过几面之缘,相处的也算愉快,能帮的忙我一定帮;况且我也不想让米奇散在主人手里。不过正如你所说,主人自打从北方回来之后,不知道为什么,脾气忽然变大了,嗓门也变高了,从早到晚总沉着脸,但也别有一番魅力,如果把头发染一染的话,倒是不失为一名魅惑狂狷的帅大叔。''

无心听她说话听得头疼,不想理会。然而玛丽莲谈兴正浓,将无心上下打量了,她又有了新话题:``哇,帅哥,你够瘦的!''伸手向着下方一指:``也够细的。''

无心枕着自己一侧肩膀,对着玛丽莲苦笑了一下:``我是饿的,我很久都没有吃过东西了。''

玛丽莲正要继续评价他的形象,可是话未出口,她骤然向下一沉,消失了个无影无踪。与此同时,丁思汉夹着一只大铝盒子,叮叮当当的走回来了。

\chapter{酷刑}

丁思汉蹲在无心身前,用一把银色的长柄小刀子轻轻蹭着他的小腿。无心的皮肤呈现出一种干燥的蜡白色,仿佛将要自行脱水风干,刀背摩擦着他的皮肤,感觉皮肤已经类似皮革。

用刀尖戳了戳关节清晰的膝盖骨,丁思汉抬头向上仰视了无心一眼,握着刀子的右手随即猛一用力,让刀锋斜斜的割开了小腿皮肤。瘦骨嶙峋的两条腿果然一起颤抖了,带出了一串脚踝铁链的铿锵声响。他不为所动的继续向下切割,艰难的滞涩的,像是切割一块坚韧的树皮,右手费了偌大的力气,也只用小刀子切下了薄薄的一小片。

一小片皮肉到了他的手里,半透明的带着弧度。而无心的小腿创面上只呈现出了淡淡的粉色,连一颗血珠子都没能渗出。

丁思汉捏着那一片皮肉起了身,在日光灯的光芒下反复的看。看到最后他``嗤''的一笑,转向无心问道:``感觉如何?''

无心仰靠在十字架上,一言不发的紧闭了双眼。丁思汉没有等待答案,于是随手把刀子丢进地上的大铝盒子里,然后伸手一捏无心的下巴,把手中的皮肉塞进了他的嘴里。

无心含着自己的皮肉,先是不动,后来他缓缓的活动牙关开始咀嚼,面无表情的自己吞咽了自己。

在此期间,丁思汉一直默默的凝视着他,花白头发凌乱的垂在额前,遮住了他的眼睛。

丁思汉很想吃了无心。

他认为自己早已超凡脱俗的不算了人,所以一贯认为吃活人不算什么。``食其肉寝其皮''之类恶狠狠的古话,对他来讲,也完全可以做到。对于不死的无心,他想不出哪种刑罚足够残酷。当然,杀人不成,可以诛心,问题是如今无心的心中好像空空荡荡,并没有什么牵肠挂肚的``天下第一''可以让他去杀去诛。

丁思汉没了办法。对于无心,无论是一百年前的``她'',还是此时此刻的``他'',都时常是无计可施。

夹着他的大铝盒子出了地下室,他站在别墅门口,去看远方叠嶂的山。冻雨连绵许久了,浓绿的草木全挂了水滴冰珠。畏寒似的把手揣进棉衣口袋里,他又掀起了棉衣后面的帽子戴好。帽子边缘镶着一圈人造毛,黑白混杂,像他的头发。一名保镖拿着一把兵工铲,正在专心致志的清除门前地面的冰。冰是半融化的,更像坚固的水,带着黏性,非常的滑。保镖是个黑黝黝的小个子,干活的动作十分利落。丁思汉望着身体前任主人给自己留下的家业和人马,不由得生出了一种坐享其成的得意。

几十年来他作为丁思汉的影子,一直只能做一名旁观者。旁观者有旁观者的好处,比如一旦有了机会,他可以即刻走马上任,毫无破绽的取代真正的丁思汉。

转身走回客厅,他让保镖去弄一些热糖水,喂给无心。

一名人高马大的保镖用大号的可乐瓶子装了满满一瓶糖水,进入地下室去喂无心。跟随老丁先生许多年了,保镖也修炼出了一脸不阴不阳的鬼气。举着可乐瓶子站在无心面前,保镖看无心像个饿极了的婴儿,眼睛都没有睁,完全是凭着直觉和本能一口叨住了瓶嘴。又因为无需换气,所以他咕咚咕咚的一味只是痛饮。糖水越来越少,瓶底越举越高。无心追着瓶嘴向下歪了脑袋,一瓶糖水喝光了,他还不肯松口。

保镖强行从他口中拔出了瓶嘴,塑料瓶嘴变了形,上下带着清清楚楚的两道牙印。向下一瞟无心的身体,他看到了无心微微隆起的圆肚皮。

无心的嘴唇受了糖水的滋润,隐隐透出了一层血色:``我还要。''

保镖没言语,拿着变了形的可乐瓶子上楼去见了丁思汉:``先生,他说他还要。''

丁思汉一点头:``给他,要多少给多少。''

保镖不肯轻易解开无心手脚的镣铐,于是只用面粉调成了面糊,填鸭似的一次次灌饱他。而在无心饥不择食的大喝特喝之时,史高飞已经梦游似的到了昆明。坐在一家小饭店里,他一边吃着滚烫的豆花米线,一边看着一份云南省地图。及至把米线吃光了,他起身出发去了长途汽车站。粉红色的小书包已经脏的不见了本来面目,印着的美羊羊图案也脱落成了花脸羊妖怪。抬手摁了摁头顶的厚绒棒球帽,棒球帽是他在路上为自己添置的,左右两边各支着一只三角猫耳朵,其中一只耳朵边缘绽了线,露出了一缕白色太空棉。风餐露宿的在外面跑了一个多月,他晒黑了,上嘴唇长出了一抹小胡子的雏形。警惕而仇恨的注视着面前来来往往的行人,他随时预备着和邪恶的地球人决一死战。

然而地球人见了他与众不同的形象,都纷纷绕着他走,连车站外面招揽旅馆生意的大妈和伺机行窃的小贼们都不敢招惹他。手里拿着几块刚出锅不久的夹沙荞糕,他坐上一辆长途汽车,一路吃得满手满脸全是豆沙。车上乘客几乎满员,唯独他身边空着一个座位。售票员喊破了嗓子,硬是没人敢和他并肩而坐。

几番辗转之后,在骨神的引领下,他到达了云贵交界处的昭通市。

骨神忙死了,忙得感觉自己简直不像了鬼。他的记忆力是好的,只是永远不辩东西南北,走了前路迷了后路。他忙昏了头,有时候对着史高飞长篇大论了许久之后,才发现自己没有现形,史高飞根本听不到自己的鬼话;又有时候他急匆匆的飘在路上,忽然把迎面行人吓得高叫一声昏死过去,原来是他忘记自己刚刚现了形,竟然光芒万丈的在大马路上公然飘了老远。

把史高飞引出昭通市区之后,他悬在一棵冷飕飕湿淋淋的老树下,又迷路了。

史高飞抱着热水袋站在一座小山包上,眯着眼睛眺望远方的苍翠群山。骨神远远的瞥了他一眼,发现他的目光和神情都很沧桑。

史高飞的身后,是一座小小的村落,村中的居民以汉人为主,余下的少数民族也早被汉化。骨神希望史高飞先回村中落脚,等到前途方向有眉目了再继续上路。然而史高飞抱着一只半热不冷的大水袋,很固执的向前走去了。

骨神别无选择,只好硬着头皮跟上了他。可是还未等他们走下小山包,路边树木的枝叶之中忽然吊下了一个女人头:``咦?米奇?你真的来了?''

骨神暂停在了半空中,因为一直看不上玛丽莲,所以很严肃的没有回应。

玛丽莲无论生死,永远不知道愁。骨神不理她归不理她,不影响她个人的热情。欢欢喜喜的移到了骨神近前,她快乐的笑道:``米奇,你是来找妖怪的吗?不要急着走,妖怪托我给你带句话。''

骨神很怀疑的审视着她,始终感觉她不是个正经鬼。

在玛丽莲和骨神交谈之时,丁思汉带着他的大铝盒子,又出现在了无心面前。

在狂饮了无数汤汤水水之后,无心的肌肤渐渐恢复了充盈饱满,被厚胶布撕扯掉的毛发也开始重新生长。丁思汉认为自己等待得够久了,如果再继续喂养无心的话,未免过于仁慈了。

把铝盒打开摆在水泥地上,盒子里放着七长八短的雪亮刀子。先前的丁思汉只害人,不吃人;所以他如今也只好避人耳目的开斋。当然,吃不是目的,他并不是馋嘴的人,让无心疼一疼,怕一怕,才是目的。

果然,无心真怕了。

他新生的两道眉毛非常黑,黑得几乎带了潮湿的水意。随着丁思汉的逼近,他的眉毛微微颤抖,微微凹陷的眼窝中,两只乌溜溜的大黑眼珠也是光芒闪烁。丁思汉注视着他的眼睛,忽然满心欢喜,兴奋得要叫要笑。甩手一刀扎进无心的面颊,他手腕一转,剜下了一块血淋淋的肉。无心疼得周身一起抽搐了,喷涌而出的血液却是稀薄淡红的颜色。刀尖扎着肉收到面前,丁思汉伸出舌头舔了一下,随即笑着一皱眉一扭头:``味道还是很不好。''

用固体酒精烧开了一小锅山泉水,丁思汉蹲下了身,将刀尖上的肉放到水中涮了涮。滚水之中浮出了薄薄一层血沫。肉却是粉红的没有变色。丁思汉对它吹了一口凉气,然后起身面对了无心,缓缓的张大嘴巴,用牙齿衔住了肉。

紧接着向后一仰头,他把肉从刀尖上咬了下去。上下牙关结结实实的合拢了,他盯着无心慢慢咀嚼。最后``咕噜''一声把肉咽了,他笑微微的告诉无心:``应该把你煮了吃,煮过之后,你是甜的。''

无心的一侧面颊陷下去了个血坑,隐隐露出了雪白的牙齿。定定的瞪着丁思汉,他的黑眼珠仿佛正在涣散洇染,染得白眼珠泛了蓝。忽然猛的向前一咬,他没能咬到丁思汉的手,但是咬住了丁思汉手中的刀。丁思汉很识相的立刻一松手。他松了手,无心也松了口。刀子掉落在水泥地上,刀身已经变了形。

丁思汉暗暗的心惊了,如果不是他躲得及时,也许他会被无心活活咬掉半只手掌。但是心惊之余,他又生出了一种别样的痛快。无心一定是疼极了,像他当年一样疼。有冤报冤有仇报仇的滋味真好,他一脚踢开废刀,弯腰掂起了一把新刀。挑选着无心身上的干净皮肉,他一边防备着无心的牙齿,一边好整以暇的下刀子。滚水除去了肉中的腥与涩,丁思汉慢条斯理的向无心描述着他的口感,同时看他的眼珠子越来越黑,看他被自己割成红白相间的身体抖得好像一片风中的叶子。

最后,他心满意足的剖开了无心的胸膛。用刀子向内拨弄着看了又看,他轻飘飘的说道:``你的里面,和人还是很不一样。''

无心紧闭双眼,挤出了一滴黏稠的眼泪。他疼极了,在刀尖的翻戳之下,他终于忍无可忍,战栗着发出了一声惨叫。

丁思汉的动作在他的惨叫声中停了一下。抬眼望向他,丁思汉冷静的说道:``我还以为你转了性,要在我面前充硬汉。叫吧,早该叫了。上辈子我死前也叫过,撕心裂肺,不是假的。''

话音落下,无心却是安静了。

无心一直安静,一言不发,于是丁思汉收拾了器具,转身离去。

无心站在自己的血泊中,不麻木不昏迷,周身始终是在针扎火燎的疼。地下室里的空气温暖甜腥,是他的余味。

一场酷刑过后,他极力的想要给自己一点安慰,想要用一点美好的回忆来哄自己开心,可在剧痛之中回首往事,他所珍惜所渴望的尘世间的一切,忽然和他有了十万八千里的距离,甚至在他的脑海中,连史高飞的面孔都模糊了。

他的手臂在铁链之中微微的动,全身的骨骼一起作痛做痒,他想狂奔,他想杀生。

一夜过后,他周身斑斓的伤口分别覆了一层粉红薄膜。薄膜一生,痛楚随之减了些许。可丁思汉又出现了,先是用刀子在他脸上纵横交错的乱画了一气,然后笑眯眯的阉了他。

无心成了丁思汉最爱的玩具,横竖不会死,正好可以由着他随便玩。一天傍晚他进了地下室,迎面几乎被无心吓了一跳。无心的脸上生满了七长八短的白毛,每一根都出自正在愈合中的粉红伤口。抬眼望着丁思汉,他诡异的面孔上没有表情,眼珠却是特别的大和亮。

丁思汉忽然嗅到了一丝危险气息,并且感觉他变得不大像人了。没敢贸然的再折磨他,丁思汉只是命令保镖给他的手脚加了一道铁铐。

及至丁思汉离去之后,无心侧过了脸,开始去咬缠在臂膀上的铁链。在一盏日光灯的照耀下,他瞎了似的大睁着眼睛,无知无觉的单只是咬。

不知过了多久,丁思汉又来了,手里端着一大碗晾凉了的汤圆。

他带着很厚的手套,把大碗一直送到了无心面前:``今天是正月十五,过节了。''

无心一头扎进了大碗里,连汤带水的狼吞虎咽。而丁思汉望着铁链上的斑斑牙印,知道他还是不服,自己没把他吃光,反倒吃出了他的兽性。

正月十五也算是大节日。史高飞人在一处小小的县城里,也应景吃了几只大汤圆。真正连个景都没应上的,却是史丹凤。

史丹凤找不到无心,怎么找也找不到,并且还丢了弟弟。新年前夕她接到了家里的电话,她不敢实话实说,只讲自己要和弟弟在外面过年。她妈赵秀芬不敢和儿子论理,于是牢牢的抓住了女儿,在电话中嗷嗷的叫骂咣咣的打嗝,中气十足的号称自己已经被女儿气出了病,不但生病了,而且要死了。

史丹凤被母亲骂得面红耳赤,忍气吞声的刚刚挂了电话,铃声忽然又响,一看手机屏幕,却是史一彪的号码。

史一彪虽然在金钱上从不亏待儿女,但是性情偏于粗暴,电话甫一接通,他立刻开始咆哮,让姐弟二人赶紧回家。史丹凤走投无路,随口扯了谎,说弟弟去外地旅游了。此言一出,史一彪又将她臭骂了一顿,因为她身为姐姐,居然没有对弟弟寸步不离。

史丹凤感觉自己是没活路了。

大年初一她关了手机,自己拎着一只小旅行包去了火车站。最近的一班火车是往北京去的,她漫无目的的买了票,直接奔了北京。

到北京干什么?没什么可干的,她只是感觉天下没了自己的容身之处。无心硬是没了,弟弟也联络不上。正月十五的晚上,她独自坐在宾馆楼下的一家肯德基里,要了一堆杂七杂八的食物。扭头面对着落地玻璃窗外的车水马龙,她心里茫茫然的,长久的端详自己投在玻璃窗上的影子。她瘦了,本来也不胖,如今越发瘦得四肢细长,眼下时有时无的细纹也彻底永驻了。一身的好衣服,当初是为了要配手上的钻戒,现在配了,可是又配给谁看?

史丹凤收回了目光,感觉自己是投胎投得有问题,往后再挣也挣不过命去。百无聊赖的正打算吃自己面前的一桌子零碎食物,她无意中一抬眼皮,却是骤然一怔。

在空荡的餐厅里,她看到前方角落处站着一个小男孩。小男孩穿着一身偏大的棉衣,白白的脸黑黑的眼,简直和无心是一个模子印出来的!

史丹凤愣愣的看着小男孩,看的眼睛都直了,气都不喘了。而小男孩留意到了她的目光,当即咬着手指对她一笑,然后迟迟疑疑的走向了她。

他走得越近,史丹凤看他看得越清,一颗心像被捏住了似的,一阵一阵揉搓着疼。和颜悦色的对着小男孩一笑,她含着一点眼泪问道:``小朋友,你的爸爸妈妈呢?''

小男孩开了口,小模样生得如此乖巧,却有个堪称难听的哑嗓子:``我没有爸爸妈妈,我是孤儿。''

史丹凤一听,热浪一波接一波的往脑子里冲。拿起一张餐巾纸按了按眼角,她低头又一擤鼻子。而小男孩垂下眼帘望着桌面的饮食,小声说道:``姐姐,我饿了。''

史丹凤平素连条野狗都不舍得喂的,可是此刻听了小男孩的哑嗓子,却是立刻把托盘向前一推:``喏,姐姐给你东西吃。你叫什么名字,告诉姐姐好不好?''

小男孩坐上对面的椅子,从长袖子里伸出了两只小手。仰起脸睁圆了一双楚楚可怜的大眼睛,他不假思索的答道:``我叫小猫。''

然后他张大嘴巴,将一整只鸡翅塞进了口中。

史丹凤见了他的神情举止,活脱就是个小无心。搭在桌面上的手抬了一抬,她差一点就要扑上前去抓住对方——如果小猫真没有父母的话,那她愿意收养小猫。

小猫低头吐出两根细细的鸡骨头,紧接着抬头对史丹凤一笑,伸手又去拿东西吃。史丹凤正是百感交集,手边皮包里的手机忽然响了。

手机屏幕上显示了一个陌生号码,她接通了一听,对方竟然是史高飞。不知是哪一方的信号不好,史高飞的声音断断续续不清晰。史丹凤左听右听,始终是听不清他要说什么,正是着急之时,电话彻底断了。

\chapter{懊恼的小猫}

史丹凤难得的和弟弟又有了联系,自然不能因为信号问题轻易中断。一双眼睛盯着对面正在大嚼的小猫,她一边摁键拨号,一边把手边的冰可乐推到了小猫面前。小猫吃得上气不接下气,腮帮子鼓得圆圆的。探头衔住吸管深吸了一口可乐,他打开汉堡盒子,手和嘴全都忙到了极点。史丹凤看着狼吞虎咽的小猫,越看越感觉他像无心,像得不得了。

电话连着拨了三次,第四次终于又接通了。她无暇寒暄,劈头问道:``你在哪儿呢?''

史高飞的声音中夹杂着嗤啦啦的杂声:``我在昭通。''

史丹凤没听明白:``什么交通?你到底跑到哪里去了?''

千里之外的史高飞果然是一如既往的不耐烦了:``昭通!云南昭通!''

史丹凤听到此处,差点从椅子上溜到了桌子底:``你到云南了?''

史高飞自顾自的继续嚷道:``姐,我钱不够用了。上个月公司的账目没有算,白大千手里至少还有我上万块钱。你替我向他把钱要了,立刻全打到我的银行卡里!''

史丹凤恨不能在北京给他跪下了:``小飞,人是在北边丢的,你去南边找什么呀?你有力气也不能乱用啊!姐求你了,你快回家吧!''

史高飞像头驴似的,开始在电话里咆哮:``你们懂什么?宝宝是被人当成鲑鱼抢走的!总之快把钱打给我,否则我要挨饿啦!''

然后他气冲冲的挂断了电话,不屑于和凡人多言多语。而史丹凤直眉瞪眼的攥着手机,忽然发现自己也是蠢到了家,居然企图和弟弟讲道理。

小猫吃得很快,史丹凤走了片刻的神,再清醒时发现自己面前的食物全没了,取而代之的是一堆七长八短的鸡骨头。小猫用油渍麻花的小手捧着一大杯可乐,垂着眼帘咬着吸管,吸出一片呼噜噜的空响。

史丹凤不舍得放了他,于是正色问道:``小朋友,你真的没有爸爸妈妈了吗?''

小猫抬起头,一本正经的摇了摇头:``没有。''

史丹凤又问:``那\ldots{}\ldots{}你有监护人吗?''

小猫怔怔的睁着大眼睛看她,满脸都是茫然。史丹凤立刻思索了一下,重新问道:``平时都是谁来照顾你的衣食住行呢?''

此言一出,小猫的脸上现出了恍然大悟的神情:``我自己照顾自己。''

史丹凤感觉小猫可能还是没听懂自己的话:``谁给你钱买东西吃、买衣服穿呀?''

小猫不假思索的答道:``衣服是捡的。''

史丹凤上下打量了他,发现他的衣服的确是全不合身,统一的偏大:``那吃和住怎么办?吃什么?住在哪里?''

小猫开始在座位上摇来晃去,是个小孩子无聊皮痒的模样:``没有地方住。''

史丹凤不问了,知道这孩子应该是个小流浪儿。给小猫又买了一盒蛋挞,她在小猫大吃之时,偷偷去找了正在打扫厕所的保洁员。从保洁员的嘴里,她得知小猫仿佛的确是个无主的孩子,起码在近半个月里,他每天都会跑来餐厅一两趟,默然无语笑眯眯的行乞。又因为他实在是个漂亮的小男孩,所以保安和服务员都不忍心撵他,食客也像对待可爱的流浪猫狗一样,心甘情愿的匀出些许食物给他吃。

史丹凤总怕小猫会突然消失,一边和保洁员交谈,一边用眼睛瞄着他。小猫吃蛋挞吃得兴致勃勃,忽然仰脸对着前方一呲小白牙,他似乎是在对着空气乐不可支的炫耀。

向保洁员尽可能的打听清楚了小猫的来历,史丹凤回到座位,对小猫说道:``小朋友,你今天晚上在哪里睡觉?''

小猫咬着半个蛋挞抬了头,显然是又有点发傻:``唔?''

史丹凤微笑着伸出手,摸了摸他的小脑袋:``如果没有好地方的话,姐姐可以帮你的忙。我就住在楼上的宾馆里,宾馆房间很暖和,加一个你也够住了。''

小猫迟疑着开了口:``姐姐\ldots{}\ldots{}''

史丹凤拦住了他的话头:``你多大了?''

小猫飞快的想了一下:``十岁!''

史丹凤对他柔声说道:``那你应该叫我阿姨。''

小猫很腼腆的笑了,抿着嘴摇摇头又点点头。而史丹凤等了一会儿,不见他回答,以为他是不相信自己,便又和声细语的补充道:``你如果害怕阿姨是坏人,阿姨也不会勉强你。''

小猫的耳朵微微一动,随即可怜兮兮的对着史丹凤开了口:``姐姐,我知道你是好人。''

隔着一张乱七八糟的桌子,他怯生生的向史丹凤伸出了一只手。小手伸到半路,他似乎是意识到了什么,把手却又收了回去。抓起一张餐巾纸仔细擦净了手指上的油,他羞涩的忽闪着上下两圈黑睫毛,将小手再次伸向了史丹凤。

史丹凤简直要被他的举动融化了心灵。轻轻接住了小猫的小手,她小心翼翼的握住了,心想世上竟然有个无主的小男孩,会像透了她的无心。

小猫也回握了她的手。史丹凤的手薄而软,细长的手指上戴着一枚熠熠生辉的钻戒。小猫用稚嫩的拇指蹭过戒指上的钻石,大眼睛随之亮了一下。

史丹凤带着小猫去了附近的自助银行,往史高飞的账户里转了几千块钱。

然后他们出了银行大门,手拉手的在大街上走。正月十五的夜里,有些大街喧嚣至极,有些小街寂静至极。史丹凤握着小猫的手,遥遥的望着天际盛开一朵缤纷礼花。无心依旧是没有音信,没的像是死了。她领着小猫走在路灯黯淡的小街上,只感觉无比的寂寥凄凉,仿佛小猫是无心的遗腹子。爸爸没了,妈妈和孩子怎么办?

史丹凤把小猫带进了宾馆。宾馆是家快捷酒店,环境不好不坏,倒是不辜负它的价格。史丹凤一贯省俭,如今心中没了精气神,她不省了。

房间是大床房,史丹凤看小猫是个小不点,所以不但没把他当男人,甚至根本没把他当人。走去卫生间调好了水温,她让小猫脱了衣服去洗澡。小猫不情愿,磨磨蹭蹭的脱了鞋蹲在床沿,双脚没穿袜子,小小的脚趾头向下抓着被褥。史丹凤催不动他,只好亲自动了手。``嗤啦''一声拉开他的棉衣拉链,她三下五除二的把他扒了个精光,然后拎小猴儿似的把他送进了卫生间:``好好洗,一定要打香皂。如果自己不会搓背的话,你叫阿姨,阿姨给你搓。''

小猫撅着嘴,在史丹凤的指导下拨弄冷热水龙头。花洒之中忽然喷出了暖而急的水流,劈头盖脸的正浇中了他。史丹凤向后一退,同时只听他惊慌的怪叫了一声:``嗥!''

此声一出,史丹凤没怎么样,小猫自己却是一哆嗦,如同犯了弥天大错一般,惶惶然的站在水流之中闭了眼睛。史丹凤笑了,安慰他道:``洗澡怕什么的?乖乖的快洗,洗干净了好睡觉。''

安慰完毕了,她转身想要往外走,然而一股子冷而沉重的风由外向内和她擦肩而过,竟然把她冲撞得向旁一歪。手扶门框站稳了,她狐疑的回头向内看,卫生间里一如往常,只是多了个小猫而已。

继续向外迈了步子,她感觉方才的风太重太硬了,简直不像了风。抬头向上望了望天花板,天花板上并没有换气口,史丹凤莫名其妙的进了房内,不知道屋里的风从何而来。

一手摁了电视机的电源键,史丹凤一手掏出手机,打算再和弟弟通一次话。拨通号码之后静候片刻,史高飞的大嗓门骤然震痛了她的耳膜:``姐!什么事?!''

史丹凤抓紧时间,赶着抢着告诉他:``我刚往你的账户里打了钱,你——''

如她所料,史高飞果然不肯给她把话说完的机会:``姐,我知道了!现在我要进山了,不和你讲了,拜拜。''

史丹凤心中一惊:``进山?大半夜的你进什么山?''

史高飞没头没脑的告诉他:``宝宝在山里,但是还不能确定具体位置。我手里只有一个指南针,早知道应该提前预备个GPS导航仪。''

史丹凤彻底听傻了:``你说无心在山里?谁说的?他怎么可能在山里?''

史高飞答道:``鬼说的!''

史丹凤的声音失了控,尖锐的像是鸟叫:``鬼?''

史高飞又不耐烦了:``不说了!反正是鸭子他爸要害我的宝宝!我现在先去救宝宝,等宝宝安全了,我马上去杀了鸭子和鸭子他爸!''

电话挂断,史丹凤站在原地愣了神。先前她以为弟弟纯粹只是疯,可是听了方才的一席话,她隐隐约约的感觉不对劲。被史高飞明确称为``鸭子''的人,只有丁丁一个。至于丁丁的阿爸,自己也是见过的,一个花红柳绿的小老头。叫什么名字来着?一时想不起了。

史丹凤越是思索,越感觉弟弟话里有话,并非完全的疯。心中忽然一凛,她狠狠的一攥手机,心想莫非真是丁老头绑走了无心?丁老头据说是个有邪本事的,难道他看出了无心的来历?利令智昏,有人敢偷猎东北虎,有人敢偷猎大熊猫,自然也会有人敢偷猎无心——和东北虎大熊猫相比,无心显然是更为奇妙稀罕的存在。

史丹凤越想越真,心慌意乱的稳不住了。慢慢的坐到床边,她闭上眼睛做了几个深呼吸,随即把眼一睁,她很意外的看到了水淋淋的小猫。

小猫耸着肩膀,细细的小胳膊紧缩着贴在身体两侧。大睁着眼睛望向史丹凤,他小声问道:``姐姐,无心是谁?''

史丹凤顶着一头冷汗,语无伦次的答道:``他\ldots{}\ldots{}他和你一样,你是小男孩,他是大男孩。''

小猫穿着一双塑料大拖鞋,十个脚趾头一起蜷着:``和我一样?''

史丹凤皱着眉毛翘了嘴角,又像哭又像笑:``和你一模一样的,可是他上个月丢了\ldots{}\ldots{}阿姨刚刚得到了一点新消息,明天要去云南找他,你\ldots{}\ldots{}阿姨带你一起走,好不好?''

小猫变成了一只六神无主的小白条鸡,微微张着小嘴,不说好也不说不好。冷不丁的向前踉跄了一步,他紧接着做了回答:``好,我和阿姨一起走,去云南。''

史丹凤看他踉跄得奇怪,像是被一只手从背后推了一把似的,是猝不及防的向前一仆。起身走到他面前俯□,她嗅了嗅他热气腾腾的小脑袋,嗅过之后说道:``头发怎么不用洗发水好好洗一洗?快回去,外面冷。''

她在房内穿得简单,对着小猫深深一弯腰,她专心致志的检查卫生,没有留意自己领口大开,隐隐约约的走了光。扶着小猫的肩膀正要直起腰,她忽觉胸前一凉,像是被什么东西蹭了一下似的,猛然低头去瞧,小猫的双手乖乖垂着,一双大眼睛根本就没往自己胸前瞟。

把小猫送回了水汽蒸腾的卫生间后,史丹凤自己掩着大圆领口,心想今天算是奇了怪,怎么房间里总像是多了个隐形人?

趁着小猫在洗澡,史丹凤打开电视,急急忙忙的换了一身睡衣。而在哗啦啦的水声掩护之中,光溜溜的小猫抬了头,双手合什向上拜了拜,压低声音做着口型哀求道:``琉璃哥哥不要去不要去不要去啊!我们说好了只是出来玩的,你不想继续玩了吗?再说是他先丢下我们的,我们现在为什么还要去找他?云南很远的,路上一定很辛苦,你辛苦了我的心也会苦的。''他蹙起眉头撅起小嘴,很猴急的搓着双手:``如果真的找到了他,他狼心狗肺不识好歹的,一定也不会领你的情,兴许还会以为你离不得他。哎呀想一想我都要气死了。''

一具修长的人形影子悬在了他的面前,灯光透过人形,地面居然显出了几不可见的淡淡阴影。人形的一切都太清晰太真切了,最锐利的阴阳眼也不能立刻分辨出他是真实还是虚幻。居高临下的俯视了小猫,白琉璃眨了眨蓝眼睛,认认真真的一摇头:``不,他已经离开我们半年多了,我很想见到他。''

小猫逃出了热水流,先是快速的晃去了头上身上的水珠,随即伸手向上去摸白琉璃:``我陪你玩不是一样的?他天天惹你生气,我天天哄你开心。你不要我要他?琉璃哥哥,求你了,不要去找他了。外面的姐姐有一枚很值钱的大戒指,等我夜里把它偷走卖掉。你不是喜欢看人间的热闹吗?我有了钱,就可以在外面多生活一阵子了,让你看个够!''

手指穿透了白琉璃的鬼影,居然隐隐的会有触觉,像是触到了风,像是触到了水,阴凉的一下子,稍纵即逝,无法言喻。总而言之,白琉璃不再是虚无的了。凭着他的力量,也许在不久的将来,他会为自己修炼出一副不得见光的新皮囊。

小猫作为一只两百来岁的猫头鹰精,在山林之中勤修苦练,终于在二十年前开了阴阳眼,得以见了白琉璃的真面目。他对白琉璃是死心塌地的又爱戴又崇拜,然而白琉璃不领他的好情好意,一根筋的只和无心纠缠。猫头鹰本来是个没脾气的妖精,起初作为一名旁观者,还不敢多言多语;及至在山中住得久了,他渐渐的大了胆子,再听到无心对白琉璃出言不逊,便义愤填膺的想要拔刀相助。在无心和白琉璃之间坚持不懈的挑拨离间了二十年,他终于在去年取得了一点小成绩——无心揪着他的羽毛将他毒打了一顿之后,气冲冲的自己下山了。

对于无心这根眼中钉,猫头鹰拔得委实不易,所以万万不想让白琉璃轻轻巧巧的再把钉子扎回原位。眨巴着一双泪光晶莹的大眼睛,他尿急似的夹着两条细腿扭来扭去,自认为已经可爱到爆,白琉璃只要稍有人性,就必定败在自己这一双勾魂摄魄的大眼睛前。

然而白琉璃只是静静的看着他扭,看到最后回头穿透了卫生间的薄墙,他向房内一瞟,随即一边转向小猫,一边歪着脑袋一垂眼帘,自言自语似的微笑说道:``很奇怪,总是会有漂亮女人喜欢他,不知道他这次会不会给我谈一场比较好看的恋爱。''

小猫快要哭了:``白琉璃,你听我的话嘛!''

白琉璃莫名其妙的看了他一眼,不明白自己为什么要听一只妖精的话。

因为小猫仿佛是陷在了卫生间里,所以史丹凤只好亲自出马,把他揪回了房内。他太像无心了,史丹凤怜爱的用棉被裹住了他,生怕他受了冻。他小小的躺在大床正中央,成了个沮丧而又茫然的棉被卷子。

史丹凤洗漱过后,另展开一床棉被躺在了小猫身边。她对小猫本是十分好奇的,然而此刻心中存着明日远行之事,乱纷纷的没有头绪,让她也就无暇再去询问小猫的出身来历。她不说话,小猫恨透了自己,也不肯说话——早知如此,他饿死也不会去向史丹凤讨东西吃!

史丹凤沉默良久,末了大概盘算出行程眉目了,这才转身去给小猫掖了掖被子,又情不自禁的叹了一口气:``大男孩子失踪一个月了,失踪得很蹊跷,怎么想都像是遭了绑架。阿姨一看到你,就想起了他。别的我也不奢望了,只盼着他在受苦遭罪的时候,也能有人可怜可怜他。''

小猫瞄着她手上的钻戒,想偷,又不大敢。真希望史丹凤口中的无心不是他的眼中钉无心,可是对方名叫无心,又和他相貌相似,并且是个大男孩子——这不是他的眼中钉又能是谁?

如此过了一夜,到了翌日上午,史丹凤先到附近的火车票售票点订了两张去昆明的火车票,然后领着小猫出了门,给小猫买了一身合体的新衣裤。得把小猫带着,她想,小猫跟了自己,至少在吃住两方面是有了着落。再说她也舍不得离开小猫,如果当真是找不回无心了,她自己暗暗思忖着,收养了小猫做儿子也不错。

下午她带着小猫上了火车,惶惶然的一路南下。与此同时,史高飞一无所获的退出荒山野林,正坐在一处集市的小摊子上连吃带喝。在骨神的引领下,他昨夜在山中漫无目的的转了整整一夜,几次三番的险些坠崖。本来他就觉得骨神这只鬼有点不靠谱,经过了昨夜的探险,他越发的不想再理睬对方了。

吃饱喝足之后,他摇晃着大个子给自己找了一家小旅店安身。再不睡觉就要支撑不住了,他躺在一张满是臭虫的硬板床上,一闭眼便沉入了睡眠之中。

第二天他没能起床,一身的骨骼像是全脱了节,两只大脚丫子高高的架在床头上,脚底遍布着干瘪瘪的血泡,还是前一夜彻夜奔波的恶果。挣扎着出门买了竹筒饭填饱了肚皮,他坐在床上长吁短叹。从书包里掏出无心的旧汗衫,他把汗衫铺在了枕头上。枕着枕头闭了眼睛,他想儿子,都快要想死了。

旅店老板生了一大串泥猴似的小儿女,小动物似的在旅店外面摸爬滚打。史高飞听着小崽子们的嬉笑怒骂,听得他二十五岁的年纪骤然老成了五十二。抬起脏兮兮的粗糙手背一抹眼睛,他想老板的儿女要是丢了,老板还可以和老板娘再生养,反正天下全是他们的同类;自己却是不一样,在整个地球上,自己只有一个宝宝。

他仰面朝天的瘫在床上,开始流着眼泪抽抽搭搭。他想自己已经很久都没有给宝宝买过东西吃,买过衣服穿了。

第三天,史高飞下了床,感觉胳膊腿儿又归自己所有了,便筹划着再次进山。据骨神说,关押儿子的监狱其实并不算十分偏远,只是位置刁钻,让人不能轻易找到。

然而在他出发之前,他接到了史丹凤的电话。史丹凤一手拿着手机,一手拽着小猫,肩膀上挎着旅行包,腕子上吊着个塑料袋,袋子里还装着两盒方便面以及小猫喝剩下的半瓶雪碧。站在熙熙攘攘的人群中,她高声大嗓的叫道:``小飞!我到昆明了,马上去昭通。你现在人在昭通的哪里?千万别走,乖乖等着姐!''

\chapter{非人}

丁思汉站在阴霾的天空下,挂断了手中的卫星电话。先前的丁思汉一直是个大忙人,在东南亚一带颇有名气,周游列国似的四处弄钱。弄了钱去养昆明的败家子,好个败家子,怎么养都像是要养不起,于是丁思汉快忙死了。

先前的丁思汉,如今已经成了他心底的一抹阴影。新的丁思汉并不见钱眼开,更不会为了个败家子无原则的卖命。将找上门来的生意一一推掉,他向后一抬手,把卫星电话准确无误的扔进了保镖手中。

双手十指贴着头皮,缓缓向后梳通了茂密的短发。十几岁的灵魂,几十岁的身体,他时常有些接受不了自己的老态。房内骤然传出一声惊叫,是保镖的大嗓门。片刻之后,人高马大的保镖跑出来了,用游戏的口吻小声笑道:``差一点被咬到了手。''

丁思汉没有回头,自语似的喃喃说道:``下次让岩纳去喂,岩纳的身手好。''

彪形大汉甩着手,嘿嘿的笑着答应了。丁老先生总是善解人意的,笑眯眯的永远是有话好说。虽然最近他老人家最近转了性,忽然变成了个阴森森的暴脾气,不过保镖们跟他许多年了,全能像体谅老爹似的不和他一般计较。

丁思汉在山中住了小一个月,越住越是痛苦,先前从复仇中所得的快感也淡化到无。独自坐在客厅中的一把硬木椅子上,他自己检讨内心,发现问题还是出在无心身上。

他忽然很想让无心死,无心死了,他便能了无牵挂了,便能在老死之前也出去见一见天日和世面了。可无心不死!

他不知道怎样处置无心才好了,酷刑已经施到了极致,凌迟日夜都在进行。日复一日的饱啖着无心的血肉,他简直吃到了将要呕吐的地步。

留着无心,无法处置;放了无心,他又不甘。右手下意识的从衣兜里摸出一只烟斗,他没有烟瘾,可他的身体却是一具上了年纪的老烟枪。往烟斗中填了返潮的烟丝,他吧嗒吧嗒的吸了一下午烟。吸到最后熄了烟斗,他端着一杯滚烫的普洱茶站在客厅中,对着墙壁上的镜子慢慢喝。镜子中的老脸让他有了物是人非之感,该变的不该变的全都变了,唯有他的痛苦不变。眼镜滑稽的向下滑落到了鼻尖,视野中的一切全变成了朦朦胧胧。不男不女,不老不少,超凡脱俗的优越感消失了,他低头喝了一口热茶,随即端着肩膀一笑,想自己是受虐者,也是施虐者。

喝光一杯热茶之后,他下楼去了地下室。地下室的房门大开着,岩纳正提着一只破竹筐往上走。无心的吃喝拉撒都在地下室中进行,隔三差五的就得派人进去打扫一次卫生。岩纳是个没有国籍的摆夷小子,生在边境,长在边境,起初是在雇佣军里卖命混饭吃,后来军队散了,他流浪到了丁思汉手里。手里攥着一根一米多长的铁棍,他每次在进入地下室干活之前,都会站在门口先发制人,三下五除二的把无心打到一动不动。

对着丁思汉打了招呼,岩纳拎着破竹筐上楼去了。地下室内已经被打扫干净,前方十字架下蜷缩着一只红白相间的怪物,正是无心。

无心的一只手被上方垂下的铁铐锁着,另一只手却是自由,正托着一只煮熟了的土豆。土豆腾腾的冒着热气,然而他不怕烫,低着头慌慌的连咬带吞。吃光了一个之后,他从双脚之间又拿起一个,整个儿的全填进了嘴里。

丁思汉不敢贸然靠近无心,向内迈了一步,他站住了:``无心。''

无心舔了舔掌心的土豆泥,然后拿起了最后一个土豆。土豆太大了,没有熟透,嚼得他满嘴作响。耳朵虽然听到了门口的声音,但他神情漠然,眼里心里装着的只有土豆。

丁思汉把他折磨成了一只麻木不仁的野兽。痛苦越深,回忆越浅。他所爱的人,爸爸,姐姐,已经全部淡化成了模糊的影子。坚固锋利的牙齿把土豆咔嚓咔嚓嚼成了碎泥,他低垂的眼帘随着他的咀嚼微颤。

土豆的汁水浸染了他半边面孔,半边面孔上面蒙着一层粉红薄膜,薄膜中钻出了参差不长的白毛。吞咽下了最后一口土豆,他缓缓的转向了门口。

丁思汉站稳了,一动不动的和他对视。他一直很喜欢无心的黑眼睛,天下苍生的灵气全汇聚在无心的黑眼珠里了,在最愤怒最痛苦的时候也是流光溢彩。然而自从他几天前对无心下了一次狠手之后,无心眼中的光彩便骤然消失了。

他用一把刀子,把无心的半张脸刮成了骷髅。当时无心疼到了极致,几乎快要挣断铁链的束缚。待他停了刀子之后,无心身后的钢铁十字架已经微微变形。铁链嵌入他血肉模糊的身体之中,丁思汉以为他一定要哀号了,可他张开嘴,只长长的吁出了一口气。

从那以后,他就彻底的一言不发了。

丁思汉默默的凝视着无心,看不够似的看。该报的仇已经报了——能报的,他全报了。还有一些报不了的,无法挽回的,他没办法,只好罢了。

地下室里空气污浊,然而以甜腥为主,并非恶臭。丁思汉开了口:``吃饱了吗?''

无心仰脸望着他,看他是个人,可怕的人。下意识的咬了咬牙,他的脑海中存了两个印象,一是可怕,二是人——人的可怕,可怕的人。

丁思汉转身上楼,取了两块面饼,又让岩纳去把无心重新绑回十字架。岩纳带着个帮手进了地下室,丁思汉站在门外,只听室内铿铿锵锵的乱了一阵,末了两名保镖一前一后的跑出来了,岩纳舔着手背上的一道浅浅擦伤:``先生,人绑好了。''

丁思汉进入地下室,一直走到了无心面前。用带着手套的手把面饼送到无心嘴边,丁思汉在他狼吞虎咽之时,用另一只手轻轻抚摸他的身体。指尖蹭过腰侧的一片新生嫩肉,他虽然极力加着小心,然而可能还是力气大了,因为无心含着满口的面饼猛一探头,一口咬住了他的手套。他疼得叫了一声,立刻抽出了手后退一步。

他的叫声让无心眼中闪过了一线光芒。随即无心慢慢的张开了嘴,手套先落了地,嚼烂了的面饼后落在了手套上。

丁思汉捂着掌侧痛处,不但没有愤慨,反而还有了一点隐隐的兴奋。他想自己的人生处处都是不可思议,他和无心互相折磨到了如此地步,自己对他竟然还是爱恨交织。

渐渐的,丁思汉也不大敢亲手给无心喂食了。手套连着被咬破了好几副,他老了,手脚已经不够灵活,而无心的动作又总是疾如闪电。

今年的春天来得格外晚,阴雨靡靡的一直是冷。保镖们偶尔下山去采购食物和日用品,中午出门,先向下走一段崎岖山路,然后拐入一处密林,林中停着一辆破旧的小皮卡车。有皮卡车做代步工具,他们到了傍晚便能满载而归了。

满载而过之后,是照例的一顿好吃好喝。本来丁思汉也时常和保镖们同乐,然而如今他转了性,天黑之后早早上楼去睡了觉。于是保镖们鸠占鹊巢的坐在客厅里,喝着本地产的白酒低声谈笑。

岩纳很贪酒,卤菜没吃一盘,白酒已经灌了一瓶。醉醺醺的起了身,他走到门口抄起了靠墙立着的铁棍,然后嘟嘟囔囔的一边诉苦,一边走去厨房,从大锅里挖了一小盆白米饭。端着米饭拄着铁棍,他下楼去了地下室。在头顶小灯泡的照耀下,他打开暗锁,然后在进门之前先扬起铁棍,一边向内深入一边又准又狠的敲打了无心的脑袋。

无心蹲在地上,依旧被铐镣高高吊了一只手。一声不吭的单手抱了脑袋,他照例是被铁棍打成了一团。而岩纳正是喝得周身温暖舒适,这时便很不耐烦的走到了无心面前,一手用铁棍横压了他的脑袋,一手将盆里的米饭倒在了地上。将盆沿在水泥地面磕了磕,他急归急,可是不敢大意,面对着无心一步一步的后退了,他的铁棍尖端悬在无心头顶,随时预备着狠敲下去。

就在铁棍将要远离无心之际,变故陡然发生了!

无心猛的抽出了那条被镣铐紧缠着的手臂,一跃而起扑向了岩纳。而岩纳一生中最后的记忆,便是一段附着些许淡红筋肉的臂骨。

为了能够从镣铐中得到自由,无心用牙齿啃去了自己半只手掌,以及整条小臂的皮肉。双手捧住岩纳的脑袋,他一口咬上了对方柔软的咽喉。纤细的骨骼和滑韧的筋脉在他口中吱吱咯咯的断裂开了,紧咬牙关猛一甩头,他随即用手指扒住了对方的伤口狠狠一撕!

岩纳的脑袋和身体立刻成了个藕断丝连的状态。无心松了手,一双手染透了滚烫的鲜血。伸长舌头一舔血手,他迈开大步冲向了门外的楼梯。

赤脚踏过冰冷的水泥台阶,他在倏忽间上了地面,和前方客厅中的保镖们正打了个照面。保镖们端着酒杯酒瓶,捏着鸡翅鸡腿,冷不防的见了他,统一的一起静了一瞬。

下一秒,在保镖们的惊呼声中,无心对着半开的大门一闪身,瞬间没了影子。

丁思汉被保镖从被窝里掏了出来,保镖们都是经过风浪的,所以一边掏着先生,一边急而不乱的告诉先生妖怪逃了,岩纳的脑袋也被妖怪撕掉了。丁思汉睡得正酣,此刻光着他的老胳膊老腿坐在床上,他朦胧着一双近视眼面对周围一圈大汉,先是本能似的羞涩了一下,随即猛的睁圆了近视眼:``无心逃了?''

不等保镖回答,他摸索着找到眼镜戴了上:``别围着我,快出去找!见到了用枪打,他死不了!''

保镖训练有素的立刻出门去了,而丁思汉潦草的穿好衣裤。站在黑暗中咽了口唾沫,他弯腰系好短靴鞋带,咚咚咚的也跑出去了。

在丁思汉漫山遍野的寻找无心之时,史家姐弟也出了发。

史丹凤的思维到底是比史高飞缜密许多。跑去县城买了一顶小小的野营帐篷,她感觉此地虽然不是预想中的温暖如春,但是再冷也绝冻不死人,夜里在外露宿还是不成问题。她力气小,只背着野营帐篷;史高飞力气大,负责背负食物。小猫也跟上了他们,跟的时候态度很好,姐姐长姐姐短的嘴甜如蜜,及至离开县城真上了路,他约摸着史丹凤没有时间再把自己送回县城宾馆了,便露出本来面目,开始别别扭扭的没事找事,一会儿渴了,一会儿饿了,上一步崴了脚,下一步扭了腰,总之是困得史家姐弟寸步难行。史丹凤先前看他处处像无心,偶尔露出一点小小的贱相,也很有无心的风格,然而此刻再瞧,她换了观点,发现这个崽子有时候真是太烦人了。

她不能半路扔了他,所以只好捏着鼻子牵着他走。经过了一处村庄之时,两名青年围上了他们,一团和气的问他们是不是游客——本地很有几处好风景区,每年到了旅游旺季,前来观光的旅游团一贯十分密集。不过旅行团都是成群结队走大路,敢于单枪匹马往山林里走的,一般都是探险家一流,不是探险家,也是资深驴友,以及少数不知天高地厚的傻大胆。

两名青年一高一矮,讲一口好普通话,似乎并非本地人。左右夹攻的围住了史家姐弟,他们表示自己是刚刚从山中护送出了几名外国游客。其中一人紧跟着史丹凤,热情洋溢的搭讪道:``小姐,你们是想看石刻还是看悬棺?豆沙关的悬棺看过了吗?这边山里也有,一般人绝找不到也看不到,比豆沙关的更古老。''

史丹凤先是摆手拒绝,摆着摆着,她心中一动,转而问道:``请问,前边山里还有人家吗?''

青年略一犹豫,随即答道:``差不多是\ldots{}\ldots{}没有。''

史丹凤把史高飞扯到一旁,低声说道:``要是他们真认识路,我们不如雇他们做一段向导。你不是说那房子离山下不很远吗?''

史高飞背着沉甸甸的新旅行包,脑子转了一圈,没有得出新主意,于是一点头:``行!''

史丹凤又小声问他:``你看那两个人像不像坏人?说老实话,我有点儿不敢用他们。''

史高飞看了看旁边的二位,依旧是没看出什么:``不知道。''

史丹凤抬手一指他的鼻尖:``你打起精神,万一他们是强盗,姐可指望你救命了。''

史高飞急着往山里走,听闻此言,他很不耐烦的一扭肩膀。

史丹凤不敢多说,怕惹恼了弟弟。和两名青年又讲了讲价,双方谈妥了,便一起踏着山路进了密林。史丹凤一边拽着小猫,一边提防着身边的野导。小猫哼哼唧唧的又想偷懒,结果被史高飞兜头扇了一巴掌:``再闹就滚蛋!''

小猫被他打得向前一栽,史丹凤心疼了,把小猫往自己身前一扯:``小飞!他才多大一点儿,禁得住你打吗?''

史高飞很不忿的答道:``姐,我看你是老糊涂了,没事捡个野孩子养。你还说他像宝宝——宝宝是大的,他是小的,这么明显的区别你都看不出来?真是一双老花眼!''

史丹凤如今扯着青春的尾巴,最恨旁人说自己老,听闻此言,她伸手捂住了小猫的一只耳朵,同时翕动嘴唇,无声的骂了一句。

姐弟二人嘀嘀咕咕的斗起了嘴,小猫垂头丧气的跟着史丹凤走。倒是一高一矮两名青年互相眉来眼去,一路走得东张西望。天色黑暗,史丹凤打开了一只小手电筒,向前照一步走一步,光芒微弱的可以忽略不计。史高飞跟着她走了一阵,走得磕磕绊绊十分气闷。背过一只手拉开了背包拉链,他摸黑乱掏了一阵,随即身前骤然大放光明,他双手握着一只炮筒粗的老式手电筒,一回身转向了旁边的野导,想要让他们走到前方指引方向。不料在他转身之时,两名野导正在互相耳语,冷不防的被他照了个正着。握着不知从哪里买来的超级大手电筒,史高飞暴躁的怒道:``你们敬业一点好不好?我们走出这么远了,你们屁也不放一个,由着我们往前摸黑!我雇你们是干什么的?你看你们两个的贼样子,有话不明说,非得咬耳朵,信不信老子棒打鸳鸯,在你俩中间挑一个宰了?''

史丹凤听他说话不着调,连忙出言阻拦:``小飞你别胡说八道。''又对着两名野导说道:``他不会说话,你们别往心里去。我们还是按照刚才说好的路线走——前头是不是该有平地了?''

两名青年脸上微笑,口中一边答应着,一边双手插兜走到了前方。弹簧刀的刀柄已经被他们攥热了,一旦时机成熟,他们回手一刀,今夜的财就算发了。

然而没有走出多远,他们忽然听到身后起了窸窸窣窣的声响。下意识的回头一瞧,他们只见史高飞把大手电筒夹在腋下,一手握着一只小小的青苹果,另一只手从后方背包中缓缓抽出一把半米长的砍刀。

雪亮的刀身反射了月光,史高飞一边削着苹果皮,一边抬眼望向他们,一字一句的冷冷说道:``吃个苹果,补充维他命C。''

两名青年张了嘴,吓得尿都要出来了。

\chapter{本能}

史丹凤绝没有阻拦弟弟补充维他命C的意愿,只是弟弟削皮的规模过于大了,她螃蟹似的横避到山路一侧,生怕无心没有找到,自己先被弟弟误杀。小猫迈着两条小腿紧跟慢赶,本来还思谋着再闹点事情来拦住史家姐弟的脚步,然而在见识了史高飞的大砍刀之后,他把小嘴一闭,老老实实的彻底规矩了。

和小猫一起老实的,是前方两名野导。他们在衣兜中松开了弹簧刀,手心潮腻腻的全是冷汗。一对落难鸳鸯似的互相搀扶了,他们泪眼朦胧,在身后手电筒的光芒之中向前走。好在虽然他们动机不纯,但认路的本事是真有。按照史丹凤事前的描述,他们颤巍巍的夹着尿,心慌意乱的把人领到了一片平地上。说是平地,其实并非真平,不过是相对周围的起伏山势而言。平地上的草木十分茂盛,史丹凤为了缓和弟弟散发出的恐怖空气,故作轻松的没话找话:``这一带的土地多平坦,距离山下又不远,怎么没有人家?''

前方的矮青年含泪答道:``山下有泉眼,用水方便。''

史丹凤很不自在的扭了扭脖子晃了晃肩膀,总感觉自己背上的小帐篷似乎是增加了分量:``山里有野兽吗?''

高青年出了声:``有是肯定有,去年还有人在山里被野猪撞了一下哩!''

史丹凤回头看了一眼,没看到什么,抬手摸了摸后脖颈,也没摸到什么。而小猫扭了脸向她的方向仰望,就见白琉璃盘腿坐在史丹凤后背的帐篷包上。史丹凤把长头发挽了个圆髻,圆髻下面散落了几缕弯曲长发。白琉璃拈起一缕头发,一圈一圈的往手指上缠,当然是缠不住,然而他自得其乐,玩得很来劲。

小猫叹了口气,感觉自己永远摸不清白琉璃的心思。打起精神转向前方,他拉着史丹凤的手正要加快速度,冷不防视野之中忽然金光大作,一个火流星似的鬼影从远方瞬间冲到近前。可在即将抵达史高飞面前之时,鬼影一个急刹车,对着白琉璃``哇''的惊吼了一声,紧接着贴着史高飞的鼻尖一个急转弯,倏忽间又消失在了路旁密林之中。

小猫以为对方不过是小鬼见了大鬼,自惭形秽的逃走了而已。哪知不出片刻的工夫,一个金光灿烂的大脑袋从前头树林中探出老长,探头缩脑的又开始窥视起了白琉璃。

然后,小猫听到了白琉璃的声音:``扎西贡布,不必看了,我是白琉璃。''

金色脑袋上不着天下不着地的一颤:``白琉璃,没想到你还记得我。''

白琉璃抬起头,面无表情的看了他一眼:``嗯,记得。''

金色脑袋试试探探的向他飘进了一点:``白琉璃,你怎么来了?''

白琉璃清楚而又冷淡的告诉他:``我找无心。''

金色脑袋立刻睁圆了一双大眼睛:``无心逃走了!他被一个——一个比你还要邪恶的巫师扒了几层皮,现在巫师派出了保镖,正在漫山遍野的捕捉他。''

白琉璃一点头,同时松开了史丹凤的头发:``扎西贡布,你带我去找他。''

金色脑袋也跟着他一点头:``白琉璃,以后不要再叫我扎西贡布。西康的往事我已经忘记了,现在你可以叫我的英文名米奇,或者中文名骨神——不,骨神是我随口胡诌的。你叫我米奇好了。''

白琉璃在凌乱的长发之中垂下眼帘:``好的,扎西贡布。''

骨神又对着白琉璃一摆手,随即在史高飞面前现了形。金身罗汉似的悬浮在半空中,他扯着低沉动人的大嗓门叫道:``我终于找到了巫师的家!''

史高飞刚把砍刀收回背包。此刻捏着一只细细的苹果核,他对于骨神的言语嗤之以鼻:``用不着你找,一边呆着去!''

骨神知道史高飞对自己意见不小,但是自己的话不说不行:``无心刚刚逃进了山里,你们得赶在丁思汉的头里找到他!''

史高飞很怀疑的盯着他:``真的假的?''

骨神急得一拍大腿:``爱信不信!''

话音落下,他的影子渐渐淡化在了夜色之中。前方的高矮二青年停在半路,愣了良久之后缓缓回过了头,颤着声音问道:``先生,刚才说话的大仙,是何方神圣啊?''

史高飞不耐烦的答道:``一只鬼。''

此言一出,两名青年怪叫一声,张牙舞爪的统一开始向后狂奔。不过片刻的工夫,他们已经没了影子,而史丹凤站在路边,一手拽着小猫,一手叉腰,张着嘴犯迷糊,不知道自己是否应该当场吓晕——按理来讲,自己作为一名年轻女子,该晕一晕;不过凭着弟弟的狼心狗肺,自己若是真晕了,很有被他抛在路上喂野猪的可能。

干巴巴的闭了嘴,史丹凤决定还是不晕为妙。低头看看小猫,小猫咬着一根手指头,兴许是年纪小不懂事的缘故,倒是堪称淡定。

居心叵测的野导半路逃了,史家姐弟面面相觑,有了点晕头转向的意思。忽然小猫出了声,细着嗓子的唧唧说道:``姐姐,我害怕。''

史丹凤没看他,自己摸着陡然轻松的后脖颈答道:``别怕,有阿姨呢。''

小猫的迷魂术没有施行成功。眼看白琉璃真是追着骨神远去了,他眨巴眨巴大眼睛,一转身又拉扯了史高飞的衣襟:``哥哥。''

史高飞低了头,气势汹汹的问道:``又干什么?''

一声问话出了口,他盯着小猫的眼睛不动了。而小猫对他悠悠的一转眼珠,一个小脑袋又仰向了史丹凤:``姐姐,你看我。''

史丹凤心不在焉的望向了他,本来只想扫他一眼,可是双方目光一触,她身不由己的出了神,陷在对方的黑色瞳孔中不能自拔了。

把史家姐弟全迷住了,小猫从鼻子孔里出了凉气,撅着嘴又关了他们手中的手电筒——想找无心吗?等巫师夜里把无心重新抓住了,你们再找吧!

然后退到一旁脱了衣裤,他拍拍双臂向上纵身一跃。身体腾空而起,他化为一只大猫头鹰,双目如炬的追白琉璃去了。

小猫越是飞得远,越感觉林中空气不对。居高临下的扫视地面,他看到了许多鬼魂。

鬼魂的力量有强有弱,统一像要赶集似的飘了个漫山遍野。他的大眼睛放出贼光,一路东张西望的寻找白琉璃。正是入神之际,他``咣''的一声,一头撞在了前方的山石峭壁上。伴随着一声嗥叫,他和他的羽毛一起落到了下方的树木枝叶之中。

在小猫收拢翅膀忍痛之时,白琉璃和骨神一前一后的拉开了距离。骨神还是畏惧丁思汉,面对着丁思汉手下的鬼奴隶,他生怕其中会有多事的坏鬼跑去给丁思汉通风报信。而白琉璃停在前方回了头,一脸天真无邪的大无畏:``扎西贡布?你不走了?''

骨神有点怕他,所以在回答之前先瑟缩了一下:``我怕丁思汉。今夜的情况很异常,他好像是放出了他手里全部的小鬼。

白琉璃若有所思的又问:``丁思汉在哪里?让我去看看他。''

骨神被他问住了——收到了玛丽莲的警告之后,他现在真是不大敢靠近丁思汉。

白琉璃本来对骨神也没感情,骨神又是吞吞吐吐的一问三不知,于是他打算甩了骨神自己走。可是未等他真正前行,一个女鬼忽然出现在了他和骨神之间,正是玛丽莲。玛丽莲本是有话要说,然而此刻看了看骨神,又看了看白琉璃,她抬手一捂胸口:``哇!你俩放在一起真是帅得刚柔并济呀!''

骨神对她是一贯的不客气:``我问你,丁思汉在哪里?''

玛丽莲笑嘻嘻的:``主人还在家里,你可不要去冒险哦!''然后她转向了白琉璃:``敢问这位小哥高姓大名?鬼龄几何?看你也不像个淹死鬼,可是一身湿淋淋的宛如出水芙蓉,莫非死前还特地洗了个澡?''

白琉璃正在望着天想心事,并没有回答的意思。于是玛丽莲兴致勃勃的又面对了骨神:``米奇,我知道你为什么一直不近女色了,原来是你的性取向有问题。刚才我看你像哈巴狗一样苦追这位中性风小哥,可惜人家一直不肯理你。''

骨神听到这里,当场崩溃:``玛丽莲,你少来恶心我!我虽然不是丁思汉的对手,但是收拾一个你还不在话下!''

玛丽莲见势不妙,立刻逃跑。而骨神在追杀她之前一转眼珠,发现白琉璃已然自顾自的离去了。

白琉璃像一朵沉重的云,慢慢的飘向了树木环绕之中的二层小楼。小楼是座粗糙的建筑,水泥外墙上生着脏兮兮的青苔。一楼的窗口亮了电灯,从外向内望,可以看到室内摆着稀稀落落的桌子椅子,一名背着猎枪的大汉正在桌椅之间来回踱步。

于是白琉璃追着一抹鬼气,向上升到了二楼。隔着二楼紧闭着的玻璃窗,他没敢贸然的穿墙而入,因为室内窄窄的窗台上左右各立着一支巴掌长的小黄旗子。房内不该有风,然而旗子各自向着左右猎猎的飘动,倒像是一股疾风在窗台正中兵分两路了一般。盘起双腿悬在窗外,他将双手搭上膝盖,闭着眼睛垂下了头。

窗内的风似乎越发急了,并且乱了方向,小黄旗子盘旋乱转,扑啦啦直打窗玻璃。忽然起了``砰''的一声爆响,两扇玻璃窗猛的大敞四开,两支小黄旗子则是被房内鼓出的疾风吹飞了。

通过黑洞洞的窗口,白琉璃看到屋内的小老头。小老头穿着一件毛茸茸的连帽衫,在房间正中央席地而坐。抬手一推滑到鼻尖上的半框眼镜,小老头闭着眼睛一笑:``不得了,来了个大家伙。''

白琉璃进入了房内,在小老头面前也坐下了:``丁思汉?''

丁思安微微一点头。

白琉璃又问:``能看到我吗?''

丁思汉摇了摇头:``看不到,也不必看。''

白琉璃力不能支似的向前俯下了身,轻声说道:``把无心给我。''

丁思汉笑了一下:``他刚逃了,逃得无影无踪。''

白琉璃缓缓的偏过了脸,在黑暗中向上翻起一双蓝眼睛:``去找到他,然后给我。''

丁思汉终于睁开了眼睛,他没有修炼过阴阳眼。看阳界,他用眼;看阴界,他用心。自从以着丁思汉的身份重新复活之后,他用心的时间,远远多过用眼。准确无误的对准了白琉璃,他不把人放在眼里,同样也不把鬼放在眼里。

``口气不小啊。''他闲闲的说道:``你和无心是什么关系?好朋友?有交情?''

白琉璃缓缓的一眨眼睛:``好朋友,有交情。''

丁思汉抬起了一只手,当着他的面凌空画了一道符。示威似的横了白琉璃一眼,他随即猛一挥手,半空中竟是骤然爆出了一道火光。火光直奔了白琉璃的面门,一道雷似的要把他劈成魂飞魄散。然而白琉璃纹丝不动,火光掠过他的鬼影,竟是无声无息的熄灭了。

小丁猫颇为意外的一挑眉毛,随即感觉膝盖一凉,低头看时,他预感到了不妙。

白琉璃的手介于虚实之间,是幻影,却又带着隐约的重量和温度,无声无息的搭上了丁思汉的膝盖。他缓缓的直起了腰,他的手也一路后退。手指恋恋的离开了丁思汉,他垂下头,声音轻不可闻的又道:``给我。''

然后不等丁思汉回答,他一路后退,穿墙而出。

丁思汉慌忙卷起了自己的裤腿,借着窗外的月光,他看到自己膝盖上黑了掌心大的一片。急忙用小刀子挑破乌黑的皮肤,他忍痛狠挤,竟然挤出了一股子浓稠的黑色油膏。心神不定的喘了口气,他知道自己是遇上对头了!

白琉璃决定自己去找无心。然而无心没找到,他先找到了小猫。

小猫缩着翅膀蹲在一根低低的树枝上,正在很警惕的四处张望。忽然见了白琉璃,他欢喜的一张翅膀:``琉璃哥哥,无心像只野猴子一样,没有人能找到他的!你找也找了,是不是找完了我们就可以回家了?''

白琉璃莫名其妙的答道:``我正在找。''

小猫在树枝上搔首弄姿,连着换了好几个姿势:``我替你找过了,整座大山我也都飞过一遍了,没有无心,什么都没有!''

白琉璃听了这话,感觉猫头鹰满嘴谎言,也是个骗子。心中忽然一动,他换了个话题问道:``无心的女人和女人的弟弟去哪里了?''

小猫张开了尖嘴,露出一条尖舌头:``他们——走了另一条路。''

白琉璃不再多问,也不理睬小猫,径自飘远了。而与此同时,史高飞抡着他的砍刀,正在山路上和一只纸人搏斗。原来他和史丹凤中了迷魂术,怔怔的在路上呆站了许久,直到两双手掐上了他们的脖子。他们被掐得如梦方醒,挣扎着回头一看,却是和两张描眉画眼的纸脸打了照面!

史丹凤尖叫了半声,另半声被纸人的手扼在了喉咙里。史高飞不知道害怕,此时反倒占了临危不乱的便宜,向后便是一胳膊肘,当即把纸人的白脸杵了个大窟窿。纸人全靠着一股子阴魂控制支撑,身体受了损,并不耽误它们行凶。史高飞又去用力撕扯了它的双手——一双手合在脖子上时,凉阴阴的很像人手。史高飞没能立刻扯开它的手,正是窒息之时,身边却是腾起了火光。原来史丹凤做好了露营的准备,随身携带着一块钱一个的打火机。此刻她手无寸铁,情急之下掏出打火机,噼里啪啦的乱摁一通,想要放火吓唬身后的东西。不料一点火星落到颈部的手上,腾空一团光焰过后,纸人竟是没了。

史丹凤吓了一跳,自己抬手摸摸脸,脸上不疼,五官也都还在。自知找到了克敌制胜的法宝,她一把火燎了弟弟身上的纸人,然后张皇失措的去叫小猫。

未等她找到小猫,纸人又来了,而且一起来了三个。

一块钱的打火机十分不做脸,无端的开始摁不出火。眼看纸人越来越近了,史丹凤带着哭腔问道:``这是什么东西?这是怎么回事?''

史高飞抽出砍刀,略一思索,随即有了答案:``姐,不要怕,它们应该还是鬼。一般的鬼都是3D的,但是我们面前的鬼是4D的,更先进而已。姐,这就是科技的力量。''

史丹凤带着哭腔做了回应:``科技你奶奶个腿儿呀!这下可好,咱俩全交待在山里了,我也不用伺候你一辈子了。''

史高飞听了他姐姐的牢骚,心中暗骂:``这个粗俗的地球妇女!''

骂过之后,他呐喊一声,抡着砍刀冲向了最近的一个纸人,一刀削下了对方的一条手臂。然而纸人满不在乎,仰着一张喜眉笑眼的白脸,飘飘忽忽的包抄向了他们。史高飞并不是练家子,左一刀右一刀的乱砍一气,差点把他姐也给剁了。史丹凤被一只纸人抓住了头发,自己挣脱不开,眼前又是刀光闪烁,不禁吓得吱哇乱叫。正是要命之时,路边树上忽然腾空跃下一个人影,紧接着三团火光升了空,纸人已经灰飞烟灭。

史高飞和史丹凤气喘吁吁的站稳了,大睁着眼睛去看来人。在一轮惨白的大月亮下,他们想自己是看到了无心。

夜色之中,看不清无心的面目,但他们对无心是太熟悉了,一看身形便能认出。无心□着身体,胳膊和腿都是极端的细瘦,身体依稀是个斑斓的颜色。史高飞没有耐心端详他了,大叫一声冲上前去,他欢天喜地的喊道:``宝宝!''

然而无心逆着月光,却是退了一步。

史丹凤越过了弟弟,连哭带笑的伸出了手:``无心,傻小子,过来啊!''

无心微微的侧了脸,半张面孔上的白毛在夜风中微微的抖。静静的凝视着面前的史高飞和史丹凤,他感觉有些熟悉,又有些陌生。方才是凭着本能才下树救他们的,其实他并不知道自己为什么要出手相助。

史丹凤史高飞一起向他迈了步,他们距离他越来越近了。周身皮肉忽然针扎似的剧痛起来,无心猛的一惊,一个转身冲入了路边密林之中。

无心一直跑,一直跑。四脚着地的跑,攀爬跳跃的跑。跑到最后他停在了一棵老树上。茫茫然的嚼了一嘴树叶,他瑟缩着转动了脑袋。

动物性压过了一切,支配了他的身心。他一边大把的捋下树叶往嘴里填,一边警惕的环顾四周。``咔嚓''一声咬碎了混在树叶中的一只硬壳大甲虫,树叶的绿汁顺着他的嘴角往下淌。

当树叶填饱了他的肚皮时,他溜下树,继续疯狂的向前跑。一直跑到无路可跑了,他在一片石壁前仰起头,壁立千仞,草木不生,只稀稀疏疏的点缀了几个黑影,是千百年前留下的悬棺。

抬起尚且完整的左手抚上石壁,无心问自己:``他们是谁?''

``他们''已经成了个模糊的大概念,他只知道``他们''全是人。有坏的,也有好的,好的是谁?想不清楚了。

抱着肩膀坐在了石壁下,他在寒冷的夜风中瑟瑟发抖。他真希望有一个好人肯来抱他一下,不抱他,摸他一下也好。他太恐惧太孤独了,再次抬头看清了石壁上的一处小小洞口,他颤抖着站起了身,残缺的右手举起来扳住一块凸出的尖石,他开始向洞口攀爬。

\chapter{中招}

无心蹲在石洞的边沿,右臂新生的一层粉红肉膜在方才的攀爬之中磨破了,淡红色的血水顺着胳膊肘向下滴答。他伸长了被草汁染绿的舌头,轻轻去舔自己的伤口。夜色之中有不知名的大鸟掠过,当空的大月亮已经有了西沉的趋势。

他舔了良久,直到疼痛的感觉渐渐钝化了,他才放下手臂,四脚着地的爬向了洞内深处。洞子的入口堪称干净,内中则是黑沉沉的深不可测。他抽了抽鼻子,忽然隐隐的嗅到了一股子恶臭。

于是他不动了,靠着石壁蜷缩成了一团。洞子不算宽敞,大概是一人来高一人来宽,不知是自然形成的,还是人工开凿的。无心闭了眼睛,有气无力的摸索着周遭——他想藏到地下去,先避一阵子再说。然而他现在虚弱之极,没有立刻上天入地的力量了。

他不敢回到地面上去,只想找个隐蔽地方,能容许自己慢慢的往土壤里钻。可是洞中石壁坚硬,连滴水都不能轻易渗入。

夜风从洞口灌了进来,正吹在了他的后背上。他觉出了冷,于是瑟瑟发抖的继续往洞里爬。洞子起初一段是笔直的,地面也平坦,然而越往里越崎岖曲折,冰冷潮湿的空气也渐渐升了温度。无心小心翼翼的贴着一侧石壁向内行进,忽然半路停了动作,他那残缺不全的右手猛然在空中晃了一下。一瞬间的工夫,他已经从石壁上方摘下了一只大蝙蝠。

不假思索的,他把大蝙蝠填进了嘴里。``咯吱''一声牙关紧咬,温暖的鲜血立刻溢满了他的口腔。他大口咀嚼着蝙蝠细脆的骨头和柔软的皮肉,舌头尝不出味道来,完全是出于本能在吃。在成长期间,他总是疯狂的索求着营养。

吃掉了大蝙蝠之后,他继续前进,从靠近石壁的地面上蹭了一身的蝙蝠粪。不知拐了几个弯,他开始听到了隐隐的水声。嶙峋的洞壁滑溜溜的,也凝结着一层水珠。无心的精神当即一振——他需要水。

进了肚的大蝙蝠给他增添了一点体力。他觅着水声又爬又跑又跳,末了在一面倾斜的石坡上打了滑,``咕咚''一声跌落进了一处水潭之中。水潭的水并不很凉,他一边下沉一边咕咚咕咚的痛饮,一直涨出了个大肚皮。水潭底部也是石头起伏,他在漆黑的深水中长长的伸展了身体,脑袋忽然甩出一道暗流,他用牙齿咬住了一条擦肩而过的水蛇。

双手抓住扭曲盘卷的蛇身,他仰面朝天的把自己陷在了一处石窝子里。石窝子向上开口,宛如人的臂弯,稳稳当当的托了他的后背和大腿。他专心致志的吮吸着蛇血,吮着吮着,忽然感觉此情此景似曾相识。在不久之前,或者很久之前,也曾有人这样托抱着他,给饥饿的他喂食。

蛇血从他的嘴角散逸开来,混于水中。无心正是放松惬意之时,心中无端的一凛,却是生出了不祥的预感。他此刻已然丧失了思考的能力,下意识的纵身一跃凫上水面,他嘴里叼着死蛇,手脚并用的爬上了岸。结果未等他在岸上蹲稳,水声由远及近的激烈了,面前的水潭中骤然崩出了一朵巨大水花,不知是什么东西正在水下翻江倒海。

无心连连的后退,一直退到了角落里,嘴里还叼着死蛇。原来水潭也不是他的安身之处,他可不是水中那大家伙的对手。至于大家伙到底是什么,那他还不能确定,希望是鱼,因为鱼不能上岸。

无心有些怕,沿着原路往外退。退着退着,他抱着脑袋躲到了一块突出的大石后面。与此同时,洞外起了铺天盖地的异响,正是无数大蝙蝠赶在黎明之前回洞了。

蝙蝠密密匝匝的往洞子深处钻,洞内直乱了一个多小时才恢复了太平。无心不敢和蝙蝠大军抗衡,只能被蝙蝠挤到了洞口去。天还是黑,月亮也落了,简直黑到了伸手不见五指的程度。无心倚靠着石壁坐好了,茫茫然的用牙齿撕扯蛇肉。想要做人,至少得有个人模样,人模样连着他的人心。如今他不是很有人模样,连着的人心就也不知丢去了哪里。现在他的脑子里只有两件事,第一是吃,第二是躲。

天边显出了一线鱼肚白,把群山与丛林映衬成了起伏的剪影。无心小心翼翼的从洞口伸出了一张红白相间的花脸子,眨巴着大眼睛往远方眺望。他吃饱了,肚子舒服了,然而心中依旧难过,仿佛是在思念着谁,可到底是在思念谁呢?不清楚了,不知道了。抬手轻轻挠了挠生着白毛的半边面颊,他感觉新生的嫩肉有一点痒。歪着脑袋在肩膀上又蹭了蹭脸蛋,他垂下眼帘,看向了自己搭在大石头上的双手。手很瘦也很脏,指甲缝里凝结着干涸了的蛇血。右手的一半是块粉红畸形的肉,手指的骨骼藏在肉里,还未生长成形。

他呆呆的直了目光,右眼的睫毛挑着一缕灰尘。末了向着前方一抬头,他迎着地平线上喷薄而出的漫天朝霞,微微的张开了嘴。

他是想呼唤,呼唤一个名字。名字是什么,名字是谁的,他全不知道。他只是觉得自己不应该这样孤单,他想在这个世界上,一定还有另一个人认识自己,关怀自己。

否则,自己怎么会在最痛苦的时候,感到委屈?

无心躲在洞口横生的一块石头后面,静静的回想着那个名字。怎么想也想不起来了,于是最后他垂下眼帘,默默的向后缩,一直缩进了洞中黑暗处。清晨的风在洞口盘旋而过,带着冰霜的凉和草木的香。无心冷了,有意往深处躲,可深处住着蝙蝠的大家族,上面黑压压,下面臭烘烘,让人不能轻易安身。脑子里一片空白,那个名字依然是想不起来。忽然咧嘴笑了一下,是给自己笑的,想不起来就想不起来吧,他安慰自己,哄着自己,极力想要压下自身的野性,既然无论如何都死不了,那就还得好好的活。

洞口暗了一下,是一只大猫头鹰斜斜的滑翔而过。险伶伶的在石壁前方做了个急转弯,小猫紧紧的闭了尖嘴,强忍着没有叫出声音——他看到无心了,并且被无心的模样吓坏了。

在遮天蔽日的山林里,他收拢翅膀落在了一棵矮树上。树下坐着白琉璃,垂头发话问他:``有线索了吗?''

小猫的小脑筋转了又转,随即扯着哑嗓子答道:``没有。''

白琉璃喃喃的又说:``你找不到,我也找不到。算了,让丁思汉去找。''

然后他叹了口气:``可惜我死了,很多法术,我没有办法再用。''

把胳膊肘架在两边大腿上,他俯身闭了眼睛,一动不动的没了声息。他在巫术方面本来堪称全才,可惜如今没了身体和法器,他满心的花骨朵,硬是开放不出几朵来。以他为中心,周遭几米之内的花草树木全静止了,连小虫子都停了鸣叫。

与此同时,远在几里地外的丁思汉,面孔忽然黑了一下。

他在家里实在是坐得心烦意乱,宁愿辛苦了老胳膊老腿亲自出马。一张纸符烧成灰敷上了膝盖伤处,倒也压制住了那一片乌黑。他和鬼打了几辈子交道,还没遇见过这么厉害的鬼爪子,算他一时大意,老马失蹄。

他提起了精神,决定从此开始谨慎行事。下意识抬手摸了摸脸,他又低头看了看掌心,人老手不老,他感觉自己的双手一直还算嫩,然而此刻粉白的掌心上却是笼罩了一层依稀的青气。用泛了青的手再摸摸脸,他明显觉出了异常——自己的皮肤在硬化!

他吓了一跳,当即从怀中摸出一张纸符点了火。纸符阴燃出了淡淡的烟雾,被他拿着满脸满身的熏了一遍。这一遍是用来祛阴气的,如果体内体外附了蛊虫一类,蛊虫大多属阴,经了这么一熏,必定也该有所反应了。

可是直到纸符缓缓的化为了灰烬,他的周身还是不痛不痒。仰起脸承接了茂密枝叶之中透下的细碎光斑,他慢吞吞的抬起手,很轻巧的打了个响指。

玛丽莲应声出现在了他的身边,受宠若惊的唤道:``主人,有什么吩咐?''

丁思汉低声答道:``附近藏了一位鬼巫师,去找到他。''

玛丽莲也看出他脸色有异了,不禁回想起了昨夜的奇遇。没敢当着主人的面提起米奇,她管住了自己的嘴,一路飘远找鬼巫师去了。

丁思汉带着两名背着猎枪的保镖继续走,一股子凉气如影随形的纠缠了他,一波接一波的冲击骨缝关节。丁思汉顶了片刻,感觉自己有些支持不住,便咬紧牙关脱了外衣,又用刀尖刺破手指,龙飞凤舞的在外衣背后画了一道淡淡的血符。双臂打着颤重新穿好外套,凉气的势头果然立刻减弱了许多。

``这是什么招数?''他一边走一边开动了脑筋。鬼上身不是这个感觉,况且也不会有鬼敢上他的身;可若不是鬼上身,又是什么?他玩了几辈子鬼,玩得自己都成了人不人鬼不鬼,不过话说回来,术业有专攻,他也只会摆弄小鬼。

血符是用来驱邪祟保平安的,符的图案很常见,符的力量却是取决于画符人的本事。点了一根香烟叼在嘴上,他探头做了个侧耳倾听的姿态,原来是玛丽莲回来了——她没有找到鬼巫师,但是在附近一条河边见到了史家姐弟。

史家姐弟对于丁思汉来讲,堪称一文不值。于是他一挥手赶走了玛丽莲,双手插兜继续走。

沿途不住的有小鬼给他通风报信,所以他也并非是乱走。末了停在一面峭壁之前,他仰望向上,口中轻声问道:``是在这里?''

一个声音在他耳边嘁嘁喳喳:``主人,我在洞口看到了他。''

丁思汉又问:``为什么不进去?''

那个声音含羞带愧的说道:``我\ldots{}\ldots{}不敢。''

丁思汉盯着上方那开在一具腐烂悬棺旁的洞口,洞口距离地面足有二三十米高,上不着天下不着地的,黑洞洞的莫测高深。很不最自在的耸了耸肩膀,他问身边的保镖:``我们能上去吗?''

保镖摸着下巴仰着头,很慎重的考虑了一分多钟,末了才答道:``能!''

丁思汉点了点头:``我们回去准备一下,设法进洞。''

丁思汉带着保镖回了家。上楼进了他的卧室,他急急的从床下箱子里翻出一沓纸符。纸符是前任丁思汉的存货,前任丁思汉倒是个乐观的过日子人,攒钞票,攒房产,甚至连鬼都攒。关闭门窗坐在了地面中央,他急急的将八张纸符在自己面前摆成了八卦形状。另取一张黄纸点燃了,他咬牙切齿的轻声念道:``九丑之鬼,知汝姓名,急速逮去,不得久停,急急如律令!''

话音落下,他手腕一转,八张纸符一起经了他手中之火,瞬间喷出一圈光焰。封在纸符中的凶鬼恶灵被他打成魂飞魄散,阴邪之气随之爆发向了四面八方。而一直追随着他的、说不清道不明的一股子寒意骤然受了零碎魂魄的冲击,及至阴气散尽了,寒意果然也跟着消失了。

丁思汉紧紧的一闭眼睛,又长长的呼出了一口气。他想自己真得尽快找到那名鬼巫师了,治标之法不能持久,自己须得把那巫师打成灰飞烟灭才行,否则,怕是要出大麻烦。

这时,保镖已经准备好了登山的设备。丁思汉起身出去一看,发现他们居然只带了一卷尼龙绳子和几只脏兮兮的登山镐。一个黝黑的小子笑道:``我们先爬上去,进了洞再用绳子拽先生。''

丁思汉不置可否的点了点头,想起无心,心中一阵悸动,可是想过之后他的思维分了叉,把方才玛丽莲提供的消息又捡了起来:史家姐弟来干什么?来找无心?他们怎么知道无心会在这里?莫非白大千当真是有些神通?白大千来了没有?

丁思汉一边胡思乱想,一边结结实实的吃了几大块巧克力。最后对着保镖们一扬手,他率先走出楼门,且走且伸了舌头,很费力的舔着粘在牙齿上的巧克力。

楼内除了死了的岩纳之外,一共还有八名保镖,跟着他的是四名,余下四名留下看家。花了将近一个小时的时间,他们抵达了峭壁之下。

在丁家的保镖向上攀爬之时,几十米之外的大树上,史高飞眯着眼睛,将他们的行踪看了个一清二楚。自从昨夜眼睁睁的看着无心逃走之后,他和史丹凤先是沮丧了一场,随即重新振奋了精神——原来还是试探摸索着想来碰运气,没想到一切都是真的,无心也的确是在这一片山林里,既然如此,他们找人的决心反倒更坚定了。

回头望向树下,他小声唤道:``姐,我看见鸭子他爸了!在那边的石头山底下,正往上爬呢!''紧接着他对史丹凤居高临下的又招了招手:``你那边看不见,你到我这边往前看!''

史丹凤拧着眉瞪着眼,蓬头垢面的站在三米开外:``你没拉完,我能过去吗?''

史高飞双脚叉开,蹲在两根平行伸出的粗树枝上,一个光屁股撅出老远:``我也是迫不得已,地上的虫子咬我的蛋!''

史丹凤没有好气:``别废话了,你快点儿!虫子怎么不咬我呢?''

史高飞答道:``因为你是女的,没有蛋。''

史丹凤听到这里,又颇想掐死他了。

三分钟后,史高飞提着裤子下了树。因为肚子里松快了许多,所以他立刻又向他姐要了一包干脆面。史丹凤动作缓步伐慢,但是更有韧劲。在他咔嚓咔嚓大嚼之际,她含了块水果硬糖,决定依从弟弟方才的指示,前去看看丁思汉到底在搞什么鬼。

丁思汉的保镖们,无论年龄大小,全是野小子一流,登高上远他们是行家。洞口既然不是封闭着的,想必里面也不会存着有害的气体。两个轻巧的小个子先爬进了洞口,没敢贸然往里走,而是把粗糙的尼龙绳子垂了下去,让下面的兄弟用绳子把丁老先生缠绑几圈。丁思汉生怕自己体力不足,路上又吃了不少甜食,被奥利奥糊出了一张黑嘴。这时他一边由着保镖给自己五花大绑,一边专心致志的舔牙齿舔嘴唇,越舔越黑。

然后像要上吊似的,上面的保镖开始把丁思汉往上拽。绳子绑得不妥当,丁思汉刚一离地就感觉不对劲——身体快被绳子勒断了!

于是他落了地,让保镖给自己重新绑,怎么绑都不舒服。及至他终于舒服了时,已经到了傍晚时分,天光从明亮转为了黯淡。

心惊胆战的上了洞子,丁思汉因为恐高,所以吓得双腿软成了面条,坐在洞口喘息不止。保镖为他解了绳子,把一端绳头顺手绑在了洞边突出的一块大石头上——丁老先生是值得他们费一费力气的,而下面两位大个子兄弟,就无须他们亲自去拽了。

两位大个子并没有全上来,留下了一个殿后。丁思汉见自己一方的人员已经齐了,便扶着石壁站起身,一边从裤兜里摸出一支小手电筒打开了,一边嘱咐保镖道:``都给我打起精神来,想一想岩纳是怎么死的!''

保镖一起答应了,知道无心是个厉害的家伙。同样拿出小手电筒打开,三个人各自抽出短刀,一步一步的试着往里走。

丁思汉加了十分的小心,小心翼翼的抬脚落步。如此走了没有多远,他和保镖一起停了脚步,只感觉洞子深处起了可怕的骚动,并且由内向外鼓出了一股子恶臭。

他疑惑了,回头去问保镖:``怎么回事?是不是洞里有野兽?''

保镖侧耳倾听,一脸的糊涂相:``先生,听着不像大野兽,倒像是\ldots{}\ldots{}''

话音未落,洞子深处骤然刮出一阵黑风。丁思汉大叫一声卧倒在地,后方的三名保镖也惊呼哀嚎着滚作了一团——傍晚时分,洞中的大蝙蝠倾巢而出,成群结队的觅食去了。

一个小时之后,大蝙蝠散尽。丁思汉以及他的保镖们覆着一身的蝙蝠粪,东倒西歪的站起了身。众人抬手摸了摸脸,保镖们全受了皮肉伤,龇牙咧嘴的倒也罢了。丁思汉抬手一抹眼镜片上的蝙蝠粪,却是把腰一弯,哇哇的大吐了一场。

吐过之后抬起了头,他抬袖子一抹嘴,细着嗓子呻吟了一声。保镖陪着小心问道:``先生,还往里走吗?要不然,您今晚回家休息一夜,明天再来吧!''

丁思汉幽幽的叹了一声,花白头发散了满额:``走走吧,能走多远算多远。否则白天蝙蝠回了洞,里面的路更难走。''

保镖们相视一笑,认为先生这句话说得娇声嫩气的,像个挺小的小姑娘。而丁思汉下意识的对着前方挤眉弄眼了一下,又抬手摸了摸脸——脸不舒服,皮肤发硬发紧,四肢百骸也像是灌进了凉风,冷飕飕的难受。太阳落山了,阴气随之浓重了,他硬撑着向内又走了两步,末了停在半路,他感觉自己又有了要中招的意思。

``不行!''他突然说道:``我们下去回家,明天天亮再来!''

在远方史高飞和史丹凤的注视下,保镖们齐心协力,把一脸黑气的丁思汉从洞口吊向了地面。

\chapter{洞中相聚}

丁思汉派出了无数小鬼,漫山遍野的寻找白琉璃,然而大半夜过去了,游魂们一无所获,他所承受的痛苦却是越发剧烈了。独自坐在潮湿冰冷的卧室地面上,他咬紧牙关盯着前方的一点光明。房内没开电灯,全靠着一根蜡烛照明。火光如豆,在他的眼镜片上一分为二,跳跃腾挪。

一线细细的黑血流出了他的鼻孔,他一动不动,额头皮下的毛细血管乌黑的肿胀硬化了,自上而下形成了一张越来越淡的网,正在以着极慢的速度笼罩他。他隐约明白了,自己是受了诅咒。

对于咒术,他一直是知之甚少。此刻束手无策的坐在地面上,他所能做的只有放了自己的鲜血,在四面八方一道叠一道的画下血符。为了抵挡外来的邪气,他把自己当成了鬼来处理,用层层符咒把自己给封住了。

凌晨时分,他额头上的黑网慢慢消退了,干硬的皮肤也渐渐恢复了柔软。抬手堵住一侧鼻孔,他弯腰向地面用力的呼出了一团黑色血块。今夜是熬过去了,明天怎么办?白天倒也罢了,夜里鬼巫师的力量明显强了许多。白天可以对付,夜里可是将要对付不过去了。

丁思汉左右为难,不知道自己是应该先去解决掉鬼巫师,还是先去捕捉无心。让保镖烧了一壶热水,他又洗头发又擦身。头发洗到一半,卫星电话在外面响了。保镖开了门给他递电话:``先生,小丁先生打来的。''

丁思汉光着膀子顶着满头满脸的雪白泡沫,因为被个壮汉见了自己的半裸体,故而羞得老脸通红。伸着湿手接过电话,他怒不可遏的发出一声尖叫:``干什么?''

电话那边的丁丁被他这一嗓子吓成了结巴:``阿爸,我、我想问你什么时候回、回家?''

丁思汉听了他这一分钱不值的问候,当即把电话遥遥的掷向了保镖:``拿走,出去!''

保镖连忙接住电话退出卧室。丁思汉则是环抱双臂挡住胸口,始终是不习惯自己这老头子的外形与身份。

丁思汉左思右想,末了理智败给感情,还是决定再次攀岩进洞,去找无心。一旦无心到了手,他满可以带着人立刻离开此地,把鬼巫师远远的甩开。鬼巫师的诅咒毕竟不是精确制导武器,只要自己跑得够远够快,对方的咒术再厉害也是无用。

思及至此,他带领保镖们做了一番准备。留下两个最不顶用的小子看了家,他带着余下六人出了门。翻山越岭的走了许久,他们遥遥的望到了远方峭壁。一名保镖忽然大叫一声,伸手向前一指:``看!有人在往上爬!''

丁思汉举目远眺,果然看到光秃秃的峭壁上活动着两个人形黑点。小影子一上一下的拽着自己昨夜留下的长尼龙绳,其中上方一个已经用双手扒住了洞口边沿,正在扭动着身躯往里爬。下方的人影似乎是偏于笨手笨脚,双手抓着绳子双脚蹬着石头,蛤蟆似的向上连蹿带蹦。及至上头的人爬进洞中了,下面的蛤蟆向上伸出手,被跪在洞口的前锋军一把拽了上去。

丁思汉先不忙乱,从保镖手中要过望远镜,他通过望远镜凝神细看。远方情景瞬间近到了眼前,他一皱眉毛,发现进洞的二人竟是史家姐弟!

史丹凤他是见过的,虽然当时这具身体还不属于他,但他也有意识,也有记忆。史高飞给他的印象更深刻了,这个疯疯癫癫的东西居然自称是无心的父亲!想起来就要让人感到愤慨,因为他丁思汉都没有想过要给无心当爹!

丁思汉总觉得凭着他们姐弟的智商与本领,没有千里迢迢找到此处的可能。幕后的指使者也许就是白大千——白大千时而像个人物,时而像个白痴,让人始终是摸不清他的底细。也许真是真人不露相?丁思汉越想越细,越细越糊涂。这么多人都在找无心,简直要让他酸溜溜的生气了。

丁思汉不把史家姐弟往眼里放,带着保镖继续赶路。而史高飞和史丹凤一前一后的在洞中站稳了,史高飞依旧背着大旅行包,史丹凤也依旧背着小帐篷包。昨夜他们在林子里商量了一宿,实在是很想爬到洞里看一看,然而洞子快有十层楼高,又岂是能让人轻易爬上去的?

到了凌晨时分,大蝙蝠们乱哄哄的回了洞。史高飞和史丹凤缩在小帐篷里打了个盹儿。再次清醒过来之后,两人做了决定,打算先过去攀爬一次试试看。

两人从来没做过极限运动,全都没有信心。然而扯着绳子踩上了石头,他们一点一点的往上蹭,却发现这一片岩壁是出乎意料的好爬,总有凸起的大小石块让他们踩着借力。两人险伶伶的越爬越高,末了出乎他们意料的,竟然真上去了。

史高飞有点懒驴上磨屎尿多的意思,越是要紧的关头,越是能吃能拉。转身背对了史丹凤,他开口说道:``姐,我又饿了,你给我拿点儿吃的。''

史丹凤不但没能找到无心,还弄丢了小猫,上火上得心都满了,愁得食欲全无。从背包里掏出一小包饼干递给史高飞,她开口说道:``那些泡椒豆干你就别吃了,给无心留着吧,他最爱吃这些乱七八糟的东西。''

史高飞乖乖点头,深以为然。一边咔嚓咔嚓嚼着饼干,他一边领头向里走。地上薄薄的一层蝙蝠粪经了一夜的风吹,已经没了气味。史丹凤掏出两只口罩,自己戴一只,给了弟弟一只,权当防毒面具。口罩是史丹凤在一家小学校门口的地摊上买来的,通体黑色,只在嘴巴的位置画了上下两排白色大獠牙。史高飞戴了口罩,又打开了手电筒,兴致勃勃的往洞中走:``姐,等我们找到了宝宝,就马上回家给他去办周岁宴。不过这个周岁应该怎么算呢?是从他落到地球开始,还是从他出土开始?''

史丹凤经过了两日两夜的野人生活,一身的好衣服已经全没了好。伸着脖子弯着腰,她一边试探着往里走,一边不耐烦的说道:``闭嘴吧,怪臭的。''

史高飞十分惊诧:``我只不过是两天没刷牙而已,你隔着口罩都闻到我口臭了?''

史丹凤很无奈的转向了他:``我是说洞里臭,你看这墙根底下,全是屎。''

史高飞正想回答,可是话未出口,他忽然停了脚步,惊声叹道:``姐,看哪,好多蝙蝠在睡觉!''

史丹凤顺着晃动的手电筒光向前望,只见穹顶似的洞子上方密密匝匝挂满了大蝙蝠。身上骤然起了一层鸡皮疙瘩,她和弟弟一起龇牙咧嘴了:``哎呀,好恶心哪!''

史家姐弟难得的达成了一次共识,然而光是喊恶心也没有用,该走的路还是得走。照例还是史高飞打了前锋,两人拱肩缩背弯着腰,挑着中间的道路穿过蝙蝠阵。史高飞皱着鼻子,暗想地球真是让人呆不下去了,居然藏污纳垢的养了这许多丑蝙蝠。史丹凤紧随其后,挑着地势较高的石头尖落脚。拖着两脚沉重的蝙蝠粪,她对地球倒是没意见,只在心中暗暗痛惜:``我这鞋啊\ldots{}\ldots{}''

石洞有个好处,便是没有岔路,只要胆子壮,便能心无旁骛的一条道走到黑。两人小心翼翼的经过了无数正在酣睡的大蝙蝠,竟是没有惹出什么乱子。连着拐了几个弯,史高飞握着手电筒,头也不回的小声说道:``姐,我都被臭味熏得麻木了。''

史丹凤缩脖端腔高抬腿,嘁嘁喳喳的回应道:``唉,别提了,我刚才差点儿陷进了大粪里。''

史高飞一晃手电筒:``前边的蝙蝠越来越少了,姐,我们要不要试着喊一喊宝宝?''

史丹凤活了这么大,第一次冒这般的险。伸手扯住了弟弟背包的带子,她惴惴的不敢松手:``我早就想喊了,又怕他听了我们的声音会跑。小飞,你说他跑什么呢?''

史高飞以一种很科学的态度,东张西望的答道:``孩子有孩子的心事,家长不要过分干涉。姐,你看那边有个倒吊着的石头尖,是叫钟乳石吧?哈哈,还挺好看的,真是桂林山水甲天下啊!''

史丹凤听他说话东一句西一句的全不挨着,不禁感到十分烦恼:``桂个屁啊,别扯淡了。''

史高飞和史丹凤小心避开了大大小小的石笋,一路走得东倒西歪,虽然是时常在滑腻的地面上摔跤,但跌倒之后一翻身爬起来,并不耽误他们前进的速度。一口气不知走了多久,史高飞停了脚步:``姐,前边有个湖。''

史丹凤借着他的手电筒向前观望,手电筒不老实,光柱总是乱晃,于是她摸出了自己的小LED手电筒。手指拨动开关,一道细细的白光登时直照到了对面洞壁上。而在他们和洞壁之间,果然蓄着一大池水。

史丹凤比史高飞更有学问,此刻便忖度着说道:``这个\ldots{}\ldots{}叫做地下暗河吧?''

史高飞转动了手电筒的方向,想要看清暗河的全貌。原来他面前的这一片水,比池大比湖小,应该算是个中等尺寸的水潭。水潭的三面全是石壁,其中对面和左侧的石壁直上直下,而他们脚下的一面却是个斜坡。圆圆的水潭在右侧收了口,缩成了一条细长的水路继续向深处流淌,倒是一条名副其实的小河了。小河一侧还有窄窄的岸,高高低低的全是石头,只适合身怀轻功的高人行走。

史高飞试探着伸出了腿,想要沿着斜坡往水边走。史丹凤伸着手电筒往地下看,忽然发出一声惊呼,她一把揪住了史高飞的背包:``小飞,你低头看!''

史高飞不但低头,而且弯腰。昏黄的大光圈投在滑溜溜的石坡上,他看到了一大条子黑色痕迹,是有东西从岸边一直滑进水中,蹭掉了一路的青苔。

与此同时,史丹凤转身照向了来路——一路光顾着走了,竟然没想到看看地面有没有活物留下的痕迹。洞里黑漆漆的,单单薄薄的一道光线根本照不清远方的面貌。小心翼翼的横着挪了一步,她脚下忽然一滑。站稳之后向下一看,她看到了半截没了脑袋的死蛇。

惊叫被她咽进了喉咙里,只挤出``嘎''的一声余音。一转身面向了史高飞,她怀疑水里有吃肉的猛兽,正要把弟弟从斜坡拉扯上来。不料抬头一看,她发现史高飞不知何时又向下走了好几步,此刻竟然已经险伶伶的蹲在了水边。伸长一只手去撩了撩水,史高飞回头说道:``姐,这水好像挺干净。''

史丹凤对着他疯狂的招手:``你快上来,水里好像有蛇!''

史高飞把手电筒夹到腋下,想要摘了脸上的口罩喘口气,可是刚刚抬手摸到耳朵,他却是歪着脑袋骤然愣住了。

手电筒的光芒斜斜射入水面,在波光粼粼的清澈水中,他看到了无心的眼睛!

无心悬浮在水潭的一角,面无表情的仰脸凝视着他,不知已经看了多久。史高飞怔怔的和他对视了一瞬,随即大叫出声,张牙舞爪的就扑向了水中。史丹凤吓了一跳,蹭下斜坡想要揪住他。然而她追不上史高飞,史高飞也追不上无心。一只脚踩进水中,他眼看着水下白影一闪,无心的黑眼睛不见了。

史高飞发了疯。

他在水中乱踢乱打,乱捞乱抓,又把口罩摘下来狠狠掼到水中:``姐,全怪你,非得让我带这个破口罩!宝宝肯定是被我们吓跑了!''

史丹凤没有亲眼见到无心,所以不知道他疯得有没有理。手足无措的站在岸边,她被愁绪和弟弟内外交攻,恨不能一头扎进水里淹死。手忙脚乱的下了石坡,她试图拽住想要下水的弟弟,可是未等她拽着弟弟的背包带子发力,身后忽然起了声音:``史小姐,史先生,你们发现他了?''

史丹凤和史高飞登时统一的做了个向后转。黑暗之中活跃着七长八短的光束,其中一道光自上而下的直射洞顶,中间正是托出了丁思汉的面孔。目光锐利的盯着史家姐弟,他点头一笑:``好久不见了。''

史高飞本来打算对着他姐发疯,如今见了丁思汉,他立刻换了对象。抬手对着丁思汉一指,他高声咆哮道:``你这老不死的鸭子精!''

然后他把身后的背包往下一甩,拉开拉链抽出砍刀。史丹凤一把从后搂住了他的腰:``别去,人家有枪!''

丁思汉压下了身后保镖抬起来的散弹枪枪管,倒是保持了良好的风度:``白大师来了吗?''

史高飞虽然满心狂怒,但是见了对方的枪口,他很识相的放下了砍刀,心中暗想:``我不能和这帮地球人硬碰硬,我要是死了,宝宝就变成孤儿了。等我以后占了上风,再剁掉老鸭子的杂毛脑袋!''

``他来个屁!''史高飞对丁思汉嚷道:``宝宝又不是他的儿子!''

丁思汉点了点头,显而易见的事实反倒容易让人心生疑虑,史高飞如此明目张胆的胡言乱语,让他怀疑对方也是个高人。手里拄着一根充作登山杖的粗木棍,他打算再酝酿几句话敲打敲打对方的底细,可是在他开口之前,水潭里忽然咕嘟嘟的冒了泡开了锅。史丹凤之所以一直没言语,就是因为感觉水潭里安静得不对劲。如今终于生了变化,她如同吞了弹簧一般,条件反射似的猛然一窜,力大无穷的推着弟弟往上跑。丁思汉等人也下意识的跟着后退了几步,可是站定之后再看,一潭的水哗啦啦的打了漩涡,可是并没有继续卷出大浪。

``怎么回事?''丁思汉喃喃自语:``难道水里有东西?''

史高飞拎着砍刀站稳当了,眼看黑沉沉的水面上,一个漩涡眼越转越浅,最后消失在了那条通往洞内深处的暗河之中。忽然把丁思汉抛到了脑后,他一咬牙下了决心。弯腰放下砍刀和手电筒,他解开鞋带倒了倒水,然后重新穿好直起了腰。

``姐,我们沿着河走。''他抬手指向暗河:``宝宝一定是被我们吓得逃跑了。''

史丹凤感觉他这话完全没有准,不过因为走投无路,所以愿意试试弟弟的疯主意。丁思汉站在远处听得清楚,知道他们肯定是已经捕捉到了无心的影踪——既然无心的确是在这座洞中,那就一切都好办了。

在手电筒的光线

边缘,他的耳朵耸了一下。通风报信的小鬼正在叽叽喳喳的向他说话,在洞子的极深处,在暗河尽头的石头岸上,小鬼发现了陌生的鬼魂。鬼魂已经快要修炼成煞,也许正是主人所要寻找的鬼巫师。

丁思汉蹲了下去,在四面八方的手电筒照耀下,他打开了随身携带的背包。虽然隔行如隔山,但是鬼巫师再厉害也只是个鬼。对于诅咒,他只有招架的份,可对于鬼,他素来很有手段。

史家姐弟侧身踏上了暗河边的尖锐石头,丁思汉也在手电筒下摆开了道场。与此同时,暗河远方的水面上水花一闪,是水中的无心探出了头。

仰起脸面向了前方拐角处的石壁,他静静的望着悬浮于半空中的白色鬼魂。他是水淋淋的,那鬼魂也是水淋淋的,带着生机勃勃的邪气。闭着眼睛抱住了肩膀,他在刺骨的寒意打了个冷战。

那鬼魂很美,像是一轮温柔的明月化成了人形。可是他已经看不出美丑,睁开眼睛的时候,他只想逃只想躲。

宛如堕入了饿鬼道,他如今最清晰的感觉便是饥渴与恐慌。在水中无声的退却了,闭上眼睛的时候,他倒是感觉对方的气息仿佛存有几丝亲切和熟悉。瘦削的脊梁骨划开水面,他向后一直退到了暗河一边的石壁上。

他想离开,想要沉入水中,鱼一样的迅速溜走,钻进更深更远更黑暗的地方去。可是一丝说不清道不明的留恋让他留在了原地。肩胛骨轻轻磕打磨蹭着粗糙的石头,他怕到了浑身颤抖的地步。鼻尖掠过隐隐的阴风,是那鬼魂向下靠近了他。

他再次睁开了眼睛,看到鬼魂盘腿降落到了水面。将右胳膊肘架在了膝盖上,鬼魂倾斜身体歪着脑袋,从凌乱披散的潮湿长发中向他一笑,然后抬起左手,作势摸他:``无心,你的脸怎么了?你被人扒了皮吗?''

无心姿态僵硬的微微一扭头,仿佛是想要避开对方的触碰。于是那鬼魂又说话了:``无心,你还在记恨我?''

话音落下,他收回右手一拍膝盖,毫无预兆的笑出了声音:``扎西贡布在天亮之前告诉我你在洞里,我从凌晨找到现在,终于找到了你。不要生气啦,无心,你当然比猫头鹰重要。真是有趣,你竟然和一只鸟赌气。哈哈。''

无心终于开了口,声音很轻很哑:``我忘记了你是谁。''

鬼魂收敛了笑容,用蓝眼睛很认真的看了他半晌,末了答道:``我是白琉璃,我来救你,我还会给你报仇。''

无心垂下眼帘,偏过脸面对着墨汁一样漆黑深沉的水面,口中轻声自语:``白琉璃\ldots{}\ldots{}''

白琉璃又对他伸出了手,他抖得厉害,仿佛是在害冷。白琉璃想给他一点温暖,可惜自己也没有热度,只是一团阴冷的鬼影。苍白的手徒劳的穿过了无心的头脸,他无能为力的叹了口气:``再过几个月,或者几年,等我有了身体,就抱你一下,再吃顿重庆火锅。''

无心定定的看着他,不是很信他,也不是很怕他。身体缓缓沉入水下,他不置可否的藏在了一道石缝之中。

白琉璃很孤独的悬在水上,声音很低的自言自语:``龟儿子,竟然不理老子。''

\chapter{吸血鬼}

白琉璃言而有信,说要给无心报仇,就一定不会半路收兵。无心沉在水中,不理他,不离开,也不露面。他不知道对方这是在闹哪一出,也懒得问,更懒得哄。几十年相处下来,白琉璃发现无心仿佛是有点贱性,如果过分的善待他了,他很可能会得寸进尺的讨人厌。

他顺着水流的方向飘远了,想要找个清净地方,专心致志的作法念咒。把一个活人从有到无的活活咒死,实在不是一件容易事情,尤其对方也不是无能的善类。洞子里黑漆漆的永远不见天日,即便是在正午时分,阴气也重得如同午夜。白琉璃很喜欢这种环境,只是遗憾自己没有身体,好些本事都不能施展。如果他有身体——哪怕只有一只手呢,也能多出好几种方法来替无心报仇。

但是现在想不得那许多了,他只有念力可以运用。念咒实在是件耗精力的事情,当年在西康和扎西贡布斗法,因为双方都是有备而战无懈可击,他无计可施,只好硬着头皮足足念了十天的咒。等到扎西贡布通体乌黑的死去时,他累得气息奄奄,也算是丢了大半条命。如果当时他丢了整条性命,也不稀奇,也无话可说。横竖是个愿赌服输的事情,所以如今扎西贡布再见了他,也是一样的没怨气。

暗河的河床越来越高,河水越流越浅,最后断在了一片斜斜的石滩上。沿着石滩往里走,还有着深不可测的空间。白琉璃不肯再在路途上面浪费时间了,向上一直升到了洞顶,他停留在了几根尖锐的钟乳石间。摆好了架势正要开工,身下的暗河却是有了动静。白琉璃垂下头,看到一道乌黑的脊背在水面上一闪而逝,不像蛇,也不像鱼,体积仿佛是非常的大,然而很灵动轻巧,只让暗河涨潮似的漾了几波。

白琉璃望着水面出了一会儿神。一只小鬼在远方探头缩脑的窥视着他,看他始终是一动不动,便奓着胆子靠近又靠近。及至近到了相当的程度,白琉璃身形一闪,随即小鬼消失无踪,正是被他吞了。

然而小鬼是死不绝的,在他闭目凝神之时,又来了几只小鬼,远远的悬在洞顶,一声不响的静盯着他。

白琉璃开始念咒,念得前仰后合如痴如醉,如此只过了几分钟,遥远处的丁思汉便有感觉了。头脸的粗细血管一起肿胀硬化成了一张网,冷森森的束缚着他的血肉。于是他加快了速度。把刚刚画好的一沓血符摆在正前方,他又拿起最后一张黄纸摁在了地面上。刺破了的中指指尖往纸上一点,他随即``咝''的吸了一口凉气,同时像被烫着了似的,猛然高高的抬起了手。

他画符是画得太熟了,饶是手抬得快,可在方才的一瞬间里,他还是在纸上弯弯曲曲的抹了一下子,留下的痕迹不是红色,而是黑色。立刻掏出打火机把纸烧了,他心中一阵乱跳——血符借的就是鲜血中的一股子阳气,鲜血加上念力,算是双保险。可如今鲜血变成了毒血,谁知道会画出一张什么邪符?

让个牛似的大个子保镖割破了中指,丁思汉又抽出一张黄纸,蘸着他的鲜血把余下的血符画完。外人的血到底是外人的血,比不得自己的鲜血纯粹,堪称美中不足,但是无可奈何,只得如此。拿着厚厚一沓血符站起了身,他虽然没有照镜子,但是很有自知之名的避开了保镖的手电筒。征途刚刚开始,战场尚未到达,他不想提前吓走了自己的军队。

领着六个生龙活虎的小伙子,他人在前方,头也不回的说道:``走,我们去追他们。''

然后低着一张黑网密布的恐怖面孔,他返老还童一般,大踏步的率先前进了。保镖们当即不假思索的追上——跟着丁老先生混久了,他们什么没见过?

沿着斜坡向右走,直接能走到暗河右侧的石岸。石岸太窄了,大模大样的走肯定是不行,侧身背靠着岸边石壁横着走,也有困难。六名保镖加上丁思汉,一起效仿了螃蟹。手电筒的白色光束满洞里乱晃,没有一支是能照到点子上的。丁思汉从裤兜里掏出一支神火手电筒——他这一支是真货,保镖手里的全是山寨货。

手电筒的光芒直射前方,他想寻找史家姐弟的踪迹,然而前方影影绰绰的是一堵石墙,原来暗河在前头来了个急拐弯,史家姐弟如果没有掉进河里淹死的话,想必就是已然拐弯走远了。

丁思汉一挑眉毛,心想这两个资质平庸的货都能走得太平,可见前途道路崎岖得有限,只要小心一点,还是有路可走。心中燃起了一股子希望的小火苗,他来了精神。抽出一张血符平铺在左掌心中,他念念有词的用右手拇指重新描了血符一遍,随即猛一甩手。血符平平的飞过暗河,无声的粘在了对面的嶙峋石壁上。

丁思汉长吁了一口气,然后继续横着往前挪。一只鬼魂敢把自己逼到这般地步,显然是采取了杀敌一千自损八百的战术。自己先挺着熬着,等到真把他找到了,再和他当面锣对面鼓的打一场。到时候,他不逃,没有生路;他逃,生路却又被自己布了阵,逃命等于自投罗网。在无边无际的大石山中,鬼们穿墙遁地的本领全都等于了零。墙才多厚?山又有多厚?反正凭他几世的经历来看,他还没有见过能穿山的鬼魂。

险伶伶的走到了前方拐角,丁思汉往洞顶又甩出一张血符。面前水流平稳,脚下可以用来借力的石块石笋也多不胜数。他平平安安的拐了弯,这时用手电筒再往前一照,他看到了史家姐弟的背影。

史家姐弟距离他们太远了,不过也是同样的做螃蟹状,并且把背包全反背到了胸前,以便让后背贴住石壁,站得更稳当。丁思汉把他们当成了自己的先遣军,他们走得越快越远,说明自己的路途越平坦。

他因为年老体衰了,所以格外的小心,一步一步都是看清了才落。旁边的保镖们由于太灵活,反倒吃了粗心大意的亏。一名大个子一脚绊在了凸起的石笋上,摇晃着向前一栽,膝盖和头脸全都拍进了河水里。同伴成了他的救命恩人,一把揪住了他的腰带,把他硬生生的拽回了原位。大个子方才刚被先生选中割破了手指,如今又差点落了水,自己也觉着怪倒霉的,不由得一边挤着浸了河水的手指伤口,一边讪讪的苦笑。正当此时,河面忽然起了一溜波动,像是水底射过了一只长箭。丁思汉停了脚步想要看个分明,不料前方的水面上猛的爆发出了一朵大浪,浪花之中一只通体乌黑的活物昂然而起,七只手电筒的光芒汇聚到了一处,白光之中只见那活物长条条的足有水缸粗细,然而非蛇非鱼,周身软腻腻的乌黑发亮,一圈一圈有着无数的环节。高昂的头上无眼无口,笼统的只是一只边缘外翻的吸盘,吸盘中央活动着三瓣软颚。软腭本是围成一圈,可是居高临下的对准了岸边活人,那软腭骤然扩大到了极致,居然足有大脸盆大。在众人的惊呼声中,怪物目标明确的直扑而下,软皮管子似的直接套住了大个子。大个子攥着自己受了伤的手指头,呆呆的连叫都没有叫出一声,便被怪物吞进了肚子里去,只留下一双穿着运动鞋的脚伸在外面。而怪物的口颚立刻收缩,让那一双脚也迅速下沉入了它的腹中。在三瓣颚片合拢之时,口颚之中仿佛包不住了似的,喷出了一线细细的鲜血。鲜血从天而降,正好洒到了大个子的救命恩人身上。救命恩人的年纪也不大,顶着满头满脸的甜腥鲜血仰着头,他崩溃似的嚎叫了一声,随即举起手中的散弹枪扣动了扳机。弹丸打在柔软的怪物身躯上,竟是毫无杀伤力。而怪物再次昂首扑向下方,一口吞了这半身鲜血的新猎物。

在这天下大乱的时候,丁思汉忽然压低声音喝道:``不要动,不要叫!''

他说话还是有分量的,保镖们立刻全噤了声。而那怪物沉入水中,也不知道是走没走,总之水面缓缓恢复了平静,仿佛先前什么事情都没有发生过。

丁思汉双腿打颤,继续横着往前走,一边走一边咬牙说道:``你们说,它像什么?''

保镖们沉默了片刻,殿后的一个人伸着脖子作了回答:``像蚂蝗。''

中间也有人说了话:``蚂蝗不可能长这么大,也许是蟒蛇吧?''

没人搭茬了,于是丁思汉轻声说道:``我看\ldots{}\ldots{}也是蚂蝗。''

后面的话他不说了——蚂蝗嗜血,而死掉的两个小子,无一例外的全都沾了血。当然自己身上也有伤,也会有血腥气,可是鬼巫师的诅咒让自己的血液变了质,也许自己因祸得福,反而捡了一条老命。

史高飞和史丹凤依稀听到了身后的狂呼乱叫,但丁思汉看得清他们,他们凭着手里一大一小两只粗制滥造的手电筒,却是看不清丁思汉等人。两人一前一后的横着走,走得还挺稳当,只是身后的石壁越发不平了,移动之时不是前仰就是后合。史高飞弯了腰,撅着屁股从一块凌空突出的大石头下蹭过。史丹凤瞟了他一眼,当即开始唠叨:``腿不能再往下弯着点儿吗?大屁股撅那么高,怕石头尖刮不破你的裤子?''

史高飞是副大骨架子,方才已经是极力的蜷缩了,听了史丹凤的话,他下降成了半蹲之势,同时不耐烦的作出答复:``姐你真烦人。''

两人全是个要吵架的语气,其实并没有要吵架的打算。史丹凤对弟弟是一贯的不肯客气,史高飞对于姐姐也从来不知尊敬。前方又出现了一根斜刺向上的大石笋,史高飞纵身一跃跳了过去,跳过之后自己纳罕,没想到自己轻功盖世。史丹凤没有他的本领,对着石笋做出种种姿势,怎么着都是过不去。史高飞正要伸手拉她一把,可是藉着手电筒的光芒,他忽然发现水中又闪过了白色影子。一大步跳入水中,他伸展双臂做了个自由泳的姿态,想要乘风破浪直追上前。不料双脚结结实实的落了地,他低头一看,发现河水竟然只没过了自己的腰。眼看白影蜿蜒着要游远了,他双腿运力向前一蹦,直挺挺的拍向了前方。手指下意识的猛一合拢,他紧紧抓住了无心的脚踝。一只脚卡在河底的石头缝里,他动弹不得,反倒占了便宜。运足力气大喝一声,他拼命的往回一收手。水面起了一线雪白的浪,他把无心搂到了怀里:``哈!宝宝!''

无心的上半身被他搂住了,实在动弹不得,只能活动下面两条□长腿。双脚惊恐的蹬住岸边石头,他在史高飞的怀里摇头摆尾。史高飞的力量和温度都让他感到了无比的怕。水淋淋的双手推开了对方兴高采烈的笑脸,他怒不可遏的睁大了双眼,紧接着扭头一口咬上了史高飞的手臂。

史高飞穿着一件薄薄的棉服,如今挨了他这狠狠的一口,虽然隔着几层布棉,不至于受伤,但还像是被人掐了一把似的,疼得他扯着嗓子嚎叫了一声。无心仰起脑袋使劲一晃,从口中吐出一片碎布和几缕棉花。身体依旧被对方的手臂紧箍着,他困兽一般的再次抬头,这回一口咬上了史高飞的面颊。

史高飞方才被咬破了衣服,叫得声震云霄;如今被咬到了肉,反倒沉默了。紧锁眉头忍住脸上剧痛,他死死的抱着无心,就是不松手。温暖的鲜血顺着伤口流入了无心的口中,带着热度。无心吮了一下,然后缓缓的松了口,忽然感觉这个怀抱似曾相识。

他歪着脑袋去看史高飞的脸,史高飞的老式手电筒落进水中,已然熄灭。借着史丹凤的手电筒光芒,史高飞看不清他,他却能看清史高飞。怔怔的端详了对方的面孔,他没看出什么来,只见那张脸上印着一圈血红的牙印。鲜血慢慢的又渗出来了,顺着面颊往下淌。于是他下意识的凑上去,噙住伤口又吮了一口。

史高飞腾出一只手,去摩挲了他的脸:``宝宝,我是爸爸啊!你不要怕,爸爸来救你了。''

深深的低下头,史高飞亲了亲他的额头和眉毛:``爸爸来了,姐姐也来了。我们带你回家去。''

无心睁大眼睛瞪着他,忽然左右为难的痛苦了。对方是个人,而他不知道自己还应不应该继续和人在一起。本能似的战栗了,他还是想远离。

他抬手推了推史高飞的胸膛,推不动,胸膛湿淋淋的宽硬成了一堵墙。史高飞开始一步一步的向后退,一直退到了岸边。一屁股坐在了一块大石头上,他把无心的双腿也托上了岸。

史丹凤已然越过了石笋。小心翼翼的在弟弟身边蹲下了,她试探着抬手摸了摸无心的湿头发,然而她刚一摸,无心便在史高飞怀中猛然一挣。黑眼珠子向上翻去,他懵懂惶惑的去看史丹凤。

史丹凤看清了他的脸,惊讶的``哟''了一声。史高飞也看清了,双脚在水中有一搭没一搭的踢着水花,他柔声问道:``宝宝,你怎么不认识爸爸了?''

无心不回答,单是警惕的看着他们。史丹凤忍不住,又去摸了他的头发:``小飞,你看他的脸,肯定是被那个老不死的欺负狠了。''

史高飞用手背轻轻去蹭无心的脸:``宝宝,你不要怕。虽然你现在看起来像遭了核辐射似的,不过在爸爸眼中,你永远都像刚出土的那晚一样可爱。''然后他用手指一点无心的半边白脸:``牛奶。''再一点无心的半边粉脸:``草莓。''

话音落下,他自己嘿嘿的笑了,向史丹凤寻求共鸣:``搅一搅就是草莓奶昔了。''

史丹凤叹了口气,一瞬间把往事全想起来了。望着无心抚今思昔,她心里一阵难受:``原来都长得好好的了,现在被人祸害成了这样子,连我们都不认识了。''话音落下她忽然史一彪附体,粗豪的骂道:``就应该剁了姓丁那个老东西!做大孽的,不得好死!''

拉过无心推在史高飞胸膛上的右手,史丹凤细细的瞧,瞧到最后又送到了史高飞眼前:``你看,你看,手没了一半,气死我了!''

史高飞忽然明白了事理,一本正经的教训他

姐:``你不要吵了,我们得先把宝宝送到安全地方去。他现在被吓坏了,万一一会儿又跑了怎么办?真是的,难道我不知道应该剁了丁思汉?事情总得一样一样的办啊,我先走后剁行不行?''

史丹凤被他说得哑口无言,于是只好迂回的进行反击:``你总拨我的手干什么?我不能碰他了?''

史高飞抱着无心一侧身:``我的!''

史丹凤嘴上不说,心中暗骂:``你的?他给你买结婚戒指了吗?''

无心仰卧在史高飞的臂弯里,只感觉这个姿势很熟悉,头顶上方的女人散发着似有似无的甜香气味,这也很熟悉。于是他瑟缩着躺住了,并没有再挣扎。

史高飞把背包给了史丹凤,自己则是把无心背了起来。姐弟二人既然找到了无心,便打算沿着原路返回。然而想到半路上的丁思汉,他们又犯了难——对方人多势众,自己实在不是对手。

无心忽然开了口,他轻声说道:``白琉璃。''

史高飞没听清楚,侧过脸发出疑问:``嗯?''

无心趴在史高飞的后背上,茫然之中只感觉自己不能停留,应该顺着河流的方向往前走。暗河的尽头有个白琉璃,他隐约感觉自己和白琉璃之间有着很长久的交情,所以不能把白琉璃独自留在那里。

按照无心的指示,史高飞和史丹凤决定继续往前走。万一前方有好地方可以藏身,他们也可以等丁思汉等人走过去了,再偷偷的踏上归途。史丹凤依旧是背对着石壁,手里拎着弟弟的旅行包,胸前挂着自己的帐篷包。史高飞则是转换姿势面对了石壁,因为背后多了个无心。扭头面对着前方,一束光线从史丹凤举起的手电筒中射出来,遥遥的给他照亮了脚下地势。

无心转过了脸,默默的去看史丹凤。其实在最初的时候,史丹凤根本没把他当个东西看,可是不知怎的,他从个吃货怪物变成了宝贝。他每遭一次难,她就要忍不住多疼他几分。此刻歪着脑袋面对着无心,她生怕对方真的再也不认识了自己,所以拼命的向他微笑,笑得龇牙咧嘴。再美的女人也禁不住这么自我丑化,于是无心收紧了环在史高飞脖子上的双臂,感觉史丹凤是要吃人。

史高飞一伸舌头,险些被他勒断了气。把背后的无心又往上托了托,他一步一步横挪得十分来劲。鲜血还在顺着面颊往下淌,他光顾着高兴了,也觉不出疼。无心嗅着淡淡的血腥气,忽然说道:``快走!''

话音落下,他伸长舌头,在史高飞的脸上又舔了一口。史高飞不介意,一边迈大步,一边问道:``宝宝,你像个小吸血鬼似的,是不是饿了?''

无心哑着嗓子,扭过脸望向了史丹凤:``吸血鬼在水下,快走!''

\chapter{攻击}

史高飞很听儿子的话,儿子让他``快走'',他横着调动了两条长腿,当真是把速度加快到了极致,因为方才在暗河中灌了两鞋的水,所以他踩得一步一咕唧,走得还挺热闹。史丹凤的身材比他小了一号,也比他更柔软苗条,走起险路反倒占了便宜。高抬腿轻落步的迈过一根根石笋,她分量轻,无论大石头小石头,全能禁得住她。

史高飞侧着脸往前头看,脸蛋痒痒的,是无心伸长了脖子和舌头在舔他的伤口。冰凉的舌头湿漉漉的拖过痛处,他忍不住想要笑,感觉儿子像只小狗。空气中弥漫开了隐隐的甜腥气味,是无心咬破了舌尖,要用自己的鲜血盖住史高飞散发出的人血腥气。

险伶伶的又拐了一个急弯,史高飞的立足之地细成了窄窄一道,双手托着无心的大腿,他极力的保持平衡往直了站,然而鼻尖还是将要蹭到粗糙石壁。正是摇晃着要落水之时,他忽然感觉手中一滑背上一轻,抬头看时,竟是无心摁着他的肩头向上一窜,猴子似的攀到了斜上方的石壁上。

史高飞和史丹凤一起惊呼了,史丹凤一举手电筒,高声叫道:``又要跑?''

史高飞同时也开了口:``不要跑!''

无心灵活的换了个大头朝下的姿势,壁虎似的贴在了石壁上。居高临下的望着他们,他的嘴唇动了动,最终却是没能说出话。手指脚趾一起抓着凹凸石块,他猛的转身向前爬了一米多远,随即回头去看史家姐弟。

史高飞还在发怔,史丹凤出于女性的直觉,却是立时明白了:``走,小飞,他给咱们带路呢!也可能是怕你累着!''

史高飞立时感激涕零:``真是大孝子!''

经过了一段特别崎岖的石头路,史丹凤眼看河水似乎是越来越浅了,心头不由得一阵轻松。而史高飞一直眼睁睁的盯着上方的无心,生怕儿子再逃了。末了石头路实在是窄得走不成,史高飞试探着下了水,发现水面刚刚没过小腿。很痛快的踢出一溜水花,他出声唤道:``姐,下来走!''

史丹凤宁可踩蝙蝠粪也不愿意淌水:``你快给我上岸,水里也许有蛇呢!''

史高飞背对着史丹凤答道:``就不上就不上!''紧接着抬头对石壁上的无心张开了双臂:``宝宝,来,爸爸背你走!''

听了史高飞的呼唤,史丹凤也把手电筒转向了无心。无心雪白的缠在一根树干粗的钟乳石上,迎着手电筒的光芒,他的黑眼睛骤然一亮,随即低声吼道:``走!''

史丹凤看他像条蛇似的,正在担心他会掉下来摔出个好歹,冷不防听了他恶狠狠的催促,虽然心里没能领会意思,但是一双脚比脑子更有主意,自作主张的先加了紧。史高飞也噼里啪啦的由走变跑,然而没等他跑出几步,身后骤然起了一阵狂风暴雨,大浪劈头盖脸的把他浇了个透心凉。在史丹凤的惊呼声中转过身,他抹着脸上的水仰脸一瞧,当即叫道:``真他妈丑啊!''

类似蚂蝗的大水怪不知道是什么时候潜随而来,居然始终无声无息。此刻它毫无预兆的亮了相,故技重施的张开三瓣口颚,居高临下的俯冲向了史高飞。史丹凤看在眼里,眼都红了,不假思索的就要往水里跑,可是在她动作之前,无心忽然挟着疾风从天而降,把手伸向了她拎着的大旅行包。旅行包里还存着大半包的零食,拉链没有拉严,一处开口中露出了砍刀的刀柄。无心握住刀柄纵身向上一跃。双脚蹬着石壁狠狠的借了力,他直接横窜出去,直奔了水怪张开的巨口。在落入巨口的同时,他将砍刀横架在了巨口两端,双手死死扳住刀背,他由着刀锋切上了坚韧滑腻的水怪口颚。身体陷入怪物的体腔,他只觉体腔内壁的软肉分泌了黏液,正在一点一点的把自己往深处吸。如同陷入了沼泽中一般,他把全部力气都运用到了双手上。趁着一个脑袋还露在口颚之外,他大声喊道:``走啊!''

史丹凤吓得面无人色,脑筋已经停了转,完全是依靠着本能跳入水中。一把抓住了史高飞的手臂,她尖锥锥的抬头高叫道:``无心,跳出来!快!''

无心手扳着刀背,极力想要把身体往上撑,因为将要力不能支了,所以他发出了一声怒吼:``走!''

史丹凤被他这一嗓子震得一哆嗦,随即像条听话的好猎犬似的,拽着史高飞扭头就往前跑。无心在水怪口中越陷越深,眼看刀尖已经斜斜的滑过口颚边缘要和自己一起落肚了,他正是无计可施,然而脚下忽然有了着力点,不知是什么东西堵在了水怪的体腔里,又冷又滑软中带硬。无心的两只脚有了着落,站稳之后拉长了身体,他把砍刀向上举了举,让它重新横架上了口颚两端。及至砍刀稳当了,他重新把力气全运上了双手,扳着刀背缓缓向上,拼了命的对抗水怪体内蠕蠕的吸力。

他赤身露体长条条的,类似一条水蛇,在水怪的体腔之中上下都容易。滑溜溜的从水怪的口颚中探出了上半身,他俯身趴在水怪的吸盘之上向下一溜。带着满身粘稠的液体,他``扑通''一声向下落入水中。一个鲤鱼打挺爬起身,他并不直接狂奔,而是斜斜的冲向了一侧石壁。后方水怪再昂着吸盘大口去追他时,他已经窜上石壁高处,隐没在了无数倒垂着的钟乳石中。

水怪似乎是毫无智慧,一味只是昂首去追,对着无心藏身的方向发出猛冲。尖锐的钟乳石以刚克柔的迎接了它的大口。水怪很快撞碎了几根奇长的钟乳石,而锋利的石头渣滓显然也没饶了水怪。水怪毕竟是肉做的,几个回合之后它趴伏回了水中,环节密布的后背忽然起了涌动,它张开大口,却是缓缓的吐出了两具结了茧一般的尸首。浑圆的身体渐渐的扁了,它像一片柔软的叶子一样,波浪起伏的向后退去,很快和漆黑的暗河融为了一体。

无心静候了片刻,见水中是真安静了,便跳跃着落了地。低头看了看躺在浅水中的尸首,他见尸首从头到尾亮晶晶的蒙了一层透明的黏胶,皮肤则是干枯的紧贴着骨骼,面部表情尤其狰狞,全是死不瞑目的样子。

无心看清之后不再停留,转而一路狂奔向前。在暗河尽头的石岸上,他和史家姐弟会了面。

史丹凤还在发傻,方才跑着跑着忽然不跑了,因为感觉自己跑得不对。史高飞糊里糊涂的被她扯出老远,如今她骤然刹了闸,史高飞也跟着停了步子。忽见无心鬼影似的追上来了,史丹凤像是服了一剂活血化瘀的猛药一般,周身经脉立刻畅通了,脑子里也有了思想。史高飞则是``呼''的长出了一口气。

两个人一起往前迎,都想第一个拉住无心。然而无心半路拐了弯,一边拐弯,一边又向他们招了招手。他不想让他们死,可若是由着他们自由行动,又会必死无疑,所以他打算找个好地方安置他们。石岸边的石壁上开了个离地两米高的孔洞,洞前有钟乳石垂下,堪称是天然的掩护。无心让史丹凤先上,史丹凤懵里懵懂的,一脚踩着石头,一脚没地方蹬,正是上得为难,冷不防一双手推了她的屁股。她回头一瞧,见无心半蹲着身体,正要咬着牙齿向上托举自己。

史丹凤笨手笨脚的爬上去了,心里特别的舒服,暗想:``别看他没和我说话,其实他还是和我好。''

随后史高飞也进了孔洞。两人在里面抱着膝盖蜷成一团,正要极力的给无心匀出空间,不料无心站在下方,将一根手指竖到唇边,向他们``嘘''了一声。

史高飞会意,压低声音说道:``宝宝,上来呀。''

无心摇了摇头,随即答道:``危险,不要出声,等我来找你们。''

然后他一转身,鬼影似的瞬间一闪。史丹凤再拿手电筒去照耀,洞外地面上已经空无一物。

谨记着无心的嘱咐,史丹凤关了手电筒,拼命的让弟弟往里缩。孔洞实在是小,稍不留神就要露出胳膊腿儿。史丹凤轻声说道:``小飞,你把腿再往里收一收。咱们不能给无心帮忙,也不能给无心添乱。刚才多危险哪,你说那东西是什么?是不是蛇?''

史高飞答道:``姐你别说话,宝宝不让我们出声。''

史丹凤果然闭了嘴——刚才受的刺激太大,她居然麻木不仁的没有很怕。恐怖情景存在她的脑子里,此刻正好让她慢慢的消化消化。

知道史家姐弟会乖乖的窝在洞内了,无心仿佛放下了一桩大心事似的,从头到脚一阵轻松。理智已经失去了,他只好凭着直觉行事。直觉告诉他史家姐弟是好的,那么他就相信他们是真的好。

一个人走在黑暗中,他遥遥的望见了一抹白光。心中无端的快乐了,他连跑带跳的到了白光近前,仰起头小声叫道:``白琉璃。''

白琉璃忙着念咒,和他也没有什么话说,于是闭着眼睛垂着脑袋,没有理睬他。

无心虽然还是一脑子乱麻,但是感觉对方是个很亲切的鬼魂。他不理睬自己,自己仿佛习以为常了似的,也并不生气:``我\ldots{}\ldots{}我刚才遇到了我爸,我爸对我很好。''

白琉璃用一根食指轻轻一敲膝盖,算作回答。

无心抬手攥住一根倒垂下来的钟乳石,自顾自的又道:``我也想吃火锅了——我什么都想吃,我要饿死了。''

他很认真的向白琉璃征求意见:``你说,如果我去向我爸要东西吃,他会不会给我?''

白琉璃睁开了一只眼睛看他:``吵死了,走开。''

无心不走,执着的又问:``你在干什么?''

白琉璃把睁开的眼睛重新闭了上:``我在给你报仇。''

无心歪着脑袋想了又想,最后自己点了点头。一双眼睛忽然黑出了贼光,已经缓缓消退了的兽性重新复燃,他饿极了,不但想要生吞活剥,而且还要敲骨吸髓——只是不知道他的仇人兼猎物应该是谁。

无心喃喃的和白琉璃说话,因为自己的思路太乱,所以想要请白琉璃让自己清醒清醒。可白琉璃并没有爱心和他抚今思昔嚼舌头。对着嗡嗡乱叫的无心猛一挥手,他很不耐烦的蹙起了两道长眉。而无心身不由己的向后直飞,结结实实的撞到了一块大石头上。

滚落在地伸长了两条腿,无心六神无主的坐起身,望着前方又道:``白琉璃,你知道吗?水里有一条大蚂蝗,那么大。那么大的蚂蝗还是蚂蝗吗?不是蚂蝗了吧?''

一句话让他说的颠颠倒倒啰啰嗦嗦,但是他自得其乐,说得甚至忘记了饥饿。白琉璃不肯分心,有一搭没一搭的告诉他:``是蚂蝗。''

无心摇了摇头:``太大了,还是蚂蝗?''

白琉璃第无数次的发现无心是真烦人,恨不能找块大石头一举将他砸晕:``不是普通的蚂蝗\ldots{}\ldots{}是蛊中之精\ldots{}\ldots{}此地属阴适宜养蛊\ldots{}\ldots{}别和我说话。''

无心听到这里,脑子忽然灵了:``蛊?既然是蛊,就必定有养蛊的人。养蛊的人在哪里?''

白琉璃被他问得愁容满面,简直快要发火:``不知道,如果活着,一定不会远;如果死了,就不一定了。''

无心和白琉璃有问有答的说了半天话,感觉自己似乎是越来越聪明了,甚至已经能够开始思考:``不会远\ldots{}\ldots{}对,养蛊不容易,养成了的蛊虫,谁会舍得随便抛弃?不会远\ldots{}\ldots{}''他扭头望向了漆黑的洞中深处:``你说这座洞子会通到哪里去?里面会不会还有活物?一直走下去的话,能不能找到养蛊的人?我去走着试试看,如果能够走出一条新路,我就不必去杀蚂蝗了。''

说到这里,他一翻身爬起来,当真是攀援跳跃着冲入了黑暗。白琉璃面无表情的撩了他一眼,心想:``终于滚了。''

然后将手指搭上膝盖,他集中了全部精力继续念咒。一团幽幽的寒气笼罩了他的全身,先前藏在附近窥视他的小鬼已经全不见了,有些是被他吓跑了,有些则是被他吃掉了。

他身上的光芒越盛,石岸上的丁思汉越痛苦。抬手轻轻抚摸着自己的面孔,他已经可以摸到一丝丝坚硬的毛细血管——血管已经枝枝杈杈的在他脸上显出了形状。

他的外套后面连着帽子,抬手掀起帽子扣在了头上,他不想让保镖们看到自己的异象。保镖们自从见识了大蚂蝗的胃口之后,先前那种天不怕地不怕的锐气全散尽了,变成了一小队肌肉发达的小绵羊,双腿打着晃横行。

小心翼翼的转过最后一道弯,丁思汉闭着眼睛停顿了一下,随即将手中的血符贴到了身后石壁上。前方的史家姐弟是彻底失踪了,他现在只能自己摸索着走。贴过血符之后,他在裤子一侧用力的蹭了蹭手指,生怕自己染了血腥气,会再招惹来大蚂蝗,虽然血符上的鲜血早已经干透了。

渐渐的,他感觉自己距离鬼巫师越来越近了。

他和鬼打了几辈子交道,完全的不怕鬼。鬼的气息他很熟悉,然而鬼巫师和一般的鬼不一样。一般的鬼都是阴气重,而鬼巫师则是邪气重。在偶尔的疏忽之时,他甚至会搞不清鬼巫师到底是生是死。说不清,与其揣测他是人是鬼,莫不如说他更像妖魔。

丁思汉每每想到这里,都很庆幸,因为鬼巫师的确是鬼。幸亏他是鬼,否则自己就全无还手之力了。试探着将一只脚迈下石岸,河水只没过了他的鞋面。还是通达大路走着舒服,他趟起了水,一路哗啦哗啦的往前走。走到水与岸的交界处,他停下脚步,向后方的保镖伸出了手。

从保镖为他撑开的背包里,他拿出了两只小黄旗子。双手执旗单膝跪地,他把旗子立在潮湿的石坡上,口中低声念道:``天清地灵,兵随将令,兵随印转,将随令行,速速领令启程奉行,神兵火急如律令!''

话音落下,他松开双手一拍地面,两只小黄旗子竟是自行立住,丝毫不动。一股子凉风瞬间从后方吹过来了,无形刀剑一般穿过了两只黄旗之间。周围的邪气太重了,吓得他的小鬼不敢靠前,于是他充当开路将军施了一道阴兵咒,在弥漫着的邪气之中开了一道小门,让小鬼们能够通过小门继续前行。

凉风穿过双旗之后,立刻就弱化成了似有似无。这一段洞窟已经被白琉璃的念力镇得密不透风,小鬼们即便有了通道,也无法长驱直入。

丁思汉另有一番主意。拔了小旗向前走了几步,他故技重施,重新立旗念咒,引着小鬼们又向前行进了一段路途。感觉自己距离白琉璃实在是很近了,他收起小旗,从袋子里又掏出一只小盒子。盒子打开来,里面是满满一盒腥红油脂,乍一看仿佛印泥,其实是经过了炮制的尸油。挑了一指头抹在地上,他慢条斯理的描出弧线,最后正是画成了一个极大的圆圈。尸油是纯阴之物,这一个圆圈也就是他为小鬼们暂时划出的安身之处。有了尸油的安慰,也许小鬼们不会立刻急着逃跑。

把小鬼们暂且圈禁住了,丁思汉面对着白琉璃所在的方向盘腿坐下,身边正挨着他的乌合之鬼们。保镖们则是远远的立在了一旁——在岸边一块大石头下,他们刚刚看到了两名同伴的尸体。

把大敞四开的背包摆在一旁,他抬头对着前方冷笑了一下,随即把手伸入背包之中,摸出了一沓符。一招鲜,吃遍天,单凭着一手好符,他便可以在阴阳两界畅行无阻。手里的符干燥而又柔韧,是半透明的人皮,用烙铁在活人背上烫出咒文,烫到人死,符便成了。人皮主人的魂魄全被封在人皮符里,封得越久,怨气越重,一旦释放,必成凶灵。

丁思汉此刻并不需要凶灵作祟,所以一手托着人皮符,另一只手从背包里抓了一把朱砂。将朱砂抹在人皮符上,符中的鬼是阴的,朱砂却是鬼的克星。将一张人皮符细细的抹匀了,他拿起第二张接着涂抹。一张一张的涂抹过了,他紧闭双眼定了定神——头脸的皮肤像是要被硬化的血管勒碎了,他的时间已经很有限。

最后在自己面前点起半截蜡烛头,他拈起一张人皮符在火苗上一燎,随即猛的挥向了前方。人皮符沾火即燃,在脱手而出的瞬间已经烧成了一团火流星。滴溜溜的直飞到了洞窟高处,人皮符在白琉璃面前彻底化灰,符中的魂魄受了朱砂与烈火的冲击,在自由的同时魂飞魄散。而在魂魄分崩离析的一刹那间,爆发出的凶杀之气直冲向了白琉璃。

白琉璃本是不怕鬼的,可万没想到丁思汉会把鬼当成高射炮弹轰击自己。他稳住心神正想还击,然而第二张人皮符又到了。

他被第二张人皮符狠狠的``震''了一下。慌忙向后退却了,他无论生死,一直是个幕后的人物,从来没有明刀明枪的上过真战场。他能大门不出二门不迈的咒死活人,却抵挡不住一个小孩子的拳脚。先前他打了丁思汉的软肋,如今丁思汉也打了他的软肋。人皮符接二连三的对他紧追不放,他仿佛陷在了开花炮阵里,但是他没有慌。一甩袖子退入洞中深处,他想找个僻静地方重起炉灶另开张。

火流星随着他换了方向。他集中了念力预备对抗,可在火流星穿越身体的一瞬间,他的影子忽然闪烁了一下。

不是人皮符了,他想,丁思汉换了招数!

丁思汉的确是换了招数。白琉璃毕竟是个鬼,而他没有必要用鬼打鬼。将他的先遣队尽数祭出之后,他进入正题,一挺身起了立。对着白琉璃的方向迈上一步,他一边结着手印,一边口中诵道:``临兵斗者皆列阵前行!''

话音落下,他向前一甩手,发出的却是一张最普通不过的纸符。纸符是常见的镇邪祟符,但因画符人是他丁思汉,所以纸符拥有了名副其实的力量,当真是把白琉璃镇了一下——一下而已,并没镇住。

丁思汉是要穷追猛打,白琉璃自然也不会坐以待毙。接连又挨了几张镇邪祟符,他有心立刻退却,可很快发现一味的退也不是长久之计,丁思汉明显是有备而战,怀里仿佛藏着无穷无尽的纸符。想到对方先是伤害无心,如今又要伤害自己,白琉璃忽然怒不可遏的高高举起了双手,大吼一声狠狠拍下。洞窟之中的空气骤然激荡了,一根尖锐的钟乳石锥断裂脱落,直刺向了丁思汉的头顶心。丁思汉侧身一躲,让石锥紧贴自己碎在了地面。与此同时,他从怀中掏出了最后一张符。口中低声念诵了咒语,他目中精光大盛——白琉璃已经乱了方寸,正好让他发动最后一击!

纸符平平的穿过了白琉璃的身体,白琉璃的影子随之一闪,紧接着凭空消失。纸符缓缓的落下,在它即将着地之时,洞中起了``啪''的一声爆响,纸符碎成无数细屑,白琉璃则是缓缓升回空中,鬼影已经变得忽明忽暗。

丁思汉万没想到他如此难缠,凭着自己连珠炮似的打法,他居然既不就范,也没有魂飞魄散。正想对他再补一招,暗中却是无声的冲出了一个白影,炮弹似的合身冲撞向他。未等保镖赶来救援,他已经被撞了个仰面朝天。躺在地上一歪脑袋,他又惊又喜的睁大了眼睛:``无心!''

不等无心回应,他一翻身扑向前方,两条腿还未站直,双臂却是先他一步的抱住了无心的腰。无心低头一看,正是面对了他恐怖的面孔。短暂的怔了一下,无心抬手一把掐住了他的脖子,又垂下头,一口咬上了他的额头。

丁思汉惨叫一声,在窒息的痛苦中断断续续的说道:``又来了\ldots{}\ldots{}又来了\ldots{}\ldots{}又来杀我了\ldots{}\ldots{}''

无心合拢牙关一甩头,从他额头上撕下了一片黑血淋漓的皮肉。远方暗中灯光闪烁,是保镖们摇晃着手电筒赶来救主了。无心见到了``人'',不禁一阵心悸。松开双手转身狂奔向了洞穴深处。

丁思汉捂着喉咙站起了身——无心逃了,鬼巫师,本来将要成为自己的囊中之物了,如今趁乱也逃了。额头显出了黑糊糊的血洞,虽然惨不忍睹,但是反倒比他先前的模样更正常,因为是个受了伤的人模样,不再像妖魔鬼怪。

宛如中了毒一般,他的神经有些麻木,觉不出剧痛。转身向后走了几步,他放出了尸油圈中的小鬼们。小鬼们是没有价值的,但是可以给他通风报信,还可以冲淡鬼巫师留在此处的邪气。

用纸符烧成灰糊住伤口,他不顾保镖们的关怀,只自顾自的悠然想道:``我刚才又抱了他。''

他想了又想,想得十分细致,并且还自作主张的横生出了许多枝节。及至他想过瘾了,一弯腰拎起他的背包,他握着手电筒,一步一顿的走向了洞穴深处。

\chapter{洞的主人}

无心一口气逃出老远,末了壁虎一样爬过洞窟的穹顶,他在一块凸出的大石上落了脚。白琉璃飘飘忽忽的悬在他的前方,鬼影时明时暗的很不稳定。无心伸手去触碰他,手指掠过他的鬼影,空荡荡的毫无感觉,一直阴冷浓烈的邪气也淡得几近于无了。

无心忽然恐慌了:``你会散吗?''

白琉璃把头垂到了胸前,低声答道:``不会。''

无心向他招手,让他靠近自己:``你虚弱的像只新死鬼。''

白琉璃一点头:``嗯,我伤了元气。''

无心像捧一轮月亮似的,把他拢到了自己胸前:``你不能抱我了,也不能吃火锅了。''

白琉璃蜷缩在一团微弱的白光之中,有了点落花流水的意思,声音也软了:``只是看你很可怜,才想抱抱你。其实抱不抱的,没有关系。''

他的鬼影闪烁了一下:``不过重庆火锅,是真的很想吃。上一次吃时我是十二岁,还没有进西康。''

无心忽然难过了:``对不起。''

白琉璃歪着脑袋,从湿漉漉的长发中去看他,看着看着,笑了一下:``等到出了洞,你谈恋爱给我看吧!''

无心的脑筋没有跟上他的要求,不过不假思索的点了头:``好,我谈恋爱给你看。''

看谈恋爱的快乐弥补了吃不到火锅的遗憾,于是白琉璃开始寻找藏身之处。无心想要像藏匿史家姐弟一样,找个洞把白琉璃封住。然而石壁上洞眼虽然不少,可他不会画符,无从封起。他的感觉最为灵敏,耳朵在暗中动了一动,他隐隐听到了远方一步一停的脚步声音。灵机一动有了主意,他让白琉璃飘进了一道又窄又深的石缝中。咬破舌头反复的舔了石缝两边,他的鲜血成了拦路门神,至少可以挡住普通的小鬼。

白琉璃缩在深处,也不敢靠近他的血。眼看无心伸长舌头左舔舔右舔舔,像只水鬼或者野猴,白琉璃一蹙眉毛,又有些讨厌他了。

无心先是安顿了史家姐弟,如今又安顿了白琉璃。收回舌头落了地,他想了想自己的所作所为,心中无端的生出了一阵喜悦。从前的往事虽然还是影影绰绰不明朗,但是他把眼下的局面大致弄清楚了。向前又跑了一段路,他爬上石壁做了埋伏。在等待敌人逼近的时间里,他的眼前不断浮现出种种片段,有花有雪,有一望无际的林海,有温暖的怀抱,有堆积如山的奶粉罐子。抬手搭上一小块凸起的石头,他收拢手指抓了抓,然后自己一抿嘴——还有个又甜又香的女人。

正当此时,脚步声音越来越近了。

声音很轻,显然抬脚落步都很有控制,然而一直不见有光出现,也许是他们在摸黑前进。无心静静的等待着,一直等到他们出现在了自己的视野中。对他来讲,丁思汉的保镖比丁思汉本人更富有杀伤力,因为他们身大力不亏,凶神恶煞奈何不了无心,他们却是能够三拳两脚的把无心打成俘虏。

所以无心一动不动,等他们走得更近一点。

在无心等待之时,丁思汉也在等待,等待无心和鬼巫师的出现。鬼巫师虽然没有被自己打散,但也弱到了半死不活的地步。也许正是因为他的弱,丁思汉极力的集中了精神,却是始终觉察不出鬼巫师特有的邪气。身边围绕着的小鬼又太多了,几乎对他造成了干扰。

他不着急,慢慢的往前走,头顶骤然掠过一道疾风,后方一名保镖发出了惨叫。几支手电筒瞬间一起开了,沿着地上一片细细的血点子瞧,他们在通道一边发现了濒死的同伴。同伴的颈侧动脉是血糊糊的一片,皮肉仿佛是被生生撕扯开的,鲜血滔滔的淌了一地。救人是绝对的来不及了,他们眼看着同伴在血泊之中抽搐成了一具尸体。

起初谁也没有说话,后来有人开了口:``难道除了水里,地上也有东西?''

丁思汉无言的摇了摇头——他不敢确定,也许是新敌人,也许是无心。有小鬼的声音响在他的耳边,嘁嘁喳喳指指点点。他猛的晃动手电筒照向了斜上方,一簇钟乳石中闪过了一条白影,他的手慢了一秒钟,甚至没能分辨出无心消失的方向。

关闭了手中的手电筒,丁思汉悠然神往的叹了口气。保镖的死亡并不能让他动心,让他念念不忘的还是无心。无心在的时候,他终日思索着如何折磨对方;无心不在了,逃了,他又像个贱种似的,开始饥渴的思念对方。他简直搞不清了自己是男是女是老是少,几辈子的光影记忆全重叠在了一起,男女老少四个字似乎也根本无法将他描述清楚了。他像个情窦初开的小姑娘一样对无心又痴又迷,也像个身经百战的老色胚一样,恨不能垂涎三尺的一口吞了对方。羞涩而兴奋的回忆着自己方才对无心的一抱,他走了神,任由黑血顺着自己的鼻尖往下滴答。

小鬼们献媚似的挤到他的耳边,一个说无心在这边,另一个说无心在那边,他被小鬼们指挥得团团乱转,可是连无心的毛都没能觑见。正是茫然之时,身后又起了一声哀嚎。三束手电筒光芒一起向后转了,走在中间的一名保镖随即猛一闭眼,因为迎面被喷了满头满脸的热血。

无心又咬死了一名殿后的保镖。和保镖们相比,他的武器只有牙齿。锋利的牙齿咬断颈侧动脉,一断即逃,他在地面上几乎是不停留。有人开枪了,一边射击一边崩溃似的怒吼出声。潮湿的碎石屑应声飞溅,死去的保镖软软的瘫在地上,无心则是再一次不知所踪。

丁思汉身边只剩下了两个活人,一个活人在叱天骂地的开枪乱打,另一个活人则是靠了边,自作主张的想要沿着原路往回走。丁思汉其实一直很依赖保镖们的功夫和力量,可惜保镖们的精神不如他们的肉体结实,而且水中的蚂蝗和地上的无心,已经解决掉了他们中的三分之二。

暗暗的叹了一口气,丁思汉蹲下了身,先是念念有词的在地上画出了两道符,随即双手一拍地面,口中轻声喝道:``起!''

躺在地面上的两具新尸首,一具近一点,一具远一点,全有了反应,摇摇晃晃的先后站起了身。他让小鬼附上了他们新鲜的尸体,虽然行尸走肉总比不得活人灵活,但是聊胜于无,可以勉强一用。

死人活了,活人则是吓傻了。丁思汉用手电筒向他们晃了一下,声音轻而沙哑,在洞窟之中引出了空旷的回声:``不要怕,跟我走。''

死人跟上了他,活人犹犹豫豫的,不知要不要跟。不知怎的,他们忽然感觉眼前的丁老先生不再是他们印象中的丁老先生了。印象中的丁老先生虽然也有点神鬼莫测的意思,可是老头挺和气,不冒险。跟着老先生混饭吃,是个缺德不缺钱的太平差事。

活人是有思想的,真到了紧要关头,他们不会给丁老先生陪葬。握着散弹枪不进反退,他们起了二心。洞子地面高低崎岖,上下还横贯着七长八短的钟乳石,撒腿狂奔是做不到的,但是只要够机灵,一路跳跃着还是能够逃出速度。互相对视了一眼,两个活人互通有无的交流了眼神。

然而在他们即将撤退之时,一股子寒气直刺心肺。头脑瞬间晕了一下,他们的魂魄已被外界的小鬼冲出了身体。

丁思汉带着四具活尸,一往无前的继续走。同时,他放心大胆、肆无忌惮的开了口:``无心,我知道你在附近!''

在四具活尸的包围下,他料想无心没有办法从天而降的咬死自己:``无心,上辈子的事情,你还记得吗?你和我,我和你,还记得吗?''

无心没有回应,而他翻尸倒骨抚今思昔,却是自己把自己感动了:``无心,你看我——''他的气息颤了一下:``你看我为了找你,人都不要了,命都不要了。''

话到此处,他咳嗽了一声,震得额头黑血飞溅:``无心,你早已忘了我的名字,我却是想了你将近一百年!''

尾音陡然上扬,他调出了尖利的嗓子:``无心!世上可有第二个人,能像我一样对你念念不忘一百年?爱之深恨之切,你懂吗?''

洞窟幽深,声音能够传出老远。躲在石缝里的白琉璃倾听良久,感觉自己是得知了天大的秘密。把秘密翻过来掉过去的细想了想,像被雷劈了似的,他骤然明白了,随即险些笑死在了石缝里。

无心攀在一根粗壮的钟乳石上,感觉十分尴尬。可是如果认认真真的下去批驳对方的歪理,结果一定不会美妙,兴许只会让他更尴尬。

丁思汉意犹未尽的站在原地,感觉自己还有千言万语要说——他想吞了无心,不是玩笑,不是赌气,他是真的想吞了无心。吞了无心,无心就永远都是他的了!

``我恨你!''他毫无预兆的转了口风:``负心薄幸,你害死我了!''

双手攥成拳头,他额头的伤口像是开了闸一般,汩汩的向外涌出黑血。如同他的缠杂不清的灵魂一样,他的感情也是同样的缠杂不清。他对无心时而爱得要死时而恨得要命,无论爱恨,全是真的。

恍惚中忘记了自己衰老的皮囊,他上一世死于十四岁,于是这一世生于十四岁。苍老的声音被他扯得又细又高,他颠动着一头花白的短发,甩出了一片漆黑的血珠:``出来!你给我出来!无心,事情没有完,只要不合我的意,就永远不会完。这辈子死了,我还有下辈子,下辈子死了,我还有下下辈子。无心,你何其荣幸,能让我为了这么一点蠢事与你纠缠三生!''

说到这里,他咬紧牙关屏住呼吸,对着自己点了点头:``蠢啊,真蠢。''

洞窟之中漆黑如水,只有丁思汉手中的小手电筒放射出了长远的光束。他笔直的站在四具活尸中间,一番独角戏似的长篇大论过后,他仿佛是疲倦了,歪着脑袋望着前方,他面无表情的只是喘气。

无心始终是不吭声。没有声,没有光,他便无法确定无心的位置,连一枪把对方轰下来都不能够。突然深深的委屈了,丁思汉几乎要落了眼泪——他心里真苦啊,全是无心欺负了他!恨不能在光天化日之下和无心对着开膛剖肚,互相的晾一晾肺腑。他自认为是一腔真感情,没掺杂,最纯粹。

他敢晾,无心敢吗?所以还是无心的错,全是无心的错!丁思汉提起一口气——对于无心,他的话还没说完!

``我知道你恨我折磨了你,可在上一世,你又是如何把我送到死路上去的?无心,你够狠啊——''

话没说完,洞子深处忽然传出了柔和的男声:``大爷,你还唠叨没完啦?''

丁思汉一愣,当即抬头觅声望去:``谁?''

手电筒的光芒一晃,下方的丁思汉和上方的无心一起睁大了眼睛。如果没走眼的话,他们统一的认为自己看到了一只穿着运动服的直立大蜥蜴。

无心依旧是按兵不动,丁思汉则是有些傻眼:``什么东西?''

来者站在手电筒的光束之中,不但头角峥嵘,露出的四肢也是鳞甲赫然。看个头它得有个一米七高,看模样,两个鼓泡眼一个长扁嘴,正是个典型的蜥蜴面孔。

丁思汉意识到自己是遇到了成精的妖物,幸而身边立着四具活尸,纵算妖精敢袭击自己,自己也有还手之力。下意识的后退一步站稳当了,他只听妖精坦然的答道:``我是一只蛇精。''

丁思汉没看出它像蛇精,不过也懒得刨根问底。暗暗做好了驱使活尸的准备,他一团和气的又问:``是不是我惊扰了大仙?''

自称是蛇精的大蜥蜴用爪子扯了扯身上的运动服,言谈举止慢悠悠的,是个很讲理的模样:``可不是?您跟哪位大娘吵架呢?我可听你闹半天了。''

丁思汉一咧嘴:``我\ldots{}\ldots{}我没和女人吵架?''

大蜥蜴一点头,嘴边耷拉着分叉的细舌头:``不是女人,那就是男人啰?你们这些中老年同志的感情出了问题,应该自找地方解决,不该拿到我家里来吵嘛!你这样影响我的生活,我简直不知道要不要吃掉你。''

丁思汉皱起了眉毛:``这是你家?''

大蜥蜴答道:``没错,我都住了二十多年了。''

丁思汉抬手向后一指:``那河里的大蚂蝗——''

大蜥蜴颇斯文的答道:``鄙人的宠物。''

丁思汉的脑筋立刻左三圈右三圈的开始转动了。他没和妖精打过交道,不知道前面这蜥蜴会有多么大的力量。看蜥蜴的意思,显然是要让自己滚蛋。自己滚了倒是容易,可洞里的无心怎么办?无心是一步也放松不得的,天知道他有多么会逃,也许一眼照顾不到,自己这一辈子乃至下一辈子,都见不到他了。

思及至此,丁思汉一言不发的抬起了手,对着大蜥蜴虚空画符。最后一笔收了尾,他紧接着向前狠狠的一挥手:``去!''

大蜥蜴属于妖邪之物,自然也有符咒可以制它。而随着丁思汉的号令,一具尸首拖着双腿,直着眼睛直奔了大蜥蜴而去。大蜥蜴先是受了一道无形的符,如今又猝不及防的被一具活死人推了个跟头。张开大嘴``哈''的一声,它对着上方的活尸喷出了一口黑烟,随即歪着脑袋一嘴叼住了对方的脖子。摇头晃脑的一扯,大蜥蜴咬断了活尸的脖子,一个圆滚滚的人头顺着地势骨碌碌滚出了老远。再一口咬中活尸的臂膀,大蜥蜴力大无穷,把对方的胳膊也撕掉了。

最后把活尸分了个七零八落,大蜥蜴站起了身再往前看,发现丁思汉和活尸们已经一起消失无踪。

大蜥蜴保持着人形,正想继续驱逐家中的不速之客。哪知未等它开始行动,沿着一根细长的钟乳石,一张雪白的面孔倒吊着缓缓伸到了他的面前。

无心为了表示恭敬,在说话之前先双手合什拜了拜,然后才小声开了口:``大仙,我是被刚才那个老头子逼到这里来的,早就想走了,一直没能走成。你老人家明辨是非,千万不要迁怒于我啊!''

大蜥蜴被他吓了一跳,张着嘴没言语。而无心伸手向着后方一指,小声说道:``他们刚往那边跑去了。''

大蜥蜴将无心上下打量了一番,末了闭了大嘴,一个鹞子翻身腾空而起,扁扁的附上了洞顶。又因它着实是个蜥蜴的身体,且又穿着半袖运动衫和五分裤,所以一条尾巴无处安置,只好从一侧裤管中伸了出来。一瞬间爬没了影子,它显然是追丁思汉去了。而无心抱着钟乳石想了想,随后猴子似的纵身一跃,一路也悠荡而去了。

平心而论,无心在洞中的行动,比大蜥蜴更为灵活利落。细长的四肢全调动了,他闭着眼睛往前冲,忽然感觉前方阴气极重,应该是个鬼魂聚集的所在。他提起了精神,磨牙霍霍的向下一跳。

然而鬼魂们簇拥着的并不是丁思汉。他刚一落地,便被一名沉重的活尸压住了。未等他出手反抗,双臂已经被活尸紧紧的箍住。另一双干枯的老手从天而降捧住了他的脸,他听到了丁思汉低哑的笑声:``嘿嘿嘿,你往哪里逃?''

与此同时,远方河边的孔洞之中,史高飞无视了史丹凤的强烈阻挠,自顾自的伸腿下了洞。他的大砍刀被水流冲到了岸边,此刻正静静的躺在浅水之中。他好了伤疤忘了疼,把大蚂蝗抛到了脑后。走过去弯腰捡起了刀,他回到孔洞前仰头说道:``姐,我去看看情况,你放心,我不和人打架。''

然后他拿着一只备用的小手电筒,蹑手蹑脚的向前走去了。

\chapter{归于尘土}

无心对大蜥蜴的印象很好,感觉它虽然是个妖精,但是一张嘴就带着股通情达理的劲儿,而且已经和丁思汉结了仇,如今三方动了手,它必定会站到自己这一边来。然而大蜥蜴爬了个无影无踪,留他一个人在地上翻翻滚滚,死活挣脱不开活尸的压制束缚。咬破舌尖狠啐了活尸的面孔,他想让对方尝尝自己鲜血的厉害,可连着啐了好几口,唾沫里连点血星都没有。

活尸已经是相当的有分量,偏偏活尸上面又压了一个丁思汉。丁思汉生怕他跑了,但又没有办法控制住他。一只手死死的揪住他的耳朵,丁思汉将另一只手拍在地上画起了符。石头洞子,不是土地,地下自然也不可能埋有尸骸,让他玩不成借尸还魂的把戏。手指肚在粗糙的石头地上磨出了血,余下两具活尸不知跑到哪里去了,丁思汉一边画符,一边感觉自己是有命无运——好端端的,岩洞里居然会藏着一只成了精的大蜥蜴。大蜥蜴不但打断了自己的真情告白,还嫌自己吵闹,要吃了自己。

他画得太用力了,把手掌磨成了鲜血淋漓。眼看无心挣扎得越来越激烈,他垂死挣扎似的狠狠一拍地面,口中大声喝道:``起!''

一声过后,他凭着直觉,感到远方似乎有了回应。自己派出的小鬼必定是找到了宿主,虽然不知道那宿主是死人还是死狗。不管死人死狗,反正都是自己的帮手。稍稍的轻松了一点,丁思汉忽然意识到自己正抓着无心的耳朵,不由得心中一荡。手指用力拧出了无心的惨叫,他越过活尸的大脑袋,侧脸在无心的额角上亲了一下。

亲过之后,他从头到脚的一起发起了烧。黑血滴滴答答的落在活尸的后脑勺上,仿佛是刚刚意识到似的,他想自己中了鬼巫师的诅咒,其实也活不了多久了。活不了多久也没关系,反正他还有下辈子下下辈子。盯紧了无心找个好人家投胎,也许十几年之后,自己和他可以真真正正、正正经经的相爱一次。

他越想越对,最后忍不住嘿嘿嘿的笑出了声。老头子的苍老声音把他吓了一跳,他当即闭了嘴,因为自己也在嫌弃着自己。手指依旧死死的攥着无心的耳朵,把软软的一片耳朵攥成了奇形怪状。两人之间的活尸实在是太碍事了,丁思汉一声号令驱逐了他,只让他起身摁住无心的脑袋。无心的牙齿太厉害了,丁思汉不想再被他咬下一块肉去。

这回结结实实的趴下了,丁思汉握住无心瘦削的肩膀。洞里太黑了,什么都看不见,只有触觉是真的。身体忽然哆嗦起来,丁思汉骤然感到了无比的痛苦:``无心,别动,你听我说——''

话只说到了这里,因为一束光线忽然打在了他的身上。随即上方响起了一声怒吼:``天杀的恋童癖老变态!竟敢非礼我的宝宝!''

丁思汉猛一扭头,只见史高飞双手高举着一把大砍刀,狂风似的席卷而来。立刻命令了活尸前去迎战,他虽然见无心已经反抗到了精疲力竭的地步,可是用尽力气压着无心的手脚,他还是不敢有丝毫的躲闪和懈怠。

史高飞跑得太快了,活尸刚刚起身前去抵挡,史高飞已经奔到了近前。一道长长的影子骤然斜刺里冲出,把活尸直撞出去了一米来远。而史高飞面前陡然没了障碍,高举的砍刀当即落下,只听``咔嚓''一声,锐利的刀锋沉沉的落上了丁思汉的脊背,竟然一举将他砍成了两截!刀身继续下切,把无心的胸膛也长长的割开了一道。无心本来已是耗尽了力气,此刻皮肉冷不防的剧痛了一下,他像受了一大惊似的,反倒重新生出了精气神。抬手推开丁思汉的上半截身体,他眼看一团光芒从对方身上缓缓的升起来了,连忙不假思索的纵身向前一扑——依他的意思,他是慌了,下意识的想要把那一团魂魄扑住,不让它再去转世作乱。然而他体内仅有的鲜血顺着胸膛刀口汩汩的流淌,像个血葫芦似的,他合身直冲进了那一团光芒之中。

洞中仿佛响起了一声深长的叹息,丁思汉的魂魄一刹那间光芒大盛,随即越来越明亮越来越饱满,最后笼罩了无心的全身。

在短暂的光明过后,阴气渐渐的由浓转淡,无心仰起头环顾四周,只见光团缓缓分散成了明明暗暗的小星星,星星很多,因为那是两个人的魂魄。

抬手捂住胸前的伤口,他知道对方终于是彻底的魂飞魄散了。丁思汉上一辈子到底是叫什么名字来着?真是记不得了。记不得就记不得吧,记得了又有什么用?纵算是记得了,迟早也还是要忘的。

抬头望向面前的史高飞,无心很甜蜜的委屈了一下,随即张开双臂走上前去,抱住了对方:``爸。''

史高飞方才行凶之时把手电筒扔了,并不知道自己砍出了什么成绩。此刻抬手紧紧的搂住了儿子,他耳听周遭万籁俱寂,便很笃定的认为自己是把敌人全砍死了。

``宝宝。''他用他的大巴掌一下一下抚摸无心的后脑勺:``不要怕,坏人已经被爸爸杀掉了。''

话音落下,一束光芒照亮了他的侧影,一个男中音随即响起:``其实主要是我杀的。''

史高飞意外的一扭头,正和大蜥蜴打了个照面。对着大蜥蜴一瞪眼睛,他情不自禁的怪叫了一声:``哇操!什么玩意儿?''

大蜥蜴握着史高飞丢下的手电筒,不急不躁的答道:``我是一只蛇精。''

史高飞把无心揽在怀里,又用巴掌遮了他的眼睛,生怕儿子会被对方的尊容吓坏:``蛇精?是不是——''他一清喉咙开始唱:``青城山下白素贞,洞中千年修此身,啊\ldots{}\ldots{}啊\ldots{}\ldots{}勤修苦练来得道,脱胎换骨变成人\ldots{}\ldots{}''

史高飞越唱越长,一本正经的唱到了最后:``嗨呀嗨嗨哟,嗨呀嗨嗨哟,渡一渡我素贞出凡尘。''

大蜥蜴起初似乎是想要打断他,但是大嘴张了一张,却是并没有真开口。及至史高飞唱完了,他才答道:``你唱得真好,不过我和白素贞不同,我是公的。''

史高飞紧紧的搂着无心,感觉儿子瘦得好像一根刺:``除了是公的之外,你还是一只四脚蛇吧?实不相瞒,我看你根本就不像蛇,倒像大蜥蜴。''

大蜥蜴眨了眨眼睛,没说话。无心怕史高飞把大蜥蜴惹恼了,连忙挣开了他的怀抱,转向大蜥蜴说道:``大仙,我们是被那些人追杀进洞的,现在那些人死绝了,我们也就要告辞走了。无端惊扰了大仙的山中岁月,我们真是有罪啊。''

大蜥蜴收回了耷拉在嘴边的半截分叉舌头,态度十分和气:``好的,我送你们出洞。''

大蜥蜴一边走一边收拾沿途的残尸,在拐角处将其堆成了一小堆,说是到时候可以喂给他的宠物蚂蝗吃。史高飞背起了无心,因见大蜥蜴的一侧裤管中伸出了一条长尾巴,就很好奇的弯腰伸手抻了一把:``你怎么不在屁股后面开个洞,直接把尾巴伸出来?''

大蜥蜴礼貌的侧身一躲:``哦,我下山还要穿这条裤子呢,开了洞就不雅观了。''

史高飞十分惊讶:``下山?你们蜥蜴星人已经在地球建立基地了?真厉害啊,我和我儿子还无依无靠的漂着呢!''

大蜥蜴显然是没听懂他的话,不过彬彬有礼的作了解释:``现在社会发展很快,如果总是与世隔绝的话,我怕自己会被时代所抛弃,所以每个月末会下山到县里的网吧里包宿一次。在比较温暖的季节,我还会去工厂打打零工,赚点小钱添置生活用品。''

史高飞快把眼珠子瞪出眼眶了:``你这样的还能去打零工?你们已经和地球人结成联盟了?''

大蜥蜴微微一笑:``我下山的时候,会变成人的模样。事实上我和人类的关系一直不错,遇到迷路的小朋友我一定会把他们送回家,见到摔倒的老人家也一定会去搀扶。不过最近行善的风险渐渐增大了,上个月我被一位老人家讹了两百块钱,还被她儿子揍了一顿。''

史高飞立刻义愤填膺:``妈的,要是让我遇上了这种老×,一定当场打死!''

大蜥蜴笑了笑,显然是不甚赞同史高飞的暴脾气。

无心趴在史高飞的背上,就听这二人且行且谈,居然聊得有说有笑。暗暗的抹了一把冷汗,他想看来这大蜥蜴是个吃软不吃硬的脾气,幸好丁思汉当初先起了杀机,否则大蜥蜴若是被他笼络住了,自己这边非彻底完蛋不可。

这时史高飞又继续问道:``蜥蜴星人,你为什么要住在山洞里?你在山下不是有基地吗?''

大蜥蜴悠悠的作了回答,原来他十几年前和附近镇上的一位姑娘产生了爱情,姑娘密谋着要和他私奔,都筹划得清清楚楚了,姑娘却是死于了一场疾病。大蜥蜴痛不欲生,夜里偷出了姑娘的尸首,一路逃进了山中。而这座岩洞虽然漆黑恐怖,但是暗不透光,且通长河,正是一处极阴的地点。阴气重到了一定的程度,可保尸身不腐,于是他带着姑娘的尸首进了洞,还把自养的大蚂蝗扔进暗河,充当了看门狗。守着姑娘的尸首,大蜥蜴已经独自生活了许多年。

这一段故事讲完,史高飞唏嘘不已,无心把下巴搭在史高飞的肩膀上,也跟着感慨了好几声。忽然抬头向上一看,他开口说道:``爸,你停一停,我上去接个朋友下来。''

一边说话一边溜下了史高飞的后背,他在攀爬石壁之前回头又问道:``爸,你有水吗?''

史高飞问道:``渴了?''

无心摇了头:``不,不是要喝。''

史高飞弯腰解了鞋带,把脚上一只圆头圆脑的大皮鞋递给了儿子:``我刚才在河里又趟了一次。''

无心托着一皮鞋的河水,向上一直爬到了藏匿白琉璃的石缝前。用手蘸水细细的擦去了石缝两边的干血,他小声唤道:``白琉璃,出来吧,天下太平了。''

白琉璃像一只半透明的糯米团子,从窄窄的石缝中挤了出来。对着无心一翘嘴角,他想起丁思汉那一番如泣如诉的告白,忍不住哈哈又笑了。

无心没理他,下到地面让史高飞穿了鞋,又自动的跳回了对方的后背上。史高飞自然而然的背过双手托住了他的腿,因为和大蜥蜴谈得正酣,所以也没想着问问他接的是什么朋友。而白琉璃跟在后方,看无心在史高飞的后背上趴得十分坦然稳当,便暗暗的纳罕,没想到世上还有人肯把他当宝贝。

一人一妖一鬼一无心走过长长的岩洞,末了在暗河岸边停了脚步。暗河边上躺着两具尸首,还是大蚂蝗当初的呕吐物。尸身全都伸展了胳膊腿儿,显然曾经一度被小鬼俯了身,可惜丁思汉一死,小鬼们也随之离了体。

史丹凤无须呼唤,自动的踩着石头跳出了洞。远远的看见无心和弟弟了,她心中一阵狂喜,放心大胆的一路小跑到了近前,结果没等她收住脚步,大蜥蜴从史高飞的身后转了出来。

史丹凤和大蜥蜴迎面相遇,直着眼睛望着前方,她嘴上一声没出,一脑袋头发却是险些齐根立起。大蜥蜴见了异性,非常客气的一点头——可能还笑了,不过大嘴咧到太阳穴,史丹凤完全欣赏不了他的笑容。

对着史丹凤打过招呼之后,他两脚踢开尸首,自己慢条斯理的唠叨道:``回来之后有我忙的了,这些尸首都得处理掉才行,也不知道大黑肯不肯吃。''然后他蹲在河边,从裤兜里掏出一支折叠小牙刷。将牙刷头伸到水里涮了涮,他开始张着大嘴刷牙。

史家姐弟和无心全傻了眼,站成一排看他刷牙。而大蜥蜴刷牙完毕起了身,挺不好意思的解释道:``我近一年一直在吃素,刚才咬了满嘴的血,感觉口气很不清新。''

史家姐弟一起乖乖点头,史高飞是心悦诚服,史丹凤则是感觉自己在梦游。唯有无心蹲在河边,悄悄的捧起河水也漱了漱口。

大蜥蜴直起了身,对着黑沉沉的河面发出了一串蛤蟆叫。水面由远及近的起伏了,正是大蚂蝗一路波浪式的游了过来。扁扁的浮在浅水之中,它成了一大片软腻腻的薄叶子。大蜥蜴率先趟水踏上了大蚂蝗的背,然后回头对着众人招手:``请上来吧,大黑游得又稳又快,会把你们一直送到洞口去。''

史高飞先上去了,一只手拉着无心。无心也跟着上去了,一只手拉着史丹凤。史丹凤依旧是一言不发,因为始终感觉自己是在做噩梦。

大蚂蝗的速度果然很了不得,众人在蚂蝗背上也没觉出乘风破浪,可是不过片刻的工夫,他们已经进了洞口的圆形水潭。史家姐弟已经忘记了自己是什么时候进洞的,在大蜥蜴的引领下向外转了几个弯,他们眯着眼睛,看到了洞口一片碧蓝的天空。

能喘气的都长长的呼出了一口气,不能喘气的,比如无心,眼睛和心也随之一亮。小心翼翼的踏过长长一大片蝙蝠粪,一缕明媚阳光照在了领路的大蜥蜴身上。史高飞眼前一花,发现大蜥蜴居然瞬间变成了人模样。背靠岩壁侧身站在洞口,他对着面前这一队不速之客笑道:``我送到这里,就不出洞了。''

史高飞怔怔的看着他,发现他变成人后还挺帅,是个二十多岁的青年模样,皮肤白皙,头发泛黄,清澈透明的灰眼珠中有着狭长的黑瞳孔。史丹凤也看傻了,情不自禁的``哟''了一声。

大蜥蜴对着他们微笑点头,又特地对史高飞说了话:``其实我并不是蛇精,我的确是一只蜥蜴。不过蛇精听着总像是更浪漫一点,所以你就当我是蛇好了。''

史高飞抬手和他行了个拥抱礼,然后神情坚定的说道:``虽然我们现在暂时留在了地球上,但是我相信总有一天你还会回到蜥蜴星,我的母舰也会接我和宝宝回家。即便母星抛弃了我们,我想我们也还是能在地球上做出一番大事业的!莱因哈特说过,我们的征途,是星辰大海!''

说完了话,他又用力的拍了拍大蜥蜴的后背,拍得大蜥蜴浑身乱颤。小心翼翼的斜着眼睛瞥向史高飞,大蜥蜴很识相的没有多问。

扯着丁思汉一伙当初留下的尼龙绳子,史高飞等人络绎的降到了地面。大蜥蜴跪在洞口,在春风之中向他们用力的挥了挥手。史高飞也仰头扬起了手:``再见,蜥蜴星人!''

大蜥蜴大声的作了回答:``再见,外星人!你唱歌真好听!''

史高飞弯腰捡起一块石头,在地面上画出了浅浅的阿拉伯数字,画过之后扔掉石头拍拍手,他直起腰对着上方喊道:``蜥蜴星人,记住我的电话号码,上午九点到下午四点开机。有空的话找我玩!''

然后对着大蜥蜴又招了招手,他把赤条条的无心扯到背上背好,带着史丹凤走向了密林之中。

\chapter{回家去}

史高飞背着无心走山路,一步一步走得精神焕发;史丹凤紧跟在一旁,将一只手偷偷搭上了无心的光脊背。无心的两只大眼睛忙得快要不敷分配——双手搂着史高飞的脖子,他不是往左去瞟白琉璃,就是往右去瞟史丹凤。常年和鬼妖厮混在一起,导致他对``人''是特别的敏感,又因为他自己是个男人,所以越发的对女人有兴趣。兴许是心情轻松的缘故,他的脑子活络了许多,并不久远的往事一桩一桩的浮现在了眼前,忽然好像醍醐灌顶似的,他大喊了一声:``姐!''

史丹凤扭头看他,先是看,后是笑——无心总算是知道叫她一声了。

白琉璃伤了元气,在阳光下虚弱得快要飘不动。灵机一动的升到了树梢,他附上了一只正在打哈欠的大灰雀。他没做过鸟,想当然的张开翅膀向下一栽,他在半空中连着翻了好几个筋斗,最后准确无误的掉到了无心的颈窝里。无心和他太熟了,一看鸟眼睛中的光彩,便知道鸟身体里住了白琉璃。腾出一只手拂去脸上的凌乱鸟毛,他张开手指,不松不紧的抓住了大灰雀。又把大灰雀送向史丹凤,用短短的鸟嘴去触她的面颊:``啄!''

史丹凤扭头一躲,暗暗乐成了一只心里美的萝卜,口吻却还是一贯的:``哎呀,别烦人。''

与此同时,在不远处的一棵老樟树上,一只猫头鹰含着眼泪,直直的望着无心等人越行越远——白琉璃又被无心拐走了!

其实他早就知道白琉璃进了岩洞,可是岩洞里面有鬼气有妖气还有蝙蝠的粪臭,他在洞口徘徊了许久,茶不思饭不想的瘦了一圈,可硬是没敢进去闯一趟。悲伤的低头望着自己紧抓树枝的两只大爪子,他并不承认自己怯懦,反倒是更恨无心了。

史高飞在一块大石头前停了脚步,因为忽然意识到了儿子还光着屁股,山里的风也还带着凉意。他身上的衣服不是脏就是湿,唯有紧贴身的一件保暖内衣还算干爽。史丹凤到底是比他想得周全,眼看无心还是一张半红半白的阴阳脸,她解开身上的羊绒大衣,把自己的一件小毛衣脱了下来。

把小毛衣给无心套了上,她又让史高飞脱了身上的连帽棉外套。穿上外套戴上帽子,无心的面孔正好能被遮掩大部分。又把自己套在秋裤外面的羊绒保暖裤也脱了,史丹凤见无心正坐在大石头上,便要让他穿了保暖裤遮羞。然而未等把裤子递到他的面前,她忽然``呀''了一声,一直把头低到了无心的腿间。下意识的拈起了他的命根子,她大惊失色的叫道:``怎么少了半截?''

话一出口,她紧接着面红耳赤的松了手,感觉自己是办错了事说错了话。她不怕无心挑自己的理,可是担心弟弟因此胡言乱语。哪知直起腰一望史高飞,史高飞却是板着脸向她使了个眼色,是个不让说的意思。

无心接了保暖裤,抬脚往裤管里蹬,一边穿一边仰脸告诉史丹凤:``还能长的。''

史丹凤没敢再搭茬。打开史高飞的背包,背包里进了水,开了封的面包和饼干全泡烂了,倒是密封的几样零食还安然无恙。撕开一袋泡椒豆干给了无心,史丹凤正要换个话题开口,然而史高飞却是招手把她唤到了一旁。

在距离无心三米远的一棵树后,史高飞把声音压到极低,对史丹凤耳语道:``姐,你不要在宝宝面前再提鸭子他爸。''

史丹凤抬头看他:``我还想问呢,姓丁的老不死到底是——''

不等她说完,史高飞做了抢答:``鸭子他爸是个变态,已经被我剁了。不过在剁之前,我正好看到他在压着宝宝耍流氓。宝宝还小,被老变态非礼后会产生心理阴影的,所以你不要总提,让宝宝把老流氓忘记才好。知道了吗?''

史丹凤瞠了眼睛:``你杀人了?''

史高飞理直气壮的一点头:``对呀,杀啦!姐你是没看见,我杀得可帅了,`嚓'的一刀,把老变态剁成了两截!''

此言一出,史丹凤彻底老实了,不但自己不肯再提洞中之事,并且嘱咐史高飞也要把嘴闭紧,万万不可再对别人吹嘘他那``嚓''的一刀。

两人达成了共识,转身回到了大石头前。无心手里的泡椒豆干已经只剩了个空包装袋,一手抓着大灰雀,一手托着包装袋,他把舌头伸进了袋子里面大舔特舔,舔干净了抬起头,他辣得说不出话,只顾着咝咝吸气。白琉璃看了他的吃相,当即神情漠然的眯起了眼睛。

及至无心重新趴上了史高飞的后背,白琉璃不以为然的彻底闭了眼睛,心想:``还要背?''

然后他的身体一腾空,果然是随着无心一起升了高度。很笨拙的缩起了一对灰翅膀,白琉璃无论如何想不通。凭着他对无心的了解,他认为无心贱头贱脑,定会立刻被他们惯坏;紧接着他又思索道:``难道无心很讨人喜欢吗?''

低头啄了啄无心的手指,白琉璃认定他们是受了无心的骗。

史高飞因为心中喜悦,所以力大无穷,越走越有劲。史丹凤在山洞里毫无发言权,见了天日之后却是机灵了。她引着史高飞绕开了丁家别墅,直直的走上了下山的路。走着走着,她忽然偷偷的叹了口气,因为想起了小猫。这一趟冒险之旅虽然是圆满结束了,但是细想一想,弟弟杀了个人,自己弄丢了个小孩,堪称是双料的犯罪。张嘴吸了一口气,她下意识的想和无心讲一讲小猫的故事,但是话到嘴边,她瞟了弟弟一眼,因为怕自己说错了话,会触动弟弟的逆鳞,故而还是没敢真出声。

几个小时的长路走下来,他们出了山林地界。穿过山脚下的小村庄,在一条坑坑洼洼的土路上,他们挤上了一辆长途汽车。这一条线路乃是由私人承包了的,汽车直通县城。此刻车上乘客稀少,无心坐在了史高飞的大腿上,蒙着帽子趴上了前方的座椅靠背。两只赤脚垂在椅子下面,旁人乍一看,倒也看不出他的异常。如此颠簸了三个多小时,史高飞背着无心领着姐姐,在县城内的长途汽车站里下车了。

县城里宾馆不少,史丹凤挑了一家好的进去,用自己的身份证开了两间房。拿着房卡走出宾馆大门,她对等在外面的史高飞说道:``快进去吧,人家不查身份证。''

史高飞背着无心就往大门里走,史丹凤紧随其后,生怕前台的服务员会对无心起疑。然而两名小服务员统一的在低头玩手机,对他们是一眼不瞧。做贼一般的上了二楼,史丹凤先把弟弟和无心护送进了客房。客房收拾得窗明几净,阳光洒了满床。史高飞先把无心放在床上,然后直起腰反手捶了捶后背。史丹凤把房门关好了,转身走进房内。看了看无心又看了看弟弟,她从鼻子里重重的呼出了两道粗气,感觉往昔的生活又回来了,天下真的太平了。

大乱的时候,她没主意,只能跟着弟弟跑;如今太平了,她的用武之地就重新浮出了水面。脱了身上的羊绒大衣,她先走进卫生间里检查了一遍,一边检查一边发号施令:``小飞,你别歇着,先带无心洗个澡。你俩慢点儿洗,我出去给你们买身干净衣服回来。脱了的脏衣服别往床上放,晚上想吃点什么?''

在确定了花洒之中的确能够放出热水之后,史丹凤没等弟弟回答,一阵风似的开门刮出去了。不出一个小时,她大包小裹的又刮了回来。进门之后向内一瞧,她发现弟弟和无心全都光溜溜的只裹了一条浴巾。无心瘦得像猴子似的,弟弟耸着两片肩胛骨,则是十分的像刀螂。下意识的照了照墙上的玻璃镜,史丹凤发现自己的形象也很成问题,正是介于猴子与刀螂之间,脸都成了锥子。

见她回来了,猴子和刀螂一拥而上,猴子看食物,刀螂看衣服。看过之后,猴子像发了疟疾一般,伸手捞起炒面就往嘴里塞,一边吃一边浑身发抖。刀螂则是出言不逊:``姐,你又买便宜货,这玩意儿穿出去,不够丢人的。''

不等史丹凤回答,他又转向了无心:``宝宝,你怎么了?''

无心左右开弓的往嘴里塞炒面,同时含糊的答道:``爸,我饿死了。''

话音落下,他缠在腰间的浴巾自动松脱开了,顺着两条细腿滑落到了地面上。白琉璃蹲在枕头边的阴影中,看他吃疯了似的浑身乱颤,不由得要替他难为情。而无心被食欲刺激得头脑一片空白,凭着本能胡吃海塞。等他鼓着大肚皮恢复神智了,放眼往桌面上一瞧,他发现自己居然一人吃了三人的份,只给爸爸和姐姐留下了一摞油腻腻的空饭盒。

颇为惶恐的抬起头,他的半边白脸隐隐泛了红:``我\ldots{}\ldots{}''

史丹凤不怕他吃得多,只怕他不肯吃。快手快脚的收拾了空饭盒,她抢着笑道:``我再去买,楼下就有小饭店,买什么都容易。''

拎着空饭盒出了门,史丹凤急急的重新买了晚餐上楼。推门进房之后,她发现无心竟然已经睡了。长条条的仰卧在床上,他身上搭了一条薄薄的毯子。而史高飞坐在床边,本是在低头看他,此刻抬头将一根手指竖到唇边,他对着史丹凤``嘘''了一声:``姐,宝宝刚睡。''

史丹凤果然加了小心。放下饭盒走到床边,她做贼似的小声说道:``你也吃吧,好几天没正经吃饭了。''

史高飞重新垂下了头,盯着无心的睡颜说道:``姐,我不饿,我先陪他一会儿。姐,你看他睡得多可爱,像死了似的,乖极了。''

史丹凤听了他这绝妙的譬喻,不禁一皱眉毛。蹑手蹑脚的走上前去,她也微微俯下了身,不敢出声,只用气流送出语言:``刚吃饱就睡呀?''

史高飞不错眼珠的盯着无心:``他说他累死了。''

伸手给无心拉了拉毯子,他又恶狠狠的从牙关中挤出了话:``不知道是哪个该杀的王八蛋——兴许就是鸭子他爸那个老变态——在宝宝胸前划了那么深的一刀,心疼死我了。''

史丹凤抬手把鬓边的碎头发掖到耳后,发现弟弟干什么都不行,唯独做父亲做得很合格。床上的无心翻了个身,右手搭在枕边,已经分化出了拇指食指的形状。史高飞轻轻的捏了捏他粉红色的新生手掌。他捏过了,史丹凤也伸手捏了一下,捏得很小心,然而史高飞还是挑了理:``姐你轻点儿,别把宝宝的骨头捏坏了。''

史丹凤心中登时燃起了嫉妒之火:``我捏一下能怎么的?他全是你的呀?''

史高飞理直气壮的睁大了眼睛:``他当然全是我的!''

这一声嚷得高了,惊动了床上的无心。无心揉着眼睛欠了身,东张西望一圈之后却是问道:``爸,我的鸟呢?''

史高飞立刻起身走到窗前,把缩着脖子假寐的大灰雀抓到了无心枕边。无心看到了白琉璃,看到了史高飞,也看到了史丹凤。该在场的全在场,他安了心,倒下去又睡了。

无心算是从地狱一步迈进了天堂。足足的睡了一觉之后,他在天黑之后醒来了。客房是标准间,一张单人床上挤了三个大人看电视。无心上身偎在史丹凤怀里,两条腿横在史高飞怀里。史丹凤很细致的剥着一只猕猴桃,剥好了送到无心的手里。无心接了猕猴桃,仰起脸对着史丹凤笑。正在他讨好卖乖之时,房门忽然被人敲响了。

史丹凤不敢支使弟弟,自己伸腿下床走去开了门。门前走廊上立着个小小的人儿,史丹凤睁圆了眼睛:``小猫?''

小猫穿着一身松松垮垮的大衣服,一手扶上门框,他抬头对着史丹凤,把一双黑眼睛眨巴得流光溢彩:``大姐姐,我回来了。''

史丹凤心中一亮,感觉自己算是洗刷了人贩子的罪名。伸手把小猫拽进了房内,她弯腰急切的问道:``这些天你去哪里了?''

不等小猫回答,坐在床上的无心忽然向外伸了脑袋。对着小猫打量了一番,他开口怒道:``卑鄙的东西,你怎么来了?''

小猫撅了嘴,根本不想正视他:``我找琉璃哥哥。''

无心本来号称自己饿死了累死了,瘫在床上一动都不能动,可此刻他怒从心中起、恶向胆边生,一时间竟是忘了自己装娇弱博同情的大计。一抬腿跳下了床,他一把揪住小猫的衣服领子,劈面先左右开弓抽了对方十个大嘴巴。及至把小猫打得哇哇大叫了,他双手将对方横举过头,拼了命的往墙上乱摔。小猫的大衣服忽然瘪了,房内无端的飘出许多细软的羽毛。一只大猫头鹰钻出了领口,拍着大翅膀想要往天花板飞。然而无心一个箭步直窜而起,竟是单手抓住了他的尾巴。俯身把他摁到墙角落里,无心开始左一把右一把的薅羽毛。末了看猫头鹰和白条鸡也差不许多了,无心走到窗前推开窗子,抡圆胳膊把他掷了出去:``滚!''

史高飞上前关了窗子,然后问道:``宝宝,他是谁?''

无心气哼哼的答道:``是只猫头鹰精。''

史丹凤很淡定的答道:``哦,是猫头鹰精啊。''

史丹凤给无心又剥了两个猕猴桃,切了一个橙子,洗了几只大梨。把客房里的鸟毛大致收拾干净了,她到隔壁的客房里睡觉。洗漱过后躺到床上,她终于有了时间和闲心。从弟弟刨出无心开始,她慢条斯理的一直回忆到了此时此刻。她的思想素来是有条有理的,没理也能让她捋个道理出来。

她心平气和的捋了一夜,第二天凌晨起了床,她建设了整三十年的世界观彻底崩溃成渣。但是凭着她常年和弟弟斗智斗勇的丰富经验,崩溃归崩溃,并不耽误她过日子。

她穿戴利落了,下楼买了六人份的早餐,以及一塑料袋很新鲜的桃子。敲门进了史高飞的房间,她发现史高飞和无心居然早醒了,此刻正偎在被窝里看电视。大灰雀卧在床头柜上的玻璃烟灰缸里,跟着他们一起看。二人一鸟的脑袋统一转向了史丹凤,史丹凤拎着沉甸甸的七个塑料袋,忽然感觉自己像是来喂猪的。

在史高飞和无心

埋头大嚼之时,史丹凤打开了手机,看到未接来电无数,其中以白大千的号码为主。拨通号码打回去了,千里之外的白大千瞬间有了回应:``丹凤?你在哪儿呢?''

史丹凤迅速的斟酌了言辞,言简意赅的对他讲了实情。白大千一听他们居然真把无心找到了,当即骨酥肉软的瘫在了地上:``太好了,太好了\ldots{}\ldots{}''他悠悠的呻吟:``快回来吧,我想死你们了。''

原来白大千自从失去了无心之后,仿佛被人拔去了主心骨一样,有心独挑大梁,可是本事实在不济,生怕会有不开眼的客户请他捉鬼,会让他又砸招牌又送命。思来想去的,他暂时关了公司大门,对外则是放出风声,说白大师去终南山做短期的修行去了,一旦修行完毕,法力必定更上一层楼。藏头露尾的带着佳琪,他窝在出租屋里,不到天黑都不敢出门。终日心事重重的翻着日历,他只觉自己命比黄连苦,刚刚发了几天便宜财,家里的摇钱树便凭空失踪了。

史丹凤向他报了平安,允诺半个月内必定回家。挂断电话之后,未等她开口转述通话内容,手机忽然又响了。低头一看手机屏幕,她吓得一哆嗦——对方乃是史一彪。

硬着头皮接了电话,她很艰难的从嗓子眼里挤出声音:``爸\ldots{}\ldots{}''

史一彪对于女儿,一贯是肆意的咆哮:``臭丫头片子!你连着好几天不开手机,疯到哪里去了?''

史丹凤的小手机被史一彪震得嗡嗡直响:``爸,我在云南呢,我找到小飞了\ldots{}\ldots{}''

她本以为能够将功补过,然而史一彪依然有理由骂人:``云南?你可真行,让你在江口照顾小飞,你可好,把人给我照顾到云南去了,你跟你妈一个德行,你\ldots{}\ldots{}''

史丹凤憋气窝火,忍无可忍的把手机递给了史高飞。史高飞放下筷子,拿着手机开了口:``喂?爸?''

史一彪骤然听到儿子的声音,瞬间改头换面,细了喉咙做慈父状:``小飞啊,在云南玩得高兴吗?''

史高飞看了看身边连吃带喝的无心,不由得笑了:``很高兴。''

他有好一阵子没理睬过史一彪了,史一彪是极端的重男轻女,从小把他当成龙崽子养,恨不能搭块板子把他高高的供起来,虽然后来他拗不过遗传病的力量,渐渐长成了一头人高马大的疯驴,但史一彪总像是怕他怕出了惯性,哪怕对他有着天大的不满,当面也不敢轻易的指责半句。此刻听他回答``很高兴'',史一彪顺势又问:``看到孔雀和大象了吗?''

史高飞答道:``没看到。''

史一彪宛如春风一般,温柔的告诉他:``喜欢玩的话就多玩几天,不用急着回家,钱还够用吗?''

史高飞抬头问史丹凤:``姐,钱够用吗?''

史丹凤,像啐一根牙缝中的肉丝一样,力道很足的啐出两个字:``不够!''

于是史高飞立刻答道:``爸,不够。''

父子间的通话结束之后,史一彪立刻往女儿的账户里汇了款。而史丹凤本来是见钱眼开,然而如今查过账户余额之后,却是并未乐而忘忧。

她很想找个机会和无心说两句私房话。原来无心主动追求她时,她没把无心往眼里放;可现在形势发生了逆转,她对无心付出得越多,无心在她心中的分量越重。然而史高飞很彻底的霸占了无心,无心则是没心没肺,只知道吃。

无心大吃了一个礼拜,吃出了个新新鲜鲜的人模样。其间白琉璃卧在烟灰缸里冷眼旁观,感觉自己是又一次的大开眼界了。

这一天傍晚,史丹凤带着史高飞出门去订返程的火车票。无心趴在床上,捏着一点小米喂白琉璃。正是一片静谧之时,地面忽然腾起一团金光,他定睛一看,却是骨神来了。

骨神现形之后先是环顾四周,感觉蹲在烟灰缸里的大灰雀气味可疑,然而一时又看不出异常。于是转向无心笑出了一口大白牙:``终于找到你了。''

无心捧着小米上下打量他:``你这些天都跑到哪里去了?''

骨神支吾了一阵,因为没有出力去救无心,所以略感羞愧。而这不出力的原因,乃是他见白琉璃出了手——他的本领是不如白琉璃的,既然有了白琉璃这一棵葱,想必也就不需要他那一头蒜了。为了保持住自己刚刚恢复的元气和金身,他远远的遁到了县城里。而玛丽莲因为过于迷恋他,所以也开了小差,很执着的尾随了他一路。

这两只鬼在县城内的新华书店里扎了根,骨神对着半面墙大的中国地图研究了许久,末了终于确定了回家的路线。将一切主意都打定之后,他还是有些惦念无心,便偷偷摸摸的打算溜回山中探探风声。不料未等启程,他先在县城大街上看到了史高飞。

``我要回家啦。''他告诉无心:``你呢?''

无心答道:``丁思汉死了,我也要回家了。''

骨神扭头望向了大灰雀,怎么看怎么感觉它不对劲:``白琉璃还在吗?''

无心伸手摸了摸大灰雀的脊背:``白琉璃上了它的身。丁思汉够厉害,差点打散了他。''

骨神听了这话,眼珠子立刻滴溜溜一转。心怀叵测的转向了大灰雀,他忽然抬手一指,破口大骂:``臭杂种,上辈子你咒死了我,活该你也短命死得早!我要回家做金光佛去了,你继续当你的孤魂野鬼吧!''

话音落下,他瞬间逃了个无影无踪。未等无心做出反应,天花板上却是又伸进来了一张骨感大脸,正是玛丽莲:``哟,无心?越来越粉嫩了啊,看见米奇了吗?''

无心仰头反问:``丁思汉已经死了,你要跟米奇一起回家吗?''

玛丽莲仿佛是很忙,忙里偷闲的询问,忙里偷闲的点头,说话的速度也十分之快:``主人死了?死就死了吧,反正我现在已经移情别恋了。通过这几天的相处,我发现米奇真是魅力爆棚。实不相瞒我已经向他告白了十几次,但是他一直冷若冰霜,别有一种禁欲的美。跟着我的几个鬼小弟都笑话我,劝我收手,但是我意已决,必要追他到家不可。没办法嘛,爱真的需要勇气,去面对流言蜚语。他到底去哪里了?''

无心没言语,笑嘻嘻的抬手向窗外一指。天花板上光影一闪,玛丽莲也慌里慌张的没影了。

无心很悠闲的继续喂鸟,喂着喂着,史丹凤带着火车票回来了,双手拎着无数塑料袋,全是正当季的新鲜水果。她累得弯腰驼背,两只胳膊被塑料袋沉甸甸的坠着,造型已经类似了长臂猿。然而史高飞双手插兜跟在一旁,并不知道帮姐姐分担一点重量。

史高飞是一贯的不干活,无心在这一个礼拜里被人惯坏了,懒洋洋的趴在床上也不肯帮忙。用一根手指点着大灰雀的脑袋,他小声唤道:``杂种,杂种!''

紧接着他挨了史高飞的一大巴掌:``宝宝,不许说脏话!''

白琉璃闭着眼睛,在烟灰缸里蹲成了灰扑扑的一小团。尖嘴搭在烟灰缸沿上,他和无心在患难之时所见的真情,几乎要在这一个礼拜的太平生活中消耗殆尽。白琉璃认为无心应该隔三差五的受一通折磨,否则在甜蜜的好日子里,他必会慢慢变得贱而聒噪。

史丹凤像个女苦力似的,吭哧吭哧洗了无数水果,用保鲜袋一样一样的装好了,又预备了泡面香肠以及各样零食。凭着一己之力装出两只旅行袋,她在一个阳光明媚的清晨退了房,带着弟弟和无心往火车站去了。

\chapter{苦恼的姐姐}

从昭通到江口,没有直达的列车,所以史丹凤带着大爷似的弟弟和儿子似的无心,挣命似的上车下车再上车再下车。幸而无心越来越有人味,半路忽然意识到了史丹凤的辛苦,于是充当了她的小跟班。

让无心和自己一起坐在下铺的小床上了,史丹凤偷偷的伸了手让他看:``戒指漂不漂亮?''

无心笑眯眯的点头:``漂亮。''

史丹凤小声的问:``是谁给我买的?''

无心先不说话,单是笑,笑着笑着抬手一指自己的胸膛:``我。''

史丹凤谆谆善诱的继续问:``你为什么给我买戒指?''

无心抓住了她的手,眼睛黑黑的亮亮的:``结婚。''

史丹凤任他抓着自己:``我还以为你全忘了呢。''

无心凑向她耳语道:``以后不会忘了。''

下铺一端的枕头上摆着一只方方正正的纸盒子,盒子一面开了个窟窿,里面蹲着白琉璃。静静倾听着无心和史丹凤的来言去语,他听得饶有兴味,感觉他们全都是柔情蜜意的话里有话,每一句都很值得回味。相形之下,白琉璃忽然感觉自己生前的性格好像一管直通通的哑巴炮,一言不合,当即无声的开轰,真是太没有趣味了。

白琉璃津津有味的做了一路的听众,直到火车到了站,无心夹着纸盒子下了火车回家。

北方的四月天还是偏于凉,无心把纸盒子裹到怀里,自己站在火车站外东张西望。上一次坐火车是什么时候?他想了又想,想起了去年的事情——史高飞硬说地球人要迫害他们,悄悄的带他离家出走上了火车。算着时间,他并没有离开江口市很久,然而不知怎的,竟有了再世为人的感觉。火车站外的广场上排着长长一大队出租车,史高飞拉开面前一辆的车门,一歪身坐上了副驾驶座。无心和史丹凤也跟着钻进了后排。搂着纸盒子靠了一侧车门,他把额角抵上了不干不净的车窗。望着窗外大街上的车水马龙,他听到史丹凤一边检查空瘪瘪的旅行袋,一边轻声细语的发牢骚:``在火车上让你们吃,你们都不吃。看看,桃没吃完,全都烂了,扔了可惜,拎着又沉\ldots{}\ldots{}''

无心很惬意的沉默着。他愿意听女人唠叨,尤其是自己喜欢的女人。无论是什么环境,豪宅也罢蜗居也罢,非得里面有个女人忙忙碌碌啰啰嗦嗦,对他来讲,才算是家。拉开拉链掏出了怀中的纸盒子,他把有洞的一面对准了车窗,要让白琉璃也看看自己的新家乡。真是对不住白琉璃了,他想,白琉璃做了几十年的鬼,一定很想化身为人过几天新鲜日子。然而为了救自己,他险些被丁思汉打成魂飞魄散。新鲜日子自然也是过不成了,谁知道他需要多久才能恢复元气?

无心抱着盒子,心想自己不会死,和白琉璃永远是来日方长,将来总有报恩的时候。如果一直不报答的话,白琉璃是个有一搭没一搭的性格,想必也不会在意。

出租车一路疾驰,把车上三人送到了城外郊区的大工地中。而在写字楼的楼下,白大千提前得了消息,已经带着佳琪等候许久。双方见了面,无心一方的三人全都愣了——白大千死去活来的煎熬了两个来月,居然脱胎换骨的变了模样。

本来他生着一张体体面面的大白脸,背头永远乌黑锃亮,然而兴许是最近终日愁苦的缘故,他的背头是不梳了,偏长的头发未经修剪,很颓废的偏分垂下,发梢还打了几个似有似无的卷。大白脸的面积也明显缩小了一圈,导致鼻梁颧骨以及腮帮子全显出了棱角线条。裹着一件半新不旧的厚外套站在写字楼前的水泥地上,他模样一变,气质也跟着变了,好在灵魂还是先前的灵魂。对着无心三人流出了一滴柔弱的热泪,他长长的吁出了一口气,身体又有了要瘫软的趋势:``回来了好。''

佳琪穿着一身很鲜艳的运动服,则是没心没肺的笑嘻嘻,依着次序打招呼:``姐姐,宝宝,哥哥。''

史丹凤和无心答了一声,然后继续欣赏白大千的新形象。史高飞却是老实不客气的说道:``你天天穿运动服,难看死了。''

佳琪傻里傻气的继续笑,显然脾气很好:``新买的。''

白大千自从活活的愁瘦了二十斤之后,很意外的添了几分才子气,而且还是位清高落魄的老才子,仿佛前半生一直怀才不遇。把无心等人向上一直带到了九楼,他自从得到了史丹凤的消息之后,立刻兴奋的重整旗鼓再造河山,退了先前的毛坯房,在九楼租了一套号称是豪装的新房。新房依旧是三室一厅,里面家具家电一应俱全。白大千像个盼儿归的老娘一样,提前把被褥都铺好了。开门把众人全放进去,他最后进门。搓着双手站在客厅里,他感慨万千的长叹一声:``唉,算我白某人还有几分运气,一生没有坎坷到底。年前我去金光寺接佳琪回家,发现汇丰老秃驴又换车了,凭他的熊样,妈的坐六百万的车,我日啊!看得我心都碎了,你们说我比他差什么?凭什么他做宾利私家车,我坐夏利出租车?唉,老天保佑,你们把无心找到了。今天晚上我们出去吃顿大餐庆祝一下,明天公司开门,继续做生意!''

史丹凤听了他东一句西一句的言论,不知道怎么回应才好。无心还搂着他的纸盒子,屋里屋外的到处走。佳琪也不捧父亲的场,自顾自的问史高飞:``哥哥,我绣了好多十字绣,你要不要?''

史高飞答道:``好看吗?不好看不要。''

佳琪扑通扑通的跑进卧室里去了,要向史高飞献宝。白大千和史丹凤一起冷眼旁观,都感觉佳琪似乎是对史高飞有点意思。史高飞对待佳琪的态度也很异常——是非常的老实,以及非常的不客气。

想象了一下史高飞与佳琪的组合,白大千和史丹凤又统一的皱了眉头。一个疯一个傻,组成的家庭会是何等模样,简直让人不堪想象。正当此时,史丹凤的手机响了。

她掏出手机接听了,对方乃是她妈赵秀芬。说了不上三言两语,史丹凤的脸色就变了。原来赵秀芬不甘寂寞,居然又给她找了个相亲对象。说起这位对象,史丹凤还很熟悉。此人和她年龄相仿,乃是她的高中同学,前几年大学毕业之后,因为在外找不到满意的工作,所以回了火星镇,一时宣称自己要考研究生,一时又宣称自己要考公务员,狂言大话说了好几年,至今还是家里蹲,偏偏自我感觉十分之好,不但认定自己是美玉蒙尘,而且放眼天下,看谁都是大傻×。

握着手机顿了顿,史丹凤做了个深呼吸,然后勉强平静了语气答道:``妈,不用看了,我还不认识他吗?说实话吧,我看不上他,我不同意。''

赵秀芬一听,勃然大怒:``哎呀你还看不上人家?你告诉我你想找什么样的?我都这么大岁数了,你还想耗在家里当老姑娘让我伺候你啊?上次给你介绍那个县里钢厂的,你嫌他岁数大不同意;这次给你介绍了个岁数小的,你还是不同意,你想怎么着?你要活活气死我吗?你知不知道你自己是什么条件?你有什么脸挑三拣四?你老大不小的没人要,你不嫌丢人我还嫌丢人呢,我告诉你\ldots{}\ldots{}''

史丹凤低着头,先是一言不发的听,听着听着放下手机等了足有十分钟,拿起手机再听,发现她妈还在兴致勃勃的侮辱她。攥着手机的手指渐渐收紧了,她冷不丁的来了一句:``我有男朋友了。''

她妈愣了一下,立刻换了话题:``谁?多大了?什么工作?一个月挣多少?有房子吗?带不带孩子?''

史丹凤抬起头,发现无心不知何时站到了自己的正前方,正在紧张的盯着自己瞧。像是忽然有了靠山似的,史丹凤渐渐的把话说流利了:``年纪不大,也就二十多岁吧,在一家文化公司里工作,是我的同事。''

她妈安静了半分钟,随即扯了高调叫道:``二十多岁?二十多岁的能找你?你是不是让人家给骗了呀?你啊你啊,你个丢人现眼的东西,几辈子没见过男人啊?二十多岁,你——''

史丹凤从小到大,也不知怎的,在家里永远是挨骂的靶子。赵秀芬在电话里叫得嗷嗷的,整间客厅里的人都听见了,然而她像是已然麻木了,并不动气:``妈,他没什么积蓄,但是很会赚钱,对我也好。我已经决定和他结婚了。''

赵秀芬继续高叫:``结婚?呸!小凤我告诉你,你别把你那不三不四的玩意儿往家里领,你领回来了我也不让他进门!''

史丹凤答道:``好,你让钢厂工人和家里蹲进门吧。''

然后她挂断电话,把手机往沙发上一扔。扭头望着史高飞,她轻描淡写的说道:``咱妈彻底疯了。''

史高飞好奇的注视着她:``姐,你有男朋友了?''

史丹凤对着无心一抬下巴,心里像被大风刮过了一场,满腔冰冷狼藉:``是无心。''

史高飞登时目瞪口呆:``无——宝宝?''

史丹凤点了点头。

史高飞抬手一拍自己的胸膛:``我的宝宝?''

史丹凤``嗯''了一声。

史高飞在刹那间跑到无心身后,一把将他搂到了怀里:``姐你开什么玩笑?他是我儿子,又不是你儿子,你凭什么喜欢他?''

无心一直没言语,此刻却是扭头给了史高飞一个侧影:``爸,我给姐做男朋友也不是不可以啊!你看姐多可怜,天天被家里人逼着去相亲,还不是因为她没有男朋友?如果她有了男朋友,岂不是皆大欢喜,她也不用天天挨骂受气了?''

史高飞紧紧的抱着他:``胡说八道,她是你姑姑!''

无心向着史丹凤一挤眼,随即一本正经的继续说道:``爸,我们不用按照地球人的规矩排辈分。''

史高飞依旧不肯松手:``不行,你还小呢!''

无心慢悠悠的苦口婆心:``爸,助人为乐不分老幼。你真忍心看姐天天赌气?你忍心我还不忍心呢!姐天天照顾我们,给我们买东西吃帮我们洗衣服,现在她遇到了困难,我们帮她也不也是理所当然的吗?''

史高飞无端的心酸了:``宝宝,爸爸不想让你结婚,爸爸养你还没有养够呢。''

无心把嘴唇抿成了一条线,忽然感觉很幸福:``爸,我长大之后迟早是要和女人结婚的嘛!我要是和别的女人结婚了,她对你不好怎么办?与其如此,不如让我和姐姐结婚。姐姐肯定不会让我离开你的,你说呢?''

无心侃侃而谈,史高飞听得心乱如麻,一时间竟是没了主意。最后他低了头,对着儿子的后脑勺闷声闷气的说道:``宝宝,爸爸心里好难过。''

无心笔直的站了,拼命的向史丹凤使眼色。史丹凤先还不明就里,后来恍然大悟了,她坐在沙发上叹息复叹息,又幽幽的甩闲话:``反正我要被妈活活的逼死了。家我是不能回了,小飞,念在你我姐弟一场,哪天姐要是一时想不开有个三长两短的了,后面的事情你就给姐办了吧!''

史高飞莫名其妙的抬起头:``办什么?''

史丹凤扭头去望窗外:``火化呗!''

史高飞张了嘴:``啊?''

史丹凤越说越真:``也不用把我往家里送。我在家挨了三十年的骂,早挨够了。再说我丢了他们的脸面,你真把我送回去了,他们也未必肯接收。''

话到此处,她想再流几滴眼泪渲染悲情,可惜平时刚强惯了,眼泪不能呼之即来。史高飞怔怔的看着她,也知道姐姐像个奴才似的一直对自己照顾有加。很为难的垂下眼帘,他把双臂搭在无心的肩膀上,拧着眉毛思索了半天。

末了抬起了头,他认认真真的对史丹凤说道:``姐,你别寻死觅活的了,我同意把宝宝借给你,但是,只能借你一下下\ldots{}\ldots{}''他神情严肃、举止幼稚的竖起了一根手指:``一下下哦!''

放下了手重新勒住无心,他继续说道:``等到爸妈不再闹你了,你还得把宝宝还给我。刚才我心里难过死了,都要吐血了。真舍不得把宝宝借给你,可是宝宝说得也对,做人应该知恩图报。如果我现在对你不好,宝宝将来也许会跟着我学坏的。''

史丹凤听得有些迷糊,没有掌握他这话的中心思想。无心却是有了主意。志满意得的向后一仰,他靠着史高飞的胸膛大声问道:``姐,怎么才能让所有人都知道我们结婚了呢?''

史丹凤看了白大千和佳琪一眼,发现这二人全是一脸懵懂。犹犹豫豫的开了口,问题来得太急了,她一时也没有准主意:``那\ldots{}\ldots{}得回镇里办婚礼吧?''

话说到这里,就可以暂时告一段落了,因为目前史丹凤还没有回家乡的计划。史一彪是见过无心一面的,不知道他的记忆力如何,如果当真还记得无心,那她就还得再做一番解释。可是该怎么解释呢?她目前仍然没有主意。

无心挣开史高飞的双臂,偷偷溜进了卧室里。关上房门蹲到床边,他攥了拳头砸上大床,然后对着枕头上的白琉璃张大嘴巴,开始无声的大笑。

白琉璃通过一双黑豆似的鸟眼睛,看清了无心的后槽牙以及嗓子眼。他静等着无心笑完说话,可是无心从嗓子眼里往外出气,持久的哈哈不止。白琉璃等得忍无可忍,拍着翅膀向前跃了一步,他在无心的嘴唇上啄了一口。

无心果然立刻闭了嘴,然而依旧挤眉弄眼:``白琉璃,我要结婚了!''

伸出双手捧起白琉璃,他把声音压到了极低:``我又要有家了,家里有姐有爸,他们两个全都不会离开我。还有婚礼——我好像有好几百年没举行过婚礼了。白琉璃,你想不想看我做新郎官?''

白琉璃躲在大灰雀的身体里,本来是个百无聊赖的状态,可是见无心说得眉飞色舞,像是受了感染一般,他也跟着兴奋了。

无心很想和史丹凤做一对小夫妻,可同时又生怕伤了史高飞的心。第二天他随着白大千在公司里坐了半天,下午他告了假,要带史高飞去市区散心。

这一趟倒是不白走,史高飞在市中心的电子大世界里买了一台笔记本电脑。到家之后开了机,无心左手托着大灰雀,右手在键盘上试试探探的乱摁,又问:``爸,你的电影呢?我想看一看。''

史高飞没听明白:``什么电影?''

无心的声音瞬间低了:``就是男女在一起睡觉的\ldots{}\ldots{}原来我看过一次,当时你不让我看\ldots{}\ldots{}''

史高飞当即``哦''了一声:``我没有下载,你当然看不到了——还有,你小孩子不许看。''

无心以为电脑和电视机是一类的货,频道列好了,可以让人自由选择。

他很沮丧,小声问道``爸,我能看什么啊?''

史高飞答道:``桌面上不是有猫和老鼠吗?自己看吧!''

无心会用电视机,但是永远不能理解电脑的工作原理。他不会打字,也用不好鼠标,只能看猫和老鼠,以及玩连连看。玩连连看的时候白琉璃出了大灰雀的窍,坐在他的肩膀上大肆指挥,吵得无心心烦意乱。

后来趁着史高飞不在家,无心把史丹凤叫进了卧室。打开电脑摆在床上,他赖皮赖脸的笑道:``姐,你给我下个崽。''

史丹凤立刻给了他一巴掌:``放屁!胡说什么呢?''

无心把嘴唇撅到了她的耳边,嘁嘁喳喳的作了解释,然后又把史丹凤的脸扳向自己,用力的亲了一口。史丹凤象征性的躲了一下:``烦人,别闹。''

在她操作电脑的时候,无心跪在了她的身边,从头到脚全不老实:``姐,摸一下。''

史丹凤让他摸了一下,随即打开了他的手,因为感觉自己又要被他摸成大灰狼了。

两人正是动手动脚之时,客厅房门一响,是史高飞从楼下的公司上了来。在门口换了拖鞋,他大声说道:``姐,爸刚才给我打了个电话,说是过几天要来江口,顺便看看我们。''

史丹凤听闻此言,登时傻了眼——史一彪是富有智慧的,比赵秀芬更难对付。

史丹凤丢开电脑,开始绞尽脑汁的要为无心编个来历。第一天过去了,她没编出什么;第二天又过去了,她给无心编出了若干个漏洞百出的身世,没有一个是合格的。到了第三天,史一彪来了。

史一彪最近走出小镇,立足县城,放眼城市,事业有了大大的发展,座驾也从丰田霸道升级为保时捷卡宴。肉山一样坐在副驾驶座上,他身边是兼任保镖的司机,身后满满登登的挤着一排美女,乃是他近几年包养的二三四五奶。二三四五奶的年龄没有过二十三的,全都生得苗条风骚,四把小腰掐一块儿,也没有他一个人的腰粗。此行从火星镇开车到江口市,史一彪一是想为二三四五奶消费一番,二是听家里的黄脸老婆说女儿居然在外找了个小男朋友,他好奇得要死,几乎夜不能寐,非要亲自来看一眼不可。

\chapter{皆大欢喜}

史一彪在城郊写字楼前下车之时,迎接他的只有史丹凤与史高飞——谁也没料到他心宽体胖之余还能来去如风,接到电话通知之时,他的汽车已经飞驰在江口市内的大街上了。

二三四五奶全被史一彪卸到了市区内的宾馆里,所以他属于轻装上阵——也是对女儿太没信心,生怕她找了个浪里浪荡的小痞子,再惹得二三四五奶们私下笑话。一只大胖手推开车门,一只大胖脚踏上地面,史一彪找好着力点了,紧接着向外猛然一甩周身肥肉,算是使了个巧劲,硬把自己甩出了车。

威武雄壮的站在一对儿女面前,他仰起头先看写字楼的全貌,随即笑眯眯的抬手一拍史高飞的肩膀:``小飞瘦了,小凤也瘦了。''

史丹凤惶惶然的请父亲进入楼内,一边等电梯一边又道:``不知道爸能来得这么快,要是早知道的话,今天就不让无心出门了。这一阵子客户特别多,他早上五点钟就起了床,和我们老板到郊县给人看风水去了。''

史一彪环顾着周遭环境,感觉虽然外面是个大工地,但是楼内的装潢还算有档次:``无心?你那个小男朋友啊?他还会看风水?''

电梯门开了,史丹凤先让史一彪进,再让史高飞进:``他\ldots{}\ldots{}不大会,看风水主要是我们老板的工作。''

史一彪登时了然——女儿果然找了个一无所能的小混混。

电梯上了九楼,三个人进入走廊。史丹凤掏钥匙开了房门,房内收拾得窗明几净,佳琪昨天去了金光寺没回来,所以此刻客厅里的活物只有一只大灰雀。史高飞知道大灰雀是儿子的宠物,所以小心翼翼的捧起了它,一直将其送到了卧室窗台上。

史丹凤殷勤的招待着父亲,端茶递水嘘寒问暖,不说强说不笑强笑。史一彪看了她这副心虚的做派,越发不由自主的要起疑。将一盘切开的橙子摆到茶几上,史丹凤以着去卫生间洗手的借口,偷偷的叫出史高飞,暗暗的对他做了一番嘱咐。史高飞的思路素来异于常人,导致史丹凤说话说得十分费劲,正是急得她满头冒汗之时,外面房门忽然响了。

她吓得一哆嗦,当即做了个向后转,两条腿长得无边,仿佛一个劈叉便迈回了客厅。而早归的无心一手扶着门框,一手拎着个印了八卦阴阳鱼的黄绸子口袋。面对着沙发上的史一彪,他显然也是愣了一瞬。不过一瞬过后,他立刻笑着开了口:``叔叔来了。''

史一彪的颈椎骨在层层肥肉中一拧,身未动,头已转。上下将他打量了一阵,史一彪忽然抬手一拍膝盖:``哎?你不是那个谁吗?''

不等无心回答,他飞快的面对了史丹凤,瞠目结舌的问道:``他原来不是小飞的吗?怎么?小飞把他转给你啦?''

赶在史丹凤开口之前,他很灵活的又转向了无心:``我说小子,你又不搞同性恋了?''

无心张开了嘴,想要解释几句,然而史一彪自有一颗肥而不腻的七窍玲珑心,在没问够之前,根本无须答案。目光像网似的兜住了一儿一女,他自顾自的继续发言:``怎么回事?弟弟用完姐姐用?平时也没看出你俩有什么共同点,怎么在这事上面审美观这么统一?''

下一秒,他重新叨住了无心:``你到底是何方神圣?兔子还不吃窝边草呢,你说你——''

话没问完,史一彪忽然感觉自己的逻辑出了一点小问题,想了一想,他不由得困惑了:``你是兔子还是草?你们三个之间,是谁先勾搭的谁?去年小飞是不是被你拐走的?''

最后他对史高飞招了招手:``小飞,你是好孩子,跟爸说实话。''

史高飞刚受了史丹凤的嘱咐,所以现在能够一言不发的保持沉默。无心的拖鞋被史一彪穿去了,光着两只脚站在地板上,他眼睁睁的望着史丹凤,不知道自己应该作何反应。

史丹凤度过了最初一阵的慌乱,此刻倒是渐渐的镇定下来了。她支使史高飞去厨房拿纸杯倒热水,又从鞋柜里翻出拖鞋给无心穿了上。无心走回卧室去放黄绸子口袋,一边走一边压低声音对史丹凤说:``姐,这太突然了。''

史丹凤的头脑正在高速运转,简直无暇理他。忙忙碌碌的又搬出了三只塑料小板凳,她把史高飞和无心召唤到了客厅,隔着一张茶几,三个人并肩坐在了史一彪面前。史丹凤深吸了一口气,然后对着史一彪说道:``爸,其实有些话,我一直没和你说明白。无心和小飞的感情是很好,比亲兄弟还要亲,但并不是同性恋关系。而我和无心朝夕相处了几个月,互相产生了好感,他也不嫌我年纪比他大,所以\ldots{}\ldots{}''

话没说完,史高飞开了口:``哈?我刚听明白,原来竟然还有人以为我和宝宝在搞同性恋?哎呀他们内心太阴暗了!''

史一彪不敢和儿子争风头,等儿子闭嘴了,他才对着史丹凤问道:``小凤,你说的话,爸可以信。但现在爸最想知道的是——''他用胡萝卜似的食指一指无心:``这小子究竟是个什么身份?他老家在哪里?父母都是干什么的?他跟你们在一起混了小一年了,你们不应该对他一无所知吧?说句老实话,我听你说了这么半天,现在就感觉他是从天上掉下来的。''

此言一出,史高飞嗤嗤的笑了,可是笑而不语,是个莫测高深的模样。

史丹凤被父亲问到了山穷水尽的地步,不动声色的又做了个深呼吸,她正打算给父亲讲故事,不料未等她出声,无心忽然出了声:``叔叔,实不相瞒,我是个孤儿,生在大兴安岭,不知道父母是谁,从小是吃百家饭长大的。因为是黑户,所以也没有正式上过学。您说我像是从天上掉下来的,我同意。''

这一段话让他说得正正经经冷冷清清。史丹凤惊讶的扭头望了他,因为平日见惯了他嬉皮笑脸的贱样子,万没想到他也会严肃。而无心微微的垂了头,很平静的继续述说:``我一直生存得很辛苦,直到去年遇到了他们。''

向着史家姐弟的方向一偏头,他笑了一下:``他们对我非常好。''

史一彪的脖子将要有腰粗,脑袋陷在脖子里,他大幅度的点了点头,点得整个上半身一起颤了颤:``哦,原来是个苦孩子。''

无心察言观色,见史一彪已经摆出倾听的姿态了,他当即趁热打铁:``叔叔,我虽然没念过书,但是自学过文化,也有谋生的本领,结婚之后,绝对不会让姐跟着我吃苦受穷。''

史一彪眨巴眨巴陷在肉里的一对眼睛,沉吟着不肯表态。照理来讲,姑娘大了不中留,尤其姑娘自身也没有大本事,离了家庭的资助,凭着她每个月的千八百块工资,很有饥寒交迫的可能。当然,姑娘无才有貌,然而话说回来,再美今年也三十一了。放到火星镇,至多是找个二婚男人,还得是条件不大好的二婚。如此一想,他恨不能一把抓住无心,把他和史丹凤捏成一对揉成一团。想他史一彪英雄一世,家里的一双儿女却是全打光棍,无论如何都是好说不好听,所以,嫁出一个是一个。

史一彪不言不动,脑筋和心眼却是一起在飞转。两只眼睛被肉挤成了缝,以至于目光集中,分外锐利。一瞬间的工夫,他已经从婚礼酒席名单想到了无心婚后出轨的可能性。小白脸子要是出了轨,家里的臭丫头片子必定要找自己哭诉。自己怎么办?很简单,派几个人上门打断他的狗腿。他再不服,直接让他净身出户。

冷不丁的,他忽然来了一句:``无心,当我家的上门女婿行不行?''

不等无心回答,史丹凤先发了话:``不行。要说将来生了俩孩子,选一个跟我姓史可以;但是不能让他倒插门。''

史一彪虽然已经思索到了一定的深度,但在精神上还是有些恍惚——当真有人肯要自家的孩子了?家里的儿女已经被他四处推销了五六年,一直滞销,如今骤然出手了一个,让他不由得生出了难以置信之感。

于是史一彪又不说话了,把一只不锈钢的小盘子摆在自己浑圆凸起的大肚皮上,他心事重重的吃光了盘中橙子。抽出一张餐巾纸擦了擦手,他迟迟疑疑的想:``其实真把小凤给他也行。小凤不傻,也没和别的小子狗扯羊皮过。她说他好,必定是他真有优点。但还是太穷了,孤身一人,没有后盾,我是不是还得给他俩预备房子?''

思及至此,史一彪下了结论:``丫头片子,真是赔钱货!''

史一彪没说同意,也没说不同意。喝光了一纸杯热水之后,他手扶大腿起了身:``走了,外面还有人等着我呢。你俩有时间回趟家,看看你们妈。''

史丹凤很识相,满面春风的起了身,带着弟弟和无心恭送父亲的大驾。及至史一彪坐着汽车无影无踪了,史丹凤在楼下大厅中双手插兜,眯着眼撇着嘴长出了一口气:``哼哼,过了一关!''

史高飞和她擦肩而过,直奔电梯,同时头也不回的说道:``姐你表情好猥琐。''

史丹凤立刻恢复了原形:``小飞,你看出来了没有?爸至少是不反对了。
无心,你今天表现得也很好,我真怕你对着我爸说实话。''

无心伸手在裤兜里掏啊掏,掏出了一块太妃糖。慢条斯理的剥开糖纸,他低头把糖填进嘴里,然后又用胳膊肘一杵史高飞:``爸,我们以后在地球不能随便暴露身份,你要向我一样,学会保守秘密。''

史高飞认真的点头:``宝宝说得对,我应该向蜥蜴星人学习,纵算不能入乡随俗,起码也得能够自保。真是的,蜥蜴星人怎么总不给我打电话?难道他回他的母星了?不会吧,凭什么他运气这么好,想回去就能回去?''

无心含着太妃糖安慰他:``爸,你不要羡慕他,他没儿子,很孤独的。''

史高飞一听,立刻心悦诚服的又庆幸了。

史丹凤上了三楼,直接去了公司。白大千为了能够有朝一日在经济层面上挑战汇丰大师,如今斗志极强,每天挑灯夜战苦读易经,并且终止了他单枪匹马的作坊式炒作,改为雇用网络水军,请职业人士为自己出谋划策。他新购置了一台电脑,把旧货淘汰给了史丹凤,并且让她勤于上网,留意各大论坛的舆论动向。

史丹凤没了心事,一身轻松的打开电脑。片刻之后她起身绕过屏风提醒道:``白大师,今天你还没更新博客呢!''

白大千还没来得及去剪头发,金丝眼镜前几天被佳琪一屁股坐扁了,他在附近的小眼镜店里配了副黑框眼镜。听了史丹凤的话,他愣了愣,随即却是问道:``丹凤,你会用PS软件吗?''

史丹凤思索着答道:``会一点点,也就是美化照片的水平。''

白大千立刻给她派了任务。而史丹凤逃之不及,只好乖乖坐回前台,在接电话卖护身符之余,开始处理白大千近几天所照的无数照片。忙了片刻之后,她隔着屏风高声问道:``白大师,你喜欢写实风格还是梦幻风格?''

白大千扯着大嗓门答道:``写实风格,不用太帅。''

史丹凤从早忙到晚,专门为白大千PS照片,小P而已,并不伤筋动骨。而白大千每隔一两天,必定发表新博客,博客内容除了东拼西凑故作玄妙的杂文之外,必定附加几张他精挑细选出的照片。如此过了不到半个月,一家大论坛中出现了关于他的帖子,标题为《阴阳师大叔你敢不敢不要这么帅!
》帖子里面收录了他博客中的所有照片,照片中的白大千长发遮眼手指夹烟,时而行色匆匆时而静坐沉思,偶尔几张还蓄有稀疏的胡渣。帖子立刻被顶为热帖,白大千不但在网上出了名,连写字楼内的人们也开始重新审视了他。年轻的女职员们都说他有日系大叔的范儿,背地里给他起了个爱称叫欧吉桑,并且统一得了失忆症,把他先前油头粉面的唐装造型忘了个一干二净。

白大千十分得意,还给史丹凤买了一套化妆品,以示奖励。本来他对史丹凤很有一点蠢蠢欲动的小意思,但是史丹凤既然有眼无珠的看上了无心,而他又是老树发新芽,帅得一塌糊涂,自然眼界就要高于先前,不能再天天对着自家的前台小姐思春。

白大千的本事没有增长许多,但是名气一大,生意立刻就跟着红火了。史丹凤忙得坐在前台无暇起身,无心也是终日跟着白大千东跑西颠。家里只剩了史高飞和佳琪两个闲人。佳琪听见史高飞的房里有声音,便抱着一堆膨化食品走了进去:``哥哥,你干什么呢?''

史高飞坐在床上,腿上摆着笔记本电脑:``我看动画片呢。''

白大千总怕自己的女儿会被臭小子看上,所以佳琪不施粉黛,永远只有运动服穿:``我有好吃的,我们一起吃一起看吧!''

史高飞看了她一眼,又``嗯''了一声。

佳琪欢欢喜喜的坐到了史高飞身边,一边用力去撕薯片的包装袋,一边伸着脑袋去看屏幕。正好一部动画片演完了,史高飞关了浏览器,自己仰起头揉眼睛。佳琪伸手去摸了触摸板,移动了鼠标箭头乱点。浏览器忽然重新开了,屏幕先是一片黑,黑了片刻之后,开始有动感音乐响起。

佳琪往嘴里填起了薯片,史高飞也低下头睁开了眼睛。短暂的音乐过后,黑屏幕变成了一片肉色,佳琪莫名其妙的盯着看,没看明白,只听到了一个女人哼哼唧唧的叫了一长串。

下一秒,镜头拉远,她看明白了。史高飞则是睁大了眼睛,不知道自己的电脑里怎么会有爱情动作片。

史高飞瞪着眼,佳琪含着薯片,两人全盯着屏幕呆住了。

大家都很忙,所以没人去管九楼的闲人们。无心和白大千是在天黑之后才回家的,史丹凤比他们早一点,早不了几分钟,因为连衣服都还没换。

白大千到家的第一件事就是算账,想要看看自己今天发了多大的财。史丹凤系着围裙进了厨房,一边洗菜一边对无心说道:``你猜怎么着?妈上次还往死里骂我呢,这回可能是听爸说你好了,竟然让我请假带你回一趟火星镇。可我哪有时间回去?''

无心伸手端过了菜盆:``我洗,你切。''

房中众人动脑的动脑,动手的动手。史高飞晃着大个子出了卧室,低着脑袋走去了卫生间。抬手搭上卫生间的门把手,他转动眼珠瞟了瞟屋中的三个人,同时下意识的一蹙眉头,随即垂头丧气的开门进了卫生间。

白大千坐在茶几上,双手一起摁计算器,嘴里一五一十的念念有词。佳琪从史高飞的房中一闪身溜回了自己的卧室,他也没有留意。

当晚众人吃饱喝足,各自回房休息,其中白大千是无房可睡的,只能在沙发上安身。无心是个懒人,今天奔波了一整天。上床之后把大灰雀捉到枕边放好,他闭上眼睛就睡了。而史高飞仰卧在一旁,枕着双臂望着天花板。双目炯炯的望了良久,他欠起身,伸了脑袋去看无心。无心背对着他,一手缩在被窝里,一手向上搭在大灰雀的后背上。大灰雀缩成了一块灰石头,显然也是在睡。

史高飞抬手轻轻摸了摸他的头发,心中暗叹:``幸好宝宝来的时候只是一颗卵子,没有妈妈。''

低头亲了亲无心的短头发,史高飞蹑手蹑脚的下了床,把耳朵贴上墙壁,去听隔壁佳琪卧室里的动静。隔壁很安静,他赤着脚站在地板上,凉气顺着小腿往身上爬。

天刚一亮,客厅里就热闹了。

佳琪从外面买回了馅饼,史丹凤站在厨房里煮米粥,白大千和无心蹲在地上,整理一箱子崭新的五行八卦福。煮粥的空当里,史丹凤接了个电话。放下电话之后她出了厨房,对着无心又惊又笑:``妈说她给我做了八床被褥,当嫁妆。还说要把婚礼的日期定在六月,现在都快到五月了,六月怎么来得及?''

无心仰起脸望着她笑:``也来得及。''

史丹凤做了一番心算,末了点头笑了:``是来得及。''

白琉璃本来正蹲在窗台上啄小米,忽然听了无心与史丹凤的对话,他转动脑袋看了看无心,发现无心依旧低头整理着箱子,一张脸却是越来越红。一翘尾巴在窗台上拉了一点鸟屎,白琉璃横挪了一步,随即振翅一跃,扑啦啦的飞到了无心的后衣领里。无心抿嘴笑着一缩脖子,不出声。白大千忙里偷闲的瞟了他一眼,随即对史丹凤说道:``看给无心美的。''

正当此时,史高飞出了自己的卧室。佳琪站在厨房门口,向史高飞打招呼:``哥哥。''

史高飞不看她,换了鞋子推门往外走。佳琪的脑子永远比旁人慢好几拍,今天却是格外的机灵了,居然在半分钟后便意识到了史高飞情绪不对头。跑到门口趿拉了鞋,她慌里慌张的追了出去。

史丹凤回到了厨房,通过窗户往外看,看到弟弟正在一蹿一蹿的在街上走,正是个火气不小的样子。佳琪飞快的从后方追上去,且从街边的章鱼丸子摊上买了一盒丸子。殷勤的凑到他身边,佳琪把丸子往他面前送,然而他驴头驴脑的一眼不看,走得却是越发快了。

\chapter{大喜之日}

结婚这种事情,说复杂可以很复杂,想要简单,也能非常简单。无心把自己近来所赚的钱全给了史丹凤,而史丹凤给自己买了一根项链,一对耳环,以及几套鲜艳的裙装。这几样装备足以让她风风光光的回家乡,对于她妈赵秀芬,也算是有个交待了。

赵秀芬仿佛终于找到了人生的目标,一天三次的给她打电话,也不吵了也不骂了,心平气和的告诉她``饭店已经定了'',``迎亲车队凑齐了'',``婚礼司仪是你爸从县里找的'',``婚纱你不用管,我替你定了。你大表姨的二女婿的三舅母在县里做婚纱摄影,店里有全新的婚纱,没上过身的''。

史丹凤连着好几天没挨骂,很不习惯,堪称惶恐,同时发现她妈好像也不打嗝了。

史丹凤想把无心也包装一番,然而史高飞不肯。史高飞前一阵子闹邪脾气,总像是憋着要大疯一场,吓得家中其余众人全成了避猫鼠,连白大千都不敢在家高谈阔论了。然而邪脾气也是有保质期的,沉着脸过了一个礼拜,他渐渐的从阴转了晴。佳琪厚着脸皮再跟他搭话,他也肯理睬她了。

史高飞不让无心和史丹凤出门,自己却是带着他左一趟右一趟的逛大街。宝宝要结婚了,爸爸很忧伤,同时也很嫉妒,因为儿子还小,自己养他还没有养够。天气越来越温暖了,也不下雨,从早到晚永远是阳光明媚。史高飞疯狂的给无心买吃买喝买穿买戴。其中穿戴全是双份的,他自认为是在和儿子穿父子装,外人自然也管不了。史丹凤和白大千看他天天和无心穿着情侣装到处跑,统一的都愁死了,感觉他俩真是影响公司形象。

白大千给史家姐弟和无心放了一个礼拜的婚假,自己预备在公司里独挑一个礼拜的大梁。而在启程回火星镇的前一晚,史高飞带着无心去市中心吃重庆火锅。无心新得了个触屏手机,在火锅开锅之前,他调出了手机里的小游戏。白琉璃被他随身携带了来,此刻站在手机屏幕前,他开始用尖嘴去啄屏幕上的小飞机,小飞机四处乱飞,然而他一啄一个准,啄中之后发出``啪''的一声响,是小飞机四分五裂的爆炸了。

无心任着他玩,自己伸筷子从火锅里捞出了一块肉。先把肉塞到嘴里唆了唆红油,他用牙齿小小的咬下了一点肉丝,用手指捏着要去喂给白琉璃。

然而白琉璃忙着打飞机,不肯吃。

当天晚上,史丹凤和史高飞蹲在客厅里收拾行装。无心独自趴在卧室床上玩手机。白琉璃忽然脱了大灰雀的身,飘飘摇摇的悬在了他的正前方。无心抬头向他笑了笑,然后低头继续玩手机。

白琉璃开了口:``无心,很奇怪,你为什么要认一个年轻人做父亲?''

无心回头望了望房门,见房门是紧闭着的,便压低声音答道:``他对我很好,我愿意给他做儿子。况且对我来讲,年龄是没有意义的。''

白琉璃听得心悦诚服,乖乖回到了大灰雀体内。照理来讲,无心已经把话说得十分清楚了,本无须再赘言。可是盯着手机沉默片刻,他嘴皮子做痒,意犹未尽的又抬起了头:``白琉璃,算年龄的话,你也是我的灰孙子哩!''

话音落下,他被大灰雀凶猛的啄了一头包。

翌日清晨,无心跟着史家姐弟出了门。天气热了,三个人全穿着短袖T恤,又因为无心和史高飞的T恤前襟全印着一只愁眉苦脸的大狗,史丹凤看了不忿,也穿了一件同样图案的女款T恤。三个人上了火车,一名中学男生坐在他们对面,对着他们看了又看,怀疑他们是小动物保护协会的人,因为胸前的狗脸苦大仇深,正是个受了迫害的模样。

半天之后,他们在县火车站下了火车。史一彪驱车前来迎接,把他们一直送到了火星镇的家中。赵秀芬薄施脂粉淡扫蛾眉,风韵犹存的前来招待新女婿。没等无心在史家坐稳,史丹凤的七大姑八大姨闻听了消息,当即蜂拥而至,倒要看看史家的大姑娘在江口市找了个什么货色。及至进了史家的门,七大姑八大姨们围住无心,先是对着他一起愣了一下,随即笑眯眯的拐弯抹角,开始问他的岁数。无心坐在沙发上,因为从来不曾受过如此虎视眈眈的注目,所以几乎要脸红。史丹凤坐上了沙发扶手,替他答道:``二十二,还小呢!''

七大姑八大姨们知道她是找了个小女婿,然而没想到竟然小了将近十岁。欢声笑语的恭喜了一番之后,众妇女们络绎告辞。及至走远了,她们把史家的新闻重新嚼了一遍,末了达成两点共识:第一,无心肯要史丹凤,肯定是看上了史家的钱;第二,史丹凤找了个比自己小十岁的,将来必定会被小白脸子抛弃。终上所述,史家大姑娘还是不幸福。

赵秀芬并不知道老姐妹们的高论,忙忙的带着女儿试婚纱,又向家里人展示了她雇人缝纫的八床被褥。史一彪一句好话没有,说预备被褥之举纯属山炮行为。赵秀芬看了他一眼,见他穿着一身单薄的黑色衣裤,桥墩子似的站在屋子中央,身上肥肉一圈叠一圈,掐掉脑袋就是一大摞轮胎。

一声不吭的收回目光,赵秀芬忽然感觉自己不再爱史一彪了。

史一彪原地转了个方向,继续发言:``无心,有句话我早就想对你说了——你能不能把你那美瞳给我摘了?人家小姑娘爱漂亮,整俩大黑眼珠子还情有可原。你一个小伙子,跟着凑什么热闹?你说你把自己弄得像大眼贼似的,好看吗?''

无心没听懂,惶惶然的去看史丹凤。史丹凤当即伸手去揉了他的眼睛:``爸,你看清楚了再说话。什么美瞳啊,他是天生的大眼睛,天生丽质难自弃,要摘就得摘眼珠子了。''

史一彪在镇上另找了一套单元房做新房,无心和史高飞在新房子里住了两天,第三天到了婚礼吉日,他兴奋得失眠了整整一夜。凌晨时分起了床,他和史高飞在卧室里试穿西装。白琉璃蹲在床头,看史高飞蹲下去给无心整理裤脚,又听无心小声的叮嘱史高飞:``爸,千万记住了,今天当着外人的面,你别叫我宝宝,我也不叫你爸。''

史高飞抬了手,把两只胳膊肘架在了膝盖上。深深的垂下头,他用力的一吸鼻子,随即带着哭腔开了口:``你被地球人拐走了。''

然后他开始吭哧吭哧的闷声哭泣,鼻涕眼泪淌了满脸。无心扯了一大团面巾纸,俯□四脚着地的歪了脑袋,从下向上去看史高飞。史高飞的睫毛湿漉漉的,泪水大滴大滴的往下落,噼里啪啦的全砸上了地板。忽然察觉到了儿子的目光,他抬起双手一捂脸,索性哭出了声。

无心去拉他的手腕,拉不开,只好用面巾纸擦拭他指缝中溢出的眼泪。而史高飞躲在自己的大巴掌后面,哽咽着说道:``没想到\ldots{}\ldots{}你还没满周岁\ldots{}\ldots{}就被他们捉去结婚了\ldots{}\ldots{}我自己也很失败,竟然和地球人\ldots{}\ldots{}我们父子两个,在地球上真是全军覆没了\ldots{}\ldots{}''

无心听糊涂了,一时间不知如何安慰他才好,只好陪着他在地上跪了半天。幸而史高飞以大局为重,哭过一阵之后也就收了眼泪。红着眼睛和鼻尖站起身,他拉开窗帘站到窗前,在黎明的霞光之中给无心系领带结。

史一彪虽然重男轻女,但是并不肯潦草对待女儿的婚礼。东拉西扯的调动来了二十辆黑色奔驰,他组成了一支整整齐齐的迎亲车队。装点车队的鲜花则是连夜从县里运过来的,每一朵花都鲜灵灵的带着水珠。无心坐在头车的副驾驶座上,手里捧着一束沉甸甸的玫瑰花。史高飞和伴郎坐在后排,史高飞双手托着儿子的宠物大灰雀,伴郎是个十八岁的英俊小伙子,腿上放着个大旅行包,里面装着无数红包。

车队开得很慢,无心每隔一分钟便要回一次头,看看史高飞,看看白琉璃。白琉璃卧在史高飞的手里东张西望,黑豆大的两只眼睛滴溜溜乱转。及至车队到了史家楼下,在鞭炮声中,无心抱着玫瑰花下了汽车。听说史家的老姑娘终于出阁了,整座小区的居民们都起了个大早,来看看史家新女婿的真面目。新女婿穿着一身银灰色的笔挺西装,宽肩长腿浓眉大眼,面孔本是雪白的,然而被胸前的玫瑰花烘出了满脸鲜艳的朝霞,黑色瞳孔中更是映出了点点红花瓣,花瓣随着他的目光闪烁流转。

观众们都感觉自己没白起早,不虚此行。如此漂亮的新郎,实在是难得一见的。

等到摄像师摆好机器镜头了,新郎率先进入了单元楼内,紧随其后的是全副武装的伴郎,史高飞排到了第三位,手里托着一只大灰雀。

史家房门紧锁,但是无心在伴郎的指导下只略求了几句,赵秀芬便让人开了门。史家装了一屋子大姑娘小姑娘,叽叽喳喳连说带笑,因为新郎是特别的年轻貌美,所以全憋着劲要狠狠刁难他一顿——直到史高飞托着鸟进了门,面沉似水的扫视了她们。

史家大姐可以不必怕,史家二哥却是不能不怕的。姑娘们立刻端庄了许多,也不堵着卧室房门逼新郎学狗叫了。无心有些遗憾,因为难得能有机会做新郎,他不在乎学狗叫发红包。

赵秀芬和颜悦色的叫了史高飞,让他去脱了西装外套擦一擦汗。想方设法的暂时把儿子哄走了,她很得意的让女孩子们继续闹。然而女孩子们闹着闹着又不闹了,因为史高飞拿着一条毛巾回了客厅。高人一头的站在正中央,他一手托鸟,一手擦汗,表情十分不善。

赵秀芬不敢再闹,开了房门把无心放进了卧室。而史丹凤穿着婚纱坐在闺房床上,因为凌晨被本镇的首席化妆师美化了一番,所以此刻脸蛋极其红,眼皮极其亮,眼线极其黑,睫毛极其长。羞答答的瞟了无心一眼,她微微扭脸垂头,一脑袋长卷发被发胶固定在了头顶心,从背后看,脑袋上下足有两尺多长。

无心一直认为史丹凤长得好,是天生的美人坯子,然而如今见了她如鬼似魅的新形象,不由得也有些腿软。史高飞从门口向内伸了脑袋,看清他姐之后惊了一声:``我的亲娘啊!''

赵秀芬立刻殷勤答道:``儿子,叫妈有什么事?''

无心把史丹凤拦腰抱出新房,下楼送进了迎亲的头车里。又因为无心无父无母,所以车队直接开去了饭店。饭店和宴席自然也全是本镇最高级的,一共摆了一百多桌。典礼结束之后,史丹凤在饭店里脱了婚纱改穿旗袍。身体一苗条,越发显得她头型霸气。在她和无心挨桌给客人敬酒之时,史高飞追着她那个高耸入云的脑袋瞧,越瞧心里越难过,感觉自家宝宝被他姐这个老娘们儿给玷污了。

等到史丹凤和无心敬到了他这一桌,他端起酒杯碰了碰嘴唇,看他姐的面孔如同调色板一样,假睫毛也如同帽檐,脑袋更似大钻头,仿佛一个倒立就能扎到地里去了。

一句恭喜的话也没说出来,他低头对着手里的大灰雀,重重的叹了一口气。

宴席散后,史高飞跟着新婚夫妇回了新房。史丹凤钻进浴室对自己痛加涤荡,直洗了两个多小时,才洗出了自己的本来面目。擦着头发走进客厅,她疲惫的说道:``我说我要自己化妆,妈偏不同意。结果可好,把我画成妖怪了。''

无心长长的躺在沙发上,对着史丹凤呻吟了一声,他颤悠悠的伸出了一只手,细着嗓子唤道:``姐\ldots{}\ldots{}''

史丹凤不为所动的说道:``别装了,哪有那么累!''

无心讪讪的放下了手,扭脸去对史高飞说:``爸,去切个西瓜,要冰镇的。''

史高飞本是席地而坐在看电视,听了儿子的命令,他打了个哈欠起了立。而在他往厨房走时,房内的手机响了。史丹凤走去接了电话一听,对方却是白大千。

她以为白大千是来向自己道喜的,然而白大千却是劈头问道:``丹凤,你弟弟呢?他怎么不开机?''

史丹凤懒洋洋的坐在了无心身边:``他那破手机可能又没电了,白大师,你找他有事?''

白大千答道:``对,有事,他在你身边吗?''

史丹凤去厨房把手机给了史高飞,然后拧开水龙头,冲洗一只绿皮大西瓜。在哗啦啦的水声之中,她忽然听到手机听筒中传出一声怒吼:``史高飞!你是畜生!我操你娘!''

史丹凤立刻关了水龙头,听白大千把嗓子都喊劈了,而史高飞握着手机呆站在厨房门口,却是愣怔怔的一言不发。伸手夺过手机,她开口问道:``白大师,怎么了?有话慢慢说。''

白大千在电话里做了个嘶哑的深呼吸,然后答道:``丹凤,史高飞把佳琪给欺负了!他个禽兽不如的东西,把佳琪给欺负了啊''

史丹凤一听``欺负''二字,顿时有了知觉:``他——他和佳琪?佳琪说的?''

白大千忍无可忍似的又怒吼了:``佳琪会说个屁!佳琪懂什么?''

话音落下,他勉强压了压脾气,开始讲起了来龙去脉。原来他对佳琪是常年的当爸又当妈,又因为佳琪着实是智商偏低,所以他对女儿照顾得格外细心,连佳琪每个月要用的卫生巾都是他亲自预备,约摸着时间快到了,他便提前买好放到卫生间里。

然而上个月,他发现卫生巾始终是没开封。

他知道佳琪不会诉苦,生了病也不知道告诉人,于是带着佳琪去了趟医院——他以为女儿至多是内分泌失调,然而几个项目查完了,他得知佳琪已经有了将近两个月的身孕。

脑子里当场``嗡''的轰鸣了一声,他险些晕倒在了医院的妇科诊室里。强定心神带着佳琪回了家,他把门一关,开始逼问女儿``他''是谁。

佳琪害怕了,不是因为怀孕而怕,她是被白大千的凶恶神情吓住了。脑筋随之停了转,她怯生生的靠墙站着,白大千越是咆哮着问,她越是嚎啕大哭的说不出。白大千急了眼,抄起一把塑料刷子,把她狠揍了一顿。

最后,

他终于从佳琪的嘴里揍出了答案。佳琪莫名其妙的挨了顿打,坐在地上哇哇的哭,嘴咧得像瓢似的,眼泪顺着脖子淌。白大千看着佳琪,又想起了佳琪她妈,心里一下子就苦死了——他只是一眼没照顾到,他忙着赚钱,真的只是一眼没照顾到。

这事他没法对别人说,史高飞又是个疯疯癫癫的,于是他只能反复的问史丹凤:``我怎么办?你说我该怎么办?''

史丹凤也不知道该怎么办,但因自家失身的是弟弟不是妹妹,所以她比白大千要轻松一些:``白大师你别急,这件事不能就这么算了,我必定要给你一个答复。我们在一起相处得好像一家人一样,小飞要是敢胡闹,我也不会允许的。''

三言两语的先安抚了白大千,她挂断电话转向了史高飞,横眉怒目的问道:``小飞,你把佳琪怎么了?''

史高飞骤然抬手抱住了脑袋,歇斯底里的喊道:``我怎么知道电脑里会有A片?不让佳琪看,佳琪非得看!好,我错啦,我错了行了吧?!''

史丹凤嗤之以鼻:``少跟我装马景涛!我告诉你啊,你跟我叫是屁用没有。反正佳琪已经怀孕了,你身为男子汉,不能不负责。''

话音落下,她又暗地里偷偷一咧嘴,因为片子是她下的——全怪无心非得求她``下个崽'',结果下完之后,她和无心全忘了看。

史高飞侧身一靠门框,一脸茫然的问道:``我怎么负责?''

史丹凤想了一想,忽然一拍巴掌:``小飞,要不然\ldots{}\ldots{}你把佳琪娶了吧。你不总说她像李嘉欣吗?''

史高飞摇了摇头:``林嘉欣。还有,我不想和地球人结婚。''

史丹凤严肃了面容,正色问道:``和地球人睡觉的时候,你怎么没想过会有今天呢?''

她转身走到料理台前,拿起菜刀切西瓜:``我不和你说了,我去和爸说。你也别装委屈了,你有什么可委屈的?''

史高飞缓缓的向外扭头,看到无心试试探探的走来了。天气热,无心已经脱了外面衣裤,只留了一条大红色的三角裤衩。精赤雪白的停在厨房门口,他先背过手挠了挠屁股,然后向史高飞一探头:``爸?''

史高飞歪着脑袋垂下眼帘,喃喃的说道:``我不是委屈,我是后悔。我竟然和地球人□了。''

史丹凤把一大盘西瓜端给了无心:``无心你别理他,我看他是得便宜卖乖。我要是白大千,我都舍不得把佳琪嫁给他。''

无心把西瓜端到了客厅茶几上,然后回来去拽史高飞:``爸,吃西瓜了。''

拽了第一下,没拽动,第二下拽动了,无心把史高飞牵进客厅,一边牵又一边回头,对着史丹凤做了个鬼脸。

史丹凤给史一彪打了电话,如实讲述了弟弟的罪行。史高飞坐在沙发上听着,始终是垂着脑袋一言不发。入夜之后,无心说要和史丹凤同房睡觉,他``嗯''了一声,也不反对。直挺挺的躺在沙发上,他望着天花板不闭眼——其实也不是嫌佳琪不好,佳琪没什么不好的,可她毕竟是个地球人。他怎么能和地球人结婚?怎么能给小儿子找个地球人后妈?

至于佳琪的智商问题,他倒是完全没往心里放。

心事重重的翻了几个身,他睡不着觉。卧室里面隐隐响起了扑通扑通的声音,是床垫子在上下起伏。史高飞听而不闻,继续思索着自己的心事。

史丹凤本来打算在火星镇住满一个礼拜,然而在新婚第二天,史一彪和赵秀芬便难得的同行而来了。

他们不敢招惹儿子,所以躲在卧室里偷偷的问女儿。问过一场之后,赵秀芬点了点头:``哦\ldots{}\ldots{}姑娘是有点儿傻。你说她长得挺好?''

史丹凤点了头:``不是我说的,是小飞自己说的。小飞总说她像个什么明星——李嘉欣还是林嘉欣来着?''

史一彪又问:``生活能自理吗?''

史丹凤笑了:``洗衣服做饭全能,也会逛市场买东西。但你要是让她上学念书,她肯定不行。''

史一彪和赵秀芬对视一眼:``要不然,我们去亲眼看看吧!''

史一彪说到做到,当天下午便启了程。顺便带走了女儿女婿儿子以及糟糠之妻。从火星镇到江口市,大概是三百多里的路程,当晚天黑之后他们到了市郊写字楼,史丹凤率先下车往楼内走,正好遇上了三楼一家公司里的女职员。抬手向着楼上一指,她小声问道:``小张,我们公司关灯锁门了吗?''

张小姐答道:``没呢,你们欧吉桑现在天天晚上在公司里熬夜。''

史丹凤得了消息,便把父母带上了三楼,免得当着佳琪的面讨价还价,会伤了佳琪的心。到了公司门口,她让无心和史高飞留在外面,自己带着父母推门进了办公室:``白大师,是我,我回来了。''

她一边说一边绕过了屏风。而白大千本是坐在写字台后看书,此刻闻声抬起头,他扶了扶黑框眼镜,垂下的半长头发居然夹杂了几丝白色。

史一彪和赵秀芬睁大了眼睛,一起被白大千的沧桑风采震慑住了。

\chapter{双喜临门}

白大千已经过了最愤怒的气头,又连着熬了几天的夜,精气神不足,所以在见到史一彪夫妇之后,并没有奔突咆哮。让史丹凤去凑了三把椅子,他请史一彪和赵秀芬坐了,然后自己把写字台后的沙发椅拽到两人对面,他一边落座,一边沉重的叹了口气。

史一彪身为火星镇的首席土豪,本来是轻易不把人往眼里放的,然而此刻面对着忧郁落寞的白大千,他紧了紧一身的肥肉,竟是不由自主的加了小心。赵秀芬坐在史一彪身边,则是下意识的抬手摸了摸脸和头发——准亲家太帅了,导致她有一点自惭形秽。

史丹凤走到了白大千身边,弯下腰低声告诉他道:``小飞也回来了,他不懂什么,只会添乱,所以我没让他进办公室。无心在外面陪着他呢。''

话音落下,没等白大千表态,史一彪紧跟着开了口:``兄弟,有话就让咱们做长辈的说吧,我那儿子你应该也了解,说实在的,凭着他的所作所为,我今天都没脸来见你。可是为人父母的,还不能真不管他。''随即他开始大打悲情牌:``唉,不瞒你说,我们两口子都要被他折磨死了。你看我媳妇,刚五十出头,都老成什么×样了?还不都是为他愁的?''

此言一出,赵秀芬当场沉了脸,心想也就你个狼心狗肺的看我老,其实我老个屁啊!说我老,也不照镜子看看你那肥德行!

史丹凤本想坐下旁听,可是听了个开头之后,她感觉父亲虽然语言粗俗,但是姿态挺低,不至于得罪了白大千,于是便起身绕过屏风,推开大玻璃门去看无心和史高飞。见史高飞在走廊里靠墙站得挺稳当,她放了心,转身又回去了。

史一彪还在心平气和的侃侃而谈,每一句话都说得特别在理,语重心长的先道歉后负责,一边贬低自家儿子,一边抬高白家女儿,末了他起了身,一定要上楼去瞧瞧佳琪。

而在另一方面,白大千其实根本没想和史家结亲——他早下了供养女儿一辈子的决心,根本不需要再招一个精神病女婿。只是女儿肚子里的孩子不好处理,白大千真不忍心带女儿去做人工流产,但是更不忍心把女儿嫁给史高飞。

史一彪看出了白大千的犹豫,于是开始不动声色的展望未来,话里话外的又亮家底又许大愿。白大千听着听着,听出了兴趣,心想莫非姓史的小子还是个富二代?

一行人等上了楼,进门之时正赶上佳琪坐在客厅里看电视。眼见家里骤然来了陌生客人,她怯生生的站起了身,还是运动服马尾辫的造型。史一彪热情活泼的先开了口:``佳琪看电视哪?''

佳琪看他胖得有趣,不由得笑了一下:``嗯。''

赵秀芬落后了一步,两只眼睛紧盯着佳琪看。依着她的审美,她感觉佳琪的模样其实比自家女儿更好——自家女儿细胳膊长腿的,不是个富贵的体格;而佳琪偏于白胖,胳膊腿儿都浑圆有肉,眼睛明亮,头发厚密,一笑还有俩酒窝。

史一彪进入客厅,自来熟的继续和佳琪对话。佳琪虽然不认识他,但是看他笑眯眯的像尊大弥勒佛,故而也就不很怕生,他有问,她就有答。赵秀芬留着心眼侧耳倾听,发现佳琪的言谈举止都带着小女孩气,但是有条有理清清楚楚,并非胡言乱语。心里暗暗的有了计较,她转身一拽史高飞,又向佳琪的方向使了个眼色。

史高飞磨磨蹭蹭的走上前了,不情不愿的说道:``佳琪,我给你带了礼物。''

佳琪坐在沙发上,飞快的仰脸看了他一眼,然后撅着嘴低了头,小声嘀咕道:``哥哥,爸爸不让我和你说话了。''

史高飞想了想,忽然有些不耐烦:``其实也不是什么好礼物,是我们镇特产的芝麻糖,不大好吃,也不值钱。''

佳琪垂头揪着袖口上脱出的线头,略略的有一点委屈,因为前几天挨了父亲的打,而且怀孕了。怀孕之后就要生小孩,这一点她是知道的,但知道归知道,知道而已,并不能领会吸收。她不敢再和父亲提自己怀孕的事,要提只能和史高飞提,也并不是要和史高飞算总账,只是毕竟挨了一顿打,她想找个对象诉诉苦。

史一彪坐到了佳琪身边,坐得半面沙发向下一陷。对于佳琪,他是特别的和蔼可亲,一是为了向白大千示好,二是想要考察一下佳琪到底傻到了什么程度。

满面春风的看着佳琪,他先是表示惭愧,说火星镇太小,没有什么像样的好土产可以往外带,紧接着话锋一转,他从火星镇的``小'',说到了江口市的``大''。城市这么大,人们上下班可就不大容易,尤其是城郊一带没有高档社区,将来小飞和佳琪结了婚,必定得搬到市区里住。他们两个是闲人,倒也罢了,但白老弟早晚来回太不方便,终究不是长久之计,所以买车的事情不能耽搁,须得和买房同时进行——买房子要买大的,白老弟是做学问的人,将来家里添了孩子,吱哇乱叫的吵到白老弟怎么办?自家儿子不争气,做父母的免不得就要多操心,等到将来孩子出世了,可以让做婆婆的过来伺候月子兼看孩子做饭。婆婆忙不过来,再雇个保姆也就够了。

白大千听到这里,有些傻眼,心想史高飞一副欠揍的熊样,史高飞的爹却是如此豪爽。这么痛快的爹养出那么糟糕的儿子,真是可惜了史一彪这个人。

史一彪看出佳琪傻得有限,而且白大千不是很想把女儿嫁给自家的儿子。为了巩固儿子的胜利成果,他决定采取金元外交的政策——二三四五奶始终是没有为他再生出个一男半女,史高飞是他唯一的儿子。唯一的儿子要结婚了,做父亲的还能在钱上打小算盘吗?

夜深之时,他和赵秀芬告辞离去,明天再来继续商量婚事。白大千对他挽留不住,而他临走之前去了趟卫生间,出来之后往佳琪手中塞了张信用卡:``叔叔今天来得太匆忙了,什么都没给你带。明天让你哥哥陪你出去玩,想要什么自己买。你哥哥要是不听话了,你告诉叔叔,叔叔替你教训他。''

白大千见状,立刻上前阻拦。哪知史一彪匆匆穿鞋,随即身形一晃,把自己硬甩出了门;白大千拿着卡再去追赵秀芬。赵秀芬也立刻做了撤退。白大千万没想到史家夫妇全都具有移形换影之术,居然说走就走了个无影无踪。

白大千一时没了主意,史高飞生出了负罪感,也悄悄的回了卧室。房内的人各就各位休息了,楼下汽车发动起来,史一彪和赵秀芬却是还有精神。赵秀芬感觉丈夫有些过于大方了,史一彪却是不以为然:``你懂个屁!我得先哄着他们白家,让佳琪把孙子给我生下来!''

怀揣着这个指导方针,史一彪向白大千发动了进攻。白大千前半生一直是时运不济命途多舛,除了穷困就是潦倒,如今虽然赚到钱了,但一身的穷气还未褪尽,见史一彪挥金如土,便神昏目眩的把持不住,看史高飞也顺眼多了。又因为佳琪实在是喜欢史高飞,挨了胖揍之后还敢偷偷摸摸的向他搭讪,所以白大千一咬牙一狠心,同意了这桩婚事。

史丹凤作为旁观者,本以为父母对自己的婚姻大事已经够卖力气,还在暗暗的受宠若惊,哪知如今见了父母对待弟弟的卖命劲头,才意识到自己的地位根本没有上升。她结婚时,史一彪只给了她十万块钱做陪嫁,除此之外再无其它;轮到史高飞结婚了,史一彪不但在市区给他买了价值上百万的精装新房,房子楼下的车库里还停了一辆代步用的帕萨特,虽然弟弟根本不会开车。

婚礼依旧是在火星镇举行,然而佳琪的化妆师是从江口市请来的,婚纱礼服全是订制的,迎亲车队的规模也比她那时候大了三倍。婚礼这一天,佳琪被化妆师装扮得千娇百媚,脸盘也小了,眼睛也大了,本来史高飞一口咬定她像林嘉欣,可是在典礼这天,年轻的宾客们一致认定她更像韩佳人。佳琪事先受了白大千的嘱咐,在典礼上一言不发的只是笑。她不说话,旁人以为她是新娘子脸嫩,也看不出她的异常。

及至婚礼结束,史家的一双儿女算是一起出了名——一个老姑娘,一个精神病,居然一个嫁了小帅哥,一个娶了小美女,不由得要让人感叹金钱的力量。

在毫无知觉的情况下,史丹凤和史高飞成了县里的励志姐和励志哥。

婚礼结束之后,众人各归其位。白大千没时间去驾校学开车,所以只好还是住在公司楼上的出租屋里。史丹凤想和无心在公司附近租一套小房子,但是史高飞又不同意了。

``我们一起住!''他对史丹凤说:``你放心,佳琪在结婚前已经向我保证过了,她不会欺负宝宝的,将来她生了小孩子了,也不可以偏心。''

史丹凤正要回答,可话未出口,无心却是对她一挤眼睛。

史丹凤闭了嘴,在无心的授意下,她当晚回了公司去住。无心留在史高飞的新房子里,先是趴在大床上玩手机,调出了小游戏让白琉璃打飞机。等白琉璃把手机屏幕的保护膜啄出一片坑了,他扔了手机坐起身,对着正在看电视的史高飞和佳琪说道:``爸,我明天想去和姐住。''

史高飞惊讶的回头看他:``为什么?当初说好只是借给她的,怎么,她还借起没完了?''

无心溜下大床,走到两人面前站住。先从佳琪手中的小果盘里拿了一颗小番茄扔进嘴里,他随即说道:``我想和姐睡觉,就像你和佳琪睡觉一样。''

史高飞登时起了身:``开什么玩笑?你才刚满一周岁啊!''

无心在他面前晃来晃去:``我不管,我已经和姐睡过了,以后还想睡。你不同意,我就生气了。''

史高飞伸手一指他的鼻尖:``宝宝,你这么不听话,是不是欠揍了?''

无心当即往地上一坐:``你敢打我,我就满地打滚给你看!''

史高飞俯身想要去把他拽起来:``不行!你跟爸爸和佳琪一起睡!''

无心的动作极快,不等他触碰自己,已经抱着脑袋滚向了门口——依他的原意,只是吓一吓史高飞而已,哪知地板光滑,他的力道失了控制。史高飞眼前一花,发现他已经滚没了影。紧接着前方黑暗中响起一声闷响,无心哀哀的发出了声音:``爸,我的头撞到沙发腿了。''

史高飞摸黑跑了过去,摸摸索索的在沙发前找到了儿子。无心捂着头上痛处坐起了身,凑到他耳边小声说道:``爸,我和地球人不一样,我长得很快,已经长大了。我\ldots{}\ldots{}我要\ldots{}\ldots{}我想\ldots{}\ldots{}爸你知道我的意思吧?''

史高飞蹲在黑暗之中,心里为难极了。起身坐上了沙发,他把无心也拉扯到了自己的大腿上。从上到下摸着儿子的长胳膊,他低声说道:``我知道白大千为什么在婚礼那天掉眼泪了。''

无心给了他一个侧影:``爸,我结婚那天,你也哭了。''

史高飞抓起了无心的一只手,从手腕慢慢的捏到手掌,再从手掌一根一根的捏过手指。末了把这只手送到嘴边轻轻的咬了咬,他仰头望着无心又问:``那你以后不要爸爸啦?''

无心在他大腿上转了个身:``我只是和姐一起睡觉而已,你要是想我了,我就还回家来。白天你也不要闲在家里,你到公司和我一起做事赚钱。我要赚钱养姐,你也得赚钱养佳琪和小孩子啊!''

史高飞摇了摇头,小声说道:``我可以养佳琪,不过我不想养小孩子。那小孩子的血统不纯粹了,一定会长成一个彻头彻尾的地球人。宝宝,爸爸只要你一个就够了。''

无心不再说话了,只是向史高飞的怀里一靠。

佳琪穿着拖鞋,噼里啪啦的走出来了,将一只手机递给了史高飞:``有短信。''

史高飞一手搂着无心的腰,一手打开手机短信。无心和他一起低了头,只见屏幕上面写着清清楚楚的一行字:``你是外星人吗?我是蛇精。''

史高飞本是悲悲戚戚的在发牢骚,此刻见了这条短信,却是当即乐得``哈''了一声。一把将腿上的无心推到一旁,他开始认认真真的回短信。从这开始,无论是无心和他说话,还是佳琪让他吃小番茄,他都一概不理睬了。

史高飞发了半宿的短信,无心在客房陪白琉璃玩了半宿的手机游戏。白琉璃现在从早到晚的打飞机,技艺已臻化境,只要游戏音乐一响,他便对准屏幕,发了疯似的狂啄三分钟,啄得手机屏幕一脸麻子。

无心总觉得自己对不起他,所以由着他玩。长条条的趴在客房床上,他一只手扶着手机,一只手垫在头下当枕头。昏昏欲睡的闭了眼睛,他忍无可忍的打了个打哈欠,顺便对着白琉璃展示了自己的嗓子眼。

``白琉璃,不要走了。现在猫头鹰也不知跑到哪里去了,你一个人在山里多寂寞,不如跟着我混,看我现在混得多好。''他如是说道。

白琉璃啄中了最后一只小飞机后,离开鸟身现了形:``你什么时候回家?''

无心恍恍惚惚的答道:``总得再过五六十年吧?我得给爸和姐养老送终。白琉璃,不要走了,修炼算什么要紧的事?趁着我在外面,你也跟着我开开眼界吧。明天我带你去乡下捉鬼,我去把鬼打散,你去吃掉魂魄。好不好?''

白琉璃向下沉,一直沉到了被褥表面。依着他的意思,他是想要尽早回家的,因为对于鬼神精怪来讲,他的家是一块风水宝地。不过想想地堡里的那种寂寞,也的确是有些难熬。

无心睡眼朦胧,挣扎着重新启动了游戏,然后含含糊糊的说道:``我多找些鬼魂给你吃,不也是一样的?和我在一起,多快乐啊!''

然后他从鼻孔里呼出两道气流,彻底睡着了。白琉璃则是慌里慌张的附回鸟身,对着屏幕又啄了一气。

翌日清晨,无心早早起床,吃了佳琪预备的热馒头之后,他匆匆下楼去赶公共汽车去郊外写字楼。站在公共汽车站旁,他忽然看到路边的绿化带里藏着一只非常小的猫崽子。灵机一动走了过去,他对着怀里的白琉璃问道:``你想不想做猫?鸟太小了,我真怕夜里翻身时会把你压扁,你不如改做一只小猫,或者小狗。''

胸口凉了一下,是白琉璃现了身:``不好,这猫很丑。''

无心想了想,紧接着起了身,也不等车了,沿着步行道往前走——他记得前方有一家小小的兽医院,兼给宠物拉皮条以及代售小崽子。兽医院还没有开始营业,但是已经开了门,一个年轻的小伙子正在里里外外的扫地。无心进去转了一圈,再出来时手里多了东西,正是一只小小的虎斑纹猫。

白琉璃不肯去上小猫的身,因为小猫没有尖嘴,不能打飞机。无心饶有耐心的哄了他半个小时,正是口干舌燥之际,迎面却是遇上了白大千。

白大千刚下公共汽车,一路走得飘飘然,是个意气风发的模样。对着无心一点头,他开口问道:``佳琪和小飞怎么样?''

无心答道:``挺好!''

白大千又问:``从哪儿弄了一只猫?不会是买的吧?养狗多好玩,养猫有什么意思?你到办公室等我吧,我上午去趟金光寺,中午回家,下午我们一起下乡。''

无心连连答应,然后两人分道扬镳。无心继续摆弄他的猫和鸟,白大千也在路边坐上一辆出租车,直奔了金光寺。婚礼前夕,汇丰曾经派徒弟给佳琪送了一份小礼物,是白玉的小挂饰,一尊观音一尊佛,有道是``男戴观音女戴佛'',正好把小两口全照顾到了。

平心而论,东西不算昂贵,无非是一点心意。白大千若是处在往昔落魄的时候,收就收了,不会道谢。然而如今他也算是小小的发达了,不禁一身皮肉做痒,跃跃欲试的想要跑到汇丰面前自吹自赞一番。

大清早的,商场尚未营业,他又不想空手登门。在金光寺外的一家花店里,他买了一大束火百合和马蹄莲。捧着这么一大束热热闹闹的鲜花进了寺门,他洋洋得意的,还感觉自己这礼物挺高雅。

汇丰前一阵子去台湾访问了,昨天晚上刚回了来,夜里没睡好,如今又要强打精神接待冤家弟弟,不由得就憋了一肚子起床气。白大千还未开口,他坐在一把硬木椅子上,已经横眉怒目的犯了嗔戒。白大千看了他这副尊容,登时有了饱腹之感,先前预备的妙语也是一句都说不出来了。扬着大脸站在屋子中央,他吊儿郎当的说道:``告诉你一声,佳琪已经结完婚了,多谢你送的那对小玩意儿。''

汇丰横了他一眼,又做了个深呼吸,是强忍着不咆哮的模样。

白大千把手里的鲜花往他怀里一搡,然后爱答不理的说道:``走了,再会。''

房门一开一关,他是真走了。汇丰大师因为在台湾住久了,对岸文化的余波还在他的心灵中荡漾,故而此刻拿起鲜花往旁边桌上一摔,他气急败坏的嘀咕道:``阿弥陀佛,真是有够讨厌的!''

白大千离了金光寺,顺路又去看了女儿。家里只有佳琪一人在家,白大千问道:``小飞呢?''

佳琪现在变得很馋,腮帮子一鼓一鼓的总是在大嚼:``他说他要去火车站接朋友。''

白大千一愣:``他那样的还有朋友?谁啊?''

佳琪摇了摇头:``我不认识,好像是蜥蜴星人。''

\chapter{蜥蜴来访}

史高飞在清晨时分就抵达了火车站,可是直到下午才等到了他的好朋友蜥蜴星人。大蜥蜴化为人形,混在一群十七八岁的小民工里,背着个长包挎着个扁包,手里还拎着一只小塑料袋,塑料袋里放着一根黄瓜。虽然他和史高飞不是一个星球的同胞,但史高飞总觉得他和自己同命相怜,同为外星遗孤,应该联合起来在地球上做一番大事。手里举着白大千给佳琪买的小花伞,他人高马大的站在出站口外的人海中,热情洋溢的和大蜥蜴握了握手:``你们蜥蜴星球的时间和地球时间不同步,我等了你六个多小时,中间只喝了一瓶矿泉水,快要热死了!''

大蜥蜴常年在阴暗潮湿的岩洞里过日子,如今站在烈日骄阳下,一双灰眼睛不禁有些要睁不开。先掏出一副太阳眼镜戴好了,大蜥蜴躲在镜片后面审视史高飞,同时回忆往事,认定自己在短信里把话说得很明白,绝不至于让史高飞误会到提前六个小时来接站。但他作为一只善解人意的蜥蜴,认为无论如何史高飞的确是吃了苦头,自己理所当然的应该表示愧疚与同情。把手里的塑料袋打开,他把黄瓜拿出来递向了史高飞:``要不要吃?洗干净的。''

史高飞其实不大爱吃黄瓜,但是此刻饥不择食。接过黄瓜咬了一口,他一边咔嚓咔嚓的咀嚼,一边把大蜥蜴带上出租车,直奔了自己的新家。

不出片刻的工夫,他们在小区门外下了车。史高飞夹着小花伞,把大蜥蜴一路带到了自己家中。在他们进门之时,佳琪正坐在客厅里吹着空调吃西瓜。忽见史高飞把朋友带回来了,她捧着一瓣西瓜站到门口,很好奇的上下打量大蜥蜴。史高飞随手关了房门,同时开口说道:``佳琪,他就是我说的蜥蜴星人,你看他现在这样子挺帅吧?变成大蜥蜴之后更帅!嘴那么长,牙那么大!''

大蜥蜴有好几年不曾到外人家中做客了,此刻颇为拘谨的在门口放下了自己的长扁两包,他从裤兜里掏出一只皮夹,又从皮夹中抽出了一张身份证:``外星人,为了便于在人间活动,我在上次人口普查的时候,设法上了户口办了身份证,所以,请叫我的人类名字吧!''

史高飞接过身份证一瞧,随即仰天长笑:``哈哈哈,这不还是蜥蜴吗?''

佳琪听史高飞笑得热闹,好奇的也跟着伸头去看。她几乎是连小学都没上完,然而大蜥蜴的名字太简单了,连她都能够流畅的读出:``易——西——''

大蜥蜴不为史高飞的笑声所动,正色解释道:``这个名字的高妙之处在于它既简明易读,又暗示了我的身份。''

此言一出,效果等于对牛弹琴。史高飞依旧是笑,佳琪也依旧是埋头啃西瓜。等到史高飞笑够了,佳琪的西瓜也吃完了,大蜥蜴才重获关注,得以受邀进入客厅。大蜥蜴赤脚穿着凉鞋,从南到北奔波了几千里,一双脚自然洁净不到哪里去。脱了凉鞋之后,他很自觉的想要贴着墙边溜向沙发,尽量不在史家锃亮的地板上留下脚印。然而史高飞并不能够体谅他的苦心,追着他大声嚷道:``哇!蜥蜴,你脚好臭!''

不等大蜥蜴回答,他转身又对佳琪说道:``看来蜥蜴星人和我不一样,他们比较臭。你看我就不臭,宝宝也不臭。''

佳琪又端起了一大瓣西瓜,边吃边答:``你有时候也臭,宝宝从来都不臭。对了,宝宝上次放了个大屁,不臭,但是好响!''

大蜥蜴停在半路,靠墙站着,距离门口和沙发都还有着一段距离,前进也不对后退也不对,耳中只听史高飞支使佳琪道:``你去端盆水来,给蜥蜴洗爪子!''

佳琪站在茶几前一扭肩膀:``我还没吃完呢!''

史高飞把小花伞往门旁一放,换了拖鞋往里走:``你就知道吃!''

大蜥蜴像条落网之鱼一样,屁股坐在了史家的大沙发上,赤脚踩进了史家的大水盆里。温水是史高飞给他端来的,不但端了水,还附带了一只塑料大刷子,因为看楼下邻居给家里的金毛狗洗澡时,一定会用大刷子将狗从头到尾的刷一遍。而他对大蜥蜴特别有好感,所以如果大蜥蜴需要的话,他愿意亲自出手,把大蜥蜴也刷一刷。

大蜥蜴,作为蜥蜴,个子已经是相当的大,但作为人类,他只是中等身材,在史高飞的衬托之下,越发小了一号。史高飞抓过他一只手,将一只大水蜜桃放到他的手心里,随即问道:``你这回下山,打算在外面生活多久?''

大蜥蜴托着水蜜桃收回手,神情有些茫然:``不一定,因为我已经没有家了。''

史高飞立刻开动了脑筋:``你的洞被地球人抢走了?''

大蜥蜴垂下头,望着白里透红的水蜜桃叹了一口气:``说起来,还是你们上次留下了后遗症,搞得我现在无家可归了。''

大蜥蜴对着水蜜桃长篇大论,史高飞侧耳倾听,倒也明白了十之八九。原来自从他们离开岩洞回归县城之后,大蜥蜴便开始打扫家园,想要把自己先前的好环境恢复起来。不料正是在他忙碌之时,躺在岩洞最深处的未婚妻却是骤然复活了。

说是复活,其实只是借尸还魂,而且还的还不是全魂——洞子里除了有些平常的小鬼在流连游荡之外,凭空又多了几股子零碎魂魄四处乱窜。大蜥蜴毕竟是成了精的,虽然不是很通阴阳之术,然而单是靠着直觉,他也能觉出那些零碎魂魄在拼命的往一起凑,似乎还想凑成一个完整的灵魂。一波接一波的怨气在洞内来回激荡,催动着那几道魂魄横冲直撞,末了,他的未婚妻作为洞中唯一一具囫囵尸首,别无选择的成了牺牲品。

几道魂魄在未婚妻的体内凑成了一个残缺的灵魂,没有思想,只有杀气,指挥着未婚妻东倒西歪的去抓大蜥蜴。凭着大蜥蜴的力量,满可以用牙齿和爪子把未婚妻直接撕碎,然而对着未婚妻,他下不了手。

他没了办法,只好开始逃。一步跳上大蚂蝗的背,他想要顺着暗河往外走。可是没等大蚂蝗开始游动,他的未婚妻已经鬼魅一般的追了上来。大蜥蜴躲无可躲,忽见浅水之中漂着一张纸符,上面依稀画着降妖除魔的图案。走投无路的一弯腰捞出纸符,他把水淋淋的纸符拍到了未婚妻的面孔上。

未婚妻的动作立时变得沉滞了,大蚂蝗也顺着水流启了程。大蜥蜴一边和未婚妻搏斗,一边留意到岸边石壁上隔三差五便贴着纸符——有些纸符是端正的,有些则是东倒西歪,还有些干脆漂到了水里。

他不知道那是大蚂蝗在被无心催吐了一场之后,难受得在暗河之中翻江倒海,无意中破坏了丁思汉布下的阵法。手忙脚乱的爬上岩壁,他收集了一大团纸符,尽数扔到了未婚妻的身上。阵法虽然破了,但纸符本身依然具有一点辟邪的法力。附在未婚妻身上的几缕残魂被他强行驱逐了出去,而未婚妻面目狰狞的瘫在大蚂蝗背上,暂时不动了。

大蜥蜴不知道是什么灵魂会如此顽强,散都散了,还能作乱。为了不让未婚妻再受操纵,他上山挖了个坑,让未婚妻入土为安了。而洞中既然没了他所爱的人,也就不成了家。把他的宠物大蚂蝗留在洞里,他带了自己所有的家当——两套衣裤和一把吉他——下山流浪去了。

在有家的时候,他通常只在山下的城镇里打零工赚小钱;现在没了家,他便可以无牵无挂的往远走了。坐在开往北京的列车硬座车厢里,他百无聊赖的想起了史高飞。试试探探的发出一条短信,他没想到对方立刻热情洋溢的有了回应。

他并不是自来熟的人,所以很谨慎的和史高飞聊了小半夜,在确定对方并非虚情假意之后,才在列车到站之后直接买票,应邀赶来了江口市。

史高飞对于大蜥蜴的情伤不感兴趣,自顾自的问道:``你有什么计划吗?如果没有的话,就和我联手做一番大事业吧!''

大蜥蜴在史高飞的注视之下,心里七上八下的,忽然有种上了贼船的感觉:``其实我只想找份工作填饱肚皮\ldots{}\ldots{}''

未等他把话说完,史高飞抢着又问:``蜥蜴,你说说你都有什么本领?我会跟着我儿子去捉鬼,你会什么?''

大蜥蜴终于敌不过水蜜桃的诱惑,低头咬了一大口。三嚼两嚼的吞咽了,桃子汁溅上了他的脸,伸出长舌头一卷鼻尖,他很客观的答道:``我会弹吉他,刷墙漆,修电脑,抻面条,还在工地食堂里做过大锅饭。''

史高飞这么一听,感觉大蜥蜴的特长仿佛全和``事业''二字扯不上关系。弯腰抬手托了下巴,他有些失望,心想除了自己的儿子之外,其余的外星人在地球上全都堕落了。

因为史高飞不再说话,所以大蜥蜴默默的吃了许多水蜜桃。

当天晚上,他留宿在了史家。客房里的空调坏了,夜里热得让人躺不住,于是大蜥蜴在客厅沙发上安了身。连着坐了几天几夜的硬座,他也累极了,一觉睡过去,竟是在不知不觉中现了原形。史高飞半夜起床出去撒尿,回来之后强行叫醒了佳琪,鬼鬼祟祟的带她出去看蜥蜴。大蜥蜴只穿着一条花布大裤衩,睡得仰面朝天张着大嘴,分叉舌头软绵绵的耷拉在了嘴角外。

史高飞打开了一只小手电筒,一边从头到脚的照着大蜥蜴,一边小声问佳琪:``你怕不怕?反正我是不怕。''

佳琪愣怔怔的看了半天,末了答道:``壁虎嘛,怕什么?我小时候和爸爸住老房子,老房子里总有壁虎。''随即她弯腰抓住了大蜥蜴甩在一边的大尾巴:``你把他的尾巴掐掉,他很快还能再长一条。''

史高飞把手电筒交给佳琪,然后弯腰开始去掐大蜥蜴的尾巴,掐了又掐,始终是掐不断。大蜥蜴睡得太沉了,在梦里被他掐得痛不欲生,然而硬是醒不过来。最后还是佳琪阻拦了史高飞:``你别掐了,尾巴断了一定很疼。''

史高飞松了手:``好吧,那我就不掐了。不过佳琪你看,有条大尾巴也挺好玩的,只是穿裤子不方便,幸好他白天还能变成人,要不然就得在裤子后面开个洞,但是又很容易露出□和蛋。''

佳琪基本同意史高飞的一切高论,狗腿子似的连连点头。哪知史高飞一转念,忽然又有了新想法:``不对,蜥蜴屁股和人屁股是不一样的!''

鬼似的缓缓伸出两只手,史高飞自作主张的扒掉了大蜥蜴的花裤衩。佳琪举着小手电筒低头研究了半天,最后抬头告诉史高飞:``什么也没有。''

史高飞跟着看了看,的确是没分出大蜥蜴的公母。抱着肩膀在客厅的冷气中打了个寒颤,他和佳琪心满意足的回卧室睡觉去了。

如此过了一夜,到了翌日上午,白大千又来了。

他给女儿带来了许多新鲜水果,又嘱咐她不许再吃垃圾食品。因见女婿依然是无影无踪,他十分不满的问道:``小飞现在既不去公司上班,也不在家照顾你——他每天到底是在忙什么?''

佳琪坦然的答道:``哥哥和蜥蜴星人出门找工作去了。''

白大千莫名其妙:``什么乱七八糟的?蜥蜴星人到底是谁?''

佳琪想了想,感觉自己说得挺明白:``就是一只大蜥蜴啊!''

白大千开动脑筋思索了一番,隐隐的明白了些许:``哦\ldots{}\ldots{}小飞养了一只蜥蜴?''

佳琪摇头笑道:``小飞没养,蜥蜴自己什么都能做,早上还给我们煮了皮蛋瘦肉粥。''

白大千听到这里,感觉女儿已经和女婿一起疯了,自己也有要疯的趋势:``蜥蜴还能煮皮蛋瘦肉粥?!''

佳琪理直气壮的点头:``能啊!小飞不让他煮,他非煮。他说他不好意思在我家白吃白睡,还说过几天要找房子自己住。''

白大千抬手抓了抓半长的头发:``蜥蜴还能说话?!''

佳琪猛的一拍巴掌,自作主张的改了话题:``爸!你记得去告诉宝宝和姐姐,哥哥说让宝宝今天回来睡觉,姐姐可以回来,也可以不回来,但是宝宝一定要回来。如果姐姐回来的话,让姐姐在路上买几斤甜瓜,让姐姐买,别让宝宝买,宝宝买的不甜,姐姐买的甜。''

白大千被女儿说出了一脑子乱麻。他叹息了一声,又环视了女儿的新居。看房子和车子,女儿是高攀了;可是看女婿,女儿又是下嫁了。这才结婚几天啊,佳琪也学得疯疯癫癫了。

白大千给女儿提前预备出了一顿午饭,然后回到了市郊写字楼。史丹凤中午吃了一顿盒饭,此刻正坐在电脑前给他PS照片。无心靠墙站在另一边,双手捧着一本翻开了的风水大全,直着眼睛盯着书页,半晌过去了,一动不动,一页不翻。他的脚下噼里啪啦响得挺热闹,是他的宠物大灰雀正在疯狂的啄手机屏幕,而一只稚嫩的小猫崽子蹲在一旁,虎视眈眈的正在盯着大灰雀瞧。

背着手站在办公室里,白大千毫无预兆的开了口:``丹凤,无心,你们说我可怎么办?佳琪硬说小飞往家里弄了一只蜥蜴,蜥蜴还给他俩做了早饭,吃完饭小飞还带着蜥蜴出门找工作去了。''他痛心疾首的一拍胸膛:``佳琪原来不是胡说八道的孩子啊!''

史丹凤天天对着电脑,忙得头晕目眩,对于白大千的牢骚是充耳不闻。无心扭头看了他一眼,轻声反问:``蜥蜴?''

白大千长长的吁出了一口气:``对了,小飞好像是想让你们今晚去他家里一趟,路上还得给他买几斤甜瓜。''

史丹凤盯着屏幕动着鼠标,心里不想去——一进弟弟的新房子,她就不由自主的要嫉妒兼伤心。眼不见心不烦,去了不如不去。放下鼠标抄起手机,她一心二用的给弟弟打去了电话。电话接通之后,背景声音十分嘈杂,史高飞显然正在大街上。史丹凤随口扯谎,说是今晚自己和无心都没时间过去作客。

扯过谎后,她做好了和弟弟打持久战的准备,不料史高飞十分痛快,答应一声之后便匆匆忙忙的挂了电话,正是个要务缠身的模样。

这一天是平安混过去了,第二天也还是风平浪静。第三天白大千拎着几样营养品又去探望女儿,结果中午他顶着一头大汗回了来,变脸失色的叫道:``不好了,小飞和蜥蜴跑到大街上卖唱去了!''

史丹凤不在办公室,只有无心从屏风后面转了出来,怀里抱着猫和鸟:``你看到蜥蜴了?''

白大千立刻摇了头:``我看什么蜥蜴!我连蛤蟆都怕,还看蜥蜴?是佳琪告诉我的!她还说那蜥蜴有一人多长——妈的你说小飞到底是往家里弄了个什么东西?他不知道家里还有个孕妇吗?哎呀我操,气死我了!''

\chapter{他们的生活}

白大千气得指天骂地捶胸顿足,越想越感觉女儿是受了委屈,每天孤零零的在家里连吃带喝,胖得可怜见儿的。甩着偏分刘海在办公室里大规模的发了一顿牢骚,末了他对无心说道:``今天没生意,你下午去一趟市区,替我看看小飞究竟是养了个什么东西,有没有毒吃不吃人。要是猛兽的话,你立刻想办法把它扔了。''

无心把手里的小猫托给了他:``行,你帮我照顾着猫,我走了。''

白大千抱着猫,追着他做了个向后转:``鸟呢?出门还带着鸟哇?''

无心穿了一条带有大口袋的短裤。弯腰把大灰雀塞进了口袋里,他来不及回答,直接推门走了个无影无踪。

无心挤上一辆公共汽车,顶着烈日跑到了史高飞家。给他开门的照例还是佳琪,佳琪穿得挺整齐,正在往一只保鲜袋里装桃子。无心在门口脱了鞋,光脚跑到厨房喝了一通凉开水,然后抹着嘴走回客厅问道:``佳琪,你知不知道爸和蜥蜴现在在哪里?''

佳琪装了满满两大口袋水蜜桃,装得干净利落:``宝宝跟我走,我去给哥哥和蜥蜴送水果吃。''

把水蜜桃和几瓶冰镇矿泉水放到一只大布兜里,佳琪又道:``蜥蜴喜欢吃樱桃,可是樱桃太贵了,哥哥舍不得给他买。我想给蜥蜴买,蜥蜴是只好蜥蜴,哥哥把他的尾巴扭伤了,他也不生气。''

无心伸手掏了掏口袋,只掏到了一把零钱,无法资助佳琪买樱桃喂蜥蜴。伸手替佳琪拎了沉甸甸的大布兜,他跟着佳琪出门下楼。小区位于市区中的黄金地带,出了小区大门再走一站地的距离,便到达了市中心商业区。因为都是直来直往的大街,所以佳琪走得轻车熟路。十几分钟之后,她带着无心进了一条长长的地下过街通道。过街通道因为不见天日,所以脏兮兮的阴暗凉爽。在通道靠墙的一侧围了一圈花红柳绿的青年男女,无心和佳琪走近了,只听到最后一声``嗡''的琴弦响。

佳琪率先穿透了人墙,无心拎着大布兜紧随其后。当着一圈观众的面,他没敢叫爸,只对着大蜥蜴点头一笑,然后打开布兜放到他们面前:``我们带了桃子。''

史高飞和大蜥蜴本是站在墙壁前,此刻见无心和佳琪到了,史高飞立刻有了笑模样,蹲□在大布兜里挑挑拣拣。挎着吉他的大蜥蜴最爱吃水果,所以见状也跟着弯了腰。无心蹲在一旁陪伴着,只见两人面前摆了一顶小草帽,草帽里面扔了零零碎碎不少钞票,面额全在一元到十元之间。

围观的观众们站得很稳当,并未因为史高飞和大蜥蜴吃水蜜桃而离去。一个烟花烫齐刘海的女生开口问道:``帅哥,你们不唱啦?''

史高飞吭哧吭哧的啃桃,咕咚咕咚的喝水:``累了,等会儿再唱。''

女生笑嘻嘻的又道:``等会儿让我点首歌行不行?''

史高飞一摇头:``不行,你点的我都不会唱。''

女生听闻此言,芳心大乱,被他我行我素的傲人风采所倾倒:``哎呀,我可以专点你会唱的嘛!''

史高飞埋头吃桃,不回答了。吃着吃着他忽然抬了头,对着无心说道:``宝宝,天气太热了,你和佳琪回家吧,晚上我们一起出去吃大餐。''

无心``哦''了一声,乖乖的拎着大布兜和佳琪回家了。

夏季的天又长又热,无心和佳琪无所事事,又没有一直胡吃海塞的肚量,所以吹着空调犯了困,他们东一头西一头的睡在了一张大床上。正是睡得舒服之时,史高飞和大蜥蜴却是提前回来了。

佳琪睡沉了,无论如何不肯醒,并且伸胳膊伸腿的占据了大半张床,于是史高飞把半睡半醒的无心扛出了卧室,要和儿子亲热亲热。大蜥蜴热坏了,把身上的T恤一直向上卷到胸口,他肚皮贴地趴在了立式空调前方,想要截留冷空气。

无心闭着眼睛歪在沙发上,含含糊糊的问道:``爸,你回来得真早。''

史高飞当他是个小婴儿,挠挠他的肚皮,挠挠他的脚心,同时眉飞色舞高声答道:``宝宝!爸爸明天不去地下通道了!下午有个酒吧老板带我们去了他的店,要我们每天晚上到他店里去唱歌!''

无心缓缓的睁开了眼睛:``嗯?''

史高飞自顾自的继续说道:``他问我和蜥蜴以前是干什么的,还问我和蜥蜴是什么关系。嘿嘿嘿,我当然不会实话实说了,难道我要告诉他我和蜥蜴已经组成了银河系同盟军吗?''

无心渐渐的坐直了身体:``嗯?''

大蜥蜴也慢吞吞的瞟了史高飞一眼,但是涵养很好,并未作出反驳。

史高飞高声大气,连说带笑:``我呢,告诉酒吧老板,说我和大蜥蜴是好朋友,以前是在家里吃闲饭的,现在决定联手做一番事业,自食其力了!''

大蜥蜴已经自食其力了好几百年,如今听了史高飞的话,他微微的感觉有些委屈。但是因为已经看透了史高飞的本质,所以他通情达理的继续保持沉默。

无心打了个哈欠,揉了揉眼睛,终于彻底清醒了。直愣愣的望着史高飞,他真怕对方是受了坏人的骗——反正他是从来没听史高飞正经的唱过歌,不知道凭着当今的市场行情,史高飞的歌声究竟能够价值几何。

史高飞迎着他的目光,抬手捧住了他的脸揉搓了一下,背对着大蜥蜴又道:``我儿子越长越像我了,我十七八岁的时候也是大眼睛。''

探头在无心的眼皮上亲了一下,他望着儿子的黑眼珠说道:``可惜啊,后来爸爸的脸越长越大,眼睛却是不变了。''

无心眯着眼睛望着史高飞,心里想象着他十七八岁的模样,想来想去,他末了只想象出了一个巴掌脸的大眼贼。

``爸。''他懒洋洋的开了口:``明晚我陪你去酒吧。''

史高飞把他揽到胸前左右的摇晃,晃了片刻之后忽然停了,推开无心说道:``我还得再上街一趟,给蜥蜴买双新鞋。蜥蜴的凉鞋太旧了,穿过之后脚好臭!''

大蜥蜴猝不及防的又成了靶子,当即羞愧的爬起了身:``不不不,我自己去买。''

史高飞已经走到了门口:``算了吧,你又没什么钱。宝宝跟我一起走,爸爸顺便给你买香芋派。我的遮阳伞和太阳眼镜呢?宝宝你的裤兜里怎么一动一动的?哦,是鸟。''

无心一手拿着遮阳伞,一手拿着太阳眼镜,裤兜里还探出了一只灰扑扑的鸟头。手忙脚乱的穿了鞋,他跟着史高飞出门了。而大蜥蜴独自坐在地板上,只感觉自己是落在了史高飞的手掌心里,于情于理都不能逃、也逃不脱了。

无心随着史高飞去了一趟商场,吃过香芋派和汉堡包之后,他独自回了市郊写字楼,告诉白大千道:``没有什么蜥蜴,和他一起卖唱的是他的朋友。他朋友的外号叫蜥蜴。''

屏风外面的史丹凤听了,立刻起了疑问:``小飞卖唱?''

无心立刻绕过屏风向她汇报:``还有一个酒吧老板,要请他和蜥蜴去店里唱呢!''

史丹凤敲着键盘嗤之以鼻:``别听他吹,他会唱个屁。''

无心没敢替史高飞说话,因为心里也是打鼓。翌日傍晚他又进了城,随着史高飞和大蜥蜴一起去了商业区的步行街。街上有一家``星野原咖啡西餐酒吧'',上下共有二层楼,正是史高飞和大蜥蜴的目的地。

无心始终只是尾随。史高飞和大蜥蜴进门之后去见酒吧老板了,他占了个小小的座位,点了一杯果汁慢慢的喝,一边喝一边留意店内环境。一杯果汁被他啜饮到了底,史高飞和大蜥蜴终于又露面了。从员工通道中走上了酒吧一角的低矮台子,史高飞和大蜥蜴分别坐上了高脚凳,另有一名年轻的小服务生弯腰为他们整理麦克风电线。大蜥蜴抱着吉他垂着头,衣裤全是灰扑扑的半旧货,唯独一双新鞋缤纷绚烂,姹紫嫣红之余嵌着荧光绿,并且还比他的脚丫子大了一号。此鞋凝聚了史高飞对他的关爱与友情,不穿是不行的,穿了又像个变态。无可奈何的缩到史高飞身后,他把两条腿扭成了麻花,极力的把脚往暗处藏。

史高飞对着麦克风吹了口气,大蜥蜴一甩手也拨动了吉他琴弦。酒吧大厅里的人声静了一瞬,同时史高飞开了口,正是大蜥蜴最钟爱的一曲《青城山下白素贞》。

无心静听了片刻,发现史高飞扶着麦克风浅吟低唱,竟还有着清澈的大男孩声音;要说唱功,谈不上多好,但是也绝不跑调,慢悠悠的把他和大蜥蜴全唱成了温柔迷离的背景。

一曲终了,有人鼓了掌。史高飞回头和大蜥蜴说了几句话,又遥遥的对着无心一笑。

如此唱一唱歇一歇,史高飞一晚上唱了五首歌。下了台子之后走进员工通道,他和大蜥蜴去办公室签了三个月的合同。

无心在史家住了一夜,翌日上午回到了写字楼。
独自进了九楼的出租屋,他把猫崽子和大灰雀一起放在了茶几上。

白琉璃始终是不肯做猫,蹲在玻璃烟灰缸里,他东啄啄西啄啄,自得其乐的不理人。小猫饶有兴味的盯着他看,看着看着伸了爪子,想要碰他一下,哪知他猛然回头,一口正叨上了猫爪肉垫。小猫立刻缩了爪子,喵喵叫着对着他一呲尖牙。

无心在茶几前席地而坐,心里默数自己的亲人:白琉璃在烟灰缸里,姐在楼下公司里,爸在市区新房里。天下太平,万物安好。

自然而然的盘起了双腿,他隐约记起自己似乎也曾做过许多年的僧人。双手扳着膝盖闭了眼睛,他效仿小沙弥念佛经,前仰后合左摇右摆的晃了一圈。

晃过之后坐正身体,他无声的微笑了。虽然永生不死,但在人间,他也有他的轮回。

俯身把下巴抵上了茶几表面,他轻声说道:``白琉璃,我心里真清净,真快乐。''

话音落下,他被小猫一爪子挠了个满脸花。

三道红伤纵贯了无心的面孔,他算是暂时破了相。史丹凤急急忙忙的上楼给他疗伤,他也不抚今思昔的发感慨了,哭丧着脸坐在沙发上,他一边骂猫一边把脸埋到了史丹凤的胸前。史丹凤捏着个小棉球,虽然知道他是个奇异的品种,不怕猫挠,但还是想要给他擦擦伤口。可他像滩烂泥似的瘫在她的怀里,搀不起扶不起的,并且宣称自己疼得厉害,晚上要吃一盘对虾补一补身体。

史丹凤气得抬手打了他一巴掌:``吃对虾就说吃对虾,你少跟我装模作样!真是的,越来越烦人了。拱什么拱,一边呆着去!还拱?还拱?哎呀,还敢咬人\ldots{}\ldots{}不许闹了,一会儿白大师该回来了\ldots{}\ldots{}别闹\ldots{}\ldots{}你别乱扯,我自己解\ldots{}\ldots{}''

无心的兴致是忽然生出来的,而且一瞬间便野火一样把他烧成了身不由己。史丹凤要带着他回卧室去,可他急得抓心挠肝,竟是连一秒钟都等不得了。抓起沙发上的一顶大遮阳帽,他摸索着扣住了烟灰缸里的白琉璃,算是让他非礼勿视。

白琉璃蹲在乳白色的遮阳帽里,轻轻去啄帽子里垂下的线头。帽子外面正卷着惊涛骇浪,沙发吱吱嘎嘎,人也哼哼唧唧。通过透明的玻璃茶几往下瞧,他能看到无心的一只赤脚——那只脚踏在光滑的地板上,正在一蹬一蹬的借力。

无心是个懒蛋,很少做出如此卖力的姿态,所以白琉璃看得饶有兴味,甚至起了恶作剧的心思,恨不能在他的脚趾头上狠啄一口。

良久之后,沙发上的两人鸣金收兵。噼里啪啦的互相亲了十几个嘴之后,史丹凤下楼去了,无心则是去洗了个澡。

白琉璃蹲在帽子里打了个盹,清醒之后发现帽子没了,窗外也下起了淅淅沥沥的小雨。无心裹着一条毛巾被躺在沙发上,睡得如同挺尸一般。拍着翅膀飞到了无心的胸膛上,他来回踱了两步,最后收拢翅膀,在无心的肚皮上蹲下了。百无聊赖的望着天,他一点一点的往前回忆,一直回忆到了上百年前。

他不是个很有感情的人,理智也匮乏。先前在山里和无心吵架的时候,无心总说他太任性。他不知道无心的评价值不值得相信,无心总是没个正经,谁知道他的话有几分准?

如果这话是别人说的,也许能让他信上几分,但他又从来不肯听别人说话。

白琉璃难得的做了一次自我检讨,可惜这次检讨并没能触及他的灵魂。他低头啄着毛巾被上的线头,越啄越来劲,最后就把检讨的事情给忘记了。

无心当晚吃了半盘子对虾,把猫崽子捉住又揍了一顿,然后带着白琉璃出了门,专往人迹罕至的偏僻地方走,想要捉些不成气候的小鬼给他吃。

在回家的路上,他给史高飞打了电话。史高飞刚刚出了酒吧,向儿子讲述了两件事:第一,他发现了一家物美价廉的小吃店,现在正和大蜥蜴在店里喝啤酒吃烤蘑菇;第二,今晚在酒吧里,有个女的想请大蜥蜴出去吃夜宵,大蜥蜴不为美色所动,凛然拒绝了。

他说这话时,大蜥蜴拿着一串烤蘑菇坐在对面,欲言又止的抬起头又低了头,感觉自己这点隐私全被史高飞出卖了。史高飞并不能体谅他对自己这种爱恨交织的心情,只自顾自的仰头灌了半瓶啤酒,然后对着他打了个惊天动地的大酒嗝。

吃饱喝足之后,史高飞又打包了几串烤蘑菇,带回家去给佳琪吃。佳琪毫无孕妇的自觉,想方设法的四处寻觅垃圾食品往嘴里塞。史高飞给她什么,她就欢天喜地的吃什么。大蜥蜴又是太自觉了,大半夜的进了厨房,他把明天早餐需要的材料预备齐了,又轻手轻脚的擦亮了客厅地板。史高飞和佳琪全没有睡,一起在地板上打了个滚,然后坐起来对着大蜥蜴笑嘻嘻:``真干净。''

与此同时,远在城郊家中的无心也未入眠。他穿着一条紧绷绷的三角裤衩,蹑手蹑脚的潜入厨房,把晚餐剩下的半盘对虾偷偷吃光了。史丹凤睡得天昏地暗,毫无知觉;白大千在卧室规划着自己的大事业,也没留意到厨房里的动静。

白琉璃睡了,猫崽子今天挨了几顿好打,叫得精疲力竭,也睡了。万籁俱寂的一夜过后,天上飘起了小雨星子,史丹凤开了窗户仰观天象,一边看一边打了个喷嚏:``降温了。''

从这天开始,天气一天冷似一天,夏天连个尾巴都没留,让人一步跨进了秋天。几场秋风吹枯了绿叶,似乎只是一眨眼的工夫,今年冬天的第一场小雪已经落了地。

这一年的秋冬两季过得波澜不惊,公司的生意全做得顺遂,佳琪的肚子也越来越大。大蜥蜴顶住了史高飞的热情挽留,拼了命的搬了家——也没搬远,他直线下降到了史家所在高楼的地下室里。地下室常年对外出租,租客以商业区内的服务员们为主,租金十分低廉,房源也总是充足,唯一的问题是潮湿,但作为地下室,不潮湿就怪了,所以这一点也不值得挑剔。

史高飞抱着一床崭新被褥,跟着大蜥蜴下了楼,从楼外的一侧入口里往地下室走。大蜥蜴的新居只有六七平米大,进了门就得上床。史高飞把被褥往床上一扔,然后皱着眉头环顾四周,只觉得空气都是冷而湿的,让人一秒钟都不能忍受。大蜥蜴却是怡然自得的跳上了床——作为一只常年在岩洞深处生活的蜥蜴,地下室的温度与湿度都让他感觉十分舒适。铺好被褥之后抓过手机看了看,手机信号乃是满格,可见住在地下也不耽误他和地上的史高飞联络。

史高飞,因为是真心实意的拿他当朋友,所以心里有点难受,很直白的说道:``蜥蜴,你还是跟我回家吧。你看你的样子,惨兮兮的。''

大蜥蜴打定了主意,在床上蹲得很稳当,无论如何不肯再回史家。等到史高飞悻悻的离去了,他跳下床去锁好房门,然后脱了衣服恢复蜥蜴原形。很快乐的甩了甩大尾巴,他关闭电灯趴上床去,自自在在的闭眼睛睡着了。

大蜥蜴前脚一走,史家后脚又来了人——史一彪和赵秀芬带着无数礼物前来做客,一盆火炭似的笼络着佳琪和白大千,且给史高飞小两口一人买了一件貂皮大衣。史丹凤在旁边眼睁睁的看着,从头看到尾,连根貂毛都没得到。及至他们回火星镇了,史丹凤憋气窝火的回了家,连着对无心唠叨了好几天:``我倒不是眼红佳琪,我是看不惯他们偏心。平时那些小事我就不提了,可是这回——他们明知道小飞根本不喜欢貂皮大衣,还腆着脸非买不可,不要硬给。我知道我不能和小飞比,可都是一个娘肚子里出来的,他们就不能问我一句吗?''

此刻正是夜晚,无心先钻进被窝里躺下了,陪着笑仰脸哄她:``姐,你别生气,把钱取出来,我给你买。''

钱进了史丹凤的手,向来是有进无出,于是她对着无心一瞪眼睛:``买什么买!房子都还没有呢,也好意思穿貂皮?''

在史丹凤嘀嘀咕咕之时,史高飞也正在家搔首弄姿。穿着他的貂皮大衣站在穿衣镜前,他看了又看,又抬手在肩膀和袖口之处比量了许久。最后他对旁观的佳琪说道:``肩膀改一改,就能给宝宝穿了。''

佳琪愁眉苦脸的捧着大肚皮,不是因为有心事,而是纯粹因为肚皮太大,搞得她坐立不安:``你像大熊。''

史高飞耸了耸肩膀:``毛茸茸的,谁穿都会像熊,你也一样,你那件是白的,穿了会像北极熊。不过宝宝就不一样了,宝宝可爱,穿什么都好看。这件大衣又轻又暖,其实送给蜥蜴也不错。但是我想了想,又有点儿不舍得,还是给宝宝穿吧!''

脱下貂皮大衣放进专用的大皮包里,史高飞是个急性子,立刻就想去找无心量量尺寸。无心白天总不得闲,他晚上又有工作,如果要见面的话,至早也是明天夜里。史高飞踌躇了片刻,越想越是等不得。抄起电话拨通了大蜥蜴的手机,他说自己夜里要出趟门,让大蜥蜴上楼睡一夜,权当是帮自己看家。

大蜥蜴一口答应,然后在三分钟之内敲响了史家房门。把大蜥蜴放进客厅里,史高飞提着大皮包穿了大皮鞋,转身向外就走。佳琪也回房上了床,大蜥蜴则是轻车熟路的找到一条毯子,静悄悄的躺上了沙发。

如此过了许久,大蜥蜴睡得呼呼噜噜,情不自禁的又露了原形。佳琪在卧室里都哭出声音了,他还在一无所知的高卧酣睡。

佳琪是在梦里被疼醒的,说不清那是怎样一种疼法,反正肚子里绞着拧着,连带着腰都不听使唤了。她慌了神,一边哭一边喊爸爸,糊里糊涂的滚到了床下。一路挣扎着往外爬,她先是小声的哭:``爸爸\ldots{}\ldots{}爸爸救命啊!''

不知爬了多久,她进了客厅。人在地上摸不到电灯开关,她忽然想起沙发上还睡着好心肠的蜥蜴,便一边哆嗦着往前蹭,一边换了叫法:``蜥蜴\ldots{}\ldots{}蜥蜴救命啊!''

黑暗之中,沙发上高低起伏,正是大蜥蜴仰面朝天,长嘴撅起了多高。佳琪一路哭哭啼啼的往前爬,也不知道怎么就爬得那么累,及至到了沙发一边时,她已经没了哭叫的力气,呼哧呼哧的只剩了喘。一只手颤巍巍的伸向上,她本意是要去推大蜥蜴,可是手伸到半路停住了,她在一阵锐痛之中慌乱一抓,却是抓住了蜥蜴的大尾巴。

这一抓,就放不开了。

五根浮肿的手指拼命的攥紧了,她从嗓子眼里挤出一声惨叫。靠着沙发坐在地板上,她没经历过这个疼法。扯过尾巴往嘴里一填,她不假思索的狠命一咬,咬得大蜥蜴惊吼一声,一跃而起。

愣头愣脑的望了佳琪几秒钟,大蜥蜴立刻反应过来了。从佳琪手中强行夺回了自己的大尾巴,他匆匆忙忙的化出人形,又把衣裤潦草的穿好。展开毛毯包住佳琪,他拦腰抱了她就往门口跑——跑了没有两三步,他猛的停了脚步向后转。把一头热汗一脸泪的佳琪放在沙发上,他抄起电话拨了120。

救护车即刻赶来运走了佳琪,随行上车的大蜥蜴把电话打给了史高飞。史高飞今夜留宿在郊外写字楼里,本来正在幸福的搂着儿子睡大觉。冷不防被手机震醒了,他抄起电话怒问:``谁啊?''

大蜥蜴太慌乱了,虽然是在勉强保持着自己的人形,可是嘴里的舌头已经长长的分了叉,说起话来很不利落:``你快回来,佳琪要生小孩子了!''

史高飞大叫一声,扔了手机跑出卧室,在黑洞洞的客厅里高叫:``快来人啊!佳琪要生小孩子了!''

此言一出,两扇房门同时开了,出来的正是史丹凤和白大千。史高飞望着他们,把方才的话重复了一遍:``佳琪要生小孩子了,已经被蜥蜴送到医院里去了!''

史丹凤和白大千僵在了原地,异口同声的说道:``日子不对啊!''

紧接着他们在黑暗中对视了一眼:``早产了!''

话音落下,史丹凤蓬着头发转身往回跑,将衣裤往身上胡乱的套。白大千尽管也在做着同样的事情,然而要比史丹凤热闹得多——他一边穿,一边咧着大嘴开始哭,因为女儿早产了。早产和难产,听着多么的像,吓死他了。

四个人一窝蜂的下了楼,在天寒地冻的夜路上走了足有一里多地,才拦到了一辆出租车。夜里车少,显得道路特别的宽阔平坦。司机听说他们是往医院赶,很体贴的把车开得快要平地起飞。而白大千哭了一路,及至终于到医院了,他下车之后第一件事却是去了厕所。情绪太激动了,他已经快要尿裤子。

等他系着裤腰带走出卫生间时,史高飞和史丹凤已经全不见了踪影,只有无心站在一架正在上行的电动扶梯上,拼命的向他招手示意。他连滚带爬的追了上去,心里哀哀的痛骂史高飞,因为史高飞把家安在了繁华地带,导致佳琪把孩子生在了全市最豪华的医院里。豪华医院太大了,他要跑多久才能看到佳琪?

没等他跑出多远,史丹凤扯着史高飞,一阵风似的迎面冲了过来。白大千立刻停步问道:``你们干什么去?''

史丹凤头也不回的答道:``你们往前走,我们交钱去!''

白大千平日里能说会道,百般的精明,如今真是事到临头了,却是只会坐在走廊长椅上哭。哭得医生护士纷纷侧目,不知道他一个半老头子,在妇产科嚎的是哪一出戏。无心手足无措的坐在一旁,也是没了主意。唯有史丹凤风风火火,牵驴似的牵着史高飞东奔西走,该缴费缴费,该签字签字。而大蜥蜴一时没了着落,只好六神无主的在无心身边也坐下了。

凌晨时分,佳琪生了。

生了个三斤多的小男婴,一出娘胎就进了保温箱。佳琪是顺产,睡了几个小时之后也就醒了过来。睁开眼睛向外一看,她吓了一跳,发现单人病房里全是人,甚至连前几天来了又走的史一彪夫妇,也重新出现了。

史一彪在清晨时分接到了史丹凤的电话,听闻佳琪给自己生了个三斤多的活孙子,他在狂喜之下抛弃二三四五奶,带着家里的正房黄脸婆直奔了江口市。他和赵秀芬常年不和,唯独在重男轻女四个字上是一对知音。孙子,虽然只有三斤多,但单凭着他是个带把儿的崽子,就足以让史一彪夫妇对其顶礼膜拜了。

白大千什么忙也没帮上,唯一的成绩是哭哑了嗓子,以至于现在望着女儿说不出话。史高飞在病床前弯下了腰,很认真的问道:``佳琪,你不会死吧?''

佳琪在枕头上摇了摇头,气若游丝的答道:``我好饿啊。''

史高飞又问:``你想吃什么?''

佳琪气息微弱的答道:``我想吃烤鱿鱼\ldots{}\ldots{}多放辣酱\ldots{}\ldots{}''

史高飞转身就要出去买烤鱿鱼,在门口被赵秀芬拽住了。伙食从烤鱿鱼变成小米粥,佳琪一喝就是两大碗。看见自己的大肚皮平了,她挺高兴,想要看看自己生了个什么,可是小婴儿躺在保温箱里,而她又不适宜下床行走。

然后,她和史高飞一起吃起了热蛋挞,暂时把孩子忘掉了。

\begin{quote}
作者有话要说:本文将于近期完结。
\end{quote}

\chapter{家人们}

佳琪在医院里住了三天,史一彪夫妇很有一点卸磨杀驴的意思,天天只知道围着保温箱看孙子,对于佳琪是明显的不上心。白大千倒是怀有满腔父爱,可又没法亲自伺候女儿,只能是从早到晚的陪在病房里,一是给佳琪端茶递水;二是严防史高飞作乱——昨天众人一眼没照顾到,史高飞差点勾搭着佳琪走回了家。

到了第四天,佳琪仿佛已经恢复了元气似的,开始有精神对着父亲和史高飞连说带笑。白大千很高兴,不是高兴自己有了外孙,而是高兴女儿平安无事;史高飞也很高兴,因为感觉佳琪鼓着肚子的样子很怪异,现在好了,终于不鼓了。

大家忙着高兴,零七八碎的琐事全压在了史丹凤一人肩上。她也没生过孩子,没有经验没有知识,但是史一彪和赵秀芬把她当成了万事通使唤,导致她终日奔波,又要雇月嫂又要买奶粉尿不湿以及一切婴儿必需的小玩意儿。圣诞将至,天寒地冻,陪伴她的只有无心。史丹凤偶尔得了一时半刻的清闲,必会不动声色的偷偷凝视无心,凝视到了最后,她暗暗的叹了口气,心想旁人对自己都没有真感情,要说好,还得是无心。

第四天的中午,佳琪出院回了家,月嫂也就了位,站在厨房里给佳琪熬鲫鱼汤。佳琪始终是没有奶水,导致赵秀芬和史一彪背地里嚼舌头,认为儿媳妇太不顶用,既没有奶,生的孙子又只有三斤多。赵秀芬跃跃欲试的想要拿出婆婆的派头,甩给佳琪几句闲话听听;然而闲话甩是甩了,佳琪却是乐呵呵的完全没听出言外之意。她很失望,还想继续甩,结果同样没有听懂的史高飞不耐烦了:``行了,妈你吵死了!''

赵秀芬怕儿子,史高飞一发话,她立刻成了属黄花鱼的,贴着墙根溜出了卧室。

佳琪一直没能见到儿子,于是干脆利落的把儿子忘了个一干二净。史高飞对于儿子更是毫无接收的意思——他可不想弄得家里满坑满谷全是地球人!

于是在一个多月后,当他看到赵秀芬从医院抱回了六斤重的小男婴时,气得当众沉了脸:``怎么回事?全弄到我家里来啦?''

史一彪搓了搓手,没敢言语,使了个眼色让赵秀芬说话。赵秀芬捧金子似的捧着个小襁褓,也很打怵。白大千站在一旁,直着眼睛看着孩子,心想要是没有他,自己也不至于要把佳琪嫁给史高飞。

史高飞站在门口,摆出一夫当关万夫莫开的架势,想要捍卫家园的纯洁。赵秀芬抱着孩子进退两难,月嫂站在厨房门口,也是犹犹豫豫的不知该不该迎接。忽然一眼瞄到了史丹凤,赵秀芬立刻有了目标:``小凤,你可真是的,傻站着等什么呢?三十多岁的人了,一点儿眼色也没有!''

无心正在隔壁卧室里和佳琪玩电脑,客厅说话,卧室里可以听得清清楚楚。史丹凤带着个小丈夫过日子,本来就心虚,如今听她妈说她``三十多岁'',心中登时腾起一股子恼羞成怒的火。一甩手走向卧室,她头也不回的说道:``我不会抱,要抱也轮不到我。''然后站在卧室门口叫道:``无心,别玩了,我们回家!''

史高飞不想让无心走,所以回头打断了史丹凤的命令:``不行,不许回家!''紧接着向前面对了父母,他义正词严的说道:``小孩子我们不要,送给你们了,你们带走吧!''

赵秀芬和史一彪面面相觑,又求援似的一起望向了白大千。白大千手足无措的做了个深呼吸,忽然竖起一根手指轻声说道:``有办法了!''

白大千从卧室里请出了无心,让无心去和史高飞说话。无心对史高飞和佳琪有感情,对于他们的孩子却也是毫无兴趣。当着众人的面,他毫无诚意的劝了史高飞几句,结果被史高飞扯着衣领打了一巴掌,因为他``吃里扒外''了。

史一彪夫妇见儿子真动了怒,立刻识相撤退。孙子太小了,没办法直接带回火星镇。无可奈何之下,他们只好跟着白大千赶往城郊,连人带孩子一起去了写字楼上的出租屋。幸而出租屋够宽敞够明亮,设施家具也齐全,供暖尤其是好,暖气永远热得烫手。把大粽子似的襁褓放在沙发上解开了,赵秀芬小心翼翼的从里面捧出了六斤多的孙子。白大千情绪复杂的凑近看了看,先前他不看倒也罢了,如今骤然和婴儿打了照面,他心中一动,竟是猛的生出了一股子爱意。婴儿明显是病怏怏的不健壮,但是眉目五官已经长清楚了,活脱是个小佳琪的模子,唯有细枝末节是随了史高飞。

``哎哟\ldots{}\ldots{}''他很意外的惊叹了:``睁眼睛啦?''

他的气息扑到了赵秀芬的面颊上,赵秀芬飞快的瞟了他一眼,一颗沉寂了十几年的心灵生出了蠢蠢欲动的春意。毫无预兆的,她不好意思了。

``可不是睁眼睛了?''她仿佛从更年期一步退回了青春期:``瞧瞧,他看你呢。''

史一彪刚撒了泡尿,此刻推了卫生间的门走进客厅,他望着赵秀芬和白大千并肩而立的背影,忽然感觉此情此景有些不大顺眼,但是又挑不出毛病。

因为儿子坚决不允许孙子进门,孙子又弱小得不能出远门,所以别无选择的,史一彪只好让糟糠之妻留在了江口市。完全把孙子交给月嫂伺候,是不能令人放心的,非得有个亲人在旁边照应着才行。史白两家的人口加起来,唯一合适的人选便是赵秀芬了。

史一彪惦念着家里的生意,赶在年前回了火星镇。赵秀芬独自留在写字楼上的出租屋里,单枪匹马精神焕发,把孙子照顾得密不透风。她是得意了,史丹凤却是吃了苦头——赵秀芬如今有了孙子,越发的不拿她当人了。

于是不出一个月的工夫,她也搬了家。在市区边缘的一处小区里,她租下了一套三十多平方米的小房子,也没做天长地久的计划,只想暂时避开他妈的锋芒。

偌大的出租屋里骤然只剩了赵秀芬和白大千两个大人,史一彪在家里得知了消息,依然是挑不出毛病,然而越想越是不对劲。坐拥着他的二三四五奶,他时不时的就感觉自己头上发绿后背发硬,很有当王八的征兆。但是话说回来,凭着亲家公的风采和身份,找小姑娘都能找了,又怎会青睐一个当了奶奶的黄脸婆子?

思及至此,史一彪略略的放宽了心,认为自己还是想多了。转眼之间到了新年,他孤身一人又去了江口市。在出租屋里见到赵秀芬时,他吓了一跳,发现黄脸婆子居然脸也不黄了,嗝也不打了,从头到脚收拾得利利落落,堪称是徐娘半老、风韵犹存。

他留了心眼,转而再去观察白大千。白大千前些日子又受了汇丰的刺激,如今彻底陷在了钱眼里不能自拔。百忙之中见了史一彪一面,他心底无私天地宽,一清二白坦坦然然。史一彪看亲家公神采奕奕,比上次相见时又帅了几分,一颗心便是上不着天下不着地的悬在了半空中,不知道自己的隐忧到底有没有继续存留的必要。

一个电话打给了史丹凤,他吆喝狗似的,让女儿女婿马上搬回写字楼住。哪知史丹凤立下了造反的主意,不肯听他的话。嫁出去的女儿泼出去的水,史一彪在用得着女儿的时候才发现覆水难收了。

史丹凤新租的房子不但狭窄,而且陈旧;说是个一室一厅的格局,其实厅小得虽有如无,唯一可以活动的场所便是卧室。晚上下班回了家,她照例是要在满壁油烟的老厨房里炒菜做饭。和先前单身时相比,她现在出手阔绰了许多,尤其在一日三餐上很大方,绝不肯亏待了无心的嘴。煎炒烹炸的预备出了三菜一汤,她干活干惯了,也不指望着无心帮忙。一样一样的把菜端到床前的圆桌子上,她一边忙一边说话:``现在房价越来越高,我们真不能再等了,再等连城外的房子都买不起了。无心,别玩了,起来吃饭!还玩?是不是想等我把饭喂到你嘴里?我想好了,我要挑个好地点,面积大小无所谓。反正只有两个人,一间屋子也住得开。''

把两碗米饭放到桌上,她转向大床,用女低音做出震慑性呼唤:``无心!''

趴在床上玩手机的无心一翻身坐起来了,爬到床边面向了圆桌,他端起饭碗往嘴里扒了一口饭。史丹凤问道:``饭香不香?这米很贵呢。''

无心在腾腾的蒸汽中抽了抽鼻子,然后抬头答道:``香。''

史丹凤搬了个圆凳子,也在旁边坐下了。抄起筷子给无心夹了菜,她一边吃一边又道:``下午是不是又跟着白大千放血去了?自己长点心眼,别让他对你起疑心。虽说他现在和我们是亲戚了,可万一他知道了你的秘密,谁敢保证他不会卖了你?''

无心连连点头:``放心吧,我知道。''

史丹凤伸手摸了摸他的脑袋:``真不知道再过十年,你会是什么样子。你说你究竟是个什么东西呢?''

无心一晃脑袋,皱起眉毛答道:``又来了。我是个人嘛!''

史丹凤叹了口气:``你是不是人,我还不知道吗?''

无心已经被她惯出了一点小脾气:``烦死了,吃饭!''

史丹凤重新抄起筷子:``哟,小爷们儿,说你几句还不乐意了!''

无心鼓着腮帮子一嚼一嚼:``总说我不是人\ldots{}\ldots{}''往嘴里塞了一口菜:``我不爱听!''

他再闹史丹凤也不生气。感觉到自己快要惹不起他了,史丹凤宽宏大量的让了步:``好好好,不说了。''

无心把空碗向她一递:``再来一碗。''

史丹凤接了饭碗起身去厨房盛饭,无心从桌面上捏了一粒大米饭,然后俯身弯腰,从床下的一只运动鞋里抓出了白琉璃。他一手扒开白琉璃的尖嘴,一手将大米饭粒往尖嘴里塞。史丹凤端着饭碗回来了,一眼看清了他的所作所为,当即叫道:``它是吃小米的,你别乱喂!放手!好好的一只鸟,都快被你抓死了!弄只鸟不好好养,弄只猫也不好好养,真是的,猫还挺贵!对了,猫呢?你又打它了?有意思,还对一只猫记了仇!吃饭,吃完了烧热水给你洗个澡。猫到底在哪里?你把它给扔了?''

史丹凤长篇大论,一口气说了无数话,说得无心直发笑:``没扔,猫在床底下呢!''

他一笑,史丹凤也跟着笑了,因为意识到自己是个碎嘴子,居然一下子唠叨了一大串。

吃饱喝足之后,史丹凤端了一大盆热水进卧室,拧了毛巾给无心擦了擦身,又从床底下抓出小猫,给小猫也洗了个澡。小猫越长越漂亮了,见了无心如同见鬼,只跟史丹凤亲近。白琉璃站在窗台上,啄着浅浅一碟小米。无心光着屁股仰卧在床上,双手举着报纸看房产广告。看着看着架起了二郎腿,他一边晃着赤脚,一边说道:``姐,我给你唱首歌吧!''

不等史丹凤回答,他清清喉咙开了嗓:``青城山下啊啊啊白素贞\ldots{}\ldots{}''

他平时说话并无异常,一旦唱出调子了,声音却是变得微哑苍凉,仿佛唱的不是青城山下白素贞,而是雪域大漠白素贞。一曲终了,他惹出了史丹凤一声叹息:``唱得像和尚念经似的,我都要听哭了。''

无心被她兜头泼了一盆冷水,登时闭了嘴。在大床来回打了几个滚儿,他不甘寂寞的又开了口:``姐,上床吧,我们一起看报纸。''

史丹凤洗漱完毕上了大床,和无心并肩趴着浏览房产广告,一边浏览一边点评,又在手边摆了个计算器加减乘除。无心的心思漂移不定,不是扯一扯史丹凤的头发,就是掀一掀史丹凤的睡衣。史丹凤一心二用的撵着他哄着他,最后撵也撵不走哄也哄不住了,她无可奈何,索性侧身一解睡衣纽扣:``小宝宝,给你吃奶,别闹我了!''

无心如愿以偿,立刻向前伸了手,嘴里自言自语的小声嘀咕:``两只大兔子!''

史丹凤一手拿着报纸,一手拿着计算器,被他逗笑了。

下一秒,无心用双手抓住了兔子之一。张大嘴巴``啊呜''一口,他作势要咬,吓得史丹凤用报纸抽了他一下:``敢?!''

无心既不敢,也不想。他只是一口叼住大兔子,亟不可待的开始吮,仿佛真是小娃娃在吃救命的奶。嘴里叼着一只,手里又抓了另一只,两只兔子,全是他的。

史丹凤天天看房产广告,从冬天看到春天,又从春天看到夏天。在大半年的光阴里,无心身边发生了以下事件:第一,天天挨揍的小猫崽子趁着无心和史丹凤不在家,把大灰雀咬死了。白琉璃走投无路,只好做猫。猫爪子拍在手机屏幕上,一架飞机也拍不碎。人生乐趣瞬间降至零点,他面无表情的蹲在窗台上,又想回家了。

第二,大猫头鹰历尽千难万险,居然找到了无心的家。无心并没有亲眼见到他,因为白琉璃直接隔着纱窗打发了他,让他自己先回大兴安岭。大猫头鹰可怜兮兮的听了他的话,拍着大翅膀继续往北飞。又因为据白琉璃描述,大猫头鹰形象极其狼狈,已经类似秃鹫;所以无心听得满心欢喜,别有一种幸灾乐祸的痛快。

第三,风头正劲的白大千在走夜路时,被本市一位鼎鼎大名的半仙买凶揍了一顿,半死之时偶遇英雄,英雄拔刀相助救了他一命,而他为了报恩,立刻将英雄聘为公司保安,月薪高达一千八百元。此英雄名叫李光明,即史高飞在火星镇的邻居兼校友。得知史高飞的儿子和史高飞的姐姐结婚了,李光明惊得张大嘴巴,一张面孔由正方变为长方,同时深感世事无常,再也不相信爱情了。

第四,史高飞和大蜥蜴的歌唱事业平稳进行,工资也涨了三成。史高飞的保留曲目是《青城山下白素贞》,每天晚上必唱一次,唱的时候时常能够在客人中带起一轮小合唱。照理来讲,他生得高大英俊,又是坐在台子前方,必能吸引所有目光。然而一名珠光宝气的小富婆眨巴着一双慧眼,却是看上了阴暗处的大蜥蜴。她想方设法的和大蜥蜴搭了好几次话,大蜥蜴温文尔雅,不冷不热的对谁都是一视同仁。小富婆遭遇了几次婉拒之后,反而越发的爱他了。

第五,史高飞的地球儿子已经有了乳名,叫做嘟嘟,是赵秀芬起的。赵秀芬和白大千朝夕相对,白大千冰清玉洁,不耽误她单方面的胡思乱想。但是慑于史一彪的剽悍和白大千的潇洒,她决定走保守路线,对亲家公过过眼瘾也就罢了。而史高飞和佳琪夫唱妇随,每天要么吃要么玩。白大千见女儿是真快乐,也就不甚甘心的认命了。

第六,无心跑了好几趟火星镇,终于上了户口有了身份。办好身份证后回了江口市,他在闲暇之时又去驾校学了一个多月。通过考试得了驾照之后,史家车库里常年蒙尘的帕萨特立刻归了他。

第七,史丹凤下定决心,终于决定买房,并且还是全款买房。房子距离史高飞家只有一站地的距离,正如她所愿,的确是黄金地点,可惜只有五十多平方米,是弟弟住宅的三分之一。无心身为史丹凤的小爷们儿,深感经济压力巨大,可饶是巨大,他被白琉璃折磨得没了办法,还是向史丹凤开了口:``姐,你听说过IPAD吗?''

史丹凤直接告诉他:``没有钱,不给买!''

无心讪讪的舔了舔嘴唇,抱着猫走出一站地,上楼去了史高飞家。

站在史高飞面前,他略略的理直气壮了一点:``爸,你听说过IPAD吗?''

史高飞翻箱倒柜的找了一通,末了从个乱七八糟的抽屉里找到了一只长圆形的硬壳盒子:``宝宝,爸爸只有个PSP,你要玩吗?''

无心打开盒子看了看,然后摇了头:``不行,它太小了。有没有大的?''

史高飞没听明白:``大?多大?''

无心一抬小猫的前腿:``按键要像猫爪子一样大。''

史高飞立刻摇了头:``没有。''

无心伸出了一只手:``爸,我的钱都给姐了。现在我想去买个IPAD,你给我钱好不好?''

史高飞二话不说,当即从抽屉里翻出了自己的大皮夹。皮夹里面放着一张照片,照片里是粉红色的一大坨,任谁也看不出它是什么,连无心自己都伸着脖子瞧了半天。末了瞧明白了,他抬手抓了抓头,忽然感觉很羞涩:``爸\ldots{}\ldots{}''

史高飞也是低头盯着照片看:``宝宝,你看你小时候,多像一条毛毛虫。''

无心难为情了:``我\ldots{}\ldots{}''

史高飞一边从皮夹里抽出银行卡,一边温柔的低声说道:``还是小时候最可爱。你还记不记得爸爸抱着你喂奶的事了?''

此言一出,无心怀里的白琉璃立刻回头看了他一眼,随即把脑袋探向了史高飞的皮夹。一双溜圆的猫眼睛盯住了照片,他一时间看清楚了,登时竖起了一身的毛。

无心捂住他的猫眼睛,把他强行摁回了怀里。接过史高飞的银行卡,他狼心狗肺的转身就跑。史高飞本来还想亲他一口,然而一步上前抓了个空,硬是没能亲到。

无心抱着白琉璃下了楼,一路往市中心走。胸口隐隐的凉了一下,是白琉璃自下而上的现出了一个脑袋:``你是毛毛虫?''

无心连忙否认。

白琉璃一脸狐疑的审视着他,无心迎着他的目光,一本正经的说道:``我是神仙。''

白琉璃缩回了猫身之中,忽然发现自己并不是很了解他。将一只猫爪子搭上了他的手臂,白琉璃抬头望着车水马龙的大街,耳听无心喃喃的又道:``白琉璃,今天我给你买新游戏机,以后不许你再往我的枕头上撒尿。我年纪比你大得多,你作为我的灰孙子,应该尊敬我,照顾我。我回家的时候你应该给我叼拖鞋,我趴上床了你应该给我踩后背。姐给我预备的水果你不应该偷吃,要吃也只能一个一个的吃,不能每个都只咬一口。还有\ldots{}\ldots{}''

白琉璃静静倾听着他的长篇大论,越听越生气,恨不能直接把他挠死。

无心下午出门,傍晚回家。一手抱着白琉璃,一手拎着个纸袋子,他垂头丧气的站在门口,身上的单薄T恤破烂成了渔网,连肚脐眼都见了光。

史丹凤正在厨房淘米,闻声赶来一看,登时大惊失色:``让人劫了?''

无心慢吞吞的摇了头,委委屈屈的答道:``让猫挠了。''

史丹凤看清了纸袋子中的白色包装盒,明白他终究还是把钱花了出去。鼻孔呼出两道凉气,她转身走回了厨房:``挠得好!替天行道,大快人心,省得我亲自挠了!''

\begin{quote}
——全文完
\end{quote}

\begin{quote}
\end{quote}

\begin{quote}
作者有话要说:因为未来的事情我也不知道,所以《无心法师》写到这里就结束了。去年刚刚开坑时,本来只是想写个鬼故事,没想到会越写越长,最后竟然成了我所写过的最长的文。
\end{quote}

\begin{quote}
感谢大家对本文的喜爱,感谢给我写长评扔地雷的同学,感谢文下每一条评论,感谢大家对我的鼓励和支持。非常非常的感谢。
\end{quote}
\enddocument
