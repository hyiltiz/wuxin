\part{民国时期}

\chapter{法师的来历}

无心法师永远不老,永远不死。

如此说来,他仿佛已经类似于神,可事实上他毫无神通,只是不老,只是不死。和凡人一样,他饿了要吃,渴了要喝,冷了要穿,累了要歇。所以在他无边无涯的人生之中,最紧要的一件事便是设法生存。当然,不吃不喝不穿不睡他也能活,至多是渐渐熬成一具人干,掩人耳目的蛰伏在僻静处守株待兔。然而饥寒交迫的感觉太不好受,而且无始无终的长久持续,让无心法师以为自己是堕进了阿鼻地狱。

无心法师不知道自己是从何处来,往何处去。太久远的往事他已经记不起了,他好像是从天而降落到人间,着陆之后就再没人管他。他不生不灭无魂无魄,只有一具不朽的躯壳。

因为头发至多只能长到睫毛的长度,所以无心在大部分的岁月里都在做和尚,做和尚好活,比卖苦力强。他自称会念经,会算命,会看风水,还会驱妖捉鬼。其中念经是真的,驱妖捉鬼也是真的,算命全是瞎诌,看风水更是胡说八道。凭着以上几样绝技,他浑浑噩噩的活了千百年,活到最后,就活腻歪了,不想活了。

无心法师的皮囊很体面,有着白皙的皮肤,浓秀的眉毛,眼窝微微凹陷着,由于常年的不想活,故而目光也是忧郁动人。他自认为挺英俊,可是难得拥有爱情,因为没有故乡,没有来历,没有家庭,没有亲人,又穷。凭他的资格,似乎只适合做上门女婿,但他的秘密瞒得过一时,瞒不过一世;一个永葆青春的女婿,足以令岳家上下毛骨悚然。况且根本无需一世的光阴,朝夕相处的日子过得稍微久一点,他的疑点便足以让家宅内外一起不宁了。

无心一度很爱和人亲近,想要找个姑娘作伴,结果天长日久露出马脚,被人当成妖怪烧过打过许多次。烧和打对他来讲,感觉都是统一的疼。他很伤心,并且也怕疼,所以渐渐离群索居,继续做他的游方和尚。

大概是在同治年间,无心法师终于坠入了爱河。一个十七八岁的丫头爱上了他,知道了他的所有底细之后,还依然爱他。无心法师快乐之极,当场脱了僧衣自行还俗,并且在瓜皮小帽后面掖了一条假辫子。带着媳妇在京城里过了十五年,媳妇长成了他的老大姐,两人就迁去了直隶一带居住。在直隶文县又过了十年,媳妇看起来开始像了他的娘。察觉到左邻右舍起闲话了,无心法师带着媳妇进了山,与世隔绝的度起了时光。媳妇最后是老死的,安安详详的无疾而终。无心法师含着眼泪伐大树做棺材,媳妇下葬这天,他稳稳当当的蹲在坟前,用媳妇留下的旧手帕蒙住了眼睛。

其实眼睛对他来讲,本是可有可无。他周身每一寸皮肤都能感知到颜色与光、空气与风。抬手向上招招摇摇,媳妇的魂魄缱绻缠绵,夏风一样掠过了他的指尖。

``玉儿,走吧。''他喃喃的说:``谢谢你用一生陪伴我,谢谢你。''

夏风稍纵即逝,旧手帕上还残留着玉儿的气息。无心法师在山里穷得很,平常的衣裳破到不能再穿,只好翻出了古旧的僧袍往身上套。午后的太阳照得他身上暖洋洋,像是玉儿伸出苍老干枯的双手,温柔的抚过了他的头脸。

在吃光家里最后一口杂合面之后,无心法师因为扛不住饿,所以独自下山谋生去了。

他当初上山之时,宣统皇帝还没有退位;如今下了山一打听,才知道民国的大总统都已经换了好几茬。坐在街边支起算命摊子,他打算糊弄几个钱买馒头吃,然而街上众人看了他的年轻面孔,一致认为他还是个小伙子,会算个屁。

无心法师没了生意,转而想去驱妖捉鬼。可镇子里面天下太平,并无妖鬼。无可奈何之下,他只得忍饿挨饥的踏上路途,直奔附近的文县而去。不料走到半路,他竟然出乎意料的得了个伴儿。

伴儿是个十七岁的姑娘,姓李,大名就叫月牙。月牙生得美人颈、流水肩、杨柳腰,身影比脸面更好看,当然脸面也不丑,明眸皓齿大辫子,是个干干净净的伶俐模样。月牙是从家里私逃出来的,因为爹娘要把她送给债主做八姨太。债主都六十二了,半脸褶子半脸麻,满嘴黄灿灿的大马牙。月牙不能坐以待嫁,于是趁着夜色深沉,收拾出个小包袱就跑了。

月牙一家是从关外迁过来的,家里丫头都不兴裹脚。月牙平日做惯活计,身体强健,又是一双大脚,奔跑起来分外得力。凌晨时分天蒙蒙亮,通往文县的小路上就只有她和无心两个人,她是有备而来,一边走一边从包袱里掏出一个棒子面窝头,一口一口的咬着吃。无心不远不近的跟在一旁,因为有日子没见干粮了,所以垂涎三尺,恨不能当场实行抢劫。

然而最后他并未真抢,因为月牙等他看到一定的程度了,主动掰了半块窝头递给了他:``师父,吃吧。''

无心几十年没有伪装过和尚,几乎连佛号都生疏了。对着月牙笑了一下,他接过窝头就往嘴里塞。而月牙看了他一眼,随即就转向了前方,不知怎的,忽然生出一阵心疼。

然后她自嘲的笑了,因为自己都是自身难保,居然还有闲情去心疼路人。

无心狼吞虎咽的吃了窝头,意犹未尽的伸舌头又舔了舔嘴唇上的渣滓。加快速度跟上了月牙的步伐,他终于开口说道:``姑娘,谢谢你。''

月牙自顾自的往前走,一边走一边又道:``文县外面的山上有座大庙,庙里和尚不少,也都吃得挺胖。你过去问问吧,要是能收了你,你不就有着落了?''

无心感觉到了对方的好意,于是跟得越发紧密:``姑娘,你是要去文县?''

月牙眼望前方,茫茫然的点了点头。到了文县又当如何?她不知道。

无心继续说道:``我也去文县。文县很大,我一定能弄到钱。等我有钱了,我请你去馆子里吃宴席。''

月牙本来都要愁死了,可是骤然听了无心的许诺,不由得愣了一下:``你个当和尚的,还要下馆子?''

无心望着月牙,不置可否的又是一笑。

月牙有一个好处,就是尽管时常感觉自己要``愁死了'',可是一分一秒的熬下去,她总有主意,从来没真愁死过。一个身无分文的大姑娘,回了家就得嫁给老头子做妾,离开家又无处投奔,怎么想怎么都没活路,身边还跟着一个招人心疼的怪和尚。和尚傻乎乎的真好看,让她看了心里难受得慌。为什么难受?说不清。总而言之,愁死了。

月牙存了寻死的心,什么都不在乎了,一边走一边对无心讲了自己的烦恼。无心歪着脑袋认真倾听,及至她说完了,两人也到了文县城门。

此时天已大亮,城门洞里人来人往,把姑娘和尚当成一对稀罕来看。月牙连活都不想活了,自然也就暂时不要了脸。而无心则是全不在意,只对月牙说道:``不至于。''

月牙十岁入关,身心都带着关外丫头的印记,问无心道:``啥不至于?''

无心从僧袍袖子里抽出一条旧手帕,双手抻开蒙上双眼。将手帕两端在脑后打了个活结,他迈步向前走去,同时头也不回的说道:``不至于死,也不至于愁!''

月牙拔脚追上了他:``你有眼睛不用,闹什么幺蛾子呢?''

无心灵灵巧巧的绕过脚下一块石头,然后轻声答道:``我在寻找财路。否则你没有钱,我也没有钱,到了中午,又该饿了!''

月牙连忙说道:``我包袱里还有一个窝头,一人一半,中午也能对付了——你慢点走,前面有臭水沟!''

无心不再理会她。长而柔软的僧袍袖子垂下来遮住了他的双手。他逆着晨风一路疾行。魂魄的光芒扑面而来,闭上眼睛,他才能看出人间有多拥挤。如此不知走了多久,张开的五指忽然合拢,他在袖内暗暗攥了拳头,鼻端掠过一丝阴冷的风。

天无绝人之路,文县果然没有让他失望。抬手解下眼上手帕,他扭头望向一旁,发现月牙已经追出了一头的热汗。月牙真不愿意追他,满大街的人都把他和她当疯子看,可是不追他追谁去?月牙现在没亲人了,就是走,也想在临走之前留给他半个窝头。

转回前方望出去,面前是两扇气派堂皇的黑漆大门。大门关得严丝合缝,无心伸出手去,猛然捶出一声大响。

门黑,显得他的手异常苍白。而院门后面立刻有了回应,声音苍老而又疲惫:``谁啊?''

无心清晰的答道:``法师!''

一阵铿锵之声过后,大门欠开一条大缝。一个形容枯槁的老头子探出头来,眯着眼睛去看无心:``谁?''

无心背过双手,直望进了老头子的浑浊眼中:``你家有鬼!''

此言一出,老头子当即一哆嗦。一只枯树枝似的老手伸出来,慌乱的扯住了无心的僧袍:``师父,请进来说——不,不,你别进来,我出去,我带你去找顾大人!''

\chapter{一梦}

顾大人走进文县家里时,正遇上一名小道士站在东厢房外,和房内的无心一应一答。房门是锁着的,因为他怕外人冒冒失失的闯了进去。

小道士神色俨然,穿得也是格外体面。忙里偷闲的对着顾大人一施礼,他同时就听房内问道:``你师祖为什么不回来?''

小道士理直气壮的答道:``师祖说了,他好害怕。''

然后房内的声音换了对象:``顾大人?''

顾大人站在院子里,摘了军帽满头抹汗:``啊,是我。''

无心说道:``顾大人,你进来。''

顾大人开了门上的锁,一闪身钻进房内。片刻之后他溜出来了,向小道士递出了一封信:``他给你师祖的信,一定得送到了。''

小道士立刻接了信往怀里揣:``好嘞,我下午赶火车回北京,晚上就能见到师祖。''

打发走了小道士之后,顾大人又回了东厢房。无心光着屁股趴在被窝里,一边肩膀晾在外面,本来是露出了白骨的,然而经过一天一夜的休养,白骨上面已然生出了一层粉红色的肉膜。顾大人忙得很,长安县的军头决定投到老帅麾下,于是很有保留的投了降。而他作为老帅的全权代表,当然不能藏起来不管事。

一屁股坐在床边,他挺费劲的弯腰脱马靴,床上摆着一张黄灿灿的大纸,上面用朱砂画了个乱七八糟,是出尘子特地派徒孙从北京送过来的,说是无心一定用得上。结果他带兵上山之后,才发现无心凭着一己之力,已然大功告成。

天气热,顾大人穿着大马靴奔波良久,如今大脚丫子见了凉空气,惬意的无法言喻。很自觉的把两只脚伸远了,他在无心身边躺了下去。龇牙咧嘴的抻了个懒腰,他又打了个气吞山河的大哈欠。

``怎么样?''他开口问道:``还疼不疼了?''

无心慢慢的把黄纸折好,塞进一只大信封里:``好多了,不妨事。''

顾大人仰面朝天的枕着双臂,扭头对他笑了一下:``说说吧,怎么回事?昨天把你弄回来之后,一直没抽出时间和你说话。''

无心侧身躺好了,面对着顾大人说道:``我把岳绮罗拖进了鬼洞里,我逃了出来,她留下了。''

顾大人眨巴眨巴眼睛:``不对啊,你不是说不能杀她吗?''

无心问道:``顾大人,你记不记得我们去年冬天最后一次经过鬼洞?当时是有丁大头的士兵来追杀我们,我们从猪嘴镇一直逃进了猪头山。''

顾大人想了想,随即一点头:``记得,我和月牙在树上蹲了半天,看着那帮小兵接二连三的下洞,下去的基本就都没上来。不是还有个闹诈尸的吗?让你抓住烧了,烧完之后你还跳进了洞,我和月牙在树上来不及拦你,急得我俩一边下树一边骂\ldots{}\ldots{}''

无心没有顺着顾大人的话头追忆往昔,只又问:``你猜我当时为什么进洞?''

顾大人摇了摇头:``有话直说!''

无心翻了个身,也向上面对了天花板:``那一夜连着死了许多人,可是我发现洞里洞外都很干净,尸首没有,魂魄也没有。可见\ldots{}\ldots{}''

顾大人略略的明白了:``那地方是有进无出,就算她有转世的本领,不得自由也是白搭,对不对?''

无心点了点头:``没错。我虽然不知道其中的道理是什么,但是洞里的确吸收了许多冤魂,这很奇怪,也很可怕。所以,我给出尘子写了一封信。''

顾大人看着他:``给老道写信干什么?''

无心叹息一声:``让老道来善后吧!或许可以把洞口永远堵死,上面再修座塔压住——他也不是完全的浪得虚名,应该总比我懂得多。让他考量着做吧,以后的事情,我不再管了。''

顾大人跟着叹息:``对,不管了。俩腿都没了,也够卖力气了。''

话音落下,无心没有回应。房内寂静,院里也寂静。无心透过玻璃窗子向外望,能看到半开半掩的厨房门。

顾大人今非昔比,没有时间天天守着无心,可是又不能让外人见了真相。命令卫兵牢牢的把守了院门,他每天早上都会把一天的饭菜端进房内,马桶也摆在床边。然后一把锁头扣住房门,屋子里就剩下了无心一个人。无心坐在床上,怔怔的去看对面的西厢房,看够了,再去看斜前方的厨房。厨房里的灶台上还摆着一只长柄铁勺,是月牙常用的,去猪嘴镇的前一晚摆在那里,从此再也没人动过。

天黑之后,顾大人通常会带着一份热饭热菜回来。无心在成长的阶段里总是胃口惊人,顾大人叼着烟卷靠墙站着,看他捧着海碗埋头大嚼,就不由得想起了天津岁月。那时候他和月牙心惊胆战的怀着希望,一天一天的把个怪物养成了人形。一颗心忽然不可思议的柔软了,他不假思索的开了口:``别成天愁眉苦脸的了,等你长齐全了,我再给你找个媳妇。老子有钱有势,别说你模样还不赖,就算你长成狗头蛤蟆眼了,我照样能给你弄个黄花大姑娘!''

无心对着海碗笑了一下:``万一将来她发现我不对劲了,怎么办?''

顾大人蛮横的嗤之以鼻:``怎么办?继续过呗,敢闹事就往死了揍!嫁太监的还有呢,你不比太监强?没事,你放心吧,真出乱子了,我替你做主!她敢不服,我烧了她的娘家!''

无心听到这里,发现顾大人的坏劲又上来了。顾大人不出头也就罢了,一旦出人头地,将来必定不少作孽。无心素来不喜欢坏人,可是对于顾大人,他只感觉无可奈何。

顾大人的主意,当然是馊主意,无心当个乐子听,听过也就算了。每个人都有自己的姻缘生死,他不能因为失去了自己的月牙,就出手去抢别人的月牙。

顾大人收拾了碗筷,因为懒,所以带着一身汗臭上了床。马桶还是摆在了床尾,他告诉无心:``夜里要是想撒尿了,就推我。使劲推,我睡觉沉。''

展开一床棉被躺下去,他关了电灯,在黑暗中又道:``师父,真的,人只要活着,就得向前看。月牙没了,我心里也难受,可是难受有什么用?难受她也活不了啊!月牙临走的时候嘱咐过我,让我照顾着你,这话我永远记得,我骗谁也不能骗她。现在仇也报了,你也没什么牵挂了,往后就跟着我吧。你应该看得出来,凭我的本领和志气,绝对不是平地卧的角色,养活一个你,肯定不成问题。''

无心笑了笑,没言语。他当然相信顾大人的诺言,可惜,顾大人再好,不是月牙。顾大人将来有妻有妾有儿有女,无须久,只要过上十年二十年,顾大人就无法向亲人们解释他的存在了。

他身上的破绽太多,比如,他不会老。

``顾大人。''他突然说了话:``你知道我为什么不做正经营生,专在鬼神身上挣饭吃吗?''

顾大人立刻答道:``我看你就是个懒蛋,根本没有上进的心思!''

无心继续说道:``我是想让人怕我,远离我。''

顾大人在朦朦胧胧的夜色中看了他一眼:``别胡说八道了,赶紧睡吧。''

无心又道:``自从玉儿死后,就再也没有人善待过我。我没想到会同时遇到月牙和你。这一百来年,我的运气还真是不错。''

顾大人心中涌出了一股子悲凉,当即翻身背对了无心:``行了行了,听你说话都瘆得慌。''

无心不说话了,悄悄从怀里取出他和月牙的合影。把照片摆在顾大人的后脑勺前,他们三个人,还是在一起。

一个月后,无心恢复了人样子。

在一个花红柳绿的五月清晨,他换了一身利利落落的单薄裤褂,说是要去青云观看望出尘子。出尘子新近从北京回来了,似乎是听从了无心在信中的建议,当真要去猪头山修塔。

顾大人睡懒觉睡得睡眼朦胧,蓬着头发光着膀子眯着眼睛,坐在床上一边挠大腿一边问道:``去青云观?行啊,让小马开汽车送你去吧!''

然后他伸脚下床,想要去趟茅房。不料无心站在门口,拦住了他的去路。

顾大人不挠大腿了,改摸下巴上的青胡子茬。无心定定的看他,他莫名其妙,也看无心。无心的眼睛是特别的黑,黑而幽深,是要把他的影子印刻吸收。

顾大人和他对了半天的眼,渐渐的醒透了,不由得抬手揉去眼角的眼屎:``看什么呢?你不是要走吗?''

无心收回目光,忽然张开双臂拥抱了他。手臂紧紧箍住他的赤裸上身,顾大人猝不及防,险些被他勒断了气,并且有点不好意思:``哎,哎,干嘛呀?大早上的别挡道,我还憋着尿呢!''

无心抬手拂乱了他油腻粗硬的短头发,随即松手后退一步。

看不够似的看着顾大人,他微笑说道:``可能要在青云观住上几天,你一个人在家,多保重。''

顾大人不以为然的一挥手:``滚吧!住个三五天就回来,咱们下个礼拜可能就要回天津了。''

在清凉的晨风中,无心对着顾大人点头一笑,然后转身走向了院门。

五天之后,顾大人派小马去青云观接无心,然而小马开着空汽车回了来,站在他面前说道:``观里的出尘子道长说,无心师父只在观里住了一夜,四天前就下山走了。''

顾大人听闻此言,不知怎的,浑身汗毛竖起了一层。撒开人马布下天罗地网,他开始四处寻找无心,然而人仰马翻的找了大半个月后,却是一无所获。

顾大人独自坐在院子里,顶着烈日骄阳发呆。忽然打了一个冷战,他怀疑自己是做了一年的大梦,梦里有个月牙,还有个无心。现在,梦醒了。

顾大人再次和无心相遇,是在十年之后。

那时他已经改名叫做顾庆宣,半俗半雅的,正好符合他越来越高的身份。人无千日好,花无百日红。因为专权和贪婪,他终于在过完四十整寿之后,被他的敌人们联合起来赶下台去了。

顾大人想得开,不犯愁,下台之后住进了天津租界里,领着一大家子继续过阔日子。在一个阳光明媚的午后,他带着两个儿子去逛百货公司,两个儿子全很像他,是儿童的年纪,少年的身量,别别扭扭的都不听话,一路把他扯了个东倒西歪。他本来就是个高大的坯子,如今又发了福,站在街上像个巨大的不倒翁,一手一个的拽着儿子,嘴里气得骂骂咧咧。眼角余光忽然仿佛瞥到了什么,他猛的回头,依稀看到了一个熟悉的背影。正要定睛细看,两个儿子又闹起来了:``爸爸你带我们去吃冰激凌,要不然我们都不走了!''

顾大人一头大汗的转向两个儿子:``吃你妈了个×!再闹就把你们两个小子撕了喂鹰!''

大儿子不怕他,继续耍赖:``不吃也行,你给我十块钱,我自己去吃!''

顾大人又回了一次头,心想:``我看见谁了?''

他也不知道自己是看见了谁,于是在两个儿子的胁迫下,像座大山似的继续前进了。

无心站在街角,隔着人潮去望顾大人的背影。

顾大人老了,胖了,有了一点老太爷的意思。从报纸上读到了顾大人的坏消息,他放心不下,所以特地赶来天津,想要偷偷看他一眼。

还好,顾大人虽然在仕途上受了挫折,然而精气神都足,并不是一蹶不振的颓丧模样。顾大人的儿子也很好,看起来活蹦乱跳,也许长大之后会比顾大人更有出息。

转身背对了顾大人的方向,无心沿着马路向前走去。阳光暖融融的洒了他一头一脸,在金黄色的幻觉之中,他看到年轻的顾大人在小四合院里抽烟望天,月牙则是系着围裙走出厨房,没说话,只对他粲然一笑。

面颊绯红,眼神明亮。她笑得真美,是他记忆中一朵不凋零的花。

\begin{quote}
作者有话要说:
\end{quote}

\begin{quote}
无心和月牙、顾大人的故事,到此就结束了。
\end{quote}

\begin{quote}
接下来我打算休息几天。几天之后,我或许是继续再写一个无心的故事;也或许是完结本文,另开一个新坑。
\end{quote}

\begin{quote}
感谢大家对本文的喜爱与支持,非常感谢O(∩\_∩)O\textasciitilde{}
\end{quote}

\part{抗战时期}

\chapter{设法过冬}

一九四三年秋,上海。

无心在一座无名荒山里度过了整个夏季,因为荒山里人少食多。在长达三个月的时间里,他吃了很多田鼠与蝙蝠,唯一一次遇到不幸,是睡觉的时候被野猪啃了一口。

夏季结束之后,山里的天气渐渐变得不适宜人居,于是他拎着一只帆布旅行袋下了山。有车坐车,有船坐船,他糊里糊涂的到了上海。抗日战争打了六年,战况很不分明,到处都不太平,倒是大都会里更安全。在一间小小的公寓里面,无心找到了容身之处。

一套公寓共有三间房屋,分别出租给了三位落魄的单身汉。一位是个小犹太,没有国籍;一位是个老白俄,没有祖国;无心作为第三位,没有财产。

去年他也曾经挣到过一大笔款子,可是他的人生无边无际,简直无法计划经营,所以采取了今朝有酒今朝醉的活法。如今将仅有的一点余钱交到房东手里,他拿着钥匙进了自己的小房间。一丝不苟的关上房门,他慢慢坐在吱嘎作响的铁架子床上,终于是一无所有了。

房里有个小洋炉子,炉膛里面挺干净,显然是三季没用过了,就等着入冬。无心虽然在山里混了许久,但是并未和现实社会脱节。战事日益激烈,煤炭一天一个价钱,凭着他的资本,连饭都吃不上,怎会有钱买煤?

无心一想起自己的衣食住行,就恨不得钻进地下,效仿蟒蛇冬眠。一动不动的坐在床上,他没有呼吸也没有表情,甚至心中都没有心事。怔怔的望着前方白墙,他百无聊赖的消耗着无尽时光。

木雕泥塑似的从下午坐到翌日晚上,最后还是难耐的饥饿催动了他。他懒洋洋的站起身,心想单是坐着也不成,还是得行动,还是得设法过冬。

摸黑走过去打开电灯,他把一只手举到了小灯泡前。长久的忍饥挨饿让他消瘦了,然而皮肉并未干枯松懈,而是渐渐硬化,似乎要与骨骼融为一体。在灯光下,他单薄的手掌呈现出了蜡质的半透明。缓缓的把另一只手也抬起来,他往墙壁上投了个手影。影子大鹏展翅,是只雄鹰。自得其乐的笑了一下,他又双手合作,映出了一只模模糊糊的狗头。

然后把手伸进怀中,他摸出了一张纸符。轻轻一拍电灯开关,他在骤然降临的黑暗中捏住纸符两端,``嚓''的一声撕成两半。一股子寒气随着破裂声音窜上他的鼻端,他的小喽啰在黑暗中幻化出了影子。

小喽啰看起来只有八九岁大,做着白衬衫背带裤的小学生打扮,衬衫很白,所以显得胸前一滩鲜血很红,一侧的耳朵脖子也是血肉模糊,永不愈合。

他叫小健,放学的路上不听话,跑到大马路上跳舞给保姆看,结果一辆电车刹车不及,当场把他碾死。大千世界,无奇不有。他也算是一奇,死后竟成了个漂泊无依的小鬼,并且结结实实魂魄不散。作恶的本事他没有;恶作剧的主意却是层出不穷。一个礼拜之前,他竭尽全力的搬运了一点火苗,想要去吓无心一跳,结果反被无心当成试验品练了手。无心花了十年时间学画符,成绩相当之差,但还是把他封在了一张纸符里。

七天之中,无心忙着找房安身,只能忙里偷闲的偶尔放他出来,当他是个小朋友。小健很不愿意被他关押,可还是立刻就认他做了大哥,因为无心看得见他,能和他说话。自从他被电车轮子碾过之后,已经连着两年没人理睬他了。

将一只血迹斑斑的小手拍向无心的大腿,小健仰起头笑嘻嘻:``大哥哥,你有房子住了?''

小手只是一个凄惨的影子,还停留在横死时的模样。畅通无阻的掠过了无心的身体,只留下一抹似有似无的寒意。

无心转身走到了小窗户前,推开窗扇探出脑袋。窗下是一条繁华的小街,油炸臭豆腐的味道一直向上冲到三楼,冲进了他的鼻端。

小街对面矗立着一座巍峨的大厦,从无心的角度望出去,可以看到无数灯火通明的后阳台。大厦里面也是公寓房子,不过价值极高,非得阔人才有资本入住。有女仆站在阳台里面淘米择菜,也有老爷少爷坐在阳台上读报喝茶。无心嗅着空气中似有似无的饭香,忽然起了劫富济贫的心思。

当然,凭着他的本领,去打劫肯定是不成。扭头看了看飘在自己肩上的小健,他心中像开水冒泡似的,咕嘟咕嘟的起了坏主意。弯腰从墙角捡起前任租客留下的空酒瓶,他把酒瓶横放在窗台上一转。酒瓶原地转过几圈之后,细长的瓶嘴向窗外定了方向。无心顺着瓶嘴一瞧,正看到了一面紧挨着后阳台的大玻璃窗,窗子没有拉拢窗帘,可见里面灯光辉煌,正是一户很富足的人家。

无心点了点头,心想:``就是它吧!''

与此同时,对面楼中享受着辉煌灯光的马家姐弟,莫名的一起打了个冷战。

马家姐弟是一对龙凤胎,当初他们的母亲怀孕之时,有经验的老妈妈看了她的形容举止,都认定腹中该是一对双生女。不料其中一位比较狡猾,居然在胎里男扮女装。马老爷偶然灵感发作,提前为女儿们拟出了一对野心勃勃的名字。及至孩子出世,真相大白,他一时失落,索性将错就错;于是女婴理直气壮,大名叫做赛维,是要赛过英国女王维多利亚;男婴含羞带愧,大名叫做胜伊,是要胜过英国女王伊利莎白。

马家在北京城中也算大户,成员十分复杂。赛维和胜伊因为是同胞的姐弟,所以在大家庭中分外亲近。时光易逝,转眼间他们进入了青春发育的时期,虽然生活优渥、营养充足,但是统一消瘦的如同野狗一般。赛维升入比利时女中,成绩介于平凡与糟糕之间,唯一的事业是舞动着两条细胳膊打排球,没有男朋友,只有女朋友。而胜伊尽管体态几乎类似豆芽,却有一颗早熟又骚动的心灵,常年在各大女校门口徘徊。可惜凭着他小鸡崽子似的风采,根本不能打动少女的芳心。以至于他在女校周边踏破铁鞋,不但一点罗曼司都不曾发生,反倒落下了个不甚光彩的外号,人称马浪蹄子。

这样一对无人问津的姐弟,浑浑噩噩的混到中学毕业。从此无所事事,越发游手好闲。在家里混了一年半载,他们合谋向父亲敲了一大笔钱,以探望姑母为名离开北京,跑来了上海。

此刻坐在吊灯下的羊毛地毯上,赛维正在和胜伊算账。两人在上海肆无忌惮的挥霍了一阵子,如今闹起了经济危机。赛维自认为比胜伊更有头脑,于是想要和他分家,从此各花各的,谁先空了手,谁就回北京去。反正公寓房子是租了半年整,足够他们住了。

赛维剪着齐耳的短发,头发先前是烫过的,剪过之后还可以看到焦黄的发梢。穿着长裤盘腿而坐,当着自家兄弟,她大模大样的低头数钱。马家的孩子说起来是成长在锦绣丛中,其实一个个见钱眼开,所受竞争的激烈程度,大概一般的孤儿院也望尘莫及。双目炯炯有神的盯着钞票,她嘴里一五一十的念念有词;胜伊伸着脖子,睁大眼睛去看她快速捻动的手指。

一时数清了数目,赛维俯身拿起铅笔,在白纸簿子上记下了一笔。记完之后她叹了口气:``娘在信里说,爸爸上个月给老四买了一件银狐斗篷。''

老四是指马家的四小姐,和他们不是一个娘,并且十年如一日的为敌。马老爷给四女儿花大钱,赛维和胜伊都嫉妒得眼红,并且全忘了自己也曾向父亲要过巨款,否则怎么可能如此舒适的跑来上海过生活?

赛维把钞票分成两部分,想要继续说话,不料在她开口之前,头顶的吊灯忽然一闪。两人一起抬了头,就听上方响起了嘶嘶啦啦的电流声音。而灯光稳定了不过几秒钟,随着声音又开始闪烁了。

赛维和胜伊全都没有生活的常识,不知道吊灯是犯了什么毛病,扬着脑袋就只是看。结果在短暂的黑暗之中,他们一起瞥到了屋角的小小人影!

猛然扭头望过去,随着电灯恢复明亮,人影却又消失无踪。赛维攥着一沓子钞票,张着嘴转向了胜伊。胜伊伸长了他的细脖子,一双黑眼睛睁得又圆又大:``姐,我们是不是\ldots{}\ldots{}看见了什么?''

赛维向角落中又看一眼,角落空空荡荡,干干净净。

抬手揉了揉眼睛,她对胜伊问道:``我们眼花了?''

然后两人一起点头,承认自己的确是眼花。赛维恋恋不舍的攥着钞票,盘算着想要从胜伊的份里克扣一些。胜伊则是向她伸出了手:``姐,钱——''

话音未落,吊灯骤然全灭!

胜伊的手停在半路,同时就觉头顶寒气一闪。伴着电流的噪音,一圈灯泡此起彼伏的亮了又灭,灭了又亮。每当黑暗笼罩之时,就会有小孩子的身影在他们的视野边缘掠过。赛维和胜伊惊声尖叫抱作一团,一起趴倒在地。侧过头去面对了沙发四条短腿,他们猛的一抖,就见沙发下面影影绰绰的,现出了一个小孩子的下半张脸——尖尖的下巴,稚嫩的脸蛋,可惜一侧面颊血肉模糊,甚至露出了苍白的骨头。柔软的嘴角微微一翘,鬼脸向他们笑了。

赛维和胜伊怔了一瞬,随即发出了惊天动地的怪叫。一只灯泡在叫声中自动爆裂,``啪''的一声,碎玻璃渣四散飞溅,全落在了两个人的短头发上。

午夜时分,小健穿过玻璃窗子飘回了家。无心没有睡,正蹲在地上整理他的招牌幌子。小健围着他转了一圈,得意洋洋的开口笑道:``他们家里有一个大哥哥,还有一个大姐姐,现在正哭着呢。''

无心不置可否的一挑眉毛:``嗯。''

小健又道:``他们家里,满地都是钞票。''

无心抬头看着小健,笑了一下。

小健落在了他的头顶上:``大哥哥,我看你不大喜欢我。''

无心终于出了声音:``你要是个人,我就喜欢你了。''

他把破旧的布幌子折叠起来,继续说道:``我很久都没有和人交过朋友了,真想找个活人说说话;不说话,让我摸一下也好。等我弄到了钱,我想养一条狗。小健,你要黑狗还是白狗?''

小健听了他的实话,心里有一点难过,低声说道:``花狗。''

无心一本正经的点了点头:``好,等我买够了粮食和煤,就养一条小花狗。''

\chapter{番外——无心和白琉璃(四)}

无心让白琉璃去弄个胶皮嘴的玻璃奶瓶回来,白琉璃外出四处找了一圈,然而一无所获。

白琉璃的儿子已经睁开了眼睛,眼珠子是深沉的蓝黑色,有点老谋深算的意思。无心从早到晚的用小勺子舀了羊奶喂他,喂得不胜其烦。单手把婴儿托到母羊肚子底下,无心捏了羊□往他的嘴里送。母羊的奶水太充足了,无心的手指轻轻一捏,雪白的羊奶便喷射了婴儿一头一脸。婴儿呱呱的嚎哭起来,摇头摆尾张牙舞爪。白琉璃在房内听见了,隔着大开的窗户向无心怒吼:``你在干什么?''

无心跪在地上,扭头对着他正要回答,不料白琉璃怒不可遏的又叫道:``不要欺负我的儿子!''

无心把婴儿从羊肚子下面抱了出来,没好气的反驳道:``我是想要找个喂奶的新办法!''

白琉璃气势汹汹的伸手一指他:``你喂!就要你喂!''

无心微微张着嘴看他,胸膛里像是藏了一座火山。岩浆憋在嗓子眼里,随时能喷白琉璃一脸。

``你妈的。''他喃喃的骂道,抱着婴儿往远走,想要避开白琉璃的监视。白琉璃终日袖着双手,什么也不干,专门盯着他。婴儿略有哭闹,白琉璃便要痛心疾首的对他大呼小叫。

婴儿一到傍晚就哭,喂饱了也哭,哭得抽抽搭搭委委屈屈。无心抱着婴儿坐在门外的大石头上,手足无措的把臂弯晃成了摇篮。白琉璃困惑而又心痛的凑过来了,用手指逗弄着儿子的嫩下巴。婴儿哭得很卖力气,面红耳赤大汗淋漓。白琉璃急了,指尖轻轻去碰儿子的小嘴:``无心,他为什么一直哭?''

无心也是摸不清头脑:``你去找个养过孩子的女人问一问。''

话音落下,婴儿忽然安静了,小嘴吮住白琉璃的指尖,他仿佛得了某种安慰似的,一吮一吮的闭了眼睛,偶尔抽一口气。

无心恍然大悟:``哦,他要娘呢!孩子天生就离不得娘嘛!''

白琉璃抽出了手指——他的手不干净,不敢让儿子肆意的又吸又舔。一双蓝眼睛望向了无心,他脑筋一转,忽然有了高招。一挺身站起来,他快步进房拧了一把湿毛巾,随即回到无心面前,不由分说的扒开了无心的袍襟。手掌裹着湿毛巾胡乱擦拭了无心的胸膛,他夺过儿子就往对方胸前送。无心目瞪口呆的愣在大石头上,就见白琉璃准确利落的把婴儿小嘴贴上了自己的一侧□。而婴儿仿佛出自天性一般,竟然一口就把他叼住了。

``哎,白琉璃!''无心怕伤了孩子,所以姑且没有躲闪:``你过分了啊!''

白琉璃很专注的盯着儿子:``虽然小了一点,不过小孩子也不懂,能够骗他不哭就好。''

无心后仰着躲了一下,没躲开:``你没有吗?你自己骗去!''

白琉璃摇了摇头:``你没有毒,就用你吧!''

无心气得七窍生烟:``白琉璃,我不和你过了!''

白琉璃这才抬头面对了他,满脸的莫名其妙:``为什么?''

无心张口结舌,因为原因太多,一时也不能尽数。而白琉璃腾出一只手拍了拍他的肩膀:``我们还是过下去吧。自从你来了,我每天都很快乐。''

无心简直要落泪了:``你是快乐了,可我呢?''

白琉璃垂下眼帘望着儿子,用轻快的声音回答:``啊,不知道。''

无心瞪了他半天,然而白琉璃无动于衷。最后无心把脸转向了远方深深的夜色,胸前热烘烘的,还拱着个小猪似的活物。

这天晚上,无心是分外的垂头丧气,甚至有种受辱的感觉。白琉璃和他说话,他也不理了,倒在床上闷头就睡。白琉璃不睡,摸着黑逗儿子玩。婴儿躺在床上叽叽嘎嘎,声音不高,有种心平气和的乖。

如此到了翌日天明,白琉璃在吃过了一大盘土豆泥后,亲自用小勺子喂儿子喝羊奶。无心本来想去河里洗澡,袍子都脱了,然而半路又被白琉璃喊了回来。死气活样的把孩子抱稳当了,他百无聊赖的斜着眼睛,看白琉璃一小勺一小勺的舀起羊奶,送到婴儿的小嘴边,一次也就喂出一滴的分量。

及至喂光了一碗底的羊奶,白琉璃用**的小勺子刮了刮无心的□,想在这代用品上增加一点奶水气息,以便以假乱真。放下勺子小碗,他起身绕到无心身后,又把手伸到前方,在对方胸膛上捏起了一把肉:``儿子,看,妈妈。''

无心忍无可忍的仰起了头,拖着长声表示抱怨:``哎——呀——''

长声结束,无心用肩头狠狠撞开了白琉璃:``你还没完了?''

白琉璃一个踉跄跌坐下去。直眉瞪眼的想了想,他一翻身爬起来,却是钻进了他的密室。

片刻过后,他拎着一只绣花大荷包出来了。让无心抱着孩子在房内的床上坐好,他郑重其事的关了门窗,然后在无心面前打开荷包,从里面掏出了一沓崭新的钞票。捏着钞票向无心抖了抖,他压低声音说道:``我的钱,以后都归你管。你听我的话,我们好好过日子吧!''

无心一手抱着婴儿,一手把钞票接过来看了看:``这是哪国的钱?''

白琉璃郑重其事的答道:``是英镑,三百英镑。''然后他低头抻开荷包口:``除了英镑,还有几十块钱的法币。''

无心若有所思的点了点头:``英镑\ldots{}\ldots{}很值钱吧?''

白琉璃一扬眉毛:``当然。''

无心的眼睛亮了一下。

白琉璃把钞票放回大荷包里,又抽紧了荷包口。把荷包放到无心的手里,他很友爱的又拍了拍无心的胳膊。

无心一闲下来,就攥着白琉璃的大荷包浮想联翩。傍晚时分望着窗外的晚霞,他坐在阴暗的房内,满脑子都是活络主意。白琉璃和他的儿子全都吃饱喝足了,正在嬉闹。白琉璃捏着一根草,先是扫了扫无心的胸膛,又扫了扫儿子的小脸。婴儿躺在无心的臂弯里,扬起小手追逐草叶,追得哈哈大笑。白琉璃把婴儿的目光引到了无心身上,又用清朗的声音催促道:``吃奶,去,吃他的奶!''

小婴儿兴奋的``噢''了一声,然后在父亲的托举下,欢天喜地的扑向了无心。

无心没有做无谓的反抗。垂下眼帘望着身前的父子二人,他看到白琉璃还在逗蛐蛐似的用一根草秆逗着婴儿。

``真够讨厌的!''无心暗想:``我又要干活,又要照顾婴儿,还要被他当成玩物。妈的,老子不伺候了!''

无心一旦生出了``不伺候''的心思,立刻感觉天宽地阔。如此熬了十几天,他终于等到白琉璃又出了门。用一根布条把婴儿绑在床上,他揣起荷包,从床下翻出一双鞋穿好。推开房门东张西望了一番,他见远近无人,便撒腿跑了。

他是有备而跑,一路直奔四川,姑且不提。只说白琉璃当晚回了家,远远看到家里黑洞洞的没有点灯,心中就是一惊。及至距离家门近了,他听房内婴儿啼哭不止,房外的铁锅也是冷冷清清。推门进房一瞧,他见儿子在床上又拉又尿,嚎的上气不接下气。门外的母羊也跟着咩咩上了,吵得人心烦意乱。

慌忙挤了羊奶堵住儿子的嘴,他抱着婴儿房前房后跑了一圈,一边跑一边就听见自己在呼呼的喘粗气:``无心!''他大声的呼喊:``无心!''

四野寂静,哪里有人回答?

白琉璃单手抱着儿子,飞身上马跑向远方,一边跑一边继续呐喊:``无心!无心你回来啊!''

后半夜,白琉璃抱着哭累了的儿子回家了。

他自己也哑了嗓子。扯下床单扔在地上,他带着儿子往床上一躺。突然双眼一睁,他一个鲤鱼打挺坐起来,从床上到床下摸了一通,发现自己的大荷包也没有了。

人没了,钱也没了。他从中午到现在,还没有吃过一口饭。无心明明都答应和他一起过日子了,却又不声不响的偷偷携款逃走。想到无心骗了自己,白琉璃气得浑身颤抖。双手抓住被褥扭绞了一阵,他不解恨,攥了拳头向下狠狠一捶床板,随即开始满床打滚,一边打滚一边呻吟。婴儿窝在床角,好奇的睁大眼睛看着父亲,连哭都忘了。

白琉璃把床板捶得山响,``咕咚''一声滚到床下,他坐起来,一边扯着自己的袍子和腰带,一边伸腿用力去蹬前方的墙壁。两只脚敲鼓似的在墙上乱蹬了一气,他颤抖着骂了一声``骗子'',随即咬着手指起身冲出去,跪在门前地上仰天长啸。两只手薅住被母羊啃短了的青草,他拔一把向上一扔,再拔一把向上一扔。忽然看到无心常用的一只饭碗摆在锅子旁边,他跑过去拿起碗,高高举起摔在草地上,然后一脚接一脚把碗往土里踩:``骗子,骗子!''

白琉璃在门外一直闹到天亮,还是没能完全泄愤。铁锅已经被他不知扔到了哪里去,石头堆成的炉灶也被他拆了。他抹了自己一脸黑灰,滚得满头满脸都是草屑。最后在房内儿子的哭声中坐起身,他俯身一头撞向地面,抬起头又抽了自己两个大嘴巴。末了抬起袖子一抹眼睛,他也哭了。

\begin{quote}
——番外完
\end{quote}

\begin{quote}
作者有话要说:番外到此结束。接下来开始写本文第三部,讲述文革时期的故事。依旧是三人行,分别为无心,白琉璃的鬼魂,以及一位漂亮小姑娘。
\end{quote}

\part{文革时期}

\chapter{苏桃}

一九六七年春,河北。

苏桃斜挎着一只帆布书包,战战兢兢的走上了二楼。楼是旧式的小洋楼,坐落在文县一隅,还是清末时期的建筑,近十年来一直是空置着的。上个月随着父亲逃来此处之后,她始终是没有心思打扫环境,所以楼内处处肮脏;角落结着长长的灰尘,本是静止不动的,然而如今树欲静而风不止,在楼外一声高过一声的口号震动中,灰尘也柔曼的开始飘拂了。

父亲坐在门旁靠墙的硬木椅子上,见她来了,就仰起了一张苍老的面孔。苏桃停住脚步转向了他,茫然而又恐慌的唤了一声:``爸爸。''

老苏是个军人,人生经历就是一首陕北的信天游。年轻的时候是``骑洋马,挎洋枪,三哥哥吃了八路军的粮,有心回家看姑娘,打日本就顾不上。''人到中年了,又是``三八枪,没盖盖,八路军当兵的没太太,待到那打下榆林城,一人一个女学生。''虽然他打的不是榆林城,但的确是娶了个女学生。

女学生是中等地主家的女儿,又在中等城市里念了书,集小农与小布尔乔亚两种气质于一身,最终升华出了一个娇滴滴的苏桃。女学生一辈子看不上丈夫,带着独生女儿和丈夫两地分居。老苏倒是很爱她的,单相思,相思着倒好,因为见了面也没话说。

文化大革命开始不久,老苏就被打成了反革命黑帮分子。眼看他的上级保护伞们都被分批打倒且被踩上了一万只脚,他决定不能坐以待毙。然而未等他真正行动,就听说远在外省的妻子被当地红卫兵们推上了万人批斗大会的台子,当众用皮带劈头盖脸的抽,抽完了又剃阴阳头。大会结束后她回了家,当天夜里就跳楼自杀了。

等到女儿苏桃单枪匹马的逃到身边之后,老苏趁着自己只受批斗未受监视,在一位军中老友的保护下,火速逃来了文县,不显山不露水的暂时藏进了一所鬼宅似的小楼里。未等他喘匀了气,老友也完蛋了,被造反派押去了北京交代问题。老苏从首长落成了孤家寡人,并且不知怎的走漏风声,引来了新一批人马的围攻。

老苏依然是个行动派,趁夜用铁丝和铜锁死死封住了外面院门,又用湿泥巴和碎玻璃在墙头布了一道荆棘防线。但是他能拦得住人,拦不住声,而且拦也是暂时的拦,拦不长久。于是他彻夜未眠,一夜的工夫,把什么都想明白了。

苏桃站在门口,不敢往窗前凑。透过窗子可以清清楚楚的看到楼外情景。楼外的人员很杂,有红卫兵,也有本地工厂里的造反派,平时看着可能也都是一团和气的好人,不知怎的被邪魔附体,非要让素不相识的父亲投降,父亲不投降,就让父亲灭亡。忽然意识到了父亲的注目,她有点不好意思,扶着门框垂下了头。

老苏凝视着她,看她像她妈妈,是个美人。用粗糙的大手攥了攥女儿的小手,他开口问道:``东西都收拾好了?''苏桃点了点头,小声答道:``收拾好了。''老苏笑了一下,笑得满脸沟壑纵横:``好,收拾好了就快走。他们要往里冲了,院门挡不了多久。''

苏桃撩了他一眼,几乎被他惊人的老态刺痛了眼睛。从小到大,她一年能见父亲一面,因为不亲近,每次见面的印象反倒特别深刻。在她的印象中,父亲还是一个满面红光、高声大嗓的中年人。

``爸爸,一起走吧。''她带了哭腔:``妈妈没了,你不能留下我一个人,我一个人活不了啊!''老苏的嗓子哑了,喉咙像是被壅塞住了:``我目标太大,不利于你安全转移。''大巴掌狠狠一握女儿的手,他深深吸了一口气:``桃桃,对于爸爸来讲,杀头,我不怕;侮辱,我不受!''

随即他松了手。一双眼睛定定的盯着女儿。女儿十五岁,美得像一朵正当季节的桃花。暗暗的把牙一咬,他逼回了自己的眼泪,起身对着门外一挥手:``快走。非常时期,不要优柔寡断错失良机!''苏桃双手一起扳住了门框,惶恐悲伤的哭出了声:``爸爸,一起走吧,我求你了,一起走吧。要不然我和你一起死,我没家了,我没地方去!''

老苏屏住自己的呼吸和眼泪。拦腰抱起哇哇大哭的女儿,他一路咚咚咚的走下楼梯。脚步沉重,震得满地生尘。楼下一间小佛堂里,搬开佛龛有个锁着小铁门的暗道。老友在把他藏匿到此处时曾经告诉过他,说是暗道能用,直通外界。门锁被他夜里撬开了,铁门半开半掩的露出里面黑洞洞的世界。

把痛哭流涕的女儿强行塞进小铁门里,他拼了命的挤出声音:``我锁门了,你赶紧走!你想回来也没有路!''然后他``咣当''一声关了铁门,当真用锁头把铁门锁住了。重新把佛龛搬回原位,他小心翼翼的除去了自己留下的指纹。外面响起了哗啷啷的声音,他们当真开始冲击院门了。

老苏摸了摸绑在腰间的一圈炸药,以及插在手枪皮套里的配枪。两条腿忽然恢复了活力,他往楼上跑去,想要寻找一处绝佳的射击点。在老苏躲在窗边清点子弹、苏桃在漆黑的地道里绝望撼动铁门之时,无心随着人潮,涌出了文县火车站。

全国学生大串联的余波未尽,火车上的乘客之多,唯有沙丁鱼罐头可以与之媲美。无心在天津上车时,根本就没有走车门的心思。人在月台上做好准备,未等火车停稳,他就直接扒上车窗,像条四脚蛇似的游了进去。眼看身边的三人座位下面是个空当,他一言不发的继续钻,占据了座位下面的幽暗空间。

舒舒服服的侧身躺好了,他和苏桃一样,也有个帆布书包。书包里空空的,被他卷成一团当枕头。枕了片刻之后他一抬头,忽然想起书包里还有一条小白蛇。连忙欠身打开书包,他低头向内望去,就见小白蛇歪着脑袋,正用一只眼睛瞪他。

小白蛇是他从大兴安岭带出来的,蛇身上附着白琉璃的鬼魂。自从赛维和胜伊去世后,他就跑去了大兴安岭。山林已经变了模样,大片的树木都被砍伐了,大卡车昼夜不停的向山外运送木材。但是白琉璃所在的禁地还是老样子。一是因为此地偏僻,二是伐木工人不敢来。山中树木遮天蔽日,大白天的都闹鬼。

他在地堡中找到了白琉璃。白琉璃看了二十多年的花和雪,看得百无聊赖,见他忽然出现了,真是又惊又喜:``你来了?''无心在地堡中来回的走:``外面不大好混,不如到山里做野人。''白琉璃又问:``你是一个人?''无心坐在一口破木箱上:``嗯,我太太去年饿死了。''

赛维和胜伊,都没能度过大饥荒。胜伊一生结了两次婚又离了两次婚。感情生活的不幸让他活成了一个幽怨的小孩子。在长久的粗茶淡饭之后,他固执的闭了嘴,拒绝吃糠。可是赛维当时只能找到糠。胜伊胖胖的死了,营养不良导致他身体浮肿到变了形。

全城里都没有粮。无心把自己的棒子面糊糊留给赛维,想要出去另寻食物。然而城中的飞禽走兽全进了人的肚子。他往城外走,道路两边的树皮都被剥光了。树木白花花的晾在空气中,像是夹道欢迎的两排白骨。

后来,赛维也不吃了。赛维把仅有的一点棒子面熬成稀粥,然后关了房门,不让无心再走。一小锅稀粥就是无心接下来的饮食,她气若游丝的躺在床上,要无心陪陪自己,要自己一睁眼睛,就能看到无心。

她没有浮肿,是瘦成了皮包骨头的人干。十几年来她一手把握着整个家庭,像个大家长似的挣钱花钱,在体面的时候设法隐藏财富,在拮据的时候设法保留体面。她始终是不敢堂堂正正的抛头露面,因为父亲是大汉奸马浩然。藏头露尾的经营至今,她也累了。

她不让无心走,无心就不走。无心躺在她的身边,两人分享着一个被窝。他是她的丈夫,也像她的孩子。赛维一过三十岁,在街上见到同龄的妇人领着小儿女,也知道眼馋了。

赛维枕着他的手臂,很安静的走了。无心用手指描画着她的眉眼,想起了两人十几年的争吵,想起了她年轻时候的清秀模样。想到最后,他的眼睛涌出一滴很大的眼泪。眼泪是粘稠透明的胶质,凝在脸上不肯流。

无心在安葬了赛维之后,就开始了他的流浪。和白琉璃在地堡里住了几年,他得知外面的大饥荒已经彻底过去了,便又起了活动的心思。听闻他要走,白琉璃当即附在一条白蛇身上:``把我也带上吧!我在地堡里住太久了,想去看看外面的世界。''无心大摇其头:``不带不带,我烦你。''

白琉璃没说什么。等到无心睡着了,他盘在无心的脖子上,张嘴露出倒钩尖牙,对着无心的鼻尖就是一口。无心差点没疼死,白琉璃沾染了无心的鲜血,也险些魂飞魄散。双方两败俱伤,只好和谈。和谈的结果是双方各退一步,无心带白琉璃出门见世面,但是白琉璃路上必须听话。

无心在山里住了四年,万没想到四年之后,天地剧变,竟然换了一个世界。他审时度势,立刻学会了不少崭新的革命词,并且凭着自己面嫩,冒充大中学生,拿着伪造的介绍信混到各地的红卫兵接待站中骗吃骗喝。混着混着混到了文县,他出了火车站,独自走在一条安静小街上,并不知道自己在一个小时之后,就会遇到漂亮的小姑娘苏桃了。

\chapter{他们的岁月}

对于无心来讲,时间是没有意义的。

天气热了又冷,冷了又热。山外的知青们来了又走,走了又来。机器与刀斧的力量终究是有限的,无心在山里活得安静而又安全。起伏的密林与恐怖的传说,为他隔离出了一个孤独的小世界。

山中有一条小河,不知源头在哪里,总之春天汹涌,夏天平缓,入秋之后河水渐渐干涸,到了冬天,便冻成了一条薄薄的冰带子。小河两岸盛开着鲜花,花朵颜色新鲜浓烈,美得怪异,惊心动魄。无心的赤脚趟过牵扯勾连的花草丛,初秋的阳光晒热了他的屁股脊梁。

他活成野人了,甚至省略掉了衣裤鞋袜。在足够暖和的天气里,他直接赤身露体的东跑西颠。停在一片野葡萄藤前,他咽了口唾沫。野葡萄四处攀爬,结成了一面郁郁葱葱的绿墙。紫色的果实垂垂累累,其中大部分都酸,不过只要熟透了,酸也酸得有限。

无心摘了一串葡萄,想要坐到旁边的大石头上慢慢吃,可是未等坐稳,他猛然向上一窜,开始捂着屁股骂骂咧咧。原来大石头被太阳暴晒了一天,如今的热度已经可以媲美火炭了。

无心拎着葡萄向林子里走,一侧屁股蛋被烫红了,红了一路总不见好。他素来怕疼,此刻自然满心牢骚。然而自怜自艾不耽误他觅食。大猫头鹰在林子里找到他时,他已经收获颇丰,虽然依旧红着屁股。

大猫头鹰还是没有学会说人话,对着无心高一声低一声的嗥叫了一阵,无心大概听明白了:``白琉璃又下山去了?''

然后他举起手中的一根树枝,张嘴去吃结在树枝上的野果子:``他要去就让他去嘛!''

大猫头鹰的羽毛中溢出了隐隐的一团黑雾。黑雾渐渐笼罩了他,他不见了,取而代之的站起了一个小男孩。小男孩围着无心团团乱转,一手抓住无心的腕子,一手往山下的方向指,是非让他把白琉璃找回来的架势。无心不去,不但不去,而且不耐烦,弯腰一口咬上了小男孩的咽喉。小男孩吓得一闭眼睛,一动不动的老实了。

小男孩逃离了无心的牙齿,自己跑向山下去找白琉璃,跑着跑着他变成了猫头鹰,飞着飞着他落了地,又变成了小男孩。连跑带飞的没走多远,他和白琉璃来了个顶头碰。他还没有修炼出一双阴阳眼,看不见白琉璃的影踪,可是出于妖精的直觉,他闭着眼睛都能找到对方。``扑通''一声跪在草地上,他张开双臂抱住了眼前的大白鹅,又很快乐的叫了一声:``呼!''

附在大白鹅身上的白琉璃愣了一下,随即一嘴把他啄开了。

白琉璃当蛇当得百无聊赖,于是转而做鹅。心安理得的把大白鹅交给小男孩,他溜出鹅身,一路高高兴兴的先飘向前了。在林子边缘,他啼笑皆非的遇到了无心。

无心一手倒拎着一只死鸟,一手举着一枝结满野果的绿树枝。不知道是刚刚想起了什么美事,他下面通红的支起了一根棒槌,棒槌上面缠着葡萄藤,坠着沉甸甸的两大串野葡萄。嘴里一左一右含着两枚大鸟蛋,他对着白琉璃眨巴眼睛,意思是``你回来了?''。

白琉璃被他的形象逗笑了,笑得上气不接下气,恨不能就地打滚。满山的生灵死灵加在一起,谁也没有白琉璃活得欢乐。生前藏而不发的活泼劲儿全施展在死后了,他时常笑得像个疯子。等到由着性子笑够了,他才飘到无心身边,像个活人似的陪着他并肩走:``你知道吗?山下的知青都撤走了。''

无心想要找到一块平整地方吃东西,于是一边走一边东张西望。

白琉璃又道:``知青在闹事,说是要回城。''

无心把手里的果树枝和死鸟放在了一棵老树下。自己坐在凸起的老树根上,他先吐出嘴里的鸟蛋,然后低头解开了命根子上的野葡萄藤。白琉璃为了表示自己也是通人情的,特地问道:``你想女人了?''

无心``嗯''了一声,摘了葡萄往自己嘴里送。

他已经沉默寡言了许久。白琉璃记得他死了上一个老婆之后,虽然在地堡里也哭丧了几天,但是几天之后就又嬉皮笑脸了。疑团终于有了答案,白琉璃想,越来(原来)他是特别的喜欢苏桃。

无心吃了葡萄野果,又撕开死鸟生吃了它的肉。最后带着两枚鸟蛋爬上了树,他舒舒服服的躺稳当了,不知道什么时候才能再落地。白琉璃在枝叶之间飘来飘去,想让无心带自己再下山逛上一圈。无心用一片大树叶挡住了眼睛,低声答道:``我不去。''

白琉璃告诉他:``山下有很多女知青,你可以捉一个陪你睡觉。''

无心叹了口气,不想理睬白琉璃。他和白琉璃的感情全迸发在久别重逢的一刹那,千万可别相处久了。一旦过上了朝夕相对的生活,他们迟早是要相看两相厌,比如现在,他真想把胡言乱语的白琉璃一指头弹飞。

无心躺在树上不言不动,缓慢的消化着肚中的食物。一周之后他落了地,半死不活的再次觅食。

花草渐渐凋谢了,小河渐渐消瘦了。季节周而复始的变换着,山外的知青也彻底走光了。山中才一日,世上已千年。无心长久的坐在树上,看月亮升太阳落,看星星排着阵法,一夜一夜的划过漆黑天幕。桃桃现在长大了吧?桃桃现在毕业了吧?桃桃现在结婚了吧?一滴很大的眼泪凝结在了他的腮上,是透明的胶质,最后风干,如同一颗琥珀。

在一个寂静的夜里,他又想:``桃桃现在生小孩子了吧?''

桃桃和他最初相遇的时候,也是个小孩子,孤苦伶仃,哭得上气不接下气。无心从来不做梦,可是此刻第一次体会到了做梦的感觉——他和苏桃相处的两年,就是一梦。

当无心算到``桃桃的孩子也长大了吧''的时候,苏桃已经在河北文县的县医院里工作了将近二十年。

她没有读军校,因为还是嫌军队里不自由,怕有朝一日无心回来了,组织会不同意自己和他结婚。退伍之后她主动要求分配到了文县,其实文县也不错,地方不大不小,既不落后闭塞,也不繁华喧闹。县医院是个好单位,她在医院里熬成了护士长,工资比上不足比下有余,够她活了。

她始终是没有结婚,在军队里,田兴邦曾经惊天动地的追求过她;后来到了医院,她也成了不少年轻医生的水中月镜中花。无数天作之合一般的好姻缘都被她冷漠的斩断了,她活成了医院里面有名的老处女。

她白白的美丽了一世,对于她所处的大世界,她永远是冷若冰霜、心如铁石。

在晴朗无风的周末午后,苏桃会一个人出门散步。文县越来越大了,她沿着街道慢慢走,要走好久才能到达一中门口。一中所占的还是二十年前的老楼,校园对面的破厂房成了三不管的地界。她的身体已经不复少年时代的轻盈,又顾忌着脚上的一双新皮鞋,所以在厂房废墟之中走得磕磕绊绊。最后她坐在了半截砖墙上,在阳光下举目远眺,去看砖石堆中生出的一丛丛野草闲花。

她没有读书,没有提干,没有结婚,没有生子。她以自己的人生为筹码,对无心赌了二十年的气。她坚信无心总有一天还会从天而降,就像他第一次出现时一样。到时候他老了,她也老了,她要让他读读自己一生的故事,她要让他知道他有多错!

与此同时,千里之外的无心睡在树上,很难得的做了个梦。

他梦见了一大片随风摇曳的波斯菊,盛开在那年炮火纷飞的春天里。

\begin{quote}
作者有话要说:
\end{quote}

\begin{quote}
\end{quote}

\begin{quote}
第三部到此结束,感谢大家的喜爱和阅读。
\end{quote}

\begin{quote}
\end{quote}

\begin{quote}
接下来我休息两天,如果一切顺利的话,两天之后我开始写第四部。第四部的时间背景为二十一世纪,敬请期待O(∩\_∩)O\textasciitilde{}
\end{quote}

\begin{quote}
祝大家元宵节快乐。
\end{quote}

\part{廿一世纪}

\chapter{精神病人}

在一个晴朗的四月午后,攀附在大货车顶的无心被交警发现了。当时他被牵连不清的绳网牵扯纠缠了住,否则凭着他的身手,他绝不会趴在车上束手就擒。大货车满载货物,长宽高已经几乎相等,跳车等于跳楼。交警费了老大的劲,蹬着梯子往车上爬。司机早下了车,手搭凉棚往上望,一边望一边和身边的交警解释:``我真不认识他,我能把我认识的人往车顶上放吗?哎呦我操,你们说他是怎么上去的?''

爬上车顶的交警解开了无数半死不活的大绳扣,让无心的胳膊腿儿得了自由。无心跪坐在了大货箱上,怔怔的望着面前的小交警。小交警有恐高症,一边四脚着地的往后倒退,一边怒道:``你是猴儿哇?''

话音落下,交警眼前一花,无心没了。

然后小交警在自己的惊叫声中,看到一个灰扑扑的人影斜刺里穿越国道,刹那间冲入路旁树林,从此消失无踪。

无心一路狂奔,在穿越了一片小树林后,他上了一条柏油路。路边立着个大铁牌子,上写六个大字:火星镇欢迎您。

无心仰头望着牌子,又发了半天的呆。简化字在他眼里总像是缺胳膊少腿,怎么看怎么不对劲,六个字让他翻来覆去读了好几遍。末了心里明白了,他惶惶然的迈开步子,向前走入了火星镇。在大兴安岭的深山老林里隐居了将近四十年,如今骤然回归人间,他发现人间竟然大大的变了模样——变化之剧烈,简直要让他惊恐了。

山外的人们已经不认得他手中仅有的几张旧人民币,粮票也成了天方夜谭般的往事。他的假介绍信假证明更是一分钱不值,现在的人可以随便走随便住,而且都有身份证。他穿着一身几近褴褛的旧军装走在人群中,引得人们纷纷对他行注目礼,看一个浓眉大眼的小白脸子,竟然穿戴成了乞丐模样,而且还是怪模怪样的乞丐,像是从革命时期穿越而来的。

他难得的懵懂怯懦了。扒着一辆运输木材的火车走了一段路,火车到站,他茫茫然的也到了站。在火车站外爬上一辆大货车。货车司机无知无觉的上了路,带着他疾驰了将近一天,直到交警发现了他。

无心此刻饥肠辘辘,决定去火星镇打食。千变万化的新人间虽然吓得他左一跳右一跳,但还是要比山里强。白琉璃彻底被大猫头鹰哄住了,一鬼一妖合作欺负他一个,横竖知道他死不了,所以下手格外狠辣。大猫头鹰当年一脸忠厚老实相,原来也不是个好东西。山中日月成全了一个他,几十年中他妖术大有长进,已经敢和无心蹬鼻子上脸了。

于是无心自作主张的下了山,不和他们过了。

无心沿着柏油路往前走,路是好路,路两边有田地有房屋,乃是火星镇外围的一处大村庄。此时正是四月时节,待种的田地都被翻过了,黑土被晒了一整天,此刻已经干爽松软。无心一边走一边东张西望,心想野地里不会有野菜野果,自己还是得往人的身上打主意。要说人,眼前倒是有现成的一个,看背影是个青年人,打扮得西装革履,然而双臂环抱在胸前,腰也弓着,显然是在搂抱着什么。青年人步伐匆匆,越走越快;无心连跑带跳的追上了他,侧着脸想要和他搭话,然而定睛一瞧,他心中一惊,原来青年双眼通红,满面泪痕,嘴唇紧紧的抿成了直线。西装前襟只系了一枚纽扣,下摆偶尔随风飘起,无心瞪大了眼睛,怀疑自己是看到了一圈炸弹。

看到的是一圈,看不到的,被青年双臂环绕着的,不知还有多少。一条穿着桃红背心的白哈巴狗从前头颠颠的来了,伸着舌头且颠且喘,又对着青年``汪''了一声。

未等白狗闭嘴,柏油路上爆发出了惊天动地的巨响。无心、青年、白狗瞬间化为乌有,道路两边的大树也被气浪摧成了骨断筋折。附近的房屋玻璃全起了共鸣,连远方一座小楼内的史高飞都被震得打了哆嗦。一哆嗦,手里的面巾纸失了准头,他上面望着电脑屏幕里的南波杏,下面一波接一波的射了一裤子。

一惊之后,史高飞慌忙低了头。裤子被他退到了大腿处,如今前门拉链已经被他的万子千孙彻底糊住。匆匆忙忙的用纸擦了,他心怀鬼胎的提了裤子往窗口跑。``哗''的一声拉开拉窗,他探出上半身向外张望,想要查看巨响的来源。然而窗外风景一如往常,只有一只大灰雀趁虚而入,扑啦啦的飞进了房内。

史高飞来不及驱赶鸟类。转身出了房门穿越客厅,他推开向外的楼门,几大步蹿进了院子里。院子是大院,一半铺了水泥地,一半种了花花草草。另有一棵吃里扒外的老果树紧挨院门,每年都要无私的向院外奉献出几枝子沙果。史高飞别有心事,一味的只往大门口跑。然而未等他打开左右合拢的黑漆铁栅栏门,他的眉心之间忽然落了一滴暖暖的雨。下意识的抬手一摸,他随即对着手指头直了眼——不是雨,是血!

猛然抬头向上望去,在老果树的密集枝杈之间,他看到了一只白色的狗头。狗头保持着龇牙咧嘴的神情,脖子往下一无所有,只垂了丝丝缕缕的几条鲜红筋肉。狗嘴毫无预兆的上下一张,一小块粉红色的肉垂直落到了黑土地上。

在和狗头对视了片刻之后,史高飞和狗头一样龇牙咧嘴了,恶心得恨不能就地呕吐一场。举起一根竹竿捅下狗头,他薅着狗耳朵将其扔到了院外。随即跟着狗头一起出了门,他一路小跑的看热闹去了。

史高飞本名史鸿鹏,乃是本镇首富之子。他幼年兼生了倾国倾城的貌以及多愁多病的身,把他上面的一个姐姐比得狗屁不如。不过一个男孩子一味的娇弱也不是长久之计,后来经过高人相看之后,他换汤不换药的改了名字——由具体的``鸿鹏'',改成了抽象的``高飞''。

名字一改,果然立竿见影,史高飞改头换面,从小病秧子变成了一名高大英俊的精神病患者。从十五岁疯到了二十五岁,他坚信自己是一名外星遗孤,有朝一日必将回归母星。他妈赵秀芬为他嚎得肝肠寸断,并且在丈夫史一彪心中彻底失宠——当年在赵秀芬年轻貌美之时,史一彪忘了赵秀芬的妈和妹妹曾经先后声称自己是狐狸大仙和九天神女。赵家八辈贫农,全国劳苦大众都翻身了他家也没翻身,留给子孙后代唯一的遗产就是精神病。史一彪重男轻女,恨不能练就神功,把儿子的精神病转给姑娘。姑娘三十了,生得花容月貌,袅袅娜娜,曾经是火星镇的林黛玉,还念过三年大专,可如今硬是没人敢娶,因为都怕她会随了她妈,再养出个疯儿痴女。

史一彪对于家庭彻底失望,尤其恨老婆恨得牙痒,常年不肯回家。他身为本镇的娱乐业巨头,经营着今夜星辰夜总会,明日之星KTV,快乐时光咖啡屋,以及酷龙连锁网吧三家。既然拥有如此可观的家业,他自然不会无处落脚。而赵秀芬进入更年期,天天在家要死要活,专跟着女儿较劲。女儿名叫史丹凤,既没事业也没爱情,连她妈都不肯高看她,甚至认为她一个人也挺好,将来正好照顾儿子一辈子。反正儿子疯得全镇出名,想必也找不到媳妇伺候他一生。史丹凤看她妈把心偏到了胳肢窝里,自然也有意见。总而言之,史家全体成员之中,只有史高飞的痛苦程度较轻——他一心等待母舰降临接他回家,对于家中三个地球人,他一般懒得搭理。

在柏油路上的村民群中凑了半天热闹,因为警察封锁了现场,所以他也没看到什么,只知道路面被炸出了一个大坑。傍晚时分,观众们纷纷回家做饭,他也跟着回了自己所住的小楼。小楼一共有二层,当初史一彪想在农村发展一点副业,才盖起了小楼大院。后来副业胎死腹中,小楼空着没人住;而史高飞去年年末被家人强行送进精神病院住了一阵子,出院之后和地球人越发势不两立,索性独自进了村,要安安静静的过几天田园生活。

没滋没味的锁了院门进了楼,他穿过客厅往卧室里走,一边走一边自己叹息:``我还以为是飞船来了呢!''

电脑屏幕上的视频已经播放完毕,不速之客大灰雀也早没影了。他牢牢骚骚的蹲到电脑桌下,想要清理白天乱扔的面巾纸团。不料在一团半干半黏的面巾纸下,他意外的发现了一枚大豆子。此豆十分古怪,竟然是个心形,如果把它比作人的话,必定是个连体婴。史高飞四体不勤五谷不分,不知道豆子也会畸形。捏着豆子端详了半天,他扪心自问:``我白天射豆子了?''

随即他把裤子一脱,仔细检查了自己的先天条件,最后认定这应该是不可能,因为他的那条播种的道路长而狭窄,不足以孕育出尺寸如此壮观、形象如此美好的种子。拈着豆子站起身,他忽然打了个激灵,心里又生出了邪主意:莫非方才自己的卧室内有人来过了?莫非这豆子承载着母星传递给自己的信息?光天化日的,总不会无端的发生大爆炸,必有玄妙在里面!

可他马上又犯了难:母星的使者也太不体谅人了,他在地球过了二十多年,现在哪里还能和同类心有灵犀?掂着豆子出了许久的神,他坐卧不安,实在是揣摩不出豆中的深意,又不敢贸然把豆子剖开或者嚼碎。抓心挠肝的熬到午夜,他终于浮想联翩的思索出了眉目:``这是一颗种子啊!''

午夜时分,众人皆睡,唯有史高飞独醒。站在土质最为肥沃的老果树下,他挥舞着一把大铁锹,挖了个半米多深的圆坑。恭而敬之的把心形豆子放入坑底,他双膝跪地,亲自伸手捧土填坑,一边填一边又默默祈祷:``种子啊,你快长大快显灵吧。他们都不相信我的话,还丧心病狂的诬陷我,说我是精神病。你一定要长成个了不起的宝贝,好向他们证明我的身份!''

虔诚的撒下最后一把土,他双手合什又拜了拜。最后意犹未尽的站起身,他垂着两只泥手仰望苍穹,心想满天的星星有明有暗,不知道哪一颗才是我的家。人在异星,没个知音,真是遭罪啊!

村口柏油路上的爆炸案上了各大网站的头条,捎带着火星镇一起出了名。一个月后,案子基本破了,原来是场未遂的情杀——一男一女搞对象搞出了仇,男方是个亡命徒,绑了一身炸药往女方家去,本意是要趁着傍晚女家人齐全,点燃导火索来个一锅端。没想到炸药本身出了问题,走到半路,自行炸了,炸得什么都不剩,导致警察须得四处走访调查,一点一点的拼出事实真相。

村里常年太平,近几年连去世的老人都少有,所以一桩爆炸案足以让村庄沸腾许久,唯有史高飞极其冷静,满眼满心只装着他的种子。在等待种子发芽的期间里,他连爱情动作片都没心思下载了,成天无欲无求的蹲在树下,直勾勾的只盯着土地使劲;饭也时常是一顿管一天,饿得他一米九的身高只有一百五十斤,扛着宽肩膀垂着头,他支起后背两大片肩胛骨,乍一看好像一只秃毛又折翼的大天使。

勤勤恳恳的浇了两个月的水,他天天对着一片土地望眼欲穿。如此熬到了七月,头顶的果树已经结出了累累的小绿果子,可是他的种子依旧毫无动静。

他等不得了。在一个狂风大作的夜晚,他欲哭无泪的蹲在树下,预备对种子做出一番控诉,然后把它挖出来就地踩扁。然而在他顶风开口之前,空中忽然裂过一道闪电。随即在震天撼地的雷声中,史高飞睁大眼睛,发现一贯平坦的地面竟然隐隐鼓凸,仿佛是有什么东西将要破土而出了!

颤巍巍的伸出一只手,史高飞轻轻的拨开了最表面的一层浮土。浮土之下露出了一小块粉红的皮肉,皮肉中钻出几根东倒西歪的白毛,正在暴雨来临之前的疾风中微微抖动。

史高飞忽略了地上的风与天上的雷。他屏住呼吸张大了嘴,用十根手指又挖又掘。末了在第一颗大雨点子砸向他时,他从土里刨出了一只半人长的大毛毛虫。``扑通''一声跪在泥水之中,他激动得又哭又笑,又捶大腿又甩泥巴。原来母星的同胞并没有忘记他,原来同胞所给他的,真是一粒种子!

脱下身上的T恤裹住大毛毛虫,他在大雨之中站起了身,抱着毛毛虫趿着人字拖,他一路噼里啪啦的跑进楼里去了。

\chapter{家人们}

佳琪在医院里住了三天,史一彪夫妇很有一点卸磨杀驴的意思,天天只知道围着保温箱看孙子,对于佳琪是明显的不上心。白大千倒是怀有满腔父爱,可又没法亲自伺候女儿,只能是从早到晚的陪在病房里,一是给佳琪端茶递水;二是严防史高飞作乱——昨天众人一眼没照顾到,史高飞差点勾搭着佳琪走回了家。

到了第四天,佳琪仿佛已经恢复了元气似的,开始有精神对着父亲和史高飞连说带笑。白大千很高兴,不是高兴自己有了外孙,而是高兴女儿平安无事;史高飞也很高兴,因为感觉佳琪鼓着肚子的样子很怪异,现在好了,终于不鼓了。

大家忙着高兴,零七八碎的琐事全压在了史丹凤一人肩上。她也没生过孩子,没有经验没有知识,但是史一彪和赵秀芬把她当成了万事通使唤,导致她终日奔波,又要雇月嫂又要买奶粉尿不湿以及一切婴儿必需的小玩意儿。圣诞将至,天寒地冻,陪伴她的只有无心。史丹凤偶尔得了一时半刻的清闲,必会不动声色的偷偷凝视无心,凝视到了最后,她暗暗的叹了口气,心想旁人对自己都没有真感情,要说好,还得是无心。

第四天的中午,佳琪出院回了家,月嫂也就了位,站在厨房里给佳琪熬鲫鱼汤。佳琪始终是没有奶水,导致赵秀芬和史一彪背地里嚼舌头,认为儿媳妇太不顶用,既没有奶,生的孙子又只有三斤多。赵秀芬跃跃欲试的想要拿出婆婆的派头,甩给佳琪几句闲话听听;然而闲话甩是甩了,佳琪却是乐呵呵的完全没听出言外之意。她很失望,还想继续甩,结果同样没有听懂的史高飞不耐烦了:``行了,妈你吵死了!''

赵秀芬怕儿子,史高飞一发话,她立刻成了属黄花鱼的,贴着墙根溜出了卧室。

佳琪一直没能见到儿子,于是干脆利落的把儿子忘了个一干二净。史高飞对于儿子更是毫无接收的意思——他可不想弄得家里满坑满谷全是地球人!

于是在一个多月后,当他看到赵秀芬从医院抱回了六斤重的小男婴时,气得当众沉了脸:``怎么回事?全弄到我家里来啦?''

史一彪搓了搓手,没敢言语,使了个眼色让赵秀芬说话。赵秀芬捧金子似的捧着个小襁褓,也很打怵。白大千站在一旁,直着眼睛看着孩子,心想要是没有他,自己也不至于要把佳琪嫁给史高飞。

史高飞站在门口,摆出一夫当关万夫莫开的架势,想要捍卫家园的纯洁。赵秀芬抱着孩子进退两难,月嫂站在厨房门口,也是犹犹豫豫的不知该不该迎接。忽然一眼瞄到了史丹凤,赵秀芬立刻有了目标:``小凤,你可真是的,傻站着等什么呢?三十多岁的人了,一点儿眼色也没有!''

无心正在隔壁卧室里和佳琪玩电脑,客厅说话,卧室里可以听得清清楚楚。史丹凤带着个小丈夫过日子,本来就心虚,如今听她妈说她``三十多岁'',心中登时腾起一股子恼羞成怒的火。一甩手走向卧室,她头也不回的说道:``我不会抱,要抱也轮不到我。''然后站在卧室门口叫道:``无心,别玩了,我们回家!''

史高飞不想让无心走,所以回头打断了史丹凤的命令:``不行,不许回家!''紧接着向前面对了父母,他义正词严的说道:``小孩子我们不要,送给你们了,你们带走吧!''

赵秀芬和史一彪面面相觑,又求援似的一起望向了白大千。白大千手足无措的做了个深呼吸,忽然竖起一根手指轻声说道:``有办法了!''

白大千从卧室里请出了无心,让无心去和史高飞说话。无心对史高飞和佳琪有感情,对于他们的孩子却也是毫无兴趣。当着众人的面,他毫无诚意的劝了史高飞几句,结果被史高飞扯着衣领打了一巴掌,因为他``吃里扒外''了。

史一彪夫妇见儿子真动了怒,立刻识相撤退。孙子太小了,没办法直接带回火星镇。无可奈何之下,他们只好跟着白大千赶往城郊,连人带孩子一起去了写字楼上的出租屋。幸而出租屋够宽敞够明亮,设施家具也齐全,供暖尤其是好,暖气永远热得烫手。把大粽子似的襁褓放在沙发上解开了,赵秀芬小心翼翼的从里面捧出了六斤多的孙子。白大千情绪复杂的凑近看了看,先前他不看倒也罢了,如今骤然和婴儿打了照面,他心中一动,竟是猛的生出了一股子爱意。婴儿明显是病怏怏的不健壮,但是眉目五官已经长清楚了,活脱是个小佳琪的模子,唯有细枝末节是随了史高飞。

``哎哟\ldots{}\ldots{}''他很意外的惊叹了:``睁眼睛啦?''

他的气息扑到了赵秀芬的面颊上,赵秀芬飞快的瞟了他一眼,一颗沉寂了十几年的心灵生出了蠢蠢欲动的春意。毫无预兆的,她不好意思了。

``可不是睁眼睛了?''她仿佛从更年期一步退回了青春期:``瞧瞧,他看你呢。''

史一彪刚撒了泡尿,此刻推了卫生间的门走进客厅,他望着赵秀芬和白大千并肩而立的背影,忽然感觉此情此景有些不大顺眼,但是又挑不出毛病。

因为儿子坚决不允许孙子进门,孙子又弱小得不能出远门,所以别无选择的,史一彪只好让糟糠之妻留在了江口市。完全把孙子交给月嫂伺候,是不能令人放心的,非得有个亲人在旁边照应着才行。史白两家的人口加起来,唯一合适的人选便是赵秀芬了。

史一彪惦念着家里的生意,赶在年前回了火星镇。赵秀芬独自留在写字楼上的出租屋里,单枪匹马精神焕发,把孙子照顾得密不透风。她是得意了,史丹凤却是吃了苦头——赵秀芬如今有了孙子,越发的不拿她当人了。

于是不出一个月的工夫,她也搬了家。在市区边缘的一处小区里,她租下了一套三十多平方米的小房子,也没做天长地久的计划,只想暂时避开他妈的锋芒。

偌大的出租屋里骤然只剩了赵秀芬和白大千两个大人,史一彪在家里得知了消息,依然是挑不出毛病,然而越想越是不对劲。坐拥着他的二三四五奶,他时不时的就感觉自己头上发绿后背发硬,很有当王八的征兆。但是话说回来,凭着亲家公的风采和身份,找小姑娘都能找了,又怎会青睐一个当了奶奶的黄脸婆子?

思及至此,史一彪略略的放宽了心,认为自己还是想多了。转眼之间到了新年,他孤身一人又去了江口市。在出租屋里见到赵秀芬时,他吓了一跳,发现黄脸婆子居然脸也不黄了,嗝也不打了,从头到脚收拾得利利落落,堪称是徐娘半老、风韵犹存。

他留了心眼,转而再去观察白大千。白大千前些日子又受了汇丰的刺激,如今彻底陷在了钱眼里不能自拔。百忙之中见了史一彪一面,他心底无私天地宽,一清二白坦坦然然。史一彪看亲家公神采奕奕,比上次相见时又帅了几分,一颗心便是上不着天下不着地的悬在了半空中,不知道自己的隐忧到底有没有继续存留的必要。

一个电话打给了史丹凤,他吆喝狗似的,让女儿女婿马上搬回写字楼住。哪知史丹凤立下了造反的主意,不肯听他的话。嫁出去的女儿泼出去的水,史一彪在用得着女儿的时候才发现覆水难收了。

史丹凤新租的房子不但狭窄,而且陈旧;说是个一室一厅的格局,其实厅小得虽有如无,唯一可以活动的场所便是卧室。晚上下班回了家,她照例是要在满壁油烟的老厨房里炒菜做饭。和先前单身时相比,她现在出手阔绰了许多,尤其在一日三餐上很大方,绝不肯亏待了无心的嘴。煎炒烹炸的预备出了三菜一汤,她干活干惯了,也不指望着无心帮忙。一样一样的把菜端到床前的圆桌子上,她一边忙一边说话:``现在房价越来越高,我们真不能再等了,再等连城外的房子都买不起了。无心,别玩了,起来吃饭!还玩?是不是想等我把饭喂到你嘴里?我想好了,我要挑个好地点,面积大小无所谓。反正只有两个人,一间屋子也住得开。''

把两碗米饭放到桌上,她转向大床,用女低音做出震慑性呼唤:``无心!''

趴在床上玩手机的无心一翻身坐起来了,爬到床边面向了圆桌,他端起饭碗往嘴里扒了一口饭。史丹凤问道:``饭香不香?这米很贵呢。''

无心在腾腾的蒸汽中抽了抽鼻子,然后抬头答道:``香。''

史丹凤搬了个圆凳子,也在旁边坐下了。抄起筷子给无心夹了菜,她一边吃一边又道:``下午是不是又跟着白大千放血去了?自己长点心眼,别让他对你起疑心。虽说他现在和我们是亲戚了,可万一他知道了你的秘密,谁敢保证他不会卖了你?''

无心连连点头:``放心吧,我知道。''

史丹凤伸手摸了摸他的脑袋:``真不知道再过十年,你会是什么样子。你说你究竟是个什么东西呢?''

无心一晃脑袋,皱起眉毛答道:``又来了。我是个人嘛!''

史丹凤叹了口气:``你是不是人,我还不知道吗?''

无心已经被她惯出了一点小脾气:``烦死了,吃饭!''

史丹凤重新抄起筷子:``哟,小爷们儿,说你几句还不乐意了!''

无心鼓着腮帮子一嚼一嚼:``总说我不是人\ldots{}\ldots{}''往嘴里塞了一口菜:``我不爱听!''

他再闹史丹凤也不生气。感觉到自己快要惹不起他了,史丹凤宽宏大量的让了步:``好好好,不说了。''

无心把空碗向她一递:``再来一碗。''

史丹凤接了饭碗起身去厨房盛饭,无心从桌面上捏了一粒大米饭,然后俯身弯腰,从床下的一只运动鞋里抓出了白琉璃。他一手扒开白琉璃的尖嘴,一手将大米饭粒往尖嘴里塞。史丹凤端着饭碗回来了,一眼看清了他的所作所为,当即叫道:``它是吃小米的,你别乱喂!放手!好好的一只鸟,都快被你抓死了!弄只鸟不好好养,弄只猫也不好好养,真是的,猫还挺贵!对了,猫呢?你又打它了?有意思,还对一只猫记了仇!吃饭,吃完了烧热水给你洗个澡。猫到底在哪里?你把它给扔了?''

史丹凤长篇大论,一口气说了无数话,说得无心直发笑:``没扔,猫在床底下呢!''

他一笑,史丹凤也跟着笑了,因为意识到自己是个碎嘴子,居然一下子唠叨了一大串。

吃饱喝足之后,史丹凤端了一大盆热水进卧室,拧了毛巾给无心擦了擦身,又从床底下抓出小猫,给小猫也洗了个澡。小猫越长越漂亮了,见了无心如同见鬼,只跟史丹凤亲近。白琉璃站在窗台上,啄着浅浅一碟小米。无心光着屁股仰卧在床上,双手举着报纸看房产广告。看着看着架起了二郎腿,他一边晃着赤脚,一边说道:``姐,我给你唱首歌吧!''

不等史丹凤回答,他清清喉咙开了嗓:``青城山下啊啊啊白素贞\ldots{}\ldots{}''

他平时说话并无异常,一旦唱出调子了,声音却是变得微哑苍凉,仿佛唱的不是青城山下白素贞,而是雪域大漠白素贞。一曲终了,他惹出了史丹凤一声叹息:``唱得像和尚念经似的,我都要听哭了。''

无心被她兜头泼了一盆冷水,登时闭了嘴。在大床来回打了几个滚儿,他不甘寂寞的又开了口:``姐,上床吧,我们一起看报纸。''

史丹凤洗漱完毕上了大床,和无心并肩趴着浏览房产广告,一边浏览一边点评,又在手边摆了个计算器加减乘除。无心的心思漂移不定,不是扯一扯史丹凤的头发,就是掀一掀史丹凤的睡衣。史丹凤一心二用的撵着他哄着他,最后撵也撵不走哄也哄不住了,她无可奈何,索性侧身一解睡衣纽扣:``小宝宝,给你吃奶,别闹我了!''

无心如愿以偿,立刻向前伸了手,嘴里自言自语的小声嘀咕:``两只大兔子!''

史丹凤一手拿着报纸,一手拿着计算器,被他逗笑了。

下一秒,无心用双手抓住了兔子之一。张大嘴巴``啊呜''一口,他作势要咬,吓得史丹凤用报纸抽了他一下:``敢?!''

无心既不敢,也不想。他只是一口叼住大兔子,亟不可待的开始吮,仿佛真是小娃娃在吃救命的奶。嘴里叼着一只,手里又抓了另一只,两只兔子,全是他的。

史丹凤天天看房产广告,从冬天看到春天,又从春天看到夏天。在大半年的光阴里,无心身边发生了以下事件:第一,天天挨揍的小猫崽子趁着无心和史丹凤不在家,把大灰雀咬死了。白琉璃走投无路,只好做猫。猫爪子拍在手机屏幕上,一架飞机也拍不碎。人生乐趣瞬间降至零点,他面无表情的蹲在窗台上,又想回家了。

第二,大猫头鹰历尽千难万险,居然找到了无心的家。无心并没有亲眼见到他,因为白琉璃直接隔着纱窗打发了他,让他自己先回大兴安岭。大猫头鹰可怜兮兮的听了他的话,拍着大翅膀继续往北飞。又因为据白琉璃描述,大猫头鹰形象极其狼狈,已经类似秃鹫;所以无心听得满心欢喜,别有一种幸灾乐祸的痛快。

第三,风头正劲的白大千在走夜路时,被本市一位鼎鼎大名的半仙买凶揍了一顿,半死之时偶遇英雄,英雄拔刀相助救了他一命,而他为了报恩,立刻将英雄聘为公司保安,月薪高达一千八百元。此英雄名叫李光明,即史高飞在火星镇的邻居兼校友。得知史高飞的儿子和史高飞的姐姐结婚了,李光明惊得张大嘴巴,一张面孔由正方变为长方,同时深感世事无常,再也不相信爱情了。

第四,史高飞和大蜥蜴的歌唱事业平稳进行,工资也涨了三成。史高飞的保留曲目是《青城山下白素贞》,每天晚上必唱一次,唱的时候时常能够在客人中带起一轮小合唱。照理来讲,他生得高大英俊,又是坐在台子前方,必能吸引所有目光。然而一名珠光宝气的小富婆眨巴着一双慧眼,却是看上了阴暗处的大蜥蜴。她想方设法的和大蜥蜴搭了好几次话,大蜥蜴温文尔雅,不冷不热的对谁都是一视同仁。小富婆遭遇了几次婉拒之后,反而越发的爱他了。

第五,史高飞的地球儿子已经有了乳名,叫做嘟嘟,是赵秀芬起的。赵秀芬和白大千朝夕相对,白大千冰清玉洁,不耽误她单方面的胡思乱想。但是慑于史一彪的剽悍和白大千的潇洒,她决定走保守路线,对亲家公过过眼瘾也就罢了。而史高飞和佳琪夫唱妇随,每天要么吃要么玩。白大千见女儿是真快乐,也就不甚甘心的认命了。

第六,无心跑了好几趟火星镇,终于上了户口有了身份。办好身份证后回了江口市,他在闲暇之时又去驾校学了一个多月。通过考试得了驾照之后,史家车库里常年蒙尘的帕萨特立刻归了他。

第七,史丹凤下定决心,终于决定买房,并且还是全款买房。房子距离史高飞家只有一站地的距离,正如她所愿,的确是黄金地点,可惜只有五十多平方米,是弟弟住宅的三分之一。无心身为史丹凤的小爷们儿,深感经济压力巨大,可饶是巨大,他被白琉璃折磨得没了办法,还是向史丹凤开了口:``姐,你听说过IPAD吗?''

史丹凤直接告诉他:``没有钱,不给买!''

无心讪讪的舔了舔嘴唇,抱着猫走出一站地,上楼去了史高飞家。

站在史高飞面前,他略略的理直气壮了一点:``爸,你听说过IPAD吗?''

史高飞翻箱倒柜的找了一通,末了从个乱七八糟的抽屉里找到了一只长圆形的硬壳盒子:``宝宝,爸爸只有个PSP,你要玩吗?''

无心打开盒子看了看,然后摇了头:``不行,它太小了。有没有大的?''

史高飞没听明白:``大?多大?''

无心一抬小猫的前腿:``按键要像猫爪子一样大。''

史高飞立刻摇了头:``没有。''

无心伸出了一只手:``爸,我的钱都给姐了。现在我想去买个IPAD,你给我钱好不好?''

史高飞二话不说,当即从抽屉里翻出了自己的大皮夹。皮夹里面放着一张照片,照片里是粉红色的一大坨,任谁也看不出它是什么,连无心自己都伸着脖子瞧了半天。末了瞧明白了,他抬手抓了抓头,忽然感觉很羞涩:``爸\ldots{}\ldots{}''

史高飞也是低头盯着照片看:``宝宝,你看你小时候,多像一条毛毛虫。''

无心难为情了:``我\ldots{}\ldots{}''

史高飞一边从皮夹里抽出银行卡,一边温柔的低声说道:``还是小时候最可爱。你还记不记得爸爸抱着你喂奶的事了?''

此言一出,无心怀里的白琉璃立刻回头看了他一眼,随即把脑袋探向了史高飞的皮夹。一双溜圆的猫眼睛盯住了照片,他一时间看清楚了,登时竖起了一身的毛。

无心捂住他的猫眼睛,把他强行摁回了怀里。接过史高飞的银行卡,他狼心狗肺的转身就跑。史高飞本来还想亲他一口,然而一步上前抓了个空,硬是没能亲到。

无心抱着白琉璃下了楼,一路往市中心走。胸口隐隐的凉了一下,是白琉璃自下而上的现出了一个脑袋:``你是毛毛虫?''

无心连忙否认。

白琉璃一脸狐疑的审视着他,无心迎着他的目光,一本正经的说道:``我是神仙。''

白琉璃缩回了猫身之中,忽然发现自己并不是很了解他。将一只猫爪子搭上了他的手臂,白琉璃抬头望着车水马龙的大街,耳听无心喃喃的又道:``白琉璃,今天我给你买新游戏机,以后不许你再往我的枕头上撒尿。我年纪比你大得多,你作为我的灰孙子,应该尊敬我,照顾我。我回家的时候你应该给我叼拖鞋,我趴上床了你应该给我踩后背。姐给我预备的水果你不应该偷吃,要吃也只能一个一个的吃,不能每个都只咬一口。还有\ldots{}\ldots{}''

白琉璃静静倾听着他的长篇大论,越听越生气,恨不能直接把他挠死。

无心下午出门,傍晚回家。一手抱着白琉璃,一手拎着个纸袋子,他垂头丧气的站在门口,身上的单薄T恤破烂成了渔网,连肚脐眼都见了光。

史丹凤正在厨房淘米,闻声赶来一看,登时大惊失色:``让人劫了?''

无心慢吞吞的摇了头,委委屈屈的答道:``让猫挠了。''

史丹凤看清了纸袋子中的白色包装盒,明白他终究还是把钱花了出去。鼻孔呼出两道凉气,她转身走回了厨房:``挠得好!替天行道,大快人心,省得我亲自挠了!''

\begin{quote}
——全文完
\end{quote}

\begin{quote}
\end{quote}

\begin{quote}
作者有话要说:因为未来的事情我也不知道,所以《无心法师》写到这里就结束了。去年刚刚开坑时,本来只是想写个鬼故事,没想到会越写越长,最后竟然成了我所写过的最长的文。
\end{quote}

\begin{quote}
感谢大家对本文的喜爱,感谢给我写长评扔地雷的同学,感谢文下每一条评论,感谢大家对我的鼓励和支持。非常非常的感谢。
\end{quote}
